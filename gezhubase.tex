%# -*- coding: utf-8 -*-
% !TeX encoding = UTF-8 Unicode
% !TeX spellcheck = en_US
% !TeX TS-program = xelatex
%~ \XeTeXinputencoding "UTF-8"
% vim:ts=4:sw=4
%
% 以上設定默認使用 XeLaTex 編譯,並指定 Unicode 編碼,供 TeXShop 自動識別
%
%%%%%%%%%%%%%%%%%%%%%%%%%%%%%%%%%%%%%%%%%%%%%%%%%%%%%%%%%%%%

\renewcommand\KG{{ }}

\usepackage{fancyhdr}
\pagestyle{fancyplain}
\fancyhf{}
\renewcommand{\headrulewidth}{0pt}
%\def\thepage{\Chinese{page}}
\renewcommand{\thepage}{\Chinese{page}}
\fancyfoot[R]{\thepage}






\usepackage{gezhu}
%\setgezhuraise{-6pt}
%\everygezhu={\zihao{5}\linespread{.9}\selectfont\itshape}
%\expandafter\everywithgezhu\expandafter{\the\everywithgezhu \baselineskip=16pt \parskip=1em}
\everygezhu{\fontsize{6}{7}\selectfont}
\setgezhuraise{-1.5pt}
\setgezhulines{2}

%\parindent=2em
\setgezhushipoutlevel{2}
%\gezhuraggedfalse
%\gezhuraggedtrue
\gezhunormalizetrue


%\def\skipzhfullstop{\hskip 0pt plus 0.3em}






\comments{
% 使用 footnote 替代 \gezhu
%\makeatletter
%\let\oldfootnote\footnote
%\renewcommand\footnote[1]{\gezhu{#1}} % {\gezhu{\fntfmyJiaozhu\tiny #1}}
%\makeatother

%\makeatletter
%\let\oldmarginnote\marginnote
%\renewcommand\marginnote[1]{\gezhu{#1}} % {\gezhu{\fntfmyJiaozhu\tiny #1}}
%\makeatother

}%comments

\comments{

\def\myonkyoh#1{%
  \unless\ifdefined\isfirsttime
    \def\isfirsttime{yes}%
    \withgezhu
  \fi
}

% 设置章节
% http://wiki.lyx.org/Tips/ModifyChapterEtc
\let\oldchap=\chapter
\renewcommand*{\chapter}{%
  \secdef{\Chap}{\ChapS}%
}
%\newcommand\ChapS[1]{\singlespacing\oldchap*{#1}\doublespacing}
%\newcommand\Chap[2][]{\singlespacing\oldchap[#1]{#2}\doublespacing}
\newcommand\ChapS[1]{\oldchap*{#1}\myonkyoh{#1}}
\newcommand\Chap[2][]{\oldchap{#2}\myonkyoh{#2}}
%\newcommand\ChapS[1]{\ifdefined\gezhu\endwithgezhu\fi\clearpage\oldchap*{#1}\withgezhu}
%\newcommand\Chap[2][]{\ifdefined\gezhu\endwithgezhu\fi\clearpage\oldchap{#2}\withgezhu}

%\newcommand\ChapS[1]{\ifdefined\gezhu\endwithgezhu\fi\oldchap*{#1}\withgezhu}
%\newcommand\Chap[2][]{\ifdefined\gezhu\endwithgezhu\fi\oldchap[#1]{#2}\withgezhu}

%\newcommand\ChapS[1]{\ifdefined\gezhu\endwithgezhu\fi\oldchap*{#1}\withgezhu}
%\newcommand\Chap[2][]{\ifdefined\gezhu\endwithgezhu\fi\oldchap[#1]{#2}\withgezhu}

\def\endmyonkyoh{\ifdefined\gezhu\endwithgezhu\fi}

}



\comments{

\newcommand{\BeginChapter}{}
\newcommand{\AtBeginChapter}[1]{%
    \renewcommand{\BeginChapter}{#1}%
}

\newcommand{\EndChapter}{}
\newcommand{\AtEndChapter}[1]{%
    \renewcommand{\EndChapter}{#1}%
}

\makeatletter

\newenvironment{tablehere}
{\def\@captype{table}}
{}

\let\original@chapter\chapter
\def\@first@chapter{1}
\renewcommand{\chapter}{%
    \ifnum\@first@chapter=1 \gdef\@first@chapter{0}\else\EndChapter\fi
    \original@chapter\BeginChapter}
\makeatother

%\AtBeginChapter{\centerline{$+++++$}}
%\AtEndChapter{\centerline{$******$}}

%\AtBeginChapter{\withgezhu}
%\AtEndChapter{\ifdefined\gezhu\endwithgezhu\fi}
}





