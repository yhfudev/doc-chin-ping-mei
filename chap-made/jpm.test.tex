%# -*- coding: utf-8 -*-
%!TEX encoding = UTF-8 Unicode
%!TEX TS-program = xelatex
% vim:ts=4:sw=4
%
% 以上设定默认使用 XeLaTex 编译,并指定 Unicode 编码,供 TeXShop 自动识别

\commentsr{
\chapter*{测试 1 (1) test}
\addcontentsline{toc}{chapter}{测试 1}

漢字源𣴑考


%\usepackage{CJKfntef}
%\CJKunderwave{漢字源𣴑考}

\begin{table}[ht]
\caption{生僻字测试}
\centering
\begin{tabular}{|cc|cc|cc|cc|}
𠀀 & 20000 & 𠀁 & 20001 & 𠀂 & 20002 & 𠀃 & 20003 \\
𠀄 & 20004 & 𠀅 & 20005 & 𠀆 & 20006 & 𠀇 & 20007 \\
𠀈 & 20008 & 𠀉 & 20009 & 𠀊 & 2000A & 𠀋 & 2000B \\
𠀌 & 2000C & 𠀍 & 2000D & 𠀎 & 2000E & 𠀏 & 2000F \\
𠀐 & 20010 & 𠀑 & 20011 & 𠀒 & 20012 & 𠀓 & 20013 \\
𠀔 & 20014 & 𠀕 & 20015 & 𠀖 & 20016 & 𠀗 & 20017 \\
𠀘 & 20018 & 𠀙 & 20019 & 𠀚 & 2001A & 𠀛 & 2001B \\
𠀜 & 2001C & 𠀝 & 2001D & 𠀞 & 2001E & 𠀟 & 2001F \\

{\char"028C46} & 028C46 &
{\char"021C1E} & 021C1E & % 𡰞
%\XeTeXglyph35897 & 35897 & % Here we use an index number to display a glyph.
\\
\end{tabular}
\end{table}

一丁上下

目录 %\footnote{test 1 footnote}

第二回

第二囘


頭髮䯼髻 %\marginnote{margin note example}

抹胸兒重重紐扣,

褲腿兒臟頭垂下

紅紗膝褲扣鶯花 %\marginnote{1第1个边角註.}
%\marginnote{2这个是第2个.}

常在公門操鬬毆
我卻怎生鬬得過他

此事便獲得着

言文真飴\marginnote[1第1个边角註(左).]{2这个是第2个(右).}

为為了

青月令育 %\footnote{test 2 footnote}

者老出山

追活沾信

華女紫瓜

波稃浮


\ifnum\strcmp{\myfnotemode}{\detokenize{gezhu}}=0

\chapter*{测试 2 (2) test}
\addcontentsline{toc}{chapter}{测试 2}

(GEZHU在 chapter內)

\begin{withgezhu}


話說武松自從搬離哥後,撚指不覺雪晴,過了十數日光景。都說本縣知縣,自從到任以來,都得二年有餘,
轉得許多金銀,
\gezhu{(轉得許多金銀)「轉」容本《忠義水滸傳》做「撰」,稍後如楊定見本、芥子園本、金聖歎貫華堂七十回本水滸傳均做「賺」。而前此世德堂本《西遊記》,已用「賺」代「撰」。本書「轉」、「撰」、「賺」並用。第五十三回:「賺得些中錢,又來撒漫了」;第七十六回:「家中胡亂積賺了些小本經紀」;第九十八回:「別無生意,只靠老婆錢賺」。下凡「轉」、「撰」取錢物,均統一為「賺」,不再一一出校。}
\marginnote{good}
要使一心腹人,送上東京親眷處收寄。三年任滿朝覲,打點上司。一來都怕路上小人,須得一個有力量的人去方好。猛可想起都頭武松,須得此人英雄膽力,方了得此事。當日就喚武松到衙內商議,道:「我有個親戚,在東京城內做官,姓朱名勔,見做殿前太尉之職。要送一擔禮物,
稍封書
\footnote{(稍封書)「稍」此處代「捎」。劉改「捎」。下凡以「稍」代「捎」,隨文改正,不再出校。}
去問安。只恐途中不好行,須得你去方可。你休推辭辛苦,回來我自重賞你!」武松應道:「小人得蒙恩相抬舉,安敢推辭?既蒙差遣,只得便去。小人自來也不曾到東京,就那裡觀光上國景致,走一遭,也是恩相抬舉。」知縣大喜,賞了武松三盃酒,十兩路費,不在話下。

\end{withgezhu}




%\begin{withgezhu}

%\chapter{测试 3 (3) test}
%(GEZHU包括 chapter)


%話說武松自從搬離哥後,撚指不覺雪晴,過了十數日光景。都說本縣知縣,自從到任以來,都得二年有餘,
%轉得許多金銀,
%\gezhu{(轉得許多金銀)「轉」容本《忠義水滸傳》做「撰」,稍後如楊定見本、芥子園本、金聖歎貫華堂七十回本水滸傳均做「賺」。而前此世德堂本《西遊記》,已用「賺」代「撰」。本書「轉」、「撰」、「賺」並用。第五十三回:「賺得些中錢,又來撒漫了」;第七十六回:「家中胡亂積賺了些小本經紀」;第九十八回:「別無生意,只靠老婆錢賺」。下凡「轉」、「撰」取錢物,均統一為「賺」,不再一一出校。}
%\marginnote{good}
%要使一心腹人,送上東京親眷處收寄。三年任滿朝覲,打點上司。一來都怕路上小人,須得一個有力量的人去方好。猛可想起都頭武松,須得此人英雄膽力,方了得此事。當日就喚武松到衙內商議,道:「我有個親戚,在東京城內做官,姓朱名勔,見做殿前太尉之職。要送一擔禮物,
%稍封書
%\footnote{(稍封書)「稍」此處代「捎」。劉改「捎」。下凡以「稍」代「捎」,隨文改正,不再出校。}
%去問安。只恐途中不好行,須得你去方可。你休推辭辛苦,回來我自重賞你!」武松應道:「小人得蒙恩相抬舉,安敢推辭?既蒙差遣,只得便去。小人自來也不曾到東京,就那裡觀光上國景致,走一遭,也是恩相抬舉。」知縣大喜,賞了武松三盃酒,十兩路費,不在話下。

%\end{withgezhu}

\fi
}
