%# -*- coding: utf-8 -*-
%!TEX encoding = UTF-8 Unicode
%!TEX TS-program = xelatex
% vim:ts=4:sw=4
%
% 以上設定默認使用 XeLaTex 編譯,並指定 Unicode 編碼,供 TeXShop 自動識別

\chapter*{制作說明}
\addcontentsline{toc}{chapter}{制作說明}

本電子書嘗試提供一個基本無錯誤的文學著作底本。
其特點在於,使用同一個精校文件數據源的情況下,自動生成所需要的PDF閱讀輸出版式,
而不必像使用 M\$ Word 那樣,需要針對輸出不同版面手工分別調整。
這樣可以達到事半而功倍的效果。

基於以上思想,本項目能夠自動化生成如下任意組合的PDF輸出文件:
\begin{enumerate}
  \item 任意紙張大小。可以根據打印或閱讀需求,如電子書閱讀器,以合適的大小輸出頁面;
  \item 頁面可以在直式、橫式中選擇;
  \item 可以生成含有或者不含批評、校記、注疏的版本;
  \item 批評、校記、注疏的排版位置可以是腳注或者割注的形式;
  \item 內容排版可以是橫排或者豎排;
\end{enumerate}

%而更關鍵是,本項目將采取開源方式運作,目的是讓大家都能參與進來,使其效果最大化。


本項目所服務的目標群體有

\begin{itemize}
  \item 文學研究者,提供精准的名著底本,同時輔助以各家批評、校記等以助研究。真正做到一冊在手,別無所求。
  \item 語言研究者,名著中提供供了豐富的語言素材。
  \item 一般讀者,提供經過專家校正過的版本,輔之以精心設計的版面,欣賞原汁原味的文學名著。
\end{itemize}



本項目計劃:

\begin{enumerate}
  \item 實現可輸出各種版面的工具框架;(初版已经完成) %合適可用的工具框架,可以融合各個參與者的貢獻
  \item 提供精校的名著底本;
  \item 開發工具,使之支持用原刻本上的字形排版精校底本;一方面滿足部分人懷舊之需,另外更重要是,可以用於對比來輔助校對文本。
\end{enumerate}


本項目需要的開發人員:

\begin{itemize}
  \item 文學研究者,校對名著底本;
  \item 語言研究者,校對、對其中的疑難提出校對意見;
  \item 美工,封面設計、字體搭配等
  \item 計算機程序人員,程序腳本開發;%\LaTeX 開發等。
  \item 其他人員,可以提出、改進任何你認為需要改進的地方。
\end{itemize}


項目籌備聯系郵件:\url{mailto:yhfudev@gmail.com}


\section*{凡例}

\jpmShowZhuInfo


%如需本書電子版最新版,請關注本項目開發主頁 \url{https://github.com/}
