%# -*- coding: utf-8 -*-
%!TEX encoding = UTF-8 Unicode
%!TEX TS-program = xelatex
% vim:ts=4:sw=4
%
% 以上设定默认使用 XeLaTex 编译,并指定 Unicode 编码,供 TeXShop 自动识别

\chapter*{金瓶梅出版情况}
\addcontentsline{toc}{chapter}{金瓶梅出版情况}

金瓶梅词话100回

(明)兰陵笑笑生著;梅节校订、陈诏、黄霖注释

香港:梦梅馆,1993

17X约48;1375页;200图

该出版社于1988年出版了全校本《金瓶梅词话》,被誉为最接近《金瓶梅》原本。此重校本对全校本的校订,主要有以下几个方面:原本不误,全校本误改的;原本有误,全校本只部分改正,并不彻底的;原本有误,全校本存而未改的;原本有误,全校本未发现,或虽发现而尚未悉致误之由,无法订正的。由陈诏和黄霖作注释,最后经梅节审定增删。书前附有梅节所作的〈全校本金瓶梅词话前言〉,内容与香港星海文化出版有限公司 于1987年出版的《金瓶梅词话》(全校万历本)梅节所作的〈全校本金瓶梅词话前言〉大致相同,主要交代所根据的本子及校点原则。重校本直接删去衍文,不再用圆括号()标记;增文也不用方斜括号〔〕作标记。崇祯 本200幅插图,全校本原附在卷末,此本则分插在每回之前。此重校本增加了陈诏、黄霖两家的注释,删去全校万历本的〈词典〉。




金瓶梅词话(全校万历本)100回

(明)兰陵笑笑生著

香港:星海文化出版有限公司,1987

17X约47;1300页;图200幅

据梅节的〈全校本金瓶梅词话前言〉所说,此书以日本大安株式会社出版的《金瓶梅词话》为底本,原系根据日光山轮王寺慈眼堂及德山毛利氏栖息堂所藏《金瓶梅词话》补配。参考台北联经朱墨二色套印本和北京文学古籍刊行 社1957年重印本,校以以下几个本子:

1)《新刻绣像批评金瓶梅》,台北天一出版社影印日本内阁文库藏本。此本旧称“崇祯本”,不确,今简称“廿卷本”。(本馆藏有此影印本)

2)《皋鹤堂批评第一奇书金瓶梅》,在兹堂本,简称“大字张本”。台北里仁书局影印本。

3)《皋鹤堂批评第一奇书金瓶梅》,崇经堂巾箱本,简称“小字张本”,香港文乐出版社与在兹堂本合印本。

此本同时还参考了几种近人点校的本子:

1)《中国文学珍本丛书》本《金瓶梅词话》(删节本),施蛰存点校,上海杂志出版公 司1936年出版,简称“施本”。

2)全标点本《金瓶梅词话》,毛子水序,台北增你智文化事业有限公 司1976年出版,简称“增本”。

3)《中国文学名著》本《金瓶梅》(删节本),刘本栋校订,台北三民书 局1980年出版,简称“刘本”。

4)《中国小说史料丛书》本《金瓶梅词话》(删节本),戴鸿森校点,北京人民文学出版 社1985年出版,简称“戴本”。

〈校记〉共五千余条,分系各回之后。正文校改、增文二字以上用方斜括号〔〕,衍文二字以上用圆括号(),夹文用方括 号[ ],阙文用方框□,书名及词曲牌名用尖括号〈〉,书末附二十卷本的明代插图和梅节主编的〈金瓶梅词话辞典〉。该出版社还另外出版全校本的普及本,不附校记。





金瓶梅词话万历本2函20册

(明)兰陵笑笑生

台北:联经出版事业公司,1978

11X24;1579页;图200幅

据此本的〈出版说明〉,万历丁巳(1617)刊本《金瓶梅词话》是《金瓶梅》一书现存的最早木刻版本, 共20册,目前珍藏在台北故宫博物院。1933年,北京古佚小说刊行会据以影印一百部行世。

这部联经版的《金瓶梅词话》,就是依据傅斯年先生所藏古佚小说刊行会影印本,并比对故宫博物院珍藏的万历丁巳(1617)本,整理后影印。

此本原无插图,北平古佚小说刊行会影印本补入崇祯本木刻插图200幅。每回各两图,在每回之前,与马廉藏崇祯本《新刻绣像金瓶梅》相同。此本书眉及行间间有朱笔批语。




金瓶梅词话(万历本)6卷

(明)笑笑生作

东京:大安株式会社,1963

11X24;2941页;图200幅

据此本的〈例言〉,“吾邦所传明刊本金瓶梅词话之完全者有两部,日光轮王寺慈眼堂所藏本与德山毛利氏栖息堂所藏本者是也。两者仅 第5回末叶异版。今以慈眼堂所藏本认为初版,附栖息堂所藏本书影于第一卷末。今以两部补配完整,一概据原刊本而不妄加臆改。至于原本文字不详之处,于卷末附一表。”

此本卷1至5为正文,卷6为插图,插图甚为清晰。 卷6前有〈目录〉,题“清宫珍宝縫美图总目”。



金瓶梅词话100回

(明)笑笑生

(美国国会图书馆影前北平图书馆善本RollNo.615616)

11X24;图200幅

此本为明万历间刻本,有序,序末作“欣欣子书于明一贤里之轩”。有〈金瓶梅序〉,于万历丁巳(1617)季冬东吴弄珠客漫书于金阊道中;有〈新刻金瓶梅词话目录〉,书末有附图。

藏中央图书馆缩微胶卷复制部(前南洋大学图书馆图书登记簿):1919-1920

