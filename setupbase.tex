%# -*- coding: utf-8 -*-
%!TEX encoding = UTF-8 Unicode
%!TEX TS-program = xelatex
% vim:ts=4:sw=4
%
% 以上设定默认使用 XeLaTex 编译,并指定 Unicode 编码,供 TeXShop 自动识别
%


%%%%%%%%%%%%%%%%%%%%%%%%%%%%%%%%%%%%%%%%%%%%%%%%%%%%%%%%%%%%%%%%%%
\usepackage{ifthen}
\usepackage{ifpdf}
\usepackage{ifxetex}
\usepackage{ifluatex}

\ifxetex % xelatex
\else
    %The cmap package is intended to make the PDF files generated by pdflatex "searchable and copyable" in acrobat reader and other compliant PDF viewers.
    \usepackage{cmap}%
\fi
% ============================================
% Check for PDFLaTeX/LaTeX
% ============================================
\newcommand{\outengine}{xetex}
\newif\ifpdf
\ifx\pdfoutput\undefined
  \pdffalse % we are not running PDFLaTeX
  \ifxetex
    \renewcommand{\outengine}{xetex}
  \else
    \renewcommand{\outengine}{dvipdfmx}
  \fi
\else
  \pdfoutput=1 % we are running PDFLaTeX
  \pdftrue
  \usepackage{thumbpdf}
  \renewcommand{\outengine}{pdftex}
  \pdfcompresslevel=9
\fi
\usepackage[\outengine,
    bookmarksnumbered, %dvipdfmx
    %% unicode, %% 不能有unicode选项,否则bookmark会是乱码
    colorlinks=true,
    citecolor=red,
    urlcolor=blue,        % \href{...}{...} external (URL)
    filecolor=red,      % \href{...} local file
    linkcolor=black, % \ref{...} and \pageref{...}
    breaklinks,
    pdftitle={\mydoctitle},
    pdfauthor={\mydocauthor},
    pdfsubject={\mydocsubject},
    pdfkeywords={\mydockeywords},
    pdfproducer={\mypdfproducer},
    pdfcreator={\mypdfcreator},
    %%pdfadjustspacing=1,
    pdfstartview={XYZ null null 1}, pdfborder=001,
    pdfpagemode=UseNone,
    pagebackref,
    pdfprintscaling=None,
    bookmarksopen=true]{hyperref}

% --------------------------------------------
% Load graphicx package with pdf if needed 
% --------------------------------------------
\ifxetex    % xelatex
    \usepackage{graphicx}
\else
    \ifpdf
        \usepackage[pdftex]{graphicx}
        \pdfcompresslevel=9
    \else
        \usepackage{graphicx} % \usepackage[dvipdfm]{graphicx}
    \fi
\fi


%%%%%%%%%%%%%%%%%%%%%%%%%%%%%%%%%%%%%%%%%%%%%%%%%%%%%%%%%%%%
%\let\oldchapter\chapter% Store \chapter in \oldchapter
%\newcounter{mycounter}
%\renewcommand{\chapter}{%
  %\setcounter{mycounter}{2}% Insert "your content" here
  %\oldchapter%
%}


%%%%%
%\makeatletter % create AtBeginChapter analog to AtBeginDocument
%\def\AtBeginChapter{\g@addto@macro\@beginchapterhook}
%\@onlypreamble\AtBeginChapter

%\let\oldchapter\chapter
%\long\def\chapter{\@beginchapterhook
  %\oldchapter}

%\ifx\@beginchapterhook\@undefined
  %\let\@beginchapterhook\@empty
%\fi
%\makeatother


%%%%%

%\newcommand{\myBeginChapter}{}
%\newcommand{\AtBeginChapter}[1]{%
    %\renewcommand{\myBeginChapter}{#1}%
%}

%\newcommand{\myEndChapter}{}
%\newcommand{\AtEndChapter}[1]{%
    %\renewcommand{\myEndChapter}{#1}%
%}

%\makeatletter

%\newenvironment{tablehere}
%{\def\@captype{table}}
%{}

%\let\original@chapter\chapter
%\def\@first@chapter{1}
%\renewcommand{\chapter}{%
    %\ifnum\@first@chapter=1 \gdef\@first@chapter{0}\else\myEndChapter\fi
    %\original@chapter
    %\myBeginChapter}
%\makeatother

%\AtEndChapter{\centerline{$***$}}



%%%%%%%%%%%%%%%%%%%%%%%%%%%%%%%%%%%%%%%%%%%%%%%%%%%%%%%%%%%%%%%%%%


%\usepackage{gezhu}
%\setgezhulines{2}
%\everygezhu{\fontsize{6}{7}\selectfont}
%\setgezhuraise{-1.5pt}

%\usepackage{atbegshi}
%\AtBeginShipout{%
  %\global\setbox\AtBeginShipoutBox\vbox{%
    %\special{pdf: put @thispage <</Rotate 90>>}%
    %\box\AtBeginShipoutBox
  %}%
%}% \usepackage{everyshi} %\EveryShipout{\special{pdf: put @thispage <</Rotate 90>>}}


%\newenvironment{showcasevert}[1]{%
  %\begin{minipage}{0.8\textwidth}
  %#1\par
  %\setbox0=\vbox\bgroup
  %\hsize=16em
  %\begin{withgezhu}
%}{%
  %\end{withgezhu}
  %\egroup
  %\hskip 0.5em\rotatebox{-90}{\copy0}
  %\bigskip
  %\end{minipage}
%}


%\AtBeginChapter{\begin{showcasevert}{}}
%\AtEndChapter{\end{showcasevert}}






%%%%%%%%%%%%%%%%%%%%%%%%%%%%%%%%%%%%%%%%%%%%%%%%%%%%%%%%%%%%
%\usepackage[perpage]{footmisc} % 让它每页计数, 本包导致 herf 不工作;用下面的方法替代
%   脚注的行号默认是从每一章开始计数的,现在要求从每一页开始计数
\makeatletter\@addtoreset{footnote}{page}
\makeatother

%%%%%%%%%%%%%%%%%%%%%%%%%%%%%%%%%%%%%%%%%%%%%%%%%%%%%%%%%%%%
\usepackage{pifont}
%\renewcommand{\thefootnote}{\textcircled{\tiny\arabic{footnote}}}
%\renewcommand{\thefootnote}{\ding{\numexpr171+\value{footnote}}}
\makeatletter
\renewcommand\thefootnote{\ding{\numexpr171+\value{footnote}}} % 脚注中的脚注序号不用上标,正文中的脚注号保持不变
\def\my@makefnmark{\hbox{\normalfont\@thefnmark\space}}
\let\my@save@makefntext\@makefntext
\long\def\@makefntext#1{{%
  \let\@makefnmark\my@makefnmark
  \my@save@makefntext{#1}}}
\makeatother



%\cndocPageWidth{160mm}%{145mm}
%\cndocPageHeight{220mm}%{210mm}
\ifnum\strcmp{\mymargin}{\detokenize{none}}=0 \else
    \geometry{margin = \mymargin}
\fi


\ifnum\strcmp{\mybackground}{\detokenize{none}}=0
    \definecolor{mybackgroundcolor}{cmyk}{0.03,0.03,0.18,0}
    \pagecolor{mybackgroundcolor}
\else
    \pagecolor{\mybackground}
\fi





%%%%%%%%%%%%%%%%%%%%%%%%%%%%%%%%%%%%%%%%%%%%%%%%%%%%%%%%%%%%

\ifnum\myusecjk=1
    \usepackage[nofonts]{ctex} %adobefonts
    %\usepackage[fallback]{xeCJK}

    \xeCJKsetup{AutoFallBack = true} % you have to use this, since fallback won't work as the xeCJK option after the ctex

    \PassOptionsToPackage{
        BoldFont,  % 允許粗體
        SlantFont, % 允許斜體
        CJKnumber,
        CJKtextspaces,
        }{xeCJK}
    \defaultfontfeatures{Mapping=tex-text} % 如果沒有它,會有一些 tex 特殊字符無法正常使用,比如連字符。

    \xeCJKsetup {
      CheckSingle = true,
     AutoFakeBold = false,
    AutoFakeSlant = false,
        CJKecglue = {},
       PunctStyle = kaiming,
    KaiMingPunct+ = {:;},
    }

    \PassOptionsToPackage{CJKchecksingle}{xeCJK}
    %\defaultCJKfontfeatures{Scale=0.5}
    %\LoadClass[c5size,openany,nofonts]{ctexbook}
    \XeTeXlinebreaklocale "zh"                      % 重要,使得中文可以正確斷行!
    \XeTeXlinebreakskip = 0pt plus 1pt minus 0.1pt  % 给予TeX断行一定自由度
    \linespread{1.3}                                % 1.3倍行距

\else
    \usepackage[fallback]{xeCJK}
\fi


\ifnum\strcmp{\myclinemode}{\detokenize{vertical}}=0
    \newfontlanguage{mylanguage}{CHN}
    % defaultCJKfontfeatures 要在 setCJKmainfont 之前设置,要不然不起作用
    \defaultCJKfontfeatures{Script=CJK}
    \defaultCJKfontfeatures{Language=mylanguage}
    %\defaultCJKfontfeatures{Vertical=RotatedGlyphs}
    \defaultCJKfontfeatures{RawFeature={vertical:+vert}}%{RawFeature={vertical:+vert:+vhal}}
\fi


%\xeCJKsetup{PunctStyle=gzvert}
% 设置标点
%\xeCJKsetwidth{。?}{0.7em}
%\xeCJKsetkern{:}{“}{0.3em}
%\xeCJKDeclarePunctStyle { gzvert }
  %{
    %fixed-punct-ratio = nan,
    %fixed-margin-width = 0 pt,
    %mixed-margin-width = \maxdimen,
    %mixed-margin-ratio = 0.5,
    %middle-margin-width = \maxdimen,
    %middle-margin-ratio = 0.5,
    %add-min-bound-to-margin = true,
    %bound-punct-width = 0 em,
    %enabled-hanging = true,
    %min-bound-to-kerning = true,
    %kerning-margin-minimum = 0.1 em
  %}

% 3000-303F, CJK标点符号, http://www.unicode.org/charts/PDF/U3000.pdf
% FF00-FFEF, 全角ASCII、全角中英文标点、半宽片假名、半宽平假名、半宽韩文字母, http://www.unicode.org/charts/PDF/UFF00.pdf
% FE10-FE1F, 中文竖排标点, http://www.unicode.org/charts/PDF/UFE10.pdf
% FE30-FE4F, CJK兼容符号(竖排变体、下划线、顿号), http://www.unicode.org/charts/PDF/UFE30.pdf
\xeCJKDeclareSubCJKBlock{zhbiaodian}{ "3000 -> "303F, "FF00 -> "FFEF, "FE10 -> "FE1F, "FE30 -> "FE4F, }


% fonts:

%# -*- coding: utf-8 -*-
%!TEX encoding = UTF-8 Unicode
%!TEX TS-program = xelatex
% vim:ts=4:sw=4
%
% 以上设定默认使用 XeLaTex 编译,并指定 Unicode 编码,供 TeXShop 自动识别
%

%%%%%%%%%%%%%%%%%%%%%%%%%%%%%%%%%%%%%%%%%%%%%%%%%%%%%%%%%%%%
% setup fonts
\newcommand\myfontdir{fonts/}

\newcommand\mycjkfzqk{fzqkbyss-1.00.ttf}
\newcommand\mycjknoto{SourceHanSans-Medium.otf}
\newcommand\mycjkkhangxi{khangxidict-1.023-full.otf}

\newcommand\mycjkming{mingliu-3.00-20923.ttc}
\newcommand\mycjkmingext{mingliub-7.01.ttc}
\newcommand\mycjknotol{SourceHanSans-Light.otf}


\newcommand\myFntAdobeKaiti{AdobeKaitiStd-Regular.otf}%{Adobe Kaiti Std}
\newcommand\myFntAdobeHeiti{AdobeHeitiStd-Regular.otf}%{Adobe Heiti Std}
\newcommand\myFntAdobeFangsong{AdobeFangsongStd-Regular.otf}%{Adobe Fangsong Std}
\newcommand\myFntAdobeMing{AdobeMingStd-Light.otf}%{Adobe Ming Std}
\newcommand\myFntAdobeSong{AdobeSongStd-Light.otf}%{Adobe Song Std}



\newcommand\mytestfontbbb{Adobe Kaiti Std}

\newcommand\FntNameFzbwksgb{\mytestfontbbb}  % 方正北魏楷书
\newcommand\FntNameFzzdxgb{\mytestfontbbb}
\newcommand\FntNameFzzdx{\mytestfontbbb} % 方正中等线
\newcommand\FntNameFzlth{Adobe Heiti Std}         % 方正兰亭黑扁
\newcommand\FntNameFzydh{\mytestfontbbb}          % 方正韵动中黑
\newcommand\FntNameFzys{Adobe Heiti Std}             % 方正中雅宋
\newcommand\FntNameFzqt{\mytestfontbbb}                  % 方正启体
\newcommand\FntNameFzsxslk{\mytestfontbbb}% 方正苏新诗柳楷简体



%\newcommand\FntNameFzqt{FZQiTi-S14}                  % 方正启体
\renewcommand\FntNameFzqt{Adobe Kaiti Std}                  % 方正启体

% 配置: 使用旧明体作为缺省字体,辅助用新明体补全字形
\newcommand\mytestfont{\mycjkming}%{TypeLand.com 康熙字典體}%{Adobe Heiti Std}
\newcommand\mycjkfallbackfontA{\mycjkmingext}%{TypeLand.com 康熙字典體}%{Adobe Heiti Std}
\newcommand\mycjkfallbackfontB{\mycjkkhangxi}%{TypeLand.com 康熙字典體}%{Adobe Heiti Std}
\newcommand\mycjkfallbackfontE{\myFntAdobeFangsong}%{Noto Sans S Chinese}%{Adobe Heiti Std}

\newcommand\mycjkboldfont{\mytestfont}%{TypeLand.com 康熙字典體}%{Adobe Heiti Std}
\newcommand\mycjkitalicfont{\mytestfont}%{FZKaiT-Extended}%{全字庫正楷體}
\newcommand\mycjkmainfont{\mytestfont}%{Adobe Ming Std}%{花園明朝A}%{TypeLand.com 康熙字典體}%{新細明體}
\newcommand\mycjksansfont{\mytestfont}%{Adobe Ming Std}%{花園明朝A}%{TypeLand.com 康熙字典體}%{新細明體}
\newcommand\mycjkmonofont{\mytestfont}%{WenQuanYi Micro Hei Mono}

%%%%%%%%%%%%%%%%%%%%%%%%%%%%%%%%%%%%%%%%%%%%%%%%%%%%%%%%%%%%
%\cndocFntForTitle{}
%\cndocFntForMain{}




    % 配置: 使用Adobe作为缺省字体,辅助用新明体补全字形,最后用SourceHanSans扫尾。
    \renewcommand\mytestfont{\myFntAdobeFangsong}
    \renewcommand\mycjkfallbackfontA{\mycjkmingext}
    \renewcommand\mycjkfallbackfontB{\mycjkkhangxi}
    \renewcommand\mycjkfallbackfontE{\mycjknotol}

    \renewcommand\mycjkboldfont{\myFntAdobeHeiti}
    \renewcommand\mycjkitalicfont{\myFntAdobeKaiti}
    \renewcommand\mycjkmainfont{\myFntAdobeFangsong}
    \renewcommand\mycjksansfont{\myFntAdobeHeiti}
    \renewcommand\mycjkmonofont{\myFntAdobeFangsong}



    % 配置: 使用Noto Sans S Chinese作为缺省字体,辅助用康熙字典體补全字形,最后用SourceHanSans扫尾。
    \renewcommand\mytestfont{\mycjknotol}%{TypeLand.com 康熙字典體}%{Adobe Heiti Std}
    \renewcommand\mycjkfallbackfontA{\mycjkkhangxi}%{TypeLand.com 康熙字典體}%{Adobe Heiti Std}
    \renewcommand\mycjkfallbackfontB{\mycjkmingext}%{TypeLand.com 康熙字典體}%{Adobe Heiti Std}
    \renewcommand\mycjkfallbackfontE{\mycjknotol}%{Noto Sans S Chinese}%{Adobe Heiti Std}

    \renewcommand\mycjkboldfont{\mytestfont}%{TypeLand.com 康熙字典體}%{Adobe Heiti Std}
    \renewcommand\mycjkitalicfont{\mytestfont}%{FZKaiT-Extended}%{全字庫正楷體}
    \renewcommand\mycjkmainfont{\mytestfont}%{Adobe Ming Std}%{花園明朝A}%{TypeLand.com 康熙字典體}%{新細明體}
    \renewcommand\mycjksansfont{\mytestfont}%{Adobe Ming Std}%{花園明朝A}%{TypeLand.com 康熙字典體}%{新細明體}
    \renewcommand\mycjkmonofont{\mytestfont}%{WenQuanYi Micro Hei Mono}


\ifnum\strcmp{\myclinemode}{\detokenize{vertical}}=0
    % 配置: 使用方正清刻体作为缺省字体,辅助用康熙字典體补全字形,最后用SourceHanSans扫尾。
    \renewcommand\mytestfont{\mycjkfzqk}%{TypeLand.com 康熙字典體}%{Adobe Heiti Std}
    \renewcommand\mycjkfallbackfontA{\mycjkkhangxi}%{TypeLand.com 康熙字典體}%{Adobe Heiti Std}
    %\renewcommand\mytestfont{\mycjkkhangxi}%{TypeLand.com 康熙字典體}%{Adobe Heiti Std}
    %\renewcommand\mycjkfallbackfontA{\mycjkfzqk}%{TypeLand.com 康熙字典體}%{Adobe Heiti Std}
    \renewcommand\mycjkfallbackfontB{\mycjkmingext}%{TypeLand.com 康熙字典體}%{Adobe Heiti Std}
    \renewcommand\mycjkfallbackfontE{\mycjknotol}%{Noto Sans S Chinese}%{Adobe Heiti Std}

    \renewcommand\mycjkboldfont{\mytestfont}%{TypeLand.com 康熙字典體}%{Adobe Heiti Std}
    \renewcommand\mycjkitalicfont{\mytestfont}%{FZKaiT-Extended}%{全字庫正楷體}
    \renewcommand\mycjkmainfont{\mytestfont}%{Adobe Ming Std}%{花園明朝A}%{TypeLand.com 康熙字典體}%{新細明體}
    \renewcommand\mycjksansfont{\mytestfont}%{Adobe Ming Std}%{花園明朝A}%{TypeLand.com 康熙字典體}%{新細明體}
    \renewcommand\mycjkmonofont{\mytestfont}%{WenQuanYi Micro Hei Mono}

\fi



% 用於章節中夾注評論批評
\newCJKfontfamily{\fntfmyJiaozhu}{Adobe Ming Std}%{{[Path=\myfontdir,]{\mycjkkhangxi}}}  % 校注字体

% 用於章節中整段整段的評論批評
\newCJKfontfamily{\fntfmyPinlun}{Adobe Kaiti Std}%{{[Path=\myfontdir,]{\mycjkkhangxi}}}  % 校注字体



\newCJKfontfamily{\fzwkai}   {\FntNameFzbwksgb}  % 方正北魏楷书
\newCJKfontfamily{\fzzhdxian}{\FntNameFzzdxgb} % 方正中等线
\newCJKfontfamily{\fzltheib} {\FntNameFzlth}         % 方正兰亭黑扁
\newCJKfontfamily{\fzydzhhei}{\FntNameFzydh}          % 方正韵动中黑
\newCJKfontfamily{\fzzhysong}{\FntNameFzys}             % 方正中雅宋
\newCJKfontfamily{\fzqiti}   {\FntNameFzqt}                  % 方正启体
\newCJKfontfamily{\fzliukai}[GB18030=\FntNameFzqt]{\FntNameFzsxslk}% 方正苏新诗柳楷简体


\setCJKmainfont[Path=\myfontdir,
    %FallBack=\mycjkfallbackfontE,
    zhbiaodian={[Path=\myfontdir,]{\mycjkkhangxi}},%mycjkfzqk mycjknoto mycjkkhangxi mycjkming mycjknotol myFntAdobeKaiti myFntAdobeHeiti myFntAdobeFangsong myFntAdobeMing myFntAdobeSong
    BoldFont=\mycjkboldfont,
    ItalicFont=\mycjkitalicfont,
    ItalicFeatures={RawFeature={slant=0.17}}, % 定义 \itshape 为斜体
    BoldItalicFeatures={RawFeature={slant=0.17}}, % 定义 \bfseries\itshape 为斜体
    AutoFakeBold,AutoFakeSlant]{\mycjkmainfont}

% FallBack 的值为空,将设置 备用字体
\setCJKmainfont[FallBack,Path=\myfontdir,]{\mycjkfallbackfontE}
\addCJKfontfeatures{Scale=1.5}

\setCJKsansfont[Path=\myfontdir,
    FallBack=\mycjkfallbackfontE,
    BoldFont=\mycjkboldfont,
    ItalicFont=\mycjkitalicfont,
    ItalicFeatures={RawFeature={slant=0.17}}, % 定义 \itshape 为斜体
    BoldItalicFeatures={RawFeature={slant=0.17}}, % 定义 \bfseries\itshape 为斜体
    AutoFakeBold,AutoFakeSlant]{\mycjksansfont}

%\setCJKfallbackfamilyfont{\CJKrmdefault}
%  [BoldFont={FZHei-B01_GB18030},ItalicFont={FZKai-Z03_GB18030}]{FZShuSong-Z01_GB18030}
%\setCJKfallbackfamilyfont{\CJKsfdefault}
%  [BoldFont={FZHei-B01_GB18030},ItalicFont={FZKai-Z03_GB18030}]{FZLanTingHei-R_GB18030_YS}
%\setCJKfallbackfamilyfont{\CJKttdefault}
%  [BoldFont={FZHei-B01_GB18030},ItalicFont={FZKai-Z03_GB18030}]{FZFangSong-Z02_GB18030}

\setCJKfallbackfamilyfont{\CJKrmdefault}[AutoFakeSlant]{
%\setCJKfallbackfamilyfont{rm}[AutoFakeSlant]{
    [Path=\myfontdir]{\mycjkfallbackfontA},
    [Path=\myfontdir]{\mycjkfallbackfontB},
    [Path=\myfontdir]{\mycjkfallbackfontE}
}
\setCJKfallbackfamilyfont{\CJKsfdefault}[AutoFakeSlant]{
    [Path=\myfontdir]{\mycjkfallbackfontA},
    [Path=\myfontdir]{\mycjkfallbackfontB},
    [Path=\myfontdir]{\mycjkfallbackfontE}
}
\setCJKfallbackfamilyfont{\CJKttdefault}[AutoFakeSlant]{
    [Path=\myfontdir]{\mycjkfallbackfontA},
    [Path=\myfontdir]{\mycjkfallbackfontB},
    [Path=\myfontdir]{\mycjkfallbackfontE}
}




%%%%%%%%%%%%%%%%%%%%%%%%%%%%%%%%%%%%%%%%%%%%%%%%%%%%%%%%%%%%

%\geometry{
    %papersize={145mm,210mm},
    %textwidth=110mm,
    %lines=30,
    %inner=15mm,
    %top=20mm,
    %bindingoffset=5mm,
    %headheight=10mm,
    %headsep=6mm,
    %foot=5mm}

\geometry{
    papersize={\mypagewidth,\mypageheight},
    }

\newcommand\tmprotated{0}
\ifnum\strcmp{\mypageorien}{\detokenize{landscape}}=0
    \renewcommand\tmprotated{1}
\fi

\ifnum\strcmp{\myclinemode}{\detokenize{vertical}}=0
    \ifnum\strcmp{\mypageorien}{\detokenize{landscape}}=0
    \else % fix page, fake portrait
        \renewcommand\tmprotated{1}
        \geometry{
            papersize={\mypageheight,\mypagewidth},%{\mypagewidth,\mypageheight},
            }
    \fi
\fi


\ifnum\tmprotated > 0
    % 页面旋转90度
    \usepackage{everyshi}
    \makeatletter
    \EveryShipout{
        \global\setbox\@cclv\vbox{%
            \special{pdf: put @thispage << /Rotate 90 >>}%
            \box\@cclv
        }%
    }%
    \makeatother
\fi


\ifnum\myusecjk=1
    \usepackage{cndocstory}
\fi




\newcommand{\myfootnote}[1]{
    \footnote{#1}
}
\ifnum\strcmp{\myfnotemode}{\detokenize{gezhu}}=0
    %# -*- coding: utf-8 -*-
% !TeX encoding = UTF-8 Unicode
% !TeX spellcheck = en_US
% !TeX TS-program = xelatex
%~ \XeTeXinputencoding "UTF-8"
% vim:ts=4:sw=4
%
% 以上設定默認使用 XeLaTex 編譯,並指定 Unicode 編碼,供 TeXShop 自動識別
%
%%%%%%%%%%%%%%%%%%%%%%%%%%%%%%%%%%%%%%%%%%%%%%%%%%%%%%%%%%%%

\renewcommand\KG{{ }}

\usepackage{fancyhdr}
\pagestyle{fancyplain}
\fancyhf{}
\renewcommand{\headrulewidth}{0pt}
%\def\thepage{\Chinese{page}}
\renewcommand{\thepage}{\Chinese{page}}
\fancyfoot[R]{\thepage}






\usepackage{gezhu}
%\setgezhuraise{-6pt}
%\everygezhu={\zihao{5}\linespread{.9}\selectfont\itshape}
%\expandafter\everywithgezhu\expandafter{\the\everywithgezhu \baselineskip=16pt \parskip=1em}
\everygezhu{\fontsize{6}{7}\selectfont}
\setgezhuraise{-1.5pt}
\setgezhulines{2}

%\parindent=2em
\setgezhushipoutlevel{2}
%\gezhuraggedfalse
%\gezhuraggedtrue
\gezhunormalizetrue


%\def\skipzhfullstop{\hskip 0pt plus 0.3em}






\comments{
% 使用 footnote 替代 \gezhu
%\makeatletter
%\let\oldfootnote\footnote
%\renewcommand\footnote[1]{\gezhu{#1}} % {\gezhu{\fntfmyJiaozhu\tiny #1}}
%\makeatother

%\makeatletter
%\let\oldmarginnote\marginnote
%\renewcommand\marginnote[1]{\gezhu{#1}} % {\gezhu{\fntfmyJiaozhu\tiny #1}}
%\makeatother

}%comments

\comments{

\def\myonkyoh#1{%
  \unless\ifdefined\isfirsttime
    \def\isfirsttime{yes}%
    \withgezhu
  \fi
}

% 设置章节
% http://wiki.lyx.org/Tips/ModifyChapterEtc
\let\oldchap=\chapter
\renewcommand*{\chapter}{%
  \secdef{\Chap}{\ChapS}%
}
%\newcommand\ChapS[1]{\singlespacing\oldchap*{#1}\doublespacing}
%\newcommand\Chap[2][]{\singlespacing\oldchap[#1]{#2}\doublespacing}
\newcommand\ChapS[1]{\oldchap*{#1}\myonkyoh{#1}}
\newcommand\Chap[2][]{\oldchap{#2}\myonkyoh{#2}}
%\newcommand\ChapS[1]{\ifdefined\gezhu\endwithgezhu\fi\clearpage\oldchap*{#1}\withgezhu}
%\newcommand\Chap[2][]{\ifdefined\gezhu\endwithgezhu\fi\clearpage\oldchap{#2}\withgezhu}

%\newcommand\ChapS[1]{\ifdefined\gezhu\endwithgezhu\fi\oldchap*{#1}\withgezhu}
%\newcommand\Chap[2][]{\ifdefined\gezhu\endwithgezhu\fi\oldchap[#1]{#2}\withgezhu}

%\newcommand\ChapS[1]{\ifdefined\gezhu\endwithgezhu\fi\oldchap*{#1}\withgezhu}
%\newcommand\Chap[2][]{\ifdefined\gezhu\endwithgezhu\fi\oldchap[#1]{#2}\withgezhu}

\def\endmyonkyoh{\ifdefined\gezhu\endwithgezhu\fi}

}



\comments{

\newcommand{\BeginChapter}{}
\newcommand{\AtBeginChapter}[1]{%
    \renewcommand{\BeginChapter}{#1}%
}

\newcommand{\EndChapter}{}
\newcommand{\AtEndChapter}[1]{%
    \renewcommand{\EndChapter}{#1}%
}

\makeatletter

\newenvironment{tablehere}
{\def\@captype{table}}
{}

\let\original@chapter\chapter
\def\@first@chapter{1}
\renewcommand{\chapter}{%
    \ifnum\@first@chapter=1 \gdef\@first@chapter{0}\else\EndChapter\fi
    \original@chapter\BeginChapter}
\makeatother

%\AtBeginChapter{\centerline{$+++++$}}
%\AtEndChapter{\centerline{$******$}}

%\AtBeginChapter{\withgezhu}
%\AtEndChapter{\ifdefined\gezhu\endwithgezhu\fi}
}







    \renewcommand{\myfootnote}[1]{
        \gezhu{#1}
    }

    \newenvironment{showcontents}[1]{%
        \begin{withgezhu} \Huge
    }{%
        \end{withgezhu}
    }

\else
    \newenvironment{showcontents}[1]{%
    }{%
    }

\fi
