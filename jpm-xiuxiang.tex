%# -*- coding: utf-8 -*-
%!TEX encoding = UTF-8 Unicode
%!TEX TS-program = xelatex
% vim:ts=4:sw=4
%
% 以上设定默认使用 XeLaTex 编译,并指定 Unicode 编码,供 TeXShop 自动识别
%
%%%%%%%%%%%%%%%%%%%%%%%%%%%%%%%%%%%%%%%%%%%%%%%%%%%%%%%%%%%%

\newcommand\myUseOneside{0}

% 這裏使用比較“髒”的方法,以確保同時可設置單頁並滿足Latex的documentclass必須第一個出現的要求
% 消除圖片佈置頁碼混亂導致的翻倒問題
% 只要是豎排,不區分單偶頁:
\ifdefined\clinemode
  \ifnum\strcmp{\clinemode}{\detokenize{vertical}}=0
    \renewcommand\myUseOneside{1}
  \fi
\fi
\ifnum\strcmp{\jobname}{\detokenize{type-zhmdv}}=0
    \renewcommand\myUseOneside{1}
\fi
\ifnum\strcmp{\jobname}{\detokenize{type-zhgzv}}=0
    \renewcommand\myUseOneside{1}
\fi

\ifnum\myUseOneside > 0
    \documentclass[letter,11pt,onecolumn,oneside]{book}
\else
    \documentclass[letter,11pt,onecolumn]{book}
\fi
%\documentclass[a4paper,11pt,twocolumn]{article}
%\documentclass[letter,11pt,onecolumn,adobefonts,cm-default]{ctexbook}
%\documentclass[letter,11pt,onecolumn,
    %oneside % remove blank page between chapters
    %]{book}
% or remove blank pages: \let\cleardoublepage\clearpage

%%%%%%%%%%%%%%%%%%%%%%%%%%%%%%%%%%%%%%%%%%%%%%%%%%%%%%%%%%%%

\usepackage{cndocbase}

\cndocTitle{會評會校新刻繡像批評金瓶梅}
\cndocAuthor{(明)蘭陵笑笑生}
\cndocKeywords{{金瓶梅}{蘭陵笑笑生}{The Plum in the Golden Vase}{Chin P'ing Mei}}
\cndocSubject{}

\gezhuEnablePi % 批評
\gezhuEnableZhu % 注疏
\gezhuEnableJiao % 校記


%%%%%%%%%%%%%%%%%%%%%%%%%%%%%%%%%%%%%%%%%%%%%%%%%%%%%%%%%%%%

\makeatletter
\ifnum\strcmp{\jobname}{\detokenize{type-zhmd}}=0
    % 中文现代横排(注释在底部)
    \cndocUseCjk
    \cndocFnoteMode{modern}%{modern}%{gezhu}
    \cndocClineMode{horizontal}%{horizontal}%{vertical}
    \cndocPageOrien{portrait}%{portrait}%{landscape}
    %\cndocCharScale{1.2}
    %\cndocPageWidth{145mm}
    %\cndocPageHeight{210mm}

\else\ifnum\strcmp{\jobname}{\detokenize{type-zhgzv}}=0
    % 中文割注竖排
    \cndocUseCjk
    \cndocFnoteMode{gezhu}%{modern}%{gezhu}
    \cndocClineMode{vertical}%{horizontal}%{vertical}
    %\cndocPageOrien{landscape}%{portrait}%{landscape}

\else\ifnum\strcmp{\jobname}{\detokenize{type-zhgz}}=0
    % 中文割注横排
    \cndocUseCjk
    \cndocFnoteMode{gezhu}%{modern}%{gezhu}
    \cndocClineMode{horizontal}%{horizontal}%{vertical}
    \cndocPageOrien{portrait}%{portrait}%{landscape}

\else\ifnum\strcmp{\jobname}{\detokenize{type-zhmdv}}=0
    % 中文现代竖排(注释在旁边)
    \cndocUseCjk
    \cndocFnoteMode{gezhu}%{modern}%{gezhu}
    \cndocClineMode{vertical}%{horizontal}%{vertical}
    \cndocPageOrien{landscape}%{portrait}%{landscape}

\else%\ifnum\strcmp{\jobname}{\detokenize{type-en}}=0
    % default: English text with chinese chars support
    %\usepackage[fallback]{xeCJK}
    %\newfontlanguage{mylanguage}{ENG}
\fi
\fi
\fi
\fi
\makeatother

%# -*- coding: utf-8 -*-
% !TeX encoding = UTF-8 Unicode
% !TeX spellcheck = en_US
% !TeX TS-program = xelatex
%~ \XeTeXinputencoding "UTF-8"
% vim:ts=4:sw=4
%
% 以上設定默認使用 XeLaTex 編譯,並指定 Unicode 編碼,供 TeXShop 自動識別
%


%%%%%%%%%%%%%%%%%%%%%%%%%%%%%%%%%%%%%%%%%%%%%%%%%%%%%%%%%%%%%%%%%%
\usepackage{ifthen}
\usepackage{ifpdf}
\usepackage{ifxetex}
\usepackage{ifluatex}

\ifxetex % xelatex
\else
    %The cmap package is intended to make the PDF files generated by pdflatex "searchable and copyable" in acrobat reader and other compliant PDF viewers.
    \usepackage{cmap}%
\fi
% ============================================
% Check for PDFLaTeX/LaTeX
% ============================================
\newcommand{\outengine}{xetex}
\newif\ifpdf
\ifx\pdfoutput\undefined
  \pdffalse % we are not running PDFLaTeX
  \ifxetex
    \renewcommand{\outengine}{xetex}
  \else
    \renewcommand{\outengine}{dvipdfmx}
  \fi
\else
  \pdfoutput=1 % we are running PDFLaTeX
  \pdftrue
  \usepackage{thumbpdf}
  \renewcommand{\outengine}{pdftex}
  \pdfcompresslevel=9
\fi
\usepackage[\outengine,
    bookmarksnumbered, %dvipdfmx
    %% unicode, %% 不能有unicode选项,否则bookmark会是乱码
    colorlinks=true,
    citecolor=red,
    urlcolor=blue,        % \href{...}{...} external (URL)
    filecolor=red,      % \href{...} local file
    linkcolor=black, % \ref{...} and \pageref{...}
    breaklinks,
    pdftitle={\mydoctitle},
    pdfauthor={\mydocauthor},
    pdfsubject={\mydocsubject},
    pdfkeywords={\mydockeywords},
    pdfproducer={\mypdfproducer},
    pdfcreator={\mypdfcreator},
    %%pdfadjustspacing=1,
    pdfstartview={XYZ null null 1}, pdfborder=001,
    pdfpagemode=UseNone,
    pagebackref,
    pdfprintscaling=None,
    bookmarksopen=true]{hyperref}

% --------------------------------------------
% Load graphicx package with pdf if needed 
% --------------------------------------------
\ifxetex    % xelatex
    \usepackage{graphicx}
\else
    \ifpdf
        \usepackage[pdftex]{graphicx}
        \pdfcompresslevel=9
    \else
        \usepackage{graphicx} % \usepackage[dvipdfm]{graphicx}
    \fi
\fi


%%%%%%%%%%%%%%%%%%%%%%%%%%%%%%%%%%%%%%%%%%%%%%%%%%%%%%%%%%%%
%\let\oldchapter\chapter% Store \chapter in \oldchapter
%\newcounter{mycounter}
%\renewcommand{\chapter}{%
  %\setcounter{mycounter}{2}% Insert "your content" here
  %\oldchapter%
%}


%%%%%
%\makeatletter % create AtBeginChapter analog to AtBeginDocument
%\def\AtBeginChapter{\g@addto@macro\@beginchapterhook}
%\@onlypreamble\AtBeginChapter

%\let\oldchapter\chapter
%\long\def\chapter{\@beginchapterhook
  %\oldchapter}

%\ifx\@beginchapterhook\@undefined
  %\let\@beginchapterhook\@empty
%\fi
%\makeatother


%%%%%

%\newcommand{\myBeginChapter}{}
%\newcommand{\AtBeginChapter}[1]{%
    %\renewcommand{\myBeginChapter}{#1}%
%}

%\newcommand{\myEndChapter}{}
%\newcommand{\AtEndChapter}[1]{%
    %\renewcommand{\myEndChapter}{#1}%
%}

%\makeatletter

%\newenvironment{tablehere}
%{\def\@captype{table}}
%{}

%\let\original@chapter\chapter
%\def\@first@chapter{1}
%\renewcommand{\chapter}{%
    %\ifnum\@first@chapter=1 \gdef\@first@chapter{0}\else\myEndChapter\fi
    %\original@chapter
    %\myBeginChapter}
%\makeatother

%\AtEndChapter{\centerline{$***$}}



%%%%%%%%%%%%%%%%%%%%%%%%%%%%%%%%%%%%%%%%%%%%%%%%%%%%%%%%%%%%%%%%%%


%\usepackage{gezhu}
%\setgezhulines{2}
%\everygezhu{\fontsize{6}{7}\selectfont}
%\setgezhuraise{-1.5pt}

%\usepackage{atbegshi}
%\AtBeginShipout{%
  %\global\setbox\AtBeginShipoutBox\vbox{%
    %\special{pdf: put @thispage <</Rotate 90>>}%
    %\box\AtBeginShipoutBox
  %}%
%}% \usepackage{everyshi} %\EveryShipout{\special{pdf: put @thispage <</Rotate 90>>}}


%\newenvironment{showcasevert}[1]{%
  %\begin{minipage}{0.8\textwidth}
  %#1\par
  %\setbox0=\vbox\bgroup
  %\hsize=16em
  %\begin{withgezhu}
%}{%
  %\end{withgezhu}
  %\egroup
  %\hskip 0.5em\rotatebox{-90}{\copy0}
  %\bigskip
  %\end{minipage}
%}


%\AtBeginChapter{\begin{showcasevert}{}}
%\AtEndChapter{\end{showcasevert}}






%%%%%%%%%%%%%%%%%%%%%%%%%%%%%%%%%%%%%%%%%%%%%%%%%%%%%%%%%%%%
%\usepackage[perpage]{footmisc} % 让它每页计数, 本包导致 herf 不工作;用下面的方法替代
%   脚注的行号默认是从每一章开始计数的,现在要求从每一页开始计数
\makeatletter\@addtoreset{footnote}{page}
\makeatother

%%%%%%%%%%%%%%%%%%%%%%%%%%%%%%%%%%%%%%%%%%%%%%%%%%%%%%%%%%%%
\usepackage{pifont}
%\renewcommand{\thefootnote}{\textcircled{\tiny\arabic{footnote}}}
%\renewcommand{\thefootnote}{\ding{\numexpr171+\value{footnote}}}
\makeatletter
\renewcommand\thefootnote{\ding{\numexpr171+\value{footnote}}} % 脚注中的脚注序号不用上标,正文中的脚注号保持不变
\def\my@makefnmark{\hbox{\normalfont\@thefnmark\space}}
\let\my@save@makefntext\@makefntext
\long\def\@makefntext#1{{%
  \let\@makefnmark\my@makefnmark
  \my@save@makefntext{#1}}}
\makeatother



%\cndocPageWidth{160mm}%{145mm}
%\cndocPageHeight{220mm}%{210mm}
\ifnum\strcmp{\mymargin}{\detokenize{none}}=0 \else
    \geometry{margin = \mymargin}
\fi


\ifnum\strcmp{\mybackground}{\detokenize{none}}=0
    \definecolor{mybackgroundcolor}{cmyk}{0.03,0.03,0.18,0}
    \pagecolor{mybackgroundcolor}
\else
    \pagecolor{\mybackground}
\fi





%%%%%%%%%%%%%%%%%%%%%%%%%%%%%%%%%%%%%%%%%%%%%%%%%%%%%%%%%%%%

\ifnum\myusecjk=1
    \usepackage[nofonts]{ctex} %adobefonts
    %\usepackage[fallback]{xeCJK}

    \xeCJKsetup{AutoFallBack = true} % you have to use this, since fallback won't work as the xeCJK option after the ctex

    \PassOptionsToPackage{
        BoldFont,  % 允許粗體
        SlantFont, % 允許斜體
        CJKnumber,
        CJKtextspaces,
        }{xeCJK}
    \defaultfontfeatures{Mapping=tex-text} % 如果沒有它,會有一些 tex 特殊字符無法正常使用,比如連字符。

    \xeCJKsetup {
      CheckSingle = true,
     AutoFakeBold = false,
    AutoFakeSlant = false,
        CJKecglue = {},
       PunctStyle = kaiming,
    KaiMingPunct+ = {:;},
    }

    \PassOptionsToPackage{CJKchecksingle}{xeCJK}
    %\defaultCJKfontfeatures{Scale=0.5}
    %\LoadClass[c5size,openany,nofonts]{ctexbook}
    \XeTeXlinebreaklocale "zh"                      % 重要,使得中文可以正確斷行!
    \XeTeXlinebreakskip = 0pt plus 1pt minus 0.1pt  % 给予TeX断行一定自由度
    \linespread{1.3}                                % 1.3倍行距

\else
    \usepackage[fallback]{xeCJK}
\fi


\ifnum\strcmp{\myclinemode}{\detokenize{vertical}}=0
    \newfontlanguage{mylanguage}{CHN}
    % defaultCJKfontfeatures 要在 setCJKmainfont 之前设置,要不然不起作用
    \defaultCJKfontfeatures{Script=CJK}
    \defaultCJKfontfeatures{Language=mylanguage}
    %\defaultCJKfontfeatures{Vertical=RotatedGlyphs}
    \defaultCJKfontfeatures{RawFeature={vertical:+vert}}%{RawFeature={vertical:+vert:+vhal}}
\fi


%\xeCJKsetup{PunctStyle=gzvert}
% 设置标点
%\xeCJKsetwidth{。?}{0.7em}
%\xeCJKsetkern{:}{“}{0.3em}
%\xeCJKDeclarePunctStyle { gzvert }
  %{
    %fixed-punct-ratio = nan,
    %fixed-margin-width = 0 pt,
    %mixed-margin-width = \maxdimen,
    %mixed-margin-ratio = 0.5,
    %middle-margin-width = \maxdimen,
    %middle-margin-ratio = 0.5,
    %add-min-bound-to-margin = true,
    %bound-punct-width = 0 em,
    %enabled-hanging = true,
    %min-bound-to-kerning = true,
    %kerning-margin-minimum = 0.1 em
  %}

% 3000-303F, CJK标点符号, http://www.unicode.org/charts/PDF/U3000.pdf
% FF00-FFEF, 全角ASCII、全角中英文标点、半宽片假名、半宽平假名、半宽韩文字母, http://www.unicode.org/charts/PDF/UFF00.pdf
% FE10-FE1F, 中文竖排标点, http://www.unicode.org/charts/PDF/UFE10.pdf
% FE30-FE4F, CJK兼容符号(竖排变体、下划线、顿号), http://www.unicode.org/charts/PDF/UFE30.pdf
\xeCJKDeclareSubCJKBlock{zhbiaodian}{ "3000 -> "303F, "FF00 -> "FFEF, "FE10 -> "FE1F, "FE30 -> "FE4F, }


% fonts:

%# -*- coding: utf-8 -*-
%!TEX encoding = UTF-8 Unicode
%!TEX TS-program = xelatex
% vim:ts=4:sw=4
%
% 以上设定默认使用 XeLaTex 编译,并指定 Unicode 编码,供 TeXShop 自动识别
%

%%%%%%%%%%%%%%%%%%%%%%%%%%%%%%%%%%%%%%%%%%%%%%%%%%%%%%%%%%%%
% setup fonts
\newcommand\myfontdir{fonts/}

\newcommand\mycjkfzqk{fzqkbyss-1.00.ttf}
\newcommand\mycjknoto{SourceHanSans-Medium.otf}
\newcommand\mycjkkhangxi{khangxidict-1.023-full.otf}

\newcommand\mycjkming{mingliu-3.00-20923.ttc}
\newcommand\mycjkmingext{mingliub-7.01.ttc}
\newcommand\mycjknotol{SourceHanSans-Light.otf}


\newcommand\myFntAdobeKaiti{AdobeKaitiStd-Regular.otf}%{Adobe Kaiti Std}
\newcommand\myFntAdobeHeiti{AdobeHeitiStd-Regular.otf}%{Adobe Heiti Std}
\newcommand\myFntAdobeFangsong{AdobeFangsongStd-Regular.otf}%{Adobe Fangsong Std}
\newcommand\myFntAdobeMing{AdobeMingStd-Light.otf}%{Adobe Ming Std}
\newcommand\myFntAdobeSong{AdobeSongStd-Light.otf}%{Adobe Song Std}



\newcommand\mytestfontbbb{Adobe Kaiti Std}

\newcommand\FntNameFzbwksgb{\mytestfontbbb}  % 方正北魏楷书
\newcommand\FntNameFzzdxgb{\mytestfontbbb}
\newcommand\FntNameFzzdx{\mytestfontbbb} % 方正中等线
\newcommand\FntNameFzlth{Adobe Heiti Std}         % 方正兰亭黑扁
\newcommand\FntNameFzydh{\mytestfontbbb}          % 方正韵动中黑
\newcommand\FntNameFzys{Adobe Heiti Std}             % 方正中雅宋
\newcommand\FntNameFzqt{\mytestfontbbb}                  % 方正启体
\newcommand\FntNameFzsxslk{\mytestfontbbb}% 方正苏新诗柳楷简体



%\newcommand\FntNameFzqt{FZQiTi-S14}                  % 方正启体
\renewcommand\FntNameFzqt{Adobe Kaiti Std}                  % 方正启体

% 配置: 使用旧明体作为缺省字体,辅助用新明体补全字形
\newcommand\mytestfont{\mycjkming}%{TypeLand.com 康熙字典體}%{Adobe Heiti Std}
\newcommand\mycjkfallbackfontA{\mycjkmingext}%{TypeLand.com 康熙字典體}%{Adobe Heiti Std}
\newcommand\mycjkfallbackfontB{\mycjkkhangxi}%{TypeLand.com 康熙字典體}%{Adobe Heiti Std}
\newcommand\mycjkfallbackfontE{\myFntAdobeFangsong}%{Noto Sans S Chinese}%{Adobe Heiti Std}

\newcommand\mycjkboldfont{\mytestfont}%{TypeLand.com 康熙字典體}%{Adobe Heiti Std}
\newcommand\mycjkitalicfont{\mytestfont}%{FZKaiT-Extended}%{全字庫正楷體}
\newcommand\mycjkmainfont{\mytestfont}%{Adobe Ming Std}%{花園明朝A}%{TypeLand.com 康熙字典體}%{新細明體}
\newcommand\mycjksansfont{\mytestfont}%{Adobe Ming Std}%{花園明朝A}%{TypeLand.com 康熙字典體}%{新細明體}
\newcommand\mycjkmonofont{\mytestfont}%{WenQuanYi Micro Hei Mono}

%%%%%%%%%%%%%%%%%%%%%%%%%%%%%%%%%%%%%%%%%%%%%%%%%%%%%%%%%%%%
%\cndocFntForTitle{}
%\cndocFntForMain{}




    % 配置: 使用Adobe作为缺省字体,辅助用新明体补全字形,最后用SourceHanSans扫尾。
    \renewcommand\mytestfont{\myFntAdobeFangsong}
    \renewcommand\mycjkfallbackfontA{\mycjkmingext}
    \renewcommand\mycjkfallbackfontB{\mycjkkhangxi}
    \renewcommand\mycjkfallbackfontE{\mycjknotol}

    \renewcommand\mycjkboldfont{\myFntAdobeHeiti}
    \renewcommand\mycjkitalicfont{\myFntAdobeKaiti}
    \renewcommand\mycjkmainfont{\myFntAdobeFangsong}
    \renewcommand\mycjksansfont{\myFntAdobeHeiti}
    \renewcommand\mycjkmonofont{\myFntAdobeFangsong}



    % 配置: 使用Noto Sans S Chinese作为缺省字体,辅助用康熙字典體补全字形,最后用SourceHanSans扫尾。
    \renewcommand\mytestfont{\mycjknotol}%{TypeLand.com 康熙字典體}%{Adobe Heiti Std}
    \renewcommand\mycjkfallbackfontA{\mycjkkhangxi}%{TypeLand.com 康熙字典體}%{Adobe Heiti Std}
    \renewcommand\mycjkfallbackfontB{\mycjkmingext}%{TypeLand.com 康熙字典體}%{Adobe Heiti Std}
    \renewcommand\mycjkfallbackfontE{\mycjknotol}%{Noto Sans S Chinese}%{Adobe Heiti Std}

    \renewcommand\mycjkboldfont{\mytestfont}%{TypeLand.com 康熙字典體}%{Adobe Heiti Std}
    \renewcommand\mycjkitalicfont{\mytestfont}%{FZKaiT-Extended}%{全字庫正楷體}
    \renewcommand\mycjkmainfont{\mytestfont}%{Adobe Ming Std}%{花園明朝A}%{TypeLand.com 康熙字典體}%{新細明體}
    \renewcommand\mycjksansfont{\mytestfont}%{Adobe Ming Std}%{花園明朝A}%{TypeLand.com 康熙字典體}%{新細明體}
    \renewcommand\mycjkmonofont{\mytestfont}%{WenQuanYi Micro Hei Mono}


\ifnum\strcmp{\myclinemode}{\detokenize{vertical}}=0
    % 配置: 使用方正清刻体作为缺省字体,辅助用康熙字典體补全字形,最后用SourceHanSans扫尾。
    \renewcommand\mytestfont{\mycjkfzqk}%{TypeLand.com 康熙字典體}%{Adobe Heiti Std}
    \renewcommand\mycjkfallbackfontA{\mycjkkhangxi}%{TypeLand.com 康熙字典體}%{Adobe Heiti Std}
    %\renewcommand\mytestfont{\mycjkkhangxi}%{TypeLand.com 康熙字典體}%{Adobe Heiti Std}
    %\renewcommand\mycjkfallbackfontA{\mycjkfzqk}%{TypeLand.com 康熙字典體}%{Adobe Heiti Std}
    \renewcommand\mycjkfallbackfontB{\mycjkmingext}%{TypeLand.com 康熙字典體}%{Adobe Heiti Std}
    \renewcommand\mycjkfallbackfontE{\mycjknotol}%{Noto Sans S Chinese}%{Adobe Heiti Std}

    \renewcommand\mycjkboldfont{\mytestfont}%{TypeLand.com 康熙字典體}%{Adobe Heiti Std}
    \renewcommand\mycjkitalicfont{\mytestfont}%{FZKaiT-Extended}%{全字庫正楷體}
    \renewcommand\mycjkmainfont{\mytestfont}%{Adobe Ming Std}%{花園明朝A}%{TypeLand.com 康熙字典體}%{新細明體}
    \renewcommand\mycjksansfont{\mytestfont}%{Adobe Ming Std}%{花園明朝A}%{TypeLand.com 康熙字典體}%{新細明體}
    \renewcommand\mycjkmonofont{\mytestfont}%{WenQuanYi Micro Hei Mono}

\fi



% 用於章節中夾注評論批評
\newCJKfontfamily{\fntfmyJiaozhu}{Adobe Ming Std}%{{[Path=\myfontdir,]{\mycjkkhangxi}}}  % 校注字体

% 用於章節中整段整段的評論批評
\newCJKfontfamily{\fntfmyPinlun}{Adobe Kaiti Std}%{{[Path=\myfontdir,]{\mycjkkhangxi}}}  % 校注字体



\newCJKfontfamily{\fzwkai}   {\FntNameFzbwksgb}  % 方正北魏楷书
\newCJKfontfamily{\fzzhdxian}{\FntNameFzzdxgb} % 方正中等线
\newCJKfontfamily{\fzltheib} {\FntNameFzlth}         % 方正兰亭黑扁
\newCJKfontfamily{\fzydzhhei}{\FntNameFzydh}          % 方正韵动中黑
\newCJKfontfamily{\fzzhysong}{\FntNameFzys}             % 方正中雅宋
\newCJKfontfamily{\fzqiti}   {\FntNameFzqt}                  % 方正启体
\newCJKfontfamily{\fzliukai}[GB18030=\FntNameFzqt]{\FntNameFzsxslk}% 方正苏新诗柳楷简体


\setCJKmainfont[Path=\myfontdir,
    %FallBack=\mycjkfallbackfontE,
    zhbiaodian={[Path=\myfontdir,]{\mycjkkhangxi}},%mycjkfzqk mycjknoto mycjkkhangxi mycjkming mycjknotol myFntAdobeKaiti myFntAdobeHeiti myFntAdobeFangsong myFntAdobeMing myFntAdobeSong
    BoldFont=\mycjkboldfont,
    ItalicFont=\mycjkitalicfont,
    ItalicFeatures={RawFeature={slant=0.17}}, % 定义 \itshape 为斜体
    BoldItalicFeatures={RawFeature={slant=0.17}}, % 定义 \bfseries\itshape 为斜体
    AutoFakeBold,AutoFakeSlant]{\mycjkmainfont}

% FallBack 的值为空,将设置 备用字体
\setCJKmainfont[FallBack,Path=\myfontdir,]{\mycjkfallbackfontE}
\addCJKfontfeatures{Scale=1.5}

\setCJKsansfont[Path=\myfontdir,
    FallBack=\mycjkfallbackfontE,
    BoldFont=\mycjkboldfont,
    ItalicFont=\mycjkitalicfont,
    ItalicFeatures={RawFeature={slant=0.17}}, % 定义 \itshape 为斜体
    BoldItalicFeatures={RawFeature={slant=0.17}}, % 定义 \bfseries\itshape 为斜体
    AutoFakeBold,AutoFakeSlant]{\mycjksansfont}

%\setCJKfallbackfamilyfont{\CJKrmdefault}
%  [BoldFont={FZHei-B01_GB18030},ItalicFont={FZKai-Z03_GB18030}]{FZShuSong-Z01_GB18030}
%\setCJKfallbackfamilyfont{\CJKsfdefault}
%  [BoldFont={FZHei-B01_GB18030},ItalicFont={FZKai-Z03_GB18030}]{FZLanTingHei-R_GB18030_YS}
%\setCJKfallbackfamilyfont{\CJKttdefault}
%  [BoldFont={FZHei-B01_GB18030},ItalicFont={FZKai-Z03_GB18030}]{FZFangSong-Z02_GB18030}

\setCJKfallbackfamilyfont{\CJKrmdefault}[AutoFakeSlant]{
%\setCJKfallbackfamilyfont{rm}[AutoFakeSlant]{
    [Path=\myfontdir]{\mycjkfallbackfontA},
    [Path=\myfontdir]{\mycjkfallbackfontB},
    [Path=\myfontdir]{\mycjkfallbackfontE}
}
\setCJKfallbackfamilyfont{\CJKsfdefault}[AutoFakeSlant]{
    [Path=\myfontdir]{\mycjkfallbackfontA},
    [Path=\myfontdir]{\mycjkfallbackfontB},
    [Path=\myfontdir]{\mycjkfallbackfontE}
}
\setCJKfallbackfamilyfont{\CJKttdefault}[AutoFakeSlant]{
    [Path=\myfontdir]{\mycjkfallbackfontA},
    [Path=\myfontdir]{\mycjkfallbackfontB},
    [Path=\myfontdir]{\mycjkfallbackfontE}
}




%%%%%%%%%%%%%%%%%%%%%%%%%%%%%%%%%%%%%%%%%%%%%%%%%%%%%%%%%%%%

%\geometry{
    %papersize={145mm,210mm},
    %textwidth=110mm,
    %lines=30,
    %inner=15mm,
    %top=20mm,
    %bindingoffset=5mm,
    %headheight=10mm,
    %headsep=6mm,
    %foot=5mm}

\geometry{
    papersize={\mypagewidth,\mypageheight},
    }

\newcommand\tmprotated{0}
\ifnum\strcmp{\mypageorien}{\detokenize{landscape}}=0
    \renewcommand\tmprotated{1}
\fi

\ifnum\strcmp{\myclinemode}{\detokenize{vertical}}=0
    \ifnum\strcmp{\mypageorien}{\detokenize{landscape}}=0
    \else % fix page, fake portrait
        \renewcommand\tmprotated{1}
        \geometry{
            papersize={\mypageheight,\mypagewidth},%{\mypagewidth,\mypageheight},
            }
    \fi
\fi


\ifnum\tmprotated > 0
    % 页面旋转90度
    \usepackage{everyshi}
    \makeatletter
    \EveryShipout{
        \global\setbox\@cclv\vbox{%
            \special{pdf: put @thispage << /Rotate 90 >>}%
            \box\@cclv
        }%
    }%
    \makeatother
\fi


\ifnum\myusecjk=1
    \usepackage{cndocstory}
\fi




\newcommand{\myfootnote}[1]{
    \footnote{#1}
}
\ifnum\strcmp{\myfnotemode}{\detokenize{gezhu}}=0
    %# -*- coding: utf-8 -*-
% !TeX encoding = UTF-8 Unicode
% !TeX spellcheck = en_US
% !TeX TS-program = xelatex
%~ \XeTeXinputencoding "UTF-8"
% vim:ts=4:sw=4
%
% 以上設定默認使用 XeLaTex 編譯,並指定 Unicode 編碼,供 TeXShop 自動識別
%
%%%%%%%%%%%%%%%%%%%%%%%%%%%%%%%%%%%%%%%%%%%%%%%%%%%%%%%%%%%%

\renewcommand\KG{{ }}

\usepackage{fancyhdr}
\pagestyle{fancyplain}
\fancyhf{}
\renewcommand{\headrulewidth}{0pt}
%\def\thepage{\Chinese{page}}
\renewcommand{\thepage}{\Chinese{page}}
\fancyfoot[R]{\thepage}






\usepackage{gezhu}
%\setgezhuraise{-6pt}
%\everygezhu={\zihao{5}\linespread{.9}\selectfont\itshape}
%\expandafter\everywithgezhu\expandafter{\the\everywithgezhu \baselineskip=16pt \parskip=1em}
\everygezhu{\fontsize{6}{7}\selectfont}
\setgezhuraise{-1.5pt}
\setgezhulines{2}

%\parindent=2em
\setgezhushipoutlevel{2}
%\gezhuraggedfalse
%\gezhuraggedtrue
\gezhunormalizetrue


%\def\skipzhfullstop{\hskip 0pt plus 0.3em}






\comments{
% 使用 footnote 替代 \gezhu
%\makeatletter
%\let\oldfootnote\footnote
%\renewcommand\footnote[1]{\gezhu{#1}} % {\gezhu{\fntfmyJiaozhu\tiny #1}}
%\makeatother

%\makeatletter
%\let\oldmarginnote\marginnote
%\renewcommand\marginnote[1]{\gezhu{#1}} % {\gezhu{\fntfmyJiaozhu\tiny #1}}
%\makeatother

}%comments

\comments{

\def\myonkyoh#1{%
  \unless\ifdefined\isfirsttime
    \def\isfirsttime{yes}%
    \withgezhu
  \fi
}

% 设置章节
% http://wiki.lyx.org/Tips/ModifyChapterEtc
\let\oldchap=\chapter
\renewcommand*{\chapter}{%
  \secdef{\Chap}{\ChapS}%
}
%\newcommand\ChapS[1]{\singlespacing\oldchap*{#1}\doublespacing}
%\newcommand\Chap[2][]{\singlespacing\oldchap[#1]{#2}\doublespacing}
\newcommand\ChapS[1]{\oldchap*{#1}\myonkyoh{#1}}
\newcommand\Chap[2][]{\oldchap{#2}\myonkyoh{#2}}
%\newcommand\ChapS[1]{\ifdefined\gezhu\endwithgezhu\fi\clearpage\oldchap*{#1}\withgezhu}
%\newcommand\Chap[2][]{\ifdefined\gezhu\endwithgezhu\fi\clearpage\oldchap{#2}\withgezhu}

%\newcommand\ChapS[1]{\ifdefined\gezhu\endwithgezhu\fi\oldchap*{#1}\withgezhu}
%\newcommand\Chap[2][]{\ifdefined\gezhu\endwithgezhu\fi\oldchap[#1]{#2}\withgezhu}

%\newcommand\ChapS[1]{\ifdefined\gezhu\endwithgezhu\fi\oldchap*{#1}\withgezhu}
%\newcommand\Chap[2][]{\ifdefined\gezhu\endwithgezhu\fi\oldchap[#1]{#2}\withgezhu}

\def\endmyonkyoh{\ifdefined\gezhu\endwithgezhu\fi}

}



\comments{

\newcommand{\BeginChapter}{}
\newcommand{\AtBeginChapter}[1]{%
    \renewcommand{\BeginChapter}{#1}%
}

\newcommand{\EndChapter}{}
\newcommand{\AtEndChapter}[1]{%
    \renewcommand{\EndChapter}{#1}%
}

\makeatletter

\newenvironment{tablehere}
{\def\@captype{table}}
{}

\let\original@chapter\chapter
\def\@first@chapter{1}
\renewcommand{\chapter}{%
    \ifnum\@first@chapter=1 \gdef\@first@chapter{0}\else\EndChapter\fi
    \original@chapter\BeginChapter}
\makeatother

%\AtBeginChapter{\centerline{$+++++$}}
%\AtEndChapter{\centerline{$******$}}

%\AtBeginChapter{\withgezhu}
%\AtEndChapter{\ifdefined\gezhu\endwithgezhu\fi}
}







    \renewcommand{\myfootnote}[1]{
        \gezhu{#1}
    }

    \newenvironment{showcontents}[1]{%
        \begin{withgezhu} \Huge
    }{%
        \end{withgezhu}
    }

\else
    \newenvironment{showcontents}[1]{%
    }{%
    }

\fi



%%%%%%%%%%%%%%%%%%%%%%%%%%%%%%%%%%%%%%%%%%%%%%%%%%%%%%%%%%%%
\newcommand\jpmPrefixFigures{figures-jpmxxjpg/jpmcz}
%\newcommand\jpmPrefixFigures{figures-cz/cz}
\newcommand\jpmSuffixFigures{jpg}
%\newcommand\jpmPrefixFigures{figures-czeps/jpmcz}
%\newcommand\jpmSuffixFigures{eps}

\newcommand\jpmPrefixChapter{chap-xiuxiang/jpm}
% #1: the prefix of the chapter files
%\newcommand\jpmSetupChapter[1]{
    %\renewcommand\jpmPrefixChapter{#1}
%}
%\jpmSetupChapter{chap-xiuxiang/jpm}


%# -*- coding: utf-8 -*-
%!TEX encoding = UTF-8 Unicode
%!TEX TS-program = xelatex
% vim:ts=4:sw=4
%
% 以上设定默认使用 XeLaTex 编译,并指定 Unicode 编码,供 TeXShop 自动识别

\usepackage{setspace}

% two columns footnotes
\usepackage{dblfnote}
\DFNalwaysdouble % for this example

\definecolor{darkred}{rgb}{0.5,0,0}
\definecolor{darkgreen}{rgb}{0,0.5,0}
\definecolor{darkblue}{rgb}{0,0,0.5}
\definecolor{darkpurple}{rgb}{0.5,0,0.5}
%\hypersetup{
    %colorlinks,
    %linkcolor=darkblue,
    %filecolor=darkgreen,
    %urlcolor=darkred,
    %citecolor=darkblue
%}

%%%%%%%%%%%%%%%%%%%%%%%%%%%%%%%%%%%%%%%%%%%%%%%%%%%%%%%%%%%%
% JPM related:

%\DeclareRobustCommand*\KG{}%{\kern\ccwd}
%\Huge
%\huge
%\LARGE
%\Large
%\large
%\normalsize (default)
%\small
%\footnotesize
%\scriptsize
%\tiny


% 题字签名
\newcommand\tiJPM[1]{{\bigskip\mbox{}\fzqiti\large\hfill #1 \quad}}


\newcommand{\mnote}[1]{\marginpar{\raggedright
\textsf{\footnotesize{%
\begin{spacing}{1.025}%
#1%
\end{spacing}%
}}}}

\newcommand{\mnoteb}[1]{{\marginpar{\raggedright\sffamily\footnotesize
\setstretch{1.025}%
#1}}}


\newcommand\jiaoZhanginfo{張竹坡皋鶴堂批評第一奇書批評本。齐鲁书社 1987 年版《张竹坡批评第一奇书金瓶梅》} \newcommand\zhuZhanginfo{\jiaoZhanginfo} \newcommand\piZhanginfo{\jiaoZhanginfo}
\newcommand\piZhangF[1]{ \ifnum\myEnablePi > 0  {\fntfmyJiaozhu \color{darkred} #1 (張竹坡皋鶴堂批評第一奇書批評本)} \fi} % 批評 (整段,張竹坡皋鶴堂批評第一奇書批評本批語)
\newcommand\piZhang[1]{ \ifnum\myEnablePi > 0  \footnote{{\fntfmyJiaozhu\scriptsize\linespread{0.9} \color{darkred} 張批: #1}} \fi } % 批評 (張竹坡皋鶴堂批評第一奇書批評本批語)
%\newcommand\piZhang[1]{ \ifnum\myEnablePi > 0  \footnote{張批: #1} \fi } % 批評 (張竹坡皋鶴堂批評第一奇書批評本批語)

%\renewcommand\piZhang[1]{\gezhu{#1}}



\newcommand\jiaoLiyuinfo{李漁新刻繡像批評本} \newcommand\zhuLiyuinfo{\jiaoLiyuinfo} \newcommand\piLiyuinfo{\jiaoLiyuinfo}
\newcommand\piLiyuF[1]{ \ifnum\myEnablePi > 0 {\fntfmyJiaozhu \color{darkblue} #1 (李漁新刻繡像批評本)} \fi} % 批評 (李漁新刻繡像批評本批語)
\newcommand\piLiyu[1]{ \ifnum\myEnablePi > 0  \footnote{{\fntfmyJiaozhu\scriptsize\linespread{0.9} \color{darkblue} 繡像批: #1}} \fi } % 批評 (李漁新刻繡像批評本批語)
%\newcommand\piLiyu[1]{ \ifnum\myEnablePi > 0  \footnote{繡像批: #1} \fi } % 批評 (李漁新刻繡像批評本批語)


% 文龍(字禹門)光緒年間墨批六萬餘字。這是繼張竹坡之後,對《金瓶梅》一書的重要評論,具有珍貴的文獻價值。文龍批語分回末評、眉批、夾批等部分。
\newcommand\jiaoWenglonginfo{文龍在茲堂本手書批評本} \newcommand\zhuWenglonginfo{\jiaoWenglonginfo} \newcommand\piWenglonginfo{\jiaoWenglonginfo}
\newcommand\piWenglongF[1]{ \ifnum\myEnablePi > 0  {\fntfmyJiaozhu \color{darkpurple} #1 (文龍在茲堂本手書批評本)} \fi} % 批評 (文龍在茲堂本手書批評本批語)
\newcommand\piWenglong[1]{ \ifnum\myEnablePi > 0  \footnote{{\fntfmyJiaozhu\scriptsize\linespread{0.9} \color{darkpurple} 文龍批: #1}} \fi } % 批評 (文龍在茲堂本手書批評本批語)
%\newcommand\piWenglong[1]{ \ifnum\myEnablePi > 0  \footnote{文龍批: #1} \fi } % 批評 (文龍在茲堂本手書批評本批語)


\newcommand\jiaoYLinfo{金瓶梅詞話校注,白維國、卜鍵 校注,嶽麓書社} \newcommand\zhuYLinfo{\jiaoYLinfo} \newcommand\piYLinfo{\jiaoYLinfo}
\newcommand\jiaoYL[1]{ \ifnum\myEnableJiao > 0  \footnote{{\fntfmyJiaozhu\scriptsize\linespread{0.9} \color{darkpurple} 麓校: #1}} \fi } % 校記 (金瓶梅詞話校注,白維國、卜鍵 校注,嶽麓書社)
%\newcommand\jiaoYL[1]{ \ifnum\myEnableJiao > 0  \footnote{麓校: #1} \fi } % 校記 (金瓶梅詞話校注,白維國、卜鍵 校注,嶽麓書社)
\newcommand\zhuYL[1]{ \ifnum\myEnableZhu > 0  \footnote{{\fntfmyJiaozhu\scriptsize\linespread{0.9} \color{darkpurple} 麓注: #1}} \fi } % 注疏 (金瓶梅詞話校注,白維國、卜鍵 校注,嶽麓書社)
%\newcommand\zhuMINE[1]{ \ifnum\myEnableZhu > 0  \footnote{麓注: #1} \fi } % 注疏 (金瓶梅詞話校注,白維國、卜鍵 校注,嶽麓書社)


\newcommand\jiaoMJinfo{梅節} \newcommand\zhuMJinfo{\jiaoMJinfo} \newcommand\piMJinfo{\jiaoMJinfo}
\newcommand\jiaoMJ[1]{ \ifnum\myEnableJiao > 0  \footnote{{\fntfmyJiaozhu\scriptsize\linespread{0.9} \color{darkblue} 梅校: #1}} \fi } % 校記 (梅節)
%\newcommand\jiaoMJ[1]{ \ifnum\myEnableJiao > 0  \footnote{梅校: #1} \fi } % 校記 (梅節)


\newcommand\jiaoMINEinfo{鄙人的拙作} \newcommand\zhuMINEinfo{\jiaoMINEinfo} \newcommand\piMINEinfo{\jiaoMINEinfo}
\newcommand\jiaoMINE[1]{ \ifnum\myEnableJiao > 0  \footnote{{\fntfmyJiaozhu\scriptsize\linespread{0.9} \color{darkgreen} 校: #1}} \fi } % 校記 (我的)
%\newcommand\jiaoMINE[1]{ \ifnum\myEnableJiao > 0  \footnote{校: #1} \fi } % 校記 (我的)
\newcommand\piMINE[1]{ \ifnum\myEnablePi > 0  \footnote{{\fntfmyJiaozhu\scriptsize\linespread{0.9} \color{darkgreen} 批: #1}} \fi } % 批評 (我的)
%\newcommand\piMINE[1]{ \ifnum\myEnablePi > 0  \footnote{批: #1} \fi } % 批評 (我的)
\newcommand\zhuMINE[1]{ \ifnum\myEnableZhu > 0  \footnote{{\fntfmyJiaozhu\scriptsize\linespread{0.9} \color{darkgreen} 注: #1}} \fi } % 注疏 (我的)
%\newcommand\zhuMINE[1]{ \ifnum\myEnableZhu > 0  \footnote{注: #1} \fi } % 注疏 (我的)




%%%%%%%%%%%%%%%%%%%%%%%%%%%%%%%%%%%%%%%%%%%%%%%%%%%%%%%%%%%%

\newcommand\jpmHasZhu{0}

\newcommand\jpmChkHasZhu{
\renewcommand\jpmHasZhu{0}
\ifnum\myEnablePi > 0   \renewcommand\jpmHasZhu{1} \fi
\ifnum\myEnableJiao > 0 \renewcommand\jpmHasZhu{1} \fi
\ifnum\myEnableZhu > 0  \renewcommand\jpmHasZhu{1} \fi
}

\newcommand\jpmShowZhuInfo{
\jpmChkHasZhu
\ifnum\jpmHasZhu > 0
{
本精校《金瓶梅》彙集有:

\begin{itemize}
\ifnum\myEnablePi > 0
  \item 各家批評;
    \piZhang{\piZhanginfo}
    \piLiyu{\piLiyuinfo}
    \piWenglong{\piWenglonginfo}
\fi
\ifnum\myEnableJiao > 0
  \item 各種校記;
    \jiaoYL{\jiaoYLinfo}
    \jiaoMJ{\jiaoMJinfo}
    \jiaoMINE{\jiaoMINEinfo}
\fi
\ifnum\myEnableJiao > 0
  \item 各式注疏;
    \zhuYL{\zhuYLinfo}
\fi
\end{itemize}
}
\fi
}


%%%%%%%%%%%%%%%%%%%%%%%%%%%%%%%%%%%%%%%%%%%%%%%%%%%%%%%%%%%%

\newcommand\tmpfigscale{1.1}%{1.35}

\renewcommand{\myincchapge}[2]{
    \cleardoublepage
    \pagestyle{special}
    \begin{figure}[hbtp] \vspace*{-2.1cm} \centering
        \makebox[\linewidth]{\includegraphics[width=\tmpfigscale\linewidth]{\jpmPrefixFigures.#1#2.1.\jpmSuffixFigures}}
    \end{figure}
    \begin{figure}[hbtp] \vspace*{-2.1cm} \centering
        \makebox[\linewidth]{\includegraphics[width=\tmpfigscale\linewidth]{\jpmPrefixFigures.#1#2.2.\jpmSuffixFigures}}
    \end{figure}
    \pagestyle{main}
    \input{\jpmPrefixChapter.#1#2.tex}
}


\ifnum\strcmp{\myclinemode}{\detokenize{vertical}}=0
    \usepackage{rotating}
    \ifnum\strcmp{\mypageorien}{\detokenize{landscape}}=0
        \renewcommand{\myincchapge}[2]{
            \cleardoublepage
            \pagestyle{special}
            \begin{sidewaysfigure}%\begin{figure}[ht]\centering
                %\includegraphics[height=0.85\textwidth,width=0.52\textheight,angle=180]{\jpmPrefixFigures.#1#2.1.\jpmSuffixFigures}
                %\includegraphics[height=0.85\textwidth,width=0.52\textheight,angle=180]{\jpmPrefixFigures.#1#2.2.\jpmSuffixFigures}
                \includegraphics[height=0.85\textwidth,width=0.52\textheight]{\jpmPrefixFigures.#1#2.2.\jpmSuffixFigures}
                \includegraphics[height=0.85\textwidth,width=0.52\textheight]{\jpmPrefixFigures.#1#2.1.\jpmSuffixFigures}
                \caption{This is the caption.}\label{fig:fig1}
            \end{sidewaysfigure}%\end{figure}
            \pagestyle{main}
            \input{\jpmPrefixChapter.#1#2.tex}
        }

    \else % fix page, fake portrait

        \renewcommand{\myincchapge}[2]{
            \cleardoublepage
            \pagestyle{special}
            %\renewcommand\tmpfigscale{1.4}
            \begin{sidewaysfigure}%\begin{figure}[ht]
                \vspace*{-0.5cm} \hspace*{-1.1cm}  \centering
                \makebox[\linewidth]{\includegraphics[width=\tmpfigscale\linewidth]{\jpmPrefixFigures.#1#2.1.\jpmSuffixFigures}}
            \end{sidewaysfigure}%\end{figure}
            \begin{sidewaysfigure}%\begin{figure}[ht]
                \vspace*{-0.5cm} \hspace*{-1.1cm} \centering
                \makebox[\linewidth]{\includegraphics[width=\tmpfigscale\linewidth]{\jpmPrefixFigures.#1#2.2.\jpmSuffixFigures}}
            \end{sidewaysfigure}%\end{figure}
            \pagestyle{main}
            \input{\jpmPrefixChapter.#1#2.tex}
        }
    \fi
\fi




%%%%%%%%%%%%%%%%%%%%%%%%%%%%%%%%%%%%%%%%%%%%%%%%%%%%%%%%%%%%

%opening
\title{\mydoctitle}
\author{\mydocauthor}
\date{}



\begin{document}

\maketitle

\frontmatter

%# -*- coding: utf-8 -*-
%!TEX encoding = UTF-8 Unicode
%!TEX TS-program = xelatex
% vim:ts=4:sw=4
%
% 以上設定默認使用 XeLaTex 編譯,並指定 Unicode 編碼,供 TeXShop 自動識別

\chapter*{制作說明}
\addcontentsline{toc}{chapter}{制作說明}

本電子書嘗試提供一個基本無錯誤的文學著作底本。
其特點在於,使用同一個精校文件數據源的情況下,自動生成所需要的PDF閱讀輸出版式,
而不必像使用 M\$ Word 那樣,需要針對輸出不同版面手工分別調整。
這樣可以達到事半而功倍的效果。

基於以上思想,本項目能夠自動化生成如下任意組合的PDF輸出文件:
\begin{enumerate}
  \item 任意紙張大小。可以根據打印或閱讀需求,如電子書閱讀器,以合適的大小輸出頁面;
  \item 頁面可以在直式、橫式中選擇;
  \item 可以生成含有或者不含批評、校記、注疏的版本;
  \item 批評、校記、注疏的排版位置可以是腳注或者割注的形式;
  \item 內容排版可以是橫排或者豎排;
\end{enumerate}

%而更關鍵是,本項目將采取開源方式運作,目的是讓大家都能參與進來,使其效果最大化。


本項目所服務的目標群體有

\begin{itemize}
  \item 文學研究者,提供精准的名著底本,同時輔助以各家批評、校記等以助研究。真正做到一冊在手,別無所求。
  \item 語言研究者,名著中提供供了豐富的語言素材。
  \item 一般讀者,提供經過專家校正過的版本,輔之以精心設計的版面,欣賞原汁原味的文學名著。
\end{itemize}



本項目計劃:

\begin{enumerate}
  \item 實現可輸出各種版面的工具框架;(初版已经完成) %合適可用的工具框架,可以融合各個參與者的貢獻
  \item 提供精校的名著底本;
  \item 開發工具,使之支持用原刻本上的字形排版精校底本;一方面滿足部分人懷舊之需,另外更重要是,可以用於對比來輔助校對文本。
\end{enumerate}


本項目需要的開發人員:

\begin{itemize}
  \item 文學研究者,校對名著底本;
  \item 語言研究者,校對、對其中的疑難提出校對意見;
  \item 美工,封面設計、字體搭配等
  \item 計算機程序人員,程序腳本開發;%\LaTeX 開發等。
  \item 其他人員,可以提出、改進任何你認為需要改進的地方。
\end{itemize}


項目籌備聯系郵件:\url{mailto:yhfudev@gmail.com}


\section*{凡例}


\begin{showcontents}{}

\jpmShowZhuInfo

\end{showcontents}


%如需本書電子版最新版,請關注本項目開發主頁 \url{https://github.com/}

%# -*- coding: utf-8 -*-
%!TEX encoding = UTF-8 Unicode
%!TEX TS-program = xelatex
% vim:ts=4:sw=4
%
% 以上设定默认使用 XeLaTex 编译,并指定 Unicode 编码,供 TeXShop 自动识别



\chapter*{金瓶梅序}
\addcontentsline{toc}{chapter}{金瓶梅序 -- 東吴弄珠客}

金瓶梅,穢書也。袁石公亟稱之,
亦自寄其牢騷耳,非有取於金瓶梅也。
然作者亦自有意,蓋為世戒,非為世勸也。
如諸婦多矣,而獨以潘金蓮,李瓶兒,春梅命名者,亦楚「檮杌」之意也。
蓋金蓮以姦死,瓶兒以孽死,春梅以淫死,較諸婦為更慘耳。
借(藉)西門慶以描畫世之大淨,應伯爵以描畫世之小丑(醜),諸淫婦以描畫世之丑(醜)婆淨婆,
令人讀之汗下。蓋為世戒,非為世勸也。

余嘗曰:讀金瓶梅而生憐憫心者,菩薩也;
生畏懼心者,君子也;
生歡喜心者,小人也;
生效(傚)法心者,乃禽獸耳。
余友人褚孝秀偕一少年同赴歌舞之筵,衍至「霸王夜宴」,
少年垂涎曰:「男兒何可不如此!」
褚孝秀曰:「也只為這烏江設此一着(著)耳。」
同座聞之,歎為有道之言。
若有人識得此意,方許他讀金瓶梅也。
不然,石公幾為導淫宣慾之尤矣!
奉勸世人,勿為西門慶之後車,可也。

%{\bigskip\mbox{}\fzqiti\large\hfill 東吳弄珠客題 \quad}
\tiJPM{東吳弄珠客題}
\footnote{金瓶梅词话的署名是:萬曆丁巳季冬東吴弄珠客漫書扵金閶道中}


%\piLiyuF{ % 李渔新刻绣像批评本
\chapter*{跋}
\addcontentsline{toc}{chapter}{金瓶梅跋 -- (明)谢肇淛}

%《金瓶梅》一书,不著作者名代。相传永陵中有金吾戚里,凭怙奢汰,淫纵无度,而其门客病之,采摭日逐行事,汇以成编,而托之西门庆也。书凡数百万言,为卷二十,始末不过数年事耳。
%其中朝野之政务,官私之晋接,闺闼之媟
%\zhuMINE{xie 亵狎的意思}
%语,市里之猥谈,与夫势交利合之态,心输背笑之局,桑中濮
%上之期,尊罍枕席之语,驵验{马会}之机械意智,粉黛之自媚争妍,狎客之从臾逢迎,奴怡之嵇唇淬{言卒}语,穷极境象,駥意快心。譬之范工抟泥,妍媸老少,人鬼万殊,不徒肖其貌,且并其神传之。信稗官之上乘,炉锤之妙手也。其不及《水浒传》者,以其猥琐淫蝶,无关名理。而或以为过之者,彼犹机轴相放,而此之面目各别,聚有自来,散有自去,读者竟想不到,唯恐易尽。此岂可与褒儒俗士见哉。此书向无镂版,抄写流传,参差散失。唯弇州家藏者最为完好。余于袁中郎得其十三,于丘诸城得其十五,稍为厘正,而阙所未备,以俟他日。有嗤余诲淫者,余不敢知。然溱洧之音,圣人不删,则亦中郎帐中必不可无之物也。仿此者有《玉娇丽》,然而乖彝败度,君子无取焉。

《金瓶梅》一書,不著作者名代。相傳永陵
中有金吾戚里,憑怙奢汰,淫縱無度,
而其門客病之,採摭日逐行事,匯以成編,而托之西門慶也。書凡數百萬言,為
卷二十,始末不過數年事耳。其中朝野之政務,官私之晉接,閨闥之媟\zhuMINE{xie 亵狎的意思}語,市里
之猥談,與夫勢交利合之態,心輸背笑之局,桑中濮上之期,尊罍枕席之語,駔
驓之機械意智,粉黛之自媚爭妍,狎客之從諛逢迎,奴佁之稽唇淬語,窮極境象,
駴意快心。譬之範工搏泥,妍媸老少,人鬼萬殊,不徒肖其貌,且并其神傳之。
信稗官之上乘,爐錘之妙手也。其不及《水滸傳》者,以其猥瑣淫媟,無關名理。
而或以為過之者,彼猶機軸相放,而此之面目各別,聚有自來,散有自去,讀者
意想不到,唯恐易盡。此豈可與褒儒俗士見哉。此書向無鏤版,鈔寫流傳,參差
散失。唯弇州家藏者最為完好。余于袁中郎得其十三,于丘諸城得其十五,稍為
釐正,而闕所未備,以俟他日。有嗤余誨淫者,余不敢知。然溱洧之音,聖人不
刪,則亦中郎帳中必不可無之物也。仿此者有《玉嬌麗》,然而乖彝敗度,君子
無取焉。



\tiJPM{(明)谢肇淛}
%\footnote{《金瓶梅资料汇编》190页,南开大学出版社1985年版。}
\footnote{〔明〕謝肇淛:〈金瓶梅跋〉(謝肇淛:《小草齋文集》,卷二十四),見黃霖編:《金瓶梅資料彙編》(北京:中華書局,1987),頁3-4。}
%} % 李渔新刻绣像批评本





\piZhangF{ % 张竹坡皋鹤堂批评第一奇书批评本
\chapter*{第一奇书序}
\addcontentsline{toc}{chapter}{第一奇书序 -- 张竹坡}

《金瓶》一书,传为凤洲门人之作也,或云即风洲手。然丽丽洋洋一百回内,其细针密线,每令观者望洋而叹。今经张子竹坡一批,不特照出作者金针之细,兼使其粉腻香浓,皆如狐穷秦镜,怪窘温犀、无不洞鉴原形,的是浑《艳异》旧手而出之者,信乎为凤洲作无疑也。然后知《艳异》亦淫,以其异而不显其艳;《金瓶》亦艳,以其不异则止觉其淫。故悬鉴燃犀,遂使雪月风花,瓶罄篦梳,陈茎落叶诸精灵等物,妆娇逞态,以欺世于数百年间,一旦潜形无地,蜂蝶留名,杏梅争色,竹坡其碧眼胡乎!向弄珠客教人生怜悯畏惧心,今后看官睹西门庆等各色幻物,弄影行间,能不怜悯,能不畏惧乎?其视金莲当作敝履观矣。不特作者解颐而谢觉,今天下失一《金瓶梅》,添一《艳异编》,岂不大奇!
时康熙岁次乙亥清明中浣,秦中觉天者谢颐题于皋鹤堂。

} % 张竹坡皋鹤堂批评第一奇书批评本

\piZhangF{ % 张竹坡皋鹤堂批评第一奇书批评本
\chapter*{第一奇书凡例}
\addcontentsline{toc}{chapter}{第一奇书凡例 -- 张竹坡}

一、此书非有意刊行,偶因一时文兴,借此一试目力,且成于十数天内,又非十年精思,故内中其大段结束精意,悉照作者。至于琐碎处,未暇请教当世,幸暂量之。

一、《水浒传》圣叹批,大抵皆腹中小批居多。予书刊数十回后,或以此为言。予笑曰:《水浒》是现成大段毕具的文字,如一百八人,各有一传,虽有穿插,实次第分明,故圣叹只批其字句也。若《金瓶》,乃隐大段精采于琐碎之中,只分别字名,细心者皆可为,而反失其大段精采也。然我后数十回内,亦随手补入小枇,是故欲知文字纲领者看上半部,欲随目成趣知文字细密者看下半部,亦何不可!

一、此书卷数浩繁,偶尔批成,适有工便,随刊呈世。其内或圈点不齐,或一二讹字,目力不到者,尚容细政,祈读时量之。

一、《金瓶》行世已久,予喜其文之整密,偶为当世同笔墨者闲中解颐。作《金瓶梅》者,或有所指,予则并无寓讽。设有此心,天地君亲其共恹之。

} % 张竹坡皋鹤堂批评第一奇书批评本

\piZhangF{ % 张竹坡皋鹤堂批评第一奇书批评本

\chapter*{杂录小引}
\addcontentsline{toc}{chapter}{杂录小引 -- 张竹坡}

凡看一书,必看其立架处,如《金瓶梅》内,房屋花园以及使用人等,皆其立架处也。何则?既要写他六房妻小,不得不派他六房居住。然全分开既难使诸人连合,全合拢又难使各人的事实入来,且何以见西门豪富。看他妙在将月、楼写在一处,娇儿在隐现之间。后文说挪厢房与大姐住,前又说大妗子见西门庆揭帘子进来,慌的往娇儿那边跑不迭,然则娇儿虽居厢房,却又紧连上房东间,或有门可通者也。雪娥在后院,近厨房。特特将金、瓶,梅三人,放在前边花园内,见得三人虽为侍妾,却似外室,名分不正,赘居其家,反不若李娇儿以娼家聚来,犹为名正言顺。则杀夫夺妻之事,断断非千金买妾之目。而金梅合,又分出瓶儿为一院,分者理势必然,必紧邻一墙者,为妒宠相争地步。而大姐住前厢,花园在仪门外,又为敬济偷情地步。见得西门庆一味自满托大,意谓惟我可以调弄人家妇女,谁敢狎我家春色,全不想这样妖淫之物,乃令其居于二门之外,墙头红杏,关且关不住,而况于不关也哉卜金莲固是冶容诲淫,而西门庆实是慢藏诲盗,然则固不必罪陈敬济也。故云写其房屋,是其间架处,犹欲耍狮子,先立一场;而唱戏先设一台。恐看官混混看过,故为之明白开出÷使看官如身入其中,然后好看书内有名人数进进出出,穿穿走走,做这些故事也。他如西门庆的家人妇女,皆书内听用者,亦录出之,令看者先已了了,俟后遇某人做某事,分外眼醒。而西门庆淫过妇人名数,开之足令看者伤心惨目,为之不忍也。若夫金莲,不异夏姬,故于其淫过者,亦录出之,令人知惧。

西门庆家人名数:来保(子僧保儿、小舅子刘仓)、来旺、玳安、来兴、平安、来安、书童、画童、琴
童、又琴童(天福儿改者)、棋童、来友、王显、春鸿、春燕、王经(系家丁)、来昭(暨铁棍儿)。后生(荣海)、司茶(郑纪)、烧火(刘包)、小郎(胡秀)、外甥小郎(崔本)、看坟(张安)。


西门庆家人媳妇:来旺媳妇(二,其一则宋蕙莲)、来昭媳妇(一丈青)、来保媳妇(惠祥)、来爵媳妇(惠元)、来兴媳妇(惠秀)。丫环:玉箫、小玉、兰香、小鸾、夏花、元霄儿、迎春、绣春、春梅、秋菊、中秋儿、翠儿。奶子:如意儿。

西门庆淫过妇女:李娇儿、卓丢儿、孟玉楼、潘金莲、李瓶儿、孙雪娥、春梅、迎春、绣春、兰香、宋蕙莲、来爵媳妇(惠元)、王六儿、责四嫂、如意儿、林太太、李桂姐、吴银儿、郑月儿。

意中人:何千户娘子(蓝氏)、王三官娘子(黄氏)、锦云。外宠:书童、王经、潘金莲、王六儿。

潘金莲淫过人目

张大户、西门庆、琴童、陈敬济、王潮儿。意中人:武二郎。外宠:西门庆。恶姻缘:武植。

藏春芙蓉镜:郓哥口、和尚耳,春梅秋波、猫儿眼中,铁棍舌畔、秋菊梦内。

附对:潘金莲品的箫,西门庆投的壶。

西门庆房屋

门面五间,到底七进(后要隔壁子虚房,共作花园)。

上房(月娘住)、西厢房(李娇儿住)、堂屋后三间(孙雪娥住)。

后院厨房、前院穿堂、大客屋、东厢房(大姐住)、西厢房。

仪门(仪门外,则花园也)。三间楼一院(潘金莲住)、又三间楼一院(李瓶儿住)。二人住楼在花园前,过花园方是后边。

花园门在仪门外,后又有角门,通看月娘后边也。金莲、瓶儿两院两角门,前又有一门,即花园门也。花园内,后有卷棚,翡翠轩,前有山子,山顶上卧云亭,半中间藏春坞雪洞也。花园外,即印子铺门面也。门面旁,开大门也。对门,乃要的乔亲家房子也。狮子街乃子虚迁去住者,瓶儿带来,后开绒线铺,又狮子街即打李外传处也。内仪门外,两道旁,乃群房,宋蕙莲等住者也。

} % 张竹坡皋鹤堂批评第一奇书批评本

\piZhangF{ % 张竹坡皋鹤堂批评第一奇书批评本

\chapter*{第一奇书目录}
\addcontentsline{toc}{chapter}{第一奇书目录 -- 张竹坡}


一回 热结 冷遇(悌字起)

二回 勾情 说技 三回  受贿 私挑

四回 幽欢 义愤 五回  捉奸 饮鸩

六回 瞒天 遇雨 七回  说媒 气骂

八回 占卦 烧灵 九回  偷娶 误打

十回 充配 玩赏(金瓶梅三字至此全起)

十一回 激打 梳笼 十二回 私仆 魇胜

十三回 密约 私窥 十四回 种孽 迎奸

十五回 赏灯 帮闲 十六回 择吉 追欢

十七回 弹奸 许嫁 十八回 脱祸 消魂

十九回 逻打 情感 二十回 趋奉 争风

二十一回 扫雪 替花(金瓶梅三人至此畅聚)

二十二回 偷期 正色 二十三回 输钞 潜踪

二十四回 戏娇 怒置 二十五回 秋千 醉谤

二十六回 递解 含羞 二十七回 私语 醉闹

二十八回 侥幸 糊涂

二十九回 冰鉴 兰汤(全部结果)

三十回覃恩 双喜 三十一回 构衅 为欢

三十二回 认女 惊儿 三十三回 罚唱 争风

三十四回 乞恩 说事 三十五回 报仇 媚客

三十六回 寄书 留饮 三十七回 说媒 包占

三十八回 棒槌 琵琶 三十九回 寄名 拜寿

四十回希宠 市爱 四十一回 联姻 同愤

四十二回 烟火 花灯 四十三回 争宠 卖富

四十四回 偷金 消夜 四十五回 劝当 解衣

四十六回 走雨 卜龟(两番结果)

四十七回 害主 枉法 四十八回 私情 捷径

四十九回 屈体 现身 五十回偷觑 嬉游

五十一回 品玉 输金 五十二回 山洞 花园

五十三回 惊欢 求子 五十四回 戏钏 诊瓶

五十五回 两庆 一诺 五十六回 助友 傲妻

五十七回 千金 一笑

五十八回 打狗 磨镜(孝子着书之意在此,教人以孝之意亦在此,此回以一个“孝”字照应一百回孝哥的“孝”字。)

五十九回 露阳 睹物 六十回死孽 生涯

六十一回 醉烧 病宴 六十二回 法遣 大哭

六十三回 传真 观戏 六十四回 三章 一帆

六十五回 同穴 守灵 六十六回 致赙 荐亡

六十七回 赏雪 入梦 六十八回 戏衔 密访

六十九回 初调 惊走 七十回朝房 庭参

七十一回 再梦 引奏 七十二回 抠打 义拜

七十三回 吹箫 试带 七十四回 偎玉 谈经

七十五回 含酸 撒泼(是作者一腔愤恨无可发泄处)

七十六回 娇撒 哭躲 七十七回 雪访 水战

七十八回 再战 独尝 七十九回 丧命 生儿

八十回售色 盗财 八十一回 拐财 欺主

八十二回 得双 冷面 八十三回 含恨 寄简

八十四回 碧霞 雪洞 八十五回 知情 惜泪

八十六回 唆打 解渴 八十七回 忘祸 祭兄

八十八回 感旧 埋尸 八十九回 寡妇 夫人

九十回盗拐 受辱 九十一回 爱嫁 怒打

九十二回 被陷 大闹 九十三回 义恤 娈淫

九十四回 酒楼 娼家 九十五回 窃玉 负心

九十六回 游旧 当面

九十七回 假续 真偕(一部真假总结,照转冷热二字)

九十八回 旧识 情遇 九十九回 醉骂 窃听

一百回路遇 幻化(孝字结)

} % 张竹坡皋鹤堂批评第一奇书批评本

\piZhangF{ % 张竹坡皋鹤堂批评第一奇书批评本

\chapter*{竹坡闲话}
\addcontentsline{toc}{chapter}{竹坡闲话 -- 张竹坡}

《金瓶梅》,何为而有此书也哉?曰:此仁人志士、孝子悌弟不得于时,上不能问诸天,下不能告诸人,悲愤鸣邑,而作秽言以泄其愤也。虽然,上既不可问诸天,下亦不能告诸人,虽作秽言以丑其仇,而吾所谓悲愤鸣邑者,未尝便慊然于心,解颐而自快也。夫终不能一畅吾志,是其言愈毒,而心愈悲,所谓“含酸抱阮”,以此固知玉楼一人,作者之自喻也。然其言既不能以泄吾愤,而终于“含酸抱阮”,作者何以又必有言哉?曰:作者固仁人也,志土也,孝子悌弟也。欲无言,而吾亲之仇也吾何如以处之?欲无言,而又吾兄之仇也吾何如以处之?且也为仇于吾天下万世也,吾又何如以公论之?是吾既不能上告天子以申其隐,又不能下告士师以求其平,且不能得急切应手之荆、聂以济乃事,则吾将止于无可如何而已哉!止于无可如何而已,亦大伤仁人志土、孝子悌弟之心矣。展转以思,惟此不律可以少泄吾愤,是用借西门氏以发之。虽然,我何以知作者必仁人志士、孝子悌弟哉?我见作者之以孝哥结也。“磨镜”一回,皆《蓼莪》遗
意,啾啾之声刺人心窝,此其所以为孝子也。至其以十兄弟对峙一亲哥哥,未复以二捣鬼为缓急相需之人,甚矣,《杀狗记》无此亲切也。

闲尝论之:天下最真者,莫若伦常;最假者,莫若财色。然而伦常之中,如君臣、朋友、夫妇,可合而成;若夫父子、兄弟,如水同源,如木同本,流分枝引,莫不天成。乃竟有假父、假子、假兄、假弟之辈。噫!此而可假,孰不可假?将富贵,而假者可真;贫贱,而真者亦假。富贵,热也,热则无不真;贫贱,冷也,冷则无不假。不谓“冷热”二字,颠倒真假一至于此!然而冷热亦无定矣。今日冷而明日热,则今日真者假,而明日假者真矣。今日热而明日冷,则今日之真者,悉为明日之假者矣。悲夫!本以嗜欲故,遂迷财色,因财色故,遂成冷热,因冷热故,遂乱真假。因彼之假者,欲肆其趋承,使我之真者皆遭其荼毒。所以此书独罪财色也。嗟嗟!假者一人死而百人来,真者一或伤而百难赎。世即有假聚为乐者,亦何必生死人之真骨肉以为乐也哉!

作者不幸,身遭其难,吐之不能,吞之不可,搔抓不得,悲号无益,借此以白泄。其志可悲,其心可悯矣。故其开卷,即以“冷热”为言,煞末又以“真假”为言。其中假父子矣,无何而有假母女;假兄弟矣,无何而有假弟妹;假夫妻矣,无何而有假外室;假亲戚矣,无何而有假孝子。满前役役营营,无非于假景中提傀儡。噫!识真假,则可任其冷热;守其真,则可乐吾孝悌。然而吾之亲父子已荼毒矣,则奈何?吾之亲手足已飘零矣,则奈何?上误吾之君,下辱吾之友,且殃及吾之同类,则奈何?是使吾欲孝,而已为不孝之人;欲弟,而已为不悌之人;欲忠欲信,而已放逐谗间于吾君、吾友之则。日夜咄咄,仰天太息,吾何辜而遭此也哉?曰:以彼之以假相聚故也。噫嘻!彼亦知彼之所以为假者,亦冷热中事乎?假子之子于假父也,以热故也。假弟、假女、假友,皆以热故也。彼热者,盖亦不知浮云之有聚散也。未几而冰山颓矣,未几而阀阅朽矣。当世驱己之假以残人之真者,不瞬息而己之真者亦飘泊无依。所为假者安在哉?彼于此时,应悔向日为假所误。然而人之真者,已黄土百年。彼留假傀儡,人则有真怨恨。怨恨深而不能吐,日酿一日,苍苍高天,茫茫碧海,吾何日而能忘也哉!眼泪洗面,椎心泣血,即百割此仇,何益于事!是此等酸法,一时一刻,酿成千百万年,死而有知,皆不能坏。此所以玉楼弹阮来,爱姐抱阮去,千秋万岁,此恨绵绵无绝期矣。故用普净以解冤偈结之。夫冤至于不可解之时,转而求其解,则此一刻之酸,当何如含耶?是愤已百二十分,酸又百二十分,不作《金瓶梅》,又何以消遣哉?甚矣!仁人志士、孝子悌弟,上不能告诸天,下不能告诸人,悲愤呜邑,而作秽言,以泄其愤。自云含酸,不是撒泼,怀匕囊锤,以报其人;是亦一举。乃作者固自有志,耻作荆、聂,寓复仇之义于百回微言之中,谁为刀笔之利不杀人于千古哉!此所以有《金瓶梅》也。


然则《金瓶梅》,我又何以批之也哉?我喜其文之洋洋一百回,而千针万线,同出一丝,又千曲万折,不露一线。闲窗独坐,读史、读诸家文,少暇,偶一观之曰:如此妙文,不为之递出金针,不几辜负作者千秋苦心哉!久之心恒怯焉,不敢遽操管以从事。盖其书之细如牛毛,乃千万根共具一体,血脉贯通,藏针伏线,千里相牵,少有所见,不禁望洋而退。迩来为穷愁所迫,炎凉所激,于难消遣时,恨不自撰一部世情书,以排遗闷怀。几欲下笔,而前后拮构,甚费经营,乃搁笔曰:“我且将他人炎凉之书,其所以前后经营者,细细算出,一者可以消我闷怀,二者算出古人之书,亦可算我今又经营一书。我虽未有所作,而我所以持往作书之法,不尽备于是乎!然则我自做我之《金瓶梅》,我何暇与人批《金瓶梅》也哉!

} % 张竹坡皋鹤堂批评第一奇书批评本

\piZhangF{ % 张竹坡皋鹤堂批评第一奇书批评本

\chapter*{冷热金针}
\addcontentsline{toc}{chapter}{冷热金针 -- 张竹坡}

《金瓶》以“冷热”二字开讲,抑熟不知此二字为一部之金钥乎?然于其点睛处,则未之知也。夫点睛处安在?曰:在温秀才、韩伙计。何则?韩者冷之别名,温者热之余气。故韩伙计于“加官”后即来,是热中之冷信。而温秀才自“磨镜”后方出,是冷字之先声。是知祸福倚伏,寒暑盗气,天道有然也。虽然,热与寒为匹,冷与温为匹,盖热者温之极,韩者冷之极也。故韩道国不出于冷局之后,而出热局之先,见热未极而冷已极。温秀才不来于热场之中,而来于冷局之首,见冷欲盛而热将尽也。噫嘻,一部言冷言热,何啻如花如火!而其点睛处乃以此二人,而数百年读者,亦不知其所以作韩、温二人之故。是作书者固难,而看书者为尤难,岂不信哉!

} % 张竹坡皋鹤堂批评第一奇书批评本

\piZhangF{ % 张竹坡皋鹤堂批评第一奇书批评本

\chapter*{寓意说}
\addcontentsline{toc}{chapter}{寓意说 -- 张竹坡}
稗官者,寓言也。其假捏一人,幻造一事,虽为风影之谈,亦必依山点石,借海扬波。故《金瓶》一部,有名人物不下百数,为之寻端竟委,大半皆属寓言。庶因物有名,托名摭事,以成此一百回
曲曲折折之书,如西门庆、潘金莲、王婆、武大、武二,《水浒传》中原有之人,《金瓶》因之者无论。然则何以有瓶、梅哉?瓶因庆生也。盖云贪欲嗜恶,面骸枯尽,瓶之罄矣。特特撰出瓶儿,直令千古风流人同声一哭。因瓶生情,则花瓶而子虚姓花,银瓶而银姐名银。瓶与屏通,窥春必于隙。屏号芙蓉,“玩赏芙蓉亭”盖为瓶儿插笋。而“私窥”一回卷首词内,必云“绣面芙蓉一笑开”。后“玩灯”一回《灯赋》内,荷花灯、芙蓉灯。盖金、瓶合传,是因瓶假屏,又因屏假芙蓉,浸淫以人于幻也。屏、风二字相连,则冯妈妈必随瓶儿,而当大理屏风、又点睛妙笔矣。芙蓉栽以正月,冶艳于中秋,摇落于九月,故瓶儿必生于九月十五,嫁以八月廿五,后病必于重阳,死以十月,总是《芙蓉谱》内时候。墙头物去,亲事杳然,瓶儿悔矣。故蒋文蕙将闻悔而来也者。然瓶儿终非所据,必致逐散,故又号竹山。总是瓶儿心事中生出此一人。如意为瓶儿后身,故为熊氏姓张。熊之所贵者胆也,是如意乃瓶胆一张耳。故瓶儿好倒插花,如意‘茎露独尝’,皆瓶与瓶胆之本色情景。官哥幻其名意,亦皆官窑哥窑,故以雪贼死之。瓶遇猫击,焉能不碎?银瓶坠井,千古伤心。故解衣而瓶儿死,托梦必于何家。银瓶失水矣,竹篮打水,成何益哉?故用何家蓝氏作意中人,以送西门之死,亦瓶之余意也。

至于梅,又因瓶而生。何则?瓶里梅花,春光无几。则瓶罄喻骨髓暗枯,瓶梅又喻衰朽在即。梅雪不相下,故春梅宠而雪娥辱,春梅正位而雪娥愈辱。月为梅花主人,故永福相逢,必云故主。而吴典恩之事,必用春梅襄事。冬梅为奇寒所迫,至春吐气,故“不垂别泪”,乃作者一腔炎凉痛恨发
于笔端。至周、舟同音,春梅归之,为载花舟。秀、臭同音,春梅遗臭载花舟且作粪舟。而周义乃野渡无人,中流荡漾,故永福寺里普净座前必用周义转世,为高留住儿,言须一篙留住,方登彼岸。

然则金莲,岂尽无寓意哉?莲与芰,类也;陈,旧也,败也;敬、茎同音。败茎芰荷,言莲之下场头。故金莲以敬济而败,“侥幸得金莲”,芰茎之罪。西门乃“打铁棍”,铁棍,芰茎影也,舍根而罪影,所谓糊涂。败茎不耐风霜,故至严州,而铁指甲一折即下。幸徐(山封)相救,风少劲即吹去矣。次后过街鼠寻风,是真朔风。风利如刀,刀利如风,残枝败叶,安得不摧哉!其父陈洪,已为露冷莲房坠粉红。其舅张团练搬去,又荷尽已无擎雨盖,留此败茎支持风雪,总写莲之不堪处。益知夏龙溪为金莲胜时写也。温秀才积至水秀才,至倪秀才,再至王潮儿,总言水枯莲谢,惟余数茎败叶潦倒污泥,所为风流不堪回首,无非为金莲污辱下贱写也。莲名金莲,瓶亦名金瓶,侍女偷金,莲、瓶相妒,斗叶输金,莲花飘萎,芸茎用事矣。他如宋蕙莲、王六儿,亦皆为金莲写也。写一金莲,不足以尽金莲之恶,且不足以尽西门、月娘之恶,故先写一宋金莲,再写一王六儿,总与潘金莲一而二,二而三者也。然而蕙莲,荻帘也,望子落,帘儿坠,含羞自缢,又为“叉竿挑帘”一回重作渲染。至王六儿,又黄芦儿别音,其娘家王母猪。黄芦与黄竹相类,其弟王经,亦黄芦茎之义。芦茎叶皆后空,故王六儿好干后庭花,亦随手成趣。芦亦有影,故看灯夜又用铁棍一觑春风,是芦荻皆莲之副,故曰二人皆为金莲写。此一部写金、写瓶、写梅之大梗概也。

若夫月娘为月,遍照诸花。生于中秋,故有桂儿为之女。“扫雪”而月娘喜,“踏雪”而月娘悲,月有阴晴明晦也。且月下吹箫,故用玉箫,月满兔肥,盈已必亏,故小玉成婚,平安即偷镀金钩子,到南瓦子里要。盖月照金钩于南瓦上,其亏可见。后用云里守人梦,月被云遮,小玉随之,与兔俱隐,情文明甚。

李娇儿,乃“桃李春风墙外枝”也。其弟李铭,言里明外暗,可发一笑。至贲四嫂与林太太,乃叶落林空,春光已去。贲四嫂姓叶,作“带水战”。西门庆将至其家,必云吩咐后生王显,是背面落水,显黄一叶也。林太太用文嫂相通,文嫂住捕衙厅前,女名金大姐,乃蜂衙中一黄蜂,所云蜂媒是也。此时爱月初宠,两番赏雪,雪月争寒,空林叶落,所莲花芙蓉,安能宁耐哉!故瓶死莲辱,独让春梅争香吐艳。而春鸿、春燕,又喻韶光迅速,送鸿迎燕,无有停息。来爵改名来友,见花事阑珊,燕莺遗恨。其妻惠元,三友会于园,看杜鹃啼血矣。内有玉箫勾引春风,外有玳安传消递息,箫有合欢之调,熏莲、惠元以之。箫有离别之音,故“三章约”乃阳关声。西门听之,能不动深悲耶?惹草粘花,必用玳安。一曰“嬉游蝴蝶巷”,再日“密访蜂媒”,已明其为蝶使矣,所谓“玳瑁斑花蝴蝶”非欤?书童则因箫而有名。盖篇内写月、写花、写雪,皆定名一人,惟风则止有冯妈妈。太守徐崶,虽亦一人。而非花娇月媚,正经脚色。故用书童与玉箫合,而萧疏之风动矣。未必云“私挂一帆”,可知其用意写风。然又通书为梳,故书童生于苏州府长熟县,字义可思。媚客之唱,必云“画损了掠儿稍”,接手云“贲四害怕”。“梳子在座,篦子害怕”,妙绝!《艳异》遗意,为男宠报仇。金莲必云“打了象牙”,明点牙梳。去必以瓶儿丧内,瓶坠簪折,牙梳零落,萧疏风起,春意阑珊,《阳关三叠》,大家将散场也。《金瓶》之大概寓言如此,其他剩意,不能殚述。推此观之,笔笔皆然。

至其写玉楼一人,则又作者经济学问,色色自喻皆到。试细细言之:玉楼簪上镌“玉楼人醉杏花天”,来自杨家,后嫁李家,遇薛嫂而受屈,遇陶妈妈而吐气,分明为杏无疑。可者,幸也。身毁名污,幸此残躯留于人世。而住居臭水巷。盖言元妄之来,遭此荼毒,污辱难忍,故着书以泄愤。嫁于李衙内,而李贵随之,李安往依之,以理为贵,以理为安。归于真定、枣强。真定,言吾心淡定;枣强,言黾勉工夫。所为勿助勿忘,此是作者学问。王杏庵送贫儿于晏公庙任道土为徒。晏,安也;任与人通,又与仁通。言“我若得志,必以仁道济天下,使天下匹夫匹妇,皆在晏安之内,以养其生;皆入于人伦之中,以复其性。”此作者之经济也。不谓有金道士淫之,又有陈三引之,言为今人声色货利浸淫已久,我方竭力养之教之,而今道又使其旧性复散,不可救援,相率而至于永福寺内,共作孤魂而后已。是可悲哉!夫永福寺,涌于腹下,此何物也?其内僧人,一曰胡僧,再曰道坚,一肖其形,一美其号。永福寺真生我之门死我户,故皆于死后同归于此,见色之利害。而万回长老,其回肠也哉。他如黄龙寺,脾也;相国寺,相火也。拜相国长老,归路避风黄龙,明言相火动而脾风发,故西门死气如牛吼,已先于东京言之矣。是玉皇庙,心也。二重殿后一重侧门,其心尚可问哉?故有吴道士主持结拜,心既无道,结拜何益?所以将玉皇庙始而永福寺结者,以此。

更有因一事而生数人者,则数名公同一义。如车(扯)淡、管世(事)宽、游守(手)、郝(好)贤(闲),四人共一寓意也。又如李智(枝)、黄四,梅、李尽黄,春光已暮,二人共一寓意也。又如‘带水战’一回,前云聂(捏)两湖、尚(上)小塘、汪北彦(沿),三人共一寓意也。又如安沈(枕)、宋(送)乔年,喻色欲伤生,二人共一寓意也。又有因——人而生数名者,应伯(白)爵(嚼)字光侯(喉),谢希(携)大(带)字子(紫)纯(唇),祝(住)实(十)念(年),孙天化(话)字伯(不)修(羞),常峙(时)节(借),卜(不)志(知)道,吴(无)典恩,云里守(手)字非(飞)去,白赖光字光汤,贲(背)第(地)传,傅(负)自新(心),甘(干)出身,韩道(捣)国(鬼)。因西门庆不肖,生出数名也。又有即物为名者,如吴神仙,乃镜也,名无夹,冰鉴照人无失也。黄真人,土也,瓶坠簪折,黄土伤心。末用楚云一人遥影,正是彩云易散。潘道士,撤也,死孽已成,撤着一做也。又有随手调笑,如西门庆父名达,盖明捏土音,言西门之达,即金莲所呼达达之达。设问其母何氏,当必云娘氏矣。桂姐接丁二官,打丁之人也。李(里)外传,取其传话之意。侯林儿,言树倒猢狲散。此皆掉手成趣处。他如张好问、白汝晃(谎)之类,不可枚举。随时会意,皆见作者狡滑之才。

若夫玉楼弹阮,爱姐继其后,抱阮以往湖州何官人家,依二捣鬼以终,是作者穷途有泪无可洒处,乃于爱河中捣此一篇鬼话。明亦无可如何之中,作书以自遣也。至其以孝哥结入一百回,用普净幻化,言惟孝可以消除万恶,惟孝可以永锡尔类,今使我不能全孝,抑曾反思尔之于尔亲,却是如何!千秋万岁,此恨绵绵,悠悠苍天,曷有其极,悲哉,悲哉!
\piWenglong{此批不必然,不必不然。在作者纯任其自然,批者欲求其所以然,遂未免强以为然。我谓有然有不然,不如视为莫知其然而然,斯统归于不期然而然,全付于天然,又何必争其然与不然哉!试起作者九原而问之,亦必哑然而笑,喟然而叹,悄然以悲,夷然不顾曰:其然岂其然乎?}


} % 张竹坡皋鹤堂批评第一奇书批评本

\piZhangF{ % 张竹坡皋鹤堂批评第一奇书批评本

\chapter*{苦孝说}
\addcontentsline{toc}{chapter}{苦孝说 -- 张竹坡}

夫人之有身,吾亲与之也。则吾之身,视亲之身为生死矣。若夫亲之血气衰老,归于大造,孝子有痛于中,是凡为人子者所同,而非一人独具之奇冤也。至于生也不幸,其亲为仇所算,则此时此际,以至千百万年,不忍一注目,不敢一存想,一息有知,一息之痛为无已。呜呼,痛哉!痛之不已,酿成奇酸,海枯石烂,其味深长。是故含此酸者,不敢独立默坐。苟独立默坐,则不知吾之身、吾之心、吾之骨肉,何以栗栗焉如刀斯割、如虫斯噬也。悲夫!天下尚有一境,焉能使斯人悦耳目、娱心志,一安其身也哉?苍苍高天,茫茫厚地,无可一安其身,必死用户庶几矣。然吾闻死而有有知之说,则奇痛尚在,是死亦无益于酸也。然则必何如而可哉?必何如而可,意者生而无我,死而亦无我。夫生而无我,死而亦无我,幻化之谓也。推幻化之谓,既不愿为人,又不愿为鬼,并不愿为水石。盖为水为石,犹必流石人之泪矣。呜呼!苍苍高天,茫茫厚地,何故而有我一人,致令幻化之难也?故作《金瓶梅》者,一曰“含酸”,再曰“抱阮”,结曰“幻化”,且必曰幻化孝哥儿,作者之心,其有余痛乎?则《金瓶梅》当名之曰《奇酸志》、《苦孝说》。呜呼!孝子,孝子,有苦如是!

} % 张竹坡皋鹤堂批评第一奇书批评本

\piZhangF{ % 张竹坡皋鹤堂批评第一奇书批评本

\chapter*{第一奇书非淫书论}
\addcontentsline{toc}{chapter}{第一奇书非淫书论 -- 张竹坡}

诗云“以尔车来,以我贿迁”,此非瓶儿等辈乎?又云“子不我思,岂无他人”,此非金、梅等辈乎??“狂且狡童”,此非西门、敬济等辈乎?乃先师手订,文公细注,岂不曰此淫风也哉!所以云“诗三百,一言以蔽之曰:思无邪。”注云:“诗有善有恶。善者起发人之善心,恶者惩创人之逆志。”圣贤着书立言之意,固昭然于千古也。今夫《金瓶梅》一书作者,亦是将《褰裳》、《风雨》、《箨兮》、《子衿》诸诗细为摹仿耳。夫微言之而文人知儆,显言之而流俗知惧。不意世之看者,不以为惩劝之韦弦,反以为行乐之符节,所以目为淫书,不知淫者自见其为淫耳。但目今旧板,现在金陵印刷,原本四处流行买卖。予小子悯作者之苦心,新同志之耳目,批此一书,其“寓意说”内,将其一部奸夫淫妇,翻批作草木幻影;一部淫词艳语,悉批作起伏奇文。至于以“睇”字起,“孝”字结,一片天命民彝,殷然慨侧,又以玉楼、杏庵照出作者学问经纶,使人一览无复有前此之《金瓶》矣。但恐不学风影等辈,借端恐虎,意在骗诈。夫现今通行发卖,原未禁止;小子穷愁着书,亦书生常事。又非借此沽名,本因家无寸土,欲觅蝇头以养生耳。即云奉行禁止,小子非套翻原板,固我自作我的《金瓶梅》。我的《金瓶梅》上洗淫乱而存孝悌,变帐簿以作文章,直使《金瓶》一书冰消瓦解,则算小子劈《金瓶梅》原板亦何不可!夫邪说当辟,而辟邪说者必就邪说而辟之,其说方息。今我辟邪说而人非之,是非之者必邪说也。若不予先辨明,恐当世君子为其所惑。况小子年始二十有六,素与人全无恩怨,本非借不律以泄愤懑;又非囊有余钱,借梨枣以博虚名:不过为糊口计。兰不当门,不锄何害?锄之何益?是用抒诚,以告仁人君子,共其量之。

} % 张竹坡皋鹤堂批评第一奇书批评本

\piZhangF{ % 张竹坡皋鹤堂批评第一奇书批评本

\chapter*{批评第一奇书《金瓶梅》读法}
\addcontentsline{toc}{chapter}{批评第一奇书《金瓶梅》读法 -- 张竹坡}

劈空撰出金、瓶、梅三个人来,看其如何收拢一块,如何发放开去。看其前半部止做金、瓶,后半部止做春梅。前半人家的金瓶,被他千方百计弄来,后半自己的梅花,却轻轻的被人夺去。(一)

起以玉皇庙,终以水福寺,而一回中已一齐说出,是大关键处。(二)

先是吴神仙总览其盛,后是黄真人少扶其衰,末是普净师一洗其业,是此书大照应处。(三)

“冰鉴定终身”,是一番结束,然独遗陈敬济。“戏笑卜龟儿”,又遗潘金莲。然金莲即从其自己口中补出,是故亦不遗金莲,当独遗西门庆与春梅耳。两番瓶儿托梦,盖又单补西门。而叶头陀相面,才为敬济一番结束也。(四)

未出金莲,先出瓶儿;既娶金莲,方出春梅;未娶金莲,却先娶玉楼;未娶瓶儿,又先出敬济。文字穿插之妙,不可名言。若夫夹写蕙莲、王六儿、贲四嫂、如意儿诸人,又极尽天工之巧矣。(五)

会看《金瓶》者,看下半部。亦惟会看者,单看上半部,如“生子加官”时,唱“韩湘子寻叔”、“叹浮生犹如一梦”等,不可枚举,细玩方知。(六)

《金瓶》有板定大章法。如金莲有事生气,必用玉楼在旁,百遍皆然,一丝不易,是其章法老处。他如西门至人家饮酒,临出门时,必用一人或一官来拜、留坐,此又是“生子加官”后数十回大章法。(七)

《金瓶》一百回,到底俱是两对章法,合其目为二百件事。然有一回前后两事,中用一语过节;又有前后两事,暗中一笋过下。如第一回,用玄坛的虎是也。又有两事两段写者,写了前一事半段,即写后一事半段,再完前半段,再完后半段者。有二事而参伍错综写者,有夹入他事写者。总之,以目中二事为条干,逐回细玩即知。(八)

《金瓶》一回,两事作对固矣,却又有两回作遥对者。如金莲琵琶、瓶儿象棋作一对,偷壶、偷金作一对等,又不可枚举。(九)

前半处处冷,令人不耐看;后半处处热,而人又看不出。前半冷,当在写最热处,玩之即知;后半热,看孟玉楼上坟,放笔描清明春色便知。(十)

内中有最没正经、没要紧的一人,却是最有结果的人,如韩爱姐是也。一部中,诸妇人何可胜数,乃独以爱姐守志结何哉?作者盖有深意存于其意矣。言爱姐之母为娼,而爱姐自东京归,亦曾迎人献笑,乃一留心敬济,之死靡他,以视瓶儿之于子虚,春梅之于守备,二人固当愧死。若金莲之遇

西门,亦可如爱姐之逢敬济,乃一之于琴童,再之于敬济,且下及王潮儿,何其比回心之娼妓亦不若哉?此所以将爱姐作结,以愧诸妇;且言爱姐以娼女回头,还堪守节,奈之何身居金屋而不改过海非,一竟丧廉寡耻,于死路而不返哉?(一一)

读《金瓶》,须看其大间架处。其大间架处,则分金、梅在一起,分瓶儿在一处,又必合金、瓶、梅在前院一处。金、梅合而瓶儿孤,前院近而金、瓶妒,月娘远而敬济得以下手也。(一二)

读《金瓶》,须看其入笋处。如玉皇庙讲笑话,插入打虎;请子虚,即插入后院紧邻;六回金莲才热,即借嘲骂处插入玉楼;借问伯爵连日那里,即插出桂姐;借盖卷棚即插入敬济,借翠管家插人王六儿;借翡
翠轩插入瓶儿生子;借梵僧药,插入瓶儿受病;借碧霞宫插入普净;借上坟插入李衙内;借拿皮袄插入玳安、小玉。诸如此类,不可胜数盖其用笔不露痕迹处也。其所以不露痕迹处,总之善用曲笔、逆笔,不肯另起头绪用直笔、顺笔也。夫此书头绪何限?若一一起之,是必不能之数也。我执笔时,亦必想用曲笔、逆笔,但不能如他曲得无迹、逆得不觉耳。此所以妙也。(一三)

《金瓶》有节节露破绽处。如窗内淫声,和尚偏听见;私琴童,雪娥偏知道;而裙带葫芦,更属险事;墙头密约,金莲偏看见;蕙莲偷期,金莲偏撞着;翡翠轩,自谓打听瓶儿;葡萄架,早已照人铁棍;才受赃,即动大巡之怒;才乞恩,便有平安之才;调婿后,西门偏就摸着;烧阴户,胡秀偏就看见。诸如此类,又不可胜数,总之,用险笔以写人情之可畏,而尤妙在既已露破,乃一语即解,绝不费力累赘。此所以为化笔也。(一四)

《金瓶》有特特起一事、生一人,而来既无端,去亦无谓,如书童是也。不知作者,盖几许经营,而始有书童之一人也。其描写西门淫荡,并及外宠,不必说矣。不知作者盖因一人之出门,而方写此书童也。何以言之?瓶儿与月娘始疏而终亲,金莲与月娘始亲而终疏。虽固因逐来昭、解来旺起衅,而未必至撒泼一番之甚也。夫竟至撒泼一番者,有玉箫不惜将月娘底里之言磬尽告之也。玉箫何以告之?曰有“三章约”在也。“三章”何以肯受?有书童一节故也。夫玉箫、书童不便突起炉灶,故写“藏壶构衅”于前也。然则遥遥写来,必欲其撒泼,何为也哉?必得如此,方于出门时月娘毫无怜惜,一弃不顾,而金莲乃一败涂地也。谁谓《金瓶》内有一无谓之笔墨也哉。(一五)

《金瓶》内正经写六个妇人,而其实止写得四个:月娘,玉楼,金莲,瓶儿是也。然月娘则以大纲故写之;玉楼虽写,则全以高才被屈,满肚牢骚,故又另出一机轴写之,然则以不得不写。写月娘,以不肯一样写;写玉楼,是全非正写也。其正写者,惟瓶儿、金莲。然而写瓶儿,又每以不言写之。夫以不言写之,是以不写处写之。以不写处写之,是其写处单在金莲也。单写金莲,宜乎金莲之恶冠于众人也。吁,文人之笔可惧哉!(一六)

《金瓶》内,有两个人为特特用意写之,其结果亦皆可观。如春梅与玳安儿是也。于同作丫鬟时,必用几遍笔墨描写春梅心高志大,气象不同;于众小厮内,必用层层笔墨,描写玳安色色可人。后文春梅作夫人,玳安作员外。作者必欲其如此何哉?见得一部炎凉书中翻案故也。何则?止知眼前作婢,不知即他日之夫人;止知眼前作仆,不知即他年之员外。不特他人转眼奉承,即月娘且转而以上宾待之,末路倚之。然则人之眼边前炎凉成何益哉!此是作者特特为人下砧砭也。因要他于污泥中为后文翻案,故不得不先为之抬高身分也。(一七)

李娇儿、孙雪娥,要此二人何哉?写一李娇儿,见其来遇金莲、瓶儿时,早已嘲风弄月,迎好卖俏,许多不肖事,种种可杀。是写金莲、瓶儿,乃实写西门之恶;写李娇儿,又虚写西门之恶。写出来的既已如此,其未写出来的时,又不知何许恶端不可问之事于从前也。作者何其深恶西门之如是!至孙雪娥,出身微贱,分不过通房,何其必劳一番笔墨写之哉?此又作者菩萨心也。夫以西门之恶,不写其妻作倡,何以报恶人?然既立意另一花样写月娘,断断不忍写月娘至于此也。玉楼本是无辜受毒,何忍更令其顶缸受报?李娇儿本是娼家,瓶儿更欲用之孽报于西门生前,而金莲更自有冤家债主在,且即使之为娼,于西门何损?于金莲似甚有益,乐此不苦,又何以言报也?故用写雪娥以至于为娼,以总张西门之报,且暗结宋蕙莲一段公案。至于张胜、敬济后事,则又情因文生,随手收拾。不然雪娥为娼,何以结果哉?(一八)

又娇儿色中之财,看其在家管库,临去拐财可见。王六儿财中之色,看其与西门交合时,必云做买卖,骗丫头房子,说合苗青。总是借色起端也。”(一九)

书内必写蕙莲,所以深潘金莲之恶于无尽也,所以为后文妒瓶儿时,小试行道之端也。何则?蕙莲才蒙爱,偏是他先知,亦如迎春唤猫。金莲睃见也。使春梅送火山洞,何异教西门早娶瓶儿,愿权在一块住也。蕙莲跪求,使尔舒心,且许多牢笼关锁,何异瓶儿来时,乘醉说一跳板走的话也。两舌雪娥,使激蕙莲,何异对月娘说瓶儿是非之处也。卒之来旺几死而未死,蕙莲可以不死而竟死,皆金莲为之也。作者特特于瓶儿进门加此一段,所以危瓶儿也。而瓶儿不悟,且亲密之,宜乎其祸不旋踵,后车终覆也。此深着金莲之恶。吾故曰:其小试行道之端,盖作者为不知远害者写一样子。若只随手看去,便说西门庆又刮上一家人媳妇子矣。夫西门庆,杀夫夺妻取其财,庇杀主之奴,卖朝廷之法,岂必于此特特撰此一事以增其罪案哉?然则看官每为作者瞒过了也。(二十)

后又写如意儿,何故哉?又作者明白奈何金莲,见其死蕙莲、死瓶儿之均属无益也。何则?蕙莲才死,金莲可一快。然而官哥生,瓶儿宠矣。及官哥死,瓶儿亦死,金莲又一大快。然而如意口脂,又从灵座生香,去掉一个,又来一个。金莲虽善固宠,巧于制人,于此能不技穷袖手,其奈之何?故作者写如意儿,全为金莲写,亦全为蕙莲、瓶儿愤也。(二—)

然则写桂姐、银儿、月儿诸妓,何哉?此则总写西门无厌,又见其为浮薄立品,市井为习。而于中写桂姐,特犯金莲;写银姐,特犯瓶儿;又见金、瓶二人,其气味声息,已全通娼家。虽未身为倚门之人,而淫心乱行,实臭味相投,彼娼妇犹步后尘矣。其写月儿,则另用香温玉软之笔,见西门一味粗鄙,虽章台春色,犹不能细心领略,故写月儿,又反衬西门也。(二二)

写王六儿、贲四嫂以及林太太何哉?曰:王六儿、贲四嫂、林太太三人是三样写法,三种意思。写王六儿干,专为财能致色一着做出来。你看西门在日,王六儿何等趋承,乃一旦拐财远遁。故知西门于六儿,借财图色,而王六儿亦借色求财。故西门死,必自王六儿家来,究竟财色两空。王六儿遇何官人,究竟借色求财。甚矣!色可以动人,尤未如财之通行无阻,人人皆爱也。然则写六儿,又似童讲财,故竟结入一百回内。至于贲四嫂,却为玳安写。盖言西门止知贪滥无厌,不知其左右亲随且上行下效,已浸淫乎欺主之风,而“窃玉成婚”,已伏线于此矣。若云陪写王六儿,犹是浅着。再至林太太,吾不知作者之心,有何干万愤懑,而于潘金莲发之。不但杀之割之,而并其出身之处、教习之人,皆欲致之死地而方畅也。何则?王招宣府内,故金莲旧时卖入学歌学舞之处也。今看其一腔机诈,丧廉寡耻,若云本自天生,则良心为不可必,而性善为不可据也。吾知其自二、三岁时,未必便如此淫荡也。使当日王招宣家男敦礼义,女尚贞廉,淫声不出于口,淫色不见于目,金莲虽淫荡,亦必化而为贞女。奈何堂堂招宣,不为天子招服远人,宣扬威德,而一裁缝家九岁女孩至其家,即费许多闲情,教其描眉画眼,弄粉涂朱,且教其做张做致,乔模乔样。其待小使女如此,则其仪型妻子可知矣。宜乎三官之不肖荒淫,林氏之荡闲(足俞)矩也。招宣实教之,夫复何尤!然则招宣教一金莲,以遗害无穷:身受其害者,前有武大,后有西门,而林氏为招宣还报,固其宜也。吾故曰:作者盖深恶金莲,而并恶及其出身之处,故写林太太也。然则张大户亦成金莲之恶者,何以不写?曰:张二官顶补西门千户之缺,而伯爵走动说娶娇儿,俨然又一西门,其受报亦必又有不可尽言者。则其不着笔墨处,又有无限烟波,直欲又藏一部大书于无笔处也。此所谓笔不到而意到者。(二三)

《金瓶》写月娘,人人谓西门氏亏此一人内助。不知作者写月娘之罪,纯以隐笔,而人不知也。何则?良人者,妻之所仰望而终身者也。若其夫千金买妾为宗嗣计,而月娘百依百顺,此诚《关雎》之雅,千古贤妇人也。若西门庆杀人之夫,劫人之妻,此真盗贼之行也。其夫为盗贼之行,而其妻不涕泣而告之,乃依违其间,视为路人,休戚不相关,而且自以好好先生为贤,其为心尚可问哉!至其于陈敬济,则作者已大书特书,月娘引贼人室之罪可胜言哉!至后识破奸情,不知所为分处之计,乃白口关门,便为处此已毕。后之逐敬济,送大姐,请春梅,皆随风弄柁,毫无成见;而听尼宣卷,胡乱烧香,全非妇女所宜。而后知“不甚读书”四字,误尽西门一生,且误尽月娘一生也。何则?使西门守礼,便能以礼刑其妻;今止为西门不读书,所以月娘虽有为善之资,而亦流于不知大礼,即其家常举动,全无举案之风,而徒多眉眼之处。盖写月娘,为一知学好而不知礼之妇人也。夫知学好矣,而不知礼,犹足遗害无穷,使敬济之恶归罪于己,况不学好者乎!然则敬济之罪,月娘成之,月娘之罪,西门庆刑于之过也。(二四)

文章有加一倍写法,此书则善于加倍写也。如写西门之执,更写蔡、宋二御史,更写六黄太尉,更写蔡太师,更写朝房,此加一倍热也。如写西门之冷,则更写一敬济在冷铺中,更写蔡太师充军,更写徽、钦北狩,真是加一倍冷。要之加一倍热,更欲写如西门之热者何限,而西门独倚财肆恶;加一倍冷者,正欲写如西门之冷者何穷,而西门乃不早见机也。(二五)

写月娘,必写其好佛者,人抑知作者之意乎?作者开讲,早已劝人六根清净,吾知其必以“空”结此“财色”二字也。安“空”字作结,必为僧乃可。夫西门不死,必不回头,而西门既死,又谁为僧?使月娘于西门一死,不顾家业,即削发入山,亦何与于西门说法?今必仍令西门自己受持方可。夫西门已死则奈何?作者几许踟蹰,乃以孝哥儿生于西门死之一刻,卒欲令其回头,受我度脱。总以圣贤心发菩萨愿,欲天下无终讳过之人,人无不改之过也。夫人之既死,犹望其改过于来生,然则作者之待西门何其忠厚慨恻,而劝勉于天下后世之人,何其殷殷不已也。是故既有此段大结束在胸中,若突然于后文生出一普净师幻化了去,无头无绪,一者落寻常窠臼,二者笔墨则脱落痕迹矣。故必先写月娘好佛,一路尸尸闪闪,如草蛇灰线。后又特笔
出碧霞宫,方转到雪涧,而又只一影普师,迟至十年,方才复收到永福寺。且于幻影中,将一部中有名人物,花开豆爆出来的,复一一烟消火灭了去。盖生离死别,各人传中皆自有结,此方是一总大结束。作者直欲使一部千针万线,又尽幻化了还之于太虚也。然则写月娘好佛,岂泛泛然为吃斋村妇闲写家常哉?此部书总妙在千里伏脉,不肯作易安之笔,没笋之物也是故妙绝群书。(二六)

又月娘好佛,内便隐三个姑子,许多隐谋诡计,教唆他烧夜香,吃药安胎,无所不为。则写好佛,又写月娘之隐恶也,不可不知。(二七)

内中独写玉楼有结果,何也?盖劝瓶儿、金莲二妇也。言不幸所天不寿,自己虽不能守,亦且静处金闺,令媒妁说合事成,虽不免扇坟之诮,然犹是孀妇常情。及嫁,而纨扇多悲,亦须宽心忍耐,安于数命。此玉楼俏心疡,高诸妇一着。春梅一味托大,玉楼一味胆小,故后日成就,春梅必竟有失身受嗜欲之危,而玉楼则一劳而永逸也。(二八)

陈敬济严州一事,岂不蛇足哉?不知作者一笔而三用也。一者为敬济堕落人冷铺作因,二者为大姐一死伏线,三者欲结玉楼实实遇李公子为百年知己,可偿在西门家三、四年之恨也。何以见之?玉楼不为敬济所动,固是心焉李氏,而李公子宁死不舍。天下有宁死不舍之情,非知己之情也哉?可必其无《白头吟》也。观玉楼之风韵嫣然,实是第一个美人,而西门乃独于一滥觞之金莲厚。故写一玉楼,明明说西门为市井之徒,知好淫,而且不知好色也。(二九)

玉楼来西门家,合婚过礼,以视“偷娶”“迎奸赴会”,何啻天壤?其吉凶气象已自不同。其嫁李衙内,则依然合婚行茶过礼,月娘送亲。以视老鸨争论,夜随来旺,王婆领出,不垂别泪,其明晦气象又自不同。故知作者特特写此一位真正美人,为西门不知风雅定案也。(三十)

金莲与瓶儿进门皆受辱。独玉楼自始至终无一褒贬。噫,亦有心人哉!(三一)

西门是混帐恶人,吴月娘是奸险好人,玉楼是乖人,金莲不是人,瓶儿是痴人,春梅是狂人,敬济是浮浪小人,娇儿是死人,雪娥是蠢人,宋蕙莲是不识高低的人,如意儿是顶缺之人。若王六儿与林太太等,直与李桂姐一流。总是不得叫做人。而伯爵、希大辈,皆是没良心的人。兼之蔡太师、蔡状元、宋御史,皆是枉为人也。(三二)

狮子街,乃武松报仇之地,西门几死其处。曾不数日,而于虚又受其害,西门徜徉来往。俟后王六儿,偏又为之移居此地。赏灯,偏令金莲两遍身历其处。写小入托大忘患,嗜恶不悔,一笔都尽。(三三)

《金瓶梅》是一部《史记》。然而《史记》有独传.有合传,却是分开做的。《金瓶梅》却是一百回共成一传,而千百人总合一传,内却又断断续续,各人自有一传,固知作《金瓶》者必能作《史记》也。何则?既已为其难,又何难为其易。(三四)

每见批此书者,必贬他书以褒此书。不知文章乃公共之物,此文妙,何妨彼文亦妙?我偶就此文之妙者而评之,而彼文之妙,固不掩此文之妙者也。即我自作一文,亦不得谓我之文出,而天下之文皆不妙,且不得谓天下更无妙文妙于此者。奈之何批此人之文,即若据为已有,而必使凡天下之文皆不如之。此其同心偏私狭隘,决做不出好文。夫做不出好文,又何能批人之好文哉!吾所谓《史记》易于《金瓶》,盖谓《史记》分做,而《金瓶》全做。即使龙门复生,亦必不谓予左袒《金瓶》。而予亦并非谓《史记》反不妙于《金瓶》,然而《金瓶》却全得《史记》之妙也。文章得失,惟有心者知之。我止赏其文之妙,何暇论其人之为古人,为后古之人,而代彼争论,代彼廉让也哉?(三五)

作小说者,概不留名,以其各有寓意,或暗指某人而作。夫作者既用隐恶扬善之笔,不存其人之姓名,并不露自己之姓名,乃后人必欲为之寻端竟委,说出名姓何哉?何其刻薄为怀也!且传闻之说,大都穿凿,不可深信。总之,作者无感慨,亦必不着书,一言尽之矣。其所欲说之人,即现在其书内。彼有感慨者,反不忍明言;我没感慨者,反必欲指出,真没搭撒、没要紧也。故“别号东楼”,“小名庆儿”之说,概置不问。即作书之人,亦止以“作者”称之。彼既不著名于书,予何多赘哉?近见《七才子书》,满纸王四,虽批者各自有意,而予则谓何不留此闲工,多曲折于其文之起尽也哉?偶记于此,以白当世。(三六)

《史记》中有年表,《金瓶》中亦有时日也。开口云西门庆二十七岁,吴神仙相面则二十九,至临死则三十三岁。而官哥则生于政和四年丙申,卒于政和五御丁酉。夫西门庆二十九岁生子,则丙申年;至三十三岁,该云庚子,而西门乃卒于“戊戌”。夫李瓶儿亦该云卒于政和五年,乃云“七年”,此皆作者故为参差之处。何则?此书独与他小说不同。看其三四年间,却是一日一时推着数去,无论春秋冷热,即某人生日,某
人某日来请酒,某月某日请某人,某日是某节令,齐齐整整捱去。若再将三五年间甲子次序,排得一丝不乱,是真个与西门计帐簿,有如世之无目者所云者也。故特特错乱其年谱,大约三五年间,其繁华如此。则内云某日某节,皆历历生动,不是死板一串铃,可以排头数去。而偏又能使看者五色眯目,真有如捱着一日日过去也。此为神妙之笔。嘻,技至此亦化矣哉!真千古至文,吾不敢以小说目之也。(三七)

一百回是一回,必须放开眼光作一回读,乃知其起尽处。(三八)

一百回不是一日做出,却是一日一刻创成。人想其创造之时,何以至于创成,便知其内许多起尽,费许多经营,许多穿插裁剪也。(三九)

看《金瓶》,把他当事实看,便被他瞒过,必须把他当文章看,方不被他瞒过也。(四十)

看《金瓶》,将来当他的文章看。犹须被他瞒过;必把他当自己的文章读,方不被他满过。(四一)

将他当自己的文章读,是矣。然又不如将他当自己才去经营的文章。我先将心与之曲折算出,夫而后谓之不能瞒我,方是不能瞒我也。(四二)

做文章,不过是“情理”二字。今做此一篇百回长文,亦只是“情理”二字。于一个人心中,讨出一个人的情理,则一个人的传得矣。虽前后夹杂众人的话,而此一人开口,是此一人的情理;非其开口便得情理,由于讨出这一人的情理方开口耳。是故写十百千人皆如写一人,而遂洋洋乎有此一百回大书也。(四三)

《金瓶》每于极忙时偏夹叙他事入内。如正未娶金莲,先插娶孟玉楼;娶玉楼时,即夹叙嫁大姐;生子时,即夹叙吴典恩借债;官哥临危时,乃有谢希大借银;瓶儿死时,乃人玉箫受约;择日出殡,乃有请六黄太尉等事;皆于百忙中,故作消闲之笔。非才富一石者何以能之?外加武松问傅伙计西门庆的话,百忙里说出“二两一月”等文,则又临时用轻笔讨神理,不在此等章法内算也。(四四)

《金瓶梅》妙在善于用犯笔而不犯也。如写一伯爵,更写一希大,然毕竟伯爵是伯爵,希大是希大,各人的身分,各人的谈吐,一丝不紊。写一金莲,更写一瓶儿,可谓犯矣,然又始终聚散,其言语举动,又各各不乱一丝。写一王六儿,偏又写一贲四嫂。写一李桂姐,偏又写一吴银姐、郑月儿。写一王婆,偏又写一薛媒婆、一冯妈妈、一文嫂儿、一陶媒婆。写一薛姑子,偏又写一王姑子、刘姑子。诸如此类,皆妙在、特特犯手,却又各各一款,绝不相同也。(四五)

《金瓶梅》于西门庆,不作一文笔;于月娘,不作一显笔;于玉楼,则纯用俏笔;于金莲,不作一钝笔;于瓶儿,不作一深笔;于春梅,纯用傲笔;于敬济,不作一韵笔;于大姐,不作一秀笔;于伯爵,不作一呆笔;于玳安儿,不着一蠢笔。此所以各各皆到也。(四六)

《金瓶梅》起头放过一男一女。结末又放去一男一女。如卜志道、卓丢儿,是起头放过者。楚云与李安,是结末放去者。夫起头放过去,乃云卜志道是花子虚的署缺者。不肯直出子虚,又不肯明是于十个中止写九个,单留一个缺去寻子虚顶补。故先着一人,随手去之,以出其缺,而便于出子虚,且于出子虚时,随手出瓶儿也。不然,先出子虚于十人之中,则将出瓶儿时又费笔墨。故卜志道虽为子虚署缺,又为瓶儿做楔子也。既云做一楔子,又何有顾意命名之义?而又必用一名,则只云“不知道”可耳,故云“卜志道”。至于丢儿,则又玉楼之署缺者。夫未娶玉楼,先娶此人,既娶玉楼,即丢开此人,岂如李瓶儿今日守灵,明朝烧纸,丫鬟奶子相伴空房,且一番两番托梦也。是诚丢开脑后之人,故云“丢儿”也。是其起头放过者,皆意在放过那人去,放人这人来也。至其结末放去者,’曰楚云者,盖为西门家中彩云易散作一影字。又见得美色无穷,人生有限,死到头来,虽有西子、王嫱,于我何涉?则又作者特特为起讲数语作证也。至于李安,则又与韩爱姐同意,而又为作者十二分满许之笔,写一孝子正人义士,以作中流砥柱也。何则?一部书中,上自蔡太师,下至侯林儿等辈,何止百有余人,并无一个好人,非迎奸卖俏之人,即附势趋炎之辈,使无李安一孝子,不几使良心种子灭绝手?看其写李安母子相依,其一篇话头,真见得守身如玉、不敢毁伤发肤之孝子。以视西门、敬济辈,真猪狗不如之人也。然则末节放过去的两人,又放不过众人,故特特放过此二人,以深省后人也。(四七)

写花子虚即于开首十人中,何以不便出瓶儿哉?夫作者于提笔时,固先有一瓶儿在其意中也。先有一瓶儿在其意中,其后如何偷期,如何迎奸,如何另嫁竹山,如何转嫁西门,其着数俱已算就。然后想到其夫,当令何名,夫不过令其应名而已,则将来虽有如无,故名之曰“子虚”。瓶本为花而有,故即姓花。忽然于出笔时,乃想叙西门氏正传也。于叙西门传中,不出瓶儿,何以入此公案?特叙瓶儿,则叙西门起头时,何以说隔壁一家姓花名某,某妻姓李名某也?此无头绪之笔,必不能人也。然则俟金莲进门再叙何如?夫他小
说,便有一件件叙去,另起头绪于中,惟《金瓶梅》,纯是太史公笔法。夫龙门文字中,岂有于一篇特特着意写之人,且十分有八分写此人之人,而于开卷第一回中不总出枢纽,如衣之领,如花之蒂,而谓之太史公之文哉?近人作一本传奇,于起头数折,亦必将有名人数点到。况《金瓶梅》为海内奇书哉!然则作者又不能自己另出头绪说。势必借结弟兄时,入花子虚也。夫使无伯爵一班人先与西门打热,则弟兄又何由而结?使写子虚亦在十人数内,终朝相见,则于第一回中西门与伯爵会时,子虚系你知我见之人,何以开口便提起“他家二嫂”?即提起二嫂,何以忽说“与咱院子止隔一墙?”而二嫂又何如好也哉?故用写子虚为会外之人,今日拉其人会,而因其邻墙,乃用西门数语,则瓶儿已出,邻墙已明,不言之表,子虚一家皆跃然纸上。因又算到不用卜志道之死,又何因想起拉子虚入会?作者纯以神工鬼斧之笔行文,故曲曲折折,止令看者眯目,而不令其窥彼金针之一度。吾故曰:纯是龙门文字。每于此等文字,使我悉心其中,曲曲折折,为之出入其起尽。何异人五岳三岛,尽览奇胜?我心乐此,不为疲也。(四八)

《金瓶》内,即一笑谈,一小曲,皆因时致宜,或直出本回之意,或足前回,或透下回,当于其下另自分注也。(四九)

《金瓶梅》一书,于作文之法无所不备,一时亦难细说,当各于本回前着明之。(五十)

《金瓶梅》说淫话,止是金莲与王六儿处多,其次则瓶儿,他如月娘、玉楼止一见,而春梅则惟于点染处描写之。何也?写月娘,惟“扫雪”前一夜,所以丑月娘、丑西门也。写玉楼,惟于“含酸”一夜,所以表玉楼之屈,而亦以丑西门也。是皆非写其淫荡之本意也。至于春梅,欲留之为炎凉翻案,故不得不留其身分,而止用影写也。至于百般无耻,十分不堪,有桂姐、月儿不能出之于口者,皆自金莲、六儿口中出之。其难堪为何如?此作者深罪西门,见得如此狗彘,乃偏喜之,真不是人也。故王六儿、潘金莲有日一齐动手,西门死矣。此作者之深意也。至于瓶儿,虽能忍耐,乃自讨苦吃,不关人事,而气死子虚,迎奸转嫁,亦去金莲不远,故亦不妨为之驰张丑态。但瓶儿弱而金莲狠,故写瓶儿之淫,略较金莲可些。而亦早自丧其命于试药之时,甚言女人贪色,不害人即自害也。吁,可畏哉!若蕙莲、如意辈,有何品行?故不妨唐突。而王招宣府内林太太者,我固云为金莲波及,则欲报应之人,又何妨唐突哉!(五一)

《金瓶梅》不可零星看,如零星,便止看其淫处也。故必尽数日之间,一气看完,方知作者起伏层次,贯通气脉,为一线穿下来也。(五二)

凡人谓《金瓶》是淫书者,想必伊止知看其淫处也。若我看此书,纯是一部史公文字。(五三)

做《金瓶梅》之人,若令其做忠臣孝子之文,彼必能又出手眼,摹神肖影,追魂取魄,另做出一篇忠孝文字也。我何以知之?我于其摹写奸夫淫妇知之。(五四)

今有和尚读《金瓶》,人必叱之,彼和尚亦必避人偷看;不知真正和尚方许他读《金瓶梅》。(五五)

今有读书者看《金瓶》,无论其父母师傅禁止之,即其自己亦不敢对人读。不知真正读书者,方能看《金瓶梅》,其避人读者,乃真正看淫书也。(五六)

作《金瓶》者,乃善才化身,故能百千解脱,色色皆到。不然正难梦见。(五七)

作《金瓶》者,必能转身证菩萨果。盖其立言处,纯是麟角凤嘴文字故也。(五八)

作《金瓶梅》者,必曾于患难穷愁,人情世故,一一经历过,人世最深,方能为众脚色摹神了。(五九)

作《金瓶梅》,若果必待色色历遍才有此书,则《金瓶梅》又必做不成也。何则?即如诸淫妇偷汉,种种不同,若必待身亲历而后知之,将何以经历哉?故知才子无所不通,专在一心也。(六十)

一心所通,实又真个现身一番,方说得一番。然则其写诸淫妇,真乃各现淫妇人身,为人说法者也。(六一)

其书凡有描写,莫不各尽人情。然则真千百化身现各色人等,为之说法者也。(六二)

其各尽人情,莫不各得天道。即千古算来,天之祸淫福善,颠倒权奸处,确乎如此。读之,似有一人亲曾执笔,在清河县前,西门家里,大大小小,前前后后,碟儿碗儿,一—记之,似真有其事,不敢谓为操笔伸纸做出来的。吾故曰:得天道也。(六三)

读《金瓶》,当看其白描处。子弟能看其白描处,必能自做出异样省力巧妙文字来也。(六四)

读《金瓶》,当看其脱卸处。子弟看其脱卸处,必能自出手眼,作过节文字也。(六五)

读《金瓶》,当看其避难处。子弟看其避难就易处,必能放重笔拿轻笔,异样使乖脱滑也。(六六)

读《金瓶》,当看其手闲事忙处。子弟会得,便许作繁衍文字也。(六七)

读《金瓶》,当看其穿插处。子弟会得,便许他作花团锦簇、五色眯人的文字也。(六八)

读《金瓶》,当看其结穴发脉、关锁照应处。子弟会得,才许他读《左》、《国》、《庄》、《骚》、《史》、子也。(六九)

读《金瓶》,当知其用意处。夫会得其处处所以用意处,方许他读《金瓶梅》,方许他自言读文字也。(七十)

幼时在馆中读文,见窗友为先生夏楚云:“我教你字宇想来,不曾教你囫轮吞。”予时尚幼,旁听此言,即深自儆省。于念文时,即一字一字作昆腔曲,拖长声,调转数四念之,而心中必将此一字,念到是我用出的一字方罢。犹记念的是“好古敏以求之”一句的文字,如此不三日,先生出会课题,乃“君子矜而不争”,予自觉做时不甚怯力而文成。先生大惊,以为抄写他人,不然何进益之速?予亦不能白。后先生留心验予动静,见予念文,以头伏桌,一手指文,一字一字唱之,乃大喜曰:“子不我欺”。且回顾同窗辈曰:“尔辈不若也”。今本不通,然思读书之法,断不可成片念过去。岂但读文,即如读《金瓶梅》小说,若连片念去,便味如嚼蜡,止见满篇老婆舌头而已,安能知其为妙文也哉!夫不看其妙文,然则止要看其妙事乎?是可一大揶揄。(七一)

读《金瓶》,必须静坐三月方可。否则眼光模糊,不能激射得到。(七二)

才不高,由于心粗,心粗由于气浮。心粗则气浮,气愈浮则心愈粗。岂但做不出好文,‘并亦看不出好文。遇此等人,切不可将《金瓶梅》与他读。(七三)

未读《金瓶梅》,而文字如是,既读《金瓶梅》,而文字犹如是。此人直须焚其笔砚,扶犁耕田为大快活,不必再来弄笔砚,自讨苦吃也。(七四)

做书者是诚才子矣,然到底是菩萨学问,不是圣贤学问,盖其专教人空也。若再进一步,到不空的所在,其书便不是这样做也。(七五)

《金瓶》以空结,看来亦不是空到地的,看他以孝哥结便知。然则所云“幻化”,乃是以孝化百恶耳。(七六)

《金瓶梅》到底有一种愤懑的气象,然则《金瓶梅》断断是龙门再世。(七七)

《金瓶梅》是部改过的书,观其以爱姐结便知。盖欲以三年之艾,治七年之病也。(七八)

《金瓶梅》究竟是大彻悟的人做的,故其中将僧尼之不肖处,一一写出。此方是真正菩萨,真正彻悟。(七九)

《金瓶梅》倘他当日发心不做此一篇市井的文字,他必能另出韵笔。,作花娇月媚如《西厢》等文字也。(八十)

《金瓶》必不可使不会做文的人读。夫不会做文字人读,则真有如俗云“读了《金瓶梅》”也。会做文字的人读《金瓶》,纯是读《史记》。(八一)

《金瓶梅》切不可令妇女看见。世有销金帐底,浅斟低唱之下,念一回于妻妾听者多多矣。不知男子中尚少知劝戒观感之人,彼女子中能观感者几人哉?少有效法,奈何奈何!至于其文法笔法,又非女子中所能学,亦不必学。即有精通书史者,则当以《左》、《国》、《风雅》、经史与之读也。然则,《金瓶梅》是不可看之书也,我又何以批之以误世哉?不知我正以《金瓶》为不可不看之妙文,特为妇人必不可看之书,恐前人呕心呕血做这妙文——虽本自娱,实亦欲娱千百世之锦绣才子者——乃为俗人所掩,尽付流水,是谓人误《金瓶》。何以谓西门庆误《金瓶》?使看官不作西门的事读,全以我此日文心,逆取他当日的妙笔,则胜如读一部《史记》。乃无如开卷便止知看西门庆如何如何,全不知作者行文的一片苦心,是故谓之西门庆误《金瓶梅》。然则仍依旧看官误看了西门庆的《金瓶梅》,不知为作者的《金瓶》也。常见一人批《金瓶梅》曰:“此西门之大帐簿”。其两眼无珠,可发一笑。夫伊于甚年月日,见作者雇工于西门庆家写帐簿哉?有读至敬济“弄一得双”,乃为西门大愤日:“何其剖其双珠!”不知先生又错看了也。金莲原非西门所固有,而作者特写一春梅,亦非欲为西门庆所能常有之人而写之也。此自是作者妙笔妙撰,以行此妙文,何劳先生为之旁生瞎气哉了故读《金瓶》者多,不善读《金瓶》者亦多。予因不揣,乃急欲批以请教。虽不敢谓能探作者之底里,然正因作者叫屈不歇,故不择狂瞽,代为争之。且欲使有志作文者,同醒一醒长日睡魔,少补文家之法律也。谁曰不宜?(八二)

《金瓶》是两半截书。上半截热,下半截冷;上半热中有冷,下牛冷中有热。(八三)

《金瓶梅》因西门庆一分人家,写好几分人家。如武大一家,花子虚一家,乔大户一家,陈洪一家,吴大舅一家,张大户一家,王招宣一家,应伯爵一家,周守备一家,何千户一家,夏提刑一家。他如悴云峰,在东京不算。伙计家以及女眷不往来者不算。凡这几家,大约清河县官员大户,屈指

已遍。而因一人写及一县,吁!元恶大惇论此回有几家,全倾其手,深遭荼毒也,可恨,可恨!(八四)

《金瓶梅》写西门庆无一亲人,上无父母,下无子孙,中无兄弟。幸而月娘犹不以继室自居。设也月娘因金莲终不通言对面,吾不知西门庆何乐乎为人也。乃于此不自改过自修,且肆恶无忌,宜乎就死不悔也。(八五)

书内写西门许多亲戚,通是假的。如乔亲家,假亲家也;翟亲家,愈假之亲家也;杨姑娘,谁氏之姑娘?假之姑娘也;应二哥,假兄弟也;谢子纯,假朋友也。至于花大舅、二舅,更属可笑,真假到没文理处也。敬济两番披麻戴孝,假孝子也。至于沈姨夫、韩姨夫,不闻有姨娘来,亦是假姨夫矣。惟吴大舅、二舅,而二舅又如鬼如蜮,吴大舅少可,故后卒得吴大舅略略照应也。彼西门氏并无一人,天之报施亦惨,而文人恶之者亦毒矣。奈何世人于一本九族之亲,乃漠然视之,且恨不排挤而去之,是何肺腑!(八六)

《金瓶》何以必写西门庆孤身一人,无一着己亲哉?盖必如此,方见得其起头热得可笑,后文一冷便冷到彻底,再不能热也。(八七)

作者直欲使此清河县之西门氏冷到彻底,并无一人。虽属寓言,然而其恨此等人,直使之千百年后,永不复望一复燃之灰。吁!文人亦狠矣哉!(八八)

《金瓶》内有一李安,是个孝子。却还有一个王杏庵,是个义士。安童是个义仆,黄通判是个益友,曾御史是忠臣,武二郎是个豪杰悌弟。谁谓一片淫欲世界中,天命民懿为尽灭绝也哉?(八九)

《金瓶》虽有许多好人,却都是男人,并无一个好女人。屈指不二色的,要算月娘一个。然却不知妇道以礼持家,往往惹出事端。至于爱姐,晚节固可佳,乃又守得不正经的节,且早年亦难清白。他如葛翠屏,娘家领去,作者固未定其末路,安能必之也哉?甚矣,妇人阴性,虽岂无贞烈者?然而

失守者易,且又在各人家教。观于此,可以禀型于之惧矣,齐家者可不慎哉?(九十)

《金瓶梅》内却有两个真人,一尊活佛,然而总不能救一个妖僧之流毒。妖僧为谁?施春药者也。(九一)

武大毒药,既出之西门庆家,则西门毒药,固有人现身而来。神仙、真人、活佛,亦安能逆天而救之也哉!(九二)

读《金瓶》,不可呆看,一呆看便错了。(九三)

读《金瓶》,必须置唾壶于侧,庶便于击。(九四)

读《金瓶》,必须列宝剑于右,或可划空泄愤。(九五)

读《金瓶》,必须悬明镜于前,庶能圆满照见。(九六)

读《金瓶》,必置大白于左,庶可痛饮,以消此世情之恶。(九七)

读《金瓶》,必置名香于几,庶可遥谢前人,感其作妙文,曲曲折折以娱我。(九八)

读《金瓶》,必须置香茗于案,以奠作者苦心。(九九)

《金瓶》亦并不晓得有甚圆通,我亦正批其不晓有甚圆通处也。(一百)

《金瓶》以“空”字起结,我亦批其以“空”字起结而已,到底不敢以“空”字诬我圣贤也。(百一)

《金瓶》以“空’’字起吉,我亦批其以“空”字起结而已,到底不敢以“空”字诬我圣贤也。(百二)

《金瓶》处处体贴人情天理,此是其真能悟彻了,此是其不空处也。(百三)

《金瓶梅》是大手笔,却是极细的心思做出来者。(百四)

《金瓶梅》是部惩人的书,故谓之戒律亦可。虽然又云《金瓶梅》是部人世的书,然谓之出世的书亦无不可。(百五)

金、瓶、梅三字连贯者,是作者自喻。此书内虽包藏许多春色,却一朵一朵,一瓣一瓣,费尽春工,当
注之金瓶,流香芝室,为千古锦绣才子作案头佳玩,断不可使村夫俗子作枕头物也。噫!夫金瓶梅花,全凭人力以补天王,则又如此书处处以文章夺化工之巧也夫。(百六)

此书为继《杀狗记》而作。看他随处影写兄弟,如何九之弟何十,杨大郎之弟杨二郎,周秀之弟周宣,韩道国之弟韩二捣鬼。惟西门庆、陈敬济无兄弟可想。(百七)

以玉楼弹阮起,爱姐抱阮结,乃是作者满肚皮猖狂之泪没处洒落,故以《金瓶梅》为大哭地也。(百八)

} % 张竹坡皋鹤堂批评第一奇书批评本


%\piWenglongF{ % 文龙在兹堂本手书批评本
%\chapter*{wenlong?}
%\addcontentsline{toc}{chapter}{wenlong?}
%} % 文龙在兹堂本手书批评本


%\endmyonkyoh

%\cleardoublepage
\tableofcontents

\mainmatter

%\pagestyle{main}

%%# -*- coding: utf-8 -*-
%!TEX encoding = UTF-8 Unicode
%!TEX TS-program = xelatex
% vim:ts=4:sw=4
%
% 以上设定默认使用 XeLaTex 编译,并指定 Unicode 编码,供 TeXShop 自动识别

\commentsr{
\chapter*{测试 1 (1) test}
\addcontentsline{toc}{chapter}{测试 1}

漢字源𣴑考


%\usepackage{CJKfntef}
%\CJKunderwave{漢字源𣴑考}

\begin{table}[ht]
\caption{生僻字测试}
\centering
\begin{tabular}{|cc|cc|cc|cc|}
𠀀 & 20000 & 𠀁 & 20001 & 𠀂 & 20002 & 𠀃 & 20003 \\
𠀄 & 20004 & 𠀅 & 20005 & 𠀆 & 20006 & 𠀇 & 20007 \\
𠀈 & 20008 & 𠀉 & 20009 & 𠀊 & 2000A & 𠀋 & 2000B \\
𠀌 & 2000C & 𠀍 & 2000D & 𠀎 & 2000E & 𠀏 & 2000F \\
𠀐 & 20010 & 𠀑 & 20011 & 𠀒 & 20012 & 𠀓 & 20013 \\
𠀔 & 20014 & 𠀕 & 20015 & 𠀖 & 20016 & 𠀗 & 20017 \\
𠀘 & 20018 & 𠀙 & 20019 & 𠀚 & 2001A & 𠀛 & 2001B \\
𠀜 & 2001C & 𠀝 & 2001D & 𠀞 & 2001E & 𠀟 & 2001F \\

{\char"028C46} & 028C46 &
{\char"021C1E} & 021C1E & % 𡰞
%\XeTeXglyph35897 & 35897 & % Here we use an index number to display a glyph.
\\
\end{tabular}
\end{table}

一丁上下

目录 %\footnote{test 1 footnote}

第二回

第二囘


頭髮䯼髻 %\marginnote{margin note example}

抹胸兒重重紐扣,

褲腿兒臟頭垂下

紅紗膝褲扣鶯花 %\marginnote{1第1个边角註.}
%\marginnote{2这个是第2个.}

常在公門操鬬毆
我卻怎生鬬得過他

此事便獲得着

言文真飴\marginnote[1第1个边角註(左).]{2这个是第2个(右).}

为為了

青月令育 %\footnote{test 2 footnote}

者老出山

追活沾信

華女紫瓜

波稃浮


\ifnum\strcmp{\myfnotemode}{\detokenize{gezhu}}=0

\chapter*{测试 2 (2) test}
\addcontentsline{toc}{chapter}{测试 2}

(GEZHU在 chapter內)

\begin{withgezhu}


話說武松自從搬離哥後,撚指不覺雪晴,過了十數日光景。都說本縣知縣,自從到任以來,都得二年有餘,
轉得許多金銀,
\gezhu{(轉得許多金銀)「轉」容本《忠義水滸傳》做「撰」,稍後如楊定見本、芥子園本、金聖歎貫華堂七十回本水滸傳均做「賺」。而前此世德堂本《西遊記》,已用「賺」代「撰」。本書「轉」、「撰」、「賺」並用。第五十三回:「賺得些中錢,又來撒漫了」;第七十六回:「家中胡亂積賺了些小本經紀」;第九十八回:「別無生意,只靠老婆錢賺」。下凡「轉」、「撰」取錢物,均統一為「賺」,不再一一出校。}
\marginnote{good}
要使一心腹人,送上東京親眷處收寄。三年任滿朝覲,打點上司。一來都怕路上小人,須得一個有力量的人去方好。猛可想起都頭武松,須得此人英雄膽力,方了得此事。當日就喚武松到衙內商議,道:「我有個親戚,在東京城內做官,姓朱名勔,見做殿前太尉之職。要送一擔禮物,
稍封書
\footnote{(稍封書)「稍」此處代「捎」。劉改「捎」。下凡以「稍」代「捎」,隨文改正,不再出校。}
去問安。只恐途中不好行,須得你去方可。你休推辭辛苦,回來我自重賞你!」武松應道:「小人得蒙恩相抬舉,安敢推辭?既蒙差遣,只得便去。小人自來也不曾到東京,就那裡觀光上國景致,走一遭,也是恩相抬舉。」知縣大喜,賞了武松三盃酒,十兩路費,不在話下。

\end{withgezhu}




%\begin{withgezhu}

%\chapter{测试 3 (3) test}
%(GEZHU包括 chapter)


%話說武松自從搬離哥後,撚指不覺雪晴,過了十數日光景。都說本縣知縣,自從到任以來,都得二年有餘,
%轉得許多金銀,
%\gezhu{(轉得許多金銀)「轉」容本《忠義水滸傳》做「撰」,稍後如楊定見本、芥子園本、金聖歎貫華堂七十回本水滸傳均做「賺」。而前此世德堂本《西遊記》,已用「賺」代「撰」。本書「轉」、「撰」、「賺」並用。第五十三回:「賺得些中錢,又來撒漫了」;第七十六回:「家中胡亂積賺了些小本經紀」;第九十八回:「別無生意,只靠老婆錢賺」。下凡「轉」、「撰」取錢物,均統一為「賺」,不再一一出校。}
%\marginnote{good}
%要使一心腹人,送上東京親眷處收寄。三年任滿朝覲,打點上司。一來都怕路上小人,須得一個有力量的人去方好。猛可想起都頭武松,須得此人英雄膽力,方了得此事。當日就喚武松到衙內商議,道:「我有個親戚,在東京城內做官,姓朱名勔,見做殿前太尉之職。要送一擔禮物,
%稍封書
%\footnote{(稍封書)「稍」此處代「捎」。劉改「捎」。下凡以「稍」代「捎」,隨文改正,不再出校。}
%去問安。只恐途中不好行,須得你去方可。你休推辭辛苦,回來我自重賞你!」武松應道:「小人得蒙恩相抬舉,安敢推辭?既蒙差遣,只得便去。小人自來也不曾到東京,就那裡觀光上國景致,走一遭,也是恩相抬舉。」知縣大喜,賞了武松三盃酒,十兩路費,不在話下。

%\end{withgezhu}

\fi
}


%\includechapters{4}
\myincchapge{00}{1}
\myincchapge{00}{2}
\myincchapge{00}{3}
\myincchapge{00}{4}
\myincchapge{00}{5}
\myincchapge{00}{6}
\myincchapge{00}{7}
\myincchapge{00}{8}
\myincchapge{00}{9}
\myincchapshi{0}{1}
\myincchapshi{0}{2}
\myincchapshi{0}{3}
\myincchapshi{0}{4}
\myincchapshi{0}{5}
\myincchapshi{0}{6}
\myincchapshi{0}{7}
\myincchapshi{0}{8}
\myincchapshi{0}{9}
\myincchapge{10}{0}

%\myincfig{jpmcz-004-b.jpg}{捉奸情鄆哥定記}
%\myincfig{jpmcz-013-a.jpg}{李瓶姐牆頭密約}
%\myincfig{jpmcz-026-a.jpg}{西門慶暗算來旺兒}
%\myincfig{jpmcz-037-b.jpg}{西門慶包占王六兒}
%\myincfig{jpmcz-038-b.jpg}{潘金蓮雪夜弄琵琶}
%\myincfig{jpmcz-042-a.jpg}{逞豪華門前放煙火}
%\myincfig{jpmcz-055-a.jpg}{西門慶兩番慶壽旦}
%\myincfig{jpmcz-059-b.jpg}{李瓶兒睹物哭官哥}
%\myincfig{jpmcz-089-a.jpg}{清明節寡婦上新墳}
%\myincfig{jpmcz-096-a.jpg}{春梅姐遊舊家池館}


%\endmyonkyoh



\pagestyle{special}
\appendix

%# -*- coding: utf-8 -*-
% !TeX encoding = UTF-8 Unicode
% !TeX spellcheck = en_US
% !TeX TS-program = xelatex
%~ \XeTeXinputencoding "UTF-8"
% vim:ts=4:sw=4
%
% 以上設定默認使用 XeLaTex 編譯,並指定 Unicode 編碼,供 TeXShop 自動識別

\chapter*{金瓶梅出版情况}
\addcontentsline{toc}{chapter}{金瓶梅出版情况}

金瓶梅词话100回

(明)兰陵笑笑生著;梅节校订、陈诏、黄霖注释

香港:梦梅馆,1993

17X约48;1375页;200图

该出版社于1988年出版了全校本《金瓶梅词话》,被誉为最接近《金瓶梅》原本。此重校本对全校本的校订,主要有以下几个方面:原本不误,全校本误改的;原本有误,全校本只部分改正,并不彻底的;原本有误,全校本存而未改的;原本有误,全校本未发现,或虽发现而尚未悉致误之由,无法订正的。由陈诏和黄霖作注释,最后经梅节审定增删。书前附有梅节所作的〈全校本金瓶梅词话前言〉,内容与香港星海文化出版有限公司 于1987年出版的《金瓶梅词话》(全校万历本)梅节所作的〈全校本金瓶梅词话前言〉大致相同,主要交代所根据的本子及校点原则。重校本直接删去衍文,不再用圆括号()标记;增文也不用方斜括号〔〕作标记。崇祯 本200幅插图,全校本原附在卷末,此本则分插在每回之前。此重校本增加了陈诏、黄霖两家的注释,删去全校万历本的〈词典〉。




金瓶梅词话(全校万历本)100回

(明)兰陵笑笑生著

香港:星海文化出版有限公司,1987

17X约47;1300页;图200幅

据梅节的〈全校本金瓶梅词话前言〉所说,此书以日本大安株式会社出版的《金瓶梅词话》为底本,原系根据日光山轮王寺慈眼堂及德山毛利氏栖息堂所藏《金瓶梅词话》补配。参考台北联经朱墨二色套印本和北京文学古籍刊行 社1957年重印本,校以以下几个本子:

1)《新刻绣像批评金瓶梅》,台北天一出版社影印日本内阁文库藏本。此本旧称“崇祯本”,不确,今简称“廿卷本”。(本馆藏有此影印本)

2)《皋鹤堂批评第一奇书金瓶梅》,在兹堂本,简称“大字张本”。台北里仁书局影印本。

3)《皋鹤堂批评第一奇书金瓶梅》,崇经堂巾箱本,简称“小字张本”,香港文乐出版社与在兹堂本合印本。

此本同时还参考了几种近人点校的本子:

1)《中国文学珍本丛书》本《金瓶梅词话》(删节本),施蛰存点校,上海杂志出版公 司1936年出版,简称“施本”。

2)全标点本《金瓶梅词话》,毛子水序,台北增你智文化事业有限公 司1976年出版,简称“增本”。

3)《中国文学名著》本《金瓶梅》(删节本),刘本栋校订,台北三民书 局1980年出版,简称“刘本”。

4)《中国小说史料丛书》本《金瓶梅词话》(删节本),戴鸿森校点,北京人民文学出版 社1985年出版,简称“戴本”。

〈校记〉共五千余条,分系各回之后。正文校改、增文二字以上用方斜括号〔〕,衍文二字以上用圆括号(),夹文用方括 号[ ],阙文用方框□,书名及词曲牌名用尖括号〈〉,书末附二十卷本的明代插图和梅节主编的〈金瓶梅词话辞典〉。该出版社还另外出版全校本的普及本,不附校记。





金瓶梅词话万历本2函20册

(明)兰陵笑笑生

台北:联经出版事业公司,1978

11X24;1579页;图200幅

据此本的〈出版说明〉,万历丁巳(1617)刊本《金瓶梅词话》是《金瓶梅》一书现存的最早木刻版本, 共20册,目前珍藏在台北故宫博物院。1933年,北京古佚小说刊行会据以影印一百部行世。

这部联经版的《金瓶梅词话》,就是依据傅斯年先生所藏古佚小说刊行会影印本,并比对故宫博物院珍藏的万历丁巳(1617)本,整理后影印。

此本原无插图,北平古佚小说刊行会影印本补入崇祯本木刻插图200幅。每回各两图,在每回之前,与马廉藏崇祯本《新刻绣像金瓶梅》相同。此本书眉及行间间有朱笔批语。




金瓶梅词话(万历本)6卷

(明)笑笑生作

东京:大安株式会社,1963

11X24;2941页;图200幅

据此本的〈例言〉,“吾邦所传明刊本金瓶梅词话之完全者有两部,日光轮王寺慈眼堂所藏本与德山毛利氏栖息堂所藏本者是也。两者仅 第5回末叶异版。今以慈眼堂所藏本认为初版,附栖息堂所藏本书影于第一卷末。今以两部补配完整,一概据原刊本而不妄加臆改。至于原本文字不详之处,于卷末附一表。”

此本卷1至5为正文,卷6为插图,插图甚为清晰。 卷6前有〈目录〉,题“清宫珍宝縫美图总目”。



金瓶梅词话100回

(明)笑笑生

(美国国会图书馆影前北平图书馆善本RollNo.615616)

11X24;图200幅

此本为明万历间刻本,有序,序末作“欣欣子书于明一贤里之轩”。有〈金瓶梅序〉,于万历丁巳(1617)季冬东吴弄珠客漫书于金阊道中;有〈新刻金瓶梅词话目录〉,书末有附图。

藏中央图书馆缩微胶卷复制部(前南洋大学图书馆图书登记簿):1919-1920



%# -*- coding: utf-8 -*-
%!TEX encoding = UTF-8 Unicode
%!TEX TS-program = xelatex
% vim:ts=4:sw=4
%
% 以上设定默认使用 XeLaTex 编译,并指定 Unicode 编码,供 TeXShop 自动识别

\chapter*{《金瓶梅》补删}

以下是大陆发行的《金瓶梅》的删除部分,可以用来比照所获版本是否是``纯净版''。

第二回

  更有一件,紧揪揪,红皱皱,白鲜鲜,黑裀裀,正不知是什么东西。


第四回

  但见:交颈鸳鸯戏水,并头鸾凤穿花。喜孜孜,连理枝生;美甘甘,同心带结。一个将朱唇紧贴,一个将粉脸斜偎。罗裙高挑,肩膀上露两湾新月,金钗斜坠,枕头边堆一朵乌云。誓海盟山,搏弄的千般旖妮;羞云怯雨,操搓的万种妖娆。恰恰莺声不离耳畔,津津甜唾,笑吐舌尖,扬柳腰,脉脉春浓,樱桃口,微微气喘。星眼朦胧,细细汗流香玉颗;酥胸荡漾,涓涓露滴牡丹心。直饶匹配眷姻谐,真个偷情滋味美。


  少顷,吃的酒浓,不觉烘动春心。西门庆色心就起,露出腰间那话,引妇人纤手扪弄。原来西门庆自幼常在三街四巷养婆娘,根下犹来着银打就药煮成的托子,那话约有寸许长大,红赤赤黑胡,直竖竖坚硬。好个东西,有诗单遂其能为证:

  一物从来六寸长,有时柔来有时刚。软如醉汉东西倒,硬似风僧上下狂。

  出牝入阴为本事,腰州脐下作家乡。天生二子随身便,曾与佳人斗几场。

  少顷,妇人脱了衣裳,西门庆摸见牝户上并无毳毛,犹如白馥馥,鼓蓬蓬,软浓浓,红皱皱,紧揪揪,千人爱万人贪,更不知是何物。有诗为证:

  温紧香乾口赛莲,能柔能软最堪怜。喜便吐舌开口笑,困时随力就身眠。

  内裆县里为家业,薄草崖边是故园。若遇风流清子弟,等闲战斗不开言。


第六回

  那妇人枕边风月,比娼妓犹胜,百般奉承。西门庆亦施逞枪法打动。两个女貌郎才,俱在妙龄之际,有诗单道其能,诗曰:

  寂静闺房单枕凉,才子佳人至妙顽。才去倒浇红蜡烛,忽然又掉夜行船。

  偷香粉蝶飧花蕊,戏水蜻蜓上下旋。乐极情浓无限趣,灵龟口内吐清泉。


第八回

  听够良久,只听妇人口里懒里呼叫西门庆:“达达,你只顾扇打到几时,只怕和尚听见,饶了奴,快些丢了罢。”西门庆道:“你且休慌,我还要在盖子上烧一下儿哩。”不想被这个秃厮听了个不亦乐乎。


第十回

  西门庆且不与他云雨,明知妇人第一好品萧,于是坐在青纱帐内,令妇人马爬在身边,双手轻笼金钗,捧定那话,往口里吞放。西门庆垂首观其出入之妙。呜咂良久,淫兴倍增。


  说毕,当下西门庆品萧过了,方才抱头交股而寝。正是:自有内事迎郎意,殷勤快把紫萧吹,有西江月为证:

  纱帐轻飘兰麝,娥眉惯把萧吹。雪白玉体透房帷,禁不住意飞魄荡。

  玉腕款笼金钏,两情如醉如痴。才郎情动嘱奴知,慢慢多咂一会。


第十二回

  但见:

  一个不顾纲常贵贱,一个那分上下高低,一个色胆歪邪,管甚丈夫利害;

  一个淫心荡漾,从他律犯明条。一个气喑眼瞪,好似牛吼柳影;

  一个言骄语涩,浑如莺转花间。一个耳畔许雨意云情,一个枕边说山盟海誓。

  百花园内,翻为快活排场;主母房中,变作行乐世界。霎时一滴驴精髓,倾在金莲玉体中。

第十三回

  端的二人怎样交接,但见:

  灯光影里,鲛绡帐内,一来一往,一撞一冲。

  这一个玉臂忙摇,那一个金莲高举。

  这一个莺声呖呖,那一个燕语喃喃,好似君瑞遇莺娘,犹若宋玉偷神女。

  山盟海誓依稀耳中,蝶恋蜂恣未肯即罢。

  战良久,被翻红浪,灵犀一点透酥胸;

  斗多时,帐构银钩,眉黛两弯垂玉脸。

  那正是:

  三次亲唇情越厚,一酥麻体与人偷。这房中二人云雨,不料迎春听了个不亦乐乎。


  说着,一只手把他裤子扯开,只见他那话软叮铛,银托子还带在上面。问道:“你实话,晚夕与那淫妇弄了几遭?”西门庆道:“弄的有数儿的只一遭。:妇人道:“你指着旺跳的身子赌个誓,一遭就弄的他恁软如鼻涕,浓如酱,恰似风瘫了的一般,有些硬朗气儿,也是人心。”说着,把托子一揪,挂下来,骂道:“没羞的,黄猫黑肠的强盗。”嗔道:“教我那里没寻,原来把这行货子悄地带出,和那淫妇捣去了。”


第十六回

  原来李瓶儿好马爬着,教西门庆坐在枕上,他倒插花,往来自动。两个正在美处


  说毕,妇人与西门庆互脱白绫袄,袖子里滑浪一声,掉出个物什儿来。拿在手里沉甸甸的,绍弹子大,认了半日,竟不知是什么东西,但见:

  原是番兵出产,逢人荐转在京。身躯瘦小,内玲珑,得人轻借力,辗转作蝉鸣,解使佳人心胆惧,能助肾威冈。号称金面勇先锋,战阵功第一,扬名勉子铃。

  妇人认了半日,问道:“是什么东西,见怎的把人半边胳膊都麻了。”西门庆笑道:“这物你就不知道了,名唤勉子铃,南方勉甸国出产的,好的也值四,五两银子。”妇人道:“此物便到那里。”西门庆道:“先把他放入炉内,然后行事,妙不可言。”妇人道:“你与李瓶儿也干来。”西门庆于是把晚间之事从头告诉一遍,说的金莲淫心顿起,两个白日里掩上房门,解衣上床交欢。正是:不知子晋缘何事,才学吹萧便作仙。


第十七回

  良久,春色横眉,淫心荡漾。西门庆先和妇人云雨一回,然后乘着酒兴坐于床上,令妇人横躺于衽席之上,与他品萧。但见:

  纱帐香飘兰麝,娥眉轻把萧吹。雪白玉体透香帷,禁不住魂飞魄扬。

  一点樱桃小口,两只手赛柔荑,才郎情动嘱奴知,不觉灵犀味美。

  西门庆于是醉中戏问妇人:“当初有你花子虚在时,也和他干此事不干?”妇人道:“他逐日睡生梦死,奴那里耐烦和他干这营生,他每日只在外面胡撞,就来家奴等闲也不和他沾身。况且老公公在时,和他另在一间房睡着,我还把他骂的狗血喷了头。好不好,对老公公说了,要打白棍儿也不弄人,甚麽材料儿。奴与他这般顽耍,可不寒碜杀奴罢了。谁似冤家这般可奴之意,就是医奴的药一般,白日黑夜,教奴只是想你。”两个顽耍一回,又干了一回。


第十八回

  西门庆因起早送行,着了辛苦,吃了几杯酒就醉了,倒下头鼾声如雷,不醒。那时正值七月二十头天气,夜里有些馀热,这潘金莲怎生睡的着。忽听碧纱帐内一派蚊雷,不免赤着身子起来,执着蜡满帐照蚊,照一个烧一个。回首见西门庆仰卧枕上,睡的正浓,摇之不醒,其腰间那话带着托子,累垂伟长,不觉淫心辄起,放下烛台,用纤手扪弄。弄了一回,蹲下身去用口吮之。吮来吮去,西门庆醒了,骂道:“怪小淫妇儿,你达达睡睡,就掴混死了。”一面起来坐在枕上,亦发教他在下尽力吮咂。又垂首观之,以畅其美。正是:怪底佳人风性重,夜深偷弄紫弯萧。有蚊子双关,踏纱行为证:

  我爱他身体轻盈,楚腰腻细,行行一派笙歌沸。黄昏人未掩朱扉,潜身撞入纱厨内。款傍香肌,轻怜玉体,嘴到处胭脂记,耳边厢造就百般声,夜深不肯教人睡。于是顽了有一顿饭时,

  将蜡移在床背板上,教妇人马爬在他面前,那话隔山取火,插入牝中,令其自动,在上饮酒取其快乐。妇人骂道:


第十九回

  西门庆又要顽弄妇人的胸乳,妇人一面摘下塞领子的金三事儿来,用口咬着,摊开罗衫,露出美玉无瑕,香馥馥的酥胸,紧就就的香乳,揣揣摸摸良久,用口犊之,彼此调笑,曲尽于飞。


第二十一回

  月娘道:“教你上炕,就捞定儿吃。今日只容你在我床上就够了,要思想别的事,却不能够。”那西门庆把那话露将出来,向月娘戏道:“都是你气的他,中风不语了。”月娘道:“怎的中风不语?”西门庆道:“他既不中风不语,如何大睁着眼就说不出话来?”月娘骂道:“好个汗邪的货,教我有半个眼儿看的上你。”西门庆不由分说,把月娘两只白生生腿扛在肩上,那话插入牝中,一任其莺恣蝶采,滞雨尤云,未肯即休。正是:得多少海棠枝上莺梭急,翡翠梁问燕语频。不觉到灵犀一点,美爱无加之处,麝兰半吐,脂香满唇。西门庆情极,低声求月娘叫达达。月娘亦低声帏昵,枕态有余,研口呼亲亲不绝。


第二十三回

  一面用手揪着他那话,


  西门庆脱去衣裳,白绫道袍,坐在床上。把老婆褪了裤,抱在怀里,两只脚翘在两边,那话突入牝中。两个搂抱,正作的好


第二十六回

  原来妇人夏月常不穿裤儿,只单吊着两只裙子,遇见西门庆在那里,便掀开裙子就干。口里常噙着香茶饼儿。于是二人解佩,露媛妃之玉,有几点汉署之香,双凫飞肩


第二十七回

  揭起湘裙,红琨初褪,倒踞着隔山取火,干了半晌,精还不泄,

  听够多时,听见他两个在里面正干得好。

  今日尽着你达受用。”良久,又听的李瓶儿低声叫:“亲达达,你省可的扇罢,奴身上不方便,我前番乞你弄重了些,把奴的小肚子疼起来,这两日才好些儿。”西门庆因问:“你怎的身上不方便?”


  西门庆听言满心欢喜,说道:“我的心肝,你不早说。既然如此,你爹胡乱耍耍罢。”于是乐极情浓,怡然感之,两手抱定其股,一泄如注。妇人在下亏股承受其精。良久,只闻的西门庆气喘嘘嘘,妇人莺莺声软,都被金莲听了个不亦乐乎。


  (西门庆)先将脚指挑弄其花心,挑的阴津流出,如蛙之吐涎。一面又将妇人红绣花鞋儿摘取下来戏,把他两条脚带解下来,拴其数双足,吊在两边葡萄架儿上,如金龙探爪相似,使牝户大张,红沟赤露,舌尖内吐。西门庆先倒覆着身子,执麈柄抵牝口,卖了个倒入翎花,一手掮枕,极力而提之,提的阴中淫气连绵,如数鳅行泥沼中相似。妇人在下没口子叫达达不绝。正干美处,


  春梅见把妇人两腿拴吊在架上,便说:“不知你每什么张致,大青天白日里,一时人来撞见,怪模怪样的。”西门庆问道:“角门子关了不曾?”春梅道:“我来时扣上来了。”西门庆道:“小油嘴,你看我投个肉壶,名唤金弹打银鹅。你瞧,若打中一弹,我吃一盅酒。”于是向水碗中取了枚玉黄李子,向妇人牝中一连打了三个,皆中花心。这西门庆一连吃了三盅药五香酒。又令春梅斟了一盅儿,递与妇人吃,又把一个李子放在牝内,不取出来,又不行事,急的妇人春心没乱,淫水直流,又不好叫出来的,只是朦胧星眼,四枝坦然于枕单之上,口中叫道:“好个作怪的冤家,捉弄奴死了。”莺声颤抖。


  看见妇人还吊在架上,两只白生生腿儿跷在两边,兴不可扼。因见春梅不在眼前,向妇人道:“淫妇,我丢兴你罢。”于是先扣出牝中李子,教妇人吃了。坐在一只枕头上,向纱褶子顺袋内,取出淫器包儿来,先以初使上银托子,次只用硫黄圈来。初时不肯,只在牝口子来回,擂晃不肯深入。急的妇人仰身迎播,口中不住声叫:“达达,快些进去罢,急坏淫妇了。我晓得你恼我,为李瓶儿故意使这促,却来奈何我。今日经着你的手段,再不敢惹你了。”西门庆笑道:“小淫妇,你知道就好说话儿了。”于是,一壁晃着他心子,把那话拽出来,向袋中包里,打开捻了些闺艳声娇,涂在蛙口内,顶入牝中。递了几递,须臾那话昂健奢棱,陲胞暴怒起来。垂首看着往来抽拽,观其出入之势。那妇人在枕畔朦胧星眼,呻吟不已,没口子叫:“大鸡巴达达,你不知使了什幺行子,进去又罢了,淫妇的


  心子痒到骨髓里去了,可怜见饶了罢。”淫妇口里碜死的言语都叫出来。这西门庆一上手,就是三四百回,两只手倒按住枕席,仰身竭力递播掀干,抽没至胫,复进至根者,又约一百余下。妇人以帕在下不住手搽拭牝中之津,随拭随出,衽席为之皆湿。西门庆行货子没棱露脑,往来斗留不已。因向妇人说到:“我要耍个老和尚撞钟。”忽然仰身,往前只一迸,那话攘进去了,直抵牝屋之上。牝屋者,乃妇人牝中深极处,有屋如含苞花蕊。到此处,无折男子茎首,觉翕然,畅美不可言。妇人触疼,急跨其身,只听喀嚓响了一声,把个硫黄圈子折在里面。妇人则目瞑息,微有声嘶,舌尖冰冷,四枝收坦然于衽席之上矣。西门庆慌了,急解其缚,向牝中扣出硫黄圈并勉铃来,折作两截。于是把妇人抚坐,半日星眸惊闪,苏醒过来,因向西门庆作娇泣声说道:“我的达达,你今日怎的这般大恶,险不丧了奴之性命。今后再不可这般所为。不是耍处,我如今头目森森然莫知所矣。”


第二十八回

  纤手不住只向他腰间摸弄那话。那话因惊,银托子还带在上面,软叮铛毛都鲁的,累垂伟长。西门庆戏道:“你还弄他哩,都是你头里唬出他风病来了。”妇人问他怎的风病,西门庆道:“既不是风病,如何这般软瘫热化起不来了,你还不下去央及他央及儿哩。”妇人笑瞅了他一眼,一面蹲下身子去,枕着他一只腿,取过一条裤带儿来,把他那话拴住,用手提着说道:“你这厮头里那等头铮铮,股铮铮,把人奈何昏昏的。这咋你推风症模样死儿。”提弄了一回,放在头脸上偎幌良久,然后将口吮之,又用舌尖舔其蛙口。那话登时暴怒起来,裂瓜头凹眼圆睁,落腮胡挺身直竖。西门庆亦发坐在枕头,令妇人马爬在纱帐内,尽着吮咂,以畅其美。俄而淫思益炙,复与妇人交接。妇人哀告道:“我的达达,你饶了奴罢,又要掇弄奴也。”


  有诗为证:战酣乐极,云雨歇,娇眼也斜,手持玉茎,犹坚硬。告才郎,将就些些,满饮金杯频劝,两情似醉似如痴。

  雪白玉体透廉帷,口赛樱桃手赛荑。一脉泉通声滴滴,两情吻合色迷迷。

  翻来覆去鱼吞藻,慢进轻抽猫咬鸡。灵龟不吐甘泉水,使得嫦娥敢暂离。

第二十九回

  妇人赤身露体,只着红绡抹胸儿,盖着红纱衾,枕石鸳鸯枕在凉席之上,睡思正浓,房里异香喷鼻。西门庆一见,不觉淫心顿起,令春梅带上门出去,悄悄脱了衣裤,上的床来,掀开纱被。见他玉体互相掩映,戏将两股轻开,按麈柄徐徐插入牝中,比及星眸惊欠之际,已抽拽数寸度矣。妇人睁开眼笑道:“怪强盗,三不知多咱进来,奴睡着了就不知道。奴睡的甜甜儿,掴混死了我。”西门庆道:“我便罢了,要是有个生汉子进来,你也推不知道罢。”妇人道:“我不好骂的,谁人七个头八个胆,敢进我这房里,只许了你恁没大没小的罢了。”原来妇人因前日在翡翠轩夸奖李瓶儿身上白净,就暗暗将茉莉花蕊儿搅酥油淀粉,把身子都搽遍了,搽的白腻光滑,异香可掬,使西门庆见了爱他,以夺其宠。西门庆于是见他身体雪白,穿着新作的两只大红睡鞋,一面蹲踞在上,两只手兜其股,极力而提之。垂首观其出入之势。妇人道:“怪货,只顾端详什么,奴身上黑,不似李瓶儿的身上白就是了。他怀着孩子,你便轻怜痛惜,俺每是拾儿,由着这般掇弄。”

  当下添汤换水,洗浴了一回。西门庆趁兴把妇人仰卧在浴板之上,两手执其双足,跨而提之,掀腾扇干,何止二三百回。其声如泥中螃蟹一般,响之不绝。妇人恐怕香云拖坠,一手扶着云鬓,一手折着盆沿,口中燕语莺声,百般难述。怎见这场交战,但见:

  华池荡漾波纹乱,翠帏高卷秋云暗。才郎情动要争持,愁色心忙显手段。一个颤颤巍巍挺硬枪,一个摇摇摆摆弄钢剑。一个舍死忘生生往里,一个尤云滞雨将功干。扑扑通通皮鼓催,哔哔啵啵枪对剑。啪啪嗒嗒弄响声,嘭嘭湃湃成一片。下下高高水逆流,汹汹涌涌盈清涧。滑滑绉绉怎生停,拦拦济济难存站。一来一往,一动一撞东西探,热气腾腾奴云生,纷纷馥馥香气散。一个逆水撑船将玉股摇,一个稍公把舵将金莲揩。一个紫骝猖獗逞威风,一个白面妖娆遭马战。喜喜欢欢美女情,雄雄赳赳男儿愿。翻翻复复尽欢娱,闹闹挨挨情摸乱。你死我活更无休,千战万赢心胆战。口口声声叫杀人。气气昂昂情厌,古古今今广闹争,不似这般水里战。

  当下二人水中战闹了一回,西门庆精泄而止。搽抹身体干净。


第三十四回

  于是淫心辄起,搂在怀里,两个亲嘴咂舌头。那小郎口噙香茶桂花饼,身上熏的喷鼻香。西门庆用手撩起他衣服,褪了花钗儿,摸弄他屁股。

  听见里边气乎乎,吡的地平一片声响。西门庆叫道:“我的儿,把身子吊正着,休要动。”就半日没听见动静。

第三十五回

  西门庆吐舌头,那小郎口里噙着凤香饼儿,递与他,下边又替他弄玉茎。

第三十七回

  妇人便舒手下边笼揪西门庆玉茎。彼此淫心荡漾,把酒停住不吃了。掩上房门,褪去衣裤,妇人就在里边炕床上伸开被褥。那时已是日色平西时分,西门庆乘着酒兴,顺袋内取出银托子来使上。妇人用手扪弄,见奢棱跳脑,紫光鲜沉甸甸,甚是粗大,一壁坐在西门庆怀里,一面在上两个且搂着脖子亲嘴。妇人乃跷起一足,以手导那话入牝中。两个挺一回,西门庆摸见妇人柔腻,牝毛秀,意欲交接,令妇人仰卧于床背,把双枕以手双足置于腰眼间,肆行抽送。怎见的这场云雨,但见:

  威风迷翠塌,杀气锁锦衾。珊瑚枕上施雄,翡翠帐中斗。勇男见忿怒,挺身连刺黑樱枪;女帅生嗔,拍胯着摇追命剑。一来一往,禄山曾合太真妃;一撞一动,君瑞追陪崔氏女。左右迎凑,天河织女遇牛郎;上下盘旋,仙洞妖姿逢元肇。枪来牌架,崔郎相供薛琼琼,炮打刀迎,双渐并连苏小小。一个莺声呖呖,犹如武则天遇敖曹;一个燕喘嘘嘘,好似审在逢吕雉。初战时,知枪乱刺,利剑微迎;次后来,双炮齐发,膀胛齐凑。男儿气急,使枪只去扎心窝;女帅心忙,开口要来吞脑袋。一个使双炮,往来攻打内裆兵;一个抡膀胛,上下夹迎脐下将。一个金鸡独立,高跷玉腿弄精神;一个枯树盘根,倒入翎花来刺牝。战良久,朦胧星眼,但动些麻上来;斗多时,款摆纤腰,再战百愁挨下去,散毛洞主倒上桥。放水去淹军,乌甲将军,虚点枪,侧身逃命走,脐膏落马,须曳蹂踏肉为泥。温柔妆呆,倾刻跌翻深涧底,大披挂,七零八断,犹如急雨打残花;锦套头,力尽斛轮,恰似猛风飘败叶。硫黄元帅,盔盔甲散走无门;银甲将军,守住老营还要命。正是:愁云托上九重天,一块败兵连地滚。

  原来妇人有一件毛病,但凡交媾,只要教汉子干他后庭花,在下边揉着心子绕达,不然,随问怎的,不得丢身子。就是韩道国与他相合,倒是后边去的多,前边一日走不的两三遭儿。第二件,积年好咂鸡巴,把鸡巴常放在口里,一夜他也无个足处。随问怎的出了绒,禁不得他吮舔挑弄,登时就起。这两椿,可在西门庆心坎上。


第三十八回

  妇人早已床炕上铺的厚厚的被褥,被里熏的喷鼻香。西门庆见妇人好风月,一径要打动他,家中袖了一个锦包儿来。打开里面,银托子,香思套,硫黄圈,药煮的白绫带子,悬玉环,封脐膏,勉铃,一弄儿淫器。那妇人仰卧枕上,玉腿高跷,口舌内吐。西门庆先把勉铃教妇人自放牝内,然后将银托子束其根,硫黄圈套其首,脐膏贴于脐上。妇人以手导入牝中,两相迎凑,渐入大半。妇人呼道:“达达,我只怕你墩的腿酸,拿过枕头来你垫着坐,我淫妇自家动罢。”又道:“只怕你不自在,你把淫妇腿吊着,你看好不好?”西门庆真个把他脚带解下一条来,拴他一足,吊在床格子上,低着拽,拽的妇人牝中之津如蛙之吐蜒,绵绵不绝,又拽出好多白浆子来。西门庆问道:“你如何流这些白?”才待要抹之,妇人道:“你休抹,等我吮咂了罢。”于是蹲跪在他面前,吮咂数次,呜呜有声。咂的西门庆淫心辄起,吊过身子,两个干后庭花。龟头上有硫黄圈,濡研难涩。妇人蹙眉,隐忍半晌,仅没其棱。西门庆于是颇作抽已,而妇人用手摸之,渐入大半,把屁股坐在西门庆怀里,回首流眸,作颤声叫:“达达,你慢着些,往后越发粗大,教淫妇怎生挨忍。”西门庆且扶起股,观其出入之势。因叫妇人小名王六儿:“我的儿,你达不知心里怎的,只好这一庄儿。不想今日遇上你,正可我之意,我和你明日生死难开。”妇人道:“达达,只怕后来耍的絮烦了,把奴不理怎了。”西门庆道:“相交下来,才见我不是这样人。”说话之间,两个干够一顿饭时,西门庆令妇人没高低淫声浪语叫着才过。妇人在下,一面用手举股,承受其精,乐极情浓,一泄如注。已而抽出那话来,带着圈子,妇人还替他吮咂净了,两个方才并头交股而卧。正是:一般滋味美,好耍后庭花。有诗为证:

  美冤家,一心好折后庭花,寻常只在门前里走,又被开路先锋把住了。放在户中难禁受。转丝缰,勒回马,亲得胜。弄的我身上麻。蹴损了奴的粉脸,粉脸那丹霞。


第四十二回

  见房里掌着灯烛,原来西门庆和王六儿两个在床沿子上行房。西门庆已有酒的人,把老婆按倒在床沿上,灯下褪去小衣,那话上着托子,干后庭花。一手一阵,往来扇打,何止数百回。扇打的连声响亮,其喘息之声,往来之势,犹赛折床一般,无处不听见。这小孩子正在那里明观,


第五十回

  使一些唾沫也不是人养的,我只一味干粘。


  坐在床沿上,打开淫器包儿,先把银托子来在根上,龟头上使了硫黄圈子,把胡僧与他的粉红膏子药儿,盛在个小银合儿内,捏了一厘半儿来,安放在马眼内,登时药性发作,那话暴怒起来,露棱跳脑,凹眼圆睁,横肋皆见,色若紫肝,约有六七寸长,比寻常分外粗大。西门庆心中暗喜,果然胡僧此药有些意思。妇人脱的光赤条,坐在他怀里,一面用手笼拽,说道:“怪道你要烧酒吃,原来干这个营生。”因问:“你是那里讨来的药?”西门庆急把胡僧与他药从头告诉一遍。先令妇人仰卧床上,背靠双枕,手擎那话往里放。龟头昂大,濡研半晌,方才进入些须。妇人淫津流溢,少顷滑落,已而仅没龟棱。西门庆酒兴颇作,浅抽深送,觉翕翕然,畅美不可言。妇人则淫心如醉,酥瘫于枕上,口内呻吟不止,口口声声只叫:“大鸡巴达达,淫妇今日可死也。”又道:“我央及你好歹留些工夫在后边耍耍。”西门庆于是把老婆倒蹶在床上,那话顶入户中,扶其股而极力扇嘭,扇嘭的连声响亮。老婆道:“达达,你好生扇打着淫妇,休要住了。再不,你自家拿过灯来照着顽耍。”西门庆于是移灯近前,令妇人下直竖双足,他便骑在上面,跷其股,蹲踞而提之。老婆在下,一手揉着花心,扳其股而就之,颤声不已。


  原来王六儿那里因吃了胡僧药,被药性把住了,与王六儿弄了一日,恰好过去没曾去身子,那话越发坚硬,形如铁杵。进房交迎春脱了衣裳,上床就要和李瓶儿睡。李瓶儿只说他不来,和官哥在床上已睡下了。回过头来,见是他,便道:“你在后边睡罢了,又来作什么?孩子才睡下了,睡的甜甜的,我心里不禁烦,又身上来了,不方便,你往别人屋里睡去不是,好来这里缠?”被西门庆搂过脖子来,按着就亲了一个嘴,说道:“怪奴才,你达心里要和你睡睡儿。”因把那话露出来,与李瓶儿瞧,唬的李瓶儿要不的,说道:“耶喽,你怎么弄的他这等大?”西门庆笑着告他说吃了胡僧药一节,你若不和我睡,我就急死了。李瓶儿道:“可怎样的,我身上才来了两日,还没去,亦发等等着儿去了,我和你睡罢。你今日先往五娘屋里歇一夜儿,也是一般。”西门庆到道:“我今日不知怎的,一心只要和你睡。我如今杀个鸡儿,央及你央及儿。再不你教丫头掇些水来洗洗,和我睡睡也罢了。”李瓶儿道:“我到好笑起来,你今日那里吃了酒,吃的恁醉醉儿的来家,恁歪斯缠我。就是洗了,也不干净。一个老婆的月经,沾污在男子汉身上,趱剌剌的也晦气。我到明日死了,你也只寻我?”于是乞逼勒不过,教迎掇了水下来,澡牝干净,方才上床与西门庆交房。可霎作怪,李瓶儿慢慢拍哄的官哥睡下,只刚爬过这头来,那孩子就醒了,一连醒了三次。李瓶儿教迎春拿博浪鼓儿哄着他,抱与奶子那边屋里去了。这里二人,方才自在顽耍。西门庆坐在帐子里,李瓶儿便马爬在他身边。西门庆倒插那话入牝中,已而灯下窥见那话,雪白的屁股儿,用手抱着股,目观其出入,那话已被吞进半截,兴不可遏。李瓶儿怕带出血来,不住取巾帕抹之。西门庆抽拽了一个时辰,两手抱定他屁股,只顾揉搓那话,尽入至根,不容点毛发。脐下毳毛,皆刺其股,觉翕翕然,畅美不可言。李瓶儿道:“达达,慢着些,顶的奴里边好不疼。”西门庆道:“你既害疼,我丢了罢。”于是向桌上取过茶来,呷了一口冷茶,等时精来,一泄如注。正是:四体无非畅美,一团却是阳春。西门庆方知胡僧有如此之妙药,睡下时三更天气,


第五十一回

  金莲吃了,撇了个嘴与春梅,那时春梅就知其意,那边屋里早已替他热下水。妇人抖些檀香白在里面,洗了牝,向灯下摘了头,只撇着一根金簪子,拿过镜子来,从新把嘴唇抹了些胭脂,口中噙着香茶,走过这边来。春梅床头上取过睡鞋来与他换了,带上房门出来。这妇人便将灯台挪近床边,桌上放着,一手放下半边纱帐子来,褪去红琨,露见玉体。西门庆坐在枕头上,那话带着两个托子,一位弄的大大的,露出来与他瞧。妇人灯下看见,唬了一跳,一手拽不过来,紫巍巍,沉甸甸,约有虎二。便昵瞅了西门庆一眼,说道:“我猜你没别的话,定是吃了那和尚药,弄耸的恁般大,一位要奈何老娘。好酒好肉,王里长吃的去,你在谁人跟前试了新,这回剩了些残军败将,才来我这屋里来了。俺每是谁剩鸡巴的,你还说不偏心哩。”嗔道:“那日,我不在屋里,三不知把那行货包子,偷的往他屋里去了。原来晚夕和他干这个营生,他还对着人撇清捣鬼哩。你这行货子干净是个没挽和的三寸货,想起来一白年不理你才好。”西门庆笑道:“小淫妇儿,你过来,你若有本事把他咂过了,我输一两银子与你。”妇人道:“汗邪了你了,你吃了什么行货子,我禁的过他?”于是把身子斜躺在衽席之上,双手执定那话,用朱唇吞裹。说道:“好大行货子,把人的口也撑的生疼的。”说毕,出入呜咂,或舌尖挑弄蛙口,舐其龟弦,或用口噙着,往来哺摔,或在粉脸上偎晃,百般搏弄。那话越发坚硬,造崛起来,裂瓜头凹眼圆睁,落腮胡挺身直竖。西门庆垂首,窥见妇人香肌掩映于纱帐之内,纤手捧定毛都鲁那话往口里吞放,灯下一往一来动旦。不想旁边蹲着一个白狮子猫儿,看见动旦,不知当作什么物件儿,扑向前,用爪儿来掴。这西门庆在上,又将手中拿的洒金老鸦扇儿,只顾引斗他耍子。被妇人夺过扇子来,把猫尽力打了一扇把子,打出帐子外去了。昵向西门庆道:“怪发讪的冤家,紧着这扎扎的不得人意,又引斗他上头上脸的。一时间掴了人脸,却怎样好?不好我就不干这营生了。”西门庆道:“怪小淫妇儿,会张致死了。”妇人道:“你怎的不教李瓶儿替你咂来,我这屋里尽着交你掇弄。不知吃了什么行货子,咂了这一日,亦发咂了没事没事。”西门庆于是向汗巾儿上,小银合儿里,用挑牙签挑了些粉红膏子药儿,抹在马口内,仰卧床上,交妇人骑在身上。妇人道:“等我扇着,你往里放。”龟头昂大,濡研半晌,仅没龟棱。妇人在上,将身左右捱檫,似有不胜隐忍之态。因叫道:“亲达达,里边紧涩住了,好不难捱。一面用手摸之。灯下窥见麈柄已被牝口吞进半截,撑的两边皆满,无复作往来。妇人用唾津涂抹牝户两边,已而稍宽滑落,颇作往来,一举一坐,渐没至根。妇人因向西门庆说:“你每常使的颤声娇在里头,只是一味热痒不可当,怎如和尚这药,使进去,从子宫冷森森直掣到心上。这一回把浑身上下都酥麻了,我晓的今日这命死在你手里了,好难捱忍也。”西门庆笑道:“五儿,我有个笑话儿,说与你听,是应二哥说的:一个人死了,阎王就拿驴皮披在身上,交他变驴。落后判官查簿籍,还有他十三年阳寿,又放回来了。他老婆看见,浑身都变过来了,只有阳物还是驴的,未变过来。那人道:“我往阴间换去!”他老婆慌了,说道:“我的哥哥,你这一去,只怕不放你回来怎了,由他,等我慢慢儿的捱罢。”妇人听了,笑将扇把打了一下子说道:“怪不的应二花子的二老婆,捱惯了驴的行货。碜说嘴的货,我不看世界,这一下打的你。”两个足缠了一个更次,西门庆精还不过。他在下合着眼,由着妇人蹲踞在上,极力抽提,提的龟头刮达刮打,提够良久,又吊过身子去,朝向西门庆。西门庆双足举其股,没棱露脑而提之,往来甚急。西门庆虽身接目视而犹如无物。良久,妇人情极,转过身子来,两手搂定西门庆脖项,合伏在身上,舒舌头在他口里,那话直抵牝中,只顾揉搓,没口子叫:“亲达达,罢了,五儿的死了。”须臾一阵昏迷,舌尖冰冷,泄讫一度。西门庆觉牝中一股热气,正透丹田,心中翕翕然,美快不可言也。已而淫津溢出,妇人用帕抹之。两个相搂相抱,交头叠股,呜咂其舌,那话通不拽出来。睡时没半个时辰,妇人淫情未定,爬上身去,两个又干起来。一连丢了两遭,身子亦觉稍倦。西门庆只是佯佯不采,暗想胡僧药神通。看看窗外鸡鸣,东方见白。妇人道:“我的心肝,你不过却怎样的。到晚夕,你再来,等我好歹替你咂过了罢。”西门庆道:“就咂也不得过,管情就一椿事儿就过了。”妇人道:“告我说是那一椿。”西门庆道:“法不传六,再得我晚夕来对你说。”


第五十二回

  见妇人脱的光赤条身子,坐着床沿,底垂着头,将那白生生腿儿,横抱膝上缠脚。换刚三寸,恰半窄大红平底睡鞋儿。西门庆一见,淫心辄起,麈柄挺然而兴。因问妇人要淫器包儿,妇人连忙问褥子底下摸出来递与他。西门庆把两个托子都带上,一手搂过妇人在怀里。因说:“你达今日要和你干后庭花儿,你肯不肯?”那妇人瞅了一眼,说道:“好个没廉耻冤家,你整日和书童儿小厮干的不值了,又缠起我来了,你和那奴才干去不是。”西门庆笑道:“怪小油嘴儿,罢么,你若依了我,又稀罕小厮作什么。你不知你打心里好的就是这椿儿。管情放到里头去,我就过了。”妇人被他再三缠不过,说道:“奴只怕捱不你这大行货,你把头上圈去一个,我和你耍一遭试试。”西门庆真个除去硫黄圈,根下只来着银托子,令妇人马爬在床上,屁股高蹶,将唾津涂抹在龟头上,往来濡研顶入。龟头昂健,半晌仅没其棱。妇人在下,蹙眉隐忍,口中咬汗子难捱,叫道:“达达,慢着些,这个比不得前头,撑的里头热炙火燎,疼起来。”这西门庆叫道:“好心肝,你叫达达不妨事,到明日买一套好颜色妆花纱衣服与你穿。”妇人道:“那衣服倒也有在。我昨日见李桂姐穿的那五色线掐羊皮金桃的油黛黄银条纱裙子倒好看,说是买一条我穿罢了。”西门庆道:“不打紧,我到明日替你买。”一壁说着,在上颇作抽拽,只顾没棱露脑,浅抽深送不已。妇人回首流眸叫道:“好达达,这里紧着人疼的要不的,如何只顾这般动作起来了。我央及你,好歹快些丢了罢。”这西门庆不听,且扶其股,观其出入之势。一面叫呼道:“潘五儿,小淫妇儿,你好生浪浪叫达达,哄出你达达屁儿来罢。”那妇人真个在下星眼朦胧,莺声款掉,柳腰款摆,香肌半就,口中艳声柔语,百般难述。良久,西门庆觉精来,两手扳其股,极力而扇之。扣股之声,响之不绝。那妇人在下边呻吟成一块,不能禁止。临过之时,西门庆把妇人屁股一扳,麈柄直没至根,抵于极深处,其美不可当。于是怡然感之,一泄如注。妇人承受其精。二体偎贴良久,拽出麈柄,但见惺红染茎,蛙口流涎,妇人以帕抹之,方才就寝。一宿晚景题过。


  原来西门庆走到李瓶儿房里,吃了药出来,把桂姐搂在怀中,坐于腿上,一径露出那话来与他瞧,把桂姐唬了一跳,便问:“怎的就这般大?”西门庆悉把吃胡僧药告诉了一遍,先教他低垂粉颈,款启惺唇,品咂了一回,然后轻轻扶起他刚半叉,恰三寸,好锥靶,赛藕芽,步香尘,舞翠盘,千人爱,万人贪,两只小小金莲来,跨在两边胳膊。穿着大红素缎白绫高底鞋儿,妆花金栏膝裤,腿儿用绿线带扎着,抱到一张椅子上,两个就干起来了,

  听见桂姐颤着声儿将身子只顾迎播着西门庆叫:“达达,快些了事罢,只怕有人来。”被伯爵猛然大叫一声,推开门进来,看见西门庆把桂姐扛着腿子在椅儿上,正干的好。说道:“快取水来,泼泼两个攘心的,搂到一答里了。”李桂姐道:“怪攘刀子的,猛的进来,唬了我一跳。”伯爵道:“快些儿了事,好容易,也得值那些数儿。是的,怕有人来看见,我就来了。且过来等,我抽个头儿着。”西门庆便道:“怪狗材,快出去罢,休鬼混我,只怕小厮来看见。”


  下面裙子内,却似火烧的一条硬铁,隔了衣服,只顾插将进来。那金莲也不由人,把身子一耸,那话儿就隔了衣服,热烘烘的对着了。金莲正忍不过,用手掀经济裙子,用力捏着阳物。经济慌不迭的替金莲扯下裤腰来,划的一声,却扯下一个裙裥儿。金莲笑骂道:“蠢贼奴,还不曾偷惯怎的,恁小着胆,就慌不迭,倒把裙裥儿扯吊了。”就自家扯下裤腰,刚露出牝口,一腿搭在栏干上,就把经济阳物塞进牝口。原来金莲鬼混了半晌,已是湿答答的,被经济用力一挺,便扑的进去了。经济道:“我的亲亲,只是立了不尽根,怎么处?”金莲道:“胡乱抽送抽送,且再摆布。”经济刚待抽送,


  金莲正待澡牝,西门庆把手来待摸他,金莲双手掩住,骂道:“短命的,且没要动旦,我有些不耐烦在这里。”西门庆一手抱住,一手插入腰下,竟摸着道:“怪行货子,怎的夜夜干扑扑的,今晚里面有些湿答答的。莫不是想着汉子,骚水发哩?”原来金莲想着经济,还不曾澡牝,被西门庆打着心事,一时脸红了,把言语支吾,半笑半骂,就澡牝洗脸。


  (把草戏他)的牝口。

第五十九回

  西门庆又舒手向他身上摸他香乳儿,紧紧赛麻团滑腻,一面推开衫儿观看,白馥馥犹如莹玉一般,揣摩良久,淫心辄起,腰间那话儿突然而兴,解开裤带,令他纤手笼楷,粉头见其伟是粗大,唬的吐舌害怕,双手搂西门庆脖儿,说道:“我的亲亲,你今日初会将就我,只放半截儿罢,若都放进去,我就死了,你敢吃药养的这等大,不然,如何天生恁怪刺刺的儿,红赤赤,紫绛绛,好寒碜人子。”西门庆笑道:“我的儿,你下去替我品品。”爱月儿道:“慌怎的,往后日子多如树叶儿,今日初会,人生面不熟,再来等我替你品。”说毕


  西门庆见粉头脱了衣裳,肌肤纤细,牝净无毛,犹如白面蒸饼一般,柔嫩可爱,抱了抱腰肢,未盈一掬,诚为软玉温香,千金难买,于是把他两只白生生银条般嫩腿儿夹在两边腰眼间,那话上使了托子,向花心里顶入,龟头昂大,濡搅半晌,方才没棱,那郑月儿把眉头皱在一处儿,两手攀阁在枕上,隐忍难挨,朦胧着星眼,低声说道:“今日你饶了郑月儿罢。”西门庆于是扛起他两只金莲于肩膀上,肆行抽送,不胜欢娱。正是:得多少春点碧桃红绽蓝,风欺杨柳绿翻腰,有诗为证:

  带雨尤烟匝树奇,妖娆身势似难支。水推西子无双色,春点河阳第一枝。

  浓艳正宜吟君子,功夫何用写王维。含情故把芳心束,留住东风不放归。


第六十一回

  见那边房中,亮腾腾点着灯烛,不想西门庆和老婆在房里,两个正干得好,伶伶俐俐,看见把老婆两只腿却是用脚带吊在床头上,西门庆上身只着一件绫袄儿,下身赤露,就在床沿上,两个一来一往,一动一静,扇打的连声响亮,老婆口里,百般言语,都叫将出来,淫声艳语,通作成一块儿。良久,只听老婆说:“我的亲达,你要烧淫妇,随你心里拣着那块,只顾烧淫妇,不敢拦你,左右淫妇的身子属了你,顾的那些儿了。”西门庆道:“只怕你家里的嗔是的。”老婆道:“那忘八七个头八个胆,他敢嗔,他靠着那里过日子哩。”
  方才了事,烧了五六儿心口里并盖子上尾停骨儿上,共三处香。


  见西门庆脱了衣裳坐在床沿上,妇人探出手来把裤子扯开,摸见那话儿,软叮当的,托子还带在上面,说道:“可又来,你蜡鸭子煮到锅里,身子烂了嘴头儿还硬。见放着不语先生在这里强道,和那淫妇怎么弄耸到这咱晚才来家,弄的恁软如鼻涕浓瓜酱的,嘴头儿还强哩。你赌几个誓,我叫春梅舀一瓶子凉水,你只吃了,我就算你好胆子,论起来,盐也是这般咸,醋也是这般酸,秃子包纲巾饶这一抿子儿也罢了,若是信着你意儿把天下老婆都耍遍了罢。贼没羞的货,一个大眼里火行货子,你早是个汉子,若是个老婆,就养遍街遍巷,属皮匠的,逢着的就上。”
  仰卧在枕上,令妇人:“我儿,你下去替你达品品,起来,是你造化。”那妇人一经做乔张智,便道:“好干净儿,你在那淫妇窟窿子里钻了来,教我替你咂,可不爱杀了我。”西门庆道:“怪小淫妇儿,单管胡说白道的,那里有此勾当。”妇人道:“那里有此勾当,你指着肉身子赌个誓么。”乱了一回,教西门庆下去使水,西门庆不肯下去,妇人旋向袖子里掏出通花汗巾来,将那话儿抹展了一回,方才用朱唇裹没,呜咂半晌。登时咂弄的那话奢棱跳脑,暴怒起来,乃骑在妇人身上,纵麈柄自后插入牝中,两手兜蹲踞而摆之,肆行扇打,连声响亮,灯光之下,窥观其出入之势。妇人倒伏在枕畔,举股迎凑者。久之,西门庆兴犹不惬,将妇人仰卧朝上,那话上使了粉红药儿顶入去,执其双足,又举腰没棱露脑掀腾者将二三百度。妇人禁受不的,瞑目颤声,没口子叫:“达达,你这遭儿只当将就我,不使上他也罢了。”西门庆口中呼叫道:“小淫妇儿,你怕我不怕?再敢无礼不敢?”妇人道:“我的达达,罢么,你将就我些儿,我再不敢了。达达,慢慢提,看提撒了我的头发。”


第六十七回

  西门庆乘酒兴服了药,那话上使了托子,老婆仰卧炕上,架起腿来极力鼓捣,没高低扇嘭,扇嘭的老婆舌尖水冷,淫水溢下,口中呼达达不绝。夜静时分,其声远聆数室。西门庆见老婆身上如绵瓜子相似,用一双胳膊搂着他,令他蹲下身子在被窝内咂鸡巴,老婆无不曲体承受


  西门庆于是淫心辄起,搂他在床上坐,他便仰靠梳背,露出那话来,叫妇人品萧。妇人真个低垂粉头,吞吐裹没,往来呜咂有声。西门庆见他头上戴金赤虎,分心香云上围着翠梅花钿儿,后鬓上珠翘错落,兴不可遏


第六十八回

  床上铺的被褥约一尺高,爱月道:“爹脱衣裳不脱?”西门庆道:“咱连衣耍耍罢,只怕他们前边等咱。”一面扯过夏枕来,粉头解去下衣仰卧枕畔里面,穿着红潞细底衣,褪下一只膝裤腿来。这西门庆把他两只小小金莲扛在肩头上,解开蓝绫裤子,那话使上托子,但见:花心款折,柳腰款摆,正是花嫩不禁揉,春风卒未休,花心犹未足,脉脉情无那,低低唤粉头,春宵乐未央。那当下两个至精欲泄之际,西门庆干的气喘吁吁,粉头娇声不绝,鬓云拖枕,满口只叫:“亲达达,慢着些儿。”良久,乐极情浓,一泄如注


第六十九回

  原来西门庆知妇人好风月,家中带了淫器包儿在身边,又服了胡僧药。妇人摸见他阳物甚大,西门庆亦摸其牝户,彼此欢欣,情兴如火。妇人在床旁伺候鲛绡软帕,西门庆被低预备麈柄狰狞,当下展猿臂,不觉蝶浪蜂狂,跷玉腿,那个羞怯雨。

  正是:

  纵横惯使风流阵,那管床头坠玉钗。

  有诗为证:

  兰房几曲深悄悄,香胜宝鸭睛烟袅。梦回夜月淡溶溶,展转牙床春色少。

  无心今遇少年郎,但知敲打须富商。滞情欲共娇无力,须教宋王赴高唐。

  打开重门无锁钥,露浸一枝红芍药。

  这西门庆当下竭平身本事,将妇人尽力盘桓了一场,缠至更半天气,方才精泄。妇人则发乱钗横,花柳困,莺声燕喘,依稀耳中。比及个并头交股,搂抱片时


第七十一回

  西门庆因其夜里梦遗之事,晚夕令王经拿铺盖来书房地上睡,半夜叫上床,脱的精赤条,搂在被窝内,两个口吐丁香,舌融甜唾,正是不能得与莺莺会,且把红娘去解馋。一晚题过


第七十二回

  口吐丁香蚪含珠。妇人云雨之际,百媚俱生,西门庆扣拽之后,灵犀已透,睡不着,枕上把离言深讲,交接后淫情未足,定从下品莺箫。这妇人说的无非只是要拴西门庆之心,又况抛离了半月,在家久旷幽怀,淫情似火,得到身恨不得钻入他腹中,那话把来品弄了一夜,再不离口。西门庆下床溺尿,妇人还不放,说道:“我的亲亲,你有多少尿溺在奴口里,替你咽了罢,省的冷呵呵的热身子,你下去冻着,倒值了多的。”西门庆听了越发欢喜无已,叫道:“乖乖儿,谁似你这般疼我。于是真个溺在妇人口内,妇人用口接着,慢慢一口多咽了。西门庆问道:“好吃不好吃?”金莲道:“略有些咸味儿。你有香茶与我些压压。”西门庆道:“香茶在我白绫袄内,你自家拿。”这妇人向床头拉过他袖子来,掏了几个放在口内。缠罢,侍臣不及相如渴,特赐金茎露一杯。


  精赤条搂在怀中,犹如软玉温香一般,两个酥胸相贴,玉股交匝,脸儿厮,呜咂其舌。妇人一把扣了瓜子子穰儿,用碟儿盛着安在枕头边,将口儿噙着,舌支密哺送入口中。不一时,甜唾融心,灵犀春透,妇人不住手下捏弄他那话,打开淫器包儿,把银托子。西门庆


  于是令他吊过身子去,隔山拘火,那话自后插入牝中,把手在被窝内搂抱其股,竭力扇嘭的连声响亮,一面令妇人呼叫大东大西


  妇人淫情未足,便不住只往西门庆手里捏弄那话,登时把麈柄捏弄起来,叫道:“亲达达,我一心要你身上睡睡。”一面扒在西门庆身上倒浇烛,搂着他脖子,只顾揉搓。教西门庆两手扳住他腰,扳的紧紧的,他便在上极力抽送,一面扒伏在他身上揉一回那话,渐没至根,余者被托子所阻不能入,便道:“我的达达,等我白日里替你作一条白绫带子,你把和尚与你的那末子药装些在里面,我再坠上两根长带儿,等睡时你扎在根子上,却拿这两根带扎拴后边腰里,拴的儿的人疼,又不得尽美。”西门庆道:“我的儿,你做下,药在上磁盒儿内,你自家装上就是了。”妇人道:“你黑夜好歹来,咱晚夕拿与他试试看,好不好。”于是,两个玩耍一番。


第七十三回

  用纤手向减妆磁盒儿内倾了些颤声娇药末儿装在里面,周围又进房来用倒口针儿撩缝儿,甚是细法,预备晚夕要与西门庆云雨之欢


  睡下不多时,向他腰间摸那话,弄了一回白不起,原来西门庆与春梅才行房不久,那话绵软,急切捏弄不起来。这妇人酒在腹中欲情如火,蹲身在被底,把那话用口吮咂挑弄蛙口,吞裹龟头,只顾往来不绝,西门庆猛然醒了,见他在被窝里,便道:“怪小淫妇儿,如何这咱才来。”妇人道:“俺每在后边吃酒,孟三儿又安排了两大方盒酒菜儿,郁大姐唱着,俺每陪大妗子,杨姑娘猜枚掷骰儿,又顽了这一日,被我把李娇儿先赢醉了,落后孟三儿和我两个五子三猜,俺两个倒输了好几钟酒,你倒是便宜,睡起一觉来好熬我,你看我依你不依。”西门庆道:“你整治那带子了?”妇人道:“在褥子底下不是。”一面探手取出来与西门庆看了,扎在麈柄根下,紧在腰间,拴的紧紧的。又问:“你吃了不曾?”西门庆道:“我吃了。”须曳那话乞妇人一壁厢弄起来,只见奢棱跳脑,挺身直舒,比寻常更舒七寸有余,妇人扒在身上,龟头昂大,两手扇着牝往里放,须臾突入牝中。妇人两手搂定西门庆脖项,令西门庆亦扳抱其腰,在上只顾揉搓,那话渐没至根,妇人叫西门庆:“达达,你取我的柱腰子垫在你腰底下。”这西门庆便向床头取过他大红绫抹胸儿,四折叠起垫着腰,这妇人在他身上马伏着,那消几揉,那话尽入。妇人道:“达达,你把手摸摸,全放进去了,撑的里头满满的,你自在不自在?多揉进去。”西门庆用手摸摸,见尽没至根,间不容发,止剩二卵在外,心中翕翕然畅美不可言。妇人道:“好急的慌,只是触冷咱不得拿灯儿照着干,赶不上夏天好,这冬月间只是冷的慌。”西门庆说道:“这带子比那银托子好不好?强如格的阴门生疼的。这个显的该多大,又长出许多来,你不信摸摸我小肚子七八顶到奴心。”又道:“你搂着我,等我今日一发在你身上睡一觉。”西门庆道:“我的儿,你睡达达。”搂着那妇人,把舌头放在他口里含着,一面朦胧星眼,款抱香肩,睡不多时,怎禁那欲火烧身,芳心撩乱,于是两手按着他肩膀,一举一坐,抽则至首,复送至根,叫:“亲心肝,罢了六儿的心了。”往来抽送又三百回,比及精泄,妇人口中只叫:“我的亲达达,把腰抱紧了。”一面把奶头教西门庆咂,不觉一阵昏迷,淫水溢下,停不多会,妇人两个搂抱在一处,妇人心头小鹿实实的跳,登时四肢困软,香云撩乱,于是拽出来,犹刚劲如故。妇人用帕搽之,便道:“我的达达,你不过却怎么的。”西门庆道:“等睡起一觉来再耍罢。”妇人道:“我也挨不的,身子已软瘫热化的。”当下云收雨散。


第七十四回

  妇人见他那话还直竖一条棍相似,便道:“达达,你将就饶了我吧,我来不得了,待我替你咂咂吧。”西门庆道:“怪小淫妇儿,你不若咂咂的过了,是你的造化。”这妇人真个蹲向他腰间,按着他一双腿,用口替他吮弄那话,吮够一个时分,精还不过。这西门庆用手按着粉头,只顾没棱露脑,摇挪那话,在口里吞吐不绝,抽拽的妇人口边白沫横流,残脂在茎,精欲泄之际,


  一面说着把那话放在粉头脸上只顾偎晃良久,又吞在口里挑弄蛙口一回,又用舌尖底其琴弦,搅其龟棱,然后将朱唇裹着只顾动动的,西门庆灵犀灌顶,满腔春意透脑良久精来,连声呼:“小淫妇儿,好生裹紧着我待过也。”言未绝,其精邈了妇人一口,妇人一面一口口接着多咽了。正是:自有内事挪郎意,殷勤爱把紫萧吹。


第七十五回

  “你拿那淫器包儿来与我”

  我放你去便去,不许你拿了这包子去和那歪剌骨弄答的龌龌龊龊的,到明日还要你来和我睡,好干净儿。西门庆道:“你不与我使惯了,却怎掉的缠了半日。”妇人把银托子掠与他说道:“你要拿了这个行货子去。”西门庆道:“与我这个也罢。”一面接的袖了。


  用手捏弄他那话儿上也采着托子,狰狰跳脑,又喜又怕,两个口吐丁香接在一处,西门庆见他仰卧在被窝内,脱的精赤条条,恐怕冻着他,取过他的抹胸儿替它盖着胸膛上,两手执其两足极力抽提,老婆气踹吁吁,被他得面如火热,又道

  拽着他雪白的两足,腿儿穿着一双绿罗扣花鞋儿,只顾没棱露脑,两个扇干抽提,老婆在下无般不叫出来,娇声怯怯,星眼朦胧,良久即令它马伏在下,且拿双足,西门庆披着红绫被骑在他身上,投那话儿牝中,灯光下两手只顾摘打,口中叫::“章四儿,你好去叫着亲达达,休要住了,我丢与你罢。”那妇人在下举股相就,真个口头颤声柔和呼叫不绝,足玩了一个时辰,西门庆方才精泄,良久抽出麈柄来,老婆取帕儿替他搽试。

  老婆又替吮咂,西门庆告他说:“你五姨怎的替我咂半夜,怕我害冷还尿也不叫我下来溺,都替我咽了。”老婆道:“不打紧,等我也替爹吃了就是了。”西门庆真个把泡隔夜尿都溺在老婆口内,当下两个绮妮温存,万千罗唆,捣了一夜。


  说着慢慢扶起这一只腿儿跨在胳膊上,搂抱在怀里,拽着他白生生的小腿儿,穿着大红绫子的绣鞋儿,说道:“我的儿,你达不爱你别的,只爱你这两只白腿儿,就是天下的妇人选遍了也没有你这两只褪儿柔嫩可爱。”妇人道:“好个说嘴的货,难得你那棉花嘴儿可可儿的,就是普天下妇人选遍了没有来,愁好的没有,也要千取万。不说俺每皮肉儿粗糙,你拿左话儿来右着说哩。”西门庆道:“我的心肝,我有句谎就死了。”妇人道:“怪行货子,没要紧,赌什么誓。”这西门庆说着把那话带上银托子,插放入他牝中,妇人道:“我说你,行行就下道儿来了。”便道:“且住,贼小肉儿不知替我拿下了不曾没有。”遂伸手向床褥子底下摸出绢子来,预备着搽抹,因摸见银托子,说道:“纵多咱三不知,就带上这行货子,还不趁早除下来哩。”那西门庆那里肯依,抱定他一只腿在怀里,只顾没棱露脑,浅抽深送,须臾淫水浸出,往来有声,如狗  子一般,妇人一边用绢抹之,随抹随出,口内不住的作柔颤声叫他:“达达,你省可往里去,奴这两日好不腰酸,下边流白浆子出来。”西门庆道:“我到明日问任医官讨服暖药来你吃就好了。”


第七十六回

  放在小的屁股里弄的胀胀的痛起来,我说你还不快拔出来,他又不肯拔,只顾来回动,且叫小的拿起来跑过来。他又来叫小的


第七十七回

  又用纤手掀起西门庆藕合缎子,看见他白绫裤子,西门庆一面解开裤带露出那话来叫他弄,粉头见根下来着银托子,那话狰狞跳脑,紫绛光鲜,西门庆令他品之,这粉头真个低垂粉颈轻启朱唇半吞半吐,或进或出,呜咂有声,品弄了一回灵犀已透,淫心似火,欲求讲欢,粉头便往后边去了。


  拉近枕头来解衣按在炕沿上扛起腿来就耸,那话儿上已束着银托子,刚插入牝中就拽了几拽,妇人下边淫水直流,把一条蓝布裤子都湿了,西门庆拽出那话来,向顺袋内取出包儿颤声娇来蘸了些在龟头上,攘进去方才止住淫津,肆行抽拽,妇人双手扳着西门庆的肩膊,两相迎凑,在下柔声颤语,呻吟不绝,这西门庆趁着酒兴架其两腿在胳膊上,只顾没棱露脑锐进长驱,肆行扇嘭,何止二三百度,须臾弄的妇人云蓬松,舌尖水冷,口不能言,西门庆则气喘吁吁,灵龟畅美,一泄如注,良久拽出那话来,淫水随出,用帕抹之,两个整衣采带,复理残妆。


第七十八回

  仰扇炕上,西门庆褪下裤子,扛起腿来,那话使有银托子就干起来,原来老婆好并着腿干,两只手扇着只教西门庆攘他心子,那浪水热热一阵流出来,把床褥皆湿,西门庆龟头蘸了药,攘进去两手扳着腰,只顾两相揉搓,麈柄尽入至根,不容毫发,妇人瞪目,口中只叫亲爹。

  这西门庆口中喃喃呐呐就叫:“叶五儿,不知道口里,令不,那老婆原来妮子出身,与贲四私通,被拐出来占为妻子,五短身材,两个胎眼儿,今年也是属兔的,三十二岁了,甚么事儿不知道,口里如流水连叫亲爹不绝,情浓,一泄如注,西门庆扯出麈柄要抹,妇人拦住:“休抹,等淫妇下出替你吮净了罢。”西门庆满心欢喜,妇人真个蹲下身子,双手捧定那话吮咂的干干净净才紧上裤子。


  原来西门庆家中磨枪备剑,带了淫器包儿来要安心要鏖战这婆娘,早把胡僧药用酒吃在腹中,那话上使着双托子,在被窝中架起妇人两股纵麈柄入牝中,举腰展力。那一阵掀腾鼓捣,其声犹若数尺竹泥沼中相似,连声响亮,妇人在下没口叫达达如流水。

  正是:

  照海旌幢秋色裹,击天皮鼓月明中。

  有长诗一篇道这场交战:但见

  锦屏前迷鼋阵摆,绣帏下摄魂旗开,迷鼋阵上闪出一员酒金刚,色魔王,头戴囱红盔,锦兜鍪,身穿乌油甲,绛红袍,缠筋条鱼皮带,没缝靴,使一柄黑缨枪,带的是虎眼鞭皮簿头流星槌没羽箭,跨一匹掩毛凹眼浑红马,打一面发雨翻云大帅旗。摄鼋旗下拥出一个粉骷髅,花狐狸,头戴双凤翘珠络索,身穿索罗衫,翠裙腰,白练裆,凌波蔑,鲛绡带,凤头鞋,使一条隔天边,活絮刀,不得见,泪偷垂,容瘦减,粉面阎罗帏傍骑一匹百媚千娇玉面,打一柄倒凤款莺遮日伞,须臾这阵上扑咚咚鼓振春雷,那阵上闹挨挨麝兰媛,这阵上胶溶溶被翻红浪,那阵上刷剌剌帐控银钩,被翻红浪精神健,帐控银钩情意垂,这一阵急展展二十四解任徘徊,那一阵忽剌剌一十八滚难挣扎,一个是使惯的红锦套索,鸳鸯扣,一个是好耍的拐子流星鸡心槌,一个火忿忿桶子枪,恨不的扎勾三千下,一个颤巍巍囱膀胛,巴不得榻勾五千回,这一各善贯甲披袍战,那一个能奋精吸髓垂,一个战马叭嗒嗒踏番歌舞地,一个征人软浓浓塞满密林崖,一个丑搜刚硬形骸,一个俊娇娆杏脸桃腮,一个施展他久战熬场法,一个卖弄他莺声燕语谐,一个闭良久汉浸浸钗横鬓乱,一个战多时喘吁吁枕软歪。倾刻间只觉这内裆县乞炮打成堆,个个皆肿眉嚷眼,霎时下则望那莎草场被枪打倒底,人人肉绽皮开。

  正是:

  愁云拖上九重天,一派败兵沿地滚,几番鏖战贪淫妇,不是今番这一遭,就在这婆娘心口与阴户烧了两柱香。


  在炕上斜靠着背,扯开白绫吊的绒裤子,露出那话儿来,带着银托子叫他用口吮咂,一面傍边放着果菜,斟酒自饮,呼道:“章四儿,我的儿,你用心替达达咂,我到明日寻出件好妆花段子比甲儿来,你正月十二穿。”老婆道:“看爹可怜见。”咂弄勾了一顿饭时,西门庆道:“我儿,我心里要在你身上烧柱香儿。”老婆道:“随爹你拈着烧柱香儿。”西门庆令他关上房门打裙子脱了上炕来,仰卧在炕上,底下穿着新作的大红潞袖裤儿,褪下一只裤腿来。西门庆袖内还有烧林太太剩下的三个烧酒浸的香马儿,撇去他抹骨儿,一个坐在他心口内,一个坐在他小肚儿底下,一个安在他盖子上用安息香一齐点着,那话下边便插入牝中,低着头看着拽,只顾没棱露脑往来送进不已,又取过镜台来傍边照着看,须臾那香烧到肉根前,妇人蹙眉齿止忍其疼痛,口里低声柔语哼成一块,没口子叫:“达达,爹爹,罢了我了,奴难忍也”西门庆便叫道;“章四儿淫妇,你是谁的老婆。”妇人道:“我是爹的老婆。”西门庆教于他;“你说是来旺的老婆,今日属于我的亲达达。”那妇人回应道:“淫妇原是来旺的老婆,今日属了我的亲达达了。”西门庆又问:“我会不会。”妇人道:“达达会。”两个淫声艳语,无般言语不说出来,西门庆那话粗大,撑的妇人牝户满满,被往来出入带的花心,红如鹦鹉舌,黑如蝙蝠翅一般,翻覆可爱。西门庆于是把他两股拔 在怀内,四体交匝两相迎凑,那话尽没至根,不容号发,妇人瞪目失声,淫水流下,西门庆情浓乐极,精邈如涌泉,正是:不知已透春消息,但觉形骸骨节熔。有诗为证:
  任君随意焉霞杯,满腔春事浩无涯。一身径藕东君爱,不管床头坠宝钗。
  当日西门庆烧了这老婆身上三处香。


第七十九回

  安放在麈柄根下。


  (西门庆)袖被中取出来,套在龟身下,两根锦带儿,扎在腰间,龟头又带着景东人事,用酒服下胡僧药下去,那妇人用手搏弄,弄的那话登时奢棱跳脑,横筋皆现,色若紫肝,比银托子和白绫带子,又不同。西门庆搂妇人坐在怀里,那话插进牝中,在上面两个一个递一口饮酒,咂舌头,…妇人知道,西门庆好点着灯行房,把灯台移在明间炕边一张着上安放,一面将纸门关上,澡牝干净,换了一双大红潞细白绫平底鞋儿,穿在脚上,脱了裤儿,钻在被窝里与西门庆做一处,相搂相抱,睡了一回。(原来西门庆心中,只想着何千户娘子蓝氏,情欲如火。)那话十分坚硬,先令妇人马伏在下,那话放入后庭花内,极力扇嘭了约二三百度。扇嘭的屁股连声响亮。妇人用手在下操着 心子。口中叫达达如流水,于是心中还不美意。起来披上白绫小袄,坐在一只枕头上,妇人仰卧,寻出两条脚带,把妇人两只脚拴在两边 炕柱儿上,卖了个金龙探瓜,将那话儿放入牝中,少时没棱露脑,浅抽深送,次后半出半入,才进长驱,恐其妇人害冷,亦取红绫短袄盖在他身上,这西门庆趁着其酒兴,把灯光挪近根前,垂首观其出入之势,撤至首,复送至根,又数百回。妇人口中百般柔声颤语,都叫将出来。西门庆又取粉的膏子药,涂在龟头上,攘进去,妇人阴中麻痒不能当,急令深入,两相迎就。这西门庆故作逗留,戏将龟头濡晃其牝口,又挑弄其花心,不肯深入,急的妇人淫津流出,如蜗之吐涎,往来带的牝户翻覆可爱,灯光影里,见他两只腿儿,穿着大红鞋儿,白生生腿儿,跷在两边吊的高高的,一往一来,一动一撞,其兴不可遏。…(两个)说话之间,又干勾两顿饭时,方才精泄,解卸下妇人脚带来,搂在被窝内(并头交股)。


  (妇人钻在被窝内)慢慢用手腰里摸他那话,犹如绵软,在没那硬斗气儿,更不知在谁家来。翻来覆去,(怎禁那欲火烧身,淫心荡漾)不住用手只顾捏弄,蹲下身子,被窝内替他百计品咂,只是不起,急的妇人要不的,…(药力发作起来)妇人用白绫带子拴在根上,那话跃然而起,但见裂瓜头凹眼圆睁,落腮胡挺身直竖,妇人见他只顾睡,于是骑在他身上,又取膏子药安放在马眼内,顶入牝中,只顾揉搓,那话只抵苞花窝里,觉翕翕然,浑身酥麻,畅美不可言,又两手据按,举股一起一坐,那话没棱露脑,约一二百回,初时涩滞,次后淫水浸出,稍沾滑落,西门庆由着他搓弄,只是不理,妇人情不能当,以舌亲于至根,止剩二卵在外,用手摸摸美不可言,淫水随拭随出,比三鼓,凡五换巾帕。妇人一连丢了两次。西门庆只是不泄,龟头越发胀的色若紫肝,横筋皆现,犹如火炙一回,害箍胀的慌,令妇人把根下带子去了,还发胀不已,令妇人用口吮之,这妇人亦伏在他身上,用朱唇吞进其龟头,只顾往来不已,勒够了约一顿饭时,(那管中之精,猛然一股,邈将出来,犹水银之泄筒中相似)忙用口接咽不及,(只顾流将起来)。


第八十回

  二载相逢,一朝配偶。数年姻眷,一旦和谐。一个柳腰款摆,一个玉茎忙舒。耳边诉雨意云情,枕上说山盟海誓,莺恣蝶采,猗妮搏弄百千般,狂云羞雨,娇媚施逞千万态。一个低声不住叫亲亲,一个搂抱未免呼达达,正是:得多少柳色乍翻新样,绿花容不减旧时红。


第八十二回

  妇人搂着经济,经济亦揣挨着妇人,妇人唱:六娘子,入门来,将奴搂抱在怀,奴把锦被儿伸开,俏冤家顽的十分怪, 将奴脚儿抬,脚儿抬,操乱了乌云儿歪。经济亦占回前词一首:

  雨意相投情挂牵,休要闪的人孤眠,山盟海誓说千遍,残情上放着天,放着天,你又青春咱年少。

  一面解退衣裤,就在一张春凳上,双凫飞肩,灵根半入,不胜绸缪。有生药名水仙子为证:

  当归半夏紫红石,可意槟榔招做女婿,浪荡根插入蓖麻内。母丁香左右偎,大麻花一阵昏迷,白水银扑簇簇下,红娘子心内喜,快活两片陈皮。

  (只得依他)卸下湘裙,解开裤带,仰在凳上,尽着小伙儿受用,有这等事,正是:

  明珠两颗皆无价,可奈檀郎尽得钻。


  (说着)小伙儿贴在炕上,把那话儿弄的硬硬的,直直的,直竖的一条棍,隔窗眼里舒过来,妇人一见,笑的要不的骂道:“怪贼牢骨的短命,猛可舒出你老子头来,唬了我一跳,你趁早好好抽进去,我好不好,拿针刺你一下子,叫你忍痛哩。”经济笑道:“你老人家这回儿又不待见他起来,你好歹打发他个好去处,也是你一点阴陟。”妇人骂道:“好个怪牢成九惯的囚根子。”一面向腰里摸出面青铜小镜儿来,放在牌棂上,假做勾脸照镜,一面用朱唇吞进吮咂他那话,吮咂的这小郎君一点灵犀灌顶,满腔春意融心。正是:自有内事迎郎意,殷勤爱把紫萧吹,原来妇人做作如此,若有人看见,只说他照镜勾脸么,不显其事。其淫蛊显然,通无廉耻,正咂在热闹处(忽听的有人走的脚步儿响),这妇人连忙搁下镜子,走过一边,经济便把那话抽回去。


  但见:

  情兴两和谐,楼定香肩温腮,手捻香乳绵似软,实奇哉,掀起脚儿脱绣鞋,玉体着郎怀,舌送丁香口便开,到风颠鸾云雨罢,嘱多才,明朝千万早些来。


  就不误成头。


第八十三回

  吃得酒浓上来,妇人娇眼也斜,乌云半坦,取出西门庆的淫器包儿,里面包着相思套,颤声娇,银托子,勉铃,一弄儿淫器,教经济便在灯光影下。妇人便赤身露体,仰卧在一张醉翁椅上儿,经济亦脱的上下没条丝,也对坐一椅,拿春意二十四解本儿,在灯下照着样儿行事。妇人便叫春梅,你在后面推着你姐夫,只怕他身子乏了。那春梅真个在身后推送,经济那话插入妇人牝中,往来抽送,十分畅美,不可言尽。两个对面坐着椅子,春梅便在后边推送,三个串作一处,但见:一个不顾夫主名分,一个那管上下尊卑,一个气的吁吁,犹如牛吼柳影,一个娇声历历,犹似莺啭花间,一个椅上逞雨意云情,一个耳畔说山盟海誓,一个寡妇房内翻为快活道场,一个丈母根前变作行淫世界,一个把西门庆枕边风月尽付于娇婿,一个将韩寿偷香手段,送与情娘,正是:写成今世不修书,结下生来欢喜带。


第九十回

  彼此都是旷夫寡女,恣心似火,那来旺儿缨抢强壮,尽力般弄了一回,药极精来,一泄如柱。


第九十三回

  (屁股帖着肚子)那经济推睡不理他,他把那话弄得硬硬的,直竖一条棍,抹了些唾津在头上,往他粪门里只一顶,原来经济在冷铺中,被花子飞天鬼候林儿弄过的,眼子大了,那话不觉就进去了。…淫声艳语,抠吮舔品。


  但见:

  一个玉臂忙摇,一个柳腰款摆,双睛愤火,星眼郎当,

  一个汗浃胸膛,发狠要赢三五阵,一个香消粉黛,呻吟叫够数千声。

  战良久,灵龟深入,性偏刚,战够多时,一股清泉往里邈,几翻鏖战烟 妓,不似今番这一遭。


第九十六回

  二人都醉了,这候林儿,晚夕干经济后庭花,足干了一夜。亲哥亲达达,亲汉子亲爷,口里无般不叫出来。


第九十八回

  (搂陈经济在怀)将尖尖玉手,扯下他的裤子来,两个兴如火,按纳不住,爱姐不免解衣仰卧在床上,交媾在一处。


%\chapter*{介绍}
%\addcontentsline{toc}{chapter}{介绍}

%《金瓶梅词话》是我国第一部以家庭日常生活为素材的长篇小说。




%\endmyonkyoh

% Document body: Back matter (mandatory)
\backmatter

% Back matter: Bibliography chapter (mandatory)
%\bibliography{database}

% Back matter: Index chapter (mandatory)
%\printindex

\end{document}
