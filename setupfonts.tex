%# -*- coding: utf-8 -*-
%!TEX encoding = UTF-8 Unicode
%!TEX TS-program = xelatex
% vim:ts=4:sw=4
%
% 以上设定默认使用 XeLaTex 编译,并指定 Unicode 编码,供 TeXShop 自动识别
%

%%%%%%%%%%%%%%%%%%%%%%%%%%%%%%%%%%%%%%%%%%%%%%%%%%%%%%%%%%%%
% setup fonts
\newcommand\myfontdir{fonts/}

\newcommand\mycjkfzqk{fzqkbyss-1.00.ttf}
\newcommand\mycjknoto{SourceHanSans-Medium.otf}
\newcommand\mycjkkhangxi{khangxidict-1.023-full.otf}

\newcommand\mycjkming{mingliu-3.00-20923.ttc}
\newcommand\mycjkmingext{mingliub-7.01.ttc}
\newcommand\mycjknotol{SourceHanSans-Light.otf}


\newcommand\myFntAdobeKaiti{AdobeKaitiStd-Regular.otf}%{Adobe Kaiti Std}
\newcommand\myFntAdobeHeiti{AdobeHeitiStd-Regular.otf}%{Adobe Heiti Std}
\newcommand\myFntAdobeFangsong{AdobeFangsongStd-Regular.otf}%{Adobe Fangsong Std}
\newcommand\myFntAdobeMing{AdobeMingStd-Light.otf}%{Adobe Ming Std}
\newcommand\myFntAdobeSong{AdobeSongStd-Light.otf}%{Adobe Song Std}



\newcommand\mytestfontbbb{Adobe Kaiti Std}

\newcommand\FntNameFzbwksgb{\mytestfontbbb}  % 方正北魏楷书
\newcommand\FntNameFzzdxgb{\mytestfontbbb}
\newcommand\FntNameFzzdx{\mytestfontbbb} % 方正中等线
\newcommand\FntNameFzlth{Adobe Heiti Std}         % 方正兰亭黑扁
\newcommand\FntNameFzydh{\mytestfontbbb}          % 方正韵动中黑
\newcommand\FntNameFzys{Adobe Heiti Std}             % 方正中雅宋
\newcommand\FntNameFzqt{\mytestfontbbb}                  % 方正启体
\newcommand\FntNameFzsxslk{\mytestfontbbb}% 方正苏新诗柳楷简体



%\newcommand\FntNameFzqt{FZQiTi-S14}                  % 方正启体
\renewcommand\FntNameFzqt{Adobe Kaiti Std}                  % 方正启体

% 配置: 使用旧明体作为缺省字体,辅助用新明体补全字形
\newcommand\mytestfont{\mycjkming}%{TypeLand.com 康熙字典體}%{Adobe Heiti Std}
\newcommand\mycjkfallbackfontA{\mycjkmingext}%{TypeLand.com 康熙字典體}%{Adobe Heiti Std}
\newcommand\mycjkfallbackfontB{\mycjkkhangxi}%{TypeLand.com 康熙字典體}%{Adobe Heiti Std}
\newcommand\mycjkfallbackfontE{\myFntAdobeFangsong}%{Noto Sans S Chinese}%{Adobe Heiti Std}

\newcommand\mycjkboldfont{\mytestfont}%{TypeLand.com 康熙字典體}%{Adobe Heiti Std}
\newcommand\mycjkitalicfont{\mytestfont}%{FZKaiT-Extended}%{全字庫正楷體}
\newcommand\mycjkmainfont{\mytestfont}%{Adobe Ming Std}%{花園明朝A}%{TypeLand.com 康熙字典體}%{新細明體}
\newcommand\mycjksansfont{\mytestfont}%{Adobe Ming Std}%{花園明朝A}%{TypeLand.com 康熙字典體}%{新細明體}
\newcommand\mycjkmonofont{\mytestfont}%{WenQuanYi Micro Hei Mono}

%%%%%%%%%%%%%%%%%%%%%%%%%%%%%%%%%%%%%%%%%%%%%%%%%%%%%%%%%%%%
%\cndocFntForTitle{}
%\cndocFntForMain{}




    % 配置: 使用Adobe作为缺省字体,辅助用新明体补全字形,最后用SourceHanSans扫尾。
    \renewcommand\mytestfont{\myFntAdobeFangsong}
    \renewcommand\mycjkfallbackfontA{\mycjkmingext}
    \renewcommand\mycjkfallbackfontB{\mycjkkhangxi}
    \renewcommand\mycjkfallbackfontE{\mycjknotol}

    \renewcommand\mycjkboldfont{\myFntAdobeHeiti}
    \renewcommand\mycjkitalicfont{\myFntAdobeKaiti}
    \renewcommand\mycjkmainfont{\myFntAdobeFangsong}
    \renewcommand\mycjksansfont{\myFntAdobeHeiti}
    \renewcommand\mycjkmonofont{\myFntAdobeFangsong}



    % 配置: 使用Noto Sans S Chinese作为缺省字体,辅助用康熙字典體补全字形,最后用SourceHanSans扫尾。
    \renewcommand\mytestfont{\mycjknotol}%{TypeLand.com 康熙字典體}%{Adobe Heiti Std}
    \renewcommand\mycjkfallbackfontA{\mycjkkhangxi}%{TypeLand.com 康熙字典體}%{Adobe Heiti Std}
    \renewcommand\mycjkfallbackfontB{\mycjkmingext}%{TypeLand.com 康熙字典體}%{Adobe Heiti Std}
    \renewcommand\mycjkfallbackfontE{\mycjknotol}%{Noto Sans S Chinese}%{Adobe Heiti Std}

    \renewcommand\mycjkboldfont{\mytestfont}%{TypeLand.com 康熙字典體}%{Adobe Heiti Std}
    \renewcommand\mycjkitalicfont{\mytestfont}%{FZKaiT-Extended}%{全字庫正楷體}
    \renewcommand\mycjkmainfont{\mytestfont}%{Adobe Ming Std}%{花園明朝A}%{TypeLand.com 康熙字典體}%{新細明體}
    \renewcommand\mycjksansfont{\mytestfont}%{Adobe Ming Std}%{花園明朝A}%{TypeLand.com 康熙字典體}%{新細明體}
    \renewcommand\mycjkmonofont{\mytestfont}%{WenQuanYi Micro Hei Mono}


\ifnum\strcmp{\myclinemode}{\detokenize{vertical}}=0
    % 配置: 使用方正清刻体作为缺省字体,辅助用康熙字典體补全字形,最后用SourceHanSans扫尾。
    \renewcommand\mytestfont{\mycjkfzqk}%{TypeLand.com 康熙字典體}%{Adobe Heiti Std}
    \renewcommand\mycjkfallbackfontA{\mycjkkhangxi}%{TypeLand.com 康熙字典體}%{Adobe Heiti Std}
    %\renewcommand\mytestfont{\mycjkkhangxi}%{TypeLand.com 康熙字典體}%{Adobe Heiti Std}
    %\renewcommand\mycjkfallbackfontA{\mycjkfzqk}%{TypeLand.com 康熙字典體}%{Adobe Heiti Std}
    \renewcommand\mycjkfallbackfontB{\mycjkmingext}%{TypeLand.com 康熙字典體}%{Adobe Heiti Std}
    \renewcommand\mycjkfallbackfontE{\mycjknotol}%{Noto Sans S Chinese}%{Adobe Heiti Std}

    \renewcommand\mycjkboldfont{\mytestfont}%{TypeLand.com 康熙字典體}%{Adobe Heiti Std}
    \renewcommand\mycjkitalicfont{\mytestfont}%{FZKaiT-Extended}%{全字庫正楷體}
    \renewcommand\mycjkmainfont{\mytestfont}%{Adobe Ming Std}%{花園明朝A}%{TypeLand.com 康熙字典體}%{新細明體}
    \renewcommand\mycjksansfont{\mytestfont}%{Adobe Ming Std}%{花園明朝A}%{TypeLand.com 康熙字典體}%{新細明體}
    \renewcommand\mycjkmonofont{\mytestfont}%{WenQuanYi Micro Hei Mono}

\fi



% 用於章節中夾注評論批評
\newCJKfontfamily{\fntfmyJiaozhu}{Adobe Ming Std}%{{[Path=\myfontdir,]{\mycjkkhangxi}}}  % 校注字体

% 用於章節中整段整段的評論批評
\newCJKfontfamily{\fntfmyPinlun}{Adobe Kaiti Std}%{{[Path=\myfontdir,]{\mycjkkhangxi}}}  % 校注字体



\newCJKfontfamily{\fzwkai}   {\FntNameFzbwksgb}  % 方正北魏楷书
\newCJKfontfamily{\fzzhdxian}{\FntNameFzzdxgb} % 方正中等线
\newCJKfontfamily{\fzltheib} {\FntNameFzlth}         % 方正兰亭黑扁
\newCJKfontfamily{\fzydzhhei}{\FntNameFzydh}          % 方正韵动中黑
\newCJKfontfamily{\fzzhysong}{\FntNameFzys}             % 方正中雅宋
\newCJKfontfamily{\fzqiti}   {\FntNameFzqt}                  % 方正启体
\newCJKfontfamily{\fzliukai}[GB18030=\FntNameFzqt]{\FntNameFzsxslk}% 方正苏新诗柳楷简体


\setCJKmainfont[Path=\myfontdir,
    %FallBack=\mycjkfallbackfontE,
    zhbiaodian={[Path=\myfontdir,]{\mycjkkhangxi}},%mycjkfzqk mycjknoto mycjkkhangxi mycjkming mycjknotol myFntAdobeKaiti myFntAdobeHeiti myFntAdobeFangsong myFntAdobeMing myFntAdobeSong
    BoldFont=\mycjkboldfont,
    ItalicFont=\mycjkitalicfont,
    ItalicFeatures={RawFeature={slant=0.17}}, % 定义 \itshape 为斜体
    BoldItalicFeatures={RawFeature={slant=0.17}}, % 定义 \bfseries\itshape 为斜体
    AutoFakeBold,AutoFakeSlant]{\mycjkmainfont}

% FallBack 的值为空,将设置 备用字体
\setCJKmainfont[FallBack,Path=\myfontdir,]{\mycjkfallbackfontE}
\addCJKfontfeatures{Scale=1.5}

\setCJKsansfont[Path=\myfontdir,
    FallBack=\mycjkfallbackfontE,
    BoldFont=\mycjkboldfont,
    ItalicFont=\mycjkitalicfont,
    ItalicFeatures={RawFeature={slant=0.17}}, % 定义 \itshape 为斜体
    BoldItalicFeatures={RawFeature={slant=0.17}}, % 定义 \bfseries\itshape 为斜体
    AutoFakeBold,AutoFakeSlant]{\mycjksansfont}

%\setCJKfallbackfamilyfont{\CJKrmdefault}
%  [BoldFont={FZHei-B01_GB18030},ItalicFont={FZKai-Z03_GB18030}]{FZShuSong-Z01_GB18030}
%\setCJKfallbackfamilyfont{\CJKsfdefault}
%  [BoldFont={FZHei-B01_GB18030},ItalicFont={FZKai-Z03_GB18030}]{FZLanTingHei-R_GB18030_YS}
%\setCJKfallbackfamilyfont{\CJKttdefault}
%  [BoldFont={FZHei-B01_GB18030},ItalicFont={FZKai-Z03_GB18030}]{FZFangSong-Z02_GB18030}

\setCJKfallbackfamilyfont{\CJKrmdefault}[AutoFakeSlant]{
%\setCJKfallbackfamilyfont{rm}[AutoFakeSlant]{
    [Path=\myfontdir]{\mycjkfallbackfontA},
    [Path=\myfontdir]{\mycjkfallbackfontB},
    [Path=\myfontdir]{\mycjkfallbackfontE}
}
\setCJKfallbackfamilyfont{\CJKsfdefault}[AutoFakeSlant]{
    [Path=\myfontdir]{\mycjkfallbackfontA},
    [Path=\myfontdir]{\mycjkfallbackfontB},
    [Path=\myfontdir]{\mycjkfallbackfontE}
}
\setCJKfallbackfamilyfont{\CJKttdefault}[AutoFakeSlant]{
    [Path=\myfontdir]{\mycjkfallbackfontA},
    [Path=\myfontdir]{\mycjkfallbackfontB},
    [Path=\myfontdir]{\mycjkfallbackfontE}
}
