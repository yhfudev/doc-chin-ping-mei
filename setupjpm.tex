%# -*- coding: utf-8 -*-
%!TEX encoding = UTF-8 Unicode
%!TEX TS-program = xelatex
% vim:ts=4:sw=4
%
% 以上设定默认使用 XeLaTex 编译,并指定 Unicode 编码,供 TeXShop 自动识别

\usepackage{setspace}

% two columns footnotes
\usepackage{dblfnote}
\DFNalwaysdouble % for this example

\definecolor{darkred}{rgb}{0.5,0,0}
\definecolor{darkgreen}{rgb}{0,0.5,0}
\definecolor{darkblue}{rgb}{0,0,0.5}
\definecolor{darkpurple}{rgb}{0.5,0,0.5}
%\hypersetup{
    %colorlinks,
    %linkcolor=darkblue,
    %filecolor=darkgreen,
    %urlcolor=darkred,
    %citecolor=darkblue
%}

%%%%%%%%%%%%%%%%%%%%%%%%%%%%%%%%%%%%%%%%%%%%%%%%%%%%%%%%%%%%
% JPM related:

%\DeclareRobustCommand*\KG{}%{\kern\ccwd}
%\Huge
%\huge
%\LARGE
%\Large
%\large
%\normalsize (default)
%\small
%\footnotesize
%\scriptsize
%\tiny


% 题字签名
\newcommand\tiJPM[1]{{\bigskip\mbox{}\fzqiti\large\hfill #1 \quad}}


%\newcommand{\mnote}[1]{\marginpar{\raggedright
%\textsf{\footnotesize{%
%\begin{spacing}{1.025}%
%#1%
%\end{spacing}%
%}}}}

%\newcommand{\mnoteb}[1]{{\marginpar{\raggedright\sffamily\footnotesize
%\setstretch{1.025}%
%#1}}}



\newcommand\myscriptsize{\tiny}


\newcommand\mypinlunsize{\footnotesize}


\newcommand\jiaoZhanginfo{張竹坡皋鶴堂批評第一奇書批評本。齐鲁书社 1987 年版《张竹坡批评第一奇书金瓶梅》} \newcommand\zhuZhanginfo{\jiaoZhanginfo} \newcommand\piZhanginfo{\jiaoZhanginfo}
\newcommand\piZhangF[1]{ \ifnum\myEnablePi > 0  {\fntfmyPinlun\mypinlunsize\color{darkred} #1 (張竹坡皋鶴堂批評第一奇書批評本)} \fi} % 批評 (整段,張竹坡皋鶴堂批評第一奇書批評本批語)
\newcommand\piZhang[1]{ \ifnum\myEnablePi > 0  \myfootnote{{\fntfmyJiaozhu\myscriptsize\linespread{0.9} \color{darkred} 張批: #1}} \fi } % 批評 (張竹坡皋鶴堂批評第一奇書批評本批語)
%\newcommand\piZhang[1]{ \ifnum\myEnablePi > 0  \myfootnote{張批: #1} \fi } % 批評 (張竹坡皋鶴堂批評第一奇書批評本批語)

%\renewcommand\piZhang[1]{\gezhu{#1}}



\newcommand\jiaoLiyuinfo{李漁新刻繡像批評本} \newcommand\zhuLiyuinfo{\jiaoLiyuinfo} \newcommand\piLiyuinfo{\jiaoLiyuinfo}
\newcommand\piLiyuF[1]{ \ifnum\myEnablePi > 0 {\fntfmyPinlun\mypinlunsize\color{darkblue} #1 (李漁新刻繡像批評本)} \fi} % 批評 (李漁新刻繡像批評本批語)
\newcommand\piLiyu[1]{ \ifnum\myEnablePi > 0  \myfootnote{{\fntfmyJiaozhu\myscriptsize\linespread{0.9} \color{darkblue} 繡像批: #1}} \fi } % 批評 (李漁新刻繡像批評本批語)
%\newcommand\piLiyu[1]{ \ifnum\myEnablePi > 0  \myfootnote{繡像批: #1} \fi } % 批評 (李漁新刻繡像批評本批語)


% 文龍(字禹門)光緒年間墨批六萬餘字。這是繼張竹坡之後,對《金瓶梅》一書的重要評論,具有珍貴的文獻價值。文龍批語分回末評、眉批、夾批等部分。
\newcommand\jiaoWenlonginfo{文龍在茲堂本手書批評本} \newcommand\zhuWenlonginfo{\jiaoWenlonginfo} \newcommand\piWenlonginfo{\jiaoWenlonginfo. 北京圖書館藏清在茲堂刊本《金瓶梅》,上有文龍(字禹門)光緒年間墨批六萬餘字。這是繼張竹坡之後,對《金瓶梅》一書的重要評論,具有珍貴的文獻價值。文龍批語分回末評、眉批、夾批等部分。}
\newcommand\piWenlongF[1]{ \ifnum\myEnablePi > 0  {\fntfmyPinlun\mypinlunsize\color{darkpurple} #1 (文龍在茲堂本手書批評本)} \fi} % 批評 (文龍在茲堂本手書批評本批語)
\newcommand\piWenlong[1]{ \ifnum\myEnablePi > 0  \myfootnote{{\fntfmyJiaozhu\myscriptsize\linespread{0.9} \color{darkpurple} 文龍批: #1}} \fi } % 批評 (文龍在茲堂本手書批評本批語)
%\newcommand\piWenlong[1]{ \ifnum\myEnablePi > 0  \myfootnote{文龍批: #1} \fi } % 批評 (文龍在茲堂本手書批評本批語)


\newcommand\jiaoYLinfo{金瓶梅詞話校注,白維國、卜鍵 校注,嶽麓書社} \newcommand\zhuYLinfo{\jiaoYLinfo} \newcommand\piYLinfo{\jiaoYLinfo}
\newcommand\jiaoYL[1]{ \ifnum\myEnableJiao > 0  \myfootnote{{\fntfmyJiaozhu\myscriptsize\linespread{0.9} \color{darkpurple} 麓校: #1}} \fi } % 校記 (金瓶梅詞話校注,白維國、卜鍵 校注,嶽麓書社)
%\newcommand\jiaoYL[1]{ \ifnum\myEnableJiao > 0  \myfootnote{麓校: #1} \fi } % 校記 (金瓶梅詞話校注,白維國、卜鍵 校注,嶽麓書社)
\newcommand\zhuYL[1]{ \ifnum\myEnableZhu > 0  \myfootnote{{\fntfmyJiaozhu\myscriptsize\linespread{0.9} \color{darkpurple} 麓注: #1}} \fi } % 注疏 (金瓶梅詞話校注,白維國、卜鍵 校注,嶽麓書社)
%\newcommand\zhuMINE[1]{ \ifnum\myEnableZhu > 0  \myfootnote{麓注: #1} \fi } % 注疏 (金瓶梅詞話校注,白維國、卜鍵 校注,嶽麓書社)


\newcommand\jiaoMJinfo{梅節} \newcommand\zhuMJinfo{\jiaoMJinfo} \newcommand\piMJinfo{\jiaoMJinfo}
\newcommand\jiaoMJ[1]{ \ifnum\myEnableJiao > 0  \myfootnote{{\fntfmyJiaozhu\myscriptsize\linespread{0.9} \color{darkblue} 梅校: #1}} \fi } % 校記 (梅節)
%\newcommand\jiaoMJ[1]{ \ifnum\myEnableJiao > 0  \myfootnote{梅校: #1} \fi } % 校記 (梅節)


\newcommand\jiaoMINEinfo{鄙人的拙作} \newcommand\zhuMINEinfo{\jiaoMINEinfo} \newcommand\piMINEinfo{\jiaoMINEinfo}
\newcommand\jiaoMINE[1]{ \ifnum\myEnableJiao > 0  \myfootnote{{\fntfmyJiaozhu\myscriptsize\linespread{0.9} \color{darkgreen} 校: #1}} \fi } % 校記 (我的)
%\newcommand\jiaoMINE[1]{ \ifnum\myEnableJiao > 0  \myfootnote{校: #1} \fi } % 校記 (我的)
\newcommand\piMINE[1]{ \ifnum\myEnablePi > 0  \myfootnote{{\fntfmyJiaozhu\myscriptsize\linespread{0.9} \color{darkgreen} 批: #1}} \fi } % 批評 (我的)
%\newcommand\piMINE[1]{ \ifnum\myEnablePi > 0  \myfootnote{批: #1} \fi } % 批評 (我的)
\newcommand\zhuMINE[1]{ \ifnum\myEnableZhu > 0  \myfootnote{{\fntfmyJiaozhu\myscriptsize\linespread{0.9} \color{darkgreen} 注: #1}} \fi } % 注疏 (我的)
%\newcommand\zhuMINE[1]{ \ifnum\myEnableZhu > 0  \myfootnote{注: #1} \fi } % 注疏 (我的)




%%%%%%%%%%%%%%%%%%%%%%%%%%%%%%%%%%%%%%%%%%%%%%%%%%%%%%%%%%%%

\newcommand\jpmHasZhu{0}

\newcommand\jpmChkHasZhu{
\renewcommand\jpmHasZhu{0}
\ifnum\myEnablePi > 0   \renewcommand\jpmHasZhu{1} \fi
\ifnum\myEnableJiao > 0 \renewcommand\jpmHasZhu{1} \fi
\ifnum\myEnableZhu > 0  \renewcommand\jpmHasZhu{1} \fi
}


\newcommand\jpmShowZhuInfo{
\jpmChkHasZhu
\ifnum\jpmHasZhu > 0
{
本精校《金瓶梅》彙集有:

\begin{itemize}
\ifnum\myEnablePi > 0
  \item 各家批評;
    \piZhang{\piZhanginfo}
    \piLiyu{\piLiyuinfo}
    \piWenlong{\piWenlonginfo}
\fi
\ifnum\myEnableJiao > 0
  \item 各種校記;
    \jiaoYL{\jiaoYLinfo}
    \jiaoMJ{\jiaoMJinfo}
    \jiaoMINE{\jiaoMINEinfo}
\fi
\ifnum\myEnableJiao > 0
  \item 各式注疏;
    \zhuYL{\zhuYLinfo}
\fi
\end{itemize}
}
\fi
}

\ifnum\strcmp{\myfnotemode}{\detokenize{gezhu}}=0

\renewcommand\jpmShowZhuInfo{
\jpmChkHasZhu %
%
本精校《金瓶梅》彙集有:
\ifnum\myEnablePi > 0
各家批評,如张竹坡\piZhang{\piZhanginfo} 李漁 \piLiyu{\piLiyuinfo} 文龍 \piWenlong{\piWenlonginfo} 等;
\fi
\ifnum\myEnableJiao > 0
各種校記,如
    白維國\jiaoYL{\jiaoYLinfo}
    梅節\jiaoMJ{\jiaoMJinfo}
    我的\jiaoMINE{\jiaoMINEinfo} 等;
\fi
\ifnum\myEnableJiao > 0
各式注疏,如 白維國\zhuYL{\zhuYLinfo} 等
\fi
}
\fi % gezhu


%%%%%%%%%%%%%%%%%%%%%%%%%%%%%%%%%%%%%%%%%%%%%%%%%%%%%%%%%%%%

\newcommand\tmpfigscale{1.1}%{1.35}

\renewcommand{\myincchapge}[2]{
    \cleardoublepage
    \pagestyle{special}
    \begin{figure}[hbtp] \vspace*{-2.1cm} \centering
        \makebox[\linewidth]{\includegraphics[width=\tmpfigscale\linewidth]{\jpmPrefixFigures.#1#2.1.\jpmSuffixFigures}}
    \end{figure}
    \begin{figure}[hbtp] \vspace*{-2.1cm} \centering
        \makebox[\linewidth]{\includegraphics[width=\tmpfigscale\linewidth]{\jpmPrefixFigures.#1#2.2.\jpmSuffixFigures}}
    \end{figure}
    \pagestyle{main}
    \input{\jpmPrefixChapter.#1#2.tex}
}


\ifnum\strcmp{\myclinemode}{\detokenize{vertical}}=0
    \usepackage{rotating}
    \ifnum\strcmp{\mypageorien}{\detokenize{landscape}}=0
        \renewcommand{\myincchapge}[2]{
            \cleardoublepage
            \pagestyle{special}
            \begin{sidewaysfigure}%\begin{figure}[ht]\centering
                %\includegraphics[height=0.85\textwidth,width=0.52\textheight,angle=180]{\jpmPrefixFigures.#1#2.1.\jpmSuffixFigures}
                %\includegraphics[height=0.85\textwidth,width=0.52\textheight,angle=180]{\jpmPrefixFigures.#1#2.2.\jpmSuffixFigures}
                \includegraphics[height=0.85\textwidth,width=0.52\textheight]{\jpmPrefixFigures.#1#2.2.\jpmSuffixFigures}
                \includegraphics[height=0.85\textwidth,width=0.52\textheight]{\jpmPrefixFigures.#1#2.1.\jpmSuffixFigures}
                \caption{This is the caption.}\label{fig:fig1}
            \end{sidewaysfigure}%\end{figure}
            \pagestyle{main}
            \input{\jpmPrefixChapter.#1#2.tex}
        }

    \else % fix page, fake portrait

        \renewcommand{\myincchapge}[2]{
            \cleardoublepage
            \pagestyle{special}
            %\renewcommand\tmpfigscale{1.4}
            \begin{sidewaysfigure}%\begin{figure}[ht]
                \vspace*{-0.5cm} \hspace*{-1.1cm}  \centering
                \makebox[\linewidth]{\includegraphics[width=\tmpfigscale\linewidth]{\jpmPrefixFigures.#1#2.1.\jpmSuffixFigures}}
            \end{sidewaysfigure}%\end{figure}
            \begin{sidewaysfigure}%\begin{figure}[ht]
                \vspace*{-0.5cm} \hspace*{-1.1cm} \centering
                \makebox[\linewidth]{\includegraphics[width=\tmpfigscale\linewidth]{\jpmPrefixFigures.#1#2.2.\jpmSuffixFigures}}
            \end{sidewaysfigure}%\end{figure}
            \pagestyle{main}
            \input{\jpmPrefixChapter.#1#2.tex}
        }
    \fi
\fi

