%# -*- coding: utf-8 -*-
%!TEX encoding = UTF-8 Unicode
%!TEX TS-program = xelatex
% vim:ts=4:sw=4
%
% 以上设定默认使用 XeLaTex 编译,并指定 Unicode 编码,供 TeXShop 自动识别

%第六十六回 
\chapter{翟管家寄書致賻\KG 黃真人煉度薦亡}


\begin{showcontents}{}




「八面明窗次第開,  佇看環珮下瑤臺,

閨門春色連新柳,  山嶺寒梅帶早崖;

影動梅梢明月上,  風敲竹徑故人來,

佳人留下鴛鴦錦,  都付東君仔細裁。」

話說西門慶那日陪吳大舅、應伯爵等飲酒中間,因問韓道國:「客夥中標船幾時起身?咱好收拾打包。」韓道國道:「昨日有人來會,也只在二十四日開船。」西門慶道:「過了二十念經,打包便了。」伯爵問:「這遭起身,那兩位去?」西門慶道:「三個人都去。明年先打發崔大哥押一船杭州貨來,他與來保還往江下五處,置買些布貨來發賣。家中段貨紬錦,都還有哩。」伯爵道:「哥主張極妙,常言道:要的般般有,纔是買賣。」說畢,已至起更時分。吳大舅起身說:「姐夫,你連日辛苦,俺每酒已勾了。告回,你可歇息歇息。」西門慶不肯,還要留住,令小優兒奉酒唱曲,每人吃三鍾,纔放出門。西門慶賞了小優四人六錢銀子。再三不敢接,說:「宋爺出票,叫小的每來,官身如何敢受老爺重賞?」西門慶道:「雖然官差,此是我賞你,怕怎的!」四人方磕頭領去,不在話下。西門慶便歸後邊歇去了。次日早起往衙門中去。早有玉皇廟吳道官差了一個徒弟,兩名鋪排來,在大廳上鋪設壇場,上安三清四御,中安太乙救苦天尊,兩邊東嶽酆都,下列十王九幽,冥曹幽壤,監壇神虎二大元帥,桓、劉、吳、魯四大天君,太陰神后,七真玉女,倒真懸司,提魂攝魄,一十七員神將。內外壇場,鋪設的齊齊整整;香花燈燭,擺列的燦燦輝輝。爐中都焚百合名香,周圍高懸弔挂,經筵羅列,幕走銷金,法鼓高張架,彩雲鶴旋繞。西門慶來家看見,心中大喜。打發徒弟鋪排齋食吃了,回廟中去了。隨即令溫秀才寫帖兒,請喬大戶、吳大舅、吳二舅、花大舅、沈姨夫、孟二舅、應伯爵、謝希大、常時節、吳舜臣許多親眷,并堂客,明日念經。家中廚役落作治辦齋供,不題。次日五更,道眾皆挨門進城,到於西門慶家,叫開門,進入經壇內,明起燈燭,沐手焚香,打動响樂,諷誦諸經,敷演生神玉章。鋪排大門首挂起長旛,懸弔榜文,兩邊黃紙門對一聯,大書:

「東極垂慈,仙識乘晨而超登紫府;南丹赦罪,淨魄受煉而逕上朱陵。」

榜上寫著:

「大宋國山東平府清河縣某坊居住,奉道追修孝夫信官西門慶,合家孝眷人等,即日皈誠,上干慈造,意者伏為室人李氏之靈,存日陽年二十七歲,先命辛未相正月十五日午時受生,大限於政和七年九月十七日丑時分身故。伏以伉儷情深,嘆鳳鸞之先別;閨門月冷,嗟琴瑟以斷鳴。徒追悼以何堪,憶音容而緬想。光陰易逝,五七俄臨。欲援幽魂,敬陳丹悃。謹以今月二十日伏延官道,爰就孝居,建盟真煉度齋壇,庸頒玉簡;演九轉生神寶範,奏啟琅函。迓獅馭以垂光,金燈破暗;降龍章而滅罪,鐵柱停酸。爰至深宵,度綵橋而鳴玉珮;頻餐沆瀣,登碧落而謁金真。伏願玉陛垂慈,青宮降鑒。廣覃惻隱之仁,大賜提撕之力。亡魂早超逍遙之境,滯爽咸登極樂之天。存歿眷屬,均沐休祥。愿親人等,同登道岸。凡預薦修,悉希元化,故榜。 政和年月日榜。 上清大洞經籙,九天金闕大夫,神霄玉府上筆判雷霆諸司府院事,清微弘道,體玄養素,崇教高士,領太乙官提點皇壇知磬,兼管天下道教事,高功黃元白奉行。」

大廳經壇,懸挂齊題二十字,大書:「青玄救苦,頒符告簡,五七轉經,水火煉度,薦揚齋壇。」即日黃真人穿大紅,坐牙轎,繫金帶,左右圍隨,儀從喧喝,次日高方到。吳道官率眾接至壇所,行畢禮,然後西門慶著素衣絰巾拜見。遞茶畢,洞案傍邊,安設經筵法席,大紅銷金卓幃,粧花椅褥。二道童倚立左右。其人儀偉容貌,戴王冠,韜以烏紗,穿大紅斗牛衣服,靸烏履。登文書之時,西門慶備金段一疋,金字登壇之時,換了九陽雷巾,大紅金雲白鶴法氅,與袖飛鬣,腳下白綾軟襪,朱紅登雲朝舄,朝外建天地亭,張兩把金傘盖。金童揚烟,玉女散花,執幢捧節,監壇神將,三界符使,四直功曹,城隍社令,土地祇迎,無不畢陳。高功香案上列五式天皇,號令召雷皂纛天蓬,玉尺七星寶劍,淨水法盂。先是表白宣畢齊意,齋官沐手上香詞懺,二人飄手爐向外三禮召請。然後高功繫令焚香,蕩穢淨壇,飛符召將,關發一應文書符命,啟奏三天,告盟十地。三獻禮畢,打動音樂,化財行香。西門慶與陳經濟執手爐跟隨,排軍喝路,前後四把銷金傘,三對纓絡挑搭。孝眷列於大門首,孤魂棚建於街上。場飯淨供,委付四名排軍看守。行香回來,安請監齋壇已畢,在捲棚擺齋。那日各親友街鄰夥計,送茶者絡繹不絕。西門慶悉令玳安、王經收記,打發回盒人銀錢。早辰開啟,請三寶證盟,頒告符簡,破嶽召亡。又動音樂往李瓶兒靈前攝召引魂,朝參玉陛,傍設几筵,聞經悟道。高功搭高座,演九天生神經,焚燒太乙東嶽酆都十王,冠帔雲馭。午朝高功冠裳步罡踏斗,拜進朱表,逕達東極青宮,遣差神將,飛下羅酆。原來黃真人年約三旬,儀表非常。粧束起來,午朝拜表,儼然就是個活神仙。端的生成模樣?但見:

「星冠攢玉葉,鶴氅縷金霞。神清似長江皓月,貌古如太華喬松。踏罡朱履步丹霄,步虛琅函浮瑞氣。長髯廣頰,修行到無漏之天;皓齒明眸,佩籙掌五雷之令。三島十洲,存性到洞天福地;出神游高,擡沆瀣靜裡朝元。三更步月鸞聲遠,萬里乘雲鶴背高。就是都仙太史臨凡世,廣惠真人降下方。」

拜了表文,吳道官當壇頒生天寶籙神虎玉劄。行畢午香,回來捲棚內擺齋。黃真人前,大卓面定勝 。吳道官等稍加差小。其餘散眾,俱平頭卓席。黃真人、吳道官皆襯段尺頭。四位披花,四疋絲紬,散眾各布一疋。卓面俱令人擡送廟中,散眾各有手下徒弟收入箱中,不必細說,吃畢午齋,謝了西門慶,都往花園各亭臺洞內遊玩散食去了。一面收下家火,從新擺上下卓齋饌上、來請吳大舅等眾親朋夥計來吃。正吃之間,忽報東京翟爺那裡差人來下書。西門慶即出到廳上,請來人進入。只見是府前承差幹辦,青衣窄袴,萬字頭巾,乾黃靴,全付弓箭,向前施禮。西門慶答還下禮,那人向身邊取出書來,遞上書,內封折賻儀銀十兩。問來人上姓,那人道:「小人姓王名玉,蒙翟爺差遣送此書來。不知老爹這邊有喪事,安老爹有書到京纔知道。」西門慶問道:「你安老爹書幾時到來?」那人說:「安老爹書十月纔到京。因催皇木一年已滿,陞都水司郎中。如今又奉勅條理河道,直到工完回京。」西門慶問了一遍,即令來保廂房中管待齋飯,分付明日來討回書。那人問:「韓老爹在那裡住?宅內稍信在此。小的見了還要赴往東平符下書去。」西門慶即喚出韓道國來見那人。陪吃齋食畢,同往家中去了。西門慶拆看書中之意,於是乘著喜歡,將書拿到捲棚內,教溫秀才看,說:「你照此修一封回書答他,就稍寄十方縐紗汗巾,十方綾汗巾,十副揀金挑牙,十個烏金酒杯,作回奉之禮。他明日就來取回書。」溫秀才接過書來觀看,其書曰:

「寓京都眷生翟謙頓首,書奉即擢大錦堂西門四泉親家大人門下:自京邸執手話別之後,未得從容相叙,心甚歉然。其領教之意,生已與家老爹前悉陳之矣。邇者因安鳳山書到,方知親家有鼓盆之嘆,但不能一弔為悵,奈何!奈何!伏望以禮節哀可也。外具賻儀,少表微忱,希筦納。又仰貴任榮修德政,舉民有五袴之歌,境有三留之譽。今歲考績,必有甄陞。昨日神運都功兩次工上,生已對 老爹說了,安上親家名字。工完題奏,必有恩典,親家必有掌刑之喜。夏大人年終類本,必轉京堂,指揮列銜矣。謹此預報,伏惟高照不宣。」

附云:

「此書可自省覽,不可使聞之於渠。謹密!謹密!」

又云:

「楊老爹前月二十九日卒於獄。」

下書:

「冬上澣具。」

卻說溫秀才看畢,纔待袖,早被應伯爵取過來觀看了一遍,還付與溫秀才收了,說道:「老先生把回書千萬加意做好些,翟公府中人才極多,休要教他笑話。」溫秀才道:「貂不足,狗尾續。學生匪才,焉能在班門中弄大斧?不過乎塞責而已。」西門慶道:「老先生他自有個主意,你這狗才曉的甚麼?」須臾,吃罷午齋,西門慶分付來興兒打發齋饌,送各親眷街鄰家不題。玳安回院中李桂姐、吳銀兒、鄭愛月兒、韓釧兒、洪四兒、齊香兒六家香儀人情禮去,每家還答一疋大布、一兩銀子。後晌就叫李銘、吳惠、鄭奉三個小優兒伺侯。良久,道眾陞壇發擂,上朝拜懺觀燈,解壇送聖,天色漸晚。及比設了醮,就有起更天氣。門外花大舅被西門慶留下,已不去了。喬大戶、沈姨夫、孟二舅告辭兒回家。止有吳大舅、二舅、應伯爵、謝希大、溫秀才、常時節并眾夥計在此,晚夕觀看水火煉度,就在大廳棚內搭高座、扎綵橋、安設水池火沼,放擺斛食。李瓶兒靈位,另有几筵幃幕,供獻齊整。傍邊一首魂旛,一首紅旛,一首黃旛,上書:「制魔保舉受煉南宮」,先是道眾音樂兩邊列坐,持節捧盂劍,四個道童侍立法座兩邊。黃真人頭戴黃金降魔冠,身披絳絹雲霞衣,登高座,口中念念有詞。音樂止,二人執手爐宣偈云:

「太乙慈尊降駕臨,  夜壑幽關次第開,

童子雙雙前引導,  死魂受煉步雲階。」

黃真人薰沐焚香,念曰:

「伏以玄皇闡教,廣開度於冥途;正一垂科,俾煉形而昇舉。恩沾幽爽,澤被飢噓。謹運真香,志誠上請:東極宮中大慈仁者,尋聲赴感太乙救苦天尊,青玄九陽上帝,十方救苦諸大真人,天仙、地仙,三界官屬,五岳十王,水府羅酆聖,眾伏此真,香來臨法。會伏望獅座浮,空龍旂耀。日空青枝,酒頻除熱惱;甘露食味,廣濟孤噓。今則暫供儿告頒符命,九幽滅罪,罷對停毆。切以人處塵凡,日縈俗務。不知有死,惟欲貪生。鮮能種於善根,多隨入於惡趣。昏迷弗省,恣慾貪嗔。將謂自己長存,豈信無常易到。一朝傾逝,萬事皆空。業障纏身,冥司受苦。今奉道伏為亡過室人李氏靈魂,一棄塵緣,久淪長夜。若非薦拔於愆辜,必致難逃於苦報。恭惟天尊,號隆億劫,氣應九陽。秉好生之仁,救尋聲之苦。洒甘露而普滋群類,放瑞光而遍燭昏衢。命三官寬考較之條,詔十殿閣推研之筆。開囚釋禁,宥過解冤。各隨符使,盡出幽關。咸令登火池之沼,悉蕩滌黃華之形。凡得更生,俱歸道岸。」

高功念五廚經,變食神呪,散法食:

「聞天浮九炁,九炁出乎太空之先;地凝九幽,九幽欝於重陰之壘。九炁列正,萬物並受生成,所以為天地之根,各受生於胞胎,賴三光而育養。人之有死壞者,皆所以不能受其形,保其神,貴其炁,固其根,離其本真耳。若得還生,須得濯形於太陰,煉質於太陽,復受九炁凝合,三元結成胞,乃可成形。匪伏太上之金科,玄元之秘旨,豈可開度幽魂,全形復體,駕景朝元,制魔保舉。靈寶煉形真符,謹當宣奏。」

「太微〈廴回〉黃旗,  無英命靈旛,

攝召長夜府,  開度受生魂。」

道眾先將魂旛,安於水池內,焚結靈符,換紅旛。次授火沼內,焚欝儀府,換黃旛。高功念:「天一生水,地二生火;水火交煉,乃成真形。」煉度畢,請神主冠帔,步金橋,朝參玉陛,皈依二寶。朝玉清,眾舉五供養:

「道中尊玉清,王溟滓無光。包梵炁萬象,森羅一忝珠。死魂受煉,受煉超仙界。」

朝上清五供養:

「經中尊上清,主赤明開圖。推運極元綱,流演洞渺溟。死魂受煉,受煉超仙界。」

朝太清五供養:

「師中尊太清,主道包天地。玄元始歷劫,度出迷魂。死魂受煉,受煉超仙界。」

高功曰:「既受三皈,當宣九戒:

第一戒者,敬讓,孝養父母。  第二戒者,克勤,忠於君王。

第三戒者,不殺,慈救眾生。  第四戒者,不淫,正身處物。

第五戒者,不盜,推義損己。  第六戒者,不嗔,兇怒凌人。

第七戒者,不詐,諂賊害善。  第八戒者,不驕,傲忽至真。

第九戒者,不二,奉戒專一。  汝當諦聽,戒之!戒之!」

九戒畢,道眾舉音樂,宣念符命,并十類孤魂,挂金索:

「大慈仁者,救苦青玄帝。獅座浮空,妙化成神力。清淨斛食,示現焦面鬼。注界孤魂,來受甘露味!」

「北戰南征,貫甲披袍士。捨死忘生,報效於國家。砲响一聲,身臥沙場裡。陣忘孤魂,來受甘露味!」

「好兒好女,與人為奴俾。暮打朝喝,衣不遮身體。逐趕出門,纏臥長街內。饑死孤魂,來受甘露味!」

「坐賈行商,僧道雲遊士。動歲經年,在外尋衣食。病疾臨身,旅店無依倚。客死孤魂,來受甘露味!」

「鬬惡爭強,枷鎖囹圄閉。斬絞凌遲,身喪長街裡。律有明條,犯了王法罪。刑死孤魂,來受甘露味!」

「宿世冤仇,今世來相會。暗計陰謀,毒藥攛腸胃。九竅生烟,喪了身和體。藥死孤魂,來受甘露味!」

「乳哺三年,父母恩難極。十月懷胎,坐草臨盆際。性命懸絲,子母歸陰世。產死孤魂,來受甘露味!」

「急難顛危,受忍難〈廴回〉避。私債官錢,逐日來催逼。自刎懸梁,斷了三寸氣。屈死孤魂,來受甘露味!」

「久病淹纏,氣蠱癱癆類。疥癬痍瘡,遍體膿腥氣。菽水無親,醫藥無調治。病死孤魂,來受甘露味!」

「巨浪風濤,洪水滔天至。纜斷舟沉,身喪長江裡。回首家鄉,無人稍書寄。溺死孤魂,來受甘露味!」

「回祿風烟,一時難迴避。猛火無情,燒燬身和體。爛額焦頭,死作煙薰鬼。焚死孤魂,來受甘露味!」

「附木精邪,無主魍魎輩,鱗介飛潛,莫不回生意。太上慈悲,廣垂方便澤。十類孤魂,來受甘露味!」

煉度已畢,黃真人下高座。道眾音樂,送至門外。化財焚燒箱庫回來,齋功圓滿。道眾都換了冠服,鋪排收捲道像。西門慶又早大廳上畫燭齊明,酒筵羅列。三個小優彈唱,眾親友都在堂前。西門慶先與黃真人把盞,左右捧著一疋天青雲鶴金段,一疋色段,十兩白銀,叩首回拜道:「亡室今日已賴我師經功救拔,得遂超生,均感不淺!微禮聊表寸心。」黃真人道:「小道謬參冠裳,濫膺玄教,有何德以達人天?皆賴大人一誠感格,而尊夫人已駕景朝元矣。此禮若受,實為赧顏!」西門慶道:「此禮甚薄,有褻真人,伏乞笑納。」黃真人方令小童收了。西門慶遞了真人酒,又與吳道官把盞,乃一疋金段,伍兩白銀,又是十兩經資。吳道官只受了經資,餘者不肯受,說:「小道自恁效勞,誦經追拔夫人往生仙界,以盡其心。受此經資,尚為不可,又豈當此盛禮乎?」西門慶道:「師父差矣。真人掌壇,其一應文檢法事,皆乃師父費心。此禮當與師父酬勞,何為不可?」吳道官不得已,方領下,再三致謝。西門慶與道眾遞酒已畢,然後吳大舅、應伯爵等上來,與西門慶散福遞酒。吳大舅把盞,伯爵執壺,謝希大捧菜,一齊跪下,伯爵道:「兄為嫂子今日做此好事,請得真人在此,又是吳師父費心,方纔化財,見嫂子頭戴鳳冠,身穿素衣,手執羽扇,騎著白鶴,望空騰雲而去。此賴真人追薦之力,哥的虔心,嫂子的造化,連我好不快活!」于是蒲斟一盃,送與西門慶。西門慶道:「多蒙列位,連日勞神。言謝不盡,何敢當此盛意?」說畢,一飲而盡。伯爵又斟一盞說:「哥吃酒,吃個雙盃,不要吃單盃。」希大慌忙遞一筯菜來吃了。西門慶回敬眾人畢,安席坐下。小優兒彈唱起來,廚役上來割道。當夜在席前,猜拳行令,品竹彈絲,直吃到二更時分,西門慶已帶半酣,眾人方作辭起身而去。西門慶進來,賞小優兒三錢銀子,往後邊去了。正是:

「人生有酒須當醉,  一滴何曾到九泉?」

有詩為證:

「百年方誓日,  一夕竟為雲,

飛鳳金鋼落,  翔鸞寶鏡分;

超生空自喜,  長恨不勝情,

盃物頻頻飲,  愁懷且暫清。」

畢竟不知後項如何,且聽下回分解:




\end{showcontents}


