%# -*- coding: utf-8 -*-
%!TEX encoding = UTF-8 Unicode
%!TEX TS-program = xelatex
% vim:ts=4:sw=4
%
% 以上设定默认使用 XeLaTex 编译,并指定 Unicode 编码,供 TeXShop 自动识别

%第二十二回 
\chapter{西門慶私淫來旺婦\KG 春梅正色罵李銘}

「巧厭多勞拙厭閒,  善嫌懦弱惡嫌頑,

富遭嫉妒貧遭辱,  勤怕貪圖儉怕慳;

觸事不分皆笑拙,  見機而作又疑奸,

思量那件合人意,  為人難做做人難。」

話說次日有吳大妗子、楊姑娘、潘姥姥眾堂客,都來與孟玉樓做生日。月娘在後廳與眾客飲酒,倒也罷了,其中惹出一件事來。那來旺兒,因他媳婦自家癆病死了,月娘新近與他娶了一房媳婦,娘家姓宋,乃是賣棺材宋仁的女兒。當先賣在蔡通判家房裡使喚,後因壞了事出來,嫁與廚役蔣聰為妻小;這蔣聰常在西門慶家做活答應,來旺兒早晚到蔣聰家叫蔣聰去,看見這個老婆,兩個吃酒刮言,就把這個老婆刮上了。一日,不想這蔣聰因和一般廚役分財不均,酒醉廝打,動起刀杖來,把蔣聰戳死在地,那人便越墻逃走了。老婆央來旺兒對西門慶說了,替他拿帖兒,縣裡和縣丞說,差人捉住正犯,問成死罪,抵了蔣聰命。後來,來旺兒哄月娘,只說是小人家媳婦兒,會做針指。月娘使了五兩銀子,兩套衣服,四疋青紅布,并簪環之類,娶與他為妻。月娘因他叫金蓮不好稱呼,遂改名蕙蓮。這個老婆屬馬的,小金蓮兩歲,今年二十四歲了,生的黃白淨面,身子兒不肥不瘦,模樣兒不短不長,比金蓮腳還小些兒。性明敏,善機變,會粧飾,龍江虎浪,就是嘲漢子的班頭,壞家規的領袖。若說他底本事,他也曾:

「斜筒門兒立,人來倒目隨;托腮并咬指,無故整衣裳;坐立隨搖腿,無人曲唱低;開窗推戶牖,停針不語時;未言先欲笑,必定與人私。」

初來時,同眾家人媳婦上竈,還沒甚麼粧飾,猶不作在意裡。後過了一個月有餘,看了玉樓、金蓮眾人打扮,他把䯼髻墊的高高的,梳的虛籠籠的頭髮,把水鬢描的長長的,在上邊遞茶遞水,被西門慶睃在眼裡。一日,設了條計策,教來旺兒押了五百兩銀子,往杭州替蔡太師製造慶賀生辰錦綉蟒衣,并家中穿的四季衣服,往回也有半年期程。約從十一月半頭,搭在旱路車上,起身去了。西門慶安心早晚要調戲他這老婆,不期到此,正值孟玉樓生日,月娘和眾堂客在後廳吃酒。西門慶那日在家,沒往那去,月娘分付玉筲:「房中另放桌兒,打發酒菜湯飯點心你爹吃。」西門慶因打簾內看見,惠蓮身上穿着袖對衿襖,紫絹裙子,在席上斟酒。故意問玉筲:「那個穿紅祆的是誰?」玉筲回道:「是新娶的來旺兒的媳婦子惠蓮。」西門慶道:「這媳婦子,怎的紅祆配着紫裙子,怪模怪樣!到明日對你娘說,另與他一條別的顏色裙子,配着穿。」玉筲道:「這紫裙子,還是問我借的裙子。」說了,就罷了。須臾,過了玉樓生日。一日,月娘往對門喬大戶家吃生日酒去了。約後晌時分,西門慶從外來家,已有酒了;走到儀門首,這惠蓮正往外走,兩個撞了滿懷。西門慶便一手摟過脖子來,就親了個嘴。口中喃喃吶吶說道:「我的兒,你若依了我,頭面衣服,隨你揀着用!」那老婆一聲兒沒言語,推開西門慶手,一直往前走了。西門慶歸到上房,叫玉筲送了一疋藍段子到他屋裡,如此這般對他說:「爹昨日見你酒席上斟酒,穿着紅祆配着紫裙子,怪模怪樣的,不好看;說『這紫裙子還是問我借的。』爹纔開廚櫃拿了這疋段子,使我送與你,教你做裙子穿。」這惠蓮開看,都是一疋翠藍四季團花兼喜相逢段子。說道:「我做出來,娘若見了問怎了?」玉筲道:「爹到明日還對娘說,你放心。爹說來,你若依了這件事,隨你要甚麼,爹與你買。今日趕娘不在家,要和你會會兒,你心下何如?」那老婆聽了,微笑而不言。因問:「爹多咱時分來?我好在屋裡伺候。」玉筲道:「爹說小廝每看着,不好進你這屋裡來的;教你悄悄往山子底下洞兒裡,那裡無人,堪可一會兒。」老婆道:「只怕五娘、六娘知道了,不好意思的。」玉筲道:「三娘和五娘,都在六娘屋裡下棋,你去不妨事。」當下約會已定,玉筲走來回西門慶說話,兩個都往山子底下成事,玉筲在門首與他觀風。都不想金蓮、玉樓,都在李瓶兒房裡下棋,只見小鸞來請玉樓,說:「爹來家了。」三人就散了,玉樓回後邊去了。金蓮走到房中,勻了臉,亦往後邊來。走入儀門,只門小玉立上房門首。金蓮問:「你爹在屋裡?」小玉搖手兒往前指,這金蓮就知其意。走到前邊山子角門首,只見玉筲攔着門。金蓮只猜玉筲和西門慶在此私狎,便頂進去。玉筲慌了:「五娘休進去,爹在裡面有勾當哩!」金蓮罵道:「怪狗肉,我又怕你爹了?」不由分說,進入花園裡來,各處尋了一遍。走到藏春塢山子洞兒裡,只見他兩個人在裡面纔了事。老婆聽見有人來,連忙繫上裙子,往外走。看見金蓮,把臉通紅了。金蓮問道:「賊臭肉!你在這裡做甚麼?」老婆道:「我來叫畫童兒來。」看着,一溜烟走了。金蓮進來,看見西門慶在裡邊繫褲子,罵道:「賊沒廉耻的貨,你和奴淫婦,大白日裡在這裡端的幹的勾當兒?剛纔我打與那淫婦兩個耳子纔好!不想他往外走了。原來你就是畫童兒,他來尋你!你與我實說,和這淫婦偷了幾遭?若不實說。等住回大姐姐來家,看我說不說!我若不把奴才淫婦臉,打的脹豬,也不等。俺每閒的聲喚在這裡來,你也來插上一把子。老娘眼裡都放不過!」西門慶笑道:「怪小淫婦兒!悄悄兒罷,休要嚷的人知道。我實對你說,如此這般,連今日纔一遭。」金蓮道:「一遭?二遭?我不信。你既要這奴才淫婦,兩個瞞神諕鬼,弄剌子兒,我打聽出來,休怪了我都和你每答話!」那西門慶笑的出去了。金蓮到後邊,聽見眾丫頭每說:「爹來家,使玉筲手巾裹着一疋藍段子,往前邊去,不知與誰?」金蓮就知是與來旺兒媳婦子的,對玉樓亦不題起此事。這老婆每日在那邊,或替他造湯飯,或替他做針指鞋腳,或跟着李瓶兒下棋,常賊乖趨附金蓮。被西門慶撞在一處,無人,教他兩個苟合,圖漢子喜歡。惠蓮自從和西門慶私通之後,背地不筭,與他衣服、汗巾、首飾、香茶 之類。只銀子成兩,家帶在身邊,在門首買花翠胭粉,漸漸顯露,打扮的比往日不同。西門慶又對月娘說:「他做的好湯水。」不教他上大竈,只教他和玉筲兩個,在月娘房裡,後邊小竈上,專頓茶水,整理菜蔬,打發月娘房裡吃飯,與月娘做針指,不必細說。看官聽說:凡家主切不可與奴僕并家人之婦,苟且私狎。久後必紊亂上下,竊弄奸欺,敗壞風俗,殆不可制!有詩為證:

「西門貪色失尊卑,  群妾爭妍竟莫疑,

何事月娘欺不在,  暗通僕婦亂併彝。」

一日,臘月初八日,西門慶早起,約下應伯爵,與大街坊尚推官家送殯。教小廝馬也備下兩疋,等伯爵白不見到。一面李銘來了教唱,春梅等四人彈唱。西門慶正在大廳上,圍爐坐的,教春梅、玉筲、蘭香、迎春一般兒四個,都打扮出來,看著李銘指撥,教演他彈唱。女婿陳經濟,在傍陪着說話。正唱三弄梅花,還未了,只見伯爵來,應寶跟着,夾着毡包進門。那春梅等四個,就要往後走,被西門慶喝住,說道:「左右是你應二爹,都來見見罷,躲怎的?」與伯爵兩個相見作揖,纔待坐下,西門慶令四個過來,與應二爹磕頭。那春梅等朝上磕頭下去,慌的伯爵還喏不迭,誇道:「誰似哥好有福!出落的恁四個好姐姐,水葱兒的一般,一個賽一個!都怎生好?你應二爹今日素手,促忙促急,沒曾帶的甚麼在身邊,改日送胭粉錢來罷。」少頃,春梅等四人,見了禮進去。陳經濟向前作揖,一同坐下。西門慶道:「你如何今日這咱纔來?」應伯爵道:「不好告訴你的。大小女病了一向,近日纔教好些;房下記掛着,今日接了他家來,散心住兩日,亂着,旋教應保叫了轎子,買了些東西在家,我纔來了,遲了一步兒。」西門慶道:「教我只顧等你,咱吃了粥好去了。」隨即一面分付小廝,後邊看粥來吃。只是李銘見伯爵,打了半跪。伯爵道:「李自新,一向不見你。」李銘道:「小的有。連日小的在北邊徐公公那裡,答應兩日,來爹宅裡伺侯。」說着,兩個小廝放桌兒,拿粥來吃;就是四個鹹食,十樣小菜兒,四碗頓爛,一碗蹄子,一碗鴿子雛兒 ,一碗春不老蒸乳餅 ,一碗餛飩雞兒 ,銀廂甌兒,粳米 投着,各樣榛松栗子、果仁、梅桂、白糖粥兒。西門慶陪應伯爵、陳經濟吃了,就拿小銀鍾篩金華酒 ,每人吃了三杯。壺裡還剩下上半壺酒,分付小廝畫童兒:「連桌兒擡下去,廂房內與李銘吃。」就穿衣服起身,同應伯爵並馬相行,與尚推官送殯去了。只落下李銘在西廂房,吃畢酒飯。那月娘房裡,玉筲和蘭香眾人,打發西門慶出了門,在廂房內亂廝有成一塊,一回都往對過東廂房,西門大姐房裡,摑混去了;止落下春梅一個,和李銘在這邊教演琵琶。李銘也有酒了,春梅袖口子寬,把手兜住了。李銘把他手拿起,略按重了些。被春梅怪叫起來,罵道:「好賊王八!你怎的捻我的手,調戲我?賊少死的王八!你還不知道我是誰哩?一日好酒好肉,越發養活的那王八靈聖兒出來了,平白捻我手的來了!賊王八,你錯下這個鍬撅了。你問聲兒去,我手裡你來弄鬼!等爹來家等我說了,把你這賊王八,一條棍攆的離門離戶!沒你這王八,學不成唱了?愁本司三院尋不出王八來?撅臭了你這王八了!」被他千王八、萬王八,罵的李銘拿着衣服往外,金命水命,走投無命。正是:

「兩手劈開生死路,  翻身跳出是非門。」

李銘諕的往外走了,春梅氣狠狠,直罵進後邊來。金蓮正和孟玉樓、李瓶兒并宋惠蓮,在房裡下棋,只聽見春梅從外罵將來,金蓮便問道:「賊小肉兒,你罵誰哩?誰惹你來?」氣的春梅道:「情知是誰,叵耐李銘那王八!爹臨去,好意分付小廝,留下一桌菜并粳米粥兒 與他吃。也有玉筲他每,你推我,我打你,頑成一塊,對着王八雌牙露嘴的,狂的有些褶兒也怎的。頑了一回,都往大姐那邊廂房裡去了。王八見無人,儘力向我手上捻了一下。吃的醉醉的,看着我嗤嗤待笑,我饒了他。那王八見我喓喝罵起來,他就即夾着衣裳,往外走了。剛纔打與賊王八兩個耳刮子纔好!賊王八!你也看個人兒行事,我不是那不三不四的邪皮行貨,教你這王八在我手裡弄鬼!我把王八臉打綠了!」金蓮道:「怪小肉兒!學不學沒要緊,把臉兒氣的黃黃的!等爹來家說了,把賊王八攆了去就是了。那裡緊等着供唱撰錢哩也怎的,教王八調戲我這丫頭!我知道賊王八業罐子滿了!」春梅道:「他就倒運着,量二娘的兄弟,那怕他二娘莫不挾仇打我五棍兒,也怎的?」宋惠蓮道:「論起來,你是樂工,在人家教唱,也不該調戲良人家女子!照顧你一個錢,也是養身父母;休說一日三茶六飯兒扶侍着!」金蓮道:「扶侍着,臨了還要錢兒去了。按月兒一個月與他五兩銀子。賊王八也錯上了墳,你問聲家裡這些小廝每,那個敢望着他雌牙笑一笑兒?吊個嘴兒?遇喜歡罵兩句,若不喜歡,拉倒他主子根前,就是打。着緊把他的扛的眼直直的,看不出他來。賊王八,造化低,你惹他生姜,你還沒曾經着他辣手!」因向春梅道:「沒見你!你爹去了,你進來便罷了,平白只顧和他那廂房裡做甚麼?都教那王八調戲你!」春梅道:「都是玉筲和他每,只顧頑笑成一塊,不肯進來。」玉樓道:「他三個如今還在那屋裡?」春梅道:「都往對過大姐房裡去了。」玉樓道:「等我瞧瞧去!」那玉樓起身去了。良久,李瓶兒亦回房,使綉春叫迎春去。至晚,西門慶來家,金蓮一五一十,告訴西門慶。門慶分付來興兒,今後休放進李銘來走動;自此遂斷了路兒,不敢上門。這李銘正是:

「從前作過事,  沒興一齊來。」

有詩為證:

「習教歌妓逞家豪,  每日閑庭弄錦槽,

不意李銘遭譴斥,  春梅聲價競天高。」

畢竟未知後來何如,且聽下回分解:

