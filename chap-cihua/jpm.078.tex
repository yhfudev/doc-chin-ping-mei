%# -*- coding: utf-8 -*-
%!TEX encoding = UTF-8 Unicode
%!TEX TS-program = xelatex
% vim:ts=4:sw=4
%
% 以上设定默认使用 XeLaTex 编译,并指定 Unicode 编码,供 TeXShop 自动识别

%第七十八回 
\chapter{西門慶兩戰林太太\KG 吳月娘翫燈請藍氏}


\begin{showcontents}{}




「黃鐘應律好風催,  陰伏陽生淑歲回,

葵影便移長至日,  梅花先趁大寒開;

八神表日佔和歲,  六管吹葭動細灰,

已有岸傍迎臘柳,  參差又欲領春來。」

話說當日西門慶陪大舅飲酒,到晚回家。到次日,荊都監早辰騎馬來拜謝,說道:「昨日見旨意下來,下官不勝欣喜。足見老翁愛厚費心之至,實為啣結難忘!范大人便老了,張菊軒指望陞轉他一步兒,照舊也罷了!還虧他些。」說畢,茶湯兩換,荊都監起身,因問:「雲大人到幾時請俺每吃酒?」西門慶道:「近節這兩日也是請不成,直到月間罷了。」送至大門,上馬而去。西門慶這里宰了一口鮮豬,兩壜浙江酒,一疋大紅絨金豸員領,一疋黑青粧花紵絲員領,一百菓餡金餅,謝宋御史,就差春鴻拏帖兒,送到察院去。門吏入報進去。宋御史喚至後廳火房內,賞茶吃,等寫了回帖,裝於套內封了,又賞了春鴻三錢銀子。來見西門慶,拆開觀看,上寫看:

「兩次造擾華府,悚愧殊甚!今又辱承厚貺,何以克當?外令親荊子事,已具本矣,想已知悉。連日渴仰丰標,容當面悉。使旋謹謝

(下書)侍生宋喬年拜大錦衣西門先生大人門下。」

宋御史隨即差人送了一百本歷日,四萬布,一口豬來回禮。一日上司行下文書來,吳大舅本衙到任管事。西門慶拜去,就與吳大舅三十兩銀子,四疋京段,交他上下使用,到二十四日稍閑,封了印來家,又備羊酒,花紅軸文,邀請親朋,從衛中上任回來,迎接到家,擺大酒席,與他作賀。又是何千戶東京家眷到了,西門慶寫月娘名字,送茶過去。到二十六日,玉皇廟吳道官十二個道眾,在家與李瓶兒念百日經,十回度人,整做法事,大吹大打,倡道行香。各親朋都來送茶,請吃齋供,至晚方散;俱不言表。至廿七日,西門慶打發各家禮畢,又是應伯爵、謝希大、常時節、傅夥計、甘夥計、韓道國、賁地傳、崔本,每家半口豬,半腔羊,一壜酒,一包米,一兩銀子;院中李桂姐、吳銀兒、鄭愛月兒每人一套杭州絹衣服,三兩銀子。吳月娘又與菴裏薛姑子打齋,令來安兒送香油 米麵銀錢去,不在言表。看看到年除之日,窗梅痕月,簷雪滾風,竹爆千門萬戶。家家帖春勝,處處掛桃符。西門慶燒紙,又到於李瓶兒房靈前。祭奠已畢,置酒於後堂。合家大小月娘等,李嬌兒、孟玉樓、潘金蓮、孫雪娥、西門大姐并女婿陳經濟,都遞了酒,兩旁列坐。先是春梅、迎春、玉筲、蘭香、如意兒五個磕頭,然後小玉、綉春、小鸞兒、元宵兒、中秋兒、秋菊磕頭。其次者來昭妻一丈青惠慶、來保妻惠祥、來興妻惠秀、來爵妻惠元,一般兒四個家人媳婦磕頭。然後纔是王經、春鴻、玳安、平安、來安,棋童兒、琴童兒、畫童兒,來昭兒子鐵棍兒,來保兒子僧寶兒,來興女孩兒年兒來磕頭。西門慶與吳月娘,俱有手帕汗巾銀錢賞賜。到次日,重和元年新正月元旦,西門慶早起,冠冕穿大紅,天地上炷了香,燒了紙,吃了點心,備馬就出去拜巡按賀節去了。月娘與眾婦人,早起來施朱付粉,插花插翠,錦裙綉襖,羅襪方鞋,粧點妖嬈,打扮可喜,都來後邊月娘房內,廝見行禮。那平安兒與該日節級,在門首接拜帖,落後門簿答應往來官長士夫。玳安與王經穿自新衣裳,新靴新帽,在門首踢毽子兒,放炮【火章】,又去磕瓜子兒,袖香桶兒,戴鬧娥兒。眾夥計主管,門下底人,伺候見節者不計其數,都是陳經濟一人在前邊客位管待。後邊大廳擺設錦筵卓席,單管待親朋。花園捲棚,放下毡幃煖簾,鋪陳錦裀綉毯,獸炭火盆,放着十卓,都是銷金卓幃,粧花柳甸,寶粧菓品,瓶插金花,筵開玳瑁,專一留待士大夫官長。約晌午間,西門慶往府縣拜了人回來,剛下馬,招宣府王三官兒衣巾,有四五個人跟隨,就來拜。到廳上拜了西門慶四雙八拜,然後請吳月娘出來見。西門慶請到後邊,與月娘見了,出來前廳留坐。纔拏起酒來吃了一盞,只見何千戶來拜。西門慶就教陳經濟管待陪王三官兒,他便往捲棚內陪何千戶坐去了。王三官吃了一回,告辭起身。陳經濟送出大門,上馬而去。落後又是荊都監、雲指揮、喬大戶,皆絡繹而至。西門慶待了一日人,已酒帶半酣。至晚打發人去了,歸到上房,歇了一夜。到次日早,又出去賀節。直至晚,歸家來。家中韓姨夫、應伯爵、謝希大、常時節、花子油來拜,陳經濟陪侍在廳上坐的。候至已久,西門慶到了,見畢禮,徙新擺上酒菜點心來飲酒。韓姨夫與花子油隔門,先起身去了。只見伯爵、希大、當時節,坐著如定油兒一般,還不去。又撞見吳二舅來了,見了禮,又往後邊拜見月娘,出來一處坐的。直吃到掌燈已後方散。西門慶已吃的酩酊大醉,送出伯爵等到門首,眾人去了。西門慶見玳安在旁站立,捏了一把手。玳安就知意,說道:「他屋裏沒人。」這西門就撞入他房內,老婆早已在對門裏,迎接進去。兩個也無閑話,走到裏間內,老婆脫衣解帶,仰〈扌扉〉炕上。西門慶褪下褲子,扛起腿來,那話使有銀托子,就幹起來。原來老婆好並着腿幹,兩隻手〈扌扉〉着,只教西門慶攘他心子,那浪水熱熱一陣流出來,把床褥皆濕。西門慶龜頭蘸了藥,攘進去,兩手扱着腰,只顧兩相揉搓。塵柄盡入至根,不容毫髮,婦人瞪目,口中只叫親爺。那西門慶問他:「你小名叫甚麼?說與我。」老婆道:「奴娘家姓葉,排行五姐。」這西門慶口中喃喃吶吶,就叫:「葉五兒!不知道口裏令{入日}不{入日}?」那老婆原來奶子出身,與賁四私通,被拐出來,占為妻子,五短身材,兩個鶬鶬胎眼兒,今年也是屬兔的,三十二歲了,甚麼事兒不知道;口裏如流水,連叫親爺不絕,情濃一泄如注。西門慶扯出塵柄要抹,婦人攔住:「休抹,等淫婦下去,替你吮淨了罷!」這西門慶滿心歡喜。婦人真個蹲下身子,雙手捧定那話,吮咂的乾乾淨淨。纔繫上褲子,因問西門慶:「他怎的去恁些時不來?」西門慶道:「我這里也盼他哩,只怕京中夏大人留住他使。」又與了老婆二三兩銀子盤纏。因說:「我待與你一套衣服,恐賁四知道,不好意思,不如與你些銀子兒,你自家治買罷。」開門送出來。玳安又閑在鋪子里,掩門等候的西門慶進來,方纔關上拴,西門慶便往後邊去了。看官聽說:自古上梁不正,則下梁歪,此理之自然也。如人家主子行苟且之事,家中使的奴僕,皆效尤而行。原來賁四這個老婆,不是守本分的,先與玳安有姦,落後又把西門慶勾引上了。這玳安剛打發西門慶進去了,傅夥計又沒在鋪子里上宿,他與平安兒打了兩大壺酒,就在賁四老婆屋裏,吃到有二更時分。平安在鋪子里歇了,他就和老婆在屋裏睡了一宿,有這等的事!正是:

「對人不用穿針線,  那得工夫送巧來?」

有詩為證,

「滿眼風流滿眼迷,  殘花何事濫如泥?

拾琴暫息商陵操,  惹得山禽遶樹啼。」

都說賁四老婆晚夕對玳安說:「只怕隔壁韓嫂兒傳嚷得後邊知道,也似韓夥計娘子,一時被你娘們說上幾句,羞人答答的,怎好相見?」玳安道:「如今家中除了俺大娘和五娘不言語,別的不打緊。俺大娘倒也罷了,只是五娘快出尖兒。你依我,節間買些甚麼兒進去孝順俺大娘;別的不稀罕,他平昔好吃蒸酥。你買一錢銀子菓餡蒸酥,一盒好大壯瓜子送進去。這初九日是俺五娘生日,你再送些禮去,梯已再送一盒瓜子與俺五娘。你到明日進來磕頭,管情就俺住許多口嘴。」賁四老婆真個依著玳安之言,第二日趕西門慶不在家,玳安就替他買了盒子掇進後邊月娘房中。月娘便道:「是那里的?」玳安道:「是賁四嫂送這盒點心瓜子與娘吃。」月娘道:「男子漢又不在家,那討個錢來,又交他費心!」連忙收了,又回出一盒饅頭,一盒菓子與他,說:「多上覆,多謝了。」那日西門慶拜人回家早,有玉皇廟吳道官來拜,在廳上留坐吃酒。剛打發吳道官去了,西門慶脫了衣服,使玳安:「你騎了馬,問聲文嫂兒去。俺爺今日要來拜拜太太,看他怎的說?」玳安道:「爺且不消去。頭里小的撞見文嫂兒騎着驢子,打門首過去了。他明日初四,王三官兒起身往東京與六黃公公磕頭去了。太太說,交爺初六日過去見節,他那里伺候着哩。」西門慶便道:「他真個這等說來?」玳安道:「莫不小的敢說謊?」這西門慶就入後邊去了。剛到上房坐下,忽有來安兒來報:「大舅來了。」只見吳大舅冠冕着,束着金帶,進入後堂,先拜西門慶,說道:「一言難盡!我吳鎧多蒙姐夫抬舉看顧,又破費姐夫了。多謝厚禮!日昨姐夫下降,我又不在家,失迎!空慢姐夫來了。今日敬來與姐夫磕個頭兒,恕我遲慢之罪!」說着,磕下頭去。西門慶慌忙半頭相還下來,說道:「大舅恭喜,自然之道理,至親何必計較!」吳大舅於是拜畢西門慶,月娘出來與他哥磕頭。頭戴翡白縐紗金梁冠兒,海獺臥兔,白綾對衿襖兒,沉香色遍地金比甲,玉色綾寬襴裙。耳邊二珠環兒,金鳳釵梳,胸前帶着金三事〈扌寨〉領兒,裙邊紫遍地金八條穗子的荷包,五色鑰匙線帶兒,紫遍地金扣花白綾高底鞋兒,打扮的鮮鮮兒的,向前花枝招颭,綉帶飄飄,插燭也似磕了四個頭。慌的大舅忙還半禮,說道:「姐姐兩禮兒罷!」說道:「哥哥、嫂嫂不識好歹,常來擾害你兩口兒。你哥老了,看顧看顧罷。」月娘道:「一時不到,望哥耽待些便了。」吳大舅道:「姐姐沒的說,累你兩口兒還少哩!」拜畢,西門慶留吳大舅坐,說道:「這咱晚了,料大舅也不拜人了。寬了衣裳,咱房里坐罷。」不想孟玉樓與潘金蓮兩個都在屋裏,聽見嚷吳大舅進來,連忙走出來與大舅磕頭,都是海獺臥兔兒,白綾襖兒,玉色挑線裙子;一個是綠遍地金比甲兒;一個是紫遍地金比甲兒。頭上戴的都是䯼髻,玉樓帶的是環子,金蓮是青寶石墜子,下邊尖尖趫趫顯露金蓮。與吳大舅磕了頭,逕往各人房裡去了。西門慶讓大舅房內坐的,騎火盆安放卓兒,擺上春盛菓盒,各樣熱碗夏飯,大饅頭、點心、八寶攢湯 ,一齊拿上來。小玉、玉筲都來與大舅磕頭。須臾,吃了湯飯,月娘用小金廂玳瑁鍾兒斟酒,遞與大舅。西門慶主位相陪。吳大舅讓道:「姐姐,你也來坐的。」月娘:「我就來。」又往裡間房內,拿出數樣配酒的菓菜來,都是冬笋、銀魚、黃鼠、〈魚秦〉鮓、海蜇、天花菜 、蘋婆 、螳螂、鮮柑、石榴、風菱、雪梨 之類。飲酒之間,西門慶便問:「大舅的公事都了畢停當了?」吳大舅道:「蒙姐夫抬舉,年即任便到了,上下人事,倒也都周給的七八,還有屯所裡未曾去到到任。明日是個好日期,衛中開了印來家,整理了些盒子,須得抬到屯所里到任,行牌拘將那屯頭來參見,分付分付。前官丁大人壞了事情,已是被巡撫侯爺參劾去了任。如今我接管承行,須得也要振刷在冊花戶,警勵屯頭,務要把這舊管新增,開報明白。到明日秋糧夏稅,纔好下屯衛收。」西門慶道:「通共約有多少屯田?」吳大舅道:「這屯田,不瞞姐夫說,太祖舊例,練兵衛因田養兵,省轉輸之勞,纔立下這屯田。後吃宰相王安石立青苗法,增上這夏稅。那時只是上納屯田秋糧,又不問民地。而今這濟州管內,除了拋荒葦場港隘,通共二萬七千頃頓地。每頃秋稅、夏稅,只徵收一兩八錢,不上五百兩銀子。到年終纔傾濟了,往東平府交納,轉行招商,以備軍糧馬草作用。」西門慶又問:「還有羨餘之利?」吳大舅道:「雖故還有些拋零一戶,不在冊者,鄉民頑滑,若十分進徵緊了,等秤斛斗重,恐聲口致起公論。」西門慶道:「若是有些甫餘兒也罷,難道說全徵?若徵收些出來,斛斗等秤上,也勾咱每上下攪給。」吳大舅道:「不瞞姐夫說,若會管此屯,見一年也有百十兩銀子尋。到年終,人戶們還有些雞鵝豚米面見相送。那個是各人取覓,不在數內的。只是多賴姐夫力量扶持。」西門慶道:「得勾你老人家攪給,也盡我一點之心。」正說着,月娘也走來旁邊陪坐。三人飲酒,到掌燈已後,吳大舅纔起身去了。西門慶那日就在前邊金蓮房中歇了一夜。到次日,早往衙門中開印,陞廳畫卯,發放公事。先是雲離守家發帖兒,初五日請西門慶并合衙官員吃慶官酒。西門慶次日,何千戶娘子藍氏下帖兒,初六日請月娘姊妹相會。且說那日,西門慶同應伯爵、吳大舅三人,起身到雲離守家。原來旁邊又典了人家一所房子,三間客位內擺酒,叫了一起吹打鼓樂迎接,都有卓面,吃至晚夕來家。巴不到次日,月娘往何千戶家吃酒去了。西門慶打選衣帽齊整,袖着賞賜包兒,騎馬戴眼紗,玳安、琴童跟隨,午後時分,逕來王招宣府中拜節,王三官兒不在,留下帖兒。文嫂兒又早在那裡接了帖兒,連忙報與林太太說,出來請老爺後邊坐。轉道大廳,到於後邊,進入儀門。少間,住房掀起明簾子,上面供養着先公王景崇影像,陳說兩卓春臺菓酌,朱紅公座虎皮校椅。腳下氍毹匝地,簾幙垂紅。少頃,林氏穿着大紅通袖襖兒,珠翠盈頭,粉粧膩臉,與西門慶見畢禮數,留坐待茶。分付大官把馬牽於後槽喂養。茶沒罷,讓西門慶寬衣內坐,說道:「小兒從初四日往東京與他叔父六黃太尉磕頭去了,只過了示宵纔來。」這西門慶一面喚玳安脫去上蓋,裡邊穿著白綾襖子,天青飛魚氅衣,粉底皂靴,十分綽耀。婦人房內安放卓席。黃銅四方獸面火盆。生着炭火。朝陽房屋,日色照窗。房中十分明亮,須臾,丫鬟拿酒菜上來。杯盤羅列,肴饌堆盈,酒汎金波,茶烹玉蕊。婦人錦裙綉襖,皓齒明眸。玉手傳盃,秋波送意,猜枚擲骰,笑語烘春。良久,意洽情濃。飲多時,目邪心蕩。看看日落黃昏,又早高燒銀燭。玳安、琴童下邊耳房放卓兒,自有文嫂兒主張酒饌點心管待。三官兒娘子,另在那邊角門內一所屋裡居住,自有丫鬟養娘伏侍,等閑不過這邊來。婦人又倒扣角門,僮僕誰敢擅入。酒酣之際,兩個共入裡間房內,掀開綉帳,關上窗戶。丫鬟輕剔銀釭,佳人忙掩朱戶。男子則解衣就寢,婦人即洗腳上床。枕設寶花,被翻紅浪,原來西門慶家中磨鎗備劍,帶了淫器包兒來,安心要鏖戰這婆娘,早把胡僧藥用酒吃在腹中。那話上使着雙托子,在被窩中,架起婦人兩股,縱塵柄入牝中。舉腰展力,那一陣掀騰鼓搗,其聲猶若數尺竹泥淖中相似,連聲響亮。婦人在下,沒口叫達達如流水。正是:

「照海旌幢秋色裡,  擊天鼙鼓月明中。」

有長詞一篇,道這場交戰。但見:

「錦屏前迷魂陣擺,綉幃下攝魄旗開。迷魂陣上,閃出一員酒金剛,色魔王,頭戴肉紅盔,錦兜鍪,身穿烏油甲,鋒紅袍,纏觔縧,魚皮帶,沒縫靴;使一柄黑纓鎗,帶的是虎眼鞭,皮薄頭流星槌,沒〈毛秋〉箭;跨一疋掩毛凹眼渾紅馬,打一面發雨翻雲大帥旗。攝魂旗下,擁一個粉骷髏,花狐狸,頭戴雙鳳翹,珠絡索,身穿素羅衫,翠裙腰,白練襠,凌波襪,鮫綃帶,鳳頭鞋;使一條隔天邊話絮刀,不得見,淚偷垂,容瘦減,粉面撾,羅幃傍;騎一疋百媚千嬌玉面〈毛秋〉,打一柄倒鳳顛鸞遮日傘。須臾,這陣上撲鼕鼕鼓震春雷,那陣上鬧挨挨麝蘭靉靆;這陣上腹溶溶被翻紅浪,那陣土刷剌剌帳控銀鈎。被翻紅浪精神健,帳控銀鈎情意乖。這一個急展展二十四解任徘徊,那一個忽剌剌一十八滾難掙扎;一個是慣使的紅綿套索鴛鴦扣,一個是好耍的拐子流星雞心搥。一個火忿忿桶子鎗,恨不的扎勾三千下;一個顫巍巍肉膀牌,巴不得塌勾五十回。這一個善貫甲披袍戰,那一個能奪精吸髓華。一個戰馬,叭蹋蹋蹅番歌舞地;一個征人,軟濃濃塞滿密林崖。一個醜搊搜剛硬形骸,一個俊嬌嬈杏臉桃腮。一個施展他久戰熬場法,一個賣弄他鶯聲燕語諧。一個鬬良久,汗浸浸釵橫鬢亂;一個戰多時,喘吁吁枕欹裀歪。頃刻間,只見這內襠縣,乞砲打成堆,個個皆腫眉睡眼;雯時下則望那莎草場,被鎗扎倒底,人人肉綻皮開。」正是:

「愁雲拖上九重天,  一派敗兵沿地滾;

幾番鏖戰貪淫婦,  不是今番這一遭。」

當下西門慶就在這婆娘心口與陰戶,燒了兩炷香,許下明日家中擺酒,使人請他同三官兒娘子去看燈耍子。這婦人一段身心,已是被他拴縛定了。於是滿口應承都去。這西門慶滿心歡喜,起來與他留連痛飲,至二更時分,把馬從後門牽出,作別方回家去。正是:

「不愁明日盡,  自有暗香來。」

有詩為證:

「盡日忠君倚畫樓,  相逢不捨又頻留;

劉郎莫謂桃花老,  浪把輕紅逐水流。」

都說西門慶到家,有平安迎門稟說:「今日有薛公公家差人送請帖兒,請爺早往門外皇庄看春,又是雲二叔家差人送了五個帖兒,請五位娘吃節酒。帖兒都交進去了。」西門慶聽了,沒言語,進入後邊月娘房來,只見孟玉樓、潘金蓮都在房內坐的。月娘從何千戶家赴了席來家,已摘了首飾花翠,止戴着䯼髻,撇着六根金簪子,勒着珠子箍兒。上着藍綾襖,下着軟黃綿紬裙子,坐着說話。西門慶進來,連忙道了萬福。西門慶就在正面椅上坐下,問道:「你今日往那裡,這咱纔來?」西門慶無得說:「我在應二哥家留坐,到這咱晚。」月娘便說起今日何千戶家酒席上事:「原來何千戶娘子還年小哩,今年纔十八歲!生的燈人兒也似一表人物,好標致!知今博古,透靈兒還強十分!見我去,恰似會了幾遍,好不喜狎。嫁了何大人二年光景,房裡倒使着四個丫頭,兩個養娘,兩房家人媳婦。」西門慶道:「他是內府御前生活所藍大監姪女兒,與他陪嫁了好少錢兒!」又道:「小廝對你說來?明日雲夥計家又請俺每吃節酒,送了五個帖兒,在揀粧上閣着。連薛內相家帖子,都放在一處。」因令玉筲:「拏過來與你爹瞧。」這西門慶看了薛內相家帖兒,又看雲離守家帖兒,下書他娘子兒「雲門蘇氏歛袵拜請。」西門慶說:「你每明日收拾了,都去走走。」月娘道:「留雪姐在家罷,只怕大節下,一時有個人客驀將來,他每沒處撾撓。」西門慶道:「也罷,留雪姐在家裡,你每四個去吧。明日我也不往那裡去,薛太監請我門外看春,我也懶待去。這兩日春氣發也怎的,只害這邊腰腿疼。」月娘道:「你腰腿疼,只怕是痰火。問任一官討兩服藥吃不是?只顧挨着怎的?」那西門慶道:「不妨事,由他,一發過了這兩日吃,心淨些。」因和月娘計較,到明日燈節,咱少不得置席酒兒,請請何大人娘子、連周守備娘子、荊南崗娘子、張親家母、喬親家母、雲二哥娘子,連王三官兒母親和大妗子、崔親家母,這幾位都會會,也只在十二三掛起燈來。還叫王皇親家那起小廝扮戲耍一日。爭耐去年還有賁四在家,扎了幾架烟火放。今年他東京去了,只顧不見來了,都交誰人看着扎?那金蓮在旁插口道:「賁四去了,他娘子兒扎也是一般。」這西門慶就瞅了金蓮道:「這個小淫婦兒,三句話就說下道兒去了。」那月娘、玉樓也不採顧,就罷了。因說道:「那三官兒娘,咱每與他沒有大會過,人生面不熟的,怎麼好請他?只怕他也不肯來。」西門慶道:「他既認我做親,咱送個帖兒與他,來不來隨他就是了。」月娘又道:「我明日不往雲家去罷,懷着個臨月身子,只管往人家撞來撞去的,交人家唇齒!」玉樓道:「姐姐,沒的說,怕怎麼的?你身子懷的又不顯,怕還不是這個月的孩子,不妨事。大節下,自恁散心去走走兒纔好。」說畢,西門慶吃了茶,就吃了茶,就往後邊孫雪娥房裏去了。那潘金蓮見他往雪娥房中去,叫了大姐,也就往前邊去了。西門慶到於雪娥房中,晚間交他打腿捏身上,捏了半夜。一宿晚景題過。到次日早辰,只見應伯爵走來借衣服頭面,對西門慶說:「昨日雲二嫂送了個帖兒,今日請房下陪眾嫂子坐。家中舊時有幾件衣服兒,都倒塌了。大正月出門入戶,不穿件好衣服,惹的人家笑話;敢來上覆嫂子,有上蓋衣服,借的兩套兒;頭面簪環,借的幾件兒。交他穿戴了去。」西門慶令王經:「你裡邊對你大娘說去。」伯爵道:「應寶在外邊拏着毡包并盒哩,哥哥累你拏進去,就包出來罷。」那王經接毡包進去。良久抱出來,交與應寶,說道:「里面兩套上色段子織金衣服,大小五件頭面,一雙二珠環兒。」應寶接的,往家去了。西門慶陪送伯爵吃茶,說道:「昨日房下在何大人家吃酒,來晚了。今日不想雲二哥娘子送了五個帖兒,又請房下每都會會兒。大房下又有臨月身孕,懶待去。我說他既來請,大節下你等走走去罷。我又連日不得閑。只昨日纔把人事拜了。今日咱每在雲二哥家吃了酒來,昨日家又出去有些小事,來家晚了。今日薛內相又請我門外看春,怎麼得工夫去?吳親廟裏又送帖兒初九日年例打醮,也是去不成,教小婿去了罷。這兩日不知酒多了也怎的,只害腰疼,懶待動彈。」伯爵道:「哥,你還是酒之過,濕疾流注在這下部。」西門慶道:「這節間到人家,誰是肯輕放了你我的?怎麼忌的住!」伯爵又問:「今日那幾會嫂子去?」西門慶:「大房下和第二、第三、第五的房下四人去,我在家且歇息兩日兒罷。」正說着,只見玳安拏進盒兒來,說道:「何老爹家差人送請帖兒來,初九日請吃節酒。」西門慶道:「早是你看着,人家來請,你不去?」於是看盒兒內放着三個請書兒,一個宛紅僉兒,寫着:「大寅丈四泉翁老先生大人」,一個寫「大都閫吳老先生大人」,一個寫:「大鄉望應老先生大人」:俱是「待生何永壽頓首拜。」玳安說:「他那里說不認的,教咱這里轉送送兒罷。」伯爵一見,便說:「這個都怎麼兒的?我還沒送禮兒去與他,他來請我,怎好去?」西門慶道:「我這里替你封上分帕禮兒,你差應寶早送去就是了。」一面令王經:「你封二錢銀子,一方手帕,寫你應二爹名字,與你應二爹。」因說:「你把這請帖兒袖了去,省的我又教人送。」只把吳大舅的差來安兒送去了。須臾,王經封了帕禮,遞與伯爵。伯爵打恭說道:「哥謝!容易是我後日早來會你,咱一同起身。」說畢,作辭去了。午間都表吳月娘等打扮停當,一頂大轎,三頂小轎,後面又帶着來爵媳婦兒惠元收疊衣服,一頂小轎兒;四名排軍喝道,琴童、春鴻、棋童、來安四個跟隨,往雲指揮家來吃酒。正是:

「翠眉雲鬢畫中人,  嬝娜宮腰迎出塵;

天上嫦娥元有種,  嬌羞釀出十分春。」

不說月娘與李嬌兒、孟玉樓、潘金蓮都往雲離守家吃酒去了。西門慶分付大門上平安兒:「隨問甚麼人,只說我不在。有帖兒接了,就是了。」那平安徑過一遭,那里再敢離了左右,只在門首坐的。但有人客來望,只回不在家。西門慶那日,只在李瓶山房中圍爐坐的。自從李瓶兒沒了,月娘教如意兒休勒上奶去,每日只喂奶來興女孩兒城兒。連日西門慶害腿疼,猛然想起任醫官與他延壽丹,用人乳吃。於是來到房中,教如意兒擠乳。那如意兒節間,頭上戴着黃霜霜簪環,滿頭花翠,勒着翠藍銷金汗巾,藍紬子襖兒,玉色雲段披襖兒,黃綿紬裙子,腳下沙綠潞紬,白綾高底鞋兒,粧點打扮,比昔時不同。手上戴着四個烏銀戒指兒,坐在旁邊,打發吃了藥,又與西門慶斟酒脯菜兒。迎春打發吃了飯,走過隔壁,和春梅下棋去了。要茶要水,自有綉春在廚下打發。西門慶見丫鬟都不在屋里,在炕上斜靠着背,扯開白綾吊的羢褲子,露出那話來,帶着銀托子,教他用口吮咂。一面傍邊放着菓酌,斟酒自飲。因呼道:「章四兒,我的兒,你用心替達達咂,我到日日尋出件好粧花段子比甲兒來,你正月十二日穿。」老婆道:「看爹可怜見。」吮弄勾一頓飯時,西門慶道:「我兒,我心里要在你身上燒炷香兒。」老婆道:「隨爹你揀着燒炷香兒。」西門慶令他關上房門,把裙子脫了,上炕來仰臥在枕上,底下穿着新做的大紅潞紬褲兒,褪下一隻褲腿來。西門慶袖內還有燒林氏剩下的三個燒酒浸的香馬兒,撇去他抹胸兒,一個坐在他心口內,一個坐在他小肚兒底下,一個安在他〈毛皮〉蓋子上,用安息香一齊點着。那話下邊便插進牝中,低着頭看着拽,只顧沒稜露腦,送來送進不已,又取過鏡臺來,傍邊照看。須臾,那香燒到肉根前,婦人蹙眉齧齒,忍其疼痛,口里顫聲柔語,哼成一塊,沒口子叫:「達達爹爹,罷了我了,好難忍也!」西門慶更叫道:「章四兒淫婦,你是誰的老婆?」婦人道:「我是爹的老婆。」西門慶教與他:「你說是熊旺的老婆,今日屬了我的親達達了。」那婦人回應道:「淫婦原是熊旺的老婆,今日屬了我的親達達了!」西門慶又問道:「我會{入日}不會?」婦人道:「達達會{入日}〈毛皮〉。」兩個淫聲艷語,無般言語不說出來,西門慶那話粗大,撑的婦人牝戶滿滿,彼往來出入,帶的花心,紅如鸚鵡舌,黑似蝙蝠翅一般,翻覆可愛。西門慶於是把他兩股,板抱在懷內,四體交匝,兩相迎湊,那話盡沒至根,不容毫髮。婦人瞪目失聲、淫水流下。西門慶情濃樂極,精邈如湧泉。正是:

「不知已透春消息,  但覺形骸骨節鎔。」

有詩為證:

「任君隨意薦霞盃,  滿腔春事浩無涯;

一身徑藉東君愛,  不管未頭墜寶釵。」

當日西門慶燒了這老婆身上三處香,開門尋了一件玄色段子粧花比甲兒與他。至晚月娘眾人來家,對西門慶說:「原來雲二嫂也懷着個大身子。俺兩個今日酒席上都遞了酒,說過到明日兩家若分娩了,若是一男一女,兩家結親做親家;若都是男子,同堂攻書;若是女兒,拜做姐妹,一處做針指,來往同親戚兒耍子。應二嫂做保證。」西門慶聽了話笑,言休饒舌。到第二日,都是潘金蓮上壽。西門慶早起往衙門中去了。分付小廝每抬出燈來,收拾揩抹乾淨,大廳捲棚各處掛燈,擺設錦帳圍屏,叫來興買下鮮菓,叫了小優,晚夕上壽的東西。這潘金蓮早辰打扮出來,花粧粉抹,翠袖朱唇;走來大廳上,看見玳安與琴童站着高凳,在那裡掛燈,那三大盞珠子吊掛燈。笑嘻嘻說道:「我道是誰在這里,原來是你每在這里掛燈哩。」琴童道:「今日是五娘上壽,爹分付下俺每掛了燈,明日娘的生日好擺酒,晚夕小的每與娘磕頭,娘已定賞俺每哩。」婦人道:「要打便有,要賞可沒有!」琴童道:「爺嚛!娘怎的沒打不說話,行動只把打放在頭里?小的每是娘的兒女,娘看顧看顧兒更好,如何只說打起來!」婦人道:「賊囚,別要說嘴!你與他好生仔細掛那燈,沒的例兒撦兒的,拏不牢吊將下來。前日年里為崔本來,說你爹大白日里不見了,險不險,赦了一頓打,沒曾打。這槽兒可打成了!」琴童道:「娘只說破話,小的命兒薄薄的,又諕小的!」玳安道:「娘也不打聽,這個話兒娘怎得知?」婦人道:「宮外有株松,宮內有口鍾,鍾的聲兒,松的影兒,我怎麼有個不知道的!昨日可是你爹對你大娘說,去年有賁四在家,還扎了幾架烟火放。今年他不在家,就沒人會扎。乞我說了兩句:『他不在家,左右有他老婆會扎,教他扎不是!』」玳安道:「娘說的甚麼話?一個夥計家,那里有此事?」婦人道:「甚麼話?撞木靶,有此事,真個的!畫一道兒,只怕{入日}過界兒去了!」琴童道:「娘也休聽人說,他只怕賁四來家知道。」婦人道:「瞞那傻王八千來個!我只說那王八也是明王八,怪不的他往東京去的放心,丟下老婆在家,料莫他也不肯把〈毛皮〉閑着!賊囚根子們,別要說嘴!打夥兒替你爹做牽頭,勾引上了道兒,你每好圖躧狗尾兒!說的是也不是?敢說我知道,嗔道賊淫婦買禮來!與我也罷了,又送蒸酥與他大娘。另外又送一大盒瓜子兒與我,小買住我的嘴頭子,他是會養漢兒!我就猜沒別人,就知道是玳安兒這賊囚根子替他鋪謀定計。」玳安道:「娘屈殺小的,小的平白管他這勾當怎的?小的等閑也不往他屋裡去,娘也少聽韓回子老婆說話。他兩個為孩子好不嚷亂!常言:『要好不能勾,要歹登時就一篇;房倒壓不殺人,舌頭倒壓殺人。』聽者有,不聽者無。論起來賁四娘子為人和氣,在咱門首住着,家中大小,沒曾惡識了一個人。誰人不在他屋裡討茶吃?莫不都養着?倒沒放處!」金蓮道:「我見那水眼淫婦,矮着個靶子,兩是半頭磚兒也是一個兒,把那水濟濟眼擠着,七八拏的兒舀,好個怪淫婦!他便和韓道國老婆,那長大摔瓜淫婦,我不知怎的,搯了眼兒不待見他!」正說着,只見小玉走來說:「俺娘請五娘,潘姥姥來了,要轎子錢哩。」金蓮道:「我在這里站着,他從多咱進去了?」琴童道:「姥姥打夾道里,我送進去了。一來的抬轎的,該他六分銀子轎子錢。」金蓮道:「我那得銀子來?人家來不帶轎子錢兒走!」一面走到後邊,見了他娘,只顧不與他轎子錢,只說沒有。月娘道:「你與姥姥一錢銀子,寫帳就是了。」金蓮道:「我是不惹他,他的銀子都有數兒。只教我買東西,沒教我打發轎子錢!」坐了一回,大眼看小眼。外邊抬轎子的,催着要去。玉樓見不是事,向袖中拏出一錢銀子來,打發抬轎的去了。不一時,大妗子、二妗子、大師父來了。月娘擺茶吃了。潘姥姥歸到前邊他女兒房內來,被金蓮儘力數落了一頓,說道:「你沒轎子錢,誰教你來了?恁出魄削劃的,教人家小看!」潘姥姥道:「姐姐你沒與我個錢兒與我來,老身那討個錢兒來?好容易賙辨了這分禮兒來!」婦人道:「指望問我要錢,我那里討個錢兒與你?你看着睜着眼在這里,七個窟【土竉】,到有八個眼兒等着在這里!今後你有轎子錢,便來他家來;沒轎子錢,別要來。料他家也沒少你這個窮親戚,休要傲打嘴的獻世包!關王買豆腐,人硬!我又聽不上人家那等〈毛皮〉聲顙氣。前日為你去了,和人家大嚷大鬧的,你知道?你罷了,驢糞毬兒面前光,卻不知里面受恓惶!」幾句說的潘姥姥嗚嗚咽咽哭起來了。春梅道:「娘今日怎的只顧說起姥姥來了!」一面安撫老人家在里邊炕上的,連忙了點了盞茶與他吃。潘姥姥氣的在炕上睡了一覺,只見後邊請陪大妗子吃飯,纔起來往後邊去了。西門慶從衙門中來家,正在上房擺飯,忽有玳安拏進帖兒來說:「荊老爹陞了東南統制。來拜爹。」西門慶見帖兒上寫「新陞東南統制兼督漕運總兵官荊忠頓首拜。」慌的西門慶令抬開飯卓,連忙穿衣冠帶,迎接出來。只見荊總制穿着大紅麒麟補服,渾金帶進來,後面跟着許多僚掾軍牢。一面讓至大廳上,敘禮畢,分賓主而坐。茶湯上來,待茶畢,荊統制說道:「前日陞官,勒書纔到。還未上任,逕來拜謝老翁。」西門慶道:「老總兵榮擢,恭喜!大才必有大用,自然之道。吾輩亦有光矣,容當拜賀。」一面:「請寬尊服,少坐一飯。」即令左右放卓兒。荊統制再三致謝道:「學生奉告老翁,一家尚未拜,還有許多薄冗,容日再來請教罷。」便徑起身。西門慶那里肯放,隨令左右上來,寬去衣服,登時打抹春臺,收拾酒菓上來。獸炭頓燒,煖簾低放;金壺斟玉液,翠盞貯羊羔。纔斟上酒來,只見鄭春、玉相兩個小優兒來到,扒在面前磕頭。西門慶道:「你兩箇如何這咱纔來?」問鄭春:「那一個叫甚名字?」鄭春道:「他喚王相,是王柱的兄弟。」西門慶即令拏樂器上來彈唱,與他荊爺聽。須臾,兩個小優安放樂器停當,歌唱了一套霽景融和。左右拿上兩盤攢盒點心嗄飯,打發馬上人等。荊統制道:「這等就不是了。學生叨拜,下人又蒙賜饌,何以克當!」即令上來磕頭,西門慶道:「一二日房下還要潔誠請尊正老夫人賞燈一敘,望乞下降。在座者惟老夫人、張親家夫人,同僚何天泉夫人,還有兩位舍親,再無他人。」荊統制道:「若老夫人尊票到、賊荊已定趨赴。」又問起:「周老總兵怎的不見陞轉?」荊統制道:「我聞得周菊軒也只在三月間,有京營之轉。」西門慶道:「這也罷了。」坐不多時,荊統制告辭起身。西門慶送出大門,看着上馬喝道而去。晚夕潘金蓮上壽,後廳小優彈唱,遞了酒,西門慶便起身往金蓮房中去了。月娘陪着大妗子、潘姥姥、女兒郁大姐、兩個姑子,在上房坐的飲酒。潘金蓮便陪西門慶在他房內,從新又安排上酒來,與西門慶梯已遞酒磕頭。落後潘姥姥來了,金蓮打發他李瓶兒這邊歇臥。他便陪着西門慶自在飲酒作歡,頑耍做一處。都說潘姥姥到那邊屋里,如意、迎春讓他熱炕上坐着。先是姥姥看見明間內,靈前供擺着許多獅仙五老定勝 ,樹菓柑子,石榴蘋婆 ,雪梨鮮菓蒸酥點心,饊子蔴花,滿爐焚着未子香臘,點着長明燈,卓上拴着銷金卓幃,旁邊掛着他影,穿大紅遍地金袍兒,錦裙綉襖,珠子挑牌,向前道了個問訊,說道:「姐姐好處生天去了!」因坐在炕上,向如意兒、迎春道:「你娘勾了,官人這等費心追薦,受這般大供養勾了!他是有福的。」如意兒道:「前日娘的百日,請姥姥怎的不來?門外花大妗子和大妗子,都在這里來。十二個道士念經,好不大吹大打,揚播道場,水火煉度,晚上纔去了。」潘姥姥道:「幫年逼節,丟着個孩子在家,我來家中沒人,所以就不曾來。今日你楊姑娘怎的不見?」如意兒道:「姥姥還不知道,楊姑娘老病死了。從年里俺娘念經就沒來。俺娘們都往北邊與他上祭去了。」潘姥姥道:「可傷!他大如我,我還不曉的他老人家沒了!嗔道今日怎的不見他!」說了一回楊姑娘。如意兒道:「姥姥有鍾兒甜酒兒,你老人家用些兒?」一面教:「迎春姐,你放小卓兒在炕上,篩甜酒與姥姥吃盃。」不一時取到,飲酒之間,婆子又提起李瓶兒來:「你娘好人,有仁義的姐姐,熱心腸兒。我但來這里,沒曾把我老娘當外人看成。到就是熱茶熱水與我吃,還只恨我不吃。夜間和我坐着說話兒。我臨家去,好歹包些甚麼兒與我拏了去,誓沒曾空了我。不瞞姐姐你每說,我身上穿的這披襖兒,還是你娘與我的!正經我那冤家,半個折針兒也迸不出來與我!我老身不打誑語,阿彌陀佛,水米不打牙,他若肯與我一個錢,我滴了眼睛在地!你娘與了我些甚麼兒,他還說象小眼薄皮,愛人家的東西!想今日為轎子錢,你大包家拏着銀子,就替他老身出幾分,便怎的!咬定牙兒,只說他沒有。倒教後邊西房里姐姐,拏出一錢銀子來,打發抬轎的去了。歸到屋里,還數落了我一頓,到明日有轎子錢,便教我來;沒轎子錢,休教我上門走!我這去了,不來了,來到這里,沒的受他的氣!隨他去,有天下人心狠,不似俺這短壽命!姐姐你每聽着我說,老身苦死了,他到明日不聽人說,還不知怎麼收成結果哩!想著你從七歲沒了老子,我怎的守你到如今?從小兒交你做針指,往徐秀才家上女學去,替你怎麼纏手、縛腳兒的。你天生就是這等聰明伶俐?到得這步田地,他把娘喝過來斷過去,不看一眼兒!」如意兒道:「原來五娘從小兒上學來?嗔道恁題起來,就會識字深!」潘姥姥道:「他七歲兒上女學,上了三年,字倣也曾寫過:甚麼詩詞歌賦唱本上字不認的!」正說着,只見打的角門子響。如意兒道:「是誰叫門?」使綉春:「二姐,你去瞧瞧去。」那綉春走來,說:「是春梅姐來了。」如意兒連忙捏了潘姥姥一把手,就說道:「姥姥悄悄的,春梅來了。」潘姥姥道:「老身知道。他與我那冤家一條腿兒。」只見春梅進來,頭上翠花雲髻兒,羊皮金沿的珠子箍兒,藍綾對衿襖兒,黃綿紬裙子,金燈籠墜子子,貂鼠圍脖兒,走來見眾人陪着潘姥姥吃酒,說道:「姥姥還沒睡哩?我來瞧瞧姥姥來了。」如意兒讓他坐。這春梅把裙子摟起,一屁股坐在炕上。迎春便緊挨着他坐。如意坐在右邊炕頭上,潘姥姥坐在當中。因問:「你爹和你娘睡了不曾?」春梅道:「剛纔吃了酒,打發他兩個睡下了。我來這邊瞧瞧姥姥,有幾樣菜兒,一壺兒酒,取了來和姥姥坐的。」因央及綉春:「你那邊教秋菊掇下來,我已是攢下了。」那綉春不一時,走過那邊,取了來。秋菊放盒內掇着菜兒,綉春提了一錫瓶金華酒。分付秋菊:「你往房里看去,聽着若叫我,來這里對我說。」那秋菊把嘴谷都著了去了。一面擺酒在炕卓上,都是燒鴨 火腿,薰臘鵝 、細鮓糟魚、菓仁、鹹酸蜜食、海味之類,堆滿春臺。綉春關上角門,走進在旁邊陪坐。於是篩上酒來,春梅先遞了一鍾與潘姥姥,然後遞一鍾如意兒,一鍾與迎春。綉春在旁邊炕兒上坐的,共五人坐,把酒來斟。春梅護衣碟兒內,每樣揀出遞與姥姥眾人吃,說道:「姥姥,這個都是整菜,你用些兒。」那婆子道:「我的姐姐,我老身吃。」因說道:「就是你娘,從來也沒費恁個心兒管待我管待兒,姐姐,你倒有惜孤愛老的心!你到明日,管情好一步一步自高。敢是俺那冤家,沒人心,沒人義!幾遍為他心齷齪,我也勸他,他就扛的我失了色!今早是姐姐你看着,我來你家討冷飯吃來了?你下老實那等扛我!」春梅道:「姥姥罷,你老人家只知其一,不知其二。俺娘他爭強不伏弱的性兒,比不同的六娘,錢自有。他本等手里沒錢,你只說他不與你;別人不知道,我知道。相俺爹雖是抄的銀子放在屋里,俺娘正眼兒也不看他的。若遇着買花兒東西,明公正義問他要,不恁瞞並藏背掖的;教人看小了他,他怎麼張着嘴兒說人!他本沒錢,姥姥怪他,就虧了他了。莫不我護他?也要個公道!」如意兒道:「錯怪了五娘。自古親兒骨肉,五娘有錢,不孝順姥姥,再與誰?常言道:『要打看娘面,千朵桃花一樹兒生。』到明日你老人家黃金入櫃,五娘他也沒個貼皮貼肉的親戚,就如死了俺娘樣兒!」婆子道:「我有今年沒明年,知道今死明日死?我也不怪他。」春梅見婆子吃了兩鍾酒,韶刀上來了。便叫迎春:「二姐你拏骰盆兒來,咱每個擲個骰兒搶紅耍子兒罷。」不一時,取了四十個骰兒的骰盆兒來。春梅先與如意與擲,擲了一回,又與迎春擲,都是賭大鐘子。你一盞,我一鍾,須臾竹葉穿心,桃花上臉,把一錫瓶酒吃的罄淨。迎春又拏上半鐔麻姑酒 來,也都吃了。約莫到二更時分,那潘姥姥老人家,熬不的,又早前靠後仰打起盹來,方纔散了。春梅便歸這邊來。推了推角門,開着;進入院內,只見秋菊正在明間板壁縫兒內,倚着春凳兒,聽他兩個在屋里行房,怎的作聲喚,口中呼叫甚麼。正聽的熱鬧,不防春梅走來到根前,向他腮頰上,儘力打了個耳刮子,罵道:「賊小死的囚奴,你平白在這里聽甚麼!」打的秋菊睜睜的說道:「我在這裡打盹,誰聽甚麼來?你就來打我!」不想房內婦人聽見,便問春梅:「他和誰說話?」春梅道:「沒有人。我使他關門,他不動。」於是替他摭過了。秋菊揉着眼,關上房門。春梅走到炕上,摘頭睡了,不在話下。正是:

「鷓鷓有意留殘景,  杜宇無情戀晚輝。」

一宿晚景題過。次日,潘金蓮生日,有傅夥計、某夥計、賁四娘子、崔本媳婦、段大姐、吳舜臣媳婦、鄭三姐、吳二妗子都在這里。西門慶約會吳大舅、應伯爵,整衣冠,尊瞻視,騎馬喝道,往何千戶家赴席。那日也有許多官客,四個唱的,一起雜耍;周守禦同席。飲酒至晚回家,就在前邊和如意兒歇了。到初十日,發帖兒請眾官娘子吃酒。月娘便向西門慶說:「趁着十二日看燈酒,把門外他孟大姨和俺大姐,也帶著請來坐坐,省的教他知道惱,請人不請他。」西門慶道:「早是你說。」分付陳經濟:「再寫兩個帖,差琴童兒請去。」這潘金蓮在旁聽著多心,走到屋里,一面攛掇把潘姥姥就要起身。月娘道:「姥姥,你慌去怎的?再消住一日兒是的。」金蓮道:「姐姐,大正月裏,他家里丟着孩子沒人看,教他去罷。」慌的月娘裝了兩個盒子點心菜食,又與了他一錢轎子錢,管待打發去了。因對着李嬌兒說:「他明日請他有錢的大姨兒來看燈吃酒,一個老行貨子,觀眉觀眼的,不打發去了,平白教他在屋里做甚麼?待要說是客人,沒好衣服穿;待要說是燒火的媽媽子,又不似。倒沒的教我惹氣!」西慶使玳安兒送了四個請書兒往招宣府,一個請林太太,一個請王三官兒娘子黃氏。又使他院中早叫李桂姐、吳銀兒、鄭愛月兒、洪四兒,四個唱的,李銘、吳惠、鄭奉三個小優兒。不想那日賁四從東京來家,梳洗頭臉,打選衣帽齊整,來見西門慶磕頭,遞上夏指揮回書。西門慶問他:「如何住這些時不來?」賁四具言在京感冒打寒一節,直到正月初二日,纔收拾起身回來。夏老爹多上覆老爹,多承看顧。西門慶照舊還把鑰匙教他管絨綿鋪。另外一間,教吳二舅開鋪子賣紬絹。到明日松江貨船到,都卸在獅子街房內,同來保發賣,且教賁四娘叫花兒匠在家,儹造兩架烟火,十二日要放與堂客看。早約下應伯爵、謝希大、吳大舅、常時節四位,白日在廂房內坐的。晚夕只見應伯爵領了李三見西門慶,先道當日外承携之事。坐下吃畢茶,方纔說起:「李三哥來,今有一宗買賣與你說,你做不做?」西門慶道:「端的甚麼買賣,你說來?」李三道:「今有朝廷東京行下文書,天下十三省,每省要用萬兩銀子的古器。咱這東平府坐派着二萬兩,批文在巡按處,還未下來。如今大街上張二官府破二百兩銀子幹這宗批要做,都看有一萬兩銀子尋。小人會了二叔,敬來對老爹說。老爹若做,張二官府拏出五千兩來,老爹拏出五千兩來,兩家合着做這宗買賣,左右沒人,這邊是二叔和小人與黃四哥,他那邊還有兩個夥計,二八分錢使。未知老爹意下何如?」西門慶問道:「是甚麼古器?」李三道:「老爹還不知。如今朝廷皇城內新蓋的艮嶽,改為壽岳,上面起蓋許多亭臺殿閣;又建上清寶籙宮會真堂璇神殿;又是安妃娘娘梳粧閣;都用着這珍禽奇獸,周彝商鼎,漢篆秦爐,宣王石鼓,歷代銅鞮,仙人掌,承露盤,并希世古董玩器擺設。好不大興工程,好少錢糧!」西門慶聽了,說道:「此是我與人家打夥兒做,我自家做了罷。敢量我拏不出這一二萬銀子來?」李三道:「得老爹全做,又好了!俺每就瞞着他那邊了。左右這邊二叔和俺每兩個,再沒人。」伯爵道:「哥,家里還添個人兒不添?」西門慶道:「到根前,再添上賁四替你們走跳就是了。」西門慶又問道:「批文在那里?」李三道:「還在巡按上邊,沒發下來哩。」西門慶道:「不打緊,我這差人寫封書,封些禮,問宋松原討將來就是了。」李三道:「老爹若討去,不可遲滯。自古兵貴神速,先下米的先吃飯。誠恐遲了,行到府裏,乞別人家幹的去了。」西門慶笑道:「不怕他。設使就行到府里,我也還教宋松原拏回去就是;胡府尹我也認的。」於是留李三、伯爵同吃了飯,約會我如今就寫書,明日差小价去。李三道:「又一件,宋老爹如今按院不在這里了。從前日起身,往兗州府盤查去了。」西門慶道:「你明日就同小价往兗州府走遭。」李三道:「不打緊,等我去,來回破五六日罷了。老爹差那位管家?等我會下,有了書,教他往我那里歇。明日我同他好早起身。」西門慶道:「別人你宋老爹不認的。他常喜的是春鴻,教春鴻,來爵一時兩個去罷。」於是叫他二個人到面前,會了李三,晚夕在他家宿歇。伯爵道:「這等纔好,事要早幹。多才疾足者得之!」於是與李三吃畢飯,告辭而去。西門慶隨即教陳經濟寫了書,又封了十兩葉子黃金,在書帕內與春鴻、來爵二人,分付路上仔細:「若討了批文,即便早來,若是行到府里,問你宋老爹討張票,問府里要。」來爵道:「爹不消分付,小的曾在兗州答應過徐參議,小的知道。」於是領了書禮,打在身邊,逕往李三家去了。不說十一日來爵、春鴻同李三早顧了長行頭口,往兗州府去了。都說十二日,西門慶家中請各官堂家飲酒,那日在家不出門,約下吳大舅、應伯爵、謝希大、常時節四位晚夕來在捲棚內賞燈飲酒。王皇親家樂小廝,從早辰就挑了廂子來了,在前邊廂房做戲房。堂客到,打銅鑼銅鼓迎接。周守禦娘子有眼疾,不得來,差人來回。又是荊統制娘子、張團練娘子、雲指揮娘子,并喬親家母、崔親家母、吳大姨、孟大姨都先到了。只有何千戶娘子,王三官母親林太太并王三官娘子不見到。西門慶使排軍、玳安、琴童兒來回催邀了兩三遍,又使文嫂兒催邀。午間只見林氏一頂大轎,一頂小轎跟了來。見了禮,請西門慶拜見。問:「怎的三官娘子不來?」林氏道:「小兒不在,家中沒人。」拜畢下來。止有何千戶娘子,直到晌什大錯纔來。坐着四人大轎,一個家人媳婦,坐小轎跟隨,排軍抬着衣廂,又是兩位青衣家人,緊扶着轎竿。到二門裏纔下轎,前邊鼓樂吹打迎接。吳月娘眾姊妹迎至儀門首。西門慶悄悄在西廂房放下簾來,偷瞧見這藍氏,年約不上二十歲,生的長挑身材,打扮的如粉粧玉琢。頭上珠翠堆滿,鳳翹雙插。身穿大紅通袖五彩粧花四獸麒麟袍兒,繫着金箱碧玉帶,下襯着花錦藍裙,兩邊禁步叮〈口東〉,麝蘭香噴。但見:

「儀容嬌媢,體態輕盈。姿性兒百伶百俐,身段兒不短不長。細彎彎兩道蛾眉,直侵入鬢;滴溜溜一雙鳳眼,來往踅人。嬌聲兒似囀日流鶯,嫩腰兒似弄風楊柳。端的是綺羅隊里生來,都壓豪華氣象;珠翠叢中長大,那堪雅淡梳粧。開遍海棠花,也不問夜來不少;飄殘楊柳絮,竟不知春色如何。要知他半點真情,除非是穿綺窗皓月;能施他一腔心事,都便似翻綉幌清風。輕移蓮有步,有蕊珠仙子之風流;欵蹙湘裙,似水月觀音之態度。」

正是:

「比花花解語,  比玉玉生香!」

這西門慶不見則已,一見魂飛天外,魄喪九霄。未曾體交,精魄先失。少頃,月娘等迎接,進入後堂相見。敍禮已畢,請西門慶拜見。西門慶得不還一聲,連忙整衣冠行禮,恍若瓊林玉樹臨凡,神女巫山降下。躬身施禮,心搖目蕩,不能禁止。拜見畢,下來。先在捲棚內放卓兒擺茶,極盡希奇美饌。然後大廳上坐陳水陸珍羞,正面設石崇錦帳圍屏,四下鋪玳筵廣席。花燈高挑,綵繩半拽。雕梁錦帶低垂,畫燭齊明寶蓋。魚龍山戲,恍一片珠璣;殿閣樓臺,簇千團翡翠。左邊廂,九姊十妹美人圖畫丹青;右首下,九曜八洞神仙粧成金碧。吃的是龍肝鳳髓 ,熊掌駝峰 。歌的錦瑟銀箏,鳳筲象管。龜鼓鼕鼕驚過鳥,砍喉囀囀遏行雲。席上嬌嬈,盡是珠圍翠繞;階下腳色,皆按離合悲歡。正是:

「得多少進酒了鬟雙落浦,  獻羔侍妾兩嫦娥。」

當下林太太上席,戲文扮的是小天香半夜朝元記。唱了兩摺下來,李桂姐、吳銀兒、鄭月兒、洪四兒四個唱的上去彈唱。吳大姨門外,先起身去了。唱燈詞錦綉花燈半空挑。西門慶在捲棚內,自有吳大舅、應伯爵、謝希大、常時節,李銘、吳惠、鄭春三個小優兒彈唱飲酒。不住下來大廳格子外,往里觀覷。這各家跟轎子家人伴當,自有酒饌,前廳管待,不必用說。次第明月圓,容易彩雲散;樂極悲生,否極泰來,自然之理。西門慶但知爭名奪利,縱意奢淫。殊不知天道惡盈,鬼錄來追,死限臨頭。到晚夕,堂中點起燈來,小優兒彈唱燈詞。還未到起更時分,西門慶正陪着人坐的,就在席上齁齁的打起睡來。伯爵便行令猜枚,鬼混他,說道:「哥,你今日沒高興,怎的只打睡?」西門慶道:「我昨日沒曾睡,不知怎的,今日只是沒精神打睡。」只見四個唱的下來。伯爵教兩個唱燈詞,兩個遞了酒。當下洪四兒與鄭月兒兩個彈着箏琵琶唱,吳銀兒與李桂姐遞酒。正要在熱鬧處,忽玳安來報:「林太太與何老爹娘子起身了。」這西門慶席下來,黑影里走到二門里首,偷看着他上轎。月娘眾人送出來,前邊天井內看放烟火。藍氏穿着大紅遍地金貂鼠皮襖,翠藍遍地金裙。林太太是白綾襖兒,貂鼠披襖,大紅裙,帶着金鐸玉珮。家人打着燈籠,簇擁上轎而去。這西門慶正是餓眼將穿,饞涎空嚥,恨不能就要成雙。見藍氏去了,悄悄從夾道進來。當時沒巧不成語,姻緣會湊,可霎作怪!不想來爵兒媳婦見堂客散了,正從後邊歸來開他房門。不想頂頭撞見西門慶,沒處藏躲。原來西門慶見媳婦子生的喬樣,安心已久。雖然不及來旺妻宋氏風流,也頗克得過第二。於是乘着酒興兒,雙關接進他房中親嘴。這老婆當初在王皇親家,因是養個主子,被家人不忿攘鬧,打發出來。今日又撞着這個道路,如何不從。一面就遞舌頭在西門慶口中。兩個解衣褪褲,就按在炕沿子上,掇起腿來,被西門慶就聳了個不亦樂乎。正是:

「未曾得遇鶯娘面,  且把紅娘去解饞。」

有詩為證:

「燈月交光浸玉壺,  分得清光照綠珠;

莫道使君終有婦,  教人桑下覓羅敷。」

畢竟未知後來何如,且聽下回分解:






\end{showcontents}


