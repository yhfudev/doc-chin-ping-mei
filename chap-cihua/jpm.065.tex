%# -*- coding: utf-8 -*-
%!TEX encoding = UTF-8 Unicode
%!TEX TS-program = xelatex
% vim:ts=4:sw=4
%
% 以上设定默认使用 XeLaTex 编译,并指定 Unicode 编码,供 TeXShop 自动识别

%第六十五回 
\chapter{吳道官迎殯頒真容\KG 宋御史結豪請六黃}

「齊眉相見喜柔和,  誰料參商發結歌,

殘月雲邊懸破鏡,  流光機上柳飛梭;

愁隨草色春深謝,  苦入連心夜幾何,

試問流乾多少淚,  楓林秋色一般多。」

話說到九月二十八日,李瓶兒死了二七光景,玉皇廟吳道官受齋,請了十六個道眾,在家中揚旛修建請去救苦二七齋价壇。早修之時,有官安郎中來下書。西門慶待來人去了,吳道官廟中擡了三牲祭器,湯飯盤餅饊素食,金銀錠香紙之類,又是一疋尺頭,以為奠儀。道眾繞棺傳咒,吳道官靈前展拜。西門慶與經濟回禮,謝道:「師父多有破費,何以克當?」吳道官道:「小道甚是惶愧,本當該助一經,追薦夫人。曾奈力薄,粗茶飯奠,表意而已,望乞大人笑納。」西門慶祭畢,即收了,打發擡盒人回去。那日三朝轉經,演生神章,破九幽獄,對靈攝召,拜進救苦朱表,領告諸真符命,整做法事,俱不必細說。第二日先是門外韓姨夫家來上祭。那時孟玉樓兄弟外邊做買賣去了,五六年沒來家,昨至是來家見他姐姐嫂子。西門慶這邊有喪事,跟隨姨夫那邊來上祭,討了一分孝去,送了許多人事兒。西門慶敍禮,進入玉樓房中拜見。至是堂有十數位人。

西門慶這邊亦設席管待,俱不在言表。那日午間,又是本縣知縣李拱極,縣丞錢斯成,主簿任良貴,典史夏恭基,又有陽谷縣狄斯朽,共五員官,都關了分,穿服來上紙帛弔問。西門慶備席在捲棚內管待,請了吳大舅與溫秀才相陪,三個小優兒彈唱。馬上人俱有攢盤領下去,自有坐處吃。正飲酒到熱鬧處,當時沒巧不成話。忽報管磚戍廠工部黃老爹來弔孝,慌的西門慶連忙穿孝衣靈前伺侯。溫秀才又早迎至大門外,讓至前廳,換了衣裳,跟從進來。家下人手捧香燭、紙疋、金段到靈前,用紅漆丹盤捧過香來跪下。黃主事上了香,展拜畢。西門慶同經濟下來還禮。黃主事詭:「學生不知尊閫沒了,弔遲。恕罪!恕罪!」西門慶詭:「學生一向欠恭,今又承老先生枉弔,兼辱厚儀,不勝感激。」敍畢禮,讓至棚內上面坐下,西門慶與溫秀才下邊相陪,左右捧茶上來。

吃了茶,黃主事道:「昨日宋松原多致意先生,他也聞知令夫人作過,也要來弔問。爭奈有許多事情羈絆,他如今在濟州住劄。先生還不知,朝廷如今營建艮嶽,勅旨令太尉朱勔,往江南湖湘採取花石綱,運船陸續打河道中來。頭一運將次到淮上,又欽差殿前六黃太尉來,迎取卿雲萬態奇峯,長二丈,闊數尺,都用黃毡蓋覆,張打黃旗,費數號船隻,由山東河道而來。況河中沒水,起八郡民夫牽挽。官吏倒懸,民不聊生。宋道長督率州縣,事事皆親身經歷,案牘如山,晝夜勞苦,通不得閒。況黃大尉不久自京而至,宋道長必須率三司官員,要接他一接。想此間無可相熟者,委托學生來,敬煩尊府作一東,要請六黃太尉一飯,未審尊意可允否?」因喚左右:「叫你宋老爹承差上來。」有二青衣官吏跪下,毡包內捧出一對金段,一根沉香,兩根白蠟,一分綿紙。「此乃宋凡公致賻之儀?那兩封是兩司八府官員辦酒分資。兩司官十二員,每員三兩,府官八員,每員五兩,計二十二分,共一百零六兩。」交與西門慶:「有勞盛使一備之,何如?」西門慶再三辭道:「學生有服在家,奈何,奈何!」因問:「迎接在於何時?」黃主事道:「還早哩,也得到出月半頭,黃丸監京中還未起身。」西門慶道:「學生十月十二日纔發引,既是宋公祖老先生分付,敢不領命。」又兼謝:「盛儀賻禮,且領下;分資,決不敢收。該多少卓席,只顧分付,學生無不畢具。」黃主事道:「四泉此意差矣。松原委托學生來煩瀆,此乃山東一省各官公禮,又非松之己出,何得見卻?如其不納,學生即回松原,再不敢煩瀆矣?」西門慶聽了此言,說道:「學生權且領下。」因令玳安。王經接下去。問:「備多少卓席?」黃主事道:「六黃備一張吃著大卓面,宋公與兩司都是平頭卓席。以下府官,散席而已。承應樂人,自有差撥伺侯,府上不必再叫。」說畢,茶湯兩換,作辭起身。西門慶款留,黃主事道:「學生還到尚柳塘老先生那裡拜拜他。昔年曾在學生敝處作縣令,然後轉成都府推官。如今他令郎兩泉,又與學生鄉試同年。」西門慶道:「學生不知老先生與尚兩泉相厚,兩泉亦與學生相交。」黃主事起身。西門慶道:「煩老先生多致意宋公祖,至期寒舍拱侯矣。」黃主事道:「臨期松原差人來通報,先生亦不可太奢。」西門慶道:「學生知道。」送出大門,上馬而去。

那縣中官員,聽見黃主事帶領巡按上司人來,諕的都躲在山子下小捲棚內飲酒。分付手下,把轎馬藏過一邊。當時西門慶回到捲棚,與眾官相見,具說宋巡按率兩司八府來央煩出月迎請六黃太尉之事。眾官悉言:「正是州縣不勝憂苦,這件事欽差若來,凡一應秪迎廩餼,公宴器用人夫,無不出於州縣,必取之于民,公私困極,莫此為甚。我輩還望四泉,各上司處美言提拔,足見厚愛之至。」言訖,都不久坐,告辭起身,上馬而去。話休饒舌。

到李瓶兒三七,有門外永福寺道堅長老,領十六眾上堂僧來念經。穿雲錦袈裟,戴昆盧帽,大鈸大鼓。早辰取水轉五方,請三寶浴佛;午間加持召破獄,禮拜梁皇懺,談孔雀,甚是齊整;晚夕喬大戶娘子與眾夥計娘子,與月娘等伴宿,在靈前看偶戲。西門慶與應伯爵、吳大舅、溫秀才在棚內東首另設圍屏飲酒。十月初八日是四七,請西門外寶慶寺趙喇嘛,亦十六眾,來念番經,結壇跳沙,灑花米行香,口誦真言,齋供都用牛乳茶酪 之類。懸掛都是九醜天魔變相,身披纓絡琉璃,項掛髑髏,口咬嬰兒,坐跨妖魅,腰纏蛇螭。或四頭八臂,或手執戈戟,朱髮藍面,醜惡莫比。午齋已後,就動葷酒。西門慶那日不在家,同陰陽徐先生往門外墳上破土開壙去了。後晌方回。晚夕打發喇嘛散了。

次日推運山頭酒米卓面肴品,一應所用之物。又委付主管夥計,庄上前後搭棚四五處,酒房廚坊,墳內穴邊,又起三間罩棚。先請附近地鄰來坐席面,大酒大肉管待。臨散,背肩背項負而歸,俱不必細說。十一日白日,先是歌郎并鑼鼓地弔,來靈前參靈,弔五鬼鬧判、張天師著鬼迷、鍾馗戲小鬼、老子過函關、六賊鬧彌勒、雪裡梅、莊周夢蝴蝶、天王降地、水火風洞賓飛劍斬黃龍、趙太祖千里送荊娘,各樣百戲。弔罷,堂客都在簾內觀看,參罷靈去了。內眷親戚,都來辭靈燒紙,大哭一場。

到次日發引,先絕早擡出名旌,各項旛亭紙劄。僧道鼓手,細樂人役,都來伺侯。西門慶預先問帥府周守備討了五十名巡捕軍士,都帶弓馬,全裝結束,留十名在家看守,四十名跟殯,在材前擺馬道,分兩翼而行。衙門裡又是二十名排軍打路,照管冥器。墳頭又是二十名把門,管收祭祀。那日官員士夫,親鄰朋友,來送殯者,車馬喧呼,填街塞巷。本家并親眷堂客,轎子也有百十餘頂;三院駂子粉頭,小轎也有數十。徐陰陽擇定辰時起棺。西門慶留下孫雪娥并二女僧看家。平安兒同兩名排軍把前門,那女婿陳經濟跪在柩前摔盆。六十四人上扛,有仵作一員官,立于增架上,敲响板,指撥擡材人上肩。先是請了報恩寺朗僧官來起棺,剛轉過大街口望南走,那兩邊觀看的,人山人海。那日正值晴明天氣,果然好殯!但見:

「和風開綺陌,細雨潤芳塵。東方曉日初升,北陸殘烟乍歛。鼕鼕嚨嚨,花喪鼓不住聲喧;叮叮噹噹,地吊鑼連宵振作。名旌招颭,大書九尺紅羅;起火軒天,中散半空黃霧。猙猙獰獰,開路鬼斜擔金斧;忽忽洋洋,險道神端秉銀戈。逍逍遙遙,八洞仙龜鶴遶定;窈窈窕窕,四毛女虎鹿相隨。地弔鬼晃一片鑼篩,烟火架迸千枝花炮。熱熱鬧鬧,採蓮船撒科打諢;長長大大,高橇漢貫甲頂盔。清清秀秀,小道童十六眾,眾眾都是霞衣道髻,擊坤庭之金,奏八琅之璈,動一派之仙音;肥肥胖胖,大和尚二十四個,個個都是雲錦袈娑,排大鈸,敲大鼓,轉五方之法事。一十二座大絹亭,亭亭皆綠舞紅飛,二十四座小絹亭,座座盡珠圍翠繞。左勢下天倉與地庫相連,右勢下金山與銀山作隊。掌醢廚列八珍之罐,香燭亭供三獻之儀。六座百花亭,現千團錦綉,一乘引魂轎,扎百結黃絲。這邊把花與雪柳爭輝,那邊寶盖與銀幢作隊。金字旛、銀字旛,緊護棺輿;白絹繖,綠絹繖,桐圍增架。斧符雲氣,一邊三把,皆彩畫鮮明;執罐捧巾,兩下侍妾,盡梳粧如活。功布招颭,孝眷聲哀,簇捧定五出頭六歌郎仰覆運須彌座;六十四名青衣白帽,穩穩抬定五老雲鶴華蓋頂,四垂頭流蘇帶,大紅銷金寶象花棺罩;裡面安著巍巍不動錦綉棺輿。只見那兩邊打路排軍,個個都頭戴孝巾,身穿衲襖,腰繫孝帶,腳靸腿繃〈革翁〉鞋,手執欖杵,前呼後擁。兩邊走解的,頭戴芝蔴羅萬字頭巾,撲匾金環飛於腦後。穿的是兩三領紵絲衲襖,腰繫紫纏帶,足穿鷹爪四縫乾黃靴,襯著五彩翻身搶水獸納紗襪口。賣解猶如鷹鷂,走馬好似猿猴。執著一桿明鎗,顯硃紅桿令字藍旗。豎肩樁打斤斗,隔肚穿錢,金雞獨立,仙人打過橋,鐙裡藏身。人人喝采,個個爭誇。扶肩擠背,紛紛不辨賢愚,挨覩並觀,攘攘那分貴賤。張三蠢胖,只把氣吁;李四矮矬,頻將腳蹗。白頭老叟,盡將拐捧拄髭鬚,綠鬢佳人,也帶兒童來看殯。」

正是:

「鑼鼓鼕鼕靄路塵,  花攢錦簇萬人瞻,

哀聲隱隱棺輿過,  此殯誠然壓帝京。」

吳月娘坐大轎在頭里,後面李嬌兒等,本家轎子十余頂,一字兒緊跟在後走。西門慶總冠孝衣,同眾親朋在材後里。陳經濟絮扶棺輿。走出東街口,西門慶具禮請玉皇廟吳道官來懸真,身穿大紅五彩雲霞二十四鶴鶴氅,頭戴九陽玉環雷巾,腳蹬丹舄,手執牙笏。坐在四人肩輿上,迎殯而來。將李瓶兒大影捧于手內,陳經濟跪在面前,那殯停住了,眾人聽他在上高聲宣念:

「兔走烏飛西復東,  百年光景侶風燈,

時人不悟無生理,  到此方知色是空。」

「恭惟

故錦衣西門恭人李氏之靈,存日陽年二十七歲,元命辛未相正月十五日午時受生,大限於政和七年,九月十七日丑時分身故。伏以尊靈,名家秀質,綺閣嬌妹。稟花月之儀容,蘊蕙蘭之佳氣。鬱德柔婉,賦性溫和。配我西君,克諧伉儷處閨門而賢淑,資琴瑟以好和。曾種藍田,尋嗟楚畹。正宜享福百年,可惜春光三九。嗚呼!明月易缺,好物難全。善類無常,修短有數,今則棺輿載道,丹旆迎風,良夫踊於柩前,孝眷哀矜於巷陌。離別情深而難已,音容日遠以日忘。某等謬忝冠簪,愧領玄教。愧無新坦平之神術,恪遵玄元始之遺風。徒展崔徽鏡裡之容,難返莊周夢中之蝶。漱甘露而沃瓊漿,超仙識登於紫府;披百寶而面七真,引淨魄出於冥途。一心無挂,四大皆空。空苦苦,氣化清風形歸土。一靈真性去弗迴,改頭換面無遍數。眾听末後一句咦,精爽不知歸何處,真容留與後人傳。」

吳道官念畢,端坐轎上,那轎捲坐退下去了。這種鼓樂喧天,哀聲動地,殯纔起身,迤〈辶里〉出南門。眾親朋陪西門慶,走至門上,方乘馬。陳經濟扶柩,到于山頭五里原。原來坐營張團練帶領二百名軍,同劉、薛二內相,又早在墳前高阜處,搭帳房,吹响器,打銅鑼銅鼓,迎接殯到。看著裝燒冥器紙劄,烟焰漲天。墳內有十數家收頭祭祀,皆兩院妓擺列。堂客內眷,自有幃幕。棺輿到,落下扛,徐先生率領仵作,依羅經弔向。巳時,祭告后土,方隅後,纔下葬掩土。西門慶易服,備一對尺頭禮,請帥府周守備點主。衛中官員至眾親朋夥計,皆爭拉西門慶祭畢遞酒。鼓樂喧天,烟火匝地。收祭祀者,自有所管,人役再無淆亂。那日待人齋堂,也有四五處。堂客在後捲棚內坐,各有派定人數。熱鬧豐盛,不必細說。吃畢,各有邀占庄院,設席請西門慶收頭飲酒,賞賜亦費許多。後响回靈,吳月娘坐魂轎,抱神主魂旛,陳經濟扶靈牀。都是玄色宁絲靈衣,玉色銷金走水,四角垂流蘇,弔挂大影亭,大絹亭、小絹亭、香燭亭。鼓手細樂,十六眾小道童,兩邊吹打。吳大舅并喬大戶、吳二舅、花大舅。沈姨夫、孟二舅、應伯爵、謝希大、溫秀才眾主管夥計,都陪著西門慶進城。堂客轎子壓後。到家門首,燎火而入。

李瓶兒房中安靈已畢,徐先生前廳祭神酒掃,各門戶皆貼辟非黃符。管待徐先生,備一疋尺頭,五兩銀子,相謝出門。各項人役,打發散了。拿出二十五吊錢來,五吊賞巡捕軍人,五吊與衛中排軍,十吊賞營裡人馬。拿帖兒回謝周守備、張團練、夏提刑,俱不在話下。西門慶還令左右放卓,留喬大戶、吳大舅眾人坐。眾人都不肯,作辭起身。來保回說:「搭棚在外伺侯,明日來拆棚。」西門慶道:「棚且不消拆,亦發過了你宋老爹擺酒日子來拆罷。」打發搭綵匠去了。後邊花大娘子與喬大戶娘子、眾堂客,還等著安畢靈,哭了一場,方纔去了。西門慶不忍遽捨,晚夕還來李瓶兒房中要伴靈宿歇。見靈牀安在正面,大影挂在傍邊。靈牀內安著半身,裡面小錦被褥牀几衣服粧奩之類,無不畢具。下邊放著他的一對小小金蓮,卓上香花燈燭,金碟樽俎,般般供養。西門慶大哭不止,令迎春就在對面炕上搭鋪。到夜半對著孤燈,半窗斜月,翻復無寐,長吁短嘆,思想佳人。有詩為證:

「短嘆長吁對彼窗,  舞鸞孤影寸心傷,

蘭枯楚畹三秋雨,  楓落吳江一夜霜;

夙世已逢連理願,  此生難滅返魂香,

九泉果有精靈在,  地下人間兩斷腸。」

白日間供養茶飯,西門慶在房中親看著丫鬟擺下,他便對面卓兒和他同吃。舉起筯兒來:「你請些飯兒?」行「如在」之禮。丫鬟、養娘都忍不住掩淚而哭。奶孚如意兒,無人處,常在根前遞茶水,挨挨搶搶,搯搯捏捏,插話兒應答。那消三夜兩夜,西門慶因陪人吃得醉了,進來,迎春打發歇下。到夜間要茶吃,叫迎春不應。如意兒起來遞茶,因見被拖下炕來,接過茶盞,用手扶起被。西門慶一時興動,摟過脖子就親了個嘴,遞舌頭在口內。老婆就咂起來,一聲兒不言語。西門慶令脫去衣服上炕,兩個摟接在被窩內,不勝歡娛,雲雨一處。老婆說:「既是爹擡舉,娘也沒了,小媳情願不出爹家門,隨爹收用便了。」西門慶便叫:「我兒,你只用心伏侍我,愁養活不過你來。」當下這老婆枕席之間,無不奉承。顛鸞倒鳳,隨手而轉。把西門慶歡喜要不的。次日,老婆早辰起來,與西門慶拿鞋腳疊被褥,就不靠迎春,極盡慇懃,無所不至。西門慶開門,尋出李瓶兒四根簪兒來賞他。老婆磕頭謝了。迎春亦知收用了他,兩個打成一路。老婆自恃得寵,腳跟已牢,無復求告於人。自從西門慶請了許多官客堂客,并院中李桂姐、吳銀兒、鄭月兒三個唱的,李銘、吳惠、鄭奉、鄭春四名小優兒,墳上暖墓回家。這如意兒就不同往日,打扮喬眉喬樣,在丫鬟夥兒內,說也有,笑也有,早被潘金蓮看到眼裡。

早辰西門慶正陪應伯爵坐的,忽報宋御史老爹差人來送賀黃太尉一卓金銀酒器:兩把金壺,兩副金臺盞,十副小銀鍾,兩副銀拆盂,四副銀賞鍾,兩疋大紅彩蟒,兩疋金段,十罈酒,兩牽羊。傳報:「太尉船隻,已到東昌地方。煩老爹這裡早先預備酒席,准在十八日迎請。」西門慶收入明白,與了來人一兩銀子,打柬,打發回去。隨即兌銀與賁四、來興兒,定卓面,粘菓品,買辦整理,不必細說。因向應伯爵說:「自從他不好起,到而今,我再沒一日兒心閒。剛剛打發喪事兒出去了,又鑽出這等勾當來,教我手忙腳亂。」伯爵道:「這個哥不消抱怨,你又不曾掉攬他,他上門兒來央煩你。雖然你這席酒,替他賠幾面銀子。到明日休說朝廷一位欽差,殿前大太尉來咱家坐一坐,說是山東一省官員,并巡撫巡按人馬散級,也與咱門戶添許多光輝,壓好些仗氣。」西門慶道:「不是此說。我承望他到二十已外也罷,不想十八日就迎接,忒促急促忙。這十六日又是他五七,我前日已與了吳道官寫法銀子去了,如何又改?不然雙頭火杖,都擠在一處,怎亂得過來?」應伯爵道:「這個不打緊,我算來嫂子是九月十七日沒了,此月二十一日正是五七。你十八日擺了酒,二十日與嫂子念經也不遲。」西門慶道:「你說的是了,我如今就使小廝回吳道官改日子去。」伯爵道:「哥我又一件,如今趁著來京黃真人在廟裡住,朝廷差他來泰安州進金鈴弔挂御香,建七晝夜羅天大醮。趁他未起身,倒好教吳道官請他那日來做高功,領行法事。咱圖他這個名聲,也好看。」西門慶道:「自說這黃真人有利益,少不的那日來堂添二十四眾道士,做一晝夜齋事。爭奈吳道官齋日受他祭禮,出殯又起動他懸真,道童送殯。沒的酬謝他,教他念這個經兒表意而已。今又請黃真人主行,都不難為他?」伯爵道:「齋一般還是他受,只教他請黃真人做高功就是了。哥只是多費幾兩銀子,為嫂子,沒曾為了別人。」西門慶一面教陳經濟寫帖子,又多封了五兩銀子寫法,教他早請黃真人,改在二十日念經。二十四眾道士,水火煉度一晝夜。即令玳安騎頭口回去了。

西門慶打發伯爵去訖,進入後邊,只見吳月娘說:「賁四嫂買了兩個盒兒,他女兒長姐定與人家,來磕頭。」西門慶便問:「誰家?」賁四娘子穿著藍絒紬襖兒,白絹裙子,青段披襖;他女兒穿著大紅段襖兒,黃紬裙子,戴著花翠,插燭向西門慶磕了四個頭。月娘在傍說:「咱也不知道。原來這孩子與了夏大人房裡擡舉,昨日纔相定下,這二十四日就要娶過門,只得了他三十兩銀子。論起來這子倒也好身量,不相十五歲,倒有十六七歲的。多少時不見,就長的成成的!」西門慶道:「他前日在酒席上和我說,要抬舉兩個孩子學彈唱。不知你家孩子與了他。」于是教月娘讓在房內,擺茶留坐。落後李嬌兒。孟玉樓、潘金蓮、孫雪娥、大姐卻來見禮陪坐。臨走,西門慶、月娘與了一套重絹衣服,一兩銀子,李嬌兒眾人都有與花翠汗巾脂粉之類。晚上玳安回話:「吳道官收了銀子,知道了。黃真人還在廟裡住,過二十頭纔回東京去,十九日早來鋪設壇場。」

西門慶次日家中廚役落作,治辦酒席,務要齊整。大門上扎七級彩山,廳前五級彩山。十七日宋御史差委兩員縣官來觀看筵席。廳正面屏開孔雀,地匝氍毹。都是錦綉卓幃,粧花椅甸。黃太尉便是肘件大飯簇盤,定勝方糖 ,五老豐堆高頂,吃看大插卓觀席。兩張小插卓,是巡撫、巡按陪坐。兩邊布按三司,有卓席列坐。其餘八府官,都在廳外棚內兩邊,只是五菓五菜平頭卓席。看畢,西門慶待茶,起身回話去了。

到次日,撫按率領多官人馬,早迎到船上,張打黃旗「欽差」二字,捧省勅書,在頭裡走。地方統制、守禦、都監、團練,各衛掌印武官,皆戎服甲冑,各領所部人馬圍隨。藍旗纓鎗,叉槊儀杖,擺數里之遠。黃太尉穿大紅五彩雙挂綉蟒,坐八擡八簇銀頂暖轎,張打茶褐傘。後邊名下執事人後,跟隨無數。皆駿騎咆哮,如萬花之燦錦,隨路鼓吹而行。黃土塾道,雞犬不聞,樵採遁跡。人馬過東平府,進清河縣,縣官黑壓壓跪於道傍迎接,左右喝叱起去。隨路傳報,直到西門慶家中大門首。教坊鼓樂,聲震雲霄。兩邊執事人役,皆青衣排伏,雁翅而列。西門慶青衣冠冕,望塵拱伺。

良久,人馬過盡,太尉落下轎進來。後面撫按率領大小官員,一擁而入,到於廳上,廳上又是箏阮方響,雲璈龍笛鳳管細樂响動。為首就是山東巡撫都御史候濛,巡按監察御史宋喬年參見。太尉還依禮答之。其次就是山東左布政龔夬,左參政何其高、右布政陳四箴、右參政季侃、左參議馮廷鵠、右參議汪伯彥、廉訪使趙訥、採訪使韓文光、提學副使陳正彙、兵備副使雷啟元等兩司官參見,太尉稍加優禮。及至東昌府徐崧、東平府胡師文、
兗 %袞兖兗
州府凌雲翼、徐州府韓邦奇、濟南府張叔夜、青州府王士奇、登州府黃甲、菜州府葉遷等八府官行廳參之禮,太尉答以長揖而已。至於統制、制置、守禦、都監、團練等官,太尉則端坐。各官聽其發放,各人外邊伺候。然後西門慶與夏提刑上來拜見獻茶,侯巡撫、宋巡按身向前把盞。下邊動鼓樂來與太尉簪金花、捧玉斝,彼此酬飲。遞酒已畢,太尉正席坐下,撫按下邊主席,其餘官員并西門慶等,各依次第坐了。教坊伶官,遞上手本奏樂,一應呈應彈唱隊舞四數,各有節次,極盡聲容之盛。當筵搬演的裴晉公還帶記,一摺下去,廚役割獻燒鹿花猪,百寶攢湯 大飯燒賣 。又有四員伶官,箏阮琵琶箜篌上來清彈小唱,唱了一套南呂一枝花:

「官居八輔臣,祿享千鍾近。功存遺百世,名播萬年春。拯溺亨迍,惟治國邦論。調和鼎鼐,持義節,率忠貞,都則待報主施恩;乘賢烈,秉正直,也則是清懲化民。」

唱畢,湯未兩陳,樂已三奏。下邊跟從執事官身人等,宋御史委差兩員州官,在西門慶捲棚內,自有卓席管待。守禦都監等官,西門慶都安在前邊客位,自有坐處。黃太尉令左右拿十兩銀子來賞賜各項人役,隨即看轎,就要起身。眾官上來再三款留不住,都送出大門。鼓樂笙簧迭奏,兩街儀衛喧闐。清蹕傳道,人馬森列。多官俱上馬遠送,太尉悉令免之,舉手上轎而去。宋御史、侯巡撫分付都監以下軍衛有司,直護送至皇船上來回話。卓面器皿,答賀羊酒,具手本差東平府知府胡師文與守禦周秀,親送到船所交割明白。回至廳上,拜謝西門慶說:「今日不當負累取擾華府,深感深感!分資有所不足,容當奉補。」西門慶慌躬身施禮道:「學生屢承教愛,累辱盛儀,日昨又蒙賻禮,些小微物,何足挂齒?蝸居卑陋,猶恐有不到處,萬望公祖諒宥,幸甚!」宋御史謝畢,即令左右看轎,與侯巡撫一同起身。兩司八府官員,皆拜辭而去。各項人役,一鬨而散。西門慶回至廳上,將伶官樂人賞以酒食,俱令散了。止留下四名官身小優兒伺侯。廳內外各官卓面,自有本官手下人領,不題。

西門慶見天色尚早,收拾家火停當,攢下四張卓席,佳餚堆滿,使人請吳大舅、應伯爵、謝希大、溫秀才、傅日新、甘出身、韓道國、賁四、崔本及女婿陳經濟,從五更起來,各項照管辛苦,坐飲三杯。不一時眾人來到,吳大舅與溫秀才、應伯爵、謝希大居上坐,西門慶關席,眾夥計兩邊列坐,左右擺上酒來飲酒。伯爵道:「哥今日落忙,黃太尉坐了多大一回,喜歡不喜歡?」韓道國道:「今日六黃老公公見咱家酒席齊整,無箇不喜歡的。巡撫、巡按兩位,甚是知感不盡,謝了又謝。」伯爵道:「若是第二家擺這席酒,也成不的,也沒咱家恁大地方,也沒府上這些人手。今日少說也有上千人進來,都要管待出去。哥就賠了幾兩銀子,咱山東一省,也响出名去了。」溫秀才道:「學生宗主提學陳老先生,也在這裡預席。」西門慶問其故。溫秀才道:「名陳正彙者,乃諫垣陳了翁老生乃郎,本貫河南鄄城縣人,十八歲科舉,中壬辰進士。今任本處提學琍副使,極有學問。」西門慶道:「他今年纔二十四歲。」正說著,湯飯上來,眾人吃畢。西門慶叫上四個小優兒,問道:「你四人叫甚名字?」答道:「小的叫周采、梁鐸、馬真、韓畢。」伯爵道:「你不是韓金釧兒?」一面韓畢跪下說:「金釧兒。玉釧兒,都是小的妹子。」西門慶問:「你們吃了酒飯不曾?」周采道:「小的剛纔都吃個酒飯了。」西門慶因一回想起李瓶兒來,今日擺酒就不見他,分付小優兒:「你每拿樂器過來,會唱『洛陽花梁園月』不會?唱一個我聽。」韓畢跪下:「小的與周采記的。」一面搊箏撥院板,排紅牙,唱道普天樂:

「洛陽花,梁園月,好花須買,皓月須賒。花倚欄杆看爛熳開,月曾把酒問曾團圓夜。月有盈虧,花有開謝,想人生最苦離別。花謝了,三春近也,月缺了,中秋到也;人去了,何日來也!」

唱畢,應伯爵見西門慶眼裡酸酸的,便道:「哥別人不知你心,自我略知一二。哥教唱此詞,關係心間之事,莫非想起過世嫂子來,就如同連理枝、比目魚,今分為兩下,心中甚不想念!」西門慶看見後邊上來菓碟兒,叫:「應二哥,你只嗔我說。有他在,就是他經手整定;從他沒了,隨著丫鬟掇弄,你看都相甚模樣?好應口菜也沒一根我吃。」溫秀才道:「這等盛設,老先生中饋也不謂無人,足可以勾了。」伯爵道:「哥休說此話,你心間疼不過,便是這等說。恐一時冷淡了別的嫂子們心。」這裡酒席上說話,不想潘金蓮在軟壁後聽唱,聽見西門慶說此話,走到後邊,一五一十告訴月娘。月娘道:「隨他說去就是了,你如今都怎樣的!前日是不是他在時,即許下把綉春教伏侍,他倒睜著眼和我叫,死了許多時兒,就分散他房裡丫頭。教我就一聲兒再沒言語。這兩日你著他那媳婦子和兩個丫頭,狂的有些樣兒!我但開口,就說咱每擠撮他。」金蓮道:「娘,我也見這老婆,這兩日有些別改模樣的。怕這賊沒廉耻貨,鎮日在那裡纏了這老婆也不可知的。我聽見說,前日與了他兩對簪子。老婆戴在頭上,拿與這個瞧,拿與那個瞧。」月娘道:「荳芽菜兒,有甚綑兒!」眾人背地里都不做喜歡。正是:

「遺踪堪入時人眼,  不買胭脂畫牡丹。」

有詩為證:

「襄王臺下水悠悠,  一種相思兩地愁,

月色不知人事改,  夜深還照粉墻頭。」

畢竟不知後來如何,且聽下回分解:
