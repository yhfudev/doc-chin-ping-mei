%# -*- coding: utf-8 -*-
%!TEX encoding = UTF-8 Unicode
%!TEX TS-program = xelatex
% vim:ts=4:sw=4
%
% 以上设定默认使用 XeLaTex 编译,并指定 Unicode 编码,供 TeXShop 自动识别

%第八十三回 
\chapter{秋菊含恨泄幽情\KG 春梅寄柬諧佳會}


\begin{showcontents}{}



「堪笑西門慶識未通,  惹將桃李笑春風,

滿床錦被藏賊睡,   三頓珍羞養大蟲;

愛物只圖夫婦好,   貪財常把丈人坑,

更有一件堪觀處,   穿房入屋弄乾坤。」

話說潘金蓮見陳經濟天明越墻過去了,心中又後悔。次日都是七月十五日,吳月娘坐轎子出門,往地藏庵薛姑子那裡,替西門慶燒蘭盆會箱庫去。金蓮眾人,都送月娘到大門首回來。孟玉樓、孫雪娥、西門大姐都往後邊去了。獨金蓮落後,走到前廳儀門首,撞遇經濟正在李瓶兒那邊樓上,尋了解當庫衣物抱出來。金蓮叫住,便向他說:「昨日我說了你幾句,你如何使性兒,今早就跳博出來了?莫不真個和我罷了?」經濟道:「你老人家還說哩,一夜誰睡著來?險些兒一夜沒曾把我麻犯死了!你看把我臉上肉,也撾的去了!婦人罵道:「賊短命!既不與他有首尾,賊人膽兒虛,你平日走怎的!」經濟向袖中取出了紙帖兒來,婦人打開觀看,都是寄生藥一詞,說道:

「動不動,將人罵,一徑把臉兒上撾。千般做小伏低下,但言語,便要和咱罷!罷字兒,說的人心怕。忘恩失義俏冤家,你眉兒淡了教誰畫?」

金蓮一見,笑了,說道:「既無此事,你今晚來後邊,我慢慢再問你。」經濟道:「乞你麻犯了人,一夜誰合眼兒來!等我白日裏睡一覺兒去。」婦人道:「得不去,和你算帳!」說畢,婦人回房去了。經濟拏衣物舖子裏來,做了一回買賣。歸到廂房,〈扌歪〉在床上,睡了一覺。盼望天色晚來,要往金蓮那邊去。不想比及到黃昏時分,天氣一陣陰黑來,窗外簌簌下起雨來。正是:

「蕭蕭庭院黃昏雨,  點點芭蕉不住聲。」

這經濟見那雨下得緊,說道:「好個不做美的天!他甫能教我對證話去,今日不想又下起雨來,好悶倦人也!」于是長等短等,那雨不住。簌簌直下到初更時分,下的房簷上流水。這小郎君等不的雨住,披着一條茜紅氁子臥單在身上。那時吳月娘來家,大姐與元宵兒都在後邊沒出來。于是鎖了房門,從西角門大雨裡,走入花園金蓮那邊,推了推角門。婦人知他今日晚必來,早已分付春梅,灌了秋菊幾鍾酒,同他在炕房裡先睡了。以此把角門虛掩。這經濟推了推角門,見虛掩着,便挨身而入,進到婦人臥房。見紗窗半啟,銀蠟高燒,卓上酒果已陳,金尊滿泛。兩個並肩疊股而坐。婦人便問:「婦人便問:「你既不曾與孟三兒抅搭,這簪子怎得到你手裡?」經濟道:「本是我昨日在花園荼{艹縻}架下拾的。若哄你,便促死促滅!」婦人道:「既無此事,還把這根簪子與你關頭,我不要你的。只要把我與你的簪子香囊帕兒物事收好着,少了我一件兒,我與你答話!」兩個吃酒下棋,到一更方上床就寢。顛鷥倒鳳,整狂了半夜。婦人把昔日西門慶枕邊風月,一旦盡付與情郎身上。都說秋菊在那邊屋裏,夜聽見這邊房裡,恰似有男子聲音說話,更不知是那個了。到天明雞叫時分,秋菊起來溺尿。忽聽那邊房內開的門响,朦朧月色,雨尚未止。打窗眼看見一人,披着紅臥單,從房中出去了,恰似陳姐夫一般!「原來夜夜和我娘睡!我娘自來人前會撇清,乾淨暗裏養着女婿!」次日,逕走到後邊廚房裡,就如此這般對小玉說。不想小玉和春梅好,又告訴與春梅:「你那邊秋菊說,你娘養着姐夫。昨日在房裡睡了一夜,今早出去了。大姑娘和元宵,又沒在前邊睡。」這春梅歸房,一五一十對婦人說:「娘不打與你這奴才幾下,教他騙口張舌,葬送主子,就是一般!」金蓮遂叫秋菊來,罵道:「我要你教作煎煎粥兒,就把鍋來打破了!你屁股大,吊了心也怎的?我這幾日沒曾打你,這奴才骨朵痒了!」于是拏棍子,向他脊背上儘力狠抽了三十下。打的殺猪也似叫,身上都破了。春梅走將來說:「娘沒的打他這幾下兒,與他撾痒痒兒哩!旋剝了,叫將小廝來,拏大板子,儘力砍與他二三十板,看他怕不怕!湯他這幾下兒,打水不渾的,只像鬬猴兒一般!他好小膽兒,你想他怕也怎的!做奴才,裏言不出,外言不入。都似這般,養出家生哨兒來了!」秋菊道:「誰說甚麼來?」婦人道:「還說嘴哩!賊彼家誤五鬼的奴才,還說甚麼!」幾聲喝的婦人往廚下去了。正是:

「蚊蟲遭扇打,  只為嘴傷人!」

一日八月中秋時分,金蓮夜間暗約經濟賞月飲酒,和春梅同下鱉棋兒。晚夕貪睡失曉,至茶時前後,還未起來,頗露圭角。不想被秋菊朘到眼裡,連忙走到後邊上房門首,對月娘說。不想月娘正梳頭,小玉在上房門,秋菊拉過他一邊,告他說:「俺姐夫如此這般,昨日又在我娘房裡歇了一夜,如今還未起來哩!前日為我告你說,打了我一頓。今日真實看見,我須不賴他。請奶奶快去瞧去。」小玉罵道:「張眼露睛奴才,又來葬送主子!俺奶奶梳頭哩,還不快走哩!」月娘便問:「他說甚麼?」小玉不能隱諱,只說五娘使秋菊來請奶奶說話,更不題出別的事。這月娘梳了頭,輕移蓮步,驀然來到前邊金蓮房門首。早被春梅看見,慌的先進來報與金蓮。金蓮與經濟兩個還在被窩內未起。聽見月娘到,兩個都吃了一驚,慌做手腳不迭。連忙藏經濟在床身子裡,用一床錦被遮蓋的。教春梅放小卓兒,在床上拏過珠花來,且穿珠花。不一時,月娘到到房中坐下,說:「六姐,你這裡咱還不見出門,只道你做甚,原來在屋裡穿珠花哩。」一面拏在手中觀看,誇道:「且是穿得好!正面芝蔴花,兩邊橘子眼方勝兒,周圍蜂趕菊。你看着的珠子,一個挨一個兒,湊的同心結,且是好看!到明日你也替我穿恁條箍兒戴。」婦人見月娘說好話兒,那心頭小鹿兒纔不跳了。一面令春梅倒茶來,與大娘吃。少頃,月娘吃了茶,坐了回去了,說:「六姐快梳了頭,後邊坐。」金蓮道:「知道。」打發月娘出來,連忙攛掇經濟出港,往前邊去了。春梅與婦人整捏兩把汗。婦人說:「你大娘等閒無事,他不來我這屋裏來。無甚事,他今日大清早辰來做甚麼?」春梅道:「左右是咱家這奴才戳的來。」不一時,只見小玉走來,如此這般:「秋菊後邊說去,說姐夫在這屋裡,明睡到夜,夜睡到明日。被我罵喝了他兩聲,他還不動。俺奶奶問,我沒的說,只說五娘請奶奶說話,方纔來了。你老人只放在心裡,大人不見小人過,只隄防着這奴才就是了!」看官聽說:雖是月娘不信秋菊說話,只恐金蓮少女嫩婦,沒了漢子,日久一時心邪,着了道兒,恐傳出去,被外人唇耻。西門慶為人一場,沒了多時光兒,家中婦人都弄的七顛八倒,恰似我養的這孩子,也來路不明一般!香香噴噴在家裡,臭臭烘烘在外頭。又以愛女之故,不教大姐遠出門。把李嬌兒廂房梛與大姐住,教他兩口兒搬進後邊儀門裏來。遇着傅夥計家去,教經濟輪番在舖子裡上宿取衣物藥材,同玳安兒出入。各處門戶都上了鎖鑰。丫鬟婦女無事不許往外邊去。凡是都嚴禁這潘金蓮與經濟兩個熱鬧突突,恩情都間阻了。正是:

「世間好事多間阻,  就裡風光不久長!」

有詩為證:

「幾向天台訪玉真,  三山不見海沉沉;

侯門一日深如海,  從此蕭郎是路人。」

潘金蓮自被秋菊泄露之後,月娘雖不見信,晚夕把各處門戶都上了鎖。西門大姐搬進李嬌兒房中居住經濟尋取藥材衣物,同玳安或平安眼同出入。二人恩情都間阻了,約一個多月,不曾相會一處。金蓮每日難挨綉幃孤枕,怎禁畫閣淒涼?未免害些木邊之目,田下之心,脂粉懶勻,茶飯頓減,帶圍寬腿,懨懨瘦損。每日只是思睡,扶頭不起。有春梅向前道:「娘,你這兩日怎的不去後邊坐?或是往花園中散心走走?每日短嘆長吁,端的為些甚麼?」婦人道:「你不知道我與你姐夫相交?」有鴈兒落為證:

「我與他好似並頭蓮一處生,比目魚纒成塊。初相逢熱似粘,乍怎離別難禁耐。好是怪奇哉這兩日他不進來!大娘又把門上鎖,花園中狗兒乖。難猜,奴婢們【目殳】【目慮】的怪;傷懷,這相思實難解。」

春梅道:「娘,你放心,不妨事。塌了天,還有四個大漢扶着哩!昨日大娘留下兩個姑子,今晚夕宣卷,後邊關的儀門早。晚夕我推往前邊馬坊內取草裝填枕頭,等我往前邊舖子裡叫他去。你寫下個來帖兒,與我拏着。我好歹叫了姐夫,和娘會一面。娘心下如何?」婦人道:「我的好姐姐!你若肯可憐見,叫得他來,我恩有重報,不可有忘!我的病兒好了,替你做雙滿臉花鞋兒!」春梅道:「娘說的是那裡話?你和我是一個人,爹又沒了,你明日往前後進,我情愿跟娘去;咱兩個還在一處。」婦人道:「你有此心,可知好哩!」婦人于是輕拈象管,欵拂花箋,寫就一個柬帖兒,彌封停當。到于晚夕,婦人先在後邊月娘前,假托心中不自在,得了個金蟬脫殼,歸到前邊,房中沒事。月娘後邊儀門老早關了。丫鬟婦女都放出來,聽尼僧宣卷。金蓮央及春梅遞與他柬帖,說道:「好姐姐,你快些請他去!」有河西六娘子為證:

「央及春梅好姐,你放寬洪海量些俺團圓,只在今宵夜。嗏,你把步兒快走些些,我這裡錦被兒重重等待者。」

春梅道:「等我先把秋菊那奴才,與他幾鍾酒灌醉了,倒扣他在廚房內。我方拏了筐,推往前邊馬坊中取草來填枕頭,就叫他來。」于是篩了兩大碗酒,打發秋菊吃的,扣他在廚房內。拏了婦人柬帖兒出門。有鴈兒落為證:

「我與馬坊中,推取草;到前邊,就把他來叫。歸來把狗兒藏,門上將鎖兒套。尊前酒兒篩,床上燈兒罩。帳煖度准備鳳鸞交。休教人知覺,把秋菊灌醉了。春宵,聽著花影動,知他到;今宵,管恁兩個成就了!」

春梅走到前邊,撮了一筐草,到印子舖門首叫門。正值傅夥計不在舖中,往家去了,獨有經濟在炕上,纔〈扌歪〉下。忽見有人叫門,問:「是那個?」春梅道:「是你前世娘,散相思五瘟使!」經濟開門,見是他,滿臉笑道:「原來是小大姐,沒人,請裡面坐。」進入房內,見卓上點着燭,問:「小廝們在那裡?」經濟道:「玳安和平安在那裡生藥舖中睡哩。獨我一個在此受孤恓,挨冷淡,就是小生!」春梅道:「俺娘多上覆,你好人兒,這幾日就門邊兒也不傍,往俺那屋裡走走去!說你另有了對門主顧兒了,不希罕俺娘兒們了!」經濟道:「那裡話!自從那日因些閒話,見大娘緊門緊戶,所以不耐煩走動。」春梅道:「俺娘為你這幾日,心中好生不快!逐日無心無緒,茶飯懶吃,做事沒入腳處。今日大娘留他後邊聽宣卷,也沒去就來了,一心只是牽掛想你。巴巴使我稍寄了一柬帖在此,好歹教你快去哩!」這經濟接柬帖,見封的甚密。拆開觀看,都是寄生草一詞,說道:

「將奴這桃花面,只因你憔瘦損。不是因惜花愛月傷春困,則是因今春不減前春恨!常則是淚珠兒滴盡相思症,恨的是綉幃照影兒孤,盼的是書房人遠天涯近!」

經濟一見了此詞,連忙向春梅躬身,深深地唱諾,說道:「多有起動起動,我並不知他不好,沒曾去看的你娘兒們,休怪!休怪!你且先走一步,我收拾了如今就去。」一面開櫥門,取出一方白綾汗巾,一副銀三事挑牙兒答贈。和春梅兩個摟抱,按在炕上且親嘴咂舌,不勝歡謔。正是:

「無緣得會鶯鶯面,  且把紅娘去解饞!」

有詩為證:

「淡畫眉兒斜插梳,  不欣拈弄綉工夫,

雲窗霧閣深深許,  靜坐芸窗學景書;

多豔麗,更清姝,  神仙標映世間無,

當初只說梅花似,  細看梅花卻不如。」

當下兩相戲了一回,春梅先拏着草歸到房來,一五一十對婦人說:「姐夫我叫了,他便來也!他看了你那柬帖兒,好不喜歡。與我深深作揖,與了我一方汗巾,一副銀挑牙兒相謝。」婦人便叫春梅:「你去外邊看着,只怕他來,休教狗咬。」春梅:「我把狗藏過一邊。」原來那時正值中秋八月十六七,月色正明。且說陳經濟旋那邊生藥舖叫過平安兒來這邊歇。他一個獵古調兒,前邊花園門關了,打後邊角門走入金蓮那邊,搖木槿花為號。春梅隔墻看花稍動,且連忙以咳嗽應之,報婦人。經濟推開門,挨身進入到房中。婦人迎門接着笑語,說道:「好人兒,就不進來走走兒?」經濟道:「彼此怕是非,躲避兩日兒。不知你老人家不快,有失問候!」婦人道:「有四換頭詞為證:

『赤緊的因些閒話,把海樣恩情一旦差。你這兩日門兒不抹,我心兒掛。關情的我兒,你怎生便撇的下!』

兩個坐下,春梅關上角門,房中放卓兒,擺上酒肴。婦人和經濟並肩疊股而坐。春梅打橫,把酒來斟。穿杯換盞,倚翠偎紅,吃了一回。擺下棋子,三人同下鱉棋兒。吃得酒濃上來,婦人嬌眼拖斜,烏雲半軃,取西門慶出淫器包兒,裏面包着相思套、顫聲嬌銀托子、勉鈴、一弄兒淫器,教經濟便在燈光影下。婦人便赤身露體,仰臥在一張醉椅上兒。經濟亦脫的上下沒條絲,也對坐一椅,拏春意二十四解本兒,在燈下着照樣兒行事。婦人便叫春梅:「你在後邊推着你姐夫,只怕他身子乏了。」春梅真個在身後推送,經濟那話插入婦人牝中,往來抽送,十分暢美,不可盡言。都表秋菊在後邊廚下,睡到半夜裡,起來淨手。見房門倒扣着,推不開。于是伸手出來,拔了門弔兒,大月亮地裡躡足潛踪,走到前房窗下,打窗眼裡潤破窗紙,望裡張看兒。房中掌着明晃晃燈燭,三個吃的大醉,都光赤着身子,正做得好。兩個對面坐着椅子,春梅便在後邊推車,三人串作一處。但見:

「一個不顧夫主名分,一個那管上下尊卑。一個氣的吁吁,猶如牛吼柳影;一個嬌聲嚦

嚦,猶似鶯囀花間。一個椅上逞雨雲情,一個耳畔說山盟海誓。一個寡婦房內,翻為快活道場;一個丈母銀前,變作行淫世界。一個把西門慶枕邊風月,盡付與嬌婿,一個將韓壽偷香手段,悉送與情娘。」正是:

「寫成今世不休書,  結下來生歡喜帶!」

當時都被秋菊看到眼裏,口中不說:「還只在人前撇清要打我,今日都真實被我看見了。到明日對大娘說,莫非又說騙張舌賴他不成!」于是瞧了個不亦樂乎,依舊還往廚房中睡去了。三個整狂到三更時分纔睡。春梅未曾天明,先起來。走到廚房,見廚房門開了,便問秋菊。秋菊道:「你還說哩!我尿急了,往那裡溺?我拔門了弔,出來院子裡溺尿來。」春梅道:「成精奴才,屋裏放着榪子溺不是?」秋菊道:「我不知榪子在屋裏。」兩個後邊聒譟。經濟天明起來,早往前邊去了。正是:

「兩手劈開生死路,  翻身跳出是非門。」

婦人便問春梅:「後邊亂甚麼?」這春梅如此這般,告說秋菊夜裡開門一節。婦人發恨要打秋菊。這秋菊早辰,又走來後邊報與月娘知道。被月娘喝了一聲,罵道:「賊葬弄主子的奴才!前日平空走來輕事重報,說他主子窩藏陳姐夫在屋裏,明睡到夜,夜睡到明,叫了我去。他主子正在床上放炕卓兒,穿珠花兒,那得陳姐夫來?落後陳姐夫打前邊來。恁一個弄主子的奴才!一個大人放在屋裡,端的走糖人兒木頭兒,不拘那裡安放了一個漢子,那裡發落付莫〈毛皮〉?放在眼面前不成?傳出去,知道的,是你這奴才們葬送主子;不知道的,只說西門慶平昔要的人強占多了,人死了多少時兒,老婆們一個個都弄的七顛八倒!恰似我的這孩子,也有些甚根兒不正一般!」于是要打秋菊,諕的秋菊往前邊疾走如飛,再不敢來後邊說去了。婦人聽見月娘喝出秋菊,不信其事,心中越發放下膽子來了。于是與經濟作一詞以自快。云紅綉鞋為證:

「會雲雨風般疎 透,閒是非屁似休偢,那怕無縫鎖上十字扭!輪鍬的閃了手腕,散楚的叫破咽喉,咱兩個關心的情越有!」

西門大姐聽見此言,背地裡盤問。陳經濟道:「你信那汗邪了的奴才?我昨日見在舖子上宿,幾時往花園那邊去了?花園門成日又關着。」西門大姐罵:「賊囚根,你別要嘴!你若有風吹草動,到我耳朵內,惹娘說我,你就信信脫脫去了罷,也休想在這屋裡了!」經濟道:「是非終日有,不聽自然無!怪不的說舌的奴才,到明日得了好,大娘眼見不信他。西門大姐道:「得你這般說,就好了。」正是:

「誰料郎心輕似絮,  那知妾意亂如絲!」

畢竟未知後來何如,且聽下回分解:





\end{showcontents}


