%# -*- coding: utf-8 -*-
%!TEX encoding = UTF-8 Unicode
%!TEX TS-program = xelatex
% vim:ts=4:sw=4
%
% 以上设定默认使用 XeLaTex 编译,并指定 Unicode 编码,供 TeXShop 自动识别

%第四十一回 
\chapter{西門慶與喬大戶結親\KG 潘金蓮共李瓶兒鬥氣}

「富貴雙全世業隆,  聯翩朱紫一門中,

官高位重如王導,  家盛財豐北石崇;

畫燭錦幃消夜月,  綺羅紅粉醉春風,

朝懽暮樂年年事,  豈肯潛心任始終。」

話說西門慶在家中,裁縫儹造衣服,那消兩日就完了。到十二日,喬家使人邀請。早辰,西門慶先送了禮去。那日月娘并眾姊妹、大妗子,六頂轎子,一搭兒起身,留下孫雪娥看家。奶子如意兒抱着官哥,又令來興媳婦惠秀,伏侍疊衣服;又是兩頂小轎。西門慶在家,看着賁四叫了花兒匠來,紮縛煙火,在大廳捲棚內掛燈。使小廝拏帖兒,往王皇親宅內,定下戲子,俱不必細說。後响時分,走到金蓮房中,金蓮不在家。春梅在旁伏侍茶飯,放卓兒吃酒。西門慶因對春梅說:「十四日請眾官娘子,你每四個多打扮出去,與你娘跟着遞酒,也是好處。」春梅聽了,斜靠着卓兒說道:「你若叫,只叫他三個出去,我是不出去。」西門慶道:「你怎的不出去?」春梅道:「娘每都新裁了衣裳,陪侍眾官戶娘子,便好看。俺每一個一個,只像燒糊了卷子一般,平白出去惹人家笑話!」西門慶道:「你每多有各人的衣服首飾,珠翠花朵,雲髻兒,穿戴出去。」春梅道:「頭上將就戴着罷了。身上有數那兩件舊片子,怎麼好穿?少去見人的,倒沒的羞刺刺的!」西門慶笑道:「我曉的你這小油嘴兒,你娘每做了衣裳,都使性兒起來。不打緊,叫趙裁來,連大姐帶你四個,每人都替你裁三件。一套段子衣裳,一件遍地錦比甲。」春梅道:「我不比與他。我還問你要件白綾裙兒,搭襯着大紅遍地錦比甲兒穿。」西門慶道:「你要不打緊,少不的也與你大姐裁一件。」春梅道:「大姑娘有一件罷了,我卻沒有,他也說不的。」西門慶于是拏鑰匙開樓門,揀了五套段子衣服,兩套遍地金比甲兒,一疋白綾,裁了兩件白綾對衿襖兒。惟大姐和春梅是大紅遍地錦比甲兒,迎春、玉簫、蘭香都是藍綠顏色衣服,都是大紅段子織金對衿襖,翠藍邊拖裙,共十七件。一面叫了趙裁來,都裁剪停當。又要一疋黃紗做裙腰,貼裡一色多是杭州絹兒。春梅方纔喜歡了,陪侍西門慶在屋裡吃了一日酒。按下家中不題。且說吳月娘眾姊妹到了喬大戶家,原來喬大戶娘子,那日請了尚舉人娘子,并左鄰朱臺官娘子、崔親家母,并兩個外甥姪女兒,段大姐及吳舜臣媳婦兒鄭三姐,叫了兩個妓女,席前彈唱。聽見月娘眾姊妹和吳大妗子到了,連忙出儀門首迎接,後廳敘禮。趕着月娘呼姑娘,李嬌兒眾人,都排行叫二姑娘、三姑娘,稱着吳大妗子那邊稱呼之禮。也與尚舉人朱堂官娘子,敘禮畢。段大姐、鄭三姐向前拜見了,各依次坐下。丫鬟遞過了茶,喬大戶出來拜見,謝了禮。他娘子讓進眾人房中去寬衣服,就放卓兒擺茶。無非是蒸細巧茶食,菓餡點心,酥菓甜食,諸般菓蔬,擺設甚是齊整,請堂客坐下吃茶。奶子如意和惠秀在房中等着看官哥兒,另自管待。須臾,吃了茶,到廳,屏開孔雀,褥隱芙蓉,正面設四張卓席,讓月娘坐了首位;其次就是尚舉人娘子、吳大妗子、朱堂官娘子、李嬌兒、孟玉樓、潘金蓮、李瓶兒、喬大戶娘子關席。坐位傍邊放一卓,是段大姐、鄭三姐、共十一位。尚家兩個妓女,在旁彈唱。上了湯飯,廚役上來獻了頭一道水晶鵝 ,月娘賞了二錢銀子。第二道是頓爛烤蹄兒,月娘又賞了一錢銀子。第三道獻燒鴨 ,月娘又賞了一錢銀子。喬大戶娘子下來遞酒,遞了月娘,過去又遞尚舉人娘子。月娘就下來,往後房換衣服勻臉去了。孟玉樓也跟下來。到了喬大戶娘子臥房中,只見奶子如意兒看守着官哥兒,在炕上鋪着小褥子兒躺着。他家新生的長姐也在傍邊臥着。兩個你打我下兒,我打你下兒頑耍。把月娘、玉樓見了喜歡的要不得,說道:「他兩個倒好相兩口兒。」只見吳大妗子進來。說道:「大妗子,你來瞧瞧,兩個倒相小兩口兒。」大妗子笑道:「正是,孩兒每在炕上張手兒蹬腳兒的,你打我,我打你,小姻緣一對兒耍子。」喬大戶娘子和眾堂客多進房來。吳妗子如此這般說。喬大戶娘子道:「列位親家聽着,小家兒人家,怎敢攀的我這大姑娘府上?」月娘道:「親家好說,我家嫂子是何人?鄭三姐是何人?我與你愛親做親,就是我家小兒,也玷辱不了你家小姐,如何卻說此話?」玉樓推着李瓶兒,說道:「李大姐,你怎的說?」那李瓶兒只是笑。吳妗子道:「喬親家不依,我就惱了。」尚舉人娘子和朱堂官娘子皆說道:「難為吳親家厚情,喬親家你休謙辭了。」因問:「你家長姐去十一月生的?」月娘道:「我家小兒六月廿三日生的,原大五個月,正是兩口兒。」眾人于是不由分說,把喬大戶娘子和月娘、李瓶兒拉到前廳,兩個就割了衫襟。兩個妓女彈唱着,旋對喬大戶說了,拏出菓盒、三段紅,來遞酒。月娘一面分付玳安、琴童快往家中對西門慶說。旋抬了兩罈酒,三疋段子,經綠板兒絨,金絲花,四個螺甸大菓盒,兩家席前掛紅吃酒。一面堂中畫燭高檠,花燈燦爛,麝香靉靉,喜笑匆匆。席前兩個妓女,啟朱唇,露皓齒,輕撥玉阮,斜把琵琶,唱一套鬥鵪鶉:

「翡翠窸紗,鴛鴦碧瓦,孔雀銀屏,芙蓉綉榻,幕捲輕綃,香焚睡。鴨燈上下,下這的是南省尚書,東床駙馬。」

〔紫花兒序〕「帳前軍,朱衣畫戟;門下士,錦帶吳鈎;坐上客;繡帽宮花。按教坊歌舞,依內苑奢華。板撥紅牙,一派簫韶准備,下立兩個美人如畫,粉面銀箏,玉手琵琶。」

〔金蕉葉〕「我倒見銀燭明燒絳蠟,纖手高擎著玉斝。我見他舉止處,堂堂俊雅,我去那燈影兒下,孜孜的覷著。」

〔調笑令〕「這生那里每曾見他,莫不我眼睛花?呀!我這裡手抵著牙兒事記咱,不由我眼兒裡見了他,心牽掛。莫不是五百年前,歡喜寃家?是何處綠楊曾繫馬。莫不是夢兒中雲雨巫峽?」

〔小桃紅〕「玉簫吹徹碧桃花,一刻千金價。燈影兒裡斜將眼稍兒抹,諕的我臉紅霞。酒盃中嫌殺春風凹,玉簫年當二人,未曾抬嫁,俺相公培養出牡丹芽。」

〔三鬼台〕「他說幾句淒涼話,我淚不住行兒般下。鎖不住心猿意馬,我是個嬌滴滴洛陽花,險些露出風流的話靶。這言詞道要不是要,這公事道假不是假。他那裡拔樹尋根,我這裡指鹿道馬。」

〔禿廝兒〕「我勸他似水底納瓜,他覷我似鏡裡觀花。更做道書生自來情性,要調戲咱好人家嬌娃。」

〔聖藥王〕「你看我怎救他,難按納。公孫弘東閣鬧諠譁,散了玳瑁筵,漾了這鸚鵡斝。踢番了銀燭絳籠紗,扯三尺劍離匣。」

〔尾聲〕「從來這秀才每色膽天來大,把俺這小膽文君諕殺。忒火性卓王孫,強風情漢司馬。」

當下眾堂客,與吳月娘、喬大戶娘子、李瓶兒三人,都籫了花,掛了紅,遞了酒,各人都拜了,從新復安席坐下飲酒。廚子上了一道果菓餡壽字雪花糕 ,喜重重滿池嬌並頭蓮湯 ,割了一道燒花豬肉 。月娘坐在上席,滿心歡喜。叫玳安過來,賞一疋大紅與廚役;兩個妓女,每人都是一疋。俱磕頭謝了。喬大戶娘子還不放起身,還在後堂留坐,擺了許多勸碟細菓攆盒。約吃到一更時分,月娘等方纔拜辭回家。說道:「親家,明日好歹下降寒舍,那裡久坐坐。」喬大戶娘子道:「親家盛情,家老兒說來,只怕席間不好坐的,改日望親家去罷。」月娘道:「好親家,再沒人,親家只是見外。」因留了大妗子:「你今日不去,明日同喬親家一搭兒裡來罷。」大妗子道:「喬親家,別的日子你不去罷。到十五日,你正親家生日,你莫不也不去?」喬大戶娘子道:「親家十五日好明日子,我怎敢不去?」月娘道:「親家若不去,大妗子,我交付與你,只在你身上。」于是生死把大妗子留下了,然後作辭上轎。頭裡兩個排軍,打着兩大紅燈籠。後邊又是兩個小廝,打着兩個燈籠,喝的路走。吳月娘在頭裡,李嬌兒、孟玉樓、潘金蓮、李瓶兒,一字在中間,如意兒和惠秀隨後。奶子轎子裏,用紅綾小被把官哥兒裹得沒沒的,恐怕冷,腳下還蹬着銅火爐兒。兩邊小廝圜隨,到了家門首下轎。西門慶正在上房吃酒。月娘等眾人進來,道了萬福,坐下。眾丫鬟都來磕了頭。月娘先把今日酒席上結親之話告訴了一遍。西門慶聽了,道:「今日酒席上,有那幾位堂客?」月娘道:「有尚舉人娘子、朱序班娘子、崔親家母兩個姪女。」西門慶說:「做親也罷了,只是有些不搬陪。」月娘道:「倒是俺嫂子見他家新養的姐,和咱孩子在床炕上睡着,都蓋着那被窩兒,你打我一下兒,我打你一下兒,恰是小兩口兒一般。纔叫了俺每去,說將起來,酒席上就不因不由做了這門親。我方纔使小廝來對你說,抬送了花紅菓盒去。」西門慶道:「既做親也罷了,只是有些不搬陪些。喬家雖如今有這個家事,他只是個縣中大戶,白衣人。你我如今見居着這官,又在衙門中管着事。到明日會親酒席間,他戴着小帽,與俺這官戶怎生相處?甚不雅相!就前日荊南岡央及營里張親家,再三趕着和我做親,說他家小姐今纔五個月兒,也和咱家孩子同歲。我嫌他沒娘母子,也是房裡生的,所以沒曾應承他。不想倒與他家做了親。」潘金蓮在旁接過來道:「嫌人家是房裡養的,誰家是房外養的?就是今日喬家這孩子也是房裡生的。正是險道神撞見那壽星老兒,你也休說我的長,我也休嫌你那短。」這西門慶聽了此言,心中大怒。罵道:「賊淫婦,還不過去!人這裡說話,也插嘴插舌的,有你什麼說處!」金蓮把臉羞的通紅了,抽身走出來,說道:「誰這裡說我有說處?可知我沒說處哩!」看官聽說:今日潘金蓮在酒席上,見月娘與喬大戶家做了親,李瓶兒都披紅籫花遞酒,心中甚是氣不憤。來家又被西門慶罵了這兩句,越發急了。走到月娘這邊屋裡哭去了。西門慶因問:「大妗子怎的不來?」月娘道:「喬親家母明日見有他眾官娘子,說不得來。我留下來他在那裡,教明日同他一搭兒裡來。」西門慶道:「我說自這席間坐次上,也不好相處的。到明日怎麼廝會?」說了回話,只見孟玉樓也走過這邊屋裡來,見金蓮哭泣,說道:「你只顧惱怎的?隨他說了幾句罷了!」金蓮道:「早是你在旁邊聽着,我說他什麼歹話來?又是一說,他說別家是房裡養的,我說喬家是房外養的?也是房裡生的。那個紙包兒包着,瞞得過人?賊不逢好死的強人,就睜着眼罵起我來。罵的人那絕情絕義,我怎來的沒我說處?改變了心,教他明日現報了我的眼!我不說的,喬小妗子出來,還有喬老頭子的些氣兒。你家的失迷了家鄉,還不知是誰家的種兒哩!人便圖往扳親家耍子兒,教他人拏我惹氣罵我,管我〈毛皮〉事!多大的孩子,又是我一個懷抱了尿泡種子,平白子扳親家。有錢沒處施展的,爭破臥單沒的蓋;狗咬尿胞,空喜歡。如今做濕親家還好,到明日休要做了乾親家纔難。吹殺燈擠眼兒,後來的事,看不見的勾當,做親時人家好,過後三年五載方了的,纔一個兒!」玉樓道:「如今人也賊了,不幹這個營生。論起來,也還早哩。纔養的孩子,割什麼衫襟?無過只是圖往來,扳陪着耍子兒罷了!」金蓮道:「你的便浪〈扌扉〉着圖扳親家耍子,平白教賊不合鈕的強人罵我!我養蝦蟆得水蠱兒病,着什麼來由來?」玉樓道:「誰教你說話不着個領頂兒就說出來。他不罵你罵狗?」金蓮道:「我不好說的。他不是房裡,是大老婆?就是喬家孩子,是房裡生的,還有喬老頭子的些氣兒。你家失迷家鄉,還不知是誰家的種兒哩!」玉樓聽了,一聲兒沒言語。坐了一回,金蓮歸房去了。李瓶兒見西門慶出來了,從新花枝招颺,與月娘磕頭,說道:「今日孩子的事,累姐姐費心。」那月娘笑嘻嘻,也倒身還下禮去,說道:「你喜呀。」李瓶兒道:「與姐姐同喜。」磕畢頭起來,與月娘、李嬌兒,坐着說話。只見孫雪娥、大姐來與月娘磕頭,與李嬌兒、李瓶兒道了萬福。小玉拿將茶。正吃茶,只見李瓶兒房裡丫鬟綉春來請,說:「哥兒屋裡尋哩,爹使我請娘來了。」李瓶兒道:「奶子慌的三不知就抱的屋裡去了。一搭兒去也罷了,是孩子沒個燈兒。」月娘道:「頭裡進門,我教他抱的房裡去,恐怕晚了。」小玉道:「頭裡如意兒抱着他,來安兒打着燈籠送他來。」李瓶兒道:「這等也罷了。」于是作辭月娘,回房中來。只見西門慶在屋裡,官哥兒在奶子懷裡睡着。因說:「是你如何不對我說,就抱了他來?」如意兒道:「大娘見來安兒打着燈籠,就趁着燈兒來了。哥哥哭了一回,纔拍着他睡着了。」西門慶道:「他尋了這一回,纔睡了。」李瓶兒說畢,望着他笑嘻嘻說道:「今日與孩子定了親,累你。我替你磕個頭兒。」于是插燭也似磕下去。喜歡的西門慶滿面堆笑,連忙拉起來做一處坐的。一面令迎春擺上酒兒,兩個這屋裡吃酒。且說潘金蓮到房中,使性子,沒好氣。明知西門慶在李瓶兒這邊,一經因秋菊開的門遲了,進門就打兩個耳刮子。高聲罵道:「賊淫婦奴才,怎的叫了恁一日不開?你做什麼來摺兒。我且不和你答話。」于是走到屋裡坐下。春梅走來磕頭遞茶。婦人問他:「賊奴才,他在屋裡做什麼來?」春梅道:「在院子裡坐着來。他叫了我那等推他還不理。」婦人道:「我知道他和我兩個毆業,黨太尉吃匾食,他也學人照樣兒行事,欺負我!」待要打他,又恐西門慶在那屋裡聽見;不言語,心中又氣。一面卸了濃粧,春梅與他搭了鋪,上床就睡了。到次日,西門慶衙門中去了。婦人把秋菊教他頂着大塊柱石跪在院子裡。跪的他梳了頭,教春梅扯了他褲子,拏大板子要打他。那春梅道:「好乾淨的奴才,教我扯褲子,倒沒的污濁了我的手!」走到前邊,旋叫了畫童兒小廝,扯去秋菊底衣。婦人打着他罵道:「賊奴才淫婦,你從幾時就恁大來!別人興你,我卻不興你。姐姐,你知我見的,將就膿着些兒罷了。平白撑着頭兒,逞什麼強!姐姐,你休要倚着。我到明日洗着兩個眼兒,看着你哩!」一面罵着又打,打了大罵。打的秋菊殺豬也似叫。李瓶兒那邊纔起來,正看着奶子官哥兒打發睡着了,又諕醒了。明明白白聽見金蓮這邊打丫鬟,罵的言語兒妨頭,聞一聲兒不言語,諕的只把官哥兒耳朵握着。一面使繡春:「去對你五娘說,休打秋菊罷。哥兒纔吃了些奶睡着了。」金蓮聽了,越發打的秋菊狠了。罵道:「賊奴才!你身上打着一萬把刀子,這等叫饒!我是恁性兒,你越叫,我越打!莫不為你,拉斷了路行人?人家打丫頭,也來看着?你好姐姐對漢子說,把我別變了罷!」李瓶兒這邊分明聽見指罵的是他,把兩隻手氣的冷,忍氣吞聲,敢怒而不敢言。早辰茶水也沒吃,摟着官哥兒在炕上就睡着了。等到西門慶衙門中回家,入房來看官哥兒。見李瓶兒哭的眼紅紅的,睡在炕上,問道:「你怎的這咱還不梳頭收拾?上房請你說話。你怎猱的眼恁紅紅的?」李瓶兒也不題金蓮那邊指罵之事,只說我心中不自在。西門慶告說:「喬親家那裡送你的生日禮來了。一疋尺頭,兩壜南酒 ,一盤壽桃,一盤壽麵,四樣嗄飯;又是哥兒近節的兩盤元宵 ,四盤蜜食,四盤細菓,兩掛珠子吊燈,兩座羊皮屏風燈,兩疋大紅官段,一頂青段〈扌寨〉的金八吉祥帽兒,兩雙男鞋,六雙女鞋。咱家倒還沒往他那裡去,他又早與咱孩兒近節來了。如今上房的請你計較去。只他那裡使了個孔嫂兒和喬通押了禮來。大妗子先來了,說明日喬親家母不得來,直到後日纔來。他家有一門子做皇親的喬五太太,聽見和咱門做親,好不喜歡,到十五日也要來走走。咱少不得補個帖兒請去。」李瓶兒聽了,方慢慢起來梳頭。走到後邊,拜了大妗子。孔嫂兒正在月娘房裡待茶,禮物都擺明間內,都看了。一面打發回盒起身,與了孔嫂兒、喬通每人兩方手帕,五錢銀子,寫了回帖。又差人補請帖,送與喬太太去了。正是:

「但將鐘鼓悅和愛,   好把犬羊為國羞。」

有詩為證:

「西門獨富太驕矜,   襁褓孩童結做親;

不獨資財如糞土,   也應嗟歎後來人。」

畢竟未知後來如何,且聽下回分解:
