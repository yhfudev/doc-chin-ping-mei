%# -*- coding: utf-8 -*-
%!TEX encoding = UTF-8 Unicode
%!TEX TS-program = xelatex
% vim:ts=4:sw=4
%
% 以上设定默认使用 XeLaTex 编译,并指定 Unicode 编码,供 TeXShop 自动识别

%第五十八回 
\chapter{懷妒忌金蓮打秋菊\KG 乞臘肉磨鏡叟訴冤}


「綉幃寂寂思懨懨,  萬種新愁日夜添,

一雁叫群秋度塞,  亂蛩吟苦月當簷;

藍橋失路悲紅線,  金屋無人下翠簾,

何似湘江江上竹,  至今猶被淚痕沾。」

話說當日西門慶前廳陪親朋飲酒,吃的酩酊大醉,走入後邊孫雪娥房裡來。雪娥正顧灶上看收拾家火。聽見西門慶往後邊去,慌的兩步做一步走。先前郁大姐正在他炕上坐的,一面攛掇他往月娘炕屋裡和玉簫、小玉一處睡去了。原來孫雪娥在後邊,也住著一明兩暗三間房,一間床房,一間炕房。西門慶也有一年多沒進他房中來。聽見今日進來,連忙向前替西門慶接了衣服,安頓中間椅子上坐的。一面在房中揩抹涼蓆,收拾床舖,薰香澡牝。走來遞茶與西門慶吃了,攙扶進房中,上床脫靴解帶,打發安歇;一宿無話。到次日廿八,乃西門慶正生日。剛燒畢紙,只見韓道國後生胡秀到了門首下頭口,左右稟報與西門慶。西門慶叫胡秀到廳上,磕頭見了,問他:「貨船在那裡?」這胡秀遞上書帳,悉把韓大叔在杭州置了一萬兩銀子段絹貨物,見今直抵臨清鈔關,缺少稅鈔銀兩。方纔納稅起腳,裝載進城。這西門慶一面看了書帳,心中大喜。分付棋童看飯與胡秀吃了,教他往喬親家爹那裡見見去。不一時,胡秀吃畢飯去了。西門慶進來對吳月娘說:「如此這般,韓夥計貨船到了臨清,使了後生胡秀送書帳上來。如今少不的把對門房子打掃,卸到那裡,尋夥計收拾,裝廂土庫,開舖子發賣。」月娘聽了,便說:「你上緊尋著。也不早了,還要慢慢的。」西門慶道:「如今等應二哥來,我就對他說,教他上緊尋覓。」時應伯爵來了。西門慶在廳上陪著他坐,對他說:「韓夥計杭州貨船到了,缺少個夥計發賣。」伯爵就說:「哥,恭喜!今日華誕的日子貨船到,決增十倍之利,喜上加喜。哥若尋賣手,不打緊,我有一相識,卻是父交子往的朋友,原是這段子行賣手,連年運拙,閒在家中。今年纔四十多歲,正是當年漢子。眼力看銀水是不消說,寫算皆精;又會做買賣。此人姓甘,名潤,字出身,見在石橋兒巷住,倒是自己房兒。」西門慶道:「若好,你明日請他見我。」正說著,只見李銘、吳惠、鄭奉三個先來,扒在地下磕頭,起來旁邊站立。不一時,雜耍樂工都到了,廂房中打發吃飯。就把桌子擺下,與李銘、吳惠、鄭奉三個同吃。只見答應的節級,拏票來回話:「小的叫了唱的,止有鄭愛月兒不到。他家鴇子說,收拾了纔待來,被王皇親家人攔的往宅裡唱去了。小的只叫了齊香兒、董嬌兒、洪四兒三個,收拾了便來也。」西門慶聽見他不來,便道:「胡說,怎的不來?」便叫過鄭奉問:「怎的你妹子我這裡叫他不來?果係是被王皇親家攔了去?」那鄭奉跪下便道:「小的另住,不知道。」西門慶道:「你說往王皇親家唱就罷了,敢量我就拏不得來?」便叫玳安兒近前分付:「你多帶兩個排軍,就拏我個侍生帖兒,到王皇親家宅內,見你王二老爹,就說是我這裡請幾位人吃酒,這鄭月兒答應下兩三日了,好歹放了他來。倘若推辭,連那鴇子都與我鎖了墩在門房兒裡。這等可惡,叫不得來就罷了!」一面叫鄭奉:「你也跟了去。」那鄭奉又不敢不去。走出外邊來,央及玳安兒說道:「安哥,你進去,我在外邊等著罷。一定是王二老爹府裡叫,怕不的還沒收拾去哩。有累安哥,若是沒動身,看怎的將就,教他好好的來罷。」玳安道:「若果然往王家宅裡去了,等我拏帖兒討去。若是在家藏著,你進去對他媽說,教他快收拾一答兒來。俺就與你替他回護兩句言語兒,爹就罷了。你每不知道性格,他從夏老爹宅定下,你不來,他可知惱了哩。」這鄭奉一面先往家中說去了。玳安同兩個排軍,一名節級,後邊去著。且說西門慶打發玳安、鄭奉去了,因向伯爵道:「這個小淫婦兒,這等可惡!在別人家唱,我這裡叫他不來。」伯爵道:「小行貨子,他曉的甚麼?他還不知你的手段哩。」西門慶道:「我倒見他酒席上說話兒伶俐,叫他來唱兩日試他,倒這等可惡!」伯爵道:「哥今日揀的這四個粉頭,都是出類拔萃的尖兒了。再無有出在他上的了。」李銘道:「你沒見愛香兒的。」伯爵道:「我跟你爹在他家吃酒,他還小哩。這幾年倒沒曾見,不知出落的怎樣的了?」李銘道:「這小粉頭子,雖做好個身段兒,光是一味粧飾。唱曲也會,怎生趕的上桂姐的一半兒唱?爹這裡是那裡,叫著敢不來?就是來了,虧了你,還是不知輕重。」只見胡秀來回話:「小的到喬爹那邊見了來了,伺候老爺示下。」西門慶教陳經濟:「後邊討五十兩銀子來。」令書童:「寫一封書,使了印色,差一名節級,明日早起身,一同去下與你鈔關上錢老爹,教他過稅之時,青目一二。」須臾,陳經濟取了一封銀子來交與胡秀。胡秀稟道:「小的往韓大叔家歇去。」便領文書并稅帖,次日早同起身,不在話下。忽聽喝的道子响,平安來報:「劉公公與薛公公來了。」西門慶即冠帶迎接至大廳,見畢禮數,請至捲棚內,寬去上蓋蟒衣,上面設兩張校椅坐下。應伯爵在下,與西門慶關席陪坐。薛內相便問:「此位是何人?」西門慶道:「去年老太監會過來,乃是學生故友應二哥。」薛內相道:「卻是那快耎笑的應先兒麼?」那應伯爵欠身道:「老公公還記的,就是在下。」須臾,拿茶上來吃了。只見平安走來稟道:「府裡周爺差人拏帖兒來,說今日還有一席,來遲些。教老爹這裡先坐,不須等罷。」西門慶看了帖兒,便說:「我知道了。」薛內相因問:「西門大人,今日誰來遲?」西門慶道:「周南軒那邊還有一席,使人來說,上坐休等他哩,只怕來遲些。」薛內相道:「既來說,咱虛著他席面就是。」上面只見兩個小廝上來,一邊一個打扇。正說話之間,王經拏了兩個帖兒進來:「兩位秀才來了。」西門慶見帖兒上一個是侍生倪鵬、一個溫必古。西門慶就知倪秀才舉薦了他同窗朋友來了,連忙出來迎接。見都穿衣巾著進來,且不著倪秀才,觀看那溫必古,年紀不上四旬,生的明眸皓齒,三牙鬚;丰姿洒落,舉止飄逸。未知行藏何如,見觀動靜若是。有幾句道得他好:

「雖抱不羈之才,慣遊非禮之地。功名蹭蹬,豪傑之志已灰;家業凋零,浩然之氣先喪。把文章道學,一併送還了孔夫子。將致君澤民的事業,及榮華顯親的心念,都撇在東洋大海。和光混俗,惟其利欲是前;隨方逐圓,不以廉恥為重。峨其冠,博其帶,而眼底旁若無人;席上闊其論,高其談,而胸中實無一物。三年叫案,而小考尚難,豈望月桂之高攀?廣坐啣盃,遯世無悶,且作岩穴之隱相。」

西門慶讓至廳上敘禮。每人遞書帕二事,與西門慶祝壽。交拜畢,分賓主而坐。西門慶問道:「久仰溫老先生大才,敢問尊號?」溫秀才道:「學生賤名必古,字日新,號葵軒。」西門慶道:「葵軒老先生。」又問:「貴庠?魁經?」溫秀才道:「學生不才,府學備數,初學易經。一向久仰尊府大名,未敢進拜。昨因我這敝同窗倪桂岩道及老先生盛德,敢來登堂恭謁。」西門慶道:「不敢。承老先生先施,學生容日奉拜。只因學生一個武官,粗俗不知文理,往來書柬,無人代筆。前者因在我這敝同僚府上,會遇桂岩老先生,甚是稱道老先生大才盛德。正欲趨拜請教,不意老先生下降,兼承厚貺,感激不盡。」溫秀才道:「學生匪才薄德,繆承過譽。」茶罷,西門慶讓至捲棚內,有薛、劉二老太監在座。薛內相道:「請二位老先生寬衣進來。」西門慶一面請寬了青衣,進裡面各遜讓再四,方纔一邊一位,垂首坐下。正敘談間,吳大舅、范千戶到了,敘禮坐定。不一時,玳安與同答應的和鄭奉都來回話:「四個唱的,都叫來了。」西門慶問:「是王皇親那裡不在?」玳安道:「是王皇親宅內叫。還沒起身,小的要拴他鴇子墩鎖,他慌了,纔上轎都一答兒來了。」西門慶即出來,到廳臺基上站立。只見四個唱的一齊進來,向西門慶花枝颭招,綉帶飄飄,都插燭也似磕下頭去。那鄭愛月兒穿著紫紗衫兒,白紗挑線裙子,頭上鳳釵半卸,寶髻玲瓏,腰肢嬝娜,猶如楊柳輕盈;花貌娉婷,好似芙蓉豔麗。正是:

「萬種風流無處買,  千金良夜實難消。」

西門慶便向鄭愛月兒道:「我叫你,如何不來?這等可惡,敢量我拏不得你來!」那鄭愛月兒磕了頭起來,一聲兒也不言語,笑著同眾人一直往後邊去了。到後邊與月娘眾人都磕了頭。看見李桂姐、吳銀兒都在跟前,各道了萬福,說道:「你二位來的早。」李桂姐道:「俺每兩日沒家去了。」因說:「你四個怎的這咱纔來?」董嬌兒道:「都是月姐帶累的俺每來遲了!收拾下,只顧等著他,白不起身。」那鄭愛月兒用扇兒遮著臉兒,只是笑,不做聲。月娘便問:「這位大姐是誰家的?」董嬌兒道:「娘不知道,他是鄭愛香兒的妹子鄭愛月兒,纔成人還不上半年光景。」月娘道:「可倒好個身段兒。」說畢,看茶吃了。一面放卓兒擺茶,與眾人吃。那潘金蓮且只顧揭他裙子,撮弄他的腳看,說道:「你每這裡邊的樣子,只是恁直尖了。不相俺外邊的樣子趫。俺外邊尖底停勻,你裡邊的後跟子大。」月娘向大妗子道:「偏他恁好百勝,問他怎的?」一面又取下他頭上金魚撇扙兒來瞧,因問:「你這樣兒是那裡打的?」鄭愛月兒道:「是俺裡邊銀匠打的。」須臾擺下茶,月娘便叫:「桂姐、銀姐,你陪他四個吃茶。」不一時,六個唱的做一處,同吃了茶。李桂姐、吳銀兒便向董嬌兒四個說:「你每來花園裡走走。」董嬌兒道:「等我每到後邊就來。」這李桂姐和吳銀兒就跟著潘金蓮、孟玉樓出儀門往花園中來。因有人在大捲棚內,就不曾過那邊去。只在這邊看了回花草,就往李瓶兒房裡看官哥兒。官哥心中又有些不自在,睡夢中驚哭,吃不下奶去。李瓶兒在屋裡守著,不出來。看見李桂姐、吳銀兒和孟玉樓、潘金蓮進來,連忙讓坐的。桂姐問道:「哥兒睡哩?」李瓶兒道:「他哭了這一日,我打發他面朝裡床纔睡下了。」玉樓道:「大娘說請劉婆子來看他看,你怎的不使小廝快請去?李瓶兒道:「今日他爹好的日子,明日請他去罷。」正說話中間,只見四個唱的和西門大姐、小玉走來。大姐道:「原來你每都在這裡,卻教俺花園內尋你。」玉樓道:「花園內有人在那裡,咱每不好去的。瞧了瞧兒就來了。」李桂姐問洪四兒:「你每四個在後做甚麼?這半日纔來?」洪四兒道:「俺每在後邊四娘房裡吃茶來,坐了這一回。」潘金蓮聽了,望著玉樓、李瓶兒,笑問洪四兒:「誰對你說是四娘來?」董嬌兒道:「他留俺每在房裡吃茶來,他每問來:『還不曾與你老人家磕頭,不知娘是幾娘?』他便說:『我是你四娘哩。』」金蓮道:「沒廉恥的小婦人,別人稱道你便好,誰家自己稱是四娘來?這一家大小,誰興你?誰數你?誰叫你是四娘?漢子在屋裡睡了一夜兒,得了些顏色兒,就開起染房來了。若不是大娘房裡有他大妗子,他二娘房裡有桂姐,你房裡有楊姑奶奶,李大姐便有銀姐在這裡,我那屋裡有他潘姥姥,且輪不到往你那屋裡去哩。」玉樓道:「你還沒曾見哩,今日早晨起來,打發他爹往前邊去了。在院子裡呼張喚李的,便那等花哨起來!」金蓮道:「常言道:『奴才不可逞,小孩兒不宜哄。』又問小玉:「我聽見你爹對你奶奶說,替他尋丫頭子與他。爹昨日到他屋裡,見他只顧收拾不見。問他到底是那小淫婦做勢兒,對你爹說:『我白日不得個閑,收拾屋裡,只好晚夕來這屋裡睡罷了。』你爹說:『不打緊,到明日對你娘說,尋一個丫頭子與你使便了。』真個有此話?」小玉道:「我不曉的,敢是玉簫他聽見來?」金蓮向桂姐道:「你爹不是俺各房裡有人,等閒不往他後邊去。莫不俺每背地說他,本等他嘴頭子不達時務,慣傷犯人。俺每急切不和他說話。」正說著,綉春拿了茶上來,每人一盞果仁泡茶 。正吃間,忽聽前邊鼓樂响動,荊都監眾人都到齊了,遞酒上坐。玳安兒來叫,四個唱的就往前邊去了。那日喬大戶沒來。先是雜耍百戲,吹打彈唱,隊舞弔罷,做了個笑樂院本。割切上來,獻頭一道湯飯。只見任醫官到了,冠帶著進來。西門慶迎接至廳上敘禮。任醫官令左右毡包內取出一方壽帕,二星白金來,與西門慶拜壽。說道:「昨日韓明川纔說老先生華誕,恕學生來遲。」西門慶道:「豈敢動勞車駕?又兼謝盛儀。外日多謝妙藥。」彼此拜畢,任醫官還要把盞。西門慶道:「不消。剛纔已見過禮,就是了。」一面脫了衣服,安在左手第四席,與吳大舅相近而坐。獻上湯飯,并手下攢盤,任醫官道:「多謝了。」令僕從領下去,告坐坐下。四個唱的彈著樂器,在旁唱了一套壽詞。西門慶令上席,各分投遞酒。下邊樂工呈上揭帖,到劉、薛二內相席前。揀令一段韓湘子度陳半街升仙會雜劇。纔唱得一摺,只聽喝道之聲漸近。平安進來稟報:「守備府周爺來了。」西門慶冠帶迎接,未曾相見,就先令寬盛服。周守備道:「我來非為別務,要與四哥把一盞。」薛內相向前來說道:「周大人不消把盞,只見禮兒罷。」于是二人交拜。又道:「我學生來遲,恕罪!恕罪!」敘畢禮數,方寬衣解帶,纔與眾人作揖。左首第三席安下鍾筯。下邊就是湯飯,割切一道添換,拿上來,席前打發馬上人兩盤點心、兩盤熟肉、兩瓶酒。周守備舉手謝道:「忒多了。」令左右上來領下去,然後坐下。一面劉、薛二內相,每人送周守備一大杯。觥籌交錯,歌舞吹彈,花攢錦簇飲酒。正是:

「舞低楊柳樓心月,  歌罷桃花扇底風。」

吃至日暮時分。先是任醫官隔門去的早,西門慶送出來。任醫官因問:「老夫人貴恙覺好了?」西門慶道:「拙室服了良劑,已覺好些。這兩日不知怎的,又有些不自在。明日還望老先生過來看看。」說畢,任醫官作辭;上馬而去。落後又是倪秀才、溫秀才起身。西門慶再三款留不住,送出大門,說道:「容日奉拜請教。寒家就在對門收拾一所書院,與老先生居住,連寶眷多搬來一處方便。學生每月奉上束修,以備薪水之需。」溫秀才道:「多承盛愛,感激不盡。」倪秀才道:「觀此,是老先生崇尚斯文之雅意矣!」打發二秀才去了。西門慶陪客飲酒,吃至更闌方散。四個唱的都歸在月娘房內,唱與月娘、大妗子、楊姑娘眾人聽。西門慶還在前邊,留下吳大舅、應伯爵復坐飲酒,看著打發樂工酒飯吃了,先去了。其餘席上家火都收了,鮮果殘饌,都令手下人分散吃了,先去了;分付從新後邊拿果碟兒上來,教李銘、吳惠、鄭奉上彈唱,拏大杯賞酒與他吃。應伯爵道:「哥,今日華誕設席,列位都是喜歡。」李銘道:「今日薛爺和劉爺,也費了許多賞賜。落後見桂姐、銀姐又出來,每人又遞了一包與他。只是薛爺比劉爺年小快頑些。」不一時,畫童兒拿上添換果碟兒來,都是蜜餞減碟、榛松果仁、紅菱雪藕、蓮子 、荸薺 、酥油包螺 、冰糖霜梅 、玫瑰餅 之類。這應伯爵看見酥油包螺 ,渾白與粉紅兩樣,上面都沾著飛金。就先揀了一個,放在口內,如甘露酒心,入口而化。說道:「倒好吃!」西門慶道:「我的兒,你倒肯吃,此是你六娘親手揀的。」伯爵笑道:「也是我女兒孝順之心。」說道:「老舅,你也請個兒。」于是揀了一個,放在吳大舅口內。又叫李銘、吳惠、鄭奉近前,每人揀了一個賞他。正飲酒間,伯爵向玳安道:「你去後邊叫那四個小淫婦出來,我便罷了,也教他唱個兒與老舅聽。再遲一回兒,便好去。今日連用錢,他只唱了兩套。休要便宜了他。」那玳安不動身,說道:「小的叫了他了。在後邊唱與妗子和娘每聽哩,便來。」伯爵道:「賊小油嘴,你幾時去哩?還哄我。」因叫王經:「你去。」那王經又不動。伯爵道:「我便看你每都不去,等我去罷。」于是就往後走。玳安道:「你老人家趁早休進去。後邊有狗哩,好不利害,只咬大腿。」伯爵道:「若咬了我,我直賴到你娘那炕頭子上。」玳安入後邊良久,只聽一陣香風過,覺有笑聲。四個粉頭,都用汗巾兒搭著頭出來。伯爵看見:「我的兒,誰養的你恁乖?搭上頭兒,心裡要去的情,好自在性兒!不唱個曲兒與俺每聽,就指望去,好容易!連轎子錢,就是四錢銀子。買紅梭兒來,買一石七八斗。勾你家鴇子和你一家大小吃一個月。」董嬌兒道:「哥兒,恁便益衣飯兒,你也入了籍罷了!」洪四兒道:「大爺,這咱晚七八有二更,放了俺每去罷了。」齊香兒道:「俺每明日還要起早往門外送殯去哩。」伯爵道:「誰家?」齊香兒道:「是房簷底下開門兒那家子。」伯爵道:「莫不又是王三官兒家?前日被他連累你那場事,多虧你大爹這裡人情替李桂兒說,連你也饒了。這一遭雀兒不在那窩兒罷了。」齊香兒笑罵道:「怪老油嘴!汗邪了你恁胡說!」伯爵道:「你笑話我老,我那些兒放著老?我半邊俏,把你這四個小淫婦兒還不勾擺布!」洪四兒笑道:「哥兒,我看你行頭不怎麼好,光一味好撇!」伯爵道:「我那兒,到根前看手段還錢。」又道:「鄭家那賊小淫婦兒,吃了糖五老座子兒,百不言語,有些出神的模樣。敢記掛著那孤老兒在家裡?」董嬌兒道:「他剛纔聽見你說,在這裡有些怯床。」伯爵道:「怯床不怯床,拏樂器來,每人唱一套,你每去罷。我也不留你了。」西門慶道:「也罷,你每叫兩個遞酒,兩個唱一套與他聽罷。」齊香兒道:「等我和月姐唱。」當下鄭月兒琵琶,齊香兒彈箏,坐在校床兒,兩個輕舒玉指,款跨鮫綃,啟朱唇,露皓齒,歌美韻,放嬌聲,唱了一套越調鬬鵪鶉:「夜去明來,倒有個天長地久。」當下董嬌兒遞吳大舅酒,洪四兒遞應伯爵酒,在席上交杯換盞,倚翠偎紅,翠袖慇懃,金杯瀲灧。正是:

「朝赴金谷宴,  暮伴綺樓娃,

休道歡娛處,  流光逐落霞。」

當下酒進數巡,歌吟兩套,打發四個唱的去了。西門慶還留吳大舅坐,教春鴻上來,唱南曲與大舅聽。分付棋童:「備馬來,拏燈籠送大舅。」大舅道:「姐夫不消備馬,我同應二哥一路走罷。天色晚了。」西門慶道:「無是理。如此,教棋童打燈籠送到家。」當下唱了一套,吳大舅與伯爵起身,作別道:「深擾姐夫。」西門慶送至大門首,因和伯爵說:「你明日好歹上心,約會了那位甘夥計來見了批合同。我會了喬親家,好收拾那邊房子。一兩日卸貨。」伯爵道:「哥不消分付,我知道。」一面作辭,與大舅同行。棋童打著燈籠,吳大舅便問:「剛纔姐夫說收拾那裡房子?」伯爵悉把韓夥計貨船到,無人發賣,他心內要開個段子舖,收拾對門房子,教我替他尋個夥計一節,對大舅說了。大舅道:「幾時開張?咱每親朋會定,少不的具果盒花紅,來作賀作賀。」須臾出大街,到伯爵小胡同口上。大舅要棋童打燈籠:「送你應二叔到家。」伯爵不肯,說道:「棋童,你送大舅,我不消燈籠。進巷內就是了!」一面作辭,分路回來。棋童便送大舅去了。西門慶打發李銘等唱錢,關門回後邊月娘房中歇了一夜。到次日,果然伯爵領了甘出身,穿青衣走來拜見,講說了回買賣之事。西門慶叫將崔本來,會喬大戶那邊,收拾房子卸貨,修蓋土庫局面,擇日開張舉事。喬大戶對崔本說:「將來凡一應大小事,隨你親家爹這邊只顧處,不消多計較。」當下就和甘夥計批立了合同,就立伯爵作保。譬如得利十分為率,西門慶分五分,喬大戶分三分,其餘韓道國、甘出身與崔本三分均分。一面收卸磚瓦木石,修蓋土庫裡面,裝畫牌面。待貨車到日,堆卸貨物。後邊獨自收拾一所書院,請將溫秀才來作西賓。專修書柬,回答往來士夫。每月三兩束修,四時禮物不缺。又撥了畫童兒小廝伏侍他半晚,替他拿茶飯,舀硯水。他若出門望朋友,跟他拏拜帖匣兒。西門慶家中常筵客,就請過來陪侍飲酒,俱不必細說。不覺過了西門慶生辰,第二日早辰,就請了任醫官來看李瓶兒討藥,又在對門看看收拾。楊姑娘先家去了,李桂姐、吳銀兒,還沒家去。吳月娘買了三錢銀子螃蟹,午間煮了,來往後邊院內,請大妗子、李桂姐、吳銀兒眾人,都圍著吃了一回。只見月娘請的劉婆子來看官哥兒,吃了茶,李瓶兒就陪他往前邊房裡去了。劉婆子說:「哥兒驚了,住了奶。」又留下幾服藥。月娘與了他三錢銀子,打發去了。孟玉樓、潘金蓮和李桂姐、吳銀兒、大姐都在花架底下,放小卓兒、舖氈條,同抹骨牌,賭酒頑耍。那個輸一牌,吃一大杯酒。孫雪娥吃眾人贏了七八鍾酒,又不敢久坐,坐一回又去了。西門慶在對門房子內,看著收拾打掃,和應伯爵、崔本、甘夥計吃酒,又使小廝來家要菜兒。慌的雪娥往廚下打發,只拏李嬌兒頂缺。金蓮教吳銀兒、桂姐:「你唱慶七夕俺每聽。」當下彈著琵琶,唱商調集賢賓:

「暑纔消,大火即漸西。斗柄往,次宮移。一葉梧桐飄墜,萬方秋意皆知。暮雲軒,聒聒蟬鳴;晚風輕,點點螢飛。天階夜涼清似水,鵲橋高掛偏宜。金盤內種五生,瓊樓上設筵席。」

當日眾姊妹飲酒至晚,月娘裝了盒子,相送李桂姐、吳銀兒家去了。潘金蓮吃的大醉歸房。因見西門慶夜間在李瓶兒房裡歇了一夜,早辰請任醫官又來看他,那惱在心裡。知道他孩子不好,進門不想天假其便,黑影中躧了一腳狗尿。到房中叫春梅點燈來看,大紅段子新鞋兒上,滿幫子都展污了。登時柳眉剔豎,星眼圓睜。叫春梅打著燈,把角門關了。拏大棍把那狗沒高低,只顧打,打的怪叫起來。李瓶兒那邊使過迎春來說:「俺娘說哥兒纔吃了老劉的藥,睡著了,教五娘這邊休打狗罷。」這潘金蓮坐著,半日不言語。一面把那狗打了一回,開了門,放出去了,又尋起秋菊的不是來。看著那鞋,左也惱,右也惱。因把秋菊喚至跟前說:「論起這咱晚,這狗也該打發去了。只顧還放在這屋裡做甚麼?是你這奴才的野漢子?你不發他出去,教他恁遍地撒尿。把我恁雙新鞋兒,連今日纔三四日兒,躧了恁一鞋幫子尿!知道了我來,你與我點箇燈兒出來!你如何恁推聾粧啞裝憨兒?」春梅道:「我頭裡又對他說,你趁娘不來,早喂他些飯,關到後邊院子裡去罷。他佯打耳睜的不理我,還拏眼兒瞟著我!」婦人道:「可又來,賊膽大萬殺的奴才!怎麼恁把屁股兒懶待動彈?我知道你在這屋裡成了把頭,便說你恁久慣牢頭,把這打來不作理。」因叫他到跟前,叫春梅:「拏過燈來,教他瞧綉的我這鞋上的齷齷!我纔做的恁奴心愛的鞋兒,就教你奴才遭塌了我的!」哄得他低頭瞧,提著鞋拽巴,兜臉就是幾鞋底子。打的秋菊嘴唇都破了,只顧搵著搽血。那秋菊走開一邊。婦人罵道:「好賊奴才,你走了!」教春梅:「與我採過跪著。取馬鞭子來,把他身上衣服與我扯了,好好教我打三十馬鞭子便罷。但扭一扭兒,我亂打了不算!」春梅于是扯了他衣裳。婦人教春梅把他手拴住,雨點般鞭子輪起來,打的這丫頭殺豬也似叫。那邊官哥纔合上眼兒,又驚醒了。又使了綉春來說:「俺娘上覆五娘,饒了秋菊,不打他罷。只怕諕醒了哥哥。」那潘姥姥正〈扌歪〉在裡間屋裡炕上,聽見金蓮打的秋菊叫,一〈石古〉碌子扒起來,在旁邊勸解。見金蓮不依,落後又見李瓶兒使過綉春來說,又走向前奪他女兒手中鞭子,說道:「姐姐,少打他兩下兒罷。惹的那邊姐姐說,只怕諕了哥哥。為驢扭棍不打緊,倒沒的傷了紫荊樹。」金蓮緊自心裡惱,又聽見他娘說了這一句,越發心中攛了把火一般。須臾紫漒了面皮,把手只一推,險些兒不把潘姥姥推了一交。便道:「怪老貨!你不知道,與我過一邊坐著去!不干你事,來勸甚麼?腌子!甚麼紫荊樹,驢扭棍,單管外合裏差!」潘姥姥道:「賊作死的短壽命!我怎的外合裏差?我來你家討冷飯吃?教你恁頓捽我!」金蓮道:「你明日說與我來,看那老〈毛皮〉走,怕是他家不敢拏長鍋煮吃了我。」那潘姥姥聽見女兒這等證他,走那裡邊屋裡,嗚嗚咽咽哭起來了。由著婦人打秋菊,打勾約二三十馬鞭子,然後又蓋了十闌杆,打得皮開肉綻,纔放起來。又把他臉和腮頰,都用尖指甲搯的稀爛。李瓶兒在那邊,只是雙手握著孩子耳朵腮頰痛淚,敢怒而不敢言。不想那日西門慶在對門房子裡吃酒散了,逕往玉樓房中歇了一夜。到次日,周守備家請吃補生日酒,不在家。李瓶兒見官哥兒吃了劉婆子藥,不見動靜,夜間又著驚諕,一雙眼只是往上吊吊的。因那日薛姑子、王姑子家去,來對月娘說;向房中拏出他壓被的銀獅子一對來,要教薛姑子印造佛頂心陀羅經,趕八月十五日嶽廟裡去捨。那薛姑子就要拏著走,被孟玉樓在旁說道:「師父,你且住。大娘,你還使小廝叫將賁四來,替他兌兌多少分兩,就同他往經舖裡講定個數兒來。每一部經多少銀子?咱每捨多少,到幾時有?纔好。你教薛師父去,他獨自一個怎弄的過來?」月娘道:「你也說的是。」一面使來安兒:「你去瞧,賁四來家不曾?你叫了他來。」來安兒一直去了。不一時,賁四來到。向月娘眾人作了揖,把那一對銀獅子上天平兌了,重四十九兩伍錢。月娘分付同薛師父往經舖請印造經數去了。潘金蓮隨即叫孟玉樓:「咱送送他兩位師父去。就前邊看看大姐,他在屋裡做鞋哩。」兩個攜著手兒,往前邊來。賁四同來安兒、薛姑子、王姑子往經舖裡去。金蓮與玉樓走出大廳前,來東廂房門首,見他正守著針線筐兒,在簷下納鞋。金蓮拏起來看,卻是沙綠潞紬子鞋面。玉樓道:「大姐,你不要這紅鎖線子。爽利著藍頭線兒,卻不老作些?你明日還要大紅提跟子?」大姐道:「我有一雙是大紅提根子的。這個我心裡要藍提跟子,所以使大紅線鎖口。」金蓮瞧了一回,三個都在廳臺基上坐的。玉樓問大姐:「你女婿在屋裡不在?」大姐道:「他不知那裡吃了兩鍾洒,在屋裡睡哩。」孟玉樓便向金蓮說:「剛纔若不是我在旁邊說著,李大姐恁哈帳行貨,就要把銀子交姑子拏了印經去。經也印不成,沒腳蟹行貨子,藏在那大人家,你那裡尋他去?早時我說,叫將賁四來,同他去了。」金蓮道:「你看麼,你教我幹,恁有錢的姐姐,不撰他些兒是傻子;只相牛身上拔一根毛了!你孩兒若沒命,休說捨經,隨你把萬里江山捨了,也成不的!正是饒你有錢拜北斗,誰人買得不無常?如今這屋裡,只許人放火,不許俺每點燈。大姐聽著,也不是別人。偏染的白兒不上色,偏你會那等輕狂百勢!大清早辰,刁蹬著漢子請太醫看。他亂也的,俺每又不管。每當在人前,會那等做清兒說話。我心裡不耐煩。他爹要便進我屋裡,推看孩子睡著,和我睡。誰耐煩?教我就攛掇往別人屋裡睡去了。俺每自恁好罷了,背地還嚼說俺每。那大姐姐,偏聽他一面詞兒說話。不是俺每爭這個事,怎麼昨日漢子不進你屋裡去,你使丫頭在角門子首叫進屋裡,推看孩子,你便吃藥,一徑把漢子作成在那屋裡和吳銀兒睡了一夜去了。一徑顯你那乖覺,教漢子喜歡你。那大姐姐就有的話兒說了。昨日晚夕,人進屋裡躧了一鞋狗尿,打丫頭趕狗,也嗔起來。使丫頭過來說,諕了他孩子了。俺娘那老貨,又不知道,〈扌晃〉他那嘴吃,教他那小買手,走來勸甚麼的驢扭棍傷了紫荊樹。我惱他這等輕聲浪氣,他又來我跟前說話長短。教我墩了他兩句,他今日使性子家去了。去了罷,教我說,他家有你這樣窮親戚也不多,沒你也不少!比時恁他快使性子,到明日不要來他家,怕他拏長鍋煮吃了我,隨他和他家纏去。」玉樓笑道:「你這個沒訓教的子孫,你一個親娘母,見你這等訌他?」金蓮道:「不是這等說,惱人子腸了!單管黃貓黑尾,外合裡差,只替人說話!吃人家碗半,被人家使喚。得不的人家一個甜頭兒,千也說好,萬也說好。想著迎頭兒養了這個孩子,把漢子調咬的生根也似的,把他便扶的正正兒的,把人恨不的躧到那泥裡頭還躧!今日怎的天也有眼,你的孩兒生出病來了!我只說日頭常晌午,如何也有個錯了的時節兒!」正說著,只見賁四和來安來往經舖裡交了銀子,來回月娘話。看見玉樓、金蓮和大姐都在廳臺基上坐的,只顧在儀門外立著,不敢進來。來安走來,說道:「娘每閃閃兒,賁四來了。」金蓮道:「怪囚根子!你教他進去不是,纔乍見他?」來安說了,賁四于是低著頭,一直後邊見月娘、李瓶兒,把上項:「兌了銀子四十一兩五錢,眼同兩個師父,交付與翟經兒家收了。講定印造綾壳陀羅五百部,每部五分;絹壳經一千部,每部三分。算共該五十五兩銀子。除收過四十一兩五錢,還找與他十三兩五錢。准在十四日早抬經來。」李瓶兒連忙向房裡取出一個銀香毬來,教賁四上天平兌了,十五兩。李瓶兒道:「你拏了去。除找與他,別的你收著。換下些錢,到十五日廟上捨經,與你每做盤纏就是了。省的又來問我要。」賁四于是拿香毬出門。月娘使來安送賁四出去。李瓶兒道:「四哥,多累你。」賁四躬著身說道:「小人不敢。」走到前邊,金蓮、玉樓又叫住問他:「銀子交付與經舖了?」賁四道:「已交付明白,共一千五百部經,共該給五十五兩銀子。除收過那四十一兩五錢,剛纔六娘又與了這件銀香毬。」玉樓、金蓮瞧了瞧,沒言語。賁四便回家去了。玉樓向金蓮說道:「李大姐相這等,都枉費了錢。他若是你的兒女,就是榔頭也樁不死。他若不是你兒女,你捨經造像,隨你怎的,也留不住!他信著姑子,甚麼繭兒幹不出來!剛纔不是我說著,把這些東西就託他拏的去了。這等著咱家個人兒去,卻不好?」金蓮道:「總然他背地落,也落不多兒。」兩個說了一回,都立起來。金蓮道:「咱每往前邊大門首走走去。」因問大姐:「你不出去?」大姐道:「我不去。」這潘金蓮便拉著玉樓手兒,兩個同來到大門裡首站立。因問平安兒:「對門房子都收拾了?」平安道:「這咱哩,從昨日爹看著都打掃乾淨了。後邊樓上堆貨。昨日教陰陽來破土,樓底下要裝廂三間土庫閣段子。門面打開一溜三間,舖子局面,都教漆匠裝新油漆。地下鏝磚鑲地平,打架子,要在出月開張。」玉樓又問:「那寫書溫秀才家小,搬過來了不曾?」平安道:「從昨日就過來了。今早爹分付,把後邊堆放的那一張涼床子拆了與他。又搬了兩張卓子,四張椅子,與他坐。」金蓮道:「你沒見他老婆,怎的模樣兒?」平安道:「黑影子坐著轎子來,誰看見他來?」正說著,只聽見遠遠一個老頭兒,斯琅琅搖著驚閨葉過來。潘金蓮便道:「磨鏡子的過來了。」教平安兒:「你叫住他,與俺每磨磨鏡子。我的鏡子,這兩日都使的昏了。分付你這囚根子,看著過來再不叫!俺每出來跕了多大回,怎的就有磨鏡子的過來了?」那平安一面叫住磨鏡老兒,放下擔兒。見兩個婦人在門裡首,向前唱了兩個喏,立在旁邊。金蓮便問玉樓道:「你也磨?都教小廝帶出來,一答兒里磨了罷。」于是使來安兒:「你去我屋裡,問你春梅姐討我的照臉大鏡子,兩面小鏡子兒;就把那大四方穿衣鏡也帶出來,教他好生磨磨。」玉樓分付來安:「你到我屋裡,教蘭香也把我的鏡子拏出來。」那來安兒去不多時,兩隻手提著大小八面鏡子,懷裡又抱著四方穿衣鏡出來。金蓮道:「賊小肉兒,你拏不了,做兩遭兒拏。如何恁拏出來?一時叮噹了我這鏡子,怎了?」玉樓道:「我沒見你這面大鏡子,是那裡的?」金蓮道:「是舖子人家當的。我愛他且是喨,安在屋裡,早晚照照。」因問:「你的鏡子只三面?」玉樓道:「我的大小只兩面。」金蓮道:「這兩面是誰的?」來安道:「這兩面是俺春梅姐的,稍出來也教磨磨。」金蓮道:「賊小肉兒,他放著他的鏡子不使,成日只撾著我的鏡子照。弄的恁昏昏的!」共大小八面鏡子,交付與磨鏡者叟,教他磨。當下絆在坐架上,使了水銀,那消頓飯之間,睜磨的耀眼爭光。婦人拏在手內,對照花容,猶如一汪秋水相似。有詩為證:

「蓮萼菱花共照臨,  風吹兒動影沉沉,

一池秋水芙蓉現,  好似嫦娥入月宮;

翠袖拂塵霜暈退,  朱唇呵氣碧雲深,

從教粉蝶飛來撲,  始信花香在畫中。」

那磨鏡老子須臾將鏡子磨畢。交與婦人看了,付與來安兒收進去了。玉樓便令平安問舖子裡傅夥計櫃上,要五十文錢兒與磨鏡的。那老子一手接了錢,只顧立著不去。玉樓教平安問那老子:「你怎的不去?敢嫌錢少?」那老子不覺眼中撲簌簌流下淚來哭了。平安道:「俺當家的奶奶問你,怎的煩惱?」老子道:「不瞞哥哥說,老漢今年痴長六十一歲。老漢前者丟下個兒子,二十二歲,尚未娶妻。專一狗油,不幹生理。老漢日逐出來掙錢,便養活他。他又不守本分,常與街上搗子耍錢。昨日惹了禍,同拴到守備府中,當土賊打了他二十大棍。歸來把媽媽的裙襖,都去當了。媽媽便氣了一場病,打了寒,睡在炕上半個月,老漢說了他兩句,他便走出來,不往家去。教老漢日逐抓尋他不著個下落。待要賭氣不尋他,況老漢恁大年紀,止生他一個兒子,往後無人送老。有他在家,見他不成人,又要惹氣。似這等,乃老漢的業障!有這等負屈啣冤,各處告訴,所以這等淚出痛腸。」玉樓教平安兒:「你問他,你這後娶婆兒,是今年多大年紀了?」老子道:「他今年痴長五十五歲了,男女花兒沒有。如今打了寒纔好些,只是沒將養的,心中想塊臘肉兒吃。老漢在街上恁問了兩三日,走了十數條街巷,白不討出塊臘肉兒來!甚可嗟歎人子!」玉樓笑道:「不打緊處。我屋裡抽替內,有塊臘肉兒哩。」即令來安兒:「你去對蘭香說,還有兩個餅錠,教他拿與你來。」金蓮叫那老頭子問:「你家媽媽兒,吃小米兒粥不吃?」老漢道:「怎的不吃?那裡可知好哩!」金蓮于是叫過來安兒來:「你對春梅說,把昨日你姥姥稍來的新小米兒量二升,就拏兩個醬瓜兒出來,與他媽媽兒吃。」那來安去不多時,拏出半腿臘肉,兩個餅錠,二升小米,兩個醬瓜茄 ,叫道:「老頭子過來,造化了你。你家媽媽子不是害病想吃,只怕害孩子坐月子,想定心湯吃。」那老子連忙雙手接了,安放在擔內,望著玉樓、金蓮唱了個喏,揚長挑著擔兒,搖著驚閨葉去了。平安道:「二位娘不該與他這許多東西,被這老油嘴設智誆的去了!他媽媽子是個媒人,昨日打這街上走過去不是?幾時在家不好來?」金蓮道:「賊囚!你早不說,做甚麼來?」平安道:「罷了,也是他的造化!可可二位娘出來看見,叫住他,照顧了他這些東西去了。」正是:

「閒來無事倚門楣,  正是驚閨一老來;

不獨纖微能濟物,  無緣滴水也難為。」

畢竟未知後來如何,且聽下回分解:
