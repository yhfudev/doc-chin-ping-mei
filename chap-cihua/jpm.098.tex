%# -*- coding: utf-8 -*-
%!TEX encoding = UTF-8 Unicode
%!TEX TS-program = xelatex
% vim:ts=4:sw=4
%
% 以上设定默认使用 XeLaTex 编译,并指定 Unicode 编码,供 TeXShop 自动识别

%第九十八回 
\chapter{陳經濟臨清開大店\KG 韓愛姐翠館遇情郎}


「心安茅屋穩,  性定菜根香,

世味憐方好,  人情淡最長;

因人成事業,  避難遇豪強,

今日崢嶸貴,  他年身必殃。」

話說一日周守備,濟南府知府張叔夜,領人馬征勦梁山泊,賊王宋江三十六人,萬餘草寇,都受了招安,地方平復。表奏,朝廷大喜。加陞張叔夜為都御史,山東安撫大使。陞守備周秀為濟南兵馬制置,管理分巡河道,提察盜賊。部下從征有功人員,各陞一級,軍門帶得經濟名字,陞為參謀之職,月給米二石,冠帶榮身。守備至十月中旬,領了勅書,率領人馬來家。先使人來報與春梅家中知道。春梅滿心歡喜,使陳經濟與張勝、李安,出城迎接。家中廳上排設酒筵,慶官賀喜。官員人等,來拜賀送禮者,不計其數。守備下馬,進入後堂。春梅、孫二娘接著,參拜已畢。陳經濟換了衣巾,就穿大紅員領,頭戴冠帽,腳穿皁靴,束著角帶,和新婦葛氏兩口兒拜見。守備見好個女子,賞了一套衣服,十兩銀子打頭面,不在話下。晚夕春梅和守備在房中飲酒,未免敘些家常事務:「又娶我兄弟媳婦,費許多東西。」守備道:「阿呀!你止這個兄弟投奔你來,無個妻室,不成個前程道理!就使費了幾兩銀子,不曾為了別人。」春梅道:「你今又替他掙了這個前程,足以榮身勾了!」守備道:「朝廷旨意下來,不日我往濟南府到任。你在家看家,打點些本錢,教他搭個主管,做些大小買賣。三五日教他下去查帳目一遭,轉得些利錢來,也勾他攪計。」春梅道:「你說的也是。」兩個晚夕,夫妻同歡,不必細述。在家只住了十個日子,到十一月初旬時分,守備收拾起身,帶領張勝、李安,前去濟南到任,留周仁、周義看家。陳經濟送到城南永福寺方回。一日,春梅向經濟商議:「守備教你如此這般,河下尋些買賣,搭個主管,覓得些利息,也勾家中費用。」這經濟聽言,滿心歡喜。一日,正打街前行走,尋覓主管夥計。也是合當有事,不料撞遇舊時朋友陸二哥陸秉義,作揖說:「哥,怎的一向不見?」這經濟便把亡妻為事,被楊光彥那廝拐了我半船貨物,坑陷的我一貧如洗。我如今又好了,幸得我姐姐嫁在守備府中,又娶了親事,陞做參謀,冠帶榮身。如今要尋個夥計,做些買賣,一地里沒尋處。陸秉義道:「楊光彥那廝拐了你貨物,如今搭了個姓謝的做夥計,在臨清馬頭上謝家大酒樓上,開了一座大酒店。又收錢放債,與四方趁熟窠子娼門人使,好不獲大利息!他每日穿好衣,吃好肉,騎著一疋驢兒,三五日下去走一遭,算帳收錢,把舊朋友都不理。他兄弟在家開賭場,鬬雞養狗,人不敢惹他!」經濟道:「我去年曾見他一遍,他反面無情,打我一頓,被一朋友救了我恨他入于骨髓!」因拉陸三郎入路旁一酒店內,兩個在樓上吃酒。兩人計議:「如何處置他,出我這口氣?」陸秉義道:「常言說得好:『恨小非君子,無毒不丈夫!』咱如今將理和他說,不見棺才不下淚,他必然不受。小弟有一計策,哥也不消做別的買賣,只寫一張狀子,把他告到那里,追出你貨物銀子來,就奪了這座酒店,再添上些本錢,和謝合夥,等我在馬頭上和謝三哥掌櫃發賣。哥哥,你三五日下去走一遭,查算帳日。管情見一月,你穩拍拍的有百十兩銀子利息,強如做別的生意。」看官聽說:當時不因這陸秉義說出這庄事,有分教數個人死于非命!陳經濟一種死,死之太苦;一種亡,亡之太屈!死的不好相似那五代的李存孝,漢書中彭越?正是:

「非于前定數,  半點不由人!」

經濟聽了,忙與陸秉義作揖,便道:「賢弟,你說的正是了。我到家,就對我姐夫和姐姐說。這買賣成了,就安賢弟同謝三郎做主管。」當下兩個吃了回酒,各下樓來,還了酒錢。經濟分付:「陸二哥,兄弟千萬謹言!有事我謝你去。」陸二郎道:「我知道。」各散回家。這經濟就一五一十,對春梅說。「爭奈他爺不在,如何理會?」有老家人周忠在旁,便道:「不打緊,等舅寫了一張狀子,該拐了多少銀子貨物,拏爺個拜帖兒,都封在裡面。等小的送與提刑所,兩位官府案下。把這姓楊的拏去衙門中,一頓夾打追問,不怕那廝不拏出銀子來!」經濟大喜。一面寫就一紙狀子,拏守備拜帖,彌封停當,就使老家人周忠,送到提刑院。兩位官府,正升廳問事。門上人稟進,說:「帥府周爺,差人下書。」何千戶與張二官府喚周忠進見,問周爺上任之事,說了一遍。拆開封套觀看,見了拜帖狀子,自恁要做分上。即便批行,差委緝捕番捉,往河下拏楊光彥去。回了個拜帖,付與周忠:「到家多上覆你爺、奶奶,待我這里追出銀兩,伺候來領。」周忠拏回帖到府中,回覆了春梅說話:「即時准行拏人去了。待追出銀子,使人領去。」經濟看見兩個摺帖上面,寫著侍生何永壽、張懋得頓首拜,經濟心中大喜。遲了不上兩日光景,提刑緝捕,觀察番捉,往河下把楊光彥并兄弟楊二風,都拏了到于衙門中。兩位官府據着陳經濟狀子審問,一頓夾打,監禁數日,追出三百五十兩銀子,一百桶生眼布。其餘酒店中家活,共算了五十兩。陳經濟狀上告著九百兩,還差三百五十兩銀子。把房兒賣了五十兩,家產盡絕,這經濟就把謝家大酒樓奪過來,和謝胖子合夥。春梅又打點出五百兩本錢,共湊了一千兩之數,委付陸秉義做主管,從新把酒樓粧修,油漆彩畫。闌干灼燿,棟宇光新,桌案鮮明,酒肴齊整。一日開張,鼓樂喧天,笙簫雜奏,招集往來客商,四方遊妓。陳經濟道:「那日宰豬祭祀燒紙。」常言:「啟甕三家醉,開樽十里香。神仙留玉珮,卿相解金貂。」經濟上來大酒樓上,週圍都是推窗亮隔,綠油闌干。四望雲山疊疊,上下天水相連。正東看,隱隱青螺堆岱嶽;正西瞧,茫茫蒼霧鎖皇都;正北觀,層層甲第起朱樓;正南望,浩浩長淮如素練。樓上下有百十座閣兒,處處舞裙歌妓,層層急管繁絃。說不盡肴如山積,酒若流波。正是:

「得多少舞楊柳樓心月,  歌罷桃花扇底風!」

從正月半頭,這陳經濟在臨清馬頭上大酒樓開張,見一日他發賣三五十兩銀子,都是謝胖子和主管陸秉義,眼同經手,在櫃上掌櫃。經濟三五日騎頭口,伴當小姜兒跟隨,往河下算帳一遭。若來,陸秉義和謝胖子兩個夥計,在樓上收拾一間乾淨閣兒,鋪陳床帳,安放卓椅;糊的雪洞般齊整,擺設酒席,叫四個好出色粉頭相陪,陳三兒那里往來做量酒。一日,三月住間,天光明媚,景物芬芳。翠依依槐柳盈堤,紅馥馥杏桃燦錦。陳經濟在樓上,搭伏定綠闌干,看那樓下景致,好生熱鬧!有詩為證:

「風拂煙籠錦施楊,  太平時節日初長,

能添壯士英雄膽,  善解佳人愁悶腸;

三尺曉垂楊柳岸,  一竿斜插杏花旁,

男兒未遂平生志,  且樂高歌入醉鄉。」

一日經濟在樓窗後瞧看,正臨著河邊泊著兩隻剝船。船上戴著許多箱籠卓凳家活。四五個人盡搬入樓下空屋裡來。船上有兩個婦人:一個中年婦人,長挑身材,紫膛色;一個年小婦人,搽脂抹粉,生的白淨標致,約有二十多歲。盡走入屋裡來。經濟問謝主管:「是甚麼人?不問自由,擅自搬入我屋裡來?」謝主管道:「此是兩個東京來的婦人,投親不著,一時間無尋房住,央此間鄰居范老來說,暫住兩三日便去。正欲報知官人,不想官人來問。」這經濟正欲發怒,只見那年小婦人歛袵向前,望經濟深深的道了個萬福,告說:「官人息怒,非干主管之事。是奴家大膽,一時出于無奈,不及先來宅上稟報,報乞恕罪!容略住得三五日,拜納房金,就便搬去。」這經濟見小婦人會說話兒,只顧上上下下把眼看他,那婦人一雙星眼,斜盼經濟。兩情四目,不能定神。經濟口中不言,心內暗道:「倒相那里會過,這般眼熟!」那長挑身材中年婦人,也定睛看著經濟,說道:「官人,你莫非是西門老爺家陳姑夫麼?」這經濟吃了一驚,便道:「你怎的認得我?」那婦人道:「不瞞姑夫說,奴是舊夥計韓道國渾家,這個就是我女孩兒愛姐。」經濟道:「你兩口兒在東京,如何來在這里?你老公在那里?」那婦人道:「在船上看家活。」經濟急令量酒,請來相見。不一時,韓道國走來作揖,已是摻白鬚鬢。因說起:「朝中蔡太師、童太尉、李右相、朱太尉、高太尉、李太監六人,都被太學國子生陳東,上本參劾,後被科道交章彈奏,倒了。聖旨下來,拏送三法司問罪。發煙瘴地面,永遠充軍。太師兒子禮部尚書蔡攸處斬,家產抄沒入官。我等三口兒,各自逃生,投到清河縣我兄弟第二的那里。第二的把房兒賣了,流落不知去向。三口兒顧船,從河道中來。不想撞遇姑夫在此,三生有幸!」因問:「姑夫今還在那邊西門老爺家裡?」經濟把頭一頃,說了一遍,說:「我也不在他家了。我在姐夫守備周爺府中做了參謀官,冠帶榮身,近日合了兩個夥計,在此馬頭上開了個酒店,胡亂過日子便了。你每三口兒既遇著我,也不消搬去,便在此間住也不妨。請自穩便。」婦人與韓道國一齊下禮。說罷,就搬運船上家活箱籠。經濟看得心癢,也使伴當小姜兒和陳三兒,也替他搬運了幾件家活。王六兒道:「不勞姑夫費心用力!」彼此俱各歡喜。經濟道:「你我原是一家,何消計較!」經濟見天色將晚,有申牌時分,要回家。分付主管:「咱早送些茶盒與他。」上馬,伴當跟隨來家。一夜心心念念,只是放韓愛姐不下。過了一日,到第三日早起身,打扮衣服齊整,伴當小姜跟隨,來河下大酒樓店中,看著做了回買賣。韓道國那邊使的八老來請吃茶。經濟心下正要瞧去,恰八老來請,便起身進去。只見韓愛姐見了,笑容可掬,接將出來,道了萬福:「官人請裡面坐。」經濟到閣子內坐下。王六兒和韓道國都來陪坐。少頃茶罷,彼此敘些舊時已往的話。經濟不住把眼只睃那韓愛姐。愛姐延瞪瞪秋波一雙眼,只看經濟,彼此都有意了。有詩為證:

「弓鞋窄窄剪春羅,  香體酥胸玉一窩;

麗質不勝嬝娜態,  一腔幽恨蹙秋波。」

少頃,韓道國下樓去了。愛姐因問:「官人青春多少?」經濟道:「虛度二十六歲。敬問姐姐青春幾何?」愛姐笑道:「奴與官人一緣一會,也是二十六歲。舊日又是大老爺府上相會過面,如今又幸遇在一處。正是有緣千里來相會!」那王六兒見他兩個說得入港,看見關目,推個故事也下樓去了。止有他兩人對坐。愛姐把些風月話兒把勾經濟。經濟自幼幹慣的道兒,怎不省得,一逕起身出去。這韓愛姐從東京來,一路兒和他娘也做些道路。在蔡府中答應,與翟管家做妾,詩詞歌賦,諸子百家皆通,甚麼事兒不久慣!見經濟起身出去無人處,走向前挨在他身邊坐下,作嬌作痴說道:「官人,你將頭上金簪子借我看一看。」經濟正欲拔時,被愛姐一手按住經濟頭髻,一手拔下簪子來。便起身說:「我和你去樓上說句話兒。」一頭說,一頭走。經濟不免跟上樓來。正是:

「饒你奸似鬼,  也吃洗腳水!」

經濟跟他上樓,便道:「姐姐,有甚話說?」愛姐道:「奴與你是宿世姻緣,你休要作假;願偕枕蓆之歡,共效于飛之樂!」經濟道:「只怕此間有人知覺,卻使不得。」那韓愛姐做出許多妖嬈來,摟經濟在懷,將尖尖玉手扯下他褲子來。兩個情興如火,按納不住。愛姐不免解衣,仰臥在床上,交姤在一處。正是:

「色膽如天怕甚事,  鴛幃雲雨百年情!」

經濟問:「你叫幾姐?」那韓愛姐道:「奴是端午所生,就叫五姐,又名愛姐。」說畢話。霎時雲收雨散,偎倚共坐。韓愛姐便告經濟說:「自從三口兒東京來投親不著,盤纏缺欠,你有銀子,乞借應與我父親五兩,奴按利納還,不可推阻。」經濟應允,說:「不打緊,姐姐開口,就兌五兩來。」愛姐見他依允,還了他金簪子。兩個又坐了半日,恐怕人談論,吃了一盃茶,愛姐留吃午飯。經濟道:「我那邊有事,不吃飯了。少間,就送盤纏來與你。」愛姐道:「午後,奴略備一盃水酒,官人不要見卻,好歹來坐坐。」經濟在店中吃了午飯,又在街上閑散。走了一回,撞見晏公廟師兄金宗明,作揖,把前事訴說了一遍。金宗明道:「不知賢弟在守備老爺府中認了親,在大樓開大店,有失拜望!明日就使徒弟送茶來,閑中請去廟中坐一坐。」說罷,宗明歸去了。經濟走到店中,陸主管道:「裏邊住的老韓,請官人吃酒,沒處尋。」恰好八老又來請:「官人,就請二住主管相陪,再無他客。」經濟就同陸主管,走到裏邊房內,早已安排酒席齊整,無非魚肉菜菓之類。經濟上坐,韓道國主位,陸秉義、謝胖子打橫,王六兒與愛姐旁邊僉坐。八老往來篩酒下菜。吃過數盃,兩個主管會意,說道:「官人慢坐,小人櫃上看去。」起身去了。經濟平昔酒量,不十分洪飲。又見主管去了,開懷與韓道國三口兒吃了數盃,便覺些醉將上來。愛姐便問:「今日官人不回家去罷了?」經濟道:「這咱晚了,回去不得,明日起身去罷。」王六兒、韓道國吃了一回,下樓去了。經濟向袖中取出五兩銀子,遞與愛姐收了,到下邊交與王六兒。兩個交盃換盞,倚翠偎紅,吃至天晚。愛姐卸下濃粧,留經濟就在樓上閣兒裏歇了。當下枕畔山盟,衾中海誓,鶯聲燕語,曲盡綢繆,不能悉記。愛姐將來東京,在蔡太師府中,曾扶持過老太太,也學會些彈唱,又能識字會寫。經濟聽了,歡喜不勝,就同六姐一般,正可在心上,以此與他盤桓一夜,停眠整宿。免不的第二日起來得遲,約飯時纔起來。王六兒安排些雞子肉圓子,做了個頭腦,與他扶頭。兩個吃了幾盃煖酒。少頃,主管來請經濟,那邊擺飯。經濟包巾梳洗,穿衣。吃了飯,又來辭愛姐,要回家去,那愛姐不捨,只顧拋淚。經濟道:「我到家三五日就來看你,你休煩惱。」說畢伴當跟隨騎馬往城中去了。一路上分付小姜兒:「到家休要說出韓家之事。」小姜兒道:「小的知道,不必分付。」經濟到府中,只推店中買賣忙,算了帳目,不覺天晚,歸來不得,歇了一夜,交割與春梅利息銀兩,見一遭也有三十兩銀子之數。回到家中,又被葛翠屏聐聐:「官人怎的外邊歇了一夜?想必在柳陌花術行踏,把我丟在家中,獨自空房一個,就不思想來家!」一連留住陳經濟七八日,不放他往河下來。這里韓愛姐見他一去數日光景不來,店中自使小姜兒來問主管討算利息。主管一一封了銀子去。韓道國免不得又交老婆王六兒,又招惹別的熟人兒,或是商客,來屋裏走動,吃茶吃酒。這韓道國當先嚐著這個甜頭,靠老婆衣飯肥家。況此時王六兒年約四十五六,年紀雖半百,風韻猶存。恰好又得他女兒來接代,他不斷絕這樣行業。如今索性大做了。原來不當官身衣飯,別無生意,只靠老婆賺錢,謂之隱名娼妓。今時呼為私窠子是也。當時見經濟不來,量酒陳三兒替他勾了一個湖州販絲綿客人何官人來,請他女兒愛姐,那何官人年約五十余歲,手中有千兩絲紬絹貨物,要請愛姐。愛姐一心想著經濟,推心中不快,三回五次,不肯下樓來。急的韓道國要不的。那何官人又見王六兒長挑身材,紫膛色,瓜子面皮,描眉鋪鬢,大長水鬢,涎鄧鄧一雙星眼,眼光如醉,抹的鮮紅嘴唇,料此婦人一定好風情。就留下一兩銀子,在屋裏吃酒,和王六兒歇了一夜。韓道國便躲避在外間歇了。他女兒見做娘的留下客,只在樓上,不下樓來。自此以後,那何官人被王六兒搬弄得快活,兩個打得一似火炭般熱。沒三兩日,不來婦人家裡過夜。韓道國也禁過他許多錢使。這韓愛姐兒見濟一去數十日不見來,心中思想,挨一日似三秋,盼一夜如半夏。未免害「木邊之目,田下之心。」使八老往城中守備府中探聽。看見小姜兒,悄悄問他:「官人如何不去?」小姜見說:「官人這兩日有些身子不快,不曾出門。」回來訴與愛姐。愛姐與王六兒商議,買了一副豬蹄,兩隻燒鴨 ,兩尾鮮魚,一盒酥餅,在樓上磨墨揮筆,拂開花箋,寫封柬帖。使八老送到城中與經濟去。當下把禮物裝在盒內,交八老挑著,叮嚀囑付:「你到城中,見了陳官人,須索見他親收,討回帖來。」八老懷內揣著柬帖禮物,一路無詞。來到城內守備府前,坐在沿街石臺基上。只見伴當小姜兒出來,看見八老:「你又來做甚麼?」八老與聲喏,拉在僻淨處說:「我特來見你官人,送禮來了,有話說。我只在此等你,你可通報官人知道。」小姜隨即轉身進去。不多時,只見經濟搖將出來。那時約五月,天氣暑熱。經濟穿著紗衣服,頭戴瓦瓏帽,金簪子,腳上涼鞋淨襪。八老慌忙聲喏,說道:「官人貴體好些?韓愛姐使我稍一柬帖,送禮來了。」經濟接了柬帖,說:「五姐好麼?」八老道:「五姐見官人一向不去,心中也不快。在那里多上覆官人,幾時下去走走?」經濟拆開柬帖觀看,上面寫著甚言詞:

「                        賤妾韓愛姐歛袵拜謹啟

情郎陳大官人台下:

自別尊顏,思慕之心,未嘗少怠;懸懸不忘于心。向蒙期約,妾倚門凝望,不見降臨蓬蓽。昨遣八老探問起居,不遇而回。聽聞貴恙欠安,令妾空懷悵望,坐臥悶懨。不能頓生爾翼,而傍君之足下也!君在家自有嬌妻美愛,又豈肯動念于妾?猶吐去之菓核也!茲具腥味茶盒數事,少申問安誠意。辛希笑納,情照不宣!  外具錦繡鴛鴦香囊一個,青絲一縷,少表寸心!

下書仲夏念日賤妾愛姐再拜  」

經濟看了柬帖,并香囊。香囊裏面,安放青絲一縷。香囊是鴛鴦雙口做的,扣著:「寄與情郎陳君膝下」八字。依先摺了,藏在袖中。府傍側首,有個酒店。令小姜兒:「領八老同店內吃鍾酒,等我寫回帖與你。」分付小姜兒:「把禮物收進我房裡去。你娘若問,只說河下店主人謝家送的禮物。」小姜不敢怠慢,把四盒禮物收進去了。經濟走到書院房內,悄悄寫了回柬。又包了五兩銀子,到酒店內,問八老:「吃了酒不曾?」八老道:「多謝官人好酒!吃不得了,起身去罷。」經濟將銀子并回柬付八老,說:「到家多多拜上五姐,這五兩白金與他盤纏。過三兩日,我自去看他。」八老收了銀柬下樓,經濟送出店門,八老一直去了。經濟走入房中,葛翠屏便問:「是誰家送禮物?」經濟悉言:「店主人謝胖子打聽我不快,送這禮物來問安。」翠屏亦信其實。兩口兒計議,交丫鬟金錢兒拏盤子,拏了一隻燒鴨 ,一尾鮮魚,半副蹄子,送到後邊與春梅吃。說是店主人家送的,也不查問。此事表過不題。卻說八老到河下,天已晚了。入門將銀柬都付與愛姐收了。拆開銀柬,燈下觀看。上面寫道:

經濟頓首字覆

「愛卿韓五姐粧次:向蒙會問,又承厚款,亦且雲情雨意,袵席鍾愛,無時少怠!所云期望,正欲趨會。偶因賤軀不快,有失卿之盼望!又蒙遣人垂顧,兼惠可口佳肴,不勝感激!只在二三日間,容當面布。外具白金五兩,綾帕一方,少申遠芹之敬!伏乞心鑒,萬萬!

下書經濟再拜   。」

愛姐看了,見帕上寫著四句,詩曰:

「吳綾帕兒織迴紋,  洒翰揮毫墨跡新;

寄與多情韓五姐,  永諧鸞鳳百年情。」

看畢,愛姐把銀子付與王六兒。母子千歡萬喜等候經濟,不在話下。正是:

「得意友來情不厭,  知心人至話相投。」

有詩為證:

「碧紗窗下啟箋封,  一紙雲鴻香氣濃;

知你揮毫經玉手,  相思都付不言中。」

畢竟未知後來何如,且聽下回分解:

