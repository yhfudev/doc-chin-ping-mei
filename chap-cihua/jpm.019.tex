%# -*- coding: utf-8 -*-
% !TeX encoding = UTF-8 Unicode
% !TeX spellcheck = en_US
% !TeX TS-program = xelatex
%~ \XeTeXinputencoding "UTF-8"
% vim:ts=4:sw=4
%
% 以上設定默認使用 XeLaTex 編譯,並指定 Unicode 編碼,供 TeXShop 自動識別

%第十九回
\chapter{草裡蛇邏打蔣竹山\KG 李瓶兒情感西門慶}

\begin{showcontents}{}



「花開不擇貧家地,  月照山河處處明,

世間只有人心歹,  百事還教天養人,

痴聾瘖啞家豪富,  伶俐聰明卻受貧,

年月日時該載定,  算來由命不由人。」

話說西門慶家中起蓋花園捲棚,約有半年光景,裝修油漆完備,前後煥然一新。慶房整吃了數日酒,俱不在話下。一日,八月初旬天氣,與夏提刑做生日。在新買庄上擺酒,叫了四個唱的,一起樂工,雜耍步戲。西門慶從巳牌時分,打選衣帽齊整,四個小廝跟隨,騎馬去了。吳月娘在家,整置了酒餚細果,約同李嬌兒、孟玉樓、孫雪蛾、大姐、潘金蓮眾人,開了新花園門,閒中遊賞,翫看裡面花木庭臺,一望無際,端的好座花園!但見:

「正面丈五高,心紅漆綽屑,周圍二十板,砧炭乳口泥墻。當先一座門樓,四下幾多臺榭。假山真水,翠竹蒼松,高而不尖謂之臺,巍而不峻謂之榭。論四時賞翫,各有去處:春賞燕遊堂,檜栢爭鮮:夏賞臨溪館,荷蓮鬬彩,秋賞叠翠樓,黃菊迎霜;冬賞藏春閣,白梅積雪。剛見那嬌花籠淺徑,嫩柳拂雕欄。弄風楊柳縱蛾眉,帶雨海棠陪嫩臉;燕遊堂前,金燈花似開不開;藏春閣後,白銀杏半放不放。平野橋東,幾朵粉梅開卸;臥雲亭上,數株紫荊未吐,湖山側,纔綻金錢;寶檻邊,初生石笋。翩翩紫燕穿簾幙,嚦嚦黃鶯度翠陰。也有那月窗雪洞,也有那水閣風亭;木香棚與荼{艹縻}加相連,千葉桃與三春柳作對;也有那紫丁香、玉馬櫻、金雀藤、黃剌薇、香茉莉、瑞仙花。捲棚前後,松墻竹徑,曲水方池,映階蕉棕,白日葵榴,遊魚藻內驚人,粉蝶花間對舞;正是,芍藥展開菩薩面,荔枝擎出鬼王頭。」

當下吳月娘領著眾婦人,或携手遊芳徑之中,或鬬草坐香茵之上,一個臨欄對景,戲將紅豆擲金鱗;一個伏檻觀花,笑把羅紈驚粉蝶。月娘于是走在一個最高亭子上,名喚臥雲亭,和孟玉樓、李嬌兒下棋。潘金蓮和西門大姐、孫雪蛾,都在翫花樓望下觀看。見樓前牡丹花畔,芍藥圃、海棠軒、薔薇架、木香棚,又有那耐寒君子竹,欺雪大夫松;端的四時有不卸之花,八節有長春之景。觀之不足,看之有餘。不一時,擺上酒來,吳月娘居上,李嬌兒對席,兩邊孟玉樓、孫雪蛾、潘金蓮、西門大姐,各依序而坐。月娘道:「我忘了請陳姐夫來坐坐。」一面使小玉:「前邊快請姑夫來。」不一時,經濟來到,頭上天青羅帽,身穿紫綾深衣,腳下粉頭皂靴;向前作揖,就在大姐根前坐下。傳杯換盞,吃了一回酒,吳月娘還與李嬌兒、西門大姐下棋。孫雪蛾與孟玉樓,都上樓觀看。惟有金蓮且在山子前,花池邊,用白紗團扇撲蝴蝶為戲。不妨經濟悄悄在他身背後觀戲,說道:「五娘,你不會撲蝴蝶兒,等我替你撲。這蝴蝶兒,忽上忽下,心不定有些走瀼。」那金蓮扭回粉頸,斜瞅了他一眼,罵道:「賊短命,人聽著,你待死也!我曉得你也不要命了!」那陳經濟笑嘻嘻,撲近他身來,樓他親嘴。被婦人順手只一推,把小夥兒推了一交。都不想玉樓在翫花樓遠遠瞧見,叫道:「五姐,你走這裡來,我和你說話。」金蓮方纔撇了經濟。上樓去了。原來兩個蝴蝶,也沒曾捉的住。到訂了燕約鶯期,則做了蜂鬚花嘴。正是:

「狂蜂浪蝶有時見,  飛入梨花沒處尋。」

經濟見婦人去了,默默歸房,心中怏然不樂。口占析桂令一詞,以遣其悶。

「我見他斜戴花枝,朱唇上不抹胭脂,似抹胭脂。前日相逢,今日相逢,似有情實,未見情實!  欲見許,何曾見許?似推辭,本是不推辭。約在何時?會在何時?不相逢,他又相思;既相逢,我又相思。」

且不說吳月娘等,在花園中飲酒。單表西門慶從門外夏提刑庄子上吃了酒回來,打南瓦子裡頭過。平昔在三瓦兩巷行走耍子,搗子每都認的。那時宋時謂之搗子,今時俗呼為光棍是也。內中有兩個,一名草裡蛇魯華,一名過街鼠張勝,常被西門慶資助,乃雞竊狗盜之徒。西門慶見他兩個在那裡要錢,勒住馬,近前說話。二人連忙走至根前,打個半跪,道:「大官人,這咱晚往那去來?」西門慶道:「今日是提刑所夏老爹生日,門外庄上請我吃了酒來。我有一庄事央煩你每,依我不依?」二人道:「大官人沒的說,小人平昔受恩甚多,如今使令小人之處,雖赴湯蹈火,萬死何辭!」西門慶道:「既是你二人恁說,明日來我家,我有話分付你。」二人道:「那裡等的至明日,你老人家說與小罷!端的有甚麼事?」這西門慶附耳低言,便把蔣竹山要了李瓶兒之事,說了一遍:「只要你弟兄二人,替我出這口氣便了!」因在馬上摟起衣底,順袋中,還有四五兩碎銀子,都倒與二人。便道:「你兩個拿出去打酒吃,只要替我幹得停當,還謝你二人。」魯華那肯接,說道:「小人受你老人家恩還少哩!我只道叫俺兩個往東洋大海裡,拔蒼龍頭上角,西華岳山中,取猛虎口中牙,便去不得,這些小之事,有何難哉!這個銀兩,小人斷不敢領受!」西門慶道:「你不收,我也不央及你了!」教玳安接了銀子,打馬就走。又被張勝攔住,說:「魯華,你不知他老人家性兒,你不收,恰似咱每推托的一般!」一面接了銀子,扒倒地下磕了個頭,說道:「你老人家只顧家去坐著,不消兩日,管情穩抇抇,教你笑一聲。」張勝道:「只望官府到明日,把小人送與提刑所夏老爹那裡答應,就勾了小人了。」西門慶道:「這個不打緊,何消你說!」看官聽說,後來西門慶果然把張勝送在夏提刑守備府,做了個親隨。此係後事,表過不題。那兩子搗子,得了銀子,依舊耍錢去了。西門慶騎馬進門來家,已是日西時分,月娘等眾人聽見他進門,都往後邊去了;只有金蓮在捲簾內,看收家火。西門慶不往後邊去,逕到花園裡來,見婦人在亭子上收家火,便問:「我不在,你在這裡做甚麼來?」金蓮笑道:「俺每今日和大姐開門看了看,誰知你來的恁早?」西門慶道:「今日夏大人費心,庄子上叫了四個唱的,四個搗倒小廝,只請了五位客到。我恐怕路遠,來的早。」婦人與他脫了衣裳,因說道:「你沒酒,教丫頭看酒來你吃。」西門慶分付春梅:「把別的菜蔬都收下去,只劉下幾碟細菓子兒,篩一壺葡萄酒 來我吃。」坐在上面椅子上。因看見婦人上穿沉香色水緯羅對衿衫兒,五色縐紗眉子。下著白碾光絹桃線裙子,裙邊大紅光素段子,白綾高底羊皮金雲頭鞋兒。頭上銀絲䯼髻,金廂玉蟾宮折桂分心,翠梅鈿兒,雲鬢簪著許多花翠,越顯出紅馥馥朱唇,白膩膩粉臉,不覺淫心輙起,纔著他兩隻手兒,摟抱在一處親嘴。不一時,春梅篩上酒來,兩個一遞一口兒,飲酒砸舌,砸的舌頭一片聲響。婦人一面摟起裙子,坐在身上,噙酒哺在他口裡,然後在桌上,纖手拈了個鮮蓮蓬子,與他吃。西門慶道:「澀剌剌的,吃他做甚麼?」婦人道:「我的兒,你就吊了造化了,娘手裡拿的東西兒,你不吃?」于是口中噙了一粒鮮核桃仁兒,送與他,纔罷了。西門慶又要翫弄婦人的胸乳,婦人一面摘下〈扌寨〉領子的金三事兒來,用口咬著,攤開羅衫。露見美玉無瑕,香馥馥的酥胸,緊就就的香乳,揣揣摸摸良久,用口犢之,彼此調笑,曲盡于飛。西門慶乘著喜歡,向婦人道:「我有一件事告訴你,到明日教你笑一聲,你道蔣太醫開了生藥舖,到明日管情教他臉上開菓子舖來!」婦人便問:「怎麼緣故?」西門慶悉把今日門外,撞遇魯華、張勝二人之事,告訴了一遍。婦人笑道:「你這個墮業的眾生,到日日不知作多少罪業?」又問:「這蔣太醫,不是常來咱家看病的那蔣太醫?我見他且是謙恭禮體兒的,見了人把頭兒低著,可憐見兒的,你這等作他!」西門慶道:「你看不出他。你說他低著頭兒,他專一看你的腳哩。」婦人道:「汗邪的油嘴!他可可看人家老婆的腳?」西門慶道:「你還不知他哩!也是左近一個人家,請他看病,正是街上買了一尾魚手提著,見那人請他,說:『我送了魚到家就來。』那人說:『家中有緊病,請師父就去罷!』這蔣竹山一直跟到他家。病人在樓上,請他上樓,不想是個女人不好。素體容粧,走在房來,舒手教他把脉。這廝手把著脉,想起他魚來,掛在簾鈎兒上,就忘記看脉。只顧且問:『嫂子,你下邊有貓兒也沒有?』不想他男子漢,在屋裡聽見了,走來探著毛,打了個臭死,藥錢也沒有與他,把衣服扯的稀爛,得手纔跑了。」婦人道:「可可兒的來,我不信一個文墨人兒,他幹這個營生?」西門慶道:「你看他迎面兒,就誤了勾當,單愛外裝老成,內藏奸詐!」兩個說笑了一回,不吃酒了,收拾了家火,歸房宿歇,不在話下。按下一頭,都說李瓶兒招贅了蔣竹山,約兩月光景,初時蔣竹山圖婦人喜歡,修合了些戲藥部,門前買了些甚麼景東人事,美女相思套之類,實指望打動婦人心。不想婦人曾在西門慶手裡,狂風驟雨都經過的,往往幹事不稱其意,漸漸頗生憎惡。反被婦人把淫器之物,都用石砸的稀爛,都丟吊了。又說:「你本蝦鱔,腰裡無力:平白買將這行貨子來戲弄老娘家!把你當塊肉兒,原來是個中看不中吃,鑞鎗頭,死王八!」罵的竹山狗血噴了臉。被婦人半夜三更,趕到前邊舖子裡睡;于是一心只想西門慶,不許他進房中來。每日聐聒著算帳,查算本錢。這竹山正受了一肚氣,走在舖子小櫃裡坐的,只見兩個來進來,吃的浪浪蹌蹌,楞楞睜睜,走在凳子上坐下。先是一個問道:「你這舖子有狗黃沒有?」竹山笑道:「休要作戲,只有牛黃,那討狗黃?」又問:「沒有狗黃,你有冰灰也罷,拿來我瞧,我要買你幾兩」。竹山道:「生藥行只有冰片,是南海波斯國地道出的,那討冰灰來?」那一個說道:「你休問他,量他纔開了幾日舖子,他那裡有這兩庄藥材?咱往西門大官人舖中買去了來!」那個說道:「過來!咱與他說正經話罷!蔣二哥,你休推睡裡夢裡!你三年前死了娘子兒,問這位魯大哥借的那三十兩銀子,本利也該許多,今日問你要來了。俺剛才進門,就先問你要,你在人家招贅了,初開了這個舖子,恐怕喪了你行止,顯的俺每陰騭了。故此先把幾句風話來教你認範,你不認範,他這銀子你少不得還他!」竹山聽了,諕了個立睜,說道:「我並沒借他什麼銀子。」那人道:「你沒借銀,都問你討?自古蒼蠅不鑽那沒縫的彈,快休說此話!」蔣竹山道:「我不知閣下姓甚名誰,素不相識,如何來問我要銀子?」那人道:「蔣二哥,你就差了!自古於官不貧,賴債不富。想著你當初不得地時,串鈴兒賣膏藥,也虧這位魯大哥扶持你,今日就到了這步田地來。」這個人道:「我便姓魯,叫做魯華。你某年借了我三十兩銀子,發送妻小,本利該我四十八兩銀子,少不得還我。」竹山慌道:「我那裡借你銀子來?就借了你銀子,也有文書保人。」張勝道:「我就是保人。」因向袖中取出文書,與他照了照。把竹山氣的臉臘查也似黃了,罵道:「好殺材,狗男女,你是那裡搗子?走來嚇詐我!」魯華聽了,心中大怒,隔著小櫃,風乍的一拳去,早飛到竹山面門上,就把鼻子打歪在半邊,一面把架上藥材撒了一街。竹山大罵:「好賊搗子!你如何來搶奪我貨物?」只叫天福兒來幫助,被魯華一腳踢過一邊,那裡再敢上前。張勝把竹山拖出小櫃來,攔住魯華手,勸道:「魯大哥,你多日子也耽待了,再寬他兩日兒,教他湊過與你便了。蔣二哥,你怎麼說?」竹山道:「我幾時借他銀子來?就是問你借的,也等慢慢好講,如何這等撒野?」張勝道:「蔣二哥,你這回吃了橄欖灰兒,回過味來了!打了你一面口袋,倒過醮來了。你若好好早這般,我教魯大哥饒讓你些利錢兒,你便兩三限湊了還他,纔是話。你如何把硬話兒不認,莫不人家就不問你要罷?」那竹山聽了道:「氣殺我,我和他見官去!誰見他甚麼錢來?」張勝道:「你又吃了早酒了!」不隄防魯華又是一拳,仰八叉跌了一交,臉不倒裁入洋溝裡,將髮散開,巾幘都污濁了。竹山大叫青天白日起來,被保甲上來,都一條繩子拴了。李瓶兒在房中聽見外邊人攘,走來簾下聽覷。見地方拴的竹山去了,氣了個立睜。使出馮媽媽來,把牌面幌子都收了;街上藥材被人搶了許多,一面關閉了門戶,家中坐的。早有人把這件事,報與西門慶知道。即差人分付地方,明日早解提刑院,這裡又拿帖子,對夏大人說了。次日早帶上人來,夏提刑陞聽,看了地方呈狀,叫上竹山去,問道:「你可是蔣文蕙?如何借了魯華銀子不還,反行毀罵他?其情可惡!」竹山道:「小的通不認得此人,並沒借他銀子。小人以理分說,他反不容,亂行踢打,把小人貨物都搶了。」夏提刑便叫魯華:「你怎麼說?」魯華道:「他原借小的銀兩,發送妻喪,至今三年光景,延挨不還小的;小的今日打聽他在人家招贅了,做了大買賣,問他理討,他倒百般辱罵小的,說小的搶奪他貨物。見有他借銀子的文書在此,這張勝便是保人,望爺查情!」一面懷中取出文契,遞上去。夏提刑展開觀看,上面寫著:

「立借契人蔣文蕙,係本縣醫師為因妻喪,無錢發送,憑保人張勝,借到魯名下白銀三十兩,月利三分,入手用度。約至次年本利交還,如有欠少時,家值錢物件折准。恐後無憑,立此借契為照者。」

夏提刑看了,拍案大怒,說道:「可又來,見有保人文契,還這等抵賴!看這廝咬文嚼字模樣,就相個賴債的!」喝令左右:「選大板,拿下去著實打!」當下三、四個人,不由分說,拖番竹山在地,痛責三十大板,打的皮開肉綻,鮮血淋漓。一面差兩個公人,拿著白牌,押蔣竹山到家,處三十兩銀子,交還魯華;不然,帶回衙門收監。那蔣竹山打的那兩隻腿剌八着,走到家哭哭啼啼哀告李瓶兒,問他要銀子,還與魯華。又被婦人噦在臉上,罵道:「沒羞的王八!你遞什麼銀子在我手裡?問我要銀子。我早知你這王八砍了頭是個債樁,就瞎了眼,也不嫁你這中看不中吃的王八!」那四個人,聽見婦人屋裡攘罵,不住催逼,叫道:「蔣文蕙既沒銀子,不消只管挨遲了,趁早到衛門回話去罷。」竹山一面出來安撫了公人,又去裡邊哀告婦人。直撅兒跪在地下,哭哭啼啼,說道:「你只當積陰騭,西山五舍齋僧布施這三十兩銀子了!不與,這一回去,我這爛屁股上,怎禁的拷打?就是死罷了!」婦人不得已,那三十兩雪花銀子與他,當官交與魯華,扯碎了文書,方纔了事。這魯華、張勝得了三十兩銀子,逕到西門慶家回話了。西門慶留在捲棚內,管待二人酒飯,把前事告訴一遍。西門慶滿心大喜,說:「二位出了我口氣,足可以勾了。」魯華把三十兩銀子交與西門慶,門慶那裡肯收:「你二人收去買壺酒吃,就是我酬謝你了,後頭還有事相煩。」二人臨起身,謝了又謝,拿著銀子,自行耍錢去了。正是:

「嘗將壓善欺良意,  權作尤雲殢雨心。」

都說蔣竹山提刑院交了銀子出來,歸到家中。婦人那裡容他住,說道:「你還是那人家哩,只當奴害了汗病,把這三十兩銀子,問你討了藥吃了!你趁早與我搬出去罷;再遲些時,連我這兩間房子,尚且不勾你還人!」這蔣竹山自知存身不住,哭哭啼啼,忍著兩腿疼,自去另尋房兒。但是婦人本錢買買的貨物都留下,把他原舊的藥材、藥碾、藥篩、箱籠之物,即時摧他搬去,兩個就開交了。臨出門,婦人還使馮媽媽舀了一錫盆水,趕著潑去,說道:「喜得冤家離眼前!」當日打發了竹山出門,這婦人一心只想著西門慶,又打聽得他家中沒事,心中甚是後悔。每日茶飯慵餐,蛾眉懶畫,把門倚遍,眼兒望穿,白盼不見一個人兒來!正是:

「枕上言猶在,  于今恩愛淪,

房中人不見,  無語自消魂。」

不說婦人思想西門慶,單表一日玳安騎馬打門首經過,看見婦人大門關著,藥舖不開,靜落落的,歸告訴與西門慶,門慶道:「想必那矮王八打重了,在屋裡睡哩。會勝也得半個月出不來做買賣。」遂把這事情丟下了。一日,八月十五日,吳月娘生日,家中有許多堂客來,在大廳上坐。西門慶因與月娘不說話,一逕都來院中李桂姐家坐的,分付玳安:「早回馬去罷,晚上來接我。」旋邀了應伯爵、謝希大兩個來打雙陸。那日桂卿也在家,姐兒兩個在傍陪待、勸酒。良久,都出來院子內,投壺頑耍。玳安約至日西時分,勒馬來接。西門慶正在後邊東淨裡出恭,見了玳安,問道:「家中沒事?」玳安道:「家中沒事,大廳上坐堂客都散了,家火都收了。止有大妗子與姑奶奶眾人,大娘邀的後邊坐去了。今日獅子街花二娘那裡,使了老馮與大娘送生日禮來,四盤羹菓,兩盤壽桃麵,一疋尺頭,又與大娘做了一雙鞋。大娘與了老馮一錢銀子,說爹不在家了,也沒曾請去。」西門慶因見玳安臉紅紅的便問:「你那裡吃酒來?」玳安道:「剛纔二娘使馮媽媽叫了小的去,與小的酒吃,我說不吃酒,強說著,教小的吃了兩鐘,就臉紅起來。如今二娘到悔過來,對著小的,好不哭哩。前日我告爹說,爹還不信。從那日提刑所出來,就把蔣文蕙打發去了。二娘甚是後悔,一心還要嫁爹,比舊瘦了好些兒!央及小的好歹請爹過去,討爹示下。爹若吐了口兒,還教小的回他聲去。」西門慶道:「賊賤淫婦!既嫁漢子去罷了,又來纏我怎的?既是如此,我也不得閑去。你對他說,甚麼下茶下禮,揀個好日子,擡了那淫婦來罷。」玳安道:「小的知道了。他那裡還等著小的去回他話哩!教平安、畫童兒這裡伺候爹就是了。」西門慶道:「你去我知道了。」這玳安出了院門,一面走到李瓶兒那裡,回了婦人話。婦人滿心歡喜,說道:「好哥哥!今日多有累你對爹說,成就了二娘此事。」于是親自洗手剔甲,廚下整理菜蔬,管待玳安酒飯。說道:「你二娘這裡沒人,明日好歹你來幫扶天福兒,看著人搬家火過去。」顧了五六付扛,整擡運四、五日。西門慶也不對吳月娘說,都堆在新蓋的翫花樓上。擇了八月二十日,一頂大轎,一疋段子紅,四對燈籠,派定玳安、平安、畫童、來興四個跟轎,約後晌時分,方娶婦人過門。婦人打發了兩個丫鬟,教馮媽媽領著,先來了,等的回去,方纔上轎,把房子交與馮媽媽、天福兒看守。西門慶那日不往那去,在家新捲棚內,深衣福巾坐的,單等婦人進門。婦人轎子,落在大門首半日,沒個人出去迎接。孟玉樓走來上房,對月娘說:「姐姐,你是家主,如他已是在門首,你不去迎接迎接兒,惹的他爹不怪?他爹在捲棚內坐著,轎子在門首這一日子,沒個人出去,怎麼好進來的?」這吳月娘欲待出去接他,心中惱,又不下氣;欲待不出去,又怕西門慶性子不是好的。沉吟了一回,于是輕移蓮步,款蹙湘裙,出來迎接,婦人抱著寶瓶,逕往他那邊新房裡去了。迎春、綉春兩個丫鬟,又早在房中鋪陳停當,單等西門慶晚夕進房。不想西門慶正因舊惱在心,不進他房去。到次日,教他出來,後邊月娘房裡見面,分其大小排行他是六娘。一般三日擺大酒席,請堂客,會親吃酒,只是不往他房裡去。頭一日晚夕,先在潘金蓮房中睡,金蓮道:「他是個新人兒,纔來了頭一日,你就空了他房。」西門慶道:「你不知淫婦有些眼裡火,等我奈何他兩日,慢慢進去。」到了三日,打發堂客散了,西門慶又不進入他房中,往後邊孟玉樓房裡歇去了。這婦人見漢子一連三夜不進他房來,到半夜打發兩個丫鬟睡了,飽哭了一場,可憐走在牀上,用腳帶吊頸,懸梁自縊。正是:

「連理未諧鴛帳底,  冤魂先到九重泉!」

兩個丫鬟睡了一覺醒來,見燈光昏暗,起來剔燈,猛見牀上婦人吊著,諕慌了手腳,走出隔壁,叫春梅說:「俺娘上吊哩!」慌的金蓮起來,這邊看視。見婦人穿著一身大紅衣服,直捉捉吊在牀上,連忙和春梅把腳帶割斷,解救下來。撅了半日,吐了一口精涎,方纔甦醒。即叫春梅後邊快請你爹來。西門慶正在玉樓房中吃酒,還未睡哩。先是玉樓勸西門慶說道:「你娶將他來,一連三日不往他房裡去,惹他心中不歹麼?恰似俺每把這庄事放在頭裡一般,頭上末下,就讓不得這一夜兒。」西門慶道:「待過三日兒,我去。你不知道,淫婦有些吃著碗裡,看著鍋裡。想起來,你惱不過!我來曾你漢子死了,相交到如今,甚麼話兒沒告訴我?臨了,招進蔣太醫去了,我不如那廝?今日都怎的又尋將我來?」玉樓道:「你惱的是,他也吃人念了。」正說話間,忽聽一片聲打儀門。玉樓使蘭香問,說:「是春梅來請爹,六娘在房裡上吊哩!」慌的玉樓攛掇西門慶不迭,便道:「我說教你進他房中走走,你不依,只當弄出事來。」于是打著燈籠,走來前邊看視。落後吳月娘、李嬌兒聽見,都起來,到他房中。見金蓮摟著他坐的,說道:「五姐,你灌了他些姜湯兒沒有?」金蓮道:「我救下來時,就灌了些來了。」那婦人只顧喉中哽咽了一回,方哭出聲。月娘眾人,一塊石頭纔落地。好好安撫他睡,各歸房歇息。次日,晌午前後,李瓶兒纔吃些粥湯兒。正是:

「身如五鼓啣山月,  命似三更油盡燈。」

西門慶向李嬌兒眾人說道:「你每休信那淫婦裝死兒諕人。我手裡放不過他,到晚夕等我進房裡去,親看著他上個吊兒,我瞧方信,不然,吃我一頓好馬鞭子!賊淫婦,不知把我當誰哩!」眾人見他這般說,都替李瓶兒捏兩把汗。到晚夕,見西門慶袖著馬鞭子,進他房中去了。玉樓、金蓮分付春梅把門關了,不許一個人來。都立在角門兒外,悄悄聽覷,看裡面怎的動靜。且說西門慶見婦人在牀上,倒胸著身子哭泣,見他進去,不起身,心中就有幾分不悅;先把兩個丫頭,都趕去空房裡住了。西門慶走來,椅子上坐下,指著婦人罵道:「淫婦!你既然虧心,何消來我家上吊?你跟著那矮王八過去便了!誰請你來?我又不曾把人坑了你什麼,緣何流那〈毛皮〉尿怎的?我自來不曾見人上吊,我今日看著你上個吊兒我瞧!」于是拿一繩子丟在他面前,叫婦人上吊。那婦人想起蔣竹山說的話來,說西門慶打老婆的班頭,降婦女的領袖。思量:「我那世裡晦氣?今日大睜眼,又撞入火炕裡來了。」越發煩惱痛哭起來。這西門慶心中大怒,教他下牀來,脫了衣裳跪著。婦人只顧延挨不脫,被西門慶拖翻在牀地平上,袖中取出鞭子來,抽了幾鞭子,婦人方纔脫去上下衣裳,戰兢兢跪在地平上。西門慶坐著,從頭屋尾問婦人:「我那等對你說過,教你略等等兒,我家中有些事兒;如何不依我,慌忙就嫁了蔣太醫那廝?你嫁了別人,我倒也不惱!那矮王八有甚麼起解?你把他倒踏進門,去拿本錢與他開舖子,在我眼皮子根前開舖子,要撑我的買賣!」婦人道:「奴不說的,悔也是遲了。只因你一去了不見來,把奴想的心斜了;後邊喬皇親花園裡,常有狐狸,要便半夜三更,假名托姓變做你,來攝奴精髓,到天明雞叫時分就去了,你不信,只問老馮和兩個丫頭,便知端的。後來把奴攝的看看至死,不久身亡。纔請這蔣太醫來看,恰吊在麵糊盆內一般,乞那廝局騙了;說你家中有事,上東京去了。奴不得已,纔幹下這條路。誰知這廝,砍了頭是個債樁,被人打上門來,經管動府;奴忍氣吞聲,丟了幾兩銀子,吃奴即時攆出去了。」西門慶道:「說你教他寫狀子,告我收著你許多東西,你如何今日也到我家來了!」婦人道:「你麼,可是沒的說。奴那裡有這個話,就把身子爛化了!」西門慶道:「就算有如此,我也不怕你,道說你有錢,快轉換漢子,我手裡容你不得!我實對你說罷。前者打太醫那兩個人,是如此如此,這般這般,使的手段。只略施行計,教那廝疾走無門;若稍用機關,也要連你掛了到官,弄到一個田地!」婦人道:「奴知道是你使的計兒。還是你可憐見奴,若弄到那無人烟之處,就是死罷了!」看看說的西門慶怒氣消下些來了,又問道:「淫婦你過來,我問你,我比蔣太醫那廝誰強?」婦人道:「他拿甚麼來比你,你是個天,他是塊磚,你在三十三天之上,他在九十九地之下。休說你仗義疎財,敲金擊玉,伶牙俐齒,穿羅著錦,行三坐五,這等為人上之人。自你每日吃用稀奇之物,他在世幾百年,還沒曾看見哩!他拿甚麼來比你?你是醫奴的藥一般,一經你手,教奴沒日沒夜,只是想你。」自這一句話,把西門慶歡喜無盡,即丟了鞭子,用手把婦人拉將起來,穿上衣裳,摟在懷裡,說道:「我的兒,你說的是。果然這廝他見甚麼碟兒天來大!」即叫春梅:「快放卓兒,後邊快取酒菜來。」正是:

「東邊日頭西邊雨,  道是無情卻有情。」

果竟未知後來如何,且聽下回分解:




\end{showcontents}
