%# -*- coding: utf-8 -*-
%!TEX encoding = UTF-8 Unicode
%!TEX TS-program = xelatex
% vim:ts=4:sw=4
%
% 以上设定默认使用 XeLaTex 编译,并指定 Unicode 编码,供 TeXShop 自动识别

%第四十二回 
\chapter{豪客攔鬥玩煙火\KG 貴家高樓醉賞燈}


\begin{showcontents}{}



「星月當空萬燭燒,  人間天上雨元宵,

樂和春奏聲偏好,  人蹈衣歸馬亦嬌

易老韶光休浪度,  最公白髮不相饒,

千金博得斯須刻,  分付譙更仔細敲。」

話說西門慶打發喬家去了,走來上房和月娘、大妗子、李瓶兒商議。月娘道:「他家既先來與咱家孩子送節,咱少不的也買禮過去,與他家長姐近節,就權為插定一般。」月娘道:「庶不差了禮數。」大妗子道:「咱這裡少不的立上個媒人,往來方便些。」月娘道:「是孔嫂兒,咱家安上誰好?」西門慶道:「一客不煩二主,就安上老馮罷!」于是連忙寫了請帖八個,就叫了老馮來,教他同玳安拿請帖盒兒,十五日請喬老親家母、喬五太太,并尚舉人娘子,朱序班娘子、崔親家母、段大姐、鄭三姐來赴席,與李瓶兒做生日,并吃看燈酒。一面分付來興兒拿銀子早往糖餅舖,早定下蒸酥點心,多用大方盤,要四盤蒸餅,兩盤菓餅,團圓餅 ,兩盤玫瑰元宵餅 ;買四盤鮮菓;一盤李乾、一盤胡桃,一盤龍眼,一盤荔枝;四盤羹肴:一盤燒鵝 、一盤燒雞、一聲鴿子兒、一盤銀魚乾;兩套遍地錦羅段衣服,一件大紅小袍兒、一頂金絲縐紗冠兒,兩盞雲南羊角珍燈,一盒衣翠,一對小金手鐲,四個金寶石戒指兒。應伯爵來講李智、黃四官銀子事,看見,問其所以。西門慶告訴與喬大戶結親之事:「十五日好歹請令正來陪親家坐的。」伯爵道:「嫂子呼喚,房下必定來。」西門慶道:「今日請眾堂官娘子吃酒。」十四日早裝盒擔,教女婿陳經濟和賁四穿青衣服,押送過去。喬大戶那邊,酒筵管待,重加答賀。回盒中回了許多生活鞋腳,俱不必細說。且說那日院中吳銀兒,先送了禮來,買了一盤壽桃、一盤壽麵、兩隻燒鴨 、一副豕蹄、兩方綃金汗巾、一雙女鞋來,與李瓶兒上壽,就拜乾兒相交。月娘收了禮物,打發轎子回去。李桂姐只到次日纔來。見吳銀兒在這裡,悄悄問月娘:「他多咱來了?」月娘如此這般告他說:「昨日送了禮來,拜認你六娘做乾女兒了。」李桂姐聽了,一聲兒沒言語,一日只和吳銀兒使性子,兩個不說話。都說前廳有王皇親家二十名小廝唱戲,挑了廂子來,有兩名師父領着,先與西門慶磕頭。西門慶吩咐西廂房做戲房,管待酒飯。堂客到時,吹打迎接。大廳上玳筵齊整,錦茵匝地。先是周守備娘子、荊都監母親荊太太與張團練娘子先到了,俱是大轎,排軍喝道,家人媳婦跟隨。裡邊月娘、眾姊妹,多穿着袍出來迎接,至後廳敘禮,與眾親相畢,讓坐遞茶。等着夏提刑娘子到,纔擺茶。不料等的日中,還不見來。小廝邀了兩三遍,約午後時分,纔喝了道來,擡着衣匣,家人媳婦跟隨,許多僕從擁護。鼓樂接進去後廳與眾堂客見畢禮數,依次席坐下。先在捲棚內擺茶,然後大廳上坐。春梅、玉簫、迎春、蘭香,都是雲髻珠子纓絡兒,金燈籠墜,墜遍地錦比甲,大紅段袍,翠藍織金裙兒。惟春梅寶石墜子,大紅遍地錦比甲兒,席上捧茶斟酒。那日王皇親家樂扮的是西廂記。不說畫堂深處,珠圍翠繞,歌舞吹彈飲酒。單表西門慶那日打發堂客這里上茶,就騎馬約下應伯爵、謝希大往獅子街房裡去了。分付四架烟火,拿一架那裡去;晚夕堂客根前放兩架。那裡樓上,設放圍屏卓席,掛上燈,旋叫了個廚子,生了火,家中抬了兩食盒下飯菜蔬、兩壜金華酒 ,叫了兩個唱的,董嬌兒、韓玉釧兒。原來西門慶先使玳安顧下轎子,請王六兒同往獅子街房裡去。見婦人:「爹說請韓大嬸那裡晚夕看放烟火。」那婦人笑道:「我羞刺刺,怎麼好去哩!你韓大叔知道不嗔?」玳安道:「爹對韓大叔說了,教你老人家快收拾哩。若不是,使了老馮來請你老人家。今日各宅眾奶奶吃酒,六姐見他看哥兒那裡抹嘴去。見爹巴巴使了我來。因叫了兩個唱的,沒人陪他。」那婦人聽了,還不動身。一回只見韓道國來家,玳安道:「這不是韓大叔來了!韓大嬸這裡不信我說哩!」婦子向他漢子說:「真個教我去?」韓道國道:「老爹再三說,兩個唱的,沒人陪他,請你過去,晚夕就看放烟火。等你,還不收拾哩!剛纔教我把鋪子也收了,就晚夕一搭兒裡坐坐。保官兒也往家去了,晚夕該他上宿哩。」婦人道:「不知多咱纔散?你到那裡坐回就來罷。家裡沒人,你又不該上宿。」說畢,打扮穿了衣服,玳安跟隨,逕到獅子街房裡。來昭妻一丈青,又早在房裡收拾乾淨,下床炕帳幔褥被,多是見成的。安息沉香,薰的噴鼻香。房裡吊着兩盞紗燈,地平上火盆裡籠着一盆炭火。婦人走到裡面炕上坐下。良久,來昭妻一丈青走出來,遭了萬福,拿茶吃了。西門慶與應伯爵看了回燈,纔到房子裡,兩個在樓上打雙陸。樓上掛了六扇窗戶,掛着簾子,下邊就是燈市,十分熱鬧。打了回雙陸,收拾擺飯吃了,二人在簾裡觀看燈市。但見:

「萬井人烟錦綉圍,  香車駿馬鬧如雷;

鰲山聳出青雲上,  何處遊人不看來。」

伯爵因問:「明日喬家那頭幾位人來?」西門慶道:「有他家做皇親家五太太,明日我又不在家。早辰從廟中上元醮,又是府裡周菊軒那裡請吃酒。西門慶見人叢裡謝希大、祝日念,同一個戴方巾的在燈棚下看燈,指與伯爵瞧,因問:「那戴方巾這個人,你不認的他?如何跟着他一答兒裡走?」伯爵道:「此人眼熟,不認的他。」西門慶便叫玳安:「你去下邊悄悄請了謝爹來,休教祝麻子和那人看見。」玳安小廝是眼裡說話的賊,一直走下樓來,挨到人鬧裡,待祝日念和那人先過去了,從傍邊出來把謝希大拉了一把。慌的希大回身觀着,都是他。玳安道:「爹和應二爹在這樓上,請謝爹說話。」希大道:「你去,知道了。等陪他兩個到粘梅花處,就去見你爹。」玳安便一道烟去了。不想到了粘梅花處,這希大向人鬧處就扠過一邊,由着祝日念和那一個人只顧哩尋他。便走來樓上,見西門慶、應伯爵二個作揖。因說道:「哥來此看燈,早辰就不說,呼喚兄弟一聲。」西門慶道:「我早辰對眾人不好邀你每的。已托應二哥到你家請你去,請你不在家。剛纔祝麻子沒看見你這裡來?」因問:「那戴方巾的是誰?」希大道:「那戴方巾的是王昭宣府裡王三官兒。今日和祝麻子到我家,央我同許不與先生那裡借三百兩銀子,央我和老孫、祝麻子作保,要幹前程入武學肄業。我那裡管他這閒帳!剛纔陪他燈市裡走了走,聽見哥使盛价呼喚,我只伴他到粘梅花處,交我乘人亂,就扠開了,走來見哥。」因問伯爵:「你來多大回了?」伯爵道:「哥使我先到你家,你不在,我就來了。和哥在這裡打了這回雙陸。」西門慶問道:「你吃了飯不曾?叫小廝拿飯來你吃。」謝希大道:「可知道哩!早辰從哥那裡出來,和他兩個搭了這一日,誰吃飯來?」西門慶分付玳安:「廚下安排飯來,與你謝爹吃。」不一時搽抹卓兒乾淨,就是春盤子菜,兩碗稀爛下飯,一碗〈火川〉肉粉湯 ,兩碗白米飯。希大獨自一個吃了裏外乾淨,剩下些汁湯兒,還泡了碗吃了。玳安收下家活去。希大在傍看著兩個打雙陸。只見兩個唱的,門首下了轎子,擡轎的各提着衣裳包兒笑進來。伯爵早已在窗裏看見,說道:「兩個小淫婦兒,這咱纔來。」分付玳安:「且別教他往後邊去,先叫他樓上來見我。」希大道:「今日叫的是那兩個?」玳安道:「是董嬌兒、韓玉釧兒。」忙下樓說道:「應二爹叫你說話。」兩個那裡肯來,一直往後走了。見了一丈青拜了,引他入房中。看見王六兒頭上戴着時樣扭心{髟狄}髻兒,羊皮金箍兒,身上穿紫潞紬襖兒,玄色一塊瓦領披襖兒,白桃線絹裙子,下邊顯着趫趫兩隻金蓮,穿老鴉段子紗綠鎖線的平底鞋兒,拖的水鬢長長的,紫膛色不十分搽鉛粉,學個中人打扮,耳邊帶着了香兒;進門只望着他,拜了一拜,多在炕邊頭坐了。小鐵棍拿茶來,王六兒陪着吃了。兩個唱的,上上下下,把眼只看他身上。看一回,兩個笑一回,更不知是什麼人。落後玳安進來,兩個唱的悄悄問他每:「房中那一位是誰?」玳安沒的回答,只說是俺爹大姨人家,接來這看燈。兩個聽的進房中,從新說道:「俺每頭裡不知是大姨,沒曾見的禮,休怪。」于是插燭磕了兩個頭。慌的王六兒連忙還下半禮。落後擺上湯飯來,陪着同吃,兩個拿樂器,又唱與王六兒聽。伯爵打了雙陸,下樓來小淨手。聽見後邊唱,點手兒叫過玳安,問道:「你告我說,兩個唱的在後邊唱與誰聽?」玳安只是笑,不做聲,說道:「你老人家,曹州兵備,好管事寬。唱不唱,管他怎的?」伯爵道:「好賊小油嘴!你不和我說,愁我不知道?」玳安笑道:「你老人家,知道罷了!又問怎的?」說畢,一直往後走了。伯爵上的樓來,西門慶又與謝希大打了三貼雙陸。只見李銘、吳惠兩個驀地上樓來磕頭。伯爵道:「好呀!你兩個來的正好。在那裡來?怎知道俺每在這裡?」李銘跪下,掩口說道:「小的和吳惠先到宅裡來,宅裡說爹每在這邊房子裡擺酒,前來伏侍爹們。西門慶道:「也罷!你起來伺候。玳安,快往對門請你韓大叔去。」不一時,韓道國到了,作了揖坐下。一面收拾收放卓兒,廚下拿春盤案酒來,琴童便在旁邊用銅布甑兒篩酒。伯爵與希大居上,西門慶主位,韓道國打橫坐下,把酒來斟。一面使玳安後邊請唱的去。少頃,韓玉釧兒,董嬌兒兩個慢條斯禮上樓來,望上不當不正磕下頭去。伯爵罵道:「我道是誰來,原來是這兩個小淫婦兒!頭裡知道我在這裡,我叫着怎的不先來見我?這等大膽,到明日一家不與你個功德,你也不怕。」董嬌兒笑道:「哥兒,那裡隔墻掠見腔兒,可不把我諕殺!」韓玉釧道:「你知道愛奴兒掇着獸頭城以裡掠,好個丟醜兒的孩兒。」伯爵道:「哥,你今日忒多餘了。有了李銘、吳惠在這裡唱罷了,又要這兩個小淫婦做什麼?還不趁早打發他走。大節夜還趕幾個錢兒。等住回晚,越發沒人要了!」韓玉釧兒道:「哥兒,你怎的沒着?大爹叫了俺每來答應,又不伏侍你。哥,你怎的閑出氣?」伯爵道:「俊傻小剌骨兒,你見在這裡,不伏侍我,你說伏侍誰?韓玉釧道:「唐胖子吊在醋缸裡,把你撅酸了。」伯爵道:「賊小淫婦兒,是撅酸了我。等住回散了家去時,我和你答話!我左右有兩個法兒,你原出得我手。」董嬌兒道:「哥兒,那裡兩個法兒?說來我聽!」伯爵道:「我頭一個兒,對巡捕說了,拿你犯夜。到第二日,我拿個拜帖兒對你周爺說,拶你一頓好拶子。十分不巧,只消三分銀子燒酒,把擡轎的灌醉了,隨你這小淫婦去。天晚,到家沒錢,不怕鴇不打,管我腿事。」韓玉釧道:「十分晚了,俺每不去,在爹這房裡睡。再不,教爹這裡差人送俺每。王媽媽支錢一百文,不在于你。好淡嘴女又十撇兒。」伯爵道:「我是奴才,如今年程欺保了!」拏三道三,說笑回,兩個唱的在傍彈唱了春景之詞。眾人纔拿起湯飯來吃。只見玳安兒走來,報道:「祝爹來了。」眾人多不言語。不一時,祝日念上的樓來,看見伯爵和謝希大在上面,說道:「你兩個好吃,可成個人!」因說:「謝子純,哥這裡請你,也對我說一聲兒。三不知就走的來,教我只顧在粘梅花處那裡尋你。」希大道:「我也是誤行,纔撞見哥在樓上和應二哥打雙陸。走上來作揖,被哥留住了。」西門慶因令玳安兒:「拏椅兒來,和我祝兄弟在下邊坐罷。」于是安放鍾筯,在下席坐了。廚下拿了湯飯上來,一齊同吃。西門慶只吃了一個包兒,呷了一口湯,因見李銘在旁,都遞與李銘遞下去吃了。那應伯爵、謝希大、祝日念、韓道國每人青花白地吃一大深碗八寶攢湯 ,三個大包子。還零四個挑花燒賣 ,只留了一個包兒壓碟兒。左右收下湯碗去,斟上酒來飲酒。希大因問祝日念道:「你陪他還到那裡纔拆開了?怎知道我在這裡?」祝日念于是如此這般告說:「我因尋了你一回尋不着,就同王三官到老孫會了,往許不與先生那里借三百兩子去。乞孫寡嘴老油嘴,把借契寫差了。」希大道:「你每休寫上我,我不管。左右是你與老孫作保,討保頭錢使。」因問:「怎的寫差了?」祝日念道:「我那等分付他,寫了文書滑着些,立與他三限纔還他這銀子。不依我,教我從新把文書又改了。」希大道:「你文書上怎麼寫着?念一遍我聽。」祝日念道:「依着了我這等寫:

立借契人王寀,係招宣府舍人,休說因為要錢使用,只說要錢使用。憑中見人孫天化、祝日念作保,借到許不與先生名下,不要說白銀軟斯金三百兩,每月休說利錢,只說出納梅兒五百文。約至次年交還。別要題次年,只說約至三限交還。那三限?頭一限,風吹轆軸打孤雁;第二限,水底魚兒跳上岸;第三限,水裡石頭泡得爛;這三限交還他。平白寫了垓子點頭那一年纔還他。我便說,垓子點頭,倘忽遇着一年他動,怎了?教我改了兩句,說道:如借債人東西不在,代保人門面南北躲閃。恐後無憑,立此文契不用。到後又批了兩個字:後空。」

謝希大道:「你這等寫著,還說不滑稽?及到水裡石頭爛了時,知他和尚在也不在?」祝日念道:「你到說的好,有一朝天旱水淺,朝廷挑河,把石頭乞做工的夫子兩三鐝頭,坎得稀爛,怎了?那裡少不的還他銀子。」眾人說笑了一回。看看天晚,西門慶分付樓上點起燈,又樓簷前一邊一盞羊角玲燈,甚是奇巧。不想家中月娘使棋童兒和排軍人擡送了四個攢盒,多是美口糖食,細巧菓品;也有黃烘烘金橙,紅馥馥石榴,甜蹓蹓橄欖,青翠翠蘋婆,香噴噴水梨;又有純蜜蓋柿 ,透糖大棗 ,酥油松餅 ,芝麻蔴象眼 ,骨牌減煠 、蜜潤條環 ;也有柳葉糖 ,牛皮纏 ;端的世上稀奇,寰中少有。西門慶叫棋童兒向前問他:「家中眾奶奶們,散了不曾?還在那裡吃酒?誰使你送來?」棋童道:「大娘使小的送來,與爹這邊下酒。眾奶奶們還未散哩。戲文扮了四摺,大娘留住大門首吃酒,看放烟火哩。」西門慶問:「有人看沒?」棋童道:「擠圍滿街人看。」西門慶道:「我分付下平安兒,留下四名青衣排軍,拏欄杆在大門首攔人伺候,休放閑雜人挨擠。」棋童道:「小的與平安兒兩個同排軍,多看放了烟火。眾奶奶們七八散了,大娘纔使小的來了,並沒閑雜人攪擾。」西門慶聽了,分付把桌上飲饌,多搬下去,將攢盒擺上。廚下拏上一道菓餡元宵來。兩個唱的,在度前遞酒。西門慶分付棋童回家看去。一面重篩美酒,再設珍羞,教李銘、吳惠席前彈唱了一套燈詞雙調新水令:

「鳳城佳節賞元宵,遶鰲山瑞雲籠罩。見銀河星皎潔,看天塹月輪高,動一派簫韶,開玳宴儘歡樂。」

〔川撥棹〕「花燈兒兩邊挑,更那堪一天星月皎。我到見綉帶風飄,寶蓋微搖;鰲山上燈光照耀,剪春蛾頭上挑。」

〔第七兄〕「一居廂舞著唱著共彈著,驚人的這百戲其實妙。動人的高戲怎生學,笑人的院本其實笑。」

〔梅花酒〕「呀!一壁廂舞鮑老,仕女每打扮的清標。有萬種妖嬈,更百媚千嬌。一壁廂舞迓鼓,一壁廂躧高撬,端的有笑樂。細氤氳,蘭麝飄,笑吟吟,飲香醪。」

〔喜江南〕「呀!今日喜孜孜開宴賞元宵,玉纖慢撥紫檀槽。燈光明月兩相耀,照樓臺殿開,今日個開懷沉醉樂淘淘。」

唱畢,吃了元宵,韓道國先往家去了。少頃,西門慶分付來昭將樓下開下兩間,吊掛上簾子,把烟火架擡出去。西門慶與眾人在樓上看,教王六兒陪兩個粉頭,和來昭妻一丈青,在樓下觀看。玳安和來昭將烟火安放在街心裏,須臾點着。那兩邊圍看的,挨肩擦膀,不知其數。都說:西門大官府,在此放烟火,誰人不來觀看?果然紮得停當,好烟火!但見:

「一丈五高花樁,四圍下山棚熱鬧,最高處一雙仙鶴,口裡啣著一封丹書,乃是一枝起火,起去萃山律;一道寒光,直鑽透斗牛邊。然後正當中,一個西瓜砲迸開,四下裡人物皆著。觱剝剝,萬個轟雷皆燎徹。彩蓮舫,賽月明,一個趕一個,猶如金燈沖散碧天星;紫葡萄,萬架千株,好似驪珠倒挂水晶簾泊。霸王鞭,到處响亮,地老鼠,串遶人衣。瓊盞玉臺,端的旋轉得好看;銀蛾金彈,施逞巧妙難移。八仙捧壽,名顯中通;七聖降妖,通身是火。黃烟兒,綠烟兒,氤氳籠罩萬堆霞;緊吐蓮,慢吐蓮,燦爛爭開十段錦。一丈菊,與烟蘭相對;火梨花,共落地桃爭春。樓基殿閣,頃刻不見巍峨之勢;村坊社鼓,彷彿難聞歡鬧之聲。貨郎擔兒,上下光焰齊明;鮑風車兒,首尾迸得粉碎。五鬼鬧判,焦頭爛額見猙獰;十面埋伏,馬到人馳無勝負。總然費卻萬般心,只落得火滅烟消成煨燼。」

「玉漏銅壺且莫催,  星橋火樹徹明開,

萬般傀儡皆成妄,  使得遊人一笑回。」

那應伯爵見西門慶有酒了,剛看罷烟火,下樓來,見六兒在這裡。推小淨手,拉着謝希大、祝日念,也不辭西門慶就走了。玳安便道:「二爹那裡去?」伯爵便向他耳邊,說道:「俊孩子,我頭裡說的那本帳,我若不起身,別人也只顧坐着,顯的就不趣了。等你爹問你,只說俺每多跑了。」落後西門慶見烟火放了,問伯爵等那裡去了?玳安道:「應二爹和謝爹多一路去了。小的攔不回來。多上覆爹。」西門慶就不再問了。因叫過李銘、吳惠來,每人賞了一大巨杯酒與他吃。分付:「我且不與你唱錢。你兩個到十六日早來答應。還是應二爹三個,并眾夥計當家兒,晚夕在門首吃酒。」李銘跪下道:「小的告稟爹,十六日和吳惠、左順、鄭奉三個,多往東平府新陞的胡爺那裡別任官身去,只到後晌纔得來。」西門慶道:「左右俺每晚夕纔吃酒哩!你只休誤了就是了。」二人道:「小的並不敢誤。」於是跪着吃畢酒,拜辭出門。西門慶分付:「明日家中堂客擺酒,李桂姐、吳銀姐多在這裡,你兩個好歹來走一走。」與兩個唱的,一同出門,不在話下。西門慶吩咐來昭、玳安、琴童看着收家活,滅息了燈燭,就往後邊房裡去了。」且說來昭兒子小鐵兒,正在外邊看放了烟火。見西門慶進去了,于是來樓上,見他爹老子棹一盤子雜合的肉菜,一甌子酒,和些元宵,拿到房裡,就問他娘一丈青手裡拏着燒胡鬼子,被他娘打了兩下。不妨他走在後邊院子裡頑耍,只聽正面房子裡笑聲,只說唱的還沒去哩。見房門關着,于是眼裡望裡張看,見房裡掌着燈燭。原來西門慶和王六兒兩個,在床沿子上行房。西門慶已有酒的人,把老婆倒按在床沿上,燈下褪去小衣,那話上使着托子,幹後庭花。一手一陣,往來〈扌扉〉打,何止數百回,〈扌扉〉打的連聲响亮,其喘息之聲,往來之勢,猶賽折床一般,無處不聽見。這小孩子正在那裡明覷,不妨他娘一丈青走來後邊,看見他孩子。揪着頭角兒,揪到那前邊,鑿了兩個栗爆。罵道:「賊禍根子!小奴才兒!你還少第二遭死!又往那裡聽他去。」于是與了他幾個元宵吃了,不放他出來,就嚇住他上炕睡了。西門慶和老婆足幹搗{入日}兩頓飯時,纔了事。玳安打發擡轎的酒飯吃了,跟送他到家,然後纔來。同琴童兩個,打着燈兒,跟西門慶家去。正是:

「不愁明月盡,  自有暗香來。」

有詩為證:

「南樓玩賞頓忘歸,  總有風流得幾時,

回來明月三更轉,  不覺歡娛醉似泥。」

畢竟未知後來如何,且聽下回分解:




\end{showcontents}


