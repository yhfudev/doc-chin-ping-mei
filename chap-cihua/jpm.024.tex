%# -*- coding: utf-8 -*-
%!TEX encoding = UTF-8 Unicode
%!TEX TS-program = xelatex
% vim:ts=4:sw=4
%
% 以上设定默认使用 XeLaTex 编译,并指定 Unicode 编码,供 TeXShop 自动识别

%第二十四回 
\chapter{陳經濟元夜戲嬌姿\KG 惠祥怒罵來旺婦}

「銀燭高燒酒乍醺,  當筵且喜笑聲頻,

蠻腰細舞章臺柳,  檀口輕歌上苑春;

香氣拂衣來有意,  翠微落地拾無聲,

不因一點風流趣,  安得韓生醉後醒。」

話說一日,天上元宵,人間燈夕,西門慶在家廳上張掛花燈,鋪陳綺席。正月十六,合家歡樂飲酒,正面圍着石崇錦帳圍屏,掛着三盞珠子吊燈,兩邊擺列着許多妙戲桌燈。西門慶與吳月娘居上坐,餘李嬌兒、孟玉樓、潘金蓮、李瓶兒、孫雪蛾、西門大姐,都在兩邊列坐;都穿着錦綉衣裳、白綾襖兒、藍裙子。惟有吳月娘穿着大紅遍地通袖袍兒、貂鼠皮襖,下百花袑;頭上珠翠堆盈,鳳釵半卸。春梅、玉筲、迎春、蘭香一般兒四個家樂,在傍擽箏歌板,彈唱燈詞。燭於東首設一席,與女婿陳經濟坐。一般三湯五割 ,食烹異品,菓獻時新。小玉、元宵、小亦埝鳥、綉春都在上面下來斟酒。那來旺兒媳婦宋惠蓮,不得上來坐,在穿廊下一張椅兒上,口裡磕瓜子兒,等的上邊呼喚要酒,他便揚聲叫:「來安兒、畫童兒,娘上邊要熱酒,快儹酒上來。賊囚根子!一個也沒在這裡伺候,多不知往那裡去了?」只見畫童盪酒上去。西門慶就罵道:「賊奴才!一個也不在這裡伺候,往那裡去來?賊少打的奴才!」小廝走來,說道:「嫂子,誰往那去來?就對着爹說要喝,教爹罵我!」惠蓮道:「上頭要酒,誰教你不伺候?關我甚事!不罵你罵誰?」畫童兒道:「這地上乾乾淨淨的,嫂子磕下恁一地瓜子皮,爹看見又罵了。」惠蓮道:「賊囚根子!六月債兒熱還得快,就是,甚麼打緊?教你彫佛眼兒!便當你不掃,丟着,另教個小廝掃。他問我,只說得一聲。」畫童兒道:「耶嚛,嫂子!將就些兒罷了。如何和我合氣!」于是取了苕帚來,替他掃瓜子皮兒。這宋惠蓮外邊磕瓜子兒不題。都說西門慶席上,見女婿陳經濟沒酒,吩咐潘金蓮連忙下來滿斟一杯酒,笑嘻嘻遞與經濟,說道:「姐夫,你爹吩咐,好歹飲奴這杯酒兒。」經濟一壁接酒,一面把眼兒不住斜溜婦人,說:「五娘,請尊便,等兒子慢慢吃!」婦人一徑身子把燈影着,左手執酒,剛待的經濟用手來接,右手向他手背只一捏。這經濟一面把眼瞧着眾人,一面在下戲把金蓮小腳兒上踢了一下。婦人微笑,低聲道:「怪油嘴,你丈人瞧着,待怎的?」看官聽說:兩個自知暗地裡調情頑耍,卻不知宋惠蓮這老婆又是一個兒,在槅子外窗眼裡,被他瞧了個不亦樂乎。正是:

「當局者迷,  傍觀者清。」

雖故席上眾人,到不曾看出來;都被他向窗隙燈影下,觀得仔細。口中不言,心下自思:「尋常時在俺每根前,到且提精細撇清;誰想暗地都和這小夥子兒勾搭。今日被我看出破綻,到明日再搜求我,是有話說。」正是:

「誰家院內白薔薇,  暗暗偷攀三兩枝;

羅袖隱藏人不見,  馨香惟有蝶先知。」

飲酒多時,西門慶忽被應伯爵差人請去賞燈吃酒去了。吩咐月娘:「你們自在頑耍,我往應二哥家吃酒去來。」玳安、平安兩個小廝跟隨去了。月娘與眾姊妹吃了一回,但見銀河清淺,珠斗爛班,一輪團圓皎月,從東而出,照得院宇猶如白晝。婦人或有房中換衣者,或月下整粧者,或有燈前戴花者;惟有玉樓、金蓮、李瓶兒三個并惠蓮,在廳前看經濟放花兒。李嬌兒、孫雪蛾、西門大姐,都隨月娘後邊去也。金蓮便向二人說道:「他爹今日不在家,咱對大姐姐說,往街上走走去。」惠蓮在傍說道:「娘們去,也携帶我走走。」金蓮道:「你既要去,你就往後邊問聲你大娘去,和你二娘,看他去不去?俺們在這裡等着你!」那惠蓮連忙往後邊去了。玉樓道:「他不濟事,等我親自問他聲出去。」李瓶兒道:「我也往屋裡穿件衣裳去,這回來冷,只怕夜深了。」金蓮道:「李大姐,你有披襖子,帶出件來我穿着,省得我往屋裡去走一遭。」那李瓶兒應諾去了,獨剩着金蓮一個,看着經濟放花兒。見無人,走向經濟身上捏了一把,笑道:「姐夫原來只穿恁單薄衣裳,不害冷麼?」只見大家兒子小鐵棍兒,笑嘻嘻在根前,舞旋旋的,且拉着經濟,問姑夫要炮火塀章放。這經濟恐怕打擾了事,巴不得與了他兩個元宵炮火塀章,支的他外邊耍去了。于是和金蓮打牙犯嘴,嘲戲說道:「你老人家見我身上單薄,肯賞我一件衣裳兒穿也恁的?」金蓮道:「賊短命!得其慣便了。頭裡躡了我的腳兒,我不言語;如今大胆又來問我要衣服穿!我又不是你影射,何故把與你衣服穿?」經濟道:「你老人家不與他罷,如何扎筏子來諕我?」婦人道:「賊短命!你是城樓子上雀兒,好耐驚耐怕的蟲蟻兒!」正說着,見玉樓和惠蓮出來,向金蓮說道:「大娘因身上不方便,大姐不自在,故不去了。教娘們走走,早些來家。李嬌兒害腿疼,也不走。雪蛾見大姐姐不走,恐怕他爹來家嗔他,也不出門。」金蓮道:「都不去罷,只咱和李大姐三個去罷。等他爹來家,隨他罵去。再不,把春梅小肉兒,和房裡玉筲,你房裡蘭香,李大姐房裡迎春,都帶了去,等他爹來家問,就教他答話。」小玉走來道:「俺奶奶也是不去,我也跟娘們走走。」玉樓道:「對你奶奶說了去,我前頭等着你。」良久,小玉問了月娘,笑嘻嘻出來。當下三個婦人,帶領着一簇男女。來安、畫童兩個小廝,打着一對紗吊燈跟隨。女婿陳經濟。躧着馬,擡放烟火花炮,與眾婦人瞧。宋惠蓮道:「姑夫,你好歹略等等兒;娘們携帶我走走,我到屋裡搭搭頭就來。」經濟道:「俺們如今就行。」惠蓮道:「你不等我,就是惱你一生!」于是走到屋裡,換了一套綠閃紅段子對衿矣兒,白挑線裙子。又用一方紅銷金汗巾子搭着頭額,角上貼着飛金,三個香茶并面花兒,金燈籠墜子,出來跟着眾人走百病兒。月色之下,恍若仙蛾,都是白綾祆兒,遍地金比甲。頭上珠翠堆滿,粉面朱唇。經濟與來興兒,左右一邊一個,隨路放慢吐蓮、金絲菊、一丈蘭、賽月明。出的大街市上,但見香塵不斷,遊人如蟻。花炮轟雷,燈光雜彩。簫鼓聲喧,十分熱鬧。左右見一隊紗燈,引導一簇男女過來,皆披紅垂綠。以為出於公侯之家,莫敢仰視,都躲路而行。那宋惠蓮一回叫:「姑夫,你放過桶子花我瞧。」一回又道:「姑夫,你放過元宵炮火塀章我聽。」一回又落了花翠,拾花翠;一回又吊了鞋,扶着人且兜鞋;左來右去,只和經濟嘲戲。玉樓看不上,說了兩句:「如何只見你吊了鞋?」玉簫道:「他怕地下泥,套着五娘鞋穿着哩!」玉樓道:「你叫他過來我瞧,真個穿着五娘的鞋?」金蓮道:「他昨日問我討了一雙鞋,誰知成精的狗肉,他套着穿!」惠蓮于是摟起裙子來,與玉樓看,看見他穿着兩雙紅鞋在腳上,用紗綠線帶兒,扎着褲腿,一聲兒也不言語。須臾,走過大街,到燈市裡。金蓮向玉樓道:「咱如今往獅子街,李大姐房子裡走走去。」于是吩咐畫童、來安兒打燈先行,迤〈辶里〉往獅子街來。小廝先去打門,老馮已是歇下,房中有兩個人家買的丫頭,在炕上睡。慌的老馮連忙開了門,讓眾婦女進來,旋戳開爐子頓茶,挈着壺往街上取酒。孟玉樓道:「老馮,你且住,不要去打酒,俺每在家,酒飯吃的飽飽來,你每有茶,倒兩甌子來吃罷!」金蓮道:「你既留人吃酒,先訂下菜兒纔好。」李瓶兒道:「媽媽子,一瓶兩瓶取了來,打水不渾的。勾誰吃?要取一兩壜兒來。」玉樓道:「他哄你,不消取;只看茶來罷。」那婆子方纔不動身。李瓶兒道:「媽媽子,怎的不往那邊去走走,端的不知你成日在家做些甚麼?」婆子道:「奶奶,你看丟下這兩個業障在屋裡,誰看他?」玉樓便問道:「兩個丫頭是誰家賣的?」婆子道:「一個北邊人家房裡使女,十三歲,只要五兩銀子;一個是汪序班家出來的家人媳婦,家人走了,主子把䯼髻打了,領出來賣要十兩銀子。」玉樓道:「媽媽,我說與你,有一個人要,你撰他些銀子使。」婆子道:「三娘,果然是誰要?告我說。」玉樓道:「如今你二娘房裡,只元宵兒一個不勾使,還尋大些的丫頭使喚。你到把這大的賣與他罷。」因問:「這丫頭十幾歲?」婆子道:「他今年屬牛,十七歲了。」說着,拿茶來,眾人吃了茶。那春梅、玉筲并惠蓮都前後瞧了一遍,又到臨街樓上,推開窗子瞧了一遍,陳經濟催逼說:「夜深了,看了快些家去罷。」金蓮道:「怪短命!催的人手腳兒不停住,慌的是些甚麼?」於是叫下春梅眾人來,方纔起身。馮媽媽送出門,李瓶兒因問:「平安往那裡去了?」婆子道:「今日這咱還沒來,教老身半夜三更,開門閉戶等着他。」來安兒道:「今日平安兒跟了爹往應二爹家去了。」李瓶兒吩咐:「媽媽子,早些關了門,睡了罷!他多也是不來,省的誤了你的睡頭。明日早來宅裡伺候,你是石佛寺長老,請着你就張致了。」婆子道:「誰是老身主兒,老身敢張致?」李瓶兒道:「媽媽休得多言多語,明日早與你二娘送丫頭來。」說畢,看着他關了大門,這一簇男女方纔回家。走到家門首,只聽見住房子的韓回子老婆韓嫂兒聲音,因他男子漢答應馬房內臣,他在家,跟着人走百病兒去了;醉回來家,說有人夜晚剜開他房門,偷了狗,又不見了些東西,坐在當街上,撒酒風罵人。眾婦人方纔立住了腳。金蓮使來安兒:「你去叫韓嫂兒,等俺每問他個端的。」不一時,把韓嫂兒叫到當面:「你為甚麼來?」韓嫂子不慌不忙,扠手向前拜了兩拜,說道:「三位娘在上,聽小媳婦從頭兒告訴。」唱耍孩兒為證:太平佳節元宵夜,云云。玉樓等眾人聽了,每人掏袖中些錢果子與他;叫來安兒:「你叫你陳姐夫,送他進屋裡。」那陳經濟且顧和惠蓮兩個嘲戲,不肯搊他去。金蓮使來安兒扶到他家中,吩咐:「教他明日早來宅內,漿洗衣裳。我對你爹說,替你出氣。」那韓嫂兒千恩萬謝,回家去。玉樓等剛走過門首來,只見賁四娘子,穿着紅祆,玄色段比甲,玉色裙,勒着銷金汗巾。在門首笑嘻嘻向前道了萬福,說道:「三位娘,那裡走了走?請不棄,到寒家獻茶。」玉樓道:「方纔因小兒哭,俺站住問了他聲;承嫂子厚意,天晚了,不到罷。」賁四娘子道:「耶嚛!三位娘上門,怪人家就笑話俺小家人家,茶也奉不出一杯兒來。」生死拉到屋裡。原來外邊供養觀音八難并關聖賢。當門掛着雪花燈兒一盞;掀開門簾,他十四歲女兒長姐在屋裡。桌上兩盞紗燈,擺設着春臺菓酌,與三人坐。連忙教他長姐過來,與三位娘磕頭遞茶。玉樓、金蓮每人與了他兩枝花兒;李瓶兒袖中取了方汗巾,又是一錢銀子,與他買瓜子兒磕。喜歡的賁四娘子,拜謝了又拜。款留不住,玉樓等起身,到大門首,小廝來興在門首迎接。金蓮就問:「你爹來家不曾?」來興道:「爹未回家哩!」三個婦人還看着陳經濟在門首,放了兩筒一丈菊,和一筒大烟蘭,一個金盞銀臺兒,纔進後邊去了。西門慶直至四更來家,正是:

「醉後不知天色瞑,  任他明月下西樓。」

都說陳經濟因走百病兒,與金蓮等眾婦人嘲戲了一路兒,又和來旺媳婦宋惠蓮,兩個這來語去,都有意了。次日早辰梳洗畢,也不到舖子內,逕往後邊吳月娘房裡來。只見李嬌兒、金蓮陪着吳大妗子坐的,放着炕桌兒,纔擺茶吃。月娘便往佛堂中燒香去了。這小夥兒向前作了揖,坐下。金蓮便說道:「陳姐夫,你好人兒,昨日教你送送韓嫂兒,你就不動。只當還叫你小廝送去了!且和媳婦子打牙犯嘴,不知甚麼張致?等你大娘燒了香來,看我對他說不說!」經濟道:「你老人家還說哩,昨日險些兒子腰累〈疒羅〉〈疒里〉了哩!跟了你老人家走了一路兒,又到獅子街房裡回來,該多少里地?人辛苦走了,還教我送韓回子老婆,教小廝送送也罷了。睡了多大回就天亮了,今早還扒不起來。」正說着,吳月娘從燒了香來,經濟作了揖。月娘便問:「昨日韓嫂兒,為甚麼撒酒風罵人?」經濟把因走百病,被人剜開門,不見了狗,坐在當街哭喊罵人。今早他漢子來家,一頓好打的,這咱還沒起來哩。金蓮道:「不是俺每回來,勸的他進去了。一時你爹來家撞見,甚模樣子?」說畢,玉樓、李瓶兒、大姐都到月娘屋裡吃茶,經濟也陪着吃了茶。後次大姐回房,罵經濟:「不知死的囚根子!平白和來旺媳婦子打牙犯嘴,倘忽一時傳的爹知道了,淫婦便沒事,你死也沒處死!」幾句說經濟。那日西門慶在李瓶兒房裡宿歇,起來的遲,只見荊千戶新陛一處兵馬都監,來拜。西門慶纔起來,旋梳頭,包網巾,整衣出來,陪荊都監在廳上說話,一面使平安兒進來後邊要茶,宋惠蓮正和玉筲、小玉在後邊院子裏,撾子兒,賭打瓜子,頑成一塊。那小玉把玉筲騎在底下,笑罵道:「賊淫婦!輸了瓜子,不教我打!」因叫惠蓮:「你過來,扯着淫婦一隻腿,等我{入日}這淫婦一下子。」正頑着,只見平安走來,叫:「玉筲姐,前邊荊老爹來,使我進來要茶哩。」那玉筲也不理他,且和小玉廝打頑耍,不理他。那平安兒只顧催逼說:「人坐下來這一日了。」宋惠蓮道:「怪囚根子!爹要茶,問廚房裡上竈的要去,如何只在俺這裡纏?俺這後邊,只是預備爹娘房裡用的茶,不管你外邊的帳。」那平安兒走到廚房下,那日該來保妻惠祥,惠祥道:「怪囚!我這裡使着手做飯,你問後邊要兩鍾茶出去就是了,巴巴來問我要茶!」平安道:「我到後頭來,後邊不打發茶,惠蓮嫂子說,該是那上竈的首尾,問那個要,他不管哩!」這惠祥便罵道:「賊潑婦!他認定了他是爹娘房裡人,俺天生是上灶的來?我這裡又做大家夥裡飯,又替大娘子炒素菜,幾隻手?論起就倒倒茶兒去也罷了,巴巴坐名兒來尋上竈的,上竈的是你叫的!誤了茶也罷,我偏不打發上去。」平安道:「荊老爹來坐了這一日,嫂子快些打發茶,我拿上去罷。遲了又惹爹罵!」當下這裡推那裡,那裡推這裡,就躭誤了半日。比及又等玉筲取茶菓、茶匙兒出來,平安兒拿出茶去,那荊都監坐的久了,再三要起身,被西門慶留住。嫌茶冷不好吃,唱罵平安來,另換茶上去吃了,荊都監纔起身去了。西門慶進來,問:「今日茶是誰頓的?」平安道:「是竈上頓的茶。」西門慶回到月娘上房,告訴月娘:「今日頓這樣茶去與人吃,你往廚下查那個奴才老婆上竈?採出來問他,打與他幾下。」小玉道:「今日該惠祥上竈哩。」慌的月娘說道:「這歪辣骨待死!越發頓恁樣茶上去了!」一面使小玉叫將惠祥當院子跪着,問他要打多少?惠祥答道:「因把做飯,炒大娘子素菜,使着手,茶略冷了些。」被月娘數罵了一回,饒了他起來。吩咐:「今後,但凡你爹前邊人來,教玉筲和惠蓮後邊頓茶,竈上只管大家茶飯。」這惠祥在廚下,忍氣不過,剛等的西門慶出去了,氣恨恨走來後邊,尋着惠蓮,指着大罵:「賊淫婦!趁了你的心了罷!你天生的就是有時運的,爹娘房裡人;俺每是上竈的老婆來!巴巴使小廝坐名,問上竈要茶;上竈的是你叫的?你我生米做成熟飯,你識我見的!促織不吃癩蝦钶肉,都是一鍬土上人,你恒數不是爹的小老婆就罷了;是爹的小老婆,我也不怕你!」惠蓮道:「你好沒要緊,你頓的茶不好,爹嫌你,管我甚事?你如何走來拿人散氣?」惠祥聽了此言,越發惱了,罵道:「賊淫婦!你剛纔調唆,打我幾棍兒好來!怎的不教打我?你在蔡家養的漢數不了。來這裡還弄鬼哩!」惠蓮道:「我養漢,你看見來?沒有扯臊淡哩!嫂子,你也不什麼清淨姑姑兒!」那惠祥道:「我怎不是清淨姑姑兒?蹺起腳兒來,比你這淫婦好些兒。我不說你罷,漢子有一拿小米數兒!你在外邊,那個不吃你嘲過,你說你背地幹的那營生兒,只說人不知道。你把娘們還放不到心上,何況以下的人!」惠蓮道:「我背地說甚麼來?怎的放不到心上?隨你壓我,我不怕你!」惠祥道:「有人與你做主兒,你可不怕哩!」兩個正拌嘴,被小玉兒請的月娘來,把兩個都喝開了:「賊臭肉們,不幹那營生去!都拌的是些甚麼?教你主子聽見,又是一場兒。頭裡不曾打得成,等住回都打得成了!」惠蓮道:「若打我一下兒,我不把淫婦口裡腸抅了,也不算!我破着這命擯兌了你,也不差甚麼。咱大家都離了這門罷!」說着,往前去了。後次這宋惠蓮越發猖狂起來。仗西門慶背地和他勾搭,把家中大小都看不到眼裡。逐日與玉樓、金蓮、李瓶兒、西門大姐、春梅在一處頑耍。那日馮媽媽送了丫頭來,約十三歲,先到李瓶兒房裡看了,送到李嬌兒房裡,李嬌兒用五兩銀子,買下房中伏侍,不在話下。正是:

「梅花恣逞春情性,  不怕封夷號令嚴。」

有詩為證:

「外作禽荒內色荒,  連沾些子又何妨;

早辰跨得雕鞍去,  日暮歸來紅粉香。」

畢竟未知後來何如,且聽下回分解:

