%# -*- coding: utf-8 -*-
%!TEX encoding = UTF-8 Unicode
%!TEX TS-program = xelatex
% vim:ts=4:sw=4
%
% 以上设定默认使用 XeLaTex 编译,并指定 Unicode 编码,供 TeXShop 自动识别

%第七十一回 
\chapter{李瓶兒何千戶家托夢\KG 提刑官引奏朝儀}


\begin{showcontents}{}




「整時罷鼓膝間琴,  閒把筵篇閱古今,

常嘆賢君務勤儉,  深悲痛主事荒臣;

治平端目親賢恪,  稔亂無龍近侫臣,

說破興亡多少事,  高山流水有知音」

話說西門慶同何千戶回來,走到大街,何千戶先差人去回何太監話去了。一面邀請西門慶到家一飯。西門慶再三固辭。何千戶手下把馬嚼拉住,說道:「學生還有一事與長官商議。」于是並馬相行,到宅前下馬。賁四同擡盒逕往崔中書家去了。原來何千戶盛陳酒筵,在家等候。進入廳上,但見屏開孔雀,褥隱芙蓉,獸炭焚燒,金爐香靄。正中獨獨設一席,下邊一席相陪,傍邊東首又設一席。皆盤堆異菓,花插金瓶。卓椅鮮明,幃屏齊整。西門慶問道:「長官今日筵何客?」何千戶道:「家公公今日下班,敢與長官敍一中飯。」西門慶道:「長官這等費心,盛設待學生,就不是同僚之情!」何千戶笑道:「倒是家公公主意,治此粗酌,屈尊請教。」一面看茶吃了。西門慶請老公公拜見。何千戶道:「家公公便出來。」不一時,何太監從後邊出來,穿著綠絨蟒衣,冠帽皂靴,寶石縧環。西門慶展拜四拜,請公公受禮。何太監不肯,說道:「使不的。」西門慶道:「學生與天泉同寅晚輩,老公公齒德俱尊,又係中貴,自然該受禮。」講了半日,何太監受了半禮。讓西門慶上面,他主席相陪,何千戶傍坐。西門慶道:「老公公,這個斷然使不的,同僚之間,豈可傍坐?老公公叔姪便罷了,學生使不了的。」何太監大喜道:「大人甚是知禮。罷罷,我閣老位兒傍坐罷,教做官的陪大人主席就是了。」西門慶道:「這等學生坐的也安。」于是各敍禮坐下。何太監道:「小的兒們,再燒好炭來,今日天氣寒冷些。」須臾,左右火池火叉,拿上一包暖閣水磨細炭,向中間四方黃銅火盆內只一倒,斤前放下油紙暖簾來,日光掩映,十分明亮。何老太監道:「大人請寬了盛服罷。」西門慶道:「學生裡邊沒穿甚麼衣服,使小价下處取來。」何太監道:「不消取去。」令左右接了衣服:「拿我穿的飛魚緑絨氅衣來與大人披上。」西門慶笑道:「老公公職事之服,學生何以穿得?」何太監道:「大人只顧穿,怕怎的?昨日萬歲賜了我蟒衣,我也不穿他了,就迸了大人遮衣服兒罷。」不一時,左右取上來。西門慶捏了帶,令玳安接去員領,披上氅衣,作揖謝了。又請何千戶也寬去上蓋陪坐,又拿上一道茶來吃了。何太監道:「叫小廝們來。」原來家中教了十二名吹打的小廝,兩個師範領著上來磕頭。何太監付擡出銅鑼銅鼓,放在廳前,一面吹打動起樂來。端的聲震雲霄,韻驚魚鳥。然後左右伺候酒筵上坐,何太監親自把盞。西門慶慌道:「老公公請尊便,有長官代勞。只安放鍾筯兒,就是一般。」何太監道:「我與大人遞一鍾兒。我家做官的,初入蘆葦,不知深淺。望乞大人凡事扶持一二,就是情了。」西門慶道:「老公公說那裡話!常言:『同僚三世親。』學生亦托賴老公公餘光,豈不同力相助。」何太監道:「好說,好說!共同王事,彼此扶持。」西門慶也沒等他遞酒,只接了杯兒,領到席上,隨即回奉一杯,安在何千戶并何太監席上,彼此告揖過,坐下。吹打畢,三個小廝連師範在筵前,銀箏象板,三絃琵琶,唱了一套正宮端正好:

「水晶宮,鮫綃帳;光射水晶宮,冷透鮫綃帳。夜深沉,睡不穩龍牀;離金門,私出天街上,正風雪空中降。」

〔滾綉毬〕「似紛紛蝶翅飛,如漫漫柳絮狂。舞冰花,旋風兒飄蕩,踐玉玷,腳步兒匆忙。將白襴兩袖遮,把烏紗小帽蕩。猛回頭,鳳樓凝望,全不見碧琉璃瓦甇鴛鴦。一霎時,九重殿如銀砌;半合兒,萬里乾坤,似玉粧。恰便是粉甸滿封疆。」

〔倘秀才〕「我只見鐵桶般重門閉,我將這銅獸面雙環扣響。敲門的我是萬歲山前趙大郎。堂中無客伴,燈下看文章,特來聽講。」

〔呆骨朵〕「衝寒風,冒凍雪,來相望。有些個機密事,緊要商量。忙怎麼,了事公人免禮,咱招賢宰相。這的鼎鼐三公府,那裡也剃頭髮唐三藏。這坐席間聽講書,你休來耳邊廂叫點湯。」

〔倘秀才〕「朕不學漢高皇,身居未央;朕不學唐天子,停眠在晉陽。常則是翠被生寒金鳳凰,有心傳說,無夢到高唐。這的是為君的勾當!」

〔滾綉毬〕「雖然與四海為一人,必索要正三綱謹五常。朕的年廣學鎗棒,恨則恨,未曾到孔子門墻。尚書是幾篇?毛詩共幾章?講禮記始知謙讓,論春秋可鑑興亡。朕待學禹、湯、文、武,宗堯、舜,卿可及房、杜、蕭、曹立漢唐?則要你爕理陰陽。」

〔倘秀才〕「卿道是用論語,治朝廷有方。卻原來這半部運山河在掌。聖道如天不可量談經臨絳帳。索強如開宴出紅粧,聽說罷神清氣爽。」

〔滾綉毬〕「銀臺上華燭明,金爐內寶篆香。不當煩教老兄自斟佳釀,又何須嫂嫂親捧著霞觴。卿道是,糟糠妻不下堂,朕須想貧賤交不可忘。常言道:『表壯不如里壯。妻若賢,夫免灾殃。』朕將卿如太甲逢伊尹,卿得嫂壯呵,恰便是梁鴻配孟光,則願你福壽綿長。」

〔倘秀才〕「但歇息呵,論前王後王;恰合眼,慮興邦喪邦。因此上曉夜無眠想萬方。雖不是歡娛嫌夜短,遭難道寂寞恨更長。憂愁事幾庄?」

〔滾綉毬〕「憂則憂,當站的身無挂體;憂則憂,家無隔宿粮。憂則憂,甘貧的晝眠深巷;憂則憂,讀書的夜寐寒窗。憂則憂,嚎寒妻怨夫啼;憂則憂,駕車的,恁時分萬里行商。憂則憂,行般的一江風浪;憂則憂,饑子呼娘。憂則憂,是布衣賢士無活計;憂則憂,鐵甲忙披守戰場。題將來,感嘆悲傷!」

〔倘秀才〕「憂的是百姓苦,向御榻心勞意攘。害的是不小可,教寡人眠思夢想,太原府劉素拒北方。我只待暫離丹鳳闕,親擁碧油幢,先取那河東的上黨。」

〔滾綉毬〕「卿道是錢王共李王,劉鋹與孟昹。他每多無仁政,著萬民失翼,行霸道,百姓遭殃。差何人鎮守西,命何人定兩廣。取吳越必須名將,下江南直用忠良。要定奪展江山,白玉擎天柱,索用恁極宇宙,黃金駕海梁,仔細端詳。」

〔脫布衫〕「取金陵飛渡長江,到錢塘平定他鄉。西川休辭棧恧,南蠻地莫愁烟瘴。」

〔醉太平〕「陣衝開虎狼,身冒著風霜,用六韜三略定邊疆,把元戎印掌,則要你人披鐵甲添雄壯,馬搖玉勒難遮當,鞭敲金〈革登〉響叮噹,早班師汴梁。

〔煞〕「有那等順天心,達天理,去邪歸,正有疎放;有那等霸王業,抗王師,揚威盡滅亡。休擄掠民財,休傷殘民命,休淫污民妻,休燒毀民房。恤軍馬施仁立法,實錢粮。定賞罰,保城池,討逆招安,沿路上安民挂榜。從賑濟任開倉。」

〔尾聲〕「朕專待正衣冠,尊相貌,就凌烟圖畫你那功臣像。卿幕賓,立金石銘鍾鼎,向青史標題姓字香。能用兵善為將,有心機有膽量。仰瞻天文等星象,俯察山川變形狀。決戰方將九地量,畫戟須將旗幟張。夜戰須將火鼓揚;步戰屯雲護軍帳,水戰隨風使帆槳。奇正相生兵最強,仁勇之行勇怎當。耳聽將軍定這廂,坐擬元戎取那廂,飛奏邊庭進表章,齊賀昇平回帝鄉。比及你列土分茅拜卿相,先將你各部下的軍卒,重重的賞!」

唱了一套下去,酒過數巡,食割兩道,看看天晚,秉上燈來。西門慶喚玳安拿賞賜與廚役并吹打各色人役,就要起身,回說:「學生不當厚擾,一日了,就此告回。」那公公那裡肯放,說道:「我今日正是下斑要與大人請教,有甚大酒席,只是清坐而已。教大人受饑。」西門慶道:「承老公公賜這等大美饌,如何反言受饑!學生回去歇息歇息,明早還與天泉參謁參謁兵科,好領劄付挂號。」何太監道:「既是如此,大人何必又回下處,就在我這裡歇了罷!明早好與我家做官的幹事。敢問如今下處在那裡?」西門慶道:「學生就暫借敝同僚夏龍溪令親崔中書宅中權寓,行李都在那邊。」何太監道:「這等也不難。大人何不令人把行李搬過來,我家住兩日何如?我這後園兒裡有幾間小房兒,甚是僻淨。就早晚和做官的理會些公事兒,也方便些兒,強如在人家。這個就是一家。」西門慶道:「在這裡也罷了。只是使夏公見怪的,學生疎他一般。」何太監道:「沒的說。如今時年,早辰不做官,晚夕不唱喏。衙門是恁偶戲衙門。雖故當初與他同僚,,今日前官已去,後官接管承行,與他就無干。怎生這等說?他就是個不知道理的人了。今日我定然要和大人坐一夜,不放大人去。」喚左右:「下邊房裡快放卓兒,管待你西老爹大官兒飯酒。我家差幾個人跟他,即時把行李都搬來了。」分付:「打發後花園西院乾淨,預備舖陳,炕中籠下炭火。」堂上一呼,階下百諾,答應下去了。西門慶道:「老公公盛情,只是學生得罪夏公了。」何太監道:「沒的扯淡哩!他既出了衙門,不在其位,不謀其政。他管他那裡鑾駕庫的事,管不的咱提刑所的事了,難怪于你。」不由分說,就打發玳安并馬上人吃了酒飯,差了幾名軍牢,各拿繩扛,逕往崔中書家搬取行李去了。何太監道:「又一件相煩大人,我家做官的若是到任所,還望大人那裡替他看所宅舍兒,然後好搬取家小。今先教他同夫人去,待尋下宅子,然後打發家小起身。也不多,連幾房家人,也有二三十口。」西門慶道:「天泉去了,老公公這宅子誰看守?」何太監道:「我兩個名下官兒,第二個姪兒何永福,見在庄子上,叫他來住了罷。」西門慶道:「老公公分付要看多少銀子宅舍?」何太監道:「也得千金出外銀子的房兒纔勾住?一舉兩得其便甚好!門面七間,到底五層。儀門進去大廳,兩邊廂房鹿角頂,後邊住房、花亭。周圍群房也有許多,街道又寬闊,只好天泉住。」何太監道:「他要許多價值兒?」西門慶道:「他對我說來,原是一千三百兩,又後邊添蓋了一層半房,收拾了一處花亭。老公公若要,隨公公與他多少罷了。」何太監道:「我乃托大人,隨大人主張就是了。趁今日我在家,差個人和他說去,討他那原文書我瞧瞧。難得尋下這房舍兒,我家做官的去到那裡,就有個歸著了。」不一時,只見玳安同眾人搬了行李來回話。西門慶問:「賁四、王經來了不曾?」玳安道:「王經同押了衣箱行李先來了,還有轎子,又叫賁四在那裡看守者。」西門慶因附耳低言,如此如此,這般這般,分付:「拿我帖兒,上覆夏老爹,借過那裡房子的原契來,與何公公要瞧瞧,就同賁四一答兒來。」這玳安應的去了。不一時,賁四青衣小帽,同玳安前來,拿文書回西門慶說:「夏老爹多上覆,既是何公公要,怎好說價錢?原文書都拿的來了。又收拾添蓋使費了許多。隨爹主張了罷。」西門慶把原契遞與何太監親看了一遍,見上面寫著一千二百兩,說道:「這房兒想必也住了幾年,裡面未免有些糟爛。也別要說收拾,大人面上,我家做官的既治產業,還與他原價。」那賁四連忙跪下說:「何爺說的,自古使的憨錢,治的庄田;千年房舍換百主,一番拆洗一番新。」把這何太監聽了,喜歡的要不的。便道:「你是那裡的?此人倒會說話兒!常言成大者不借小費。其實說的是。他叫甚麼名字?」西門慶道:「此是舍下夥計,名喚賁四。」何太監道:「也罷,沒個中人,你就做個中人兒,替我討了文契來。今日是個上官好日期,就把銀子兌與他罷。」西門慶道:「如今晚了,待的明日也罷了。」何太監道:「到五更,我早進去,明日太朝。今日不如先交與他銀子,就了事而已。」西門慶問道:「明日甚時駕出?」何太監道:「午時駕出到壇,三更鼓祭了,寅正一刻就回到宮裡,擺了膳,就出來設朝陞大殿又受朝賀,天下諸司都上表拜冬。次日文武百官吃慶成宴。你每是外任官,大朝引奏過,就沒你每事了。」說畢,何太監分付何千戶進後邊,連忙打點出二十四定大元寶來,用食盒擡著,差了兩個家人,同賁四、玳安押送到崔中書家交割。夏公見了銀子來,滿心歡喜,隨即親手寫了文契,付與賁四等,拿來遞與。何太監不勝歡喜,賞了賁四十兩銀子,玳安、王經每人三兩。西門慶道:「小孩子家,不當與他。」何太監道:「胡亂與他買嘴兒吃。」三人磕了頭謝了。何太監分付管待酒飯,又向西門慶唱了兩個喏:「全於大人餘光。」西門慶道:「豈有此理?還是看老公公金面。」何太監道:「還望大人對他說說,早把房兒騰出來,這裡好打發家小身。」西門慶道:「學生已定與他說,教他早騰。何長官這一去,且在衙門公廨中權住幾日。待他家小搬取京,收拾了,這裡長官小起是不遲。」何太監道:「收拾直待過年罷了,先打發家小去纔好,十分在衙門中也不方便。」說話之間,已有二更天氣,說道:「老公公請安置罷,學生亦不勝酒力了。」何太監方作辭,歸後邊暖房內宿歇去了。何千戶教家樂彈唱,還與西門慶投壼,吃了一回,方纔起身。歸至後園,正北三間書院,四面都是粉牆,臺柳湖山,盆景花木。房內絳燭高燒,疊席牀帳,錦幔倭金屏護,琴書几席清幽,翠簾低挂,舖陳整齊。爐上茶煮寶瓶,篆內香焚麝餅。何千戶又陪西門慶敍話良久,小童看茶吃了,方道安置,起身歸後邊去了。西門慶向了回火,方纔摘去冠帽;解衣就寢。王經、玳安打發脫了靴襪,合了燈燭,自往下邊暖炕被褥歇去了。這西門慶有酒的人,睡在枕畔,見都是綾錦被褥,貂鼠綉帳火箱,泥金暖閣牀。在被窩裡,見滿窗月色,番來覆去睡不著。良久,只聞夜漏沉沉,花陰寂寂,寒風吹得那窗紙有聲。況離家已久,欲待要呼王經進來陪他睡,忽然聽得窗外有婦人語聲甚低。即披衣下牀,靸著鞋襪,悄悄啟戶視之。只見李瓶兒霧鬢雲鬟,淡粧麗雅。素白舊衫籠雪體,淡黃軟軟襪襯弓鞋。輕移蓮步,立于月下。西門慶一見,挽之入室,相抱而哭,說道:「冤家,你如何在這裡?」李瓶兒道:「奴尋訪至此,對你說,我已尋了房兒了。今特來見你一面,早晚便搬取去也。」西門慶忙問道:「你房兒在于何處?」李瓶兒道:「咫尺不遠,出此大街,迤東造釜巷中間便是。」言訖,西門慶共他相偎相抱,上牀雲雨,不勝美快之極。已而整衣扶髻,徘徊不捨。李瓶兒叮嚀囑付西門慶:「我的哥哥,切記休貪夜飲,早早回家。那廝不時伺害于你,千萬忽忘此言,是必記于心者!」言訖,撒手而別,挽西門慶相送到家,走出大街,見月色如晝,果然往東轉過牌坊,到一小巷,旋踵見一座雙扇白板門,指道:「此奴之家也。」言畢,頓袖而入。西門慶急向前拉之,恍然驚覺,乃是南柯一夢。但見月影橫窗,花枝倒影矣。西門慶向褥底摸了摸,見精流滿席,餘香在被,殘唾猶甜。追悼莫及,悲不自勝正。正時:

「世間好物不堅牢,  彩雲易散琉璃脆。」

有詩為證:

「玉宇微茫霜滿襟,  疎窗淡月夢魂驚;

淒涼睡到無聊處,  恨殺寒雞不肯鳴。」

西門慶番來覆去盼雞叫,巴不得天亮。比及天亮,又睡著了。次日清辰,何千戶家童僕起來,伺候拿洗面湯手巾。王經、玳安打發西門慶梳洗畢,何千戶又早出來陪侍吃了姜茶,放卓兒請吃粥。西門慶問:「老公公怎的不見?」何千戶道:「家公公從五更鼓進內了。」須臾,拿上粥,圍著火盆,四碟齊整小菜,四大碗熬爛下飯。吃了粥,又拿上一盞肉員子餛飩雞蛋頭腦湯,金匙銀廂雕添茶鍾。一面吃著,分付出來伺候備馬。何千戶與西門慶冠冕,僕從跟隨,早進內參見兵科出來,何千戶便分路來家。西門慶又到相國寺拜智雲長老。長老又留擺齋,西門慶只吃了一個點心,餘者收下來與手下人吃了。玳安毡包內拿著金段,從東街穿過來,要往崔中書家拜夏龍溪去。因從造府巷所過,中間果見有雙扇白板門,與夢中所見一般。悄悄使玳安問隔壁賣豆腐老姬:「此家姓甚名誰?」老姬答道:「乃袁指揮家也。」西門慶于是不勝嘆異。到了崔中書家,夏公纔出馬拜人去。見西門慶到,令左右把馬牽過,迎西門慶至廳上,拜揖敍禮。西門慶令玳安拿上賀禮,青織金綾紵一端,色段一端。夏公道:「學生還不曾拜賀長官,到承長官先賜!昨者小房又煩費心,感謝不盡。」西門慶道:「何太監央學生看房一節,我因堂尊分付,就說此房來。何公到好就估著要,學生無不作成。討了房契去看了,一口就還了原價。是內臣性兒,立馬蓋橋,就成了。還是堂尊大福。」說畢,呵呵笑了。夏公道:「何天泉我也還未回拜他。」因問:「他此去,與長官同行罷了。」西門慶道:「他已會定同學生一路去,家小還且待後。昨日他老公公多致意,煩堂尊早些把房兒騰出來,搬取家眷。他如今且權在衙門裡住幾日罷了。」夏公道:「學生也不肯久稽,待這裡尋了房兒,就使人搬取家小,也只待出月罷了。」說畢,西門慶起身,又留了個拜帖與崔中書。夏公便道:「要留長官坐坐,爭奈在于客中,彼此情諒!」送出上馬,歸至何千戶家。何千戶又早伺候午飯等候。西門慶悉把拜夏公之事,說了一遍:「騰房已在出月,搬取家小。」何千戶大喜,謝道:足見長官盛情。」吃畢飯,二人正在廳上著棋,忽左右來報:「府裡翟爹那裡,差人送下程來了。抓尋到崔老爹那裡,崔老爹使他來這裡來了。」于是拿帖來,宛紅帖兒上寫著:「謹具金段一端,雲紵一端,鮮豬一口,北羊一腔,內酒二罈 ,點心二盒。眷生翟謙頓首拜。」西門慶見來人說道:「又蒙翟大爹費心。」一面收了禮物,寫回帖,賞來人二兩銀子,擡盒人五錢。說道:「客中不便,有褻管家。」那人連忙接了,說道:「小的不敢領。」西門慶道:「將就買杯酒吃便了。」那人方纔磕頭收了。王經在傍插口悄悄的說:「小的姐姐說,教我府裡去看看愛姐,有物事稍與他。」西門慶問:「甚物事?」王經道:「是家中做的兩雙鞋腳手。」西門慶道:「單單兒怎好拿去?」分付玳安:「我皮箱內有稍帶的玫瑰花餅;取兩罐兒,用小描金盒兒盛著。」就把回帖付與王經,穿上青衣,教他同跟了往府裡看愛姐不題。這西門慶寫了帖兒,送了一腔羊、一罈酒,謝了崔中書。把那一口豬、一罈酒、兩盒點心,擡到後邊:「孝順老公公在此多有打擾!」慌的何千戶就來拜謝,說道:「長官,你我一家,如何這等計較!」且說王經到府內,請出韓愛姐外廳拜見了,打扮如瓊林玉樹一般,比在家出落自是不同,長大了好些。管待了酒飯。因見王經身上穿的單薄,與了一件天青紵絲貂鼠氅衣兒,又與了五兩銀子,拿來回覆西門慶話。西門慶大喜。正與何千戶下棋,忽聞綽道之聲,門上人來報:「夏老爹來拜,拿了兩個拜帖兒。」忙的兩個整衣冠,迎接到廳敍禮。何千戶又謝昨日房子之事。夏提刑具了兩分段帕酒禮,奉賀二公。西門慶與何千戶再三致謝,令左右收了。又賞了賁四、玳安、王經十兩銀子。一面分賓主坐下。茶罷,共敍寒溫。夏公道:「請老公公拜見。」何千戶道:「家公公進內去了。」夏公又留下了一個雙紅拜帖兒,說道:「多頂上老公公,拜遲恕罪!」言畢,辭起身去了。何千戶隨即也具一分賀禮一疋金段,差人送去,不在言表。到晚夕,何千戶又在花園暖閣中擺酒,與西門慶共酌夜飲,家樂歌唱,到二更方寢。西門慶因其夜裡夢遺之事,晚夕令王經拿舖蓋來,書房地平上睡。半夜叫上牀,脫的精赤條,摟在被窩內,兩個口吐丁香,舌融甜唾。正是:

「不能得與鶯鶯會,且把紅娘去解饞。」一晚題過。到次日起五更,與何千戶一行人跟隨進朝。先到待漏院候時,等的開了東華門進入。但見:

「星斗依稀禁漏殘,  禁中環珮響珊珊,

花迎劍戟星初落,  柳拂旌旗露未乾;

瑞靄光中瞻萬歲,  祥烟影裡擁千官,

欲知今日天顏喜,  遙覩蓮萊紫氣蟠。」

少頃,只聽九重門啟,鳴噦噦之鸞聲;閶闔天開,覩巍巍之袞裳。重熙累洽之日,致履端嘉慶之時。當時天子祀畢南郊回來,文武百官聚集于宮省等候設朝。須臾鍾響罷,天子駕出宮,陞崇政大殿,受百官朝賀。須臾,香毬撥轉,簾捲扇開。怎見的當日朝儀整肅?但見:

「皇風清穆,溫溫靄靄氣氤氳;麗日當空,郁郁蒸蒸靉靆。微微隱隱,龍樓鳳閣散滿天香靄;霏霏拂拂,珠宮寶殿映萬縷朝霞。大慶殿,崇慶殿,文德殿,集賢殿,燦燦爛爛,金碧交輝;乾明宮,神寧宮,昭陽宮,合壁宮,清寧宮,光光彩彩,丹青炳燦。蒼蒼涼涼,日影著玉砌雕欄;裊裊嬰嬰,霧鎖著金椽畫棟。紫扉黃閣,寶鼎內,縹縹緲緲,沉檀香爇;丹階墀,玉砌臺,明明朗朗畫燭高焚。龍龍鼕鼕,報天敲擂叠三通;鑑鑑鍧鍧,長樂鐘撞一百八下。枝枝楂楂,叉刀手互相磕撞;挨挨曳曳,龍虎旂來往盤旋。錦衣花帽,擎著的是圓蓋傘,方蓋傘,上上下下,開展即龍蟠;駕著的是金輅輦,玉輅輦,左左右右相陣。又見那立金瓜,臥金瓜,三三兩兩;雙龍扇,平龍扇,叠叠重重。群群隊隊,金鞍馬,玉轡馬,性貌馴習;雙雙對對,寶匣象駕轅象,猛力猙獰。鎮殿將軍,一個個長長大大賽天神,甲披金葉侍朝;衛勳一人,齊齊整整如地煞,刀繫綉春。嚴嚴肅肅,殿門內擺列著紏儀御史,人人豸冠森聳,秉簡當胸;端端正正,姜擦邊立站定眾官員,個個錦衣炳煥,宣聽旨。金殿參參差差齊開寶扇,畫棟前輕輕款款高捲珠廉。文樓上,嘐嘐噦噦報時雞,人三唱;玉階前,刺刺刮刮肅靜鞭響三聲。齊齊整整列簪纓,有五等之爵;巍巍蕩蕩坐龍床倚綉褥,瞳萬乘之尊。遠遠望見頭戴十二旒平頂冠,穿赭袞龍袍,腰繫藍田玉帶,腳靸烏油舊履,手執金廂白玉圭,背靠九雷龍鳳扆。」正是:

「晴日明開青鎖闥,  天風吹下御爐香;

千條瑞靄浮金闕,  一朵紅雲捧玉皇。」

「這帝皇果生得堯眉舜目,禹背湯肩。若說這個官家,才俊過人,口工詩韻,目類群羊。善寫墨君竹,能揮薛稷書。道三教之書,曉九流之典。朝歡暮樂,依稀似劍閣孟商王;愛色貪盃,彷彿如金陸陳後主。從十八歲登基即位,二十五年倒改了五遭年號;先改建中靖國,後改崇建,改大觀,改正和。」

當下駕坐寶位,靜鞭響罷,文武百官,九卿四相,秉簡當胸,向丹墀五拜三叩頭禮,進上表章。已有殿頭官,自穿紫窄衫,腰繫金廂帶,步著金階口,傳聖勅道:「胼今即位二十禪于茲矣,專嶽告成上天降瑞。今值履端之慶,與卿共之!」言未畢,斑首中閃過一員大臣來,朝靴踏地響,袍袖列風生。官不知多大,玉帶顯功名。視之,乃左丞相崇政殿大學士兼吏部尚書太師魯國公蔡京也。幞頭象簡,俯伏金階叩首,口稱:「萬歲,萬歲,萬萬歲!臣等誠惶誠恐,稽首頓首,恭惟皇上御極二十禪以來,海宇清寧,天下豐稔。上天降鑒,禎禪叠見。日重輪,星重輝,海重闊,聖上握乾符,永享萬年之正統,天保定,地保寧,人保安,皇圖膺寶曆,益增永壽之無疆。三邊永息于兵戈,萬國來朝于天闕。銀岳排空,玉京挺秀。寶籙膺頒于昊闕,絳霄深聳于乾宮。臣等何幸,欣逢盛世,交際明良。永效華封之祝,常沾日月之光,不勝瞻天仰聖,激切屏營之至。謹獻頌以聞。」良久,聖旨下來:「賢卿獻頌,盖見忠誠,朕心加悅。」詔改明年為宣和元年,正月元旦,受定命寶,肄赦覃賞有差。蔡太師承旨下來,殿頭官口傳聖旨:「有事出班早奏,無事捲廉退朝。」言未畢,見一人出離班部,例芴躬身,緋袍象簡,玉帶金魚,跪在金階,口稱:「光祿大夫掌金吾衛事太尉太保兼太子太保臣朱,引天下提刑官員事,後面跪的兩准、兩浙、山東、山西、河南、河北、關東、關西、福建、廣南、四川等處刑獄千戶章隆等二十六員,例該考察,已更陞補,繳換劄付。合當引奏,未敢擅便,請旨定奪。」聖旨傳下來:「照例給領。」朱大尉承旨下來,天下龍袍一展,群臣皆散,駕即回宮。百官皆從端禮門兩分而出。那十二象,不待牽而先走,鎮將長隨,紛紛而散,只聽甲響;叉刀力士、團子紅軍,盡盡而出。惟見戈明。朝門外,車馬縱橫,待仗羅列。人喧呼,海沸波翻;馬嘶喊,山崩地裂。眾提刑官皆出朝上馬,都來本衙門伺候鐵桶相似。良久,只見知印局來,拿了印牌來傳道:「老爺不進衙門了,轎兒已在西華門裡安放。如今要往蔡爺、李爺宅內拜冬去了。」以此眾官都散了。西門慶與何千戶回到家中,又過了一夕。到次日,衙門中領了劄付,同眾往科中掛了號,打點殘裝,收拾行李與何千戶一同起身。何太監晚夕置置酒餞行,囑付何千戶:「凡事請教西門大人,休要自專,差了禮數。」從十一月十一日東京起身,兩家也有二十人跟隨,竟往山東大道而來,已是數九嚴寒之際,點水滴凍之時。一路上見了些荒郊野路,枯木寒鴉,疎林淡日影斜暉,暮雪凍雲迷晚渡。一山未盡一山來,後村已過前村望。比及剛過黃河,到水關八角鎮,驟然撞遇天起一陣大風。但見:

「非干虎嘯,豈是龍吟。卒律律寒飈撲面,急颼颼冷氣侵入。既不能卸柳□□,暗藏著水妖山怪。初時節無踪無影,次後來捲霧收雲。驚得那綠楊堤鷗鳥雙飛,紅蓼岸鴛鴦並起。則見那人紗窗,撲銀燈,穿畫閣,透羅裳,亂舞飄。吹花擺柳昏慘慘,走石揚砂白茫茫。刮得那大樹連聲吼刷吼刷,驚得那孤雁落深濠。須臾砂石打地,塵土遮天。砂石打地,猶如滿天驟雨即時來;塵土遮天,好相似百萬貔貅捲土至。赶趨得材落漁罷鈎,捲鈎綸疾走回家。山中樵子魂驚,掖斧斤急忙歸舍。諕得那山中虎豹縮著頭,隱著足,潛藏深壑。刮得那海底蛟拳著爪,蟠著尾,難顯猙獰。刮多時,只見那房上瓦飛似燕;吹良久,山中走石如飛。瓦飛似燕,打得客旅迷踪失道;石走怒干,諕得那商船緊纜收帆。大樹連根拔起,小樹有條無稍。這風大不大,真個是吹拆地獄門前,刮起酆都頂上塵。嫦娥急把蟾宮閉,列子空中叫救人,險些兒玉皇住不的崑崙頂,只刮的大地乾坤上下搖。」

西門慶與何千戶坐著兩頂毡幃暖轎,被風刮得寸步難行。又見天色漸晚,恐深林中撞出小人來,對西門慶說:「投奔前村安歇一夜,明日風住再行。」抓尋了半日,遠遠望見路傍一座古剎,數株疏柳,半堵橫牆。但見:

「石砌碑橫夢草遮,  迴廊古殿半欹斜;

夜深宿客無燈火,  月落安禪更可嗟!」

西門慶與何千戶入寺中投宿,見題著「黃龍寺」,見方丈內幾個僧人在那裡坐禪,又無燈火,房舍都毀壞,半用籬遮。長老出來問訊,旋炊火煮茶,伐草根喂馬。煮出來,西門慶行囊中帶得乾雞臘肉、果餅棋子之類,晚夕與何千戶胡亂食得一頓。長老爨一鍋豆粥吃了 ,過得一宿。次日風止,天氣始晴,與了老和尚一兩銀子相謝,作辭起身,往山東來。正是:

「王事驅馳豈憚勞,  關山迢遞赴京朝;

夜投古寺無煙火,  解使行人心內焦。」

畢竟未知後來如何,且聽下回分解:






\end{showcontents}


