%# -*- coding: utf-8 -*-
%!TEX encoding = UTF-8 Unicode
%!TEX TS-program = xelatex
% vim:ts=4:sw=4
%
% 以上设定默认使用 XeLaTex 编译,并指定 Unicode 编码,供 TeXShop 自动识别

%第十三回 
\chapter{李瓶兒隔墻密約\KG 迎春女窺隙偷光}

\begin{showcontents}{}



「人生雖未有千全,  處世規模要放寬,

好是但看君子語,  是非休聽小人言;

徒將世俗能歡戲,  也畏人心似隔山,

寄語知音女娘道,  莫將苦處語為甜。」

話說一日,六月十四日,西門慶從前邊來,走到月娘房中。月娘告說:「今日你不在家,花家使小廝拏帖子來請你吃酒:『若是他來家就去。』」西門慶觀看原帖子,寫着:「即午,院中吳銀家敍,希過我往,萬萬!」于是打選衣帽齊整,叫了兩箇跟隨,預備下駿馬,先逕到花家。不想花子虛不在家了。他渾家李瓶兒,夏月間,戴着銀絲䯼髻,金鑲紫瑛墜子,藕絲對衿衫,白紗挑線鑲邊裙,裙邊露一對紅鴛鳳嘴,尖尖趫趫,立在二門裡台基上。手中正拏一隻紗綠糸塀路紬鞋扇。那西門慶三不知,正進門,兩箇撞了箇滿懷。這西門慶留心已久,雖故莊上見了一面,不曾細翫其詳。于是對面見了一面。人生的甚是白淨,五短身材,瓜子面皮,生的細彎彎兩道眉兒。不覺魂飛天外,魄散九霄,忙向前深深的作揖。婦人還了萬福,轉身入後邊去了。使出一箇頭髮齊眉的丫鬟來,名喚秀春,請西門慶客位內坐。他便立在角門首,半露嬌容說:「大官人少坐一時。他適纔有些小事出去了,便來也。」少頃,使丫鬟拏出一盞茶來。西門慶吃了。婦人隔門說道:「今日他請大官人往那邊吃酒去?好歹看奴之面,勸他早些來家。兩箇小廝又都跟的去了,止是這兩箇丫鬟和奴,家中無人。」西門慶便道:「嫂子見得有理,哥家事要緊,嫂子既然分付在下,在下已定伴哥同去同來,怎肯失了哥的事?」正說着,只見花子虛來家。婦人便回房中去了。花子虛見西門慶,敍禮說道:「蒙兄下降,小弟適有些不得已小事出去望望,失迎,恕罪!」于是分賓主坐,便叫小廝看茶。須臾茶罷,分付小廝:「對你娘說,看菜兒來。我和你西門爹吃三盃起身。今日院內吳銀姐生日,請兄同往一樂。」西門慶道:「仁兄何不早說?」即令玳安:「快家去,討五錢銀子封了來。」花子虛道:「兄何故又費心?小弟到不是了。」西門慶見左右放桌兒,說道:「兄不消留坐了。咱往裡邊吃去罷。」花子虛道:「不敢久留,兄坐一回。」就是大盤大碗,雞蹄鮮肉餚饌,拏將上來。銀高腳葵花鍾,每人一鍾。又是四箇捲餅 ,吃畢,收下來與馬上人吃。少頃,問玳安取了分資來,一同起身上馬。西門慶是玳安、平安兒,花子虛是天福、天喜兒,四箇小廝跟隨,逕往抅欄後巷吳四媽家,與吳銀兒做生日。到那裡花攢錦簇,歌舞吹彈。飲酒至一更時分方散。西門慶留心,把子虛灌的酩酊大醉。又因李瓶兒央浼之言,頓得相伴他一同來家。小廝叫開大門,扶他到客位坐下。李瓶兒同丫鬟掌着燈燭出來,把子虛挽扶進去。西門慶交付明白,就要告回。婦人旋走出來拜謝西門慶,說道:「拙夫不才今貪酒,多累看奴薄面,姑待來家。官人休要笑話。」那西門慶忙屈著還喏,說道:「不敢。嫂子這裡分付,早晨一面出門,將的軍去,將的軍來。在下敢不銘心刻骨,同哥一答裡來家?非嫂子躭心,顯的在下幹事不的了。你看哥在他家,被那些人纏住了。我漒着你催哥起身。走到樂星堂兒門首,粉頭鄭愛香兒家,小名叫做鄭觀音,生的一表人物,哥就往他家去。被我再三攔住了,說道:『哥家去罷,改日再來。家中嫂子放心不下。』方纔一直來家。不然,若到鄭家,一夜不來。嫂子在上,不該我說。哥也糊塗,嫂子又青年,惹大家室,如何便丟了去,成夜不在家,是何道理?」婦人道:「正是如此。奴為他這等在外胡行,不聽人說,奴也氣了一身病痛在這裡。往後大官人但遇他在院中,好歹看奴薄面,勸他早早回家。奴恩有重報,不敢有忘。」這西門慶是頭上打一下,腳底板響的人。積年風月中走,甚麼事兒不知道。可可今日婦人到明明開了一條大路,教他入港。于是滿面堆笑道:「嫂子說那裡話!比來比來,相交朋友做甚麼?我已定苦心諫哥,嫂子放心。」婦人又道了萬福,又叫小丫鬟拿了一盞果仁泡茶 來,銀匙雕漆茶鍾。西門慶吃畢茶,說道:「我回去罷。嫂子仔細門戶。」于是告辭歸家。自此這西門慶就安心設計圖謀這婦人。屢屢安下應伯爵、謝希大這夥人,把子虛掛住在院裡,飲酒過夜,他便脫身來家,一往在門首站立着。看見婦人領着兩丫鬟,正在門首。看見西門慶在門前咳嗽,一回走過東來,又往西去。或在對門站立,把眼不住望門盼着。婦人影身在門裡。見他來,便閃進裡面。他過去了,又探頭去瞧。兩箇眼意心期,已在不言之表。一日西門慶門首正站立間,婦人使過小丫鬟秀春來請。西門慶故意問道:「姐姐,你請我做甚麼?你爹在家裡不在?」秀春道:「俺爹不在家,娘請西門爹問句話兒。」這西門慶得不的此一聲,連忙走過來。讓到客位內坐下。良久,婦人出來,道了萬福。便道:「前日多承官人厚意,奴銘刻于心,知感不盡。拙夫從昨日出去,一連兩日不來家了。不知官人曾會見他來不曾?」西門慶道:「他昨日同三四箇在鄭家吃酒,我偶然有些小事就來了。今日我不曾得進去,不知他還在那裡沒在?若是我在那裡,有箇不催促哥哥早來家的!恐怕嫂子憂心!」婦人道:「正是這般說。只是奴吃他恁不聽人說,常時在前邊眠花臥柳,不顧家事的虧。」西門慶道:「論起哥來,仁義上也好。只是有這一件兒。」說着,小丫鬟拿茶來吃了。那西門慶恐子虛來家,不敢久戀,就要告歸。婦人千叮萬囑,央西門慶明日到那裡,好歹勸他早來家:「奴恩有報,已定重謝官人。」西門慶道:「嫂子沒的說,我與哥是那樣相交。」說畢,西門慶家去了。到次日,花子虛自院中回家。婦人再三埋怨,說道:「你便外邊貪酒戀色,多虧隔壁西門大官人,兩次三番顧睦你來家。你買分禮兒知謝知謝他,方不失了人情。」那花子虛連忙買了四盒禮物,一罈酒,使小廝天福兒送到西門慶家。西門慶收下,厚賞來人不題。有吳月娘便說:「花家如何送你這分禮?」西門慶道:「此是花二哥前日請我們在院中與吳銀兒做生日,醉了,被我攙扶了他來家。又見我常時院中勸他休過夜,早早來家。他娘子兒因此感不過我的情,想對花二哥說,買了此禮來謝我。」那吳月娘聽了,與他打了箇問訊,說道:「我的哥哥,你自顧了你罷,又泥佛勸土佛。你也成日不着箇家,在外養女調婦。又勸人家漢子!」又道:「你莫不白受他這分禮?」因問:「他帖兒上寫着誰的名字?若是他娘子的名字,今日寫我的帖兒,請他娘子過來坐坐。他已只恁要來咱家走走哩!若是他男子漢名字,隨你請不請,我不管你。」西門慶道:「是花二哥名字,我明日請他便了。」次日,西門慶果然治盃,請過這花子虛來,吃了一日酒歸家。李瓶兒說:「你不要差了禮數。咱送了他一分禮,他左右還請你過去吃了一席酒。你改日另治一席酒請他。只當回席也是好處。」光陰迅速,又早九月重陽令節。這花子虛假着節下,叫了兩箇妓者,具柬請西門慶過來賞菊。又邀應伯爵、謝希大、祝日念、孫寡嘴四人相陪。傳花擊鼓,歡樂飲酒。有詩為證:

「烏兔循環似箭忙,  人間佳節又重陽,

千枝紅樹粧秋色,  三徑黃花吐異香;

不見登高烏帽客,  還思捧酒綺羅娘,

秀簾瑣闥私相覷,  從此恩情兩不忘。」

當日眾人飲酒,到掌燈之後,西門慶忽下席來,外邊更衣解手。不防李瓶兒正在遮隔子外邊站立偷覷,兩箇撞了箇滿懷。西門慶〈廴回〉避不及。婦人走于西角門首,暗暗使丫鬟秀春黑影裡走到西門慶根前,低聲說道:「俺娘使我對西門爹說,少吃酒,早早回家。如今便打發我爹往院裡歇去。晚夕娘如此這般,要和西門爹說話哩。」這西門慶聽了,歡喜不盡。小解回來,到席上連偷酒在懷,唱的左右彈唱遞酒,只是粧醉再不吃。看看到一更時分,那李瓶兒不住走來簾外窺覷。見西門慶坐在上面,只推做打盹。那應伯爵、謝希大如同箇子釘在椅子上,正吃的箇定油兒,自不起身。熬的祝日念、孫寡嘴也去了。他兩箇還不動。把箇李瓶兒急的要不的。西門慶已是走出來,被花子虛再不放,說道:「今日小弟沒敬心,哥怎的白不肯坐?」西門慶道:「我本醉了,吃不去。」于是故意東倒西歪,教兩箇小廝扶歸家去了。應伯爵道:「他今日不知怎的,白不肯吃酒。吃了沒多酒就醉了。既是東家費心,難為兩箇姐兒在此。拏大鍾來,咱每再週四五十輪散了罷。李瓶兒在簾外聽見,罵涎臉的囚根子不絕。暗暗使小廝天喜兒請下花子虛來,分付說:「你既要與這夥人吃,趁早與我院裡吃去,休要在家裡聒噪我!半夜三更,熬油費火,我那裡耐煩!」花子虛道:「這咱晚我就和他們院裡去,也是來家不成。你休再麻犯我是的。」婦人道:「你去,我不麻犯便了。」這花子虛得不的這一聲,走來對眾人說:「如此這般,我們往院裡去。」應伯爵道:「真箇嫂子有此話?休哄我!你再去問聲嫂子來,咱好起身。」子虛道:「房下剛纔已是說了,教我明日來家。」謝希大道:「可是來白吃應花子這等韶刁,哥剛纔已是討了老腳來,咱去的也放心。」于是連兩箇唱的,都一齊起身進院。天福兒、天喜兒跟花子虛等三人,到後巷吳銀兒家,已是二更天氣。叫開門,吳銀兒已是睡下。旋起來,堂中秉燭,迎接入裡面坐下。應伯爵道:「你家孤老今日請俺們賞菊飲酒,吃的不割不截的,又邀了俺每進來。你這裡有酒,拏出俺每吃。」且不說花子虛在院吃酒。單表西門慶推醉到家,走到潘金蓮房裡,剛脫了衣裳,就往前邊花園裡去坐,單等李瓶兒那邊請他。良久,只聽的那邊趕狗關門。少頃,只見丫鬟迎春黑影影裡扒着牆推叫貓,看見西門慶坐在亭子上,遞了話。這西門慶掇過一張桌凳來踏着,暗暗扒過牆來。這邊已安下梯子。李瓶兒打發子虛去了,已是摘了冠兒,亂挽烏雲,素體濃粧,立于穿廊下。看見西門慶過來,歡喜無盡。迎接進房中,掌着燈燭,早已安排一桌齊齊整整酒餚果菜。小壺內滿貯香醪。婦人雙手高擎玉斝,迎春執壺遞酒,向西門慶深深道了萬福,說道:「一向感謝官人。官人又費心相謝,使奴家心下不安。今日奴自治了這盃淡酒,請官人過來,聊盡奴一點薄情。又撞着兩箇天殺的涎臉,只顧坐住了,急的奴要不的。剛纔吃我都打發他往院裡去了。」西門慶道:「只怕二哥還來家麼?」婦人道:「奴已分付過夜,不來了。兩箇小廝都跟去了,家裡再無一人。只是這兩箇丫頭,一箇馮媽媽看門首,是奴從小兒養娘心腹人。前後門都已關閉了。」西門慶聽了,心中甚喜。兩箇于是並肩疊股,交盃換盞,飲酒做一處。迎春旁邊斟酒,秀春往來拿菜兒。吃得酒濃時,錦帳中香薰鴛被,設放珊枕。兩個丫鬟抬開酒桌,拽上門去了。兩人上床交歡。原來大人家有兩層窗寮,外面為窗,裡面為寮。婦人打發丫鬟出去,關上裡邊兩扇窗寮,房中掌着燈燭,外邊通看不見。這迎春丫鬟,今年已十七歲,頗知事體。見他兩個今夜偷期,悄悄向窗下用頭上簪子,挺簽破窗寮上紙,往裡窺覷。端的二人怎樣交接,但見:

「燈光影裡,鮫銷帳內,一來一往,一撞一沖。這一個玉臂忙搖,那一個金蓮高舉。這一個鶯聲嚦嚦,那一個燕語喃喃。好似君瑞遇鶯娘,尤若宋玉偷神女。山盟海誓,依稀耳中;蝶戀蜂恣,未肯即罷。戰良久被翻紅浪,靈犀一點透酥胸;鬬多時帳抅銀鈎,眉黛兩彎垂玉臉。」

那正是:

「三次親唇情越厚,  一酥麻體與人偷。」

這房中二人雲雨,不料迎春在窗外聽看了個不亦樂乎。聽見他二人說話。西門慶問婦人:「多少青春?」李瓶兒道:「奴屬羊的,今年二十三歲。」因問:「他大娘貴庚?」西門慶道:「房下屬龍的,二十六歲了。」婦人道:「原來長奴三歲。到明日買分禮物過去,看看大娘,只相不敢親近。」西門慶道:「房下自來好性兒。不然,我房裡怎生容得這許多人兒?」婦人又問:「你頭裡過這邊來,他大娘知道不知?倘或問你時,你怎生回答?」西門慶道:「俺房下都在後邊第四層房子裡。惟有我第五個小妾潘氏,在這前邊花園內,獨自一所樓房居住。他不敢管我。」婦人道:「他五娘貴庚多少?」西門慶道:「他與大房下都同年。」婦人道:「又好了,若不嫌奴有玷,奴就拜他五娘做個姐姐罷。到明日討他大娘和五娘的腳樣兒來,奴親自做兩雙鞋兒過去,以表奴情。」婦人便向頭上關頂的金簪兒,拔下兩根來,遞與西門慶,分付:「若在院裡,休要叫花子虛看見。」西門慶道:「這理會得。」當下二人如膠似漆,盤桓到五更時分,窗外雞鳴,東方漸白。西門慶恐怕花子虛來家,整衣而起。婦人道:「你照前越牆而過。」兩個約定暗號兒;但子虛不在家,這邊使丫鬟立牆頭上,暗暗以咳嗽為號,或先丟塊瓦兒。見這邊無人,方纔上牆叫他,西門慶便用梯凳扒過牆來。這邊早安下腳手接他。兩個隔牆酬和,竊玉偷香,又不由大門裡行走,街坊鄰舍,怎得曉的暗地裡事。有詩為證:

「吃食少添鹽醋,  不是去處休去;

要人知重勤學,  怕人知事莫做。」

卻說西門慶天明依舊扒過牆來,走到潘金蓮房裡。金蓮還睡未起,因問:「你昨日三不知又往那裡去了?一夜不來家,也不對奴說一聲兒。」西門慶道:「花二哥又使了小廝邀我往院裡去吃了半夜酒,脫身纔走來家。」金蓮雖故信了,還有幾分疑齪影在心中。一日同孟玉樓飯後的時分,在花園亭子上坐着做針指。只見掠過一塊瓦兒來,打在面前。那孟玉樓低着納鞋沒看見。這潘金蓮單單把眼四下觀盼,影影綽綽只見一個白臉在牆頭上探了探就下去了。金蓮忙推玉樓,指與他瞧,說道:「三姐姐,你看這個是隔壁花家那大丫頭,不知上牆瞧花兒,看見俺們在這裡,他就下去了。」說畢也不就罷了。到晚夕,西門慶自外趕席來家,進金蓮房中。金蓮與他接了衣裳,問他,飯不吃,茶也不吃。趔趄着腳兒,只往前邊花園裡走的。這潘金蓮賊留心,暗暗看着他坐了好一回。只見先頭那丫頭,在牆頭上打了個照面。這西門慶就踩着梯凳過牆去了。那邊李瓶兒入房中,兩個廝會,不必細說。這潘金蓮歸到房中,翻來覆去,通一夜不曾睡。到天明,只見西門慶過來,推開房門,婦人一逕睡在床上不理他。那西門慶先帶幾分愧色,挨近他床邊坐下。婦人見他來,跳起來坐着,一手撮着他耳朵罵道:「好負心的賊!你昨日端的那去來?把老娘氣了一夜!」又說:「沒曾揸住你。你原來幹的那繭兒,我已是曉的不耐煩了!趁繭實說,從前已往,隔壁花家那淫婦得手偷了幾遭?一一說出來,我便罷休。但瞞着一字兒,到明日你前腳兒但過那邊去了,後腳我這邊就吆喝起來,教你負心的囚根子,死無葬身之地!你安下人標住他漢子在院裡過夜,這裡要他老婆。我教你吃不了包着走!嗔道昨日大白日裡,我和孟三姐在花園裡做生活,只見他家那大丫頭在牆那邊探頭舒腦的。原來是那淫婦使的勾使鬼來勾你來了!你還哄我老娘,前日他家那忘八半夜叫了你往院裡去。原來他家就是院裡!」這西門慶不聽便罷,聽了此言,慌的粧矮子,只跌腳跪在地下,笑嘻嘻央及說道:「怪小油嘴兒,禁聲些。實不瞞你,他如此這般問了你兩個的年紀,到明日討了鞋樣去,每人替你做雙鞋兒。要拜認你兩個做姐,他情愿做妹子。」金蓮道:「我是不要那淫婦認甚哥哥姐姐的,他要了人家漢子,又來獻小慇懃兒,啜哄人家老公。我老娘眼裡放不下砂子的人,肯叫你在我根前弄了鬼兒去了。」說着,一隻手把他褲子扯開。只見他那話兒軟仃儅,銀托子還帶上面。問道:「你實說晚夕與那淫婦弄了幾遭?」西門慶道:「弄到有數兒的只一遭。」婦人道:「你指着你這旺跳的身子賭個誓,一遭就弄的他恁軟如鼻涕濃如醬,恰似風癱了的一般。有些硬朗氣兒也是人心!」說着把托子一揪掛下來,罵道:「沒羞的黃貓黑盡的強盜!嗔道教我那裡沒尋,原來把這行貨子悄地帶出,和那淫婦{入日}搗去了。」那西門慶便滿臉兒陪笑兒說道:「怪小淫婦兒,麻犯人死了。他再三教我稍了上覆來,他到明日過來與你磕頭,還要替你做鞋。昨日使丫頭替了吳家的樣子去了。今日教我稍了這一對壽字簪兒送你。」于是除了帽子,向頭上拔將下來,遞與金蓮。金蓮接在手內觀看,卻是兩根番紋低板石青填地金鈴瓏壽字簪兒。乃御前所製造,宮裡出來的,甚是奇巧。金蓮滿心歡喜,說道:「既是如此,我不言語便了。等你過那邊去,我這裡與你兩個觀風,教你兩個自在{入日}搗。你心下如何?」那西門慶喜歡的雙手摟抱着說道:「我的乖乖的兒,正是如此,不枉的養兒,不在阿金溺銀,只要見景生情。我到明日梯已買一套粧花衣服謝你。」婦人道:「我不信那蜜口糖舌,既要老娘替你二人週全,要依我三件事。」西門慶道:「不拘幾件,我都依。」婦人道:「頭一件,不許你往走院裡去。第二件,要依我說話。第三件,你過去和他睡了來家,就要告訴我,一字不許你瞞我。」西門慶道:「這個不打緊處,都依你便了。」自此為始,西門慶過去睡了來,就告婦人說:李瓶兒怎的生得白淨,身軟如綿花瓜子一般,好風月,又善飲。俺兩個帳子裡放着果盒,看牌飲酒,常頑耍半夜不睡。又向袖中取出一個物件的兒來,遞與金蓮瞧道:「此是他老公公內府畫出來的,俺兩個點着燈,看着上面行事。」金蓮接在手中,展開觀看。有詞為證:

「內府衢花綾表,牙籤錦帶粧成。大青大綠細描金,鑲嵌斗方乾淨。女賽巫山神女,男如宋玉郎君。雙雙帳內慣交鋒,解名二十四,春意動關情。」

金蓮從前至尾,看了一遍,不肯放手。就交與春梅:「好生收我箱子內,早晚看着耍子。」西門慶道:「你看兩日,還交與我。此是人的愛物兒,我借了他來家瞧瞧,還與他。」金蓮道:「他的東西如何到我家?我又不曾從他手裡要將來。就是也打不出去。」西門慶道:「你沒問他要,我卻借將來了。怪小奴才兒,休作耍。」因趕着奪那手卷。金蓮道:「你若奪一奪兒,賭個手段,我就把他扯得稀爛,大家看不成。」西門慶笑道:「我也沒法了。隨你看畢了,與他罷麼?你還了他這個去,他還有個稀奇物件兒哩。到明日我要了來與你。」金蓮道:「我兒誰養的你恁乖!你拿了來,我方與你這手卷去。」兩個絮聒了一回。晚夕金蓮在房中香薰鴛被,款設銀燈,豔粧澡牝,與西門慶展開手卷,在錦帳之中,效于飛之樂。看官聽說:巫蠱魘昧之事,自古有之。觀其金蓮,自從教劉瞎子回背之後,不上幾時,就生出許多枝節,使西門慶變嗔怒而為寵愛,化幽辱而為歡娛,再不敢制他出三不信我。正是:

「饒你奸似鬼,  也吃洗腳水。」

有詩為證:

「記得書齋乍會時,雲踪雨跡少人知。曉來鸞鳳栖雙枕,剔盡銀缸半吐輝。思往事,夢魂迷。今宵喜得效于飛,巔鸞倒鳳無窮樂,從此雙雙永不離。」

畢竟未知後來如何,且聽下回分解:





\end{showcontents}
