%# -*- coding: utf-8 -*-
%!TEX encoding = UTF-8 Unicode
%!TEX TS-program = xelatex
% vim:ts=4:sw=4
%
% 以上设定默认使用 XeLaTex 编译,并指定 Unicode 编码,供 TeXShop 自动识别

%第六十七回 
\chapter{西門慶書房賞雪\KG 李瓶兒夢斷幽情}


「終日思卿不見卿,  數聲寒角未堪聞,

匣中破鏡收殘月,  篋裡餘衣歛斷雲;

寒雀疎枝栖不定,  征鴻斷字嘆離群,

玉釵敲斷心難碎,  想豫傷心記未真。」

話說西門慶歸後邊,辛苦的人,直睡至次日,日色高還未起來。有來興兒進來說:「搭綵匠外邊伺候,請問拆棚。」西門慶罵了來興兒幾句,說:「拆棚教他拆就是了,只顧問怎的?」搭綵匠一面外邊七手八腳,卸下蓆繩松條,拆了送到對門房子裡堆放,不題。玉簫進房說:「天氣好不陰的重!」西門慶令他向煖炕上取衣裳穿,要起來。有吳月娘便說:「你昨日辛苦了一夜,天陰,大睡回兒起來,慌的老早就扒起去做甚麼?就是今日不往衙門裡去也罷了。」西門慶道:「我不往衙門裡去,只怕翟親家那人來討書,好打發回書與他。」月娘道:「既是恁說,你起去。我叫丫頭熬下粥等你來吃。」這西門慶也不梳頭洗臉,蓬頭,披著絨衣,戴着毡巾,逕走到花園裡藏春閣書房中。原來自從書童去了,西門慶就委王經管花園兩邊書房門鑰匙,春鴻便收拾打掃大廳前書房。冬月間,西門慶只在藏春閣書房中坐。那裡燒下的地爐煖炕,地平上又安放着黃銅火盆,放下梅梢月油單絹煖簾來。明間內擺着夾枝桃,各色菊花,清清瘦竹,翠翠幽蘭。裡面筆硯瓶梅,琴書消洒。床炕上茜紅毡條,銀花錦褥,枕橫鸂鶒,帳挂鮫絹。西門慶〈扌歪〉在牀上,王經連忙向卓上象牙盒內,炷褻龍涎於流金小篆內。西門慶使王經:「你去叫來安兒,請你應二爹去。」那王經出來,分付來安兒請去了。只見平安走來對王經說:「小周兒在外邊伺候。」那王經走入書房,對西門慶說了。西門慶叫進小周兒來,磕了頭,說道:「你來得好,且與我篦篦頭,捏捏身上。」因說:「你怎一向不來?」小周兒道:「小的見六娘沒了,忙,沒曾來。」西門慶於是坐在一張醉翁椅上,打開頭髮,教他整理梳篦。只見來安兒請的應伯爵來了,頭戴毡帽,身穿綠絨祅子,腳穿一隻舊皂靴,棕套。掀簾子進來,唱喏。西門慶正篦頭,說道:「不消聲喏,請坐。」伯爵拉過一張椅子來,就着火盆坐下了。西門慶道:「你今日如何這般打扮?伯爵道:「你不知外邊飄雪花兒哩,好不寒冷!昨日家去晚了,雞也叫了。你還使出大官兒來拉,俺每就走不的了。我見天陰上來,還付了個燈籠,和他大舅一路家去了。今日白扒不起來,不是來安兒去叫,我還睡哩。哥,你好漢!還起的早,若着我,成不的。」西門慶道:「早是你看着,我怎得個心閒?自從發送他出去了,又亂着接黃太尉,念經,直到如今,心上是那樣不遂。今早房下說,你辛苦了,太睡回起去;我又記挂着,只怕翟親家人來討回書,又看着拆棚,二十四日又打發韓夥計和小价起身,打包寫書帳。喪事費勞了。人家親朋罷了,士夫官員,你不上門謝謝孝禮,也過不去。」伯爵道:「正是我愁着哥謝孝這一節,少不的也謝,只摘撥謝幾家要緊的,胡亂也罷了。其餘相厚,若會見,告過就是了。誰不知你府上事多,彼此心照罷。」正說着,只見王經抓簾子,畫童兒用彩漆方盒銀廂雕漆茶鍾,拿了兩盞酥油白糖熬的牛奶子 。伯爵取過一盞,拿在手內,見白瀲瀲鵝脂一般,酥油飄浮盞內,說道:「好東西!滾熱。」呷在口裡,香甜美味。那消費力,幾口就呵沒了。西門慶直待篦了頭,又教小周兒替他取耳,把奶子放在卓上,只顧不吃。伯爵道:「哥,且吃些不是,可惜放冷了。相你清辰吃恁一盞兒,倒也滋補身子。」西門慶道:「我且不吃,你吃了,停會我吃粥罷!」那伯爵得不的一聲,拿在手中,一吸而盡。畫童收下鍾去。西門慶取畢耳,又叫小周兒拿木滾子〈扌袞〉身上,行按摩導引之術。伯爵問道:「哥滾着身子,也通泰自在些麼?」西門慶道:「不瞞你說,相我晚夕身上常時發酸起來,腰背疼痛。不着這般按捏,通了不得。」伯爵道:「你這胖大身子,日逐吃了這等厚味,豈無痰火?」西門慶道:「昨日任後溪常說,老先生雖故身體魁偉,而虛之太極,送了我一罐兒百補延齡丹,說是林真人合與聖上吃的,教我用人乳常清辰服;我這兩日心上亂的,也還不曾吃。你每只說我身邊人多,終日有此事;自從他死了,誰有甚麼心緒理論此事!」正說着,只見韓道國進來,作揖坐下,說:「剛纔各家多來會了,船已顧下,准在二十四日起身。」西門慶分付甘夥計,攢下帳目;兌了銀子,明日打包。因問:「兩邊鋪子裡,賣下多少銀兩?」韓道國說:「共湊六千餘兩。」西門慶道:「兌二千兩一包,着崔本往湖州買紬子去。那四千兩,你與來保往松江販布,過年趕頭本船來。你每人先拿五兩銀子,家中收拾行李去。」韓道國道:「又一件,小人身從鄆王府,要正身上直,不納官錢,如何處置?」西門慶道:「怎的不納官錢?相來保一般,也是鄆王差事,他每月只納三錢銀子。」韓道國道:「保官兒那個,虧了太師老爺那邊文書上註過去,便不敢纏擾;小人此是祖役,還要勾當餘丁。」西門慶道:「既是如此,你寫個揭帖,我央任後溪到府中替你和王奉承說,把你官字註銷,常遠納官錢罷!你每月只委付家下一個的當人打米就是了。」那韓夥計作揖謝了。伯爵道:「哥,你這一替他處了這作事,他就去也放心。」少頃,小周滾畢身上,西門慶往後邊梳頭去了,分付打發小周兒吃了點心。良久,西門慶出來,頭戴白絨忠靖冠,身披絨氅,賞了小周三錢銀子。又使王經:「請你溫師父來。」不一時溫秀才峨冠博帶而至。敘禮已畢,左右放卓兒,拿粥上來,四碟小菜,一碗頓爛蹄子,一碗黃芽韮〈火川〉驢肉 ,一碗鮓餛飩雞,一碗頓爛鴿子鶵兒,四甌軟稻粳米粥兒 ,安放四雙牙筯。伯爵與溫秀才上坐,西門慶關席,韓道國打橫。西門慶分付來安兒再取一盞粥,一雙快兒,請你姐夫來吃粥。不一時,陳經濟來到,頭戴孝巾,身穿白紬道袍,葱白段氅衣,蒲鞋絨襪,與伯爵等作揖,打橫坐下。須臾,吃了粥,收下家火去,韓道國起身去了。只有伯爵、溫秀才,在書房坐的。西門慶因問溫秀才:「書可寫了不曾?」溫秀才道:「學生已寫稿在此,與老生看過,方可謄真。」一面袖中取出,遞與西門慶觀看。其書曰:

「寓清河眷生西門慶,端肅書復大碩德柱國雲峰老親丈大人先生台下:自從京邸邂逅,數語之後,不覺違越光儀,絛忽半載。生以不幸,閨人不祿,特蒙親家致賻儀,兼領誨教,足見為我之深且厚也。感刻無任,而終身不能忘矣。但恐一時官守責成,有所疎陋之處,企仰門墻,有負薦拔耳。又賴在老翁鈞前,當為錦覆,則生始終蒙恩之處,皆親家所賜也。今因便鴻,謹候起居,不勝馳戀,伏惟炤亮不宣。外具楊州縐紗汗巾十方,色綾汗巾十方,揀金挑牙二十付,烏金酒鍾十個,少將遠意,希笑納。」

西門慶看畢,即令陳經濟書房內取出人事來,同溫秀才封了,將書謄付錦箋,彌封停當,用了圖書。另外又封五兩白銀,與下書人王玉,不在話下。一回見雪下的大了,西門慶留下溫秀才在書房中賞雪。搽抹卓兒,拿上案酒來。只見有人在煖簾外探頭兒,西門慶問:「誰?」王經說:「鄭春在這裡。」西門慶叫他進來,那鄭春手內拿着兩個盒兒,舉的高高的,跪在當面,上頭又閣着個小描金方盒兒。西門慶問:「是甚麼?」鄭春道:「小的姐姐月姐,知道昨日爹與六娘念經辛苦了,沒甚麼,送這兩盒兒茶食兒來,與爹賞人。」揭開,一盒菓餡頂皮酥,一盒酥油泡螺兒 。鄭春道:「此是月姐親手自家揀的,知道爹好吃此物,敬來孝順爹。」西門慶道:「昨日又多謝你家送茶,今日你月姐費心,又送這個來。」伯爵道:「好呀,拿過來,我正要嚐嚐。死了我一個女兒,會揀泡螺兒,如今又是一個女兒會揀了。」先捏了一個放在口內,又拈了一個遞與溫秀才,說道:「老先兒,你也嚐嚐。吃了牙老重生,抽胎換骨,眼見稀奇物,勝活十年人。」溫秀才呷在口內,入口而化,說道:「此物出于西域,非人間可有。沃肺融心,實上方之佳味。」西門慶又問:「那小盒兒內是什麼?」鄭春悄悄跪在西門慶根前,揭開盒兒,說:「此是月姐稍與爹的物事。」西門慶把盒子放在膝盖兒上,揭開纔待觀看,一邊件爵一手撾過去,打開,是一方迴紋錦雙攔子,細撮古碌錢,同心方勝結穗挑紅綾汗內兒,裡面裹着一包親口磕的瓜仁兒。這伯爵把汗巾兒掠與西門慶,將瓜仁兩把喃在口裡,都吃了。比及西門慶用手奪時,只剩下沒多些兒,便罵道:「怪狗才,你害饞癆饞痞,留些兒與我見見兒,也是人心!」伯爵道:「我女兒送來,不孝順我,再孝順誰?我兒,你尋常吃的勾了!」西門慶道:「溫先兒在此,我不好罵出來。你這狗材,或不相模樣!」一面把汗巾收入袖中,分付王經把盒兒掇在後邊去。不一時杯盤羅列,篩上酒來。纔吃了一巡酒,玳安兒來說:「李智、黃四,關了銀子,送銀子來了。」西門慶問:「多少?」玳安道:「他說一千兩,餘者再一限送來。」伯爵道:「你看這兩個天殺的,他連我也瞞了,不對我說。嗔道他昨日你這裡念經,他也不來。原來往東平府關銀子去了。你今收了,也少要發銀子出去了。這兩個光棍,他攬的人家債也多了,只怕往後,後手不接。昨日北邊徐內相,發恨要親往東平府自家抬銀子去,只怕他老生箍嘴箍了去,都不難為哥的本錢了。」西門慶道:「我不怕他。我不管甚麼徐內相、李內相,好不好我把他小廝提留在監裡坐着,不怕他不與我銀子。」一面教陳經濟:「你拿天平出去,收兌了他的,上了合同就是了;我不出去罷。」良久,陳經濟走來回話,說:「銀子已兌足一千兩,交入後邊大娘收了。黃四說,還要請爹出去說句話兒。」西門慶道:「你只說我陪着人坐着哩。左右他只要搗合同的話,教他過了二十四日來罷。」經濟道:「不是,他有庄事兒要央煩爹,請爹出去,親自對爹說。」西門慶道:「甚麼事?等我出去。」一面走到廳上。那黃四磕頭起來,說:「銀子一千兩,姐夫收了,餘者下單找還與老爹。有小人一庄事兒,今央煩老爹。」說着,磕在地下哭了。西門慶拉起來:「端的有甚麼事?你說來。」黃四道:「小的外父孫清,搭了個夥計馮二,在東昌府販綿花。不想馮二有個兒子馮淮,不守本分,要便鎖了門,出去宿娼。那日把綿花不見了兩大包,被小人丈人說了兩句,馮二將他兒子打了二下,他兒子就和俺小舅子孫文相廝打攘起來,把孫文相牙打落了一個,他亦把頭磕傷,被客夥中解勸開。不想他兒子到家遲了半月,破傷風身死。他又人是河西有名土豪白五,綽號白千金,專一與強盜作窩主,教唆馮二,具狀在巡按衙門,朦朧告下來,批雷兵備老爹問。雷老爹又伺侯皇船,不得閑,轉委本府童推官問。白家在童推官處使了錢,教鄰勸人供狀,說小人丈人在傍喝聲來,如今童推官行牌來提俺丈人。望乞老爹千萬垂憐,討封書對雷老爹說,寧可監幾日,抽上文書去,還見雷老爹問,就有生路了。他兩人廝打,委的不管小人丈人事;又係歇後身死,出于保辜限外,先是他父馮二打來,何必獨賴在孫文相一人身上?」西門慶看了說帖,寫着:「東昌府見監犯人孫清、孫文相、乞青目。」因說:「雷兵備前日在我這裡吃酒,我只會了一面,又不甚相熟,我怎好寫書與他?」那黃四就跪下,哭哭啼啼哀告說:「老爹若不可憐見,小的丈人子父兩個,就多是死數了。如今隨孫文相頭去罷了,只是分豁小人外父出來,就是老爹莫大之恩。小人外父今年六十歲,家下無人。冬寒時月,再放在監裡,就死罷了!」西門慶沉吟良久,說:「罷,我轉央鈔關錢老爹和他說說去;與他是同年,多是壬辰進士。」那黃四又磕下頭去,向袖中又取出一百石白米帖兒遞與西門慶,腰裡就解兩封銀子來。西門慶不接,說:「我那裡要你這行錢?」黃四道:「老爹不稀罕,謝錢老爹也是一般。」西門慶道:「不打緊,事成我買禮謝他。」正說着,只見應伯爵從角門首出來,說:「哥,休替黃四哥說人情,他閒時不燒香,忙時走來抱佛腿。昨日哥這裡念經,連茶兒也不送,也不來走走兒,今日還來說人情。」那黃四便與伯爵唱喏,說道:「好二叔,你老人家殺人哩!我因這件事,整走了這半月,誰得閒來?昨日又去府裡與老爹領這銀子。今日李三哥起早打卯去了,我竟來老爹這裡交銀子,就央說此事,救俺丈人。老爹再三不肯收這禮物,還是不下顧小人。」伯爵看見是一百兩雪花官銀放在面前,因問:「哥,你替他去說不說?」西門慶道:「我與雷兵備不熟,如今又轉央鈔關錢主政替他說去。到明日我買分禮,謝老錢就是了,又收他禮做甚麼?」伯爵道:「哥,你這等就不是了。難說他來說人情,哥你賠出禮去謝人?也無此道理。你不收,恰是你嫌少的一般,倒難為他了。你依我,收下他這個禮。雖你不稀罕,明日謝錢公,又是一個樣兒。黃四哥在這裡聽着,看你外父和你小舅子造化,這一回求了書去,難得兩個多沒事出來。你老爹他恒是不稀罕你錢,你老院裡老實大大擺一席酒,請俺每耍一日就是了。」黃四道:「二叔你老人家費心,小心擺酒不消,就還教俺丈人買禮來磕頭,酬謝你老人家。不瞞你,我為他爺兒兩個這一場事,晝夜上下替他走跳,還尋不出個門路來。老爺再不可憐怎了?」伯爵道:「傻瓜,你摟着他女兒,你不替他上緊,誰上緊?」黃四道:「房下在家只是哭,俺丈人便躲了,家連送飯人也沒一個兒。」當下西門慶被伯爵說着,把禮帖收了,禮物還令他拿回去。黃四道:「你老人家沒見好大事,這般多計較!」就往外走。伯爵道:「你過來,我和你說,你書幾時要?」黃四道:「如今緊等着救命,老爹今日下顧有了書,差下人,明早我使小兒同去走遭。」于是央了又央:「差那位大官兒去?我會他會。」西門慶道:「我就替你寫書。」因叫過玳安來,分付:「你明日就同黃大官一路去。」那黃四見了玳安,辭西門慶出門。走到門首,問玳安要盛銀子搭連。玳安進入後邊,月娘房裡正與玉簫、小玉裁衣裳,見玳安站看等要搭連,玉筆道:「使着手不得閒謄,教他明日來,與他就是了。」玳安道:「黃四緊等着明日早起身東昌府去,不得來了。你謄謄與他罷!」月娘便說:「你拿與他就是了,只教人家等着。」玉簫道:「銀子還在牀地平上,掠着不是?」走到裡間,把銀子往牀上只一倒,掠出搭連來,說:「拿去了,怪囚根子,那個吃了他這條搭連,只顧立虰螞蝗的要。」玳安道:「人家不要,那好來後邊取來?」于是拿出,走到儀門首,還抖出三兩一塊蔴姑頭銀子來。原來紙包破了,怎禁玉筆使性那一倒,漏下一塊在搭連底內。玳安道:「且喜得我拾個白財。」于是褪入袖中,到前邊遞與黃四搭連,約會下明早起身。且說西門慶回到書房中,即時教溫秀才修了書,付與玳安。一面覷那門外雪,紛紛揚揚,猶如風飄柳絮,亂舞朵花相似。西門慶另打開一罈雙料麻姑酒 ,教春鴻用布甑篩上來。鄭春在傍彈箏低唱。西門慶令他唱一套柳底風徵。正唱着,只見琴童進來說:「韓大叔教小的拿了這個帖兒與爹瞧。」西門慶看了,分付:「你就拿往門外任醫官家,替他說說去,教他明日到府中承奉處替他說說,註銷差事。」琴童道:「今日晚了,小的明早去罷。」西門慶道:「是了。」不一時,來安兒用方盒拿了八碗下飯:一碗黃熬山藥雞,一碗臊子韮,一碗山藥肉圓子,一碗頓爛羊頭,一碗燒豬肉,一碗肚肺羹 ,一碗血臟湯 ,一碗牛肚兒,一碗爆炒豬腰子 ;又是兩大盤玫瑰鵝油盪麵蒸餅兒,連陳經濟共四人吃了。西門慶教王經拿盤兒,拿兩碗下飯,一盤點心,與鄭春吃,又賞了他兩大鍾酒。鄭春跪稟:「小的吃不的。」伯爵道:「俊孩兒!冷呵呵的,你爹賞你不吃,你哥他怎的吃來?」鄭春道:「小的哥吃的,小的本吃不的。」伯爵道:「你吃一鍾罷,那一鍾教王經替你吃。」王經道:「二爹,小的也吃不的。」伯爵道:「你這孩兒,你就替他吃些兒也罷。休說一個大分上,自古長者賜,少者不敢辭。」一面站起來,說:「我好歹教你吃這一杯。」那王經捏着鼻子,一吸而飲。西門慶道:「怪狗材,小行貨子,他吃不的,只恁奈何他吃!」還剩下半盞,教春鴻替他吃了,令他上來排手唱南曲。西門慶道:「咱每和溫老先兒行個令,飲酒之時教他唱,便有趣。」于是教王經取過骰盆兒,就是溫老先兒先起。溫秀才道:「學生豈敢僣?還從應老翁來。」因問:「老翁尊號?」伯爵道:「在下號南坡。」西門慶戲道:「老先生,你不知他家孤老多,到晚夕桶子掇出屎來,不敢在左近倒,恐怕街坊人罵,教丫頭直掇到大南首縣倉墻底下那裡潑去,因起號叫做『南潑』。」溫秀才笑道:「此坡字不同,那潑字乃是點水邊之發,這坡字都是土字傍邊着個皮字。」西門慶道:「老先兒倒猜的着,他娘子鎮日着皮子纏着哩!」溫秀才笑道:「豈有此說!」伯爵道:「葵軒,你不知道,他自來有些快傷叔人家。」溫秀才道:「自古道:言不褻不笑。」伯爵道:「老先兒誤了咱每行令,只顧和他說甚麼?他快屎口傷人,你就在手,不勞謙遜。」溫秀才道:「擲出幾點,不拘詩詞歌賦,要個雪字上。就照依點數兒上,說過來,飲一小杯;說不過來,吃一大盞。」當夜溫秀才擲了個么點,說道:「學生有了,雪殘鸂鶒亦多時。」推過去該應伯爵行,擲出個五點來,伯爵想了半日,想不起來,說:「逼我老人家命也。」良久說道:「可怎的也有了!」說道:「雪裡梅花雪裡開,好不好?」溫秀才道:「老翁說差了,犯子兩個雪字,頭上多了一個雪字。」伯爵道:「頭上只小雪,後來下大雪來了。」西門慶道:「這狗材,單管胡說。」教王經斟上大鍾。春鴻拍手唱南曲駐馬廳:

「寒夜無茶,走向前村覓店家。這雪輕飄,僧舍密酒,歌樓遙阻歸槎。江邊乘興探梅花,庭中歡賞燒銀蠟。一望無涯,一望無涯,有似濡橋柳絮,滿天飛下。」

伯爵纔待拿起酒來吃,只見來安兒後邊拿了幾碟菓食:一碟菓餡餅,一碟頂皮酥,一碟炒栗子,一碟晒乾棗 ,一碟榛仁,一碟瓜仁,一碟雪梨 ,一碟蘋波,一碟風菱,一碟荸薺 ,一碟酥油泡螺 ,一碟黑黑的團兒,用橘葉裹着。伯爵拈將起來,聞着噴鼻香,吃了到口,猶如飴蜜 ,細甜美味,不知甚物?西門慶道:「你猜?」伯爵道:「莫非是糖肥皂?」西門慶笑道:「糖肥皂那有這等好吃?」伯爵道:「待要說是梅蘇丸 ,裡面又有核兒。」西門慶道:「狗材,過來!我說與你罷。你做夢也夢不着,是昨日小价杭州船上稍來,名喚做衣梅 。都是各樣藥料,用蜜煉製過,滾在楊梅上,外用薄荷橘葉包裹,纔有這般美味。每日清辰,呷一枚在口內,生津補肺,去惡味。煞痰火,解酒剋食,比梅蘇丸甚妙。」伯爵道:「你不說,我怎的曉的?」因說:「溫老先兒,咱再吃個兒。」教王經:「拿張紙兒來,我包兩丸兒到家,稍與你二娘吃。」又拿起泡螺兒來,問鄭春:「這泡螺果然是你家月姐親手揀的?」那鄭春跪下說:「二爹,莫不小的敢說謊?不知月姐費了多少心,揀了這幾個兒來供孝順爹。」伯爵道:「可也虧他,上頭紋溜就相螺螄兒一般,粉紅純白兩樣兒。」西門慶道:「我見此物,不免又使傷我心。惟有死了的六娘,他會揀;他沒了,如今家中誰會弄他?」伯爵道:「我頭裡不說的,我愁甚麼,死了一個女兒會揀泡螺兒孝順我,如今又鑽出個女兒會揀了!偏你也會尋,尋的多是妙人兒!」西門慶笑的兩眼沒縫兒,趕着伯爵打,說:「你這狗材,單管只胡說!」溫秀才道:「二位老先生,可謂厚之至極!」伯爵道:「老先生,你不知,他是你小姪人家。」西門慶道:「我是他家二十年舊孤老兒了。」陳經濟見二人犯言,就起身走了。那溫秀才只是掩口而笑。須臾,伯爵飲過大鍾,次該西門慶擲骰兒,于是擲出個七點來,想了半日,說:「我打香羅帶一句唱:『東君去意切,梨花似雪。』伯爵道:「你說差了,此在第八個字上了,且吃一個大鍾。」于是流沿兒斟了一銀衢花鍾,放在西門慶面前,教春鴻唱,說道:「我的兒,你肚子裡棗胡解板兒,能有幾句兒?」春鴻又排手唱前腔:

「四野彤霞,回首江山自佔涯。這雪輕如柳絮,細似鵝毛,白勝梅花。山前曲徑更添滑,村中魯酒偏增價。疊墜天花,疊墜天花,濠平溝滿,令人驚訝。」

看看飲酒至昏,掌燭上來。西門慶飲過,伯爵道:「姐夫不在,溫老先生你還該完令。」這溫秀才拿起骰兒,擲出個么點,想了想,見書房墻上挂着一幅吊屏,泥金書一聯:「風飄弱柳平橋晚,雪點寒梅小院春。」說了未後一句,伯爵道:「不算,不算。不是你心上發出來的,該吃一大鍾。」春鴻斟上,那溫秀才不勝酒力,坐在椅上只顧打盹起來,告辭。伯爵只顧留他不住。西門慶道:「罷罷,老先生他斯文人,吃不的。」令畫童兒:「你好好送你溫師父那邊歇去。」溫秀才得不的一聲,作別去了。伯爵道:「今日葵軒不濟,吃了多少酒兒就醉了。」于是又飲勾多時,伯爵起身,說:「地下黑,我也酒勾了。」因說:「哥,明日你早教玳安替他下書去。」西門慶道:「你不見我交與他書,明日早去了。」伯爵掀開簾兒,見天陰地下滑,旋要了個燈籠,和鄭春一路去。西門慶又與了鄭春五錢銀子,盒內回了一罐衣梅,稍與他姐姐鄭月兒吃。臨出門,西門慶因戲伯爵:「你哥兒兩個好好去。」伯爵道:「你多說話,父子上山,各人努力。好不好,我如今就和鄭月兒那小淫婦兒答話去。」說着,琴童送出門去了。西門慶看收了家火,扶着來安兒,打燈籠入角門,從潘金蓮門首過,見角門關着。悄悄就往李瓶兒房門首,彈了彈門,有綉春開了門,來安就出去了。西門慶進入明間,見李瓶兒影,問:「供養了羹飯不曾?」如意兒就出來應道:「剛纔我和姐供養了。」西門慶入房中,椅上坐了,迎春拿茶來吃了。西門慶令他解衣帶。如意兒就知他在這房裡歇,連忙收拾伸鋪,用湯婆熨的被窩暖洞洞的,打發他歇下。綉春把角門關了,都在明間地平上,支着板凳,打鋪睡下。西門慶要茶吃,兩個已知科範,連忙攛掇奶子進去和他睡。老婆脫了衣服,鑽入被窩內。西門慶乘酒興服了藥,那話上使了托子,老婆仰臥炕上,架起腿來,極力鼓搗,沒高低搧硼,搧硼的老婆舌尖水冷,淫水溢下,口中呼達達不絕。夜靜時分,其聲遠聆數室。西門慶見老婆身上如綿瓜子相似,用一雙胳膊摟着他,令他蹲下身子,在被窩內砸{髟巳}{髟己},老婆無不曲體承奉。西門慶說:「我兒,你原來身體皮肉也和你娘一般白淨,我摟着你,就如同和他睡一般。你須用心伏侍我,我看顧你。」老婆道:「爹沒的說,將天比地,折殺奴婢,拿甚麼比娘?奴婢男子漢已沒了,早晚爹不嫌醜陋,只看奴婢一眼兒就勾了。」西門慶便問:「你年紀多少?」老婆道:「我今年屬兔的,三十一歲了。」西門慶道:「你原來小我一歲。」見他會說話兒,枕上又好風月。早晨起來,老婆先起來伏侍,拿鞋襪,打發梳洗,極盡慇懃,把迎春、綉春打靠後。又問西門慶討葱白紬子,做披襖兒與娘穿孝,西門慶一一許他。教小廝舖子裡拿三疋葱白紬來,你每一家裁一件。以此見他兩三次,打動了心,瞞着月娘,背地銀錢衣服首飾甚麼不與他。次日,潘金蓮就打聽得知西門慶在李瓶兒房內,和奶子老婆睡了一夜。走到後邊對月娘說:「大姐姐,你不說他幾句?賊沒廉耻貨,昨日悄悄鑽到那邊房裡,與老婆歇了一夜。餓眼見瓜皮,甚麼行貨子,好的歹的攬搭下!不明不暗,到明日弄出個孩子來,算誰的?又相來旺兒媳婦子,往後教他上頭上臉,甚麼張致!」月娘道:「你每只要裁派教我說他,要了死了的媳婦子。你每背地多做好人兒,只把我合在缸底下一般。我如今又做傻子哩!你每說,只顧和他說,我是不管你這閑帳!」金蓮見月娘這般說,一聲兒不言語,走回房去了。西門慶起早,見天晴了,打發玳安往錢主事處下書去了。往衙門回來,平安兒來稟:「翟爹人來討回書。」西門慶打發去訖,因問那人:「你怎的昨日不來取?」那人說:「小的又往巡撫侯爺那裡下書來,擔閣了兩日。」說畢,領書出門。西門慶吃了飯,就過對門房子裡,看着兌銀、打包、寫書帳。二十四日燒布,打發夥計崔本、來保、并後生榮海、胡秀五人,起身往南邊去了。寫了一封書,稍與苗小湖,就謝他重禮。看看過了二十五六,西門慶謝畢孝,一日早辰,在上房吃了飯坐的。月娘便說:「這出月初一日,是喬親家長姐生日,咱也還買分禮兒送了去。常言:先親後不改。莫非咱家孩兒沒了,斷了禮不送了!」西門慶道:「怎的不送?」于是分付來興買兩隻燒鵝 ,一副豕蹄,四隻鮮雞,兩隻燻鴨 ,一盤壽麵,一套粧花段子衣服,兩方絹金汗巾,一盒花翠,寫帖兒教王經送去。這西門慶分付畢,就往前邊花園藏春閣書房中坐的。只見玳安下了書回來回話,說:「錢老爹見了爹帖子,隨即寫書,差了一吏,同小的和黃四兒子到東昌府兵備道下與雷老爹;老爹旋行牌問童推官催文書,連犯人提上去,從新問理。連他家兒子孫文相都開出來,只追了十兩燒埋錢,問了個不應罪名,杖七十,罰贖。後又到鈔關上回了錢老爹的話,討了回帖纔來了。」西門慶見玳安中用,心中大喜。拆開回帖觀看,原來雷兵備回錢主事帖子,多在裡面。上寫道:

「來諭悉已處分。但馮二已曾供子在先,何況與孫文相忿毆,彼此俱傷,歇後身死,又在保辜限外,問之抵命,難以平允。量追燒埋錢十兩,給與馮二。相應發落,謹此回覆。下書年侍生雷起元再拜。」

西門慶看了歡喜,因問:「黃四舅子在那裡?」玳安道:「他出來,都往家去了,明日同黃四來與爹磕頭。黃四丈人與了小的一兩銀子。」西門慶分付置鞋腳穿。玳安磕頭而出,西門慶就〈扌歪〉在牀炕上眠牀了。王經在卓上小篆內炷了香,悄悄出來了。良久,忽聽有人抓的簾兒响,只見李瓶兒驀地進來,身穿糝紫衫,白絹裙,亂挽烏雲,黃懨懨面容,向牀前叫道:「我的哥哥,你在這裡睡哩!奴來見你一面。我被那廝告了我一狀,把我監在獄中,血水淋漓,與穢污在一處,整受了這些時苦。昨日蒙你堂上說了人情,減了我三等之罪。那廝再三不肯,發恨還要告了來拿你。我待要不來對你說,誠恐你早晚暗遭他毒手。我今尋安身之處去也,你須防範來!沒事,少要在外吃夜酒。往那去,早早來家。千萬牢記,奴言休要忘了。」說畢,兩人抱頭放聲而哭。西門慶便問:「姐姐,你往那去?對我說。」李瓶兒頓脫撒手,都是南柯一夢。西門慶從睡夢中直哭醒來,看見簾影射入書齋,正當卓午,追思起,由不的心中痛切,正是:

「花落土埋香不見,  鏡空鸞影夢初醒。」

有詩為證:

「殘雪初晴照紙窗,  地爐灰燼冷侵牀,

個中邂逅相思夢,  風撲梅花斗帳香。」

不想早辰送了喬親家禮,喬大戶娘子使了喬通來送請帖兒,請月娘眾姊妹。小廝說爹在書房中睡哩,都不敢來問。月娘在後邊管待喬通。潘金蓮說:「拿帖兒,等我問他去。」于是驀地進書房。上穿黑青迴紋錦對衿衫兒,泥金眉子,一溜〈扌寨〉五道金,三川鈕扣兒。下着紗裙,內襯潞紬裙,羊皮金滾邊。面前垂一雙合歡鮫綃鸂鶒帶,下邊尖尖趫趫,錦紅膝褲,下顯一對金蓮。頭上寶髻雲鬟,打扮如粉粧玉琢。耳邊帶着青寶石墜子。推開書房門,見西門慶〈扌歪〉着,他一屁股坐在椅子上,說:「我的兒,獨自個自言自語,在這裡做甚麼?嗔道不見你,原在這裡好睡也!」一面說話,口中磕瓜子兒。因問西門慶:「眼怎生揉的恁紅紅的?」西門慶道:「我控着頭睡來。」婦人道:「倒只相哭的一般。」西門慶道:「怪奴才,我平白怎的哭?」金蓮道:「只怕你一時想起甚心上人兒來是的。」西門慶道:「沒的胡說,有甚心上人,心下人!」金蓮道:「李瓶兒是心上的,奶子是心下的。俺每是心外的人,入不上數。西門慶道:「怪小淫婦兒,又六說白道起來!」因問:「我和你說正話,前日李大姐裝槨,你每替他穿了甚麼衣服在身底下來?」金蓮道:「你問怎的?」西門慶道:「不怎的,我問聲兒。」金蓮道:「你問必有個緣故。上面他穿兩套遍地金段子衣服,底下是白綾襖,黃紬裙,貼身是紫綾小襖白絹裙,大紅段小衣。」西門慶點了點頭兒。金蓮道:「我做獸醫二十年,猜不着驢肚裡病,你不想他,問他怎的?」西門慶道:「我纔方夢見他來。」金蓮道:「夢是心頭想,涕噴鼻子痒。饒他死了,你還這等念他。相俺多是可不着你心的人,到明日死了苦惱,也沒那人顯念,此是想的你這心裡胡油油的!」西門慶向前一手摟過他脖子來,就親了個嘴,說:「怪小油嘴,你有這些賊嘴賊舌的。」金蓮道:「我的兒,老娘猜不着你那黃貓黑尾的心兒!」一面把磕了的瓜子仁兒,滿口哺與西門慶吃。兩個又咂了一回舌頭,自覺甜唾溶心,脂滿香唇,身邊蘭麝襲人。西門慶于是淫心輙起,摟他在牀上坐。他便仰靠梳背,露出那話來,教婦人品簫,婦人真個低垂粉頭,吞吐裹沒往來,嗚咂有聲。西門慶見他頭上戴金赤虎,分心香雲,上圍着翠梅花鈿兒,後鬢上珠翹錯落,興不可遏。正做到美處,忽聽來安兒隔簾說:「應二爹來了。」西門慶道:「請進來。」慌的婦人沒口子叫來安兒:「賊,且不要叫他進來,等我出去着。」來安兒道:「進來了,在小院內。」婦人道:「還不去教他躲躲兒?」那來安兒走去說:「二爹且閃閃兒,有人在屋裡。」這伯爵便走松墻傍邊看雪培竹子。王經掀着軟簾,只聽裙子响,金蓮一溜烟後邊走了。正是:

「雪隱鷺鷥飛始見,  柳藏鸚鵡語方知。」

伯爵進來,見西門慶唱喏坐下。西門慶道:「你連日怎的不來?」伯爵道:「哥,惱的我要不的在這裡!」西門慶問道:「又怎的惱?你告我說。」伯爵道:「不告你說,緊自家中沒錢,昨日俺房下那個,平日又桶出個孩兒來!但是人家白日裡還好撾撓,半夜三更,房下又七痛八病,少不得扒起來,收拾草紙被褥,陸續看他叫老娘去。打緊應寶又不在家,俺家兄使了他往庄子上馱草去了。百忙撾不着個人,我自家打着燈籠,叫了巷口兒上鄧老娘來。及至進門,養下來了。」西門慶問:「養個甚麼?」伯爵道:「養了個小廝。」西門慶罵道:「傻狗材,生了兒子倒不好,如何反惱?是春花兒那奴才生的?」伯爵笑道:「是你春姨人家。」西門慶道:「那賊狗掇腿的奴才,誰教你要他來?叫叫老娘還抱怨。」伯爵道:「哥,你不知,冬寒時月,比不的你每有錢的人家;家道又有錢,又有若大前程官職,生個兒子上來,錦上添花,便喜歡。俺如今自家還多着個影兒哩,家中一窩子人口要吃穿盤攪。自這兩日,媒巴劫的魂也沒了!應寶逐日該操,當他的差事去了。家中那裡是不管的,大小姐便打發出去了。天理在頭上,多虧了哥,你眼見的這第二個孩子又大了,交年便是十三歲。昨日媒人來討帖兒,我說早哩,你且去着。緊自焦的魂也沒了,猛可半夜又鑽出這個業障來!那黑天摸地,那裡活變錢去?房下見我抱怨,沒計奈何,把他一根銀插兒與了老娘,發落去了。明日洗三,嚷的人家知道了,到滿月拿甚麼使?到那日我也不在家,信信拖拖,往那寺院裡且住幾日去罷。」西門慶笑道:「你去了好了,和尚都打發來,好趕熱被窩兒。你這狗材,到底占小便益兒!」又笑了一回。那應伯爵故意把嘴谷嘟着,不做聲。西門慶道:「我的兒,不要惱。你用多少銀,一對我說,等我與你處。」伯爵道:「有甚多少?」西門慶道:「也勾你攪纏是的,到其間不勾了,又拿衣服當去。」伯爵道:「哥若肯下顧,二十兩銀子就勾了。我寫個符兒在此,費煩的哥多了,不好開口的,也下敢嗔數兒,隨哥尊意便了。」那西門慶也不接他文約,說:「沒的扯淡,好朋友家什麼符兒?」正說着,只見來安兒拿茶進來。西門慶叫小廝:「你放下盞兒,喚王經來。」不一時,王經來到,西門慶分付:「你往後邊對你大娘說,我裡間牀背閣上,有前日巡按宋老爹擺酒兩封銀子,拿一封來。」王經應諾,去不多時,拿銀子來。西門慶就遞與應伯爵,說:「這封五十兩,你多拿了使去,省的我又拆開他。原封未動,你打開看看。」伯爵道:「忒多了。」西門慶道:「多的你收着。眼下你二令愛不大了,你可也替他做些鞋腳衣裳,到滿月也好看。」伯爵道:「哥說的是。」將銀子拆開,都是兩司各府傾就分資,三兩一定,松紋足色。滿心歡喜,連忙打恭致謝,說道:「哥的盛情,誰肯真個不收符兒?」西門慶道:「傻孩兒,誰和你一般計較?左右我是你老爺老娘家。不然,你但有事來,就來纏我?這孩子也不是你的孩子,自是咱兩個分養的,實和你說,過了滿月,把春花兒那奴才叫了來,且答應我些時兒,只當利錢,不算發了眼。」伯爵道:「你春姨這兩日,瘦的相你娘那樣哩!」不說兩個在書房中說話。伯爵因問:「黃四丈人那事怎樣了?」西門慶把玳安往返的事告說了一遍:「錢龍野書到,雷兵備旋行牌提了犯人上去,從新問理,把孫文相父子兩個都開出來了,只認十兩燒理錢,罰了杖罪,沒事了。」伯爵道:「造化他了。他就點着燈兒,那裡尋這人情去?你不受他的,乾不受他的,雖然你不希罕,留送錢大人也好。別要饒了他,教他好歹擺一席大酒,裡邊請俺每坐一坐。你不說,等我和他說。饒了他小舅一個死罪,當別的小可事兒。」這裡說話。且說月娘在上房,拿銀子與王經出來,只見孟玉樓走入房來,說他兄弟孟銳,在韓姨夫那裡,如今不久又起身,往川廣販雜貨去:「今來辭辭他爹,在我屋裡坐着哩,爹在那裡?姐姐使個小廝對他爹說聲兒。」月娘道:「他在花園書房,和應二坐着哩。又說請他爹哩,頭裡潘六姐倒請的好他爹!喬通送帖兒來,等着問他爹去,就討他個話兒,到明日咱每好收拾了去。我便把喬通留下,打發吃茶。長等短等,不見來,熬的喬通也去了。半日只見他從前邊走將來,故我問他:『你對他說了不曾?』他沒的話說:『噦,我就忘了和他說。一回,應二來了,我就出來了。誰得久停久住,和他說話來?』帖子還袖在袖子裡,教我說脆幫根兒咬!早是沒甚緊勾當,教人只顧等着。你原來恁個沒尾八行貨子,不知在前頭幹甚麼營生,那半日纔進來!恰好還不曾說,乞我訌了兩句,往前去了。」少頃,來安進來,月娘使他請西門慶,說孟二舅來了。西門慶便起身,留伯爵:「你休去了,我就來。」走到後邊,月娘先把喬家送帖來請說了。西門慶說:「那日只你一人去罷。熱孝在身,莫不一家子都出來?」月娘說:「他孟二舅來辭辭你,一兩日起身往川廣去也,在那邊屋裡坐着哩。」又問:「頭裡你要那封銀子與誰?」西門慶悉把應二哥房裡春花兒,昨晚生了個兒子,問我借幾兩銀子使,告我說,他第二個女兒又大,愁的要不的,借助幾兩銀子使罷了。月娘道:「好好!他恁大年紀,也纔見這個兒子,應二嫂不知怎的喜歡哩!到明日,咱也少的送些粥米兒與他。」西門慶道:「這個不消說。到滿月,不要饒花子,奈何他好歹發帖兒,請你們往他家走走去,就瞧瞧春花兒怎麼模樣?」月娘笑道:「左右和你家一般樣兒,也有鼻兒有眼兒,莫非別些兒?」一面使來安下邊請孟二舅來。不一時,玉樓同他兄弟來拜見,敘禮已畢,西門慶陪他敘了回話,讓至前邊書房內,與伯爵相見。分付小廝後邊看菜兒,于是放卓兒,篩酒上來,三人飲酒。西門慶教再取雙鍾筯,對門請溫師父,陪你二舅坐。來安不一時回說:「溫師父不在,望倪師父去了。」西門慶說:「請你姐夫來坐坐。」良久,陳經濟來,與二舅見了禮,打橫坐下。西門慶問:「二舅幾時起身?去多少時?」孟銳道:「出月初二日准起身,定不的年歲。還到荊州買紙,川廣販香蠟,着緊一二年也不止。販畢貨,就來家了。此去從河南、陝西、漢州去,回來打水路,從峽江、荊州那條路來,往回七八千里地。」伯爵問:「二舅貴庚多少?」孟銳道:「在下虛度二十六歲。」伯爵道:「虧你年小小的,曉的這許多江湖道路。似俺每虛老了,只在家裡坐着。」須臾,添換上來,杯盤羅列。孟二舅至日西時分,告辭去了。西門慶送了回來,還和伯爵吃了一回。只見買了兩座等庫來,西門慶委付陳經濟裝庫,問月娘尋出李兒兩套錦衣,攪金銀錢紙,裝在庫內。因向伯爵說:「今日是他六七,不念經,替他燒座庫兒。」伯爵道:「好快光陰,嫂子又早沒了個半月了。」西門慶道:「這出月初五日,是他斷七,少不的替他念個經兒。」伯爵道:「這遭哥念佛經罷了。」西門慶道:「大房下說,他在時因生小兒,許了些血盆經懺;許下家中走的兩個女僧做首座,請幾眾尼僧,替他禮拜幾卷懺兒。」說畢,伯爵見天晚,說道:「我去罷,只怕你與嫂子燒布。」又深深打恭,說:「蒙哥厚情,死生難忘!」西門慶道:「難忘不難忘,我兒,你休推夢裡睡哩。你眾娘到滿月那日,買禮多要去哩。伯爵道:「又買禮做甚?我就頭著地,好歹請眾嫂子到寒家光降光降。」西門慶道:「到那日,好歹把春花兒那奴才收拾起來,牽了來我瞧瞧。」伯爵道:「你春姨他說來,有了兒子,不用着你了。」西門慶道:「別要慌,我見了那奴才,和他答話。」伯爵佯長笑的去了。西門慶令小廝收了家火,走到李瓶兒房裡。陳經濟和玳安,已把庫裝封停當。那日玉皇廟,永福寺、報恩寺多送疏。道家是寶肅昭成真君像,佛家是冥府第六殿變成大王。門外花大舅家,送了一盒擔食,十分冥紙。吳大舅子家也是如此。西門慶看着迎春擺設羹飯完備,下出匾食來,點上香燭,使綉春請了後邊吳月娘眾人來。西門慶與李瓶兒燒了布,擡出庫去,教經濟看着大門首焚化,不在話下。正是:

「芳魂料不隨灰死,  再結來生未了緣。」

畢竟未知後來如何,且聽下回分解:
