%# -*- coding: utf-8 -*-
%!TEX encoding = UTF-8 Unicode
%!TEX TS-program = xelatex
% vim:ts=4:sw=4
%
% 以上设定默认使用 XeLaTex 编译,并指定 Unicode 编码,供 TeXShop 自动识别

%第五十三回 
\chapter{吳月娘承歡求子媳\KG 李瓶兒酬愿保兒童}

「人生有子萬事足,  身後無兒總是空,

產下龍媒須保護,  欲求麟種貴陰功;

禱神且急酬心愿,  服藥還教暖子宮;

父母好將人事盡,  其間造化聽蒼穹。」

話說吳月娘與李嬌兒、桂姐、孟玉樓、李瓶兒、孫雪娥、潘金蓮、大姐混了一場,身子也有些不耐煩,徑進房去睡了。醒時約有更次,又差小玉去問李瓶兒道:「官哥沒怪哭麼?叫奶子抱得緊緊的,拍他睡好,不要又去惹他哭了。」奶子也就在炕上吃了飯,沒待下來,又丟放他在那裡。李瓶兒道:「你與我謝聲大娘,道自進了房裡,只顧呱呱的哭,打冷戰不住。而今纔住得哭,磕伏在奶子身上睡了,額上有些熱剩剩的,奶子動也不得動,停會兒我也待換他起來吃夜飯淨手哩。」那小玉進房,回覆了月娘。月娘道:「他們也不十分當緊的,那裡一個小娃兒丟放在芭蕉腳下,徑倒別的走開,吃貓諕了。如今纔是愁神哭鬼的,定要弄壞了,纔住手!」那時說了幾句,也就洗了臉,睡了一宿。到次早起來,別無他話,只差小玉問官哥下半夜有睡否?還說大娘吃了粥,就待過來看官哥了。李瓶兒對迎春道:「大娘就待過來,你快要拿臉水來,我洗了臉。」那迎春飛搶的拿臉水進來,李瓶兒急攘攘的梳了頭,交迎春不迭的燒起茶來,點些安息香在房裡。三不知小玉來報,說:「大娘進房來了。」慌得李瓶兒撲起的也似接了,月娘就到奶子床前,摸著官哥道:「不長俊的小油嘴,常時把做親娘的,平白地提在水缸裡。」這官哥兒呱的聲怪哭起來,月娘連忙引鬥了一番,就住了。月娘對如意道:「我又不得養,我家的人種,便是這點點兒。休得輕覷著他,著緊用心纔好!」奶子如意兒道:「這不消大娘分付。」月娘就待出房,李瓶兒道:「大娘來,泡一甌子茶在那裡,請坐坐去。」月娘就坐定了,問道:「六娘,你頭鬢也是亂蓬蓬的。」李瓶兒道:「因這冤家作怪□氣,頭也不得梳。又是大娘來,倉忙的扭一挽兒,胡亂磕上髮髻,不知怎模樣的做笑話!」月娘笑道:「你看是有槽道的麼!自家養的親骨肉,倒也叫他是家。學了我,成日要那冤家,也不能勾哩!」李瓶兒道:「是便這等說,沒有這些鬼病來纏擾他便好。如今不得三兩日安靜,常時一出。前日墳上去,鑼鼓諕了;不幾時,又是剃頭哭得要不的;如今又吃貓諕了。人家都是好養,偏有這東西,是燈草一樣脆的!」說了一場,月娘就走出房來,李瓶兒隨後送出。月娘道:「你莫送我,進去看官哥去罷!」李瓶兒就進了房。月娘走過房裡去,只聽得照壁後邊,賊燒紙的說什麼。月娘便立了聽著,又在板縫裡瞧著,一名是潘金蓮與孟玉樓兩個同靠著欄杆,〈口敝〉了聲氣,絮絮荅荅的講說道:「姐姐好沒正經!自家又沒得養,別人養的兒子,又去漒遭魂的掗相知呵卵脬!我想窮有窮氣,杰有杰氣,奉承他做甚的?他自長成,只認自家的娘,那個認你?」只見迎春走過去,兩個閃的走開了;假做尋貓兒喂飯,到後邊去了。月娘不聽也罷,聽了這般言語,怒生心上,恨落牙根,那時即欲叫破罵他。又是爭氣不穿的事,反傷體面,只得忍耐了。一徑進房,睡在床上,又恐丫鬟每覺著了,不好放聲哭得,只管自理自怨,短嘆長吁。真個在家不敢高聲哭,只恐猿聞也斷腸。那時日當正午,還不起身。小玉立在床邊,請大娘起來吃飯,月娘道:「我身子不好,還不吃飯。你掩上房門,且燒些茶來吃。」小玉捧了茶進房去,月娘纔起來,悶悶的坐在房裡,說道:「我沒有兒子,受人這樣懊惱。我求天拜地,也要求一個來,羞那些賊淫婦的〈毛皮〉臉!」于是走到後房,文櫃梳匣內,取出王妓子整治的頭胎衣胞來,又取出薛姑子送的藥,看小小封筒上面,刻著「種子靈丹」四字,有詩八句:

「姮娥喜竊月中砂,  笑取斑龍頂上芽,

漢帝桃花勅特降,  梁王竹葉誥曾加;

須臾餌驗人堪羨,  衰老還童更可誇,

莫作雪花風月趣,  烏鬚種子在些些」

後有讚曰:

「紅炎閃爍,宛如碾就之珊瑚;香氣沉濃,彷彿初燃之檀麝。噙之口內,則甜津湧起于牙根;置之掌中,則熱氣貫通于臍下。直可還精補液,不必他求玉杵霜;且能轉女為男,何須別覓神樓散。不與爐邊雞犬,偏助被底鴛鴦。乘興服之,遂入蒼龍之夢;按時而動,預徵飛燕之祥。求子者一投即效,修真者百日可 。」

後又曰:

「服此藥後,凡諸腦損物,諸血敗血,皆宜忌之;又忌蘿蔔蔥白 。其交接單日為男,雙日為女,惟心所願。服此一年,可得長生矣。」

月娘看畢,心中漸漸的歡喜,見封袋封得緊,用纖纖細指,緩緩輕挑,解包開看。只見烏金紙三四層,裹著一丸藥,外有飛金硃砂,粧點得十分好看。月娘放在手中,果然臍下熱起來,放在鼻邊,果然津津的滿口香唾。月娘笑道:「這薛姑子果有道行,不知那裡去尋這樣妙藥靈丹!莫不是我合當得喜,遇得這個好藥,也未可知。」把藥來看玩了一番,又恐怕藥氣出了,連忙把麵漿來,依舊封得緊緊的,原進後房,鎖在梳匣內了。走到步廊下,對天長嘆道:「若吳氏明日壬子日,服了薛姑子藥,便得種子,承繼西門香火,不使我做無祀的鬼,感謝皇天不盡了!」那時日已近晚,月娘纔吃了飯。話不再煩。西門慶到劉太監庄上,投了帖兒,那些役人報了黃主事、安主事,一齊迎住。都是冠帶,好不齊整!敘了揖坐下。那黃主事便開言道:「前日仰慕大名,敢爾輕造;不想就擾執事,太過費了!」西門慶道:「多慢為罪!」安主事道:「前日要赴敝同年胡大尹召,就告別了。主人情重,至今心領。今日都要盡歡達日纔是。」西門慶道:「多感盛情!」門子低報道:「酒席已完備了。」就邀進捲棚,解去冠帶,安席,送西門慶首坐,西門慶假意推辭,畢竟坐了首席,歌童上來,唱一隻曲兒,名喚錦登梅:

「紅馥馥的臉襯霞,黑髭髭的鬢堆鴉,料應他必是個中人,打扮的堪描畫。顫巍巍的插著翠花,寬綽綽的穿著輕紗,兀的不風韻煞人也!嗏,是誰家把我不住了偷晴兒抹!」

西門慶讚好!安主事、黃主事就送酒與西門慶,西門慶答送過了,優兒又展開檀板,唱一隻曲,名喚降黃龍袬:

「麟鴻無便,錦箋慵寫。腕鬆金,肌削玉,羅衣寬徹。淚痕淹破,胭脂雙頰,寶鑑愁臨,翠鈿羞貼。 等閑孤負,好天良夜。玉爐中,銀臺上,香消燭滅。鳳幃冷落,鴛衾虛設;玉筍頻搓,繡鞋重攧。」

那時吃到酒後,傳盃換盞,都不絮煩。卻說那潘金蓮在家,因昨日雪洞里不曾與陳經濟得手,此時趁西門慶在劉太監庄上與黃主事、安主事吃酒,吳月娘又在房中不出來,奔進奔出的,好像熬盤上蟻子一般。那陳經濟在雪洞裡跑出來,睡在店中,那話兒硬了一夜,此時西門慶不在家中,只管與金蓮兩個眉來眼去。直至黃昏時候,各房將待掌燈,金蓮躡足潛蹤,踮到捲棚後面。經濟三不知走來,隱隱的見是金蓮,遂緊緊的抱著了。把臉子挨在金蓮臉上,兩個親了十來個嘴。經濟道:「我的親親,昨夜孟三兒那冤家打開了我每,害得咱硬幫幫撐起了一宿。今早見你妖妖嬈嬈搖颭的走來,教我渾身兒酥麻了。」金蓮道:「你這少死的賊短命,沒些槽道的!把小丈母便揪住了親嘴,不怕人來聽見麼!」經濟道:「若見火光來,便走過了。」經濟口裡只故叫親親,下面單裙子內,卻似火燒的一條硬鐵,隔了衣服只顧挺將進來。那金蓮也不由人,把身子一聳,那話兒都隔了衣服,熱烘烘對著了。金蓮政忍不過,用手掀開經濟裙子,用力捏著陽物。經濟慌不迭的,替金蓮扯下褲腰來,劃的一聲,卻扯下一個裙襇兒。金蓮笑罵道:「蠢賊奴!還不曾偷慣食的,恁小著膽!就慌不迭倒把裙襇兒扯吊了。」就自家扯下褲腰,剛露出牝口,一腿翹在欄杆上,就把經濟陽物塞進牝口。原來金蓮鬼混了半晌,已是濕答答的,被經濟用力一挺,便撲的進去了。經濟道:「我的親親,只是立了不盡根,怎麼處?」金蓮道:「胡亂抽送抽送,且再擺佈。」經濟剛待抽送,忽聽得外面狗都噑噑的叫起來,卻認是西門慶吃酒回來了,兩個慌得一滾煙走開了。卻是書童、玳安兩個拿著冠帶金扇,進來亂嚷道:「今日走死人也!」月娘差小玉出來看時,只見兩個小廝,都是醉模糊的。小玉問道:「爺怎的不歸?」玳安道:「方纔我每恐怕追馬不及,問了爺,先走回來。他的馬快,也只在後邊來了。」小玉進去回覆了。不一時,西門慶已到門外,下了馬。本待到金蓮那裡睡,不想醉了,錯走入月娘房裡來。月娘暗想:「明日二十三日,乃是壬子日。今晚若留他,反挫明日大事。又是月經左來日子,也至明日潔淨。」對西門慶道:「你今晚醉昏昏的,不要在這裡鬼混。我老人家月經還未淨,不如在別房去睡了,明日來罷。」把西門慶帶笑的推出來,走到金蓮那裡去了。捧著金蓮的臉道:「這個是小淫婦了!方纔待走進來,不想有了幾杯酒,三不知走入大娘房裡去!」金蓮道:「精油嘴的東西,你便說明日要在姐姐房裡睡了。硶說嘴的,在真人前赤巴巴弔謊!難道我便信了你?」西門慶道:「怪油嘴,專要歪斯纏人!真正是這樣的。著甚緊弔著謊來?」金蓮道:「且說姐姐怎地不留你住?」西門慶道:「不知道。他只管道我醉了,推了出來,說明晚來罷!我便急急的來了。」金蓮政待澡牝,西門慶把手來待摸他。金蓮雙手掩住,罵道:「短命的,且沒要動彈!我有些不耐煩在這裡。」西門慶一手抱住,一手插入腰下,竟摸著道:「怪行貨子,怎的夜夜乾卜卜的,今晚裡面有些濕答答的。莫不想著漢子,騷水發哩?」原來金蓮想著經濟,還不曾澡牝。被西門慶無心中打著心事,一時臉通紅了,把言語支語支吾,半笑半罵,就澡牝洗臉,兩個宿了一夜不題。卻表吳月娘次早起來,卻正當壬子日了,便思想:「薛姑子臨別時,千叮嚀萬囑付,叫我到壬子日吃了這藥,管情就有喜事。今日正當壬子,政該服藥了。」又喜昨夜天然湊巧,西門慶飲醉回家,撞入房來,回到今夜。因此月娘心上,暗自喜歡。清早起來。即便沐浴梳粧完了,就拜了佛,念一遍白衣觀音經。求子的最是要念他,所以月娘念他;也是王姑子教他念的。那日壬子日,又是個緊要的日子。所以清早閉了房門,燒香點燭,先誦過了,就到後房,開取藥來,叫小玉燉起酒來。也不用粥,先吃了些乾糕餅食之類,就雙手捧藥,對天禱告。先把薛姑子一丸藥,用酒化開,異香觸鼻,做三兩口服完了。後見王姑子製就頭胎衣胞,雖則是做成末子,然終覺有些注疑,有些焦刺刺的氣子,難吃下口。月娘自忖道:「不吃他,不得見效;待吃他,又只管生疑。也罷!事到其間,做不得主了,只得勉強吃下去罷。」先將符藥一把掩在口內,急把酒來大呷半碗,幾乎嘔將出來,眼都忍紅了。又連忙把酒過下去,喉舌間只覺有些膩格格的。又吃了幾口酒,就討溫茶來漱淨口,睡向床上去了。西門慶政走過房來,見門關著,叫小玉開了。問道:「怎麼悄悄的關上房門?莫不道我昨夜去了,大娘有些二十四麼?」小玉道:「我那裡曉得來?」西門慶走進房來,叫了幾聲。月娘吃了早酒,向裡床睡著去,那裡答應他。西門慶向小玉道:「賊奴才,現今叫大娘只是不應,怎的不是氣我?」遂沒些趣向,走出房去。只見書童進來,說道:「應二爹在外邊了。」西門慶走出來,應伯爵道:「哥,前日到劉太監庄上赴黃安二公酒席,得盡歡麼?直飲到幾時分纔散了?」西門慶道:「承兩公十分相愛,他前的下顧,因欲赴胡大尹酒席,倒坐不多時。我到他那裡,卻情投意合,倒也被他多留住了,灌了好幾杯酒。直到更次,歸路又遠,醉又醉了,不知怎的了。」應伯爵道:「別處人,倒也好情分,還該送些下程與他。」西門慶道:「說的有理。」就叫書童寫起兩個紅禮帖來,分付裡面,辦一樣兩副盛禮。枝圓桃棗 ,鵝鴨羊腿鮮魚,兩罈南酒 。又寫二個謝宴名帖,就叫書童來分付了,差他送去,書童答應去了。應伯爵就挨在西門慶身邊來坐近了:「哥前日說的,曾記得麼?」西門慶道:「記甚的來?」應伯爵道:「想是忙的都忘記了。便是前日同謝子純在這裡吃酒,臨別時說的。」西門慶呆登想了一會,說道:「莫不就是李三、黃四的事麼!」應伯爵笑道:「這叫做簷頭雨滴從高下,一點也不差!」西門慶做攢眉道:「教我那裡有銀子?你眼見我前日支鹽的事,沒有銀子,與喬親家挪得五百兩湊用。那裡有許多銀子放出去?」應伯爵道:「左右生利息的,隨分箱子角頭,尋些湊與他罷。哥說門外徐四家的,昨日先有二百五十兩來了,這一半就易處了。」西門慶道:「是便是,那裡去湊?不如且回他,等討徐家銀子,一總與他罷。」應伯爵正色道:「哥,君子一言,快馬一鞭。人而無信,不知其可也。哥前日不要許我便好,我又與他每說了,千真萬真,道今日有的了,怎好去回他?他們極服你做人慷慨,直甚麼事,反被這些經紀人背地裡不服你!」西門慶道:「應二爹如此說,便與他罷。」自己走進,收拾了二百三十兩銀子。又與玉簫討昨日收徐家二百五十兩頭,一總彈准四百八十兩。走出來對應伯爵道:「銀子只湊四百八十兩,還少二十兩。有些段疋作數,可使得麼?」伯爵道:「這個卻難,他就要現銀去幹香的事。你好的段疋,也都沒放,你剩這些粉緞,他又幹不得事;不如湊現物與他,省了小人腳步。」西門慶道:「也罷,也罷!」又走進來,稱了廿兩成色銀子,叫玳安通共掇出來。那李三、黃四卻在間壁人家坐久,只待伯爵打了照面,就走進來。謝希大適值進來,李三、黃四敘揖畢了,就見西門慶。行禮畢,就道:「前日蒙大恩,因銀子不得關出,所以遲遲。今因東平府又派下二萬香來,敢再挪五百兩,暫濟燃眉之急。如今關出這批銀子,一分也不動,都盡這邊來,一齊算利奉還。」西門慶便喚玳安,舖子裡取天平,請了陳姐夫,先把他討的徐家廿五包彈准了。後把自家二百五十兩彈明了,付與黃四、李三,兩人拜謝不已,就告別了。西門慶欲留應伯爵、謝希大再坐一回,那兩個那有心想坐,只待出去與李三、黃四分中人錢了。假意說有別的事,急急的別去了。那玳安、琴童都擁住了伯爵,討些使用,買果子吃。應伯爵搖手道:「沒有,沒有。這是我認得的,不帶得來送你,這些狗弟子的孩兒!」徑自去了。只見書童走了進來,把黃主事、安主事兩個謝帖回話,說:「兩個爺說:道:『不該受禮,恐拂盛意,只得收了。多去致意你爺。』」力錢二封,西門慶就賞與他。又稱出些,把僱來的挑盤人打發了。天色已是掌燈時分,西門慶走進月娘房裡坐定。月娘道:「小玉說你曾進房來叫我,我睡著了,不得知你叫。」西門慶道:「卻又來,我早認你有些不快我哩。」月娘道:「那裡說起不快你來?」便叫小玉泡茶,討夜飯來吃了。西門慶飲了幾杯,身子連日吃了些酒,只待要睡。因幾時不在月娘房裡來,又待奉承他。也把胡僧的膏子藥來用了些,脹得陽物來鐵杵一般。月娘見了,道:「那胡僧這樣沒槽道的,諕人的弄出這樣把戲來!」心中暗忖道:「他有胡僧的法術,我有姑子的仙丹,想必有些好消息也。」遂都上床去,暢美的睡了一夜。次日起身,都至日午時候。那潘金蓮又是顛唇簸嘴,與孟玉樓道:「姐姐前日教我看幾時是壬子日,莫不是揀昨日與漢子睡的,為何恁的湊巧?」玉樓笑道:「那有這事?」正說話間,西門慶走來。金蓮一把扯住西門慶道:「那裡人家睡得這般早,起得恁的晏;日頭也沉沉的待落了,還走往那裡去?」西門慶被他鬼混了場,那話兒又硬起來。徑撇了玉樓,玉樓自進房去。西門慶按金蓮在床口上,就戲做一處,春梅就討飯來,金蓮同吃了不題。卻說那月娘自從聽見金蓮背地講他愛官哥,兩日不到官哥房裡去看。只見李瓶兒走進房來,告訴道:「孩子日夜啼哭,只管打冷戰不住,卻怎麼處?」月娘道:「你做一個擺佈,與他弄好了便好。把些香愿也許許,或是許了賽神,一定減可些。」李瓶兒道:「前日身子發熱,我許拜謝城隍土地,如今也待完了心愿。」月娘道:「是便是,你的心愿也還該再請劉婆來商議商議,看他怎地說。」李瓶兒政待走出來,月娘道:「你道我昨日成日的不得看孩子,著甚緣故不得進來?只因前日我來看了孩子,走過捲棚照壁邊,只聽得潘金蓮在那裡和孟三兒說我自家沒得養,倒去奉承別人。扯淡得沒要緊!我氣了半日的,飯也吃不下。」李瓶兒道:「這樣怪行貨,歪刺骨!可是有槽道的?多承大娘好意思,著他甚的?也在那裡搗鬼!」月娘道:「你只記在心,防了他,也沒則聲。」李瓶兒道:「便是這等。前日迎春說,大娘出房後邊,迎春出來,見他與三姐立在那裡說話。見了迎春,就尋貓去了。」政說話間,只見迎春氣吼吼的走進來。說道:「娘快來!官哥不知怎麼樣,兩隻眼不住反看起來,口裡捲些白沫出來!」李瓶兒諕得頓口無言,攢眉欲淚。一面差小玉報西門慶,一面急急歸到房裡。見奶子如意兒,都失色了。剛看時,西門慶也走進房來,見了官哥放死放活,也吃了一驚。就道:「不好了,不好了!怎麼處?婦人平日不保護他好,到這田地,就來叫我。如今怎好!」指如意兒道:「奶子不看好他,以致今日!若萬一差池起來,就搗爛你做肉泥,也不當稀罕!」那如意兒慌得口也不敢開,兩淚齊下。李瓶兒只管看了暗哭。西門慶道:「哭也沒用,不如請施灼龜來,與他灼一個龜板。不知他有恁禍福紙脉,與他完一完再處。」就問書童討單名帖,飛請施灼龜來坐下。先是陳經濟陪了吃茶,琴童、玳安點燭燒香。舀淨水,擺桌子。西門慶出門相見了。就拿龜板對天禱告作揖,進入堂中,放龜板在桌上。那施灼龜雙手接著放上龜藥,點上了火,又吃一甌茶。西門慶正坐時,只聽一聲響。施灼龜看了,停一會不開口。西門慶問道:「吉凶如何?」施灼龜問:「甚事?」西門慶道:「小兒病症,大象怎的?有紙脉也沒有?」施灼龜道:「大象目下沒甚事。只怕後來反覆牽延,不得脫然全愈。父母占子孫,子孫爻不宜晦了。又看朱雀爻大動,主獻紅衣神道城隍等類,要殺豬羊去祭他。再領三碗羹飯,一男傷,一女傷,草船送到南方去。」西門慶就送一錢銀子謝他。施灼龜極會諂媚,就千恩萬謝,蝦也似打躬去了。西門慶走到李瓶兒房裡,說道:「方纔灼龜的說,大象牽延,還防反覆。只是目下急急的該獻城隍老太。」李瓶兒道:「我前日原許的,只不曾獻得,孩子只管駁雜。」西門慶道:「有這等事!」即喚玳安:「叫慣行燒紙的錢痰火來。」玳安即便出門,西門慶和李瓶兒擁著官哥道:「孩子,我與你賽神了,你好了些,謝天謝地!」說也奇怪,那時孩子就放下眼,磕伏著有睡起來了。李瓶兒對西門慶道:「好不作怪麼,一許了獻神道,就減可了大半!」西門慶心上一塊石頭,纔得放了下來。月娘聞得了,也不勝喜歡。又差琴童去請劉婆子的來,劉婆子急波波的,一步高一步低走來。西門慶不信婆子的,只為愛著官哥,也只得信了。那劉婆子一徑走到廚房下去摸竈門,迎春笑道:「這老媽敢汗邪了!官哥倒不看,走到廚下去摸竈門則甚的?」劉婆道:「小奴才你曉得甚的,別要吊嘴說!我老人家一年也大你三百六十日哩。路上走來,又怕有些邪氣,故來灶門前走走。」迎春把他做了個臉。聽李瓶兒叫,就同劉婆進房來,劉婆磕了頭。西門慶要分付玳安稱銀子買東西,殺豬羊獻神,走出房來。劉婆便問道:「官哥好了麼?」李瓶兒道:「便是凶得緊,請你來商議。」劉婆道:「前日是我說了,獻了五道將軍就好了。如今看他氣色,還該謝謝三界土便好。」李瓶兒道:「方纔施灼龜說,該獻城隍老太。」劉婆道:「他慣一不著的,曉得甚麼來!這個原是驚,不如我收驚倒好。」李瓶兒道:「怎地收驚?」劉婆道:「迎春姐,你去取些米,舀一碗水來,我做你看。迎春取了米水來。劉婆把一隻高腳瓦鐘,放米在裡面,滿滿的。袖中摸中舊綠絹頭來,包了這鍾米,把手捏了,向官哥頭面上下手足,虛空運來運去的戰。官哥正睡著,奶子道:「別要驚覺了他。」劉婆搖手低言道:「我曉得,我曉得。運了一陣,口裡唧噥噥的念,不知是麼。中間一兩句嚮些,李瓶兒聽得是念「天驚地驚,人驚鬼驚,貓驚狗驚。」李瓶兒道:「孩子政是貓驚了起的。」劉婆念畢,把絹兒抖開了,放鍾子在桌上。看了一回,就從米搖實下的去處,撮兩粒米,設在水碗內,就曉得病在月盡好。也是一個男傷,兩個女傷,領他到東南方上去。只是不該獻城隍,還該謝土纔是。那李瓶兒疑惑了一番,道:「我便再去謝謝土地也不妨。」又叫迎春出來,對西門慶說:「劉婆看水碗說該謝土。左右今夜廟裡去不及了,留好東西,明早志誠些去。」西門慶就叫玳安:「把拜廟裡的東西及豬羊收拾好了,待明早去罷。」再買了謝土東西,炒米繭團,土筆土墨,放生麻雀鰍鱔之類,無物不備,件色整齊。那劉婆在李瓶兒房裡,走進來到月娘房裡坐了,月娘留他吃了夜飯。卻說那錢痰火到來,坐在小廳上,琴童與玳安忙不迭的扶侍他謝土。那錢痰火吃了茶,先討個意旨。西門慶叫書童寫與他,那錢痰火就帶了雷圈板巾,依舊著了法衣,仗劍執水,步罡起來,念淨壇咒。

咒曰:

「洞中玄虛,晃朗太元。八方威神,使我自然。靈寶符命,普告九天。乾羅答那,洞罡太玄。斬妖縛邪,殺鬼萬千。中山神咒,元始玉文。持誦一遍,卻病延年。按行五嶽,八海知聞。魔王柬手,侍衛我軒。兇穢消散,道氣常存。」云云。

「請祭主拈香。」西門慶淨了手,漱了口,著了冠帶,帶了兜膝。孫雪娥、孟玉樓、李嬌兒、桂姐都幫他著衣服,都嘖嘖的讚好。西門慶走出來,拈香拜佛。安童背後,扯了衣服,好不冠冕氣象。錢痰火見主人出來,念得加倍響些。那些婦人便在屏風後,瞧著西門慶,指著錢痰火,都做一團笑倒。西門慶聽見笑得慌,跪在神前又不好發話,只顧把眼睛來打抹。書童就覺著了。把嘴來一挪,那眾婦人便覺住了些。金蓮獨自後邊出來,只見轉一拐兒。驀見了陳經濟,就與他親嘴摸奶,袖裏拏出一把果子與他。又問道:「你可要吃燒酒 ?」經濟道:「多少用些也好。」遂吃金蓮乘眾人忙的時分,扯到屋裏來。叫春梅閉了門,連把幾鍾與他吃了,就說:「出去罷,恐人來,我便死也。」經濟又待親嘴,金蓮道:「硶短命,不怕婢子瞧見!」便戲發訕,打了恁一下,那經濟就慌跳走出來。金蓮就叫春梅先走,引了他出去了。正是:

「雙手劈開生死路,  一身跳出是非門。」

那時金蓮也就走外邊瞧了,不在話下。那西門慶拜了土地,跪了半晌,纔得起來,只做得開啟功德。錢痰火又將次拜懺。西門慶走到屏風後邊,對眾婦人道:「別要嘻嘻的笑,引的我幾次忍不住了。」眾婦人道:「那錢痰火是燒紙的火鬼,又不是道士的,帶了板巾,著了法衣,這赤巴巴沒廉恥的,〈口勃〉嘍嘍的臭涎唾,也不知倒了幾斛出來了!」西門慶道:「敬神如神在,不要是這樣的寡薄嘴,調笑的他苦。」錢痰火又請拜懺。西門慶走到毡單上,錢痰火通陳起頭,就念入懺科文,遂念起志心朝禮來。看他口邊涎唾捲進捲出,一個頭得上得下,好似磕頭蟲一般,笑得那些婦人做了一堆。西門慶那裡趕得他拜來,那錢痰火拜一拜,是一個神君。西門慶拜一拜,他又拜過幾個神君了。于是也顧不得他,只管亂拜。那些婦人,笑得了不的。適值小玉出來,請李桂姐吃夜飯。說道:「大娘在那裡冷清清,和大姐、劉婆、三個坐著講閑話,這裡來這樣熱鬧得很!」嬌兒和桂姐即便走進屋裡,眾人都要進來。獨那潘金蓮,還要看後邊。看見都待進來,只得進來了。吳月娘對大姐道:「有心賽神,也放他志誠些。這些風婆子都擁出去,甚緊要的?有甚活獅子相咬,去看他!」纔說得完,李桂姐進來,陪了月娘、大姐三個吃夜飯不題。卻說那西門慶拜了滿身汗,走進裡面,脫了衣冠靴帶,就走入官哥床前,摸著說道:「我的兒,我與你謝土了。」對李瓶兒道:「好呀!你來摸他額上,就涼了許多,謝天謝天!」李瓶兒笑道:「可霎作怪,一從許了謝土,就也好些。如今熱也可些,眼也不反看了。冷戰也住些了,莫道是劉婆沒有意思?」西門慶道:「明日一發去完了廟裡的事便好了。」李瓶兒道:「只是做爺的吃了勞碌了。你且揩一揩身上,吃夜飯去。」西門慶道:「這裡恐諕了孩子,我別的去吃罷。」走到金蓮那裡來,坐在椅上,說道:「我兩個腰子,落出也似的痛了!」金蓮笑道:「這樣孝心,怎地痛起來?如今叫那個替你拜拜罷。」西門慶道:「有理,有理。」就叫春梅:「喚琴童請陳姐夫替爺拜拜,送了紙馬。」誰想那經濟,在金蓮房裡灌了幾鍾酒出來,恐怕臉紅了。小廝們猜道出來,只得買了些淡酒,在舖子裡又吃了幾杯。量原不濟,一霎地醉了,齁齁的睡著了。琴童那裡叫得起來,一腳箭走來回覆,西門慶道:「睡在那裡,再叫不起。」西門慶便惱將起來,道:「可是個有槽道的?不要說一家的事,就是鄰佑人家,還要看看。怎的就早睡了!」就叫春梅來:「大娘房裡對大姐說,爺拜酸了腰子,請姐夫替拜送紙馬,問怎的再不肯來,只管睡著?」大姐道:「這樣沒長俊的!待我去叫他。」徑走出房來。月娘就叫小玉到舖子裡叫經濟來,經濟揉一揉眼,走到後邊見了大姐道:「你怎的忙不迭的叫命?」大姐道:「叫你替爺拜土送馬去。方纔琴童來叫你不應,又來與我歪斯纏。如今娘叫小玉來叫你,好歹去拜拜罷麼。」遂半推半攙的,擁了經濟到廳上,大姐便進房去了。小玉回覆了月娘,又回覆了西門慶。西門慶分付琴童、玳安等伏侍錢痰火完了事,就睡在金蓮床上不題。卻說那陳經濟走到廳上,只見燈燭輝煌,纔得醒了。掙著眼,見錢痰火政收散花錢,遂與敘揖。痰火就待領羹飯,交琴童掌燈。到李瓶兒房首,迎春接香進去,遞與如意兒,替官哥呵了一呵,就遞出來。錢痰火捏神捏鬼的念出來,到廳上,就待送馬。陳經濟拜了一回,錢痰火就送馬發檄,發了乾卦,說道:「檄向天門,一兩日就好的。縱有反覆,沒甚事。」就放生,燒紙馬,奠酒辭神,禮畢。那痰火口渴肚飢,也待要吃東西了。那玳安收家活進去了,琴童擺下桌子,就是陳經濟陪他散堂。錢痰火千百聲謝去了,經濟也進房去了。李瓶兒又差迎春送果子福物到大姐房裡來,大姐謝了不題。卻說劉婆在月娘房裡謝了出來,剛出大門,只見後邊錢痰火提了燈籠醉醺醺的撞來。劉婆便道:「錢師父,你們的散花錢可該送與我老人家麼?」錢痰火道:「那裡是你本事?」劉婆道:「是我看水碗作成你老頭子。倒不識好歹哩!下次砍落我頭,也不薦你了。」錢痰火再三不肯道:「你精油嘴老淫婦,平白說嘴!你那裡薦的我?我是舊王顧,那裡說起分散花錢?」劉婆指罵道:「餓殺你這賊火鬼纔來求我哩!」兩個鬼混的鬬口一場,去了不題。卻說西門慶次早起來,分付安童跟隨上廟。挑豬羊的挑豬羊,拏冠帶的拏冠帶,徑到廟裡。慌得那些道士一連忙舖單讀疏。西門慶冠帶拜了,求了籤,交道士解說。道士接了籤,送茶畢,即便解說:「籤是中吉。解云:病者即愈,只防反覆,須宜保重些。」西門慶打發香錢歸來了。剛下馬進來,應伯爵正坐在捲棚的下。西門慶道:「請坐,我進去來。」遂走到李瓶兒房,說求籤如此如此,這般這般。徑走到捲棚下,對伯爵道:「前日中人錢盛麼?你可該請我一請。」伯爵笑道:「謝子純也得了些,怎的獨要我請?也罷,買些東西與哥子吃也罷。」西門慶笑道:「那個真要吃你的?試你一試兒。」伯爵便道:「便是你今日豬羊上廟,福物盛得十分的,小弟又在此,怎的不散福?」西門慶道:「也說得有理。」喚琴童去請謝爹來同享。一面分付廚下,整理菜蔬出來,與應二爹吃酒。那應伯爵坐了,只等謝希大到。那得見來?便道:「我們先坐了罷!等不得這樣喬做作的。」西門慶就與應伯爵吃酒。琴童歸來說:「謝爹不在家。」西門慶道:「怎去得恁久?」琴童道:「尋得要不的。」應伯爵遂行口令,都是祈保官哥的意思,西門慶不勝歡喜。應伯爵道:「不住的來擾宅,心上不安的緊。明後日待小弟做個薄主,約諸弟兄陪哥子一杯酒何如?」西門慶笑道:「賺得些中人錢,又來撒漫了。你別要費,我有些豬羊剩的,送與你湊樣數。」伯爵就謝了道:「只覺忒相知了些。」西門慶道:「唱的優兒,都要你身上完備哩。」應伯爵道:「這卻不消說起,只是沒人伏侍,怎的好?」西門慶道:「左右是弟兄,各家人都使得的。我家琴童、玳安將就用用罷。」應伯爵道:「這卻全副了。」吃了一回,遂別去了。正是:

「百年終日醉,  也只三萬六千場。」

畢竟不知如何,且聽下回分解:
