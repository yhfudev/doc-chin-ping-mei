%# -*- coding: utf-8 -*-
% !TeX encoding = UTF-8 Unicode
% !TeX spellcheck = en_US
% !TeX TS-program = xelatex
%~ \XeTeXinputencoding "UTF-8"
% vim:ts=4:sw=4
%
% 以上設定默認使用 XeLaTex 編譯,並指定 Unicode 編碼,供 TeXShop 自動識別

%第一回
\chapter{景陽岡武松打虎\KG 潘金蓮嫌夫賣風月}

\begin{showcontents}{}



詞曰:

「丈夫隻手把吳鈎,
\jiaoYL{丈夫隻手把吳鈎 -- 此词见于元蒋子正《山房随笔》。詞牌爲(眼見媚),作者是宋代的卓田。原詞文字爲:「丈夫隻手把吳鈎,欲斷萬人頭。因何鐵石打成心性,卻爲花柔?君看項籍並劉季,一怒使人愁。只因撞着虞姬戚氏,豪傑都休。」...}
欲斬萬人頭。

如何鐵石打成心性,卻為花柔。

請看項籍并劉季,一似使人愁;

只因撞著虞姬戚氏,豪傑都休。」

此一隻詞兒,單說著情色二字,乃一體一用。故色絢于目,情感于心,情色相生,心目相視。亙古及今,仁人君子,弗合忘之。晉人云:「情之所鍾,正在我輩。」如磁石吸鐵,隔礙潛通。無情之物尚爾,何況為人終日在情色中做活計一節。須而丈夫,隻手把吳鈎。吳鈎,乃古劍也。古有「干將」、「莫邪」、「太阿」、「吳鈎」、「魚腸」、「躅鏤」之名,言丈夫心腸如鐵石,氣概貫虹蜺,不免屈志于女人。題起當時西楚霸王,姓項名籍,單名羽字。因秦始皇無道,南修五嶺,北築長城,東填大海,西建阿房,并吞六國,坑儒焚典,因與漢王劉邦,單名季字,時二人起兵,席捲三秦,滅了秦國,指鴻溝為界,平分天下。因用范增之謀,連敗漢王七十二陣。只因寵著一個婦人,名叫虞姬,有傾城之色,載于軍中,朝夕不離。一旦被韓信所敗,夜走陰陵。為追兵所逼,霸王敗向江東取救,因捨虞姬不得,又聞四面皆楚歌。事發,嘆曰:「力拔山兮氣蓋世,時不利兮騅不逝。騅不逝兮可奈何?虞兮虞兮奈若何!」歌畢,淚下數行,虞姬曰:

「大王莫非以賤妾之故,有費軍中大事?」霸王曰:「不然。吾與汝不忍相捨故耳!況汝這般容色,劉邦乃酒色之君,必見汝而納之。」虞姬泣曰:「妾寧以義死,不以苟生!」遂請王之寶劍,自刎而死。霸王因大慟,尋以自剄。史官有詩嘆曰:

「拔山力盡霸圖隳,  倚劍空歌不逝騅;

明月滿營天似水,  那堪回首別虞姬。」

那漢王劉邦原是泗上亭長,提三尺劍,硭碭山斬白蛇起手。二年亡秦,五年滅楚,掙成天下。只因也是寵著個婦人,名喚戚氏。夫人所生一子,名趙王如意,因被呂后妒害,心甚不安。一日,高祖有疾,乃枕戚夫人腿而臥。夫人哭曰:「陛下萬歲後,妾母子何所托?」帝曰:「不難。吾明日出朝,廢太子而立爾子,意下如何?」戚夫人乃收淚謝恩。呂后聞之,密召張良謀計。良舉薦商山四皓,下來輔佐太子。一日,同太子入朝,高祖見四人鬚鬢交白,衣冠甚偉。各問姓名。一名東圓公,一名綺里季,一名夏黃公,一名角里先生。因大驚曰:「朕昔求聘諸公,如何不至?今日乃從吾兒所遊?」四皓答曰:「太子乃守成之主也。」高祖聞之,愀然不悅。比及四皓出殿,乃召戚夫人指示之曰:「我欲廢太子,況彼四人輔佐,羽翼已成,卒難搖動矣!」戚夫人遂哭泣不止。帝乃作歌以解之:

「鴻鵠高飛兮羽翼,抱龍兮橫蹤四海。橫蹤四海兮,又可奈何?雖有繑繳兮,尚安所施!」

歌訖,後遂不果立趙王矣。高祖崩世,呂后酒酖殺趙王如意,人彘了戚夫人,以除其心中之患。詩人評此二君,評到個去處,說劉、項者,固當世之英雄,不免為二婦人,以屈其志氣。雖然,妻之視妾,名分雖殊,而戚氏之禍,尤慘于虞姬。然則妾婦之道,以事其丈夫,而欲保全首領于牖下,難矣!觀此二君,豈不是「撞著虞姬戚氏,豪傑都休。」有詩為證:

「劉項佳人絕可憐,  英雄無策庇嬋娟;

戚姬葬處君知否?  不及虞姬有墓田。」

說話的,如今只愛說這情色二字做甚?故士矜才則德薄,女衍色則情放。若乃持盈慎滿,則為端士淑女,豈有殺身之禍?今古皆然,貴賤一般。如今這一本書,乃虎中美女後引出一個風情故事來。

一個好色的婦女,因與了破落戶相通,日日追歡,朝朝迷戀。
後不免屍橫刀下命染黃泉,永不得著綺穿羅,再不能施朱付粉。
靜而思之,著甚來由!況這婦人他死有甚事?

貪他的,斷送了堂堂六尺之軀;

愛他的,丟了潑天閧產業。

驚了東平府,大鬧了清河縣,

端的不知誰家婦女?誰的妻小?

後日乞何人占用?死于何人之手?

正是:

「說時華岳山峰歪,  道破黃河水逆流!」

話說宋徽宗皇帝,政和年間,朝中寵信高、楊、童、蔡四個奸臣,以致天下大亂,黎民失業,百姓倒懸;四方盜賊蜂起,罡星下生人間,攪亂大宋花花世界。四處反了四大寇。那四大寇:山東宋江,淮西王慶,河北田虎,江南方臘,皆轟州劫縣,放火殺人,僣稱王號。惟有宋江替天行道,專報不平,殺天下贓官污吏,豪惡刁民。那時山東陽谷縣,有一人姓武,名植,排行大郎。有個嫡親同胞兄弟,名喚武松。其人身長七尺,膀闊三停,自幼有膂力,學得一手好鎗棒。他的哥哥武大,生的身不滿三尺,為人懦弱,又頭腦濁蠢可笑,平日本分,不惹是非。因時遭荒饉,將租房兒賣了,與兄弟分居,搬移在清河縣居住。這武松因酒醉,打了童樞密,單身獨自逃在滄州橫海郡小旋風柴進庄上,他那裡招覽天下英雄豪傑,仗義疎財,人號他做「小孟嘗君」。柴大官人迺是周朝柴世宗嫡派子孫,那裡躲逃。柴進因見武松是一條好漢,收攬在庄上。不想武松就害起瘧疾來,住了一年有餘,因思想哥哥武大,告辭歸家。

在路上行了幾日,來到陽谷縣地方。那時山東界上,有一座景陽崗,山中有一隻弔睛白額虎,食得路絕人稀。官司杖限獵戶,擒捉此虎。崗子路上兩邊都有榜文,可教過往經商,結夥成群,于巳、午、未三個時辰過崗,其餘不許過崗。這武松聽了,呵呵大笑。就在路傍酒店內,吃了幾碗酒,壯著膽。橫拖著防身稍棒,浪浪滄滄,大扠步走上崗來。不半里之地,見一座山神廟門首,貼著一張印信榜文。武松看時,上面寫道:「景陽崗上,有一隻大蟲,近來傷人甚多;見今立限各鄉并獵戶人等,打捕住時,官給賞銀三十兩。如有過往客商人等,可于巳、午、未三個時辰,結夥過崗。其餘時分,及單身客旅,白日不許過崗,恐被傷害性命不便。各宜知悉。」武松喝道:「怕什麼鳥!」且只顧上崗去,看有甚大蟲?武松將棒綰在脅下,一步步上那崗來。回看那日色,漸漸下山,此正是十月間天氣,日短夜長,容易得晚。武松走了一會,酒力發作,遠遠望見亂樹林子,直奔過樹林子,見一塊光撻撻地大青臥牛石,把那棒倚在一邊,放翻身體,卻待要睡,但見青天忽然起一陣狂風。看那風時,但見:

「無形無影透人懷,  四季能吹萬物開;

就地撮將黃葉去,  人山推出白雲來。」

原來雲生從龍,風生從虎。那一陣風過處,只聽得亂樹皆落黃葉,刷刷的響,撲地一聲,跳出一隻弔睛白額斑爛猛虎來,猶如牛來大。武松見了,叫聲「阿呀」時,從青石上翻身下來,便提稍棒在手,閃在青石背後。那大蟲又饑又渴,把兩隻爪在地上跑了一跑,打了個歡翅。將那條尾剪了又剪,半空中猛如一個焦霹靂,滿山滿嶺盡皆振響。這武松被那一驚,把肚中酒都變做冷汗出了。說時遲,那時快。武松見大蟲撲來,只一閃,閃在大蟲背後。原來猛虎項短,回頭看人教難,便把前爪搭在地下,把腰跨一伸,掀將起來;武松只一躲,躲在側邊。大蟲見掀他不著,吼了一聲,把山崗也振動。武松卻又閃過一邊。原來虎傷人,只是一撲,一掀,一剪,三般捉不著時,氣力已自沒了一半。武松見虎沒力,翻身回來,雙手輪起稍棒,盡平生氣力,只一棒,只聽得一聲響,簌簌地將那樹枝帶葉打將下來。原來不曾打著大蟲,正打在樹枝上,磕磕把那條棒折做兩截,只拏一半在手裡。這武松心中,也有幾分慌了;那虎便咆哮性發,剪尾弄風起來,向武松又只一撲,撲將來。武松一跳,卻跳回十步遠。那大蟲撲不著武松,把前爪搭在武松面前,武松將半截棒丟在一邊,乘勢向前,兩隻手撾在大蟲頂花皮,使力只一按,那虎急要掙扎,早沒了氣力。武松儘力撾定那虎,那裡肯放鬆。一面把隻腳望虎面上眼睛裏,只顧亂踢;那虎咆哮,把身底下,扒起兩堆黃泥,做了一個土坑裡。武松按在坑裡,騰出右手,提起拳頭來,只顧狠打,儘平生氣力。不消半歇兒時辰,把那大蟲打死。躺臥著,卻似一個綿布袋,動不得了。有古風一篇,單道景陽崗武松打虎。但見:

「景陽崗頭風正狂,  萬里陰雲埋日光;

焰焰滿川紅日赤,  紛紛遍地草皆黃。

觸目曉霞掛林藪,  侵人冷霧滿穹蒼;

忽聞一聲霹靂響,  山腰飛出獸中王。

昂頭踴躍逞牙爪,  谷裡獐鹿皆奔降,

山中狐兔潛蹤跡,  澗內獐猿驚且慌,

卞莊見後魂魄散,  存孝遇時心膽亡。

清河壯士酒未醒,  忽在崗頭偶相迎;

上下尋人虎飢渴,  撞著猙獰來撲人。

虎來撲人似山倒,  人去迎虎如岩傾;

臂腕落時墜飛砲,  爪牙撾處幾泥坑。

拳頭腳尖如雨點,  淋漓兩手鮮血染;

穢污腥風滿松林,  散亂毛鬚墜山崦。

近看千鈞勢未休,  遠觀八面威風減

身橫野草錦斑消,  緊閉雙睛光不閃。」

當下這隻猛虎,被武松沒頓飯之間,一頓拳腳,打的動不得了。使的這漢子,口裏兒自氣喘不息。武松放了手,來松樹邊尋那打折的稍棒;只怕大蟲不死,向身上又打了十數下,那大蟲氣都沒了。武松尋思:「我就勢把這大蟲拖下崗子去。」就血泊中雙手來捉時,那裡提得動?原來使盡了氣力,手腳都疎軟了。

武松正坐在石上歇息,只聽草坡裡刷剌剌響。武松口中不言,心下驚恐:「天色已黑了,倘或又跳出一個大蟲來,我卻怎生鬬得過他?」剛言未畢,只見坡下鑽出兩隻大蟲來,諕武松大驚道:「阿呀!今番我死也!」只見那兩個大蟲,于面前直立起來。武松定睛看時,卻是個人把虎皮縫做衣裳,頭上帶著虎磕腦。那兩人手裡各拏著一條五股剛叉,見了武松倒頭便拜,說道:「壯士,你是人也?神也?端的吃了總律心,豹子肝,獅子腿,膽倒包了身軀!不然,如何獨自一個,天色漸晚,又沒器械,打死這個傷人大蟲?我們在此觀看多時了,端的壯士高姓大名?」武松道:「我行不更名,坐不改姓。自我便是陽谷縣人氏,姓武名松,排行第二。」因問:「你兩個是甚麼人?」那兩個道:「不瞞壯士說,我們是本處打獵戶。因為崗前這隻虎,夜夜出來,傷人極多;只我們獵戶,也折了七八個,過路客人,不計其數。本縣知縣相公,著落我們眾獵戶,限日捕捉,得獲時,賞銀三十兩;不獲時,定限吃拷。叵耐這業畜勢大,難近得他,誰敢向前?我們只和數十鄉夫在此,遠遠地安下窩弓、藥箭等他。正在這裡埋伏,卻見你大刺刺從崗子上走來,三拳兩腳,和大蟲敵鬬,把大蟲登時打死了。未知壯士身上有多少力?俺眾人把大蟲綣了,請壯士下崗,往本縣去見知縣相公討賞去來。」

于是眾鄉夫獵戶,約湊有七、八十人,先把死大蟲抬在前面,將一個兜轎抬了武松,逕投本處一個土戶家。那戶里正,都在庄前迎接,把這大蟲扛在草庭上。卻有本縣里老,都來相探,問了武松姓名,因把打虎一節說了一遍。眾人道:「真乃英雄好漢!」那眾獵戶先把野味將來與武松把盞,吃得大醉。打掃客房,武松歇息。

到天明,里老先去縣裡報知,一面合具虎床,安排花紅軟轎,迎送武松到縣衙前。清河縣知縣使人來接到縣內廳上。那滿縣人民聽得說,一個壯士打死了景陽崗上大蟲,迎賀將來,盡皆出來觀看,哄動了那個縣治。武松到廳上下了轎,扛著大蟲在廳前。知縣看了武松這般模樣,心中自忖道:「不恁地,怎打得這個猛虎?」便喚武松上廳來。參見畢,將打虎首尾,訴說了一遍,兩邊官吏,都驚呆了。知縣就廳上賜了幾盃酒,將庫中眾土戶出納的賞錢三十兩,就賜與武松。武松稟道:「小人托賴相公的福蔭,偶然僥倖,打死了這個大蟲,非小人之能。如何敢受這三十兩賞賜?給發與眾獵戶,因這畜生,受了相公許多責罰。何不就把這賞給散與眾人去?也相公恩沾,小人義氣。」知縣道:「既是如此,任從壯士處分。」武松就把這三十兩賞錢,在廳上俵散與眾獵戶去了。

知縣見他仁德忠厚,又是一條好漢,有心要抬舉他。便道:「雖是陽谷縣的人民,與我這清河縣只在咫尺。我今日就參你在我這縣裡,做個巡捕的都頭。專一河東水西,擒拏盜賊,你意下如何?」武松跪謝道:「若蒙恩相抬舉,小人終身受賜。」知縣隨即喚押司去了文案,當日便參武松做了巡捕都頭。眾里正大戶,都來與武松作賀,慶喜連連誇官,吃了三五日酒。正要陽谷縣抓尋哥哥,不料又在清河縣做了都頭。一日在街上閒遊,喜不自勝。傳得東平一府兩縣,皆知武松之名。有詩為證:

「壯士英雄藝略芳,  挺身直上景陽崗;

醉來打死山中虎,  自此聲名播四方!」

按下武松,單表武大自從與兄弟分居之後,因時遭荒饉,搬移在清河縣紫石街賃房居住。人見他為人懦弱,模樣猥衰,起了他個渾名,叫做三寸丁,谷樹皮。俗語言其身上粗躁,頭臉窄狹故也。以此人見他這般軟弱樸實,多欺負他。武太並無生氣,常時迴避便了。看官聽說:世上惟有人心最歹,軟的又欺,惡的又怕;太剛則拆,太柔則廢。古人有幾句格言,說的好:

「柔軟立身之本,剛強惹禍之胎;

無爭無競是賢才,虧我些兒何礙?

青史幾場春夢,紅塵多少奇才,

不須計較巧安排,守分而今見在。」

且說武大終日挑擔子出去街上,賣炊餅 度日,不幸把渾家故了,丟下個女孩兒,年方十二歲,名喚迎兒。爺兒兩個過活,那消半年光景,又消拆了資本,移在大街坊,張大戶家臨街房居住,依舊做買賣。張宅家下人,見他本分,常看顧他,照顧他炊餅;閑時在他舖中坐,武大無不奉承。

因此張宅家下人個個都歡喜,在大戶面時,一力與他說方便。因此大戶連房錢也不問武大要。這大戶家有萬貫家財,百間房屋,年約六旬之上,身邊寸男尺女皆無。媽媽余氏,主家嚴勵,房中並無清秀使女。一日,大戶拍胸,歎了一口氣。媽媽問道:「你田產豐盛,資財充足,閑中何故歎氣?」大戶道:「我許大年紀,又無兒女,雖有家財,終何大用?」媽媽道:「既然如此說,我教媒人替你買兩個使女,早晚習學彈唱,服侍你便了。」大戶心中大喜,謝了媽媽。過了幾時,媽媽果然教媒人來,與大戶買了兩個使女,一個叫做潘金蓮,一個喚做白玉蓮。這潘金蓮卻是南門外潘裁的女兒,排行六姐。因他自幼生得有些顏色,纏得一雙好小腳兒,因此小名金蓮。父親死了,做娘的因度日不過,從九歲賣在王招宣府裡,習學彈唱,就會描眉畫眼,傅粉施朱,梳一個纏髻兒,著一件扣身衫子,做張做勢,喬模喬樣。況他本性機變伶俐,不過十五,就會描鸞刺綉,品竹彈絲,又會一手琵琶。後王招宣死了,潘媽媽爭將出來,三十兩銀子,轉賣與張大戶家,與玉蓮同時進門。大戶家習學彈唱,金蓮學琵琶,玉蓮學箏。玉蓮亦年方二八,乃是樂戶人家女子,生得白淨,小字玉蓮,這兩個同房歇臥。主家婆余氏,初是甚是抬舉二人,不曾上鍋排備洒掃,與他金銀首飾,粧束身子。後日不料白玉蓮死了,止落下金蓮一人,長成一十八歲,出落的臉襯桃花,眉灣新月,尤細尤灣;張大戶每要收他,只怕主家婆利害,不得手。一日,主家婆鄰家赴席不在,大戶暗把金蓮喚至房中,遂收用了。正是:

「美玉無瑕,一朝損壞;  珍珠何日,再得完全?」

大戶自從收用金蓮之後,不覺身上添了四五件病症,端的那五件:

第一、腰便添疼,第二、眼便添淚,第三、耳便添聾,第四、鼻便添涕,第五、尿便添滴。還有一樁兒不可說。白日間只是打盹,到晚來噴嚏也無數。後主家婆頗知其事,與大戶嚷罵了數日,將金蓮甚是苦打。大戶知不容此女,卻賭氣倒陪房奩,要尋嫁得一個相應的人家。大戶家下人,都說:「武大忠厚,見無妻小,又住著宅內房兒,堪可與他。」這大戶早晚還要看覷此女,因此不要武大一文錢,白白的嫁與他為妻。這武大自從娶的金蓮來家,大戶甚是看顧他。若武大沒本錢做炊餅,大戶私與銀伍兩,與他做本錢。武大若挑擔兒出去,大戶候無人,便踅入房中,與金蓮廝會;武大雖一時撞見,亦不敢聲言。朝來暮往,如此也有幾時。忽一日,大戶得患陰寒病症,嗚呼哀哉死了。主家婆察知其事,怒令家童將金蓮、武大即時趕出,不容在房子裡住。武大不覺又尋紫石街西王皇親房子,賃內外兩間居住,依舊賣炊餅。原來金蓮自從嫁武大,見他一味老實,人物猥衰,甚是憎嫌,常與他合氣。報怨大戶:「普天世界斷生了男子,何故將奴嫁與這樣個貨?每日牽著不走,打著倒腿的,只是一味〈口床〉酒。著緊處,都是錐扎也不動。奴端的那世裡悔氣,卻嫁了他?是好苦也!」常無人處彈個山坡羊為證:

「想當初,姻緣錯配,奴把他當男兒漢看覷。不是奴自己誇獎,他烏鴉怎配鸞凰對?奴真金子埋在土裡,他是塊高號銅,怎與俺金色比?他本是塊頑石,有甚福抱著我羊脂玉體?好似糞土上長出靈芝。奈何隨他怎樣,倒底奴心不美!聽知,奴是塊金磚,怎比泥土基?」

看官聽說:但凡世上婦女,若自己有些顏色,所稟伶俐,配個好男子便罷了!若是武大這般,雖好殺也未免有幾分憎嫌。自古佳人才子,相湊著的少,買金偏撞不著賣金的。武大每日自挑炊餅擔兒出去賣,到晚方歸。婦人在家,別無事幹,一日三餐吃了飯,打扮光鮮,只在門前簾兒下站著。常把眉目嘲人,雙睛傳意。左右街坊,有幾個奸詐浮浪子弟,睃見了武大這個老婆,打扮油樣,沾風惹草。被這干人在街上撒謎語,往來嘲戲。唱叫:「這一塊好羊肉,如何落在狗口裡?」人人自知武大是個懦弱之人,卻不知他娶得這個婆娘在屋裡,風流伶俐,諸般都好。為頭的一件,好偷漢子。有詩為證:

「金蓮容貌更堪題,  笑蹙春山八字眉;

若遇風流清子弟,  等閑雲雨便偷期。」

這婦人每日打發武大出門,只在簾子下磕瓜子兒。一徑把那一對小金蓮做露出來,勾引的這夥人,日逐在門前彈胡博詞扠兒難。口裡油似滑言語,無般不說出來。因此武大在紫石街住不牢,又要往別處搬移,與老婆商議。婦人道:「賊混沌,不曉事的!你賃人家房住,淺房淺屋,可知有小人囉躁!不如湊幾兩銀子,看相應的,典上他兩間住,卻也氣概些,免受人欺負。你是個男子漢,倒擺布不開,常交老娘受氣!」武大道:「我那裡有錢典房?」婦人道:「呸!濁才料!把奴的釵梳湊辦了去,有何難處?過後有了,再治不遲。」武大聽了老婆這般說,當下湊了十數兩銀子,典得縣門前樓上下兩層,四間房屋居住。第二層是樓,兩個小小院落,甚是乾淨。武大自從搬到縣西街上來,照舊賣炊餅。

一日,街上走過,見數隊纓鎗,鑼鼓喧天,花紅軟轎,簇擁著一個人,卻是他嫡親兄弟武松。因在景陽崗打死了大蟲,知縣相公抬舉他,新陞做了巡捕都頭。街上里老人等作賀他,送他下處去。卻被武大撞見,一手扯住,叫道:「兄弟,你今日做了都頭,怎不看顧我?」武松回頭,見是哥哥。二人相合。兄弟大喜,一面邀請家中,讓至樓上坐。房裡喚出金蓮來,與武松相見。因說道:「前日景陽崗打死了大蟲的,便是你小叔,今新充了都頭,是我一母同胞兄弟。」那婦人叉手向前,便道:「叔叔萬福!」武松施禮,倒身下拜。婦人扶住武松道:「叔叔請起,折殺奴家!」武松道:「嫂嫂受禮!」兩個相讓了一回,都平磕了頭,起來。

少頃,小女迎兒,拿茶二人吃了。武松見婦人十分妖嬈,只把頭來低着。不多時,武大安排酒飯,管待武松。說話中間,武大下樓買酒菜去了。丟下婦人獨自在樓上陪武松坐的,看了武松身材凜凜,相貌堂堂,身上恰似有千百斤氣力。不然,如何打得那大蟲?心裡尋思道:「一母所生的兄弟,又這般長大,人物壯健,奴若嫁得這個,胡亂也罷了!你看我家那身不滿尺的丁樹,三分似人,七分似鬼。奴那世裡遭瘟?直到如今!據看武松,又好氣力,何不交他搬來我家住?誰想這段姻緣,卻在這裡!」那婦人一面臉上排下笑來,問道:「叔叔,你如今在那裡居住?每日飯食,誰人整理?」武松道:「武二新充了都頭,逐日答應上司,別處住不方便,胡亂在縣前尋了個下處,每日撥兩個士兵服事做飯。」婦人道:「叔叔何不搬來家裡住,省的在縣前士兵服事,做飯腌臢。一家裡住,早晚要些湯水吃時,也方便些。就是奴家親自安排與叔叔吃,也乾淨。」武松道:「深謝嫂嫂。」婦人又道:「莫不別處有嬸嬸,可請來廝會也。」武松道:「武二並不曾婚娶。」婦人道:「叔叔青春多少?」武松道:「虛度二十八歲。」婦人道:「原來叔叔到長奴三歲。叔叔今番從那裡來?」武松道:「在滄洲住了一年有餘,只想哥哥在舊房居住,不想搬在這裡!」婦人道:「一言難盡。自從嫁得你哥哥,吃他忒善了,被人欺負;纔得到這裡。若似叔叔這般雄壯,誰敢道個不是。」武松道:「家兄從來本分,不似武松撒潑。」婦人笑道:「怎的顛倒說?常言:『人無剛強,安身不牢。』奴家平生快性,看不上這樣三打不回頭,四打連身轉的人。」有詩為證。

詩曰

「叔嫂萍蹤得偶逢,  嬌嬈遍逞秀儀容。

私心便欲成歡會,  暗把邪言釣武松。」

原來這婦人甚是言語撇清。武松道:「家兄不惹禍,免嫂嫂憂心。」二人只在樓上說話未了,只見武大買了些肉菜、果餅歸來,放在廚下,走上樓來,叫道:「大嫂,你且下來安排則個。」那婦人應道:「你看那不曉事的!叔叔在此,無人陪侍,卻交我撇了下去。」武松道:「嫂嫂請方便。」婦人道:「何不去間壁請王乾娘來安排便了,只是這般不見便!」武大便自去央了間壁王婆子來,安排端正,都拿上樓來,擺在桌子上。無非是些魚肉果菜點心之類,隨即盪上酒來。武大教婦人坐了主位,武松對席,武大打橫,三人坐下,把酒來斟,武大篩酒 在各人面前。那婦人拿起酒來,道:「叔叔休怪,沒甚管待,請盃兒水酒。」武松道:「感謝嫂嫂,休這般說。」武大只顧上下篩酒,那裡來管閑事?那婦人笑容可鞠,滿口兒叫:「叔叔,怎的肉果兒也不揀一筯兒?」揀好的遞將過來。武松是個直性漢子,只把做親嫂嫂相待。誰知這婦人是個使女出身,慣會小意兒。亦不想這婦人一片引人心,那武大又是善弱的人,那裡會管待人。婦人陪武松吃了幾盃酒,一雙眼只看著武松身上,武松乞他看不過,只低了頭不理他。吃了一歇,酒闌了,便起身。武大道:「二哥,沒事再吃幾盃兒去。」武松道:「生受!我再來望哥哥、嫂嫂罷。」都送下樓來。出的門外,婦人便道:「叔叔是必上心,搬來家裡住,若是不搬來,俺兩口兒也吃別人笑話;親兄弟,難比別人,與我們爭口氣,也是好處!」武松道:「既是吾嫂厚意,今晚有行李便取來。婦人道:「叔叔是必記心者,奴這裡專候。」正是:

「滿前野意無人識,  幾點碧桃春自開。」

有詩為證:

「可怪金蓮用意深,  包藏淫行蕩春心;

武松正大原難犯,  耿耿清名抵萬金。」

當日這婦人情意,十分慇動。卻說武松到縣前客店內,收拾行李舖蓋,交士兵挑了,引到哥家。那婦人見了,強如拾了金寶一般歡喜。旋打掃一間房,與武松安頓停當。武松分付士兵回去,當晚就在哥家宿歇。次日早起,婦人也慌忙起來,與他燒湯淨面。武松梳洗裹幘,出門去縣裡畫卯,婦人道:「叔叔畫了卯,早些來家吃飯,休去別處吃了。」武松應說,到縣裡畫卯已畢,伺候了一早晨,回到家中。那婦人又早齊齊整整,安排下飯,三口兒同吃了飯。婦人雙手便捧一盃茶來,遞與武松。武松道:「交嫂嫂生受,武松寢食不安!明日縣裡撥個士兵來使喚。」那婦人連聲叫道:「叔叔,卻怎生這般計較?自家骨肉,又不服事了別人!雖然有這小丫頭迎兒,奴家見他拏東拏西,蹀里蹀科,也不靠他。就是撥了士兵來,那廝上鍋上灶不乾淨,奴眼裡也看不上這等人。」武松道:「恁的,都生受嫂嫂了!」有詩為證:

「武松儀表甚搊搜,  阿嫂淫心不可收;

籠絡歸來家裡住,  要同雲雨會風流。」

話休絮煩。自從武松搬來哥家裡住,取些銀子出來與武大,交買餅饊茶果,請那兩邊鄰舍。都聞分子,來與武松人情。武大又按排了回席,都不在話下。過了數日,武松取出一疋彩色段子,與嫂嫂做衣服。那婦人堆下笑來,便道:「叔叔,如何使得!既然賜與奴家,不敢推辭!」只得接了,道個萬福。自此武松只在哥家歇宿。武大依前上街,挑賣炊餅。武松每日,自去縣裡承差應事,不論歸遲歸早,婦人頓羹頓飯,歡天喜地服事武松。武松倒安身不得,那婦人時常把些言語來撥他。武松是個硬心的直漢,有話即長,無話即短。不覺過了一月有餘,看看十一月天氣,連日朔風緊起。只見四下彤雲密佈,又早紛紛揚揚,飛下一天瑞雪來。但見:

「萬里彤雲密佈,空中祥瑞飄簾,瓊花片片舞前簷。

剡溪當此際,濡伋子猷船,頃刻樓臺都壓倒,

江山銀色相連,飛淺撒粉漫連天,當時呂蒙正,窑內嗟無錢。」

當日這雪直下到一更時分,都似銀粧世界,玉碾乾坤。次日,武松果去縣裡畫卯,直到日中未歸。武大被婦人早趕出去做買賣,央及間壁王婆買了些酒肉,去武松房裡,簇了一盆炭火。心裡自想道:「我今日著實撩鬬他一鬬,不怕他不動情!」那婦人獨自冷冷清清立在簾兒下,望見武松正在雪裡,踏著那亂瓊碎玉歸來。婦人推起簾子,迎著笑道:「叔叔,寒冷?」武松道:「感謝嫂嫂罣心!」入將門來,便把毡笠兒除將下來,那婦人將手去接。武松道:「不勞嫂嫂生受!」自把雪來拂了,掛在壁子上。隨即解了纏帶,脫了身上鸚哥綠紵絲衲襖,入房內。那婦人便道:「奴等了一早晨,叔叔怎的不歸來吃早飯?」武松道:「早間有一相識請我吃飯了,都纔又有一個作盃,我不耐煩,一直走到家來。」婦人道:「既恁的,請叔叔向火。」武松道:「正好。」便脫了油靴,換了一雙襪子,穿了暖鞋,掇條凳子,自近火盆邊坐的。那婦人早令迎兒把前門上了閂,後門也關了。都換些煮酒菜蔬入房裡來,擺在桌子上。武松問道:「哥哥那裡去了?」婦人道:「你哥哥每自出去做些買賣,我和叔叔自吃三盃。」武松道:「一發等哥來家吃也不遲。」婦人道:「那裡等的他?」說由未了,只見迎兒小女早煖了一注酒來。武松道:「不必嫂嫂費心,待武二自斟。」婦人也掇一條凳子,近火邊坐了。

桌上擺著盃盤,婦人拏盞酒,擎在手裡,看著武松:「叔叔滿飲此盃!」武松接過酒去,一飲而盡。那婦人又篩一盃來,說道:「天氣寒冷,叔叔飲個成雙的盞兒。」武松道:「嫂嫂自飲。」接來又一飲而盡。武松都篩一盃酒,遞與婦人,婦人接過酒來,呷了,都拏注子再斟酒,放在武松面前。那婦人一徑將酥胸微露,雲鬟半軃,臉上堆下笑來,說道:「我聽得人說,叔叔在縣前街上,養著個唱的,有這話麼?」武松道:「嫂嫂休聽的人胡說,我武二從來不是這等人!」婦人道:「我不信,只怕叔叔口頭不是心頭。」武松道:「嫂嫂不信時,只問哥哥就見了。」婦人道:「呵呀!你休說,他那裡曉得甚麼?如在醉生夢死一般!他若知道時,不賣炊餅了。叔叔且請一盃!」連篩了三四盃飲過。那婦人也有三盃酒落肚,烘動春心,那裡按納得住?慾心如火,只把閑話來說。武松也知了八、九分,自己只把頭來低了,都不來兜攬。婦人起身去盪酒,武松自在房內,都拏火筯簇火。

婦人良久煖了一注子酒來到房裡,一隻手拏著注子,一隻手便去武松肩上只一捏,說道:「叔叔,只穿這些衣服,不寒冷麼?」武松已有五七分不自在,也不理他。婦人見他不應,匹手便來奪火筯,口裡道:「叔叔你不會簇火,我與你撥火。只要一似火盆來熱,便好。」武松有八九分焦燥,只不做聲。這婦人也不看武松焦燥,便丟下火筯,卻篩一盞酒來,自呷了一口,剩下大半盞酒,看著武松道:「你若有心,吃我這半盃兒殘酒。」乞武松匹手奪過來,潑在地下。說道:「嫂嫂,不要恁的不識羞恥!」把手只一推,爭些兒把婦人推了一交。武松睜起眼來,說道:「武二是個頂天立地的噙齒戴髮的男子漢,不是那等敗壞風俗傷人倫的豬狗。嫂嫂休要這般不識羞恥,為此等的勾當!倘有些風吹草動,我武二眼裡認的是嫂嫂,拳頭都不認的是嫂嫂!再來休要如此所為。」婦人吃他幾句,搶的通紅了面皮,便叫迎兒收拾了碟盞家火。口裡指著說道:「我自作耍子,不值得便當真起來!好不識人敬!」收了家火,自往廚下去了。有詩為證:

「潑賤【言柔】心太不良,  貪淫無恥壞綱常;

席間尚且求雲雨,  反被都頭罵一場。」

這婦人見抅搭武松不動,反被他搶白了一場。武松自在房中氣忿忿的,自己尋思。天色都早申牌時分,武大挑著擔兒大雪裡歸來。推開門,放下擔兒,進的房來,見婦人一雙眼哭的紅紅的,便問道:「你和誰鬧來?」婦人道:「都是你這不爭氣的,交外人來欺負我!」武大道:「誰敢來欺負你?」婦人道:「情知是誰!爭奈武二那廝,我見他大雪裡歸來,好意安排些酒飯與他吃,他見前後沒人,便把言語來調戲我。便是迎兒眼見,我不賴他!」武大道:「我兄弟不是這等人,從來老實!休要高聲,乞鄰舍聽見笑話!」武大撇了婦人,便來武松房裡。叫道:「二哥,你不曾吃點心,我和你吃些個。」武松只不做聲。尋思了半晌,脫了絲鞋,依舊穿上油臘靴,著了上蓋,戴上毡笠兒。一面繫纏帶,一面出大門。武大叫道:「二哥你那裡去?」也不答,一直只顧去了。

武大回到房內,問婦人道:「我叫他,又不應,只顧往縣前那條路去了。正不知怎的了!」婦人罵道:「賊混沌蟲,有甚麼難見處!那廝羞了,沒臉兒見你,走了出去。我猜他一定叫個人來搬行李,不要在這裡住;都不道你留他。」武大道:「他搬了去,須乞別人笑話!」婦人罵道:「混沌魍魎!他來調戲我,到不乞別人笑話?你要便和他過去,我都做不的這樣人。你與了我一紙休書,你自留他便了!」武大那裡再敢開口,被這婦人倒數罵了一頓。

正在家兩口兒絮聒,只見武松引了個士兵,拿著條扁擔,徑來房內,收拾行李便出門。武大走出來,叫道:「二哥,做甚麼便搬了去?」武松道:「哥哥不要問,說起來裝你的幌子。只由我自去便了!」武大那裡再敢問備細,由武松搬了出去。那婦人在裡面喃喃吶吶罵道:「都也好!只道是親難轉債,人自知道。一個兄弟做了都頭,怎的養活了哥嫂。都不知反來嚼咬人!正是花木瓜,空好看,搬了去,到謝天地,且得冤家離眼前。」武大見老婆這般言語,不知怎的了,心中只是放去不下。

自從武松搬去縣前客店宿歇,武大自依前上街賣炊餅,本待要去縣前尋兄弟說話,都被這婦人千叮萬囑,分付交不要去兜攬他,因此武大不敢去尋武松。有詩為證:

「雨意雲情不遂謀,  心中誰信起戈矛;

生將武二搬離去,  骨肉番令作寇仇。」

畢竟未知後來何如,且聽下回分解:




\end{showcontents}

