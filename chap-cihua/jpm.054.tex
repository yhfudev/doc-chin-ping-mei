%# -*- coding: utf-8 -*-
%!TEX encoding = UTF-8 Unicode
%!TEX TS-program = xelatex
% vim:ts=4:sw=4
%
% 以上设定默认使用 XeLaTex 编译,并指定 Unicode 编码,供 TeXShop 自动识别

%第五十四回 
\chapter{應伯爵郊園會諸友\KG 任醫官豪家看病症}


\begin{showcontents}{}



「來日陰晴未可商,  常言極樂起憂惶,

浪遊年少耽紅陌,  薄命嬌娥怨綠窗;

乍入杏村沽美酒,  還從橘井問奇方,

人生多少悲歡事,  幾度春風幾度霜。」

話說西門慶在金蓮房裡起身,分付琴童、玳安送豬蹄羊肉到應二爹家去。兩個小廝政送去時,應伯爵政邀客回來,見了就進房,帶邀帶請的寫一張回字:「昨擾極,茲復承佳惠,謝謝!即刻屈吾兄過舍,同往郊外一樂。」寫完了,走出來,將交與玳安。玳安道:「別要寫字去了。爹差我們兩個在這裡伏侍,也不得去了。」應伯爵笑道:「怎好勞動你兩個親油嘴,折殺了你二爹哩!」就把字來袖過了。玳安道:「二爹,今日在那笪兒吃酒?我們把卓子也擺擺麼?還是灰塵的哩!」伯爵道:「好人呀,正待要抹抹。先擺在家裡吃了便飯,然後到郊園上去頑耍。」琴童道:「先在家裡吃飯,也倒有理,省得又到那裡吃飯,徑把攢盒酒小碟兒拿去罷。」伯爵道:「你兩個倒也聰明,正合二爹的粗主意。想是日夜被人鑽掘,掘開了聰明孔哩!」玳安道:「別要講閑話,就與你收拾起來。」伯爵道:「這叫做接連三個觀音堂,妙妙妙!」兩個安童剛收拾了七八分,只見搖搖擺擺的走進門來,卻是白來創。見了伯爵拱手,又見了琴童、玳安道:「這兩個小親親,這等奉承你二爹?」伯爵道:「你莫待撚酸哩!」笑了一番。白來創道:「哥請那幾客?」伯爵道:「只是弟兄幾個坐坐,就當會茶,沒有別的新客。」白來創道:「這卻妙了!小弟極怕的是面沒相識的人同吃酒。今日我們弟兄輩小敘,倒也好吃頑耍。只是席上少不得娼的,和吳銘、李惠兒彈唱彈唱,倒也好吃酒。」伯爵道:「不消分付,此人自然知趣。難道悶昏昏的,吃了一場便罷了?你幾曾見我是恁的來?」白來創道:「停當停當,還是你老幫襯。只是停會兒,少罰我的酒。因前夜吃了火酒,吃得多了,嗓子兒怪疼的要不得,只吃些茶飯粉湯兒罷。」伯爵道:「酒病酒藥醫,就吃些何妨?我前日也有些嗓子痛,吃了幾杯酒,倒也就好了,你不如依我這方,絕妙。」白來創道:「哥你只會醫嗓子,可會醫肚子麼?」伯爵道:「你想是沒有用早飯?」白來創道:「也差不遠。」伯爵道:「怎麼處?」就跑的進去了。拿一碟子乾糕、一碟子檀香餅、一壺茶出來,與白來創吃。那白來創把檀香餅一個一口,都吃盡了,讚道:「這餅卻好!」伯爵道:「糕亦頗通。」白來創就嗶嗶聲都吃了。只見琴童、玳安收迭家活,一霎地明窗淨几。白來創道:「收拾恁的整齊了,只是弟兄們還未齊。早些來頑頑也得,怎地只管縮在家裡,不知做甚的來?」伯爵政望著外邊,只見常時節走進屋裡來。琴童政掇茶出來,常時節拱手畢,便瞧著琴童道:「是你在這裡?」琴童笑而不答。吃茶畢,三人剛立起散走。白來創看見櫥上有一副棋枰,就對常時節道:「我與你下一盤棋。」常時節道:「我方走了熱剩剩的,政待打開衣帶搧搧扇子,又要下棋!也罷麼,待我胡亂下局罷。」就取下棋枰來下棋。伯爵道:「賭個東道兒麼?」白來創道:「今日擾兄了,不如著入己的,倒也徑捷些兒,省得虛脾胃,吃又吃不成。倒不如人己的有實惠。」伯爵道:「我做主人不來,你們也著東道來湊湊麼?」笑了一番。白來創道:「如今說了,著甚麼東西?還是銀子。」常時節道:「我不帶得銀子,只有扇子在此,當得二三錢銀子起的,漫漫的贖了罷。」白來創道:「我是贏別人的絨繡汗巾,在這裡也值許多,就著了罷。」一齊交與伯爵,伯爵看看,一個是詩畫的白竹金扇,卻是舊做骨子。一個是簇新的繡汗巾。說道:「都值的,徑著了罷。」伯爵把兩件拿了,兩個就對局起來。琴童、玳安見家主不在,不住的走在椅子後邊,來看下棋。伯爵道:「小油嘴,有心央及你來再與我泡一甌茶來。」琴童就對玳安暗暗裡做了一個鬼臉,走到後邊燒茶了。卻說白來創與常時節棋子原差不多,常時節略高些,白來創極會反悔,政著時,只見白來創一塊棋子,漸漸的輸倒了。那常時節暗暗決他要悔,那白來創果然要拆幾著子。一手撇去常時節著的子,說道:「差了差了,不要這著。」常時節道:「哥子來,不好了。」伯爵奔出來道:「怎的鬧起來?」常時節道:「他下了棋,差了三四著,後又重待拆起來,不算帳,哥做個明府,那裡有這等率性的事?」白來創面色都紅了,太陽裡都是青筋綻起了,滿面涎唾的嚷道:「我也還不曾下,他又撲的一著了。我政待看個分明,他又把手來影來影去,混帳得人眼花撩亂了。那一著方纔著下,手也不曾放,又道我悔了,你斷一斷,怎的說我不是?」伯爵道:「這一著便將就著了,也還不叫悔,下次再莫待恁的了。」常時節道:「便罷,且容你悔了這著。後邊再不許你『白來創』我的子了。」白來創笑道:「你是『常時節』輸慣的,倒來說我。」政說話間,謝希大也到了。琴童掇茶吃了,就道:「你們自去完了棋,待我看看。」正看時,吳典恩也正走到屋裏來了。都敘過寒溫,就問:「可著甚的來?」伯爵把二物與眾人看,都道:「既是這般,須著完了。」白來創道:「九阿哥,完了罷,只管思量甚的?」常時節政在審局,吳典恩與謝希大旁賭。希大道:「九弟勝了。」吳典恩道:「他輸了,恁地倒說勝了?賭一杯酒。」常時節道:「看看區區叨勝了。」白來創臉都紅了,道:「難道這把扇子是送你的了?」常時節道:「也差不多。」于是填完了官著,就數起來。白來創看了五塊棋頭,常時節只得兩塊。白來創又該找還常時節三個棋子,口裡道:「輸在這三著了。」連忙數自家棋子,輸了五個子。希大道:「可是我決著了。」指吳典恩道:「記你一杯酒,停會一准要吃還我。」吳典恩笑而不答。伯爵就把扇子併原梢汗巾,送與常時節。常時節把汗巾原袖了,將扇子拽開賣弄,品評詩畫,眾人都笑了一番。玳安外邊奔進來報,卻是吳銀兒與韓金釧兒兩個相牽相引,嬉笑進來了,深深的相見眾位。白來創意思遲要下盤,卻被眾人笑了。伯爵道:「罷罷,等大哥一來,用了飯,就到郊園上去。著到幾時?莫要著了。」于是琴童忙收棋子,都吃過茶。伯爵道:「大哥此時也該來了,莫待弄宴了,頑耍不來?」剛說時,西門慶來到,衣帽齊整,四個小廝跟隨,眾人都下席迎接,敘禮讓坐,兩個妓女都磕了頭。吳銘、李惠都到來磕頭過了。伯爵就催琴童、玳安拿上八個靠山小碟兒,盛著十香瓜、五方荳豉醬油浸的花椒、釅醋滴的苔菜 、一碟糖蒜 、一碟糟筍乾、一碟辣菜 、一碟醬的大通薑 、一碟香菌 擺放停當。兩個小廝見西門慶坐地,加倍小心,比前越覺有些馬前健。伯爵見西門慶看他擺放家活,就道:「虧了他兩個,收拾了許多事,替了二爹許多力氣。」西門慶道:「恐怕也伏侍不來。」伯爵道:「忒會了些。」謝希大道:「自古道強將手下無弱兵,畢竟經了他們,自然停當。」那兩個小廝擺完小菜,就拿上大壺酒來,不住的拿上廿碗下飯菜兒,蒜燒荔枝肉 、葱白椒料 檜皮煮的爛羊肉 ,燒魚、燒雞、酥鴨 、熟肚 之類,說不得許多色樣。原來伯爵在各家吃轉來,都學了這些好烹庖了,所以色色俱精,無物不妙。眾人都拏起筯來,嗒嗒聲都吃了幾大杯酒,就拿上飯來吃了。那韓金釧吃素,再不用葷,只吃小菜。伯爵道:「今日又不是初一月半,喬作衙甚的?當初有一個人,吃了一世素,死去見了閻羅王,說:『我吃了一世素,要討一個好人身。』閻王道:『那得知你吃不吃?且割開肚子驗一驗。』割開時,只見一肚子涎唾。原來平日見人吃葷,嚥在那裡的。」眾人笑得翻了。金釧道:「這樣搗鬼,是那裡來!可不怕地獄拔舌根麼?」伯爵道:「地獄裡只拔得小淫婦的舌根,道是他親嘴時會活動哩。」都笑一陣。伯爵道:「我們到郊外去一遊何如?」西門慶道:「極妙了!」眾人都說妙。伯爵就把兩個食盒,一罈酒,都央及玳安與各家人抬在河下。喚一隻小舡,一齊下了,又喚一隻空舡載人。眾人逐一上舡,就搖到南門外三十里有餘,徑到劉太監庄前。伯爵叫灣了船,就上岸,扶了韓金釧、吳銀兒兩個上岸。西門慶問道:「到那一家園上走走倒好?」應伯爵道:「就是劉太監園上也好。」西門慶道:「也罷,就是那笪也好。」眾人都到那裡,進入一處廳堂,又轉入曲廊深徑,茂林修竹,說不盡許多景致。但見:

「翠柏森森,修篁簌簌。芳草平舖青錦褥,垂楊細舞綠絲縧。曲砌重欄,萬種名花紛若綺;幽窗密牖,數聲嬌鳥弄如簧。真同閬苑風光,不減清都景致。散淡高人,日涉之以成趣;往來游女,每樂此而忘疲。果屬奇觀,非因過譽。」

西門慶攜了韓金釧、吳銀兒手,走往各處,飽玩一番。到一木香棚下,蔭涼的緊,兩邊又有老大長的石凳琴臺,恰好散坐的,眾人都坐了。伯爵就去交琴童兩個舡上人,拿起酒盒、菜蔬、風爐、器皿等上來,都放在綠蔭之下,先吃了茶,閑話起孫寡嘴、祝麻子的事。常時節道:「不然,今日也在這裡。那裡說起!」西門慶道:「也是自作自受。」伯爵道:「我們坐了罷。」白來創道:「也用得著了。」于是就擺列坐了。西門慶首席坐下,兩個妓女就坐在西門慶身邊。吳銘、李惠立在太湖石邊,輕撥琵琶,漫擎檀板,唱一隻曲,名曰水仙子:

「據著俺老母情,他則待祅廟火,刮刮匝匝烈焰生。將水面上鴛鴦,忒楞楞騰,生分開交頸。疎刺刺沙鞲雕鞍,撒了鎖鞓,廝琅琅湯偷香處喝號提鈴,支楞楞箏絃斷了不續碧玉箏。咭叮叮噹,精甎上摔碎菱花鏡,撲通通鼕,井底墜銀瓶。」

唱畢,又移酒到水池邊,舖下毡單,都坐地了。傳盃弄盞,猜拳賽色,吃得恁地熱鬧。西門慶道:「董嬌兒那個小淫婦,怎地不來?」應伯爵道:「昨日我自去約他,他說要送一個漢子出門,約午前來的。想必此時曉得我們在這裡頑耍,他一定趕來也。」白來創道:「這都是二哥的過,怎的不約實了他來?」西門慶就向白來創耳邊說道:「我們與那花子賭了。只說過了日中,董嬌兒不來,各罰主人三大碗。」白來創對應伯爵說了。伯爵道:「便罷。只是日中以前來了,要罰列位三大碗一個。」賭便一時賭了,董嬌兒那得見來?伯爵慌得只管笑。白來創與謝希大、西門慶、兩個妓女,這般這般,都定了計。西門慶假意淨手起來。分付玳安交他假意嚷將進來,只說董姑娘在外來了,如此如此。玳安曉得了。停了一會時,伯爵正在遲疑,只見玳安慌不迭的奔將來道:「董家姐姐來了!不知那裡尋的來?」那伯爵嚷道:「樂殺我老太婆也!我說就來的。快把酒來,各請三碗一個。」西門慶道:「若是我們嬴了,要你吃你怎的就肯吃?」伯爵道:「我若輸了,不肯吃,不是人了!」眾人道:「是便是了。你且去叫他進來,我們纔好吃。」伯爵道:「是了。好人口裡的言語呢!」一走出去,東西南北都看得眼花了,那得董嬌兒的魂靈?望空罵道:「賊淫婦,在二爺面上這般的拔短梯,喬作衙哩!」走進去,眾人都笑得了不的。擁住道:「如今日中過了,要吃還我們三碗一個。」伯爵道:「都是小油嘴哄我,你們倒做實了我的酒了。怎的擺佈?」西門慶不由分說,滿滿捧一碗酒,對伯爵道:「方纔說的,不吃不是人了。」伯爵接在手,謝希大接連又斟一碗來了,吃也吃不完,吳典恩又接手斟一大碗酒來了,慌得那伯爵了不的,嚷道:「不好了,嘔出來了。拏些小菜我過過,便好。」白來創倒取甜東西去。伯爵道:「賊短命,不把酸的,倒把甜的來混帳!」白來創笑道:「那一碗就是酸的來了。左右鹹酸苦辣,都待嚐到罷了。且沒慌著!」伯爵道:「精油嘴,硶誇口得好!」常時節又送一碗來了,伯爵只待奔開暫避。西門慶和兩個妓女擁住了,那裡得去?伯爵叫道:「董嬌兒賊短命小淫婦!害得老子好苦也!」眾都笑做一堆。那白來創又交玳安拿酒壺,滿滿斟著。玳安把酒壺嘴支入碗內一寸許多,骨都都只管篩,那裡肯住手。伯爵瞧著道:「痴客勸主人也罷。那賊小淫婦慣打閛閛的,怎的把壺子都放在碗內了!看你一千年,我二爺也不攛掇你討老婆哩!」韓金釧、吳銀兒各人斟了一碗送與應伯爵。應伯爵道:「我跪了殺雞罷!」韓金釧道:「都免禮,只請酒便了。」吳銀兒道:「怎的不向董家姐姐殺雞,求他來了?」伯爵道:「休見笑了,也勾吃了。」兩個一齊推酒到嘴,伯爵不好接一頭,兩手各接了一碗,就吃完了。連忙吃了些小菜,一時面都通紅了。叫道:「我被你們弄了。酒便慢慢吃還好,怎的灌得悶不轉的!」眾人只待斟酒。伯爵跪著西門慶道:「還求大哥說個方便,饒恕小人窮性命,還要留他陪客。若一醉了,便不知天好日暗,一些興子也沒有了。」西門慶道:「便罷,這兩碗一個,你且欠著,停徵了罷。」伯爵就起來謝道:「一發蠲免了罷,足見大恩!」西門慶道:「也罷,就恕了你。只是方纔說,我們不吃,不是個人。如今你漸有些沒人氣了!」伯爵道:「我倒灌醉了。那淫婦不知那裡歪斯纏去了!」

吳銀兒笑伯爵道:「咳,怎的大老官人在這裡做東道頑耍,董嬌姐也不來來?」伯爵假意道:「他是上檯盤的名妓,倒是難請的。」韓金釧兒道:「他是趕勢利去了。成甚的行貨,叫他是名妓!」伯爵道:「我曉得你想必有些吃醋的宿帳哩!」西門慶認是蔡公子那夜的故事,把金釧一看,不在話下。那時伯爵已是醉醺醺的。兩個妓女又不是耐靜的,只管調唇弄舌,一句來,一句去,歪斯纏到吃得冷淡了。白來創對金釧道:「你兩個唱個曲兒麼?」吳銀兒道:「也使得。」讓金釧先唱。常時節道:「我勝那白阿弟的扇子,倒是板骨的,倒也好打板。」金釧道:「借來打一打板。」接去看看道:「我倒少這把打板的扇子。不作我贏的棋子,送與我罷。」西門慶道:「這倒好。」常時節吃眾人攛掇不過,只得送與他了。金釧道:「吳銀姐在這裡,我怎的好獨要。我與你猜色,那個色大的,拿了罷。」常時節道:「這卻有理。」就猜一色,是吳銀兒贏了。金釧就遞與銀兒了。常時節假冠冕道:「這怎麼處?我還有一條汗巾,送與金釧姐,補了扇罷。」遂送過去。金釧接了道:「這卻撒漫了。」西門慶道:「我可惜不曾帶得好川扇兒來,也賣富賣富。」常時節道:「這是打我一下了。」那謝希大驀地嚷起來道:「我幾乎忘了!又是說起扇子來!」交玳安斟了一大杯酒,送與吳典恩道:「請完了旁賭的酒。」吳典恩道:「這罷了。停了幾時纔想出來,他每的東西都花費了,那在一杯酒?」被謝希大逼勒不過,只得呷完了。那時金釧就唱一曲,名喚荼{艹縻}香:

「記得初相守,偶爾間因循成就,美滿效綢繆。花朝月夜同宴賞,佳節須酬,到今日一旦休。常言道,好事天慳,美姻緣他娘間阻,生拆散鸞交鳳友。  坐想行思,傷懷感舊,辜負了星前月下深深咒。願不損,愁不煞,神天還祐,他有口不測相逢,話別離,情取一場消瘦。」

唱畢,吳銀兒接唱一曲,名青杏兒:

「風雨替花愁,風雨過花也應休。勸君莫惜花前醉,今朝花謝,白了人頭。  乘興再三甌,揀溪山好處追遊。但教有酒身無事,有花也,無花也,好選甚春秋?」

唱畢,李惠、吳銘排立,謝希大道:「還有這些伎藝,不曾做哩。」只見彈的彈,吹的吹,琵琶簫管,又唱一隻小梁州:

「門外紅塵滾滾飛,飛不到魚鳥清溪。綠陰高柳聽黃鸝,幽棲意,料俗客幾人知。山林本是終焉計,用之行,舍之藏兮。悼後世,追前輩;五月五日。歌楚些弔湘纍。」

唱畢,酒興將闌。那白來創尋見園廳上,架著一面小小花框羯鼓,被他馱在湖山石後,又折一枝花來,要催花擊鼓。西門慶叫李惠、吳銘擊鼓,一個眼色,他兩個就曉得了,從石孔內瞧著,到會吃的面前,鼓就住了。白來創道:「畢竟賊油嘴,有些作弊!我自去打鼓。」也弄西門慶吃了幾杯。正吃得熱鬧,只見書童搶進來,到西門慶身邊,附耳低言道:「六娘子身子不好的緊,快請爹回來。馬也備在門外接了。」西門慶聽得,連忙走起告辭。那時酒都有了,眾人都起身。伯爵道:「哥,今日不曾奉酒,怎的好去?是這些耳報法極不好。」便待留住。西門慶以實情告訴他,就謝了上馬來。伯爵又留眾人,一個韓金釧霎眼挫不見了。伯爵躡足潛踪尋去,只見在湖山石下撒尿,露出一條紅線,拋卻萬顆明珠。伯爵在隔籬笆眼,把草戲他的牝口。韓金釧撒也撒不完,吃了一驚,就立起,褌腰都濕了。罵道:「硶短命,恁尖酸的沒槽道!」面都紅了,帶笑帶罵出來。伯爵與眾人說知,又笑了一番。西門慶原留琴童與伯爵收拾家活。琴童收拾風爐食具下舡,都進城了。眾人謝了伯爵,各散去訖。伯爵就打發兩隻舡錢,琴童送進家活,伯爵就打發琴童吃酒。都不在話下。卻說西門慶來家,兩步做一步走,一直走進六娘房裡。迎春道:「俺娘了不得病,爹快看看他。」走到床邊,只見李瓶兒咿嚶的叫疼,卻是胃腕作疼。西門慶聽他叫得苦楚,連忙道:「快去請任醫官來看你。」就叫迎春:「喚書童寫帖,去請任太醫。」迎春出去說了。書童隨寫侍生帖,去請任太醫了。西門慶擁了李瓶兒,坐在床上,李瓶兒道:「恁的酒氣!」西門慶道:「是胃虛了,便厭著酒氣。」又對迎春道:「可曾吃些粥湯?」迎春回道:「今早至今,一粒米也沒有用,只吃了兩三甌湯兒。心口肚腹兩腰子,都疼得異樣的。」西門慶攢著眉,皺著眼,嘆了幾口氣。又問如意兒:「官哥身子好了麼?」如意兒道:「昨夜還有頭熱,還要哭哩!」西門慶道:「恁的悔氣!娘兒兩個都病了,怎的好?留得娘的精神,還好去支持孩子哩!」李瓶兒又叫疼起來了。西門慶道:「且耐心著,太醫也就來了。待他看過脉,吃兩鍾藥,就好了的。」迎春打掃房裡,抹淨卓椅,燒香點茶。又支持奶子,引鬬得官哥睡著。此時有更次了,外邊狗叫得不迭,卻是琴童歸來。不一時,書童掌了燈,照著任太醫四角方巾,大袖衣服,騎馬來了。進門坐在軒下。書童走進來說:「請了來了,坐在軒下了。」西門慶道:「好了,快拿茶出來。」玳安即便掇茶,跟西門慶出去迎接任太醫。太醫道:「不知尊府那一位看脉?失候了,負罪實多!」西門慶道:「昏夜勞重,心切不安。萬惟垂諒!」太醫著地打躬道:「不敢!」吃了一鍾燻豆子撒的茶,就問:「看那一位尊恙?」西門慶道:「是第六個小妾。」又換一鍾鹹櫻桃的茶 ,說了幾句閒話。玳安接鍾,西門慶道:「裡面可曾收拾?你進去話聲,掌燈出來照進去。」玳安進到房裡去話了一聲,就掌燈出來回報。西門慶就起身打躬,邀太醫進房。太醫遇著一個門口,或是階頭上,或是轉彎去處,就打一個半喏的躬,渾身恭敬,滿口寒溫。走進房裡,只見沉煙繞金鼎,蘭火爇銀缸。錦帳重圍,玉鈎齊下。真是繁華深處,果然別一洞天。西門慶看了太醫的椅子,太醫道:「不消了。」也答看了西門慶椅子,就坐下了。迎春便把繡褥來,襯起李瓶兒的手,又把錦帕來擁了玉臂,又把自己袖口籠著他纖指,從帳底下露出一段粉白的臂,來與太醫看脉。太醫澄心定氣,候得脉來 卻是胃虛氣弱,血少肝經旺,心境不清,火在三焦,須要降火滋榮。就依書據理,與西門慶說了。西門慶道:「先生果然如見,實是這樣的。這個小妾,性子極忍耐得。」太醫道:「政為這個緣故,所以他肝經原旺,人卻不知他。如今木剋了土,胃氣自弱了。氣那裡得滿?血那裡得生?水不能載火,火都升上截來。胸膈作飽作疼,肚子也時常作疼。血虛了,兩腰子渾身骨節裡頭,通作酸痛,飲食也吃不下了。可是這等的?」迎春道:「正是這樣的。」西門慶道:「真正任仙人了!貴道裡望聞問切,如先生這樣明白脉理,不消問的,只管說出來了。也是小妾有幸!」太醫深打躬道:「晚生曉得甚的?只是猜多了。」西門慶道:「太謙遜了些。」又問:「如今小妾該用什麼藥?」太醫道:「只是降火滋榮,火降了,這胸膈自然寬泰;血足了,腰脅自然不作疼了。不要認是外感,一些也不是的,都是不足之症。」又問道:「經事來得勻麼?」迎春道:「便是不得准。」太醫道:「幾時便來一次?」迎春道:「自從養了官哥,還不見十分來。」太醫道:「元氣原弱,產後失調,遂致血虛了,不是壅積了,要用疏通藥。要逐漸吃些丸藥,養他轉來才好。不然,就要做牢了病。」西門慶道:「便是極看得明白。如今先求煎劑,救得目前痛苦。還要求些丸藥。」太醫道:「當得。晚生返舍,即便送來,沒事的。只要知此症,乃不足之症;其胸膈作痛,乃火痛,非外感也;其腰脅怪疼,乃血虛,非血滯也。吃了藥去,自然逐一好起來,不須焦躁得。」西門慶謝不絕口。剛起身出房,官哥又醒覺了,哭起來。太醫道:「這位公子好聲音。」西門慶道:「便是也會生病,不好得緊。連累小妾,日夜不得安枕。」一路送出來了。卻說書童對琴童道:「我方纔去請他,他已早睡了。敲得半日門,纔有人出來。那老子一路揉眼出來,上了馬,還打盹不住,我只愁突了下來。」琴童道:「你是苦差使。我今日遊玩得了不的,又吃一肚子酒。」政在閑話,玳安掌燈,跟西門慶送出太醫來。到軒下,太醫只管走。西門慶道:「請寬坐,再奉一茶,還要便飯點心。」太醫搖頭道:「多謝盛情,不敢領了。」一直走到出來。西門慶送上馬,就差書童掌燈送去。別了太醫,飛的進去。交玳安拿一兩銀子,趕上隨去討藥。直到任太醫家,太醫下了馬,對他兩個道:「阿叔們,且坐著吃茶,我去拿藥出來。」玳安拿禮盒,送與太醫道:「藥金請收了。」太醫道:「我們是相知朋友,不敢受你老爺的禮。」書童道:「定求收了,纔好領藥。不然,我們藥也不好拿去。恐怕回家去,一定又要送來,空走腳步。不如作速收了,候的藥去便好。」玳安道:「無錢課不靈,定求收了。」太醫只得收了。見藥金盛了,就進去簇起煎劑,連瓶內丸子藥,也倒了淺半瓶。兩個小廝吃茶畢,裡面打發回帖出來,與玳安、書童。徑閉了門,兩個小廝回來。西門慶見了藥袋厚大的,說道:「怎地許多!」拆開看時,卻是丸藥也在裡面了。笑道:「有錢能使鬼推磨。方纔他說先送煎藥,如今都送了來!也好也好。」看藥袋上是寫著:「降火滋榮湯。水二鍾,姜不用,煎至捌分,食遠服,查再煎。忌食麩麪油膩炙煿等物。」又打上「世醫任氏藥室」的印記。又一封筒,大紅票簽,寫著「加味地黃丸」。西門慶把藥交迎春,先分付煎一帖起來。李瓶兒又吃了些湯,迎春把藥熬了,西門慶自家看藥,瀘清了查出來。捧到李瓶兒床前,道:「六娘,藥在此了。」李瓶兒翻身轉來,不勝嬌顫。西門慶一手拿藥,一手扶著他頭頸,李瓶兒吃了叫苦,迎春就拿滾水來過了口。西門慶吃了粥,洗了足,就伴李瓶兒睡了。迎春又燒些熱湯護著,也連衣服假睡了。說也奇怪,吃了這藥,就有睡了。西門慶也就熟睡去了。官哥只管要哭起來,如意兒恐怕哭醒了李瓶兒,把奶子來放他吃,後邊也寂寂的睡了。到次日,西門慶將起身,問李瓶兒:「昨夜覺好些兒麼?」李瓶兒道:「可霎作怪!吃了藥,不知怎地睡的熟了。今日心腹裡,都覺不十分怪疼了。學了昨的下半晚,真要痛死人也!」西門慶笑道:「謝天謝天!如今再煎他二鍾吃了,就全好了。」迎春就煎起第二鍾來吃了。西門慶一個驚魂,落向爪哇國去了。怎見得?有詩為證:

「西施時把翠蛾顰,  幸有仙丹妙入神;

信是藥醫不死病,  果然佛度有緣人。」

畢竟未知如何,且聽下回分解:




\end{showcontents}


