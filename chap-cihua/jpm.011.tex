%# -*- coding: utf-8 -*-
%!TEX encoding = UTF-8 Unicode
%!TEX TS-program = xelatex
% vim:ts=4:sw=4
%
% 以上设定默认使用 XeLaTex 编译,并指定 Unicode 编码,供 TeXShop 自动识别

%第十一回 
\chapter{金蓮激打孫雪娥\KG 西門慶梳籠李桂姐}

\begin{showcontents}{}



「婦人嫉妒非常,  浪人落魄無賴,

一聽巧語花言,  不顧新懽舊愛;

出逢紅袖相牽,  又把風情別賣,

果然寒食元宵,  誰不幫興幫敗。」

話說潘金蓮在家恃寵生驕,顛寒作熱,鎮日夜不得個寧靜。性極多疑,專一聽籬察壁,尋些頭腦廝鬧。那個春梅,又不是十分耐煩的。一日,金蓮為些零碎事情,不湊巧罵了春梅幾句。春梅沒處出氣,走往後邊廚房下去,搥檯拍盤,悶狠狠的模樣。那孫雪娥看不過,假意戲他道:「怪行貨子!想漢子便別處去想,怎的在這裡硬氣?」春梅正在悶時,聽了幾句,不一時暴跳起來:「那個歪廝纏我哄漢子!」雪娥見他性不順,只做不開口。春梅便使性做幾步,走到前邊來,如此如此這般這般,一五一十,又添些話頭道:「我和娘收了,俏一幫兒哄漢子。」挑撥與金蓮知道。金蓮滿肚子不快活,只因送吳月娘出去送殯,起身早些,也有些身子倦,睡了一覺。走到亭子上,只見孟玉樓搖颳的走來,笑嘻嘻道:「姐姐如何悶悶的不言語?」金蓮道:「不要說起,今早倦倒了不得。三姐你在那里去來?」玉樓道:「纔到後面廚房里走了一下。」金蓮道:「他與你說些什麼來?」玉樓道:「姐姐沒言語。」金蓮雖故口裡說著,終久懷記在心,與雪娥結仇,不在話下。兩個做了一回針指,只見春梅抱着湯瓶,秋菊拿了兩盞茶來。吃畢茶,兩個放卓兒,擺下棋子盤兒下棋。正下在熱鬧處,忽目看園門小廝琴童走來報道:「爹來了。」慌的兩個婦人,收棋子不迭。西門慶恰進門檻,看見二人,家常都帶着銀絲䯼髻,露着四鬢,耳邊青寶石墜子,白紗衫兒,銀紅比甲,挑線裙子,雙彎尖趫紅鴛瘦小鞋,一個個粉粧玉琢。不覺滿面堆笑,戲道:「好似一對兒粉頭,也值百十銀子。」潘金蓮說道:「俺每纔不是粉頭,你家正有粉頭在後邊哩。」那玉樓抽身就往後走,被西門慶一手扯住,說道:「你往那里去?我來了,你脫身去了。實說,我不在家,你兩個在這裡做甚麼?」金蓮道:「俺兩個悶的慌,在這裡下了兩盤棋子,時沒做賊,誰知道你就來了。」一面替他接了衣服,說道:「你今日送殯來家早。」西門慶道:「今日齋堂裡,都是內相同官,一來天氣喧熱,我不耐煩,先來家。」玉樓問道:「他大娘怎的還不來家?」西門慶道:「他的轎子也待進城,我使回兩個小廝接去了。」一面脫了衣服坐下。因問:「你兩個下棋,賭些什麼?」金蓮道:「俺兩個自恁下一盤耍子,平白賭什麼?」西門慶道:「等我和你們下一盤,那個輸了,拿出一兩銀子做東道。」金蓮道:「俺每並沒銀子。」西門慶道:「你沒銀子,拏簪子問我手裡當,也是一般。」於是擺下棋子,三人下了一盤,潘金蓮輸了。西門慶纔數子兒,被婦人把棋子撲撒亂了,一直走到瑞香花下,倚着湖山,推掐花兒。西門慶尋到那裡,說道:「好小油嘴兒,你輸了棋子,都躲在這裡。」那婦人見西門慶來,眤笑不止,說道:「怪行貨子,孟三兒輸了,你不敢禁它,都來纏我。」將手中花撮成瓣兒,灑西門慶一身。被西門慶走向前,雙關抱住,按在湖山畔,就口吐丁香,舌融甜唾,戲謔做一處。不防玉樓走到跟前,叫道:「六姐,他大娘來家了,咱後邊去來。」這婦人方纔撇了西門慶,說道:「哥兒,我回來和你答話。」同玉樓到後邊,與月娘到了萬福。月娘問:「你每笑甚麼?」玉樓道:「六姐今日和他爹下棋,輸了一兩銀子,到明日整治東道,請姐姐耍子。」月娘笑了。金蓮當下,只在月娘面前,只打了個照面兒,就走來前邊陪伴西門慶,分付春梅,房中薰下香,預備澡盆浴湯,準備晚間兩個效魚水之懽。看官聽說:家中雖是吳月娘大娘子在正房居住,常有疾病,不管家事;只是人情看往,出門走動。出入銀錢,都在唱的李嬌兒手裡。孫雪娥單管率領家人媳婦,在廚房上灶,打發各房飲食。譬如:西門慶在那房裡宿歇,或吃酒、吃飯,造甚湯水,俱經雪娥手中整理。那房裡丫頭,自往廚下拿去,此事不說。當晚西門慶在金蓮房中吃了回酒,洗畢澡,兩人歇了。次日,也是合當有事。西門慶許了金蓮,要往廟上替他買珠子要穿箍兒戴。早起來,等着要吃荷花餅 、銀絲鮓湯 。纔起身,使春梅往廚下說去。那春梅只顧不動身。金蓮道:「你休使他,有人說我縱容他,教你收了,俏成一幫兒哄漢子。百般指猪罵狗,欺負俺娘兒們;使你又使他後邊做甚麼去?」西門慶便問:「是誰說此話欺負他?你對我說。」婦人道:「說怎的?盆罐都有耳躲;你只不叫他後邊去,另使秋菊去便了。」這西門慶遂叫過秋菊,分付他往廚下,對雪娥說去。約有兩頓飯時,婦人已是把卓兒放了,白不見拿來,急的西門慶只是暴跳。婦人見秋菊不來,使春梅:「你去後邊瞧瞧,那奴才只顧生根長苗不見來?」春梅有幾分不順,使性子走到廚下,只見秋菊正在那裡等着哩。便罵道:「賊餳奴!娘要卸你那腿哩!說你怎的就不去了哩!爹緊等着吃了餅,要往廟上去。急的爹在前邊暴跳,叫我採了你去哩!」這孫雪娥不聽便罷,聽了心中大怒,罵道:「怪小淫婦兒!馬回子拜節,來到的就是鍋兒是鐵打的,也等慢慢兒的來,預備下熬的粥兒,又不吃。忽剌八新梁興出來,要烙餅 做湯,那個是肚裡蛔虫?」春梅不忿他罵,說道:「沒的扯〈毛皮〉淡!主子不使了來問你,那個好來問你要?有沒俺們到前邊自說的一聲兒,有那些聲氣的!」一隻手擰着秋菊的耳朵,一直往前邊來,雪娥道:「主子奴才,常遠似這等硬氣,有時道着!」春梅道:「中有時道使時道,沒的把俺娘兒兩個別變了罷!」於是氣狠狠走來,婦人見他臉氣的黃黃,拉着秋菊進門,便問:「怎的來了?」春梅道:「你問他,我去時還在廚房裡雌着,等他慢條絲禮兒纔和麪兒。我自不是,說了一句:『爹在前面等着,娘說你怎的就不去了;使我來叫你來了。』倒被小院兒裡的,千奴才,萬奴才,罵了我恁一頓。說爹馬回子拜節,來到的就是,只相那個調唆了爹一般。預備下粥兒不吃,平日新生發起要餅和湯;只顧在廚房裡罵人,不肯做哩。」婦人在旁便道:「我說別要使他去,人自恁和他合氣;說俺娘兒兩個把攔你在這屋裡,只當吃人罵將來。」這西門慶聽了心中大怒,走到後邊廚房裡,不由分說,向雪娥踢了幾腳,罵道:「賊歪剌骨!我使他來要餅,你如何罵他?你罵他奴才,你如何不溺胞尿把你自家照照!」那雪娥被西門慶踢罵了一頓,敢怒而不敢言。西門慶剛走出廚房門外,雪娥對着大家人來昭妻一丈青說道:「你看我今日晦氣,早是你在旁聽,我又沒曾說什麼。他走將來,兇神也一般,大吆小喝,把丫頭採的去了。反對主子面前輕事重報,惹的走來,平白把恁一場兒。我洗着眼兒看着,主子奴才長遠恁硬氣着,只休要錯了腳兒!」不想被西門慶聽見了,復回來,又打了幾拳,罵道:「賊奴才淫婦!你還說不欺負他?親耳朵聽見你還罵他!」打的雪娥疼痛難忍,西門慶便往前邊去了。那雪娥氣的在廚房裡,兩淚悲啼,放聲大哭。吳月娘正在上房,纔起來梳頭,因問小玉:「廚房裡亂的些什麼?」小玉回道:「爹要餅吃了往廟上去,說姑娘罵五娘房裡春梅來,被爹聽見了,在廚房裡踢了姑娘幾腳,哭起來。」月娘道:「也沒見,他要餅吃,連忙做了與他去就罷了,平白又罵他房裡丫頭怎的?」於是使小玉走到廚房,攛掇雪娥和家人媳婦,連忙攢造湯水,打發西門慶吃了,騎馬小廝跟隨,往廟上去不題。

這雪娥氣憤不過,走到月娘房裡,正告訴月娘此事。不防金蓮驀然走來,立于窗下潛聽。見雪娥在屋裡對月娘、李嬌兒說,他怎的把攔漢子,背地無所不為:「娘你不知,淫婦說起來比養漢老婆還浪,一夜沒漢子也成不的;背地幹的那繭兒,人幹不出,他幹出來!當初在家,把親漢子用毒藥擺死了,跟了來;如今把俺們也吃他活埋了,弄的漢子烏眼雞一般,見了俺們便不待見!」月娘道:「也沒見你,他前邊使了丫頭要餅,你好好打發他去便了。平白又罵他怎的?」雪娥道:「我罵他禿也瞎也來?那頃這丫頭在娘房裡,着緊不聽手。俺沒曾在灶上把刀背打他,娘尚且不言語;可可今日輸他手裡,便驕貴的這等的了!」正說着,只見小玉走到說:「五娘在外邊。」少頃,金蓮進房,望着雪娥說道:「比對我當初出擺死親夫,你就不消叫漢子娶我來家,省的我把攔着他,撑了你的窩兒。論起春梅,又不是我房裡丫頭,你氣不憤,還教他伏侍大娘就是了。省的你和他合氣,把我扯在裡頭。那個好意死了漢子嫁人?如今也不難的勾當,等他來家,與我一紙休書,我去就是了。」月娘道:「我也不曉得你們底事,你每大家省言一句兒便了。」孫雪娥道:「娘,你看他嘴似淮洪也一般,隨問誰也辦不過他。又在漢子根前戳舌兒,轉過眼就不認了。依你說起來,除了娘,把俺們都攆了,只留着你罷。」那吳月娘坐着,由着他那兩個,你一句,我一句,只不言語。後來見罵起來,雪娥道:「你罵我奴才,你便是真奴才!」拉些兒不曾打起來。月娘看不上,使小玉把雪娥拉往後邊去。這潘金蓮一直歸到前邊,卸了濃粧,洗了脂粉,烏雲散亂,花容不整,哭得兩眼如桃,躺在床上,到日西時分,西門慶廟上來,袖着四兩珠子,進入房中。一見,便問:「怎的來?」婦人放聲號哭起來,問西門慶要休書,如此這般,告訴一遍:「我當初又不曾圖你錢財,自恁跟了你來;如何今日交人這等欺負!千也說我擺殺漢子,萬也說我擺殺漢子。拾了本有,吊了本無。沒丫頭便罷了,如何要人房裡丫頭伏侍?吃人指罵我,一個還多着影兒哩。」這西門慶不聽便罷,聽了此言,三尸神暴跳,五陵氣沖天。一陣風走到後邊,採過雪娥頭髮來,儘力拏短棍打了幾下。多虧吳月娘向前拉住了手,說道:「沒的大家省事些兒罷了!好交你主子惹氣!」西門慶便道:「好賊歪剌骨!我親自聽見你在廚房裡罵,你還攪纏別人;我不把你下截打下來,也不算!」看官聽說:不爭今日打了孫雪娥,管教潘金蓮從前作過事,沒興一起來。有詩為證:

「金蓮恃寵仗夫君,  道使孫娥忌怨深;

自古感恩并積恨,  千年萬載不生塵。」

當下西門慶打了雪娥,走到前邊,窩盤住了金蓮,袖中取出今日廟上買的四兩珠子,遞與他,穿箍兒戴。婦人見漢子與他做主兒,出了氣,如何不喜?由是要一奉十,寵愛愈深。一日,在園中置了一席,請吳月娘、孟玉樓,連西門慶四人共飲酒。話休饒舌,那西門慶立了一夥,結識了十個人做朋友,每月會茶飲酒。頭一個名喚應伯爵,是個潑落戶出身,一分兒家財都敗沒了,專一跟着富家子弟,幫敗貼食,在院中頑耍,諢名叫做應花子。第二個姓謝名希大,乃清河衛千戶官兒,應襲子孫;自幼兒沒了父母。遊手好閒,善能踢的好氣毬,又且賭博,把前程丟了,如今做幫閒的。第三名喚吳典恩,乃本縣陰陽生,因事革退;專一在縣前與官吏保債,以此與西門慶來往。第四名孫天化,綽號孫寡嘴,年紀五十餘歲;專在院中闖寡門,與小娘傳書寄柬,勾引子弟,討風流錢過日子。第五是雲參將兄弟,名喚雲離守。第六是花太監姪兒花子虛。第七姓祝,名喚祝日念。第八姓常,名常時節。第九個姓白,名喚白來創。連西門慶共十個。眾人見西門慶有些錢鈔,讓西門慶做了大哥,每月輪流會茶擺酒。一日,輪該花子虛家擺酒會茶,就在西門慶緊隔壁,內官家擺酒再都是大盤大碗,甚是豐盛。眾人都到齊了,那日西門慶有事,約午後不見到來,都留席面。少頃,西門慶來到,衣帽整齊,四個小廝跟隨,眾人都下席迎接,敍禮讓坐。東家安席,西門慶居首席。一個粉頭,兩個妓女,琵琶箏阮,在席前彈唱。端的說不盡梨園嬌豔,色藝雙全。但見:

「羅衣疊雪,寶髻堆雲。櫻桃口,杏臉桃腮,楊柳腰,蘭心蕙性。歌喉宛囀,聲如枝上流鶯;舞態蹁躚,影似花間鳳轉。腔依古調,音出天然。舞回明月墜秦樓,歌遏行雲遮楚館。高低緊慢按宮商,吐玉噴珠。輕重疾徐依格調,鏗金戛玉。箏排鴈柱聲聲慢,板排紅牙字字新。」

少頃,酒過三巡,歌吟兩套,三個唱的,放下樂器,向前花枝搖颭,繡帶飄颻磕頭。西門慶應呼答應小使玳安,書袋內取三封賞賜,每人二錢,拜謝了下去。因問東家花子虛:「這位姐兒上姓?端的會唱!」東家未及答,在席應伯爵插口道:「大官人多忘事,就不認的了。這擽箏的,是花二哥令翠,构攔後巷吳銀兒;那撥阮的,是朱毛頭的女兒朱愛愛;這彈琵琶的,是二條巷子李三媽的女兒,李桂卿的妹子,小名叫做桂姐。你家中見放着他親姑娘,大官人如何推不認的?」西門慶笑道:「六年不見,就出落得成了人兒了!」落後酒闌,上席來遞酒,這桂姐慇懃勸酒,情話盤桓。西門慶因問:「你三媽、你姐姐桂卿在家做什麼?怎的不來我家走走,看看你姑娘?」桂姐道:「俺媽從去歲不好了一場,至今腿腳半邊通動不的,只扶着人走。俺姐姐桂卿,被淮上一個客人,包了半年,常是接到店裡住,兩三日不放來家。家中好不無人,只靠着我逐日出來供唱,答應這幾個相熟的老爹,好不辛苦。也要往宅裡看看姑娘,白不得個閒。爹許久怎的也不在裡邊走走?放姑娘家去看看俺媽。」這西門慶見他一團和氣,說話兒乖覺伶變,就有幾分留戀之意。說道:「我今日約兩位好朋友送你家去,你意下如何?」桂姐道:「爹休哄我,你肯貴人腳兒踏俺賤地?」西門慶道:「我不哄你。」到是袖中取出汗巾,連挑牙與香盒茶兒,遞與桂姐收了。桂姐道:「多咱去?如今使保兒先家去說一聲,作個預備。」西門慶道:「直待人散,一同起身。」少頃,遞畢酒,約掌燈人散時分,西門慶約下應伯爵、謝希大,也不到家,騾馬同送桂姐,逕進构攔,往李家去。正是:

「錦繡窩中,入手不如撒手美;  紅錦套裡,鑽頭容易出頭難。」

有詞為證:

「陷人坑,土窖般暗開掘;迷魂洞,囚牢般巧砌疊;檢屍場,屠舖般明排列。衢一味死溫存,活打劫。招牌兒大字書者:買俏金,哥哥休扯;纏頭錦,婆婆自接;賣花錢,姐姐不賒。」

西門慶等送桂姐轎子到門首,李桂卿迎門接入堂中,見畢禮數,請老媽出來拜見。不一時,虔婆扶拐而出,半邊胳膊動彈不得。見了西門慶道了萬福,說道:「天麼,天麼!姐夫貴人,那陣風兒刮你到于此處?」西門慶笑道:「一向窮冗,沒曾來得,老媽休怪,休怪!」虔婆便問:「這二位老爹貴姓?」西門慶道:「是我兩個好友,應二哥、謝子純,今日在花家會茶,遇見桂姐,因此同送回來。快看酒來,俺們樂飲三盃。」虔婆讓三位上首坐了。一面點了茶,一面下去打抹春檯,收拾酒菜。少頃,保兒上來放卓兒,掌上燈燭,酒餚羅列,桂姐從新房中打扮出來,旁邊陪坐。真是個風月窩,鶯花寨。免不得姊妹兩個,在旁金樽滿泛,玉阮同調,歌唱遞酒。有詩為證:

「琉璃鍾,琥珀濃,小槽酒滴珍珠紅。烹龍炮鳳玉脂泣,羅幃繡幙圍香風。吹龍笛,擊龜鼓,皓齒歌,細腰舞。況是青春莫虛度,銀缸掩映嬌娥語,酒不到劉伶墳上去!」

當下桂卿姐兒兩個,唱了一套,席上觥籌交錯飲酒。西門慶向桂卿說道:「今日二位在此,久聞桂姐善能禾唱南曲,何不請歌一詞,以奉勸二位一盃兒酒,意下如何?」那應伯爵道:「我等不當起動,洗耳願聽佳音。」那桂姐坐着只是笑,半日不動身。原來西門慶有心要梳攏桂姐,故此發言,先索落他唱。都被院中婆娘見精識精,看破了八九分。李桂卿在旁就先開口說道:「我家桂姐,從小兒養得嬌,自來生得靦腆,不肯對人胡亂便唱。」於是西門慶便叫玳安小廝,書袋內取出五兩一錠銀子來,放在卓上便說道:「這些不當甚麼,權與桂姐為脂粉之需。改日另送幾套織金衣服。」那桂姐連忙起身相謝了,方纔一面令丫鬟收下了,一面放下一張小卓兒,請桂卿下席來唱。當下桂姐不慌不忙,輕扶羅袖,擺動湘裙,袖口邊搭刺着一方銀紅撮穗的落花流水汗巾兒,歌唱一隻駐雲飛:

「舉止從容,壓盡构攔佔上風。行動香風送,頻使人欽重。嗏!玉杵污泥中,豈凡庸。一曲清商,滿座皆驚動。何似襄王一夢中,何似襄王一夢中!」

唱畢,把個西門慶喜懽的沒入腳處。分付玳安回馬家去,晚夕就在李桂卿房裡歇了一宿。緊着西門慶要梳籠這女子,又被應伯爵、謝希大兩個在根前,一力攛掇,就上了道兒。次日使小廝往家去,拏五十兩銀子,段舖內討四套衣裳,要梳籠桂姐。那李嬌兒聽見要梳籠他家中姪女兒,如何不喜?連忙拏了一錠大元寶,付與玳安,拏到院中打頭面,做衣服,定桌席,吹彈歌舞,花攢錦簇,做三日飲喜酒。應伯爵、謝希大,又約會了孫寡嘴、祝日念、常時節,每人出五分銀子人情作賀,都來囋他。鋪的蓋的,俱是西門慶出,每日大酒大肉,在院中頑耍,不在話下。

「舞裙歌板逐時新,  散盡黃金只此身;

寄語富兒休暴殄,  儉如良藥可醫貧。」

畢竟未知後來如何,且聽下回分解:





\end{showcontents}

