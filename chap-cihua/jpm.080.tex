%# -*- coding: utf-8 -*-
%!TEX encoding = UTF-8 Unicode
%!TEX TS-program = xelatex
% vim:ts=4:sw=4
%
% 以上设定默认使用 XeLaTex 编译,并指定 Unicode 编码,供 TeXShop 自动识别

%第八十回 
\chapter{陳經濟竊玉偷香\KG 李嬌兒盜財歸院}


\begin{showcontents}{}



詩曰:

「寺廢僧居少,  橋塌客過稀,

家貧奴婢懶,  官滿吏民欺;

水淺魚難住,  林疎鳥不棲,

世情看冷煖,  人面逐高低。」

上八句詩,單說着這世態炎涼,人心冷煖,可嘆之甚也!西門慶死了首七光景,玉皇廟吳道官受齋在家,攢念二七經不題,都說那日報恩寺朗僧官十六眾僧人做水陸,有喬大戶家上祭。這應伯爵約會了齋祀中幾位朋友,頭一個是應伯爵,第二謝希大,第三個花子油,第四個祝日念,第五孫天化,第六個常時節,第七個白來創,七人坐在一處。伯爵先說道:「大官人沒了,今二七光景。你我相交一場,當也曾吃過他的,也曾用過他的,也曾使過他的,也曾借過他的,也曾嚼他過的。今日他沒了,莫非推不知道?洒土也瞇了後人眼睛兒也;他就到五閻王根前,也不饒你我了。你我如今這等計數,每人各出一錢銀子,七人共湊上七錢。使一錢六分連花兒買上一張卓面,五碗湯飯,五碟菓子;使了一錢一付三牲;使了一錢五分一瓶酒;使了五分一盤冥紙香燭;使了二錢,買一錢軸子,再求水先生作一篇祭文:使一錢二分銀子顧人抬了去,大官人靈前,眾人祭奠了。咱還便益,又討了他值七分銀一條孝絹,挐到家做裙腰子。他莫不白放咱每出來?咱還吃他一陣。到明日出殯,出頭饒飽餐一頓,每人還得他半張靠山卓面,來家與老婆孩子吃着兩三日,省了買燒餅錢,這個好不好?」眾人都道:「哥說的是!」當下每人湊出銀子來,交與伯爵,整理備祭物停當。買了軸子,央門外人水秀才做了祭文。這水秀才平昔知道應伯爵這起人與西門慶,乃小人之朋,於是包含着里面作就一篇祭文。祭軸停當,把祭祀到西門慶前擺下。陳經濟穿孝,在旁還禮。伯爵為首,各人上了香。人人都粗俗,那里曉的其中滋味!澆了奠酒,只顧把祝文來宣念。其文略曰:

「維重和元年,歲戊戍二月戊子朔,越初三日庚寅,侍生應伯爵、謝希大、花子油、祝日念、孫天化、常時節、白來創謹以清酌庶羞之奠,致祭于故錦衣西門大官人之靈曰:維靈生前梗直,秉性堅剛。軟的不怕,硬的不降。常濟人以點水,容人以瀝露,助人精光。囊篋頗厚,氣概軒昂。逄藥而舉,錦陰伏降。錦襠隊中居住,團夭庫裏收藏。有八角而不用撓摑,逄虱蟣而騷庠難當。受恩小子,常在胯下隨幫。也曾在章臺而宿柳,也曾在謝館而猖狂。正宜撑頭活腦,久戰熬場;胡何一疾,不起之殃!見今你便長伸着腳子去了,丟下子如班鳩跌彈,倚靠何方?難上他花之寨,難靠他八字紅墻。再不得同席而偎軟玉,再不得並馬而傍溫香。撇的人垂頭跌腳,閃得人囊溫郎當!今特奠茲白濁,次獻寸觴。靈其不昧,來格來歌,尚享!」

眾人祭畢,陳經濟下來還禮,請去捲棚內,三湯五割 ,管待出門。那日院中李家虔婆,聽見西門慶死了,鋪謀定計,備了一張祭卓,使了李桂卿、李桂姐坐轎子來上紙弔問。月娘不出來,都是李嬌兒,孟玉樓在上房管待。李家桂卿、桂姐悄悄對李嬌兒說:「俺媽說,人已是死了,你我院中人,守不的這樣貞節。自古千里長棚,沒個不散的筵席。教你手里有東西,悄悄教李銘稍了家去防後,你還恁傻!常言道:『楊州雖好,不久戀之家。』不拘多少時,也少不的離他家門。」那李嬌兒聽記在心。不想那日韓道國妻王六兒亦備了張祭卓,喬素打扮,坐轎子來與西門慶燒紙。在靈前擺下祭祀,只顧站着。站了半日,白沒個人兒來陪侍。原來西門慶死了,首七時分,就把王經打發家去不用了。小廝每見王六兒來,都不敢進去說。那來安兒不知就裡,到月娘房里,向月娘說:「韓大嬏來與爹上紙,在前邊站了一日了。大舅使我來對娘說。」這吳月娘心中還氣忿不過,便喝罵道:「怪賊奴才!不與我走,還來甚麼韓大嬸,〈毛皮〉大嬸!賊狗攮的養漢的淫婦,把人家弄的家敗人亡,父南子北,夫逃妻散的,還來上甚麼〈毛皮〉紙!」一頓罵的來安兒摸門不着。來到靈前,吳大舅問道:「對後邊說了不曾?」來安兒嘴谷都着不言語。問了半日,再說:「娘稍出『四馬』兒來了!」這吳大舅連忙進去對月娘說:「姐姐,你怎麼這等的!快休要舒口。自古人惡禮不惡。他男子漢領着咱惹多的本錢,你如何這等待人?好名兒難得,快休如此!你就不出去,教二姐姐、三姐姐好好待他出去,也是一般。做甚麼恁樣的,教人說你不是?」那月娘見他哥這等說,纔不言語了。良久,孟玉樓還了禮,陪他在靈前坐的。只吃一鍾茶,婦人也有些省【月豈】,就坐不住,隨即告辭起身去了。正是:

「誰人汲得西江水,  難洗今朝一面羞!」

那李桂卿、桂姐、吳銀兒都在上房坐著,見月娘罵韓道國老婆淫婦長,淫婦短;砍一枝,損百株,兩個就有些坐不住。未到日落,就要家去。月娘再三留他姐兒兩個:「晚夕夥計每,伴你每看了提偶的,明日去罷。」留了半日,只桂姐、銀姐不去了,只打發他姐姐桂卿家去了。到了晚夕,僧人散了,果然有許多街坊夥計主管,喬大戶,吳大舅、吳二舅、沈姨夫、花子油、應伯爵、謝希大、常時節也有二十余人,叫了一起偶戲,在大捲棚內擺設酒席伴宿。提演的是孫榮、孫華殺狗勸夫戲文。堂客都在靈旁廳內,圍着幃屏,放下簾來,擺放卓席朝外觀看。李銘、吳惠在這里答應,晚夕也不家去了。不一時,眾人都到齊了。祭祀已畢,捲棚內點起燭來,安席坐下。打動鼓樂,戲文上開上開,直搬演到三更天氣,戲文方了。原來陳經濟自從西門慶死後,無一日不和潘金蓮兩個嘲戲。或在靈前溜眼,帳子後調笑。至是趕人散一亂中,堂客都往後邊去了,小廝每都收家活。這金蓮趕眼錯捏了經濟一把,說道:「我兒,你娘今日可成就了你罷!趁大姐在後邊,咱要就往你屋里去罷。」經濟听了,把不的一聲,先往屋里開門去了。婦人黑影里,抽身鑽入他房內。更不答話,解開裙子,仰臥在炕上,雙鳧飛肩,交陳經濟奸耍。正是:

「色膽如天怕甚事,  鴛幃雲雨百軍情。」

「二載相逄,一朝配偶;數年姻眷,一旦和諧。一個柳腰欵擺,一個玉莖忙舒。耳邊訴雨意雲情,枕上說山盟海誓。鶯恣蝶採,婍妮搏弄百千般;狂雨羞雲,嬌媚施逞千萬態。一個低聲不住叫親親,一個摟抱未免呼達達。」

正是:

「得多少柳色乍翻新樣綠,  花容不減舊時紅!」

霎時雲雨了畢,婦人恐怕人來,連忙出房,往後邊去了。到次日,這小夥兒嚐着這個甜頭兒,早辰走到金蓮房來。金蓮還在被窩里未起來,從窗眼里張看,見婦人被擁紅雲,粉腮印玉,說道:「好管庫房的,這咱還不起來?今日喬親家爹來上祭,大娘分付教把昨日擺的李三、黃四家那祭卓,收進來罷。你快些起來,且挐鑰匙出來與我。」婦人連忙教春梅拏鑰匙與經濟。經濟先教春梅樓上開門去了。婦人便從窗眼里遞出舌頭,兩個咂了一回。正是:

「得多少脂香滿口涎空嚥,  甜唾融心溢肺肝。」

有詞為證:

「恨杜鵑聲透珠簾,心似針簽,情似膠粘。我則見笑臉腮窩,愁粉黛瘦顯春纖。寶髻亂雲鬆,翠鈿睡顏酡。玉減紅添,檀口曾沾。到如今唇上猶香,想起來口猶甜。」

良久,春梅樓上開了門,經濟往前邊看搬祭祀去了。不一時,喬大戶家祭來擺下。喬大戶娘子并喬大戶多親眷,靈前祭畢,吳大舅、二舅、甘夥計陪侍,請至捲棚管待。李銘、吳惠彈唱。那日鄭愛月兒家也來上紙弔孝。月娘俱令玉樓打發了孝裙束腰,後邊與堂客一處坐的。鄭愛月兒看見吳銀姐、李桂姐都在這里,便嗔他兩個不對他說:「我若知道爹沒了,有個不來的?你們好人兒,就不會我會兒去!」又見月娘生了孩兒,說道:「娘一喜一憂,惜乎只是爹去世太早了些兒!你老人家有了主兒,也不愁。」月娘俱打發了孝,留坐至晚方散。到二月初三日,西門慶二七,玉皇廟吳道官十六個道眾,在家念經做法事,那日衙門中何千戶作創,約會了劉、薛二內相、周守禦、荊統制、張團練、雲指揮等數員武官,合着上了一壇祭。月娘這里請了喬大戶、吳大舅、應伯爵來陪侍。李銘、吳惠兩個小優兒彈唱,捲棚管待去了。俱不必細說。到晚夕念經送亡,月娘分付把李瓶兒靈床,連影抬出去,一把火焚之,將廂籠都搬到上房內堆放。奶子如意兒并迎春收在後邊答應,把綉春與了李嬌兒房內使喚。將李瓶兒那邊房門,一把鎖鎖了。可怜!正是:

「畫棟雕梁猶未乾,  堂前不見痴心客。」

有詩為證:

「襄王臺下水悠悠,  一種相思兩樣愁;

月色不知人事改,  夜深還到粉墻頭!」

那時李銘日日假以孝堂助忙,暗暗教李嬌兒偷轉東西,與他掖送到家,又來答應。常兩三夜不往家去,只瞞過月娘一人眼目。吳二舅又和李嬌兒舊有首尾,誰敢道個不字。初九日念了三七經,月娘出了暗房。四七就沒曾念經。十二日陳經濟破了土回來,二十日早發引,也有許多冥器紙劄。送殯之人,終不似李瓶兒那時稠密。臨棺材出門,陳經濟摔盆扶柩,也請了報恩寺朗僧官起棺,坐在轎上,捧的高高的,念了幾句偈文,說西門慶一生始末,道得好;

「恭維

故錦衣武略將軍西門大官人之靈:伏以人生在世,如電光易滅,石火難消。落花無返樹之期,逝水絕歸源之路。你畫堂綉閣,命盡有若風燈;極品高官,緣絕猶如作夢。黃金白玉,空為禍患之資;紅粉輕裘,總是塵勞之費。妻奴無百載之歡,黑暗有千重之苦。一朝枕上,命掩黃泉,空榜揚虛假之名,黃土埋不堅之骨。田園百頃,其中被兒女爭奪;綾錦千廂,死後無寸絲之分。風火散時無老少,溪山磨盡幾英雄。苦苦苦,氣化清風形歸土。三寸氣斷去弗〈廴回〉,改頭換面無遍數。」詩曰:

「人生最苦是無常,  個個臨終手腳忙,     地水火風相逼迫,  精神魂魄各飛揚;

生前不解尋活路,  死後知他去那廂,

一切萬般將不去,  赤條條的見閻王。」

朗僧官念畢偈文,陳經濟摔破紙盆,棺材起身,合家大小孝眷,放聲號哭動天。吳月娘坐魂轎,後面眾堂客上轎,都圍隨材走,逕出南門外五里原祖塋安厝。陳經濟備了一疋尺頭,請雲指揮點了神主,陰陽徐先生下了葬。眾孝眷掩土畢,山頭祭卓,可怜通不上幾家。只是吳大舅、喬大戶、何千戶、沈姨夫、韓姨夫與眾夥計五六處而已。吳道官還留下十二眾道童回靈,安於上房明間正寢。大小安靈,陰陽灑掃已畢,打發眾親戚出門。吳月娘等,不免伴夫靈守孝。一日煖了墓回來,答應班上排軍節級,各都告辭回衙門去了。西門慶五七,月娘請了薛姑子、王姑子、大師父、十二眾尼僧,在家誦經禮懺,超度夫主生天。吳大妗子并吳舜臣媳婦,都在家中相伴。原來出殯之時,李桂卿、桂姐在山頭,悄悄對李嬌兒如此這般:「媽說你沒量。你手中沒甚細軟東西?不消只顧在他家了。你又沒兒女,守甚麼?教你一場嚷亂,登開了罷。昨日應二哥來說,如今大街坊張二官府,要破五百兩金銀,娶你做二房娘子,當家理紀。你那里便圖出身,你在這里守到老死,也不怎麼!你我院中人家,棄舊迎新為本,趨炎附勢為強,不可錯過了時光!」這李嬌兒听記在心,過了西門慶五七之後,因風吹火,用力不多。不想潘金蓮對孫雪娥說:「出殯那日,在坟上看見李嬌兒與吳二舅,在花園小房內兩個說話來。春梅孝堂中又親眼看見李嬌兒帳子後,遞了一包東西與李銘〈扌塞〉在腰里,轉了家去。」嚷的月娘知道,把吳二舅罵了一頓,趕去鋪子里做買賣,再不許進後邊來。分付門上平安,不許李銘來往。這花娘惱羞變成怒,正尋不着這箇由頭兒哩!一日,因月娘在上房和大妗子吃茶,請孟玉樓,不請他,就惱了,與月娘兩箇大嚷大鬧,拍着西門慶靈床子,哭哭啼啼,叫叫嚎嚎,到半夜三更,在房中要行上弔。丫鬟來報與月娘。月娘慌了,與大妗子計議,請將李家虔婆來,要打發他歸院。虔婆生怕留下他衣服頭面,說了幾句言語:「我家人在你這里,做小伏低缸受氣,好容易就開交了罷?須得幾十兩遮羞錢!」吳大舅居着官,又不敢張主。相講了半日,教月娘把他房中衣服首飾,廂籠床帳家活,盡與他,打發出門。只不與他元宵、綉春兩個丫鬟去。李嬌兒一心要這兩個丫頭,月娘生死不與他,說道:「你倒好買良為娼!」一句慌了鴇子,就不敢開言,變做笑吟吟臉兒,拜辭了月娘,李嬌兒坐轎子抬的往家去了。

看官听說: 院中唱的,以賣俏為活計,將脂粉作生涯。早辰張風流,晚些李浪子。前門進老子,後門接兒子。棄舊迎新,見錢眼開,自然之理!未到家中,撾打揪撏,燃香燒剪,走死哭嫁,娶到家,改志從良。饒君千般貼戀,萬種牢籠,還銷不住他心猿意馬。不是活時偷食抹嘴,就是死後嚷鬧離門。不拘幾時,還吃舊鍋粥去了!正是:

「蛇入洞中曲性在,  鳥出籠輕便飛騰。」

有詩為證:

「堪嘆烟花不久長,  洞房夜夜換新郎,

兩隻玉腕千人枕,  一點朱唇萬客嘗;

造就百般嬌艷態,  生成一片假心腸,

饒君總有牢籠計,  難保臨時思故鄉。」

月娘於是打發李嬌兒出門,大哭了一場,眾人都在旁勸解。潘金蓮道:「姐姐罷,休煩惱了!常言道:『娶淫婦,養海青;食水不到想海東!』這個都是他當初幹的營生,今日教大姐姐這等惹氣!」家中正亂着,忽有平兒來報:「巡鹽蔡老爹來了,在廳上坐着哩。我說家老爹沒了。他問沒了幾時了,我回正月二十一日病故,到今過了五七。他問有靈沒靈?我回有靈在後邊供養着哩。他要來靈前拜拜,我來對娘說。」月娘分付:「教你姐夫出去見他。」不一時陳經濟穿上孝衣,出去拜見了蔡御史。良久後邊收拾停當,請蔡御史進來西門慶靈前參拜了。月娘穿着一身重孝,出來回禮。再不教一言,就讓月娘:「夫人請回房。」因問經濟說道:「我昔時曾在府相擾,今差滿回京去,敬來拜謝拜謝,不期作了人故!」便問:「甚麼病來?」陳經濟道:「是個痰火之疾。」蔡御史道:「可傷,可傷!」即喚家人上來,取出兩疋杭州絹,一雙羢襪,四尾白鮝,四罐蜜餞,說道:「這些微禮,權作奠儀罷!」又挐出五十兩一封銀子來:「這個是我向日曾貣過老先生些厚惠,今積了些俸資奉償,以全始終之交。」分付:「大官,交進房去。」經濟道:「老爹忒多計較了!」月娘說:「請老爹前廳坐。」蔡御史道:「也不消坐了。拏茶來,我吃一鍾就是了。」左右須臾拿茶上來,蔡御史吃了,揚長起身上轎去了。月娘得了這五十兩銀子,心中又是那歡喜,又是那慘切!想有他在時,似這樣官員來到,肯空放去了?又不知吃酒到多咱晚!今日他伸着腳子,空有家私,眼看着就無人陪侍。正是:

「人得交游是風月,  天開圖畫即江山。」

有詩為證:

「靜掩重門春日長,  為誰展轉怨流光;

更憐無爪秋波眼,  默地懷人淚兩行。」

話說李嬌兒到家,應伯爵打听得知,報與張二官兒。就拏着五兩銀子,來請他歇了一夜。原來張二官小西門慶一歲,屬兔的,三十二歲了。李嬌兒三十四歲。虔婆瞞了六歲,只說二十八歲,教伯爵應瞞着。使了三百兩銀子,娶到家中,做了二房娘子。祝日念、孫寡嘴依舊領着王三官兒還來李家行走,與桂姐打熱,不在話下。伯爵、李三、黃四借了徐內相五千兩銀子,張二官出了五千兩,做了東平府古器這批錢糧,逐日寶鞍大馬,在院中搖擺。張二官見西門慶死了,又打點了千兩金銀,上東京尋了樞密鄭皇親人情,對堂上朱大尉說,要討刑所西門慶這個缺,家中收拾買花園蓋房子。應伯爵無日不在他那邊趨奉,把西門慶家中大小之事,盡告訴與他。說:「他家中還有第五個娘子潘金蓮,排行六姐,生的極標致,上畫兒般人材!詩詞歌賦,諸子百家,折牌道字,雙陸象棋,無不通曉!又會識字,一筆好寫,彈一手好琵琶。今年不上三十歲,比唱的還喬!」說的這張二官心中火動,巴不得就要了他。便問道:「莫非是當初的賣炊餅武大郎的妻子麼?」伯爵道:「就是他。被他占來家中,今也有五、六年光景。不知他嫁人不嫁?」張二官道:「累你打听着,待有嫁人的聲口,你來對我說,等我娶了罷。」伯爵道:「我身子里有個人在他家做家人,名來爵兒,等我對他說,若有出嫁聲口,就來報你知道。難得你若娶過教這個人來家,也強如娶過唱的!當時有西門慶在,為娶他,也費了許多心。大抵物各有主,也說不的。只好有福的匹配。你如今有了這般勢耀,不得此女貌,同享榮華,枉自有許多富貴!我只叫來爵兒密密打聽,但有嫁人的風縫兒,憑我甜言美語,打動春心;你都用幾百兩銀子,娶到家中,儘你受用便了。」看官聽說:但凡世上幫閒子弟,極是勢利小人。見他家豪富,希圖衣食,便竭力承奉,稱功誦德;或肯撒漫使用,說是疎財仗義慷慨丈夫。脅肩諂笑,獻子出妻,無所不至。一見那門庭冷落,便唇譏腹誹,說他外務,不肯成家立業;祖宗不肖,有此敗兒!就是平日深恩,視如陌路。當初西門慶待應伯爵如膠似漆,賽過同胞弟兄。那一日不吃他的,穿他的,受用他的?身死未幾,骨肉尚熱,便做出許多不義之事!正是:

「畫虎畫皮難畫骨,  知人知面不知心!」

有詩為證:

「昔年意氣似金蘭,  百計趨承不等閑;

今日西門身死後,  紛紛謀妾伴入眠。」

畢竟未知後來如何,且聽下回分解:





\end{showcontents}


