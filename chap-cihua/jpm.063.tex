%# -*- coding: utf-8 -*-
%!TEX encoding = UTF-8 Unicode
%!TEX TS-program = xelatex
% vim:ts=4:sw=4
%
% 以上设定默认使用 XeLaTex 编译,并指定 Unicode 编码,供 TeXShop 自动识别

%第六十三回 
\chapter{親朋祭奠開筵宴\KG 西門慶觀戲感李瓶}


\begin{showcontents}{}



「十二瑤臺七寶欄,  瓊花落後再開難,

龍鬚煮藥醫無效,  熊胆為丸晒未乾;

蓉帳夜愁紅燭冷,  紙窗秋暮翠衾寒,

應憐失伴孤飛雁,  霜落風高一影單。」

話說當日應伯爵勸解了西門慶一回,拭淚而止。令小廝後邊看飯去了。不一時吳大舅,吳二舅都到了。靈前行畢禮,與西門慶作揖,道及煩惱之意。請至廂房中,與眾人同坐。玳安走至後邊向月娘說:「如何?我說娘每不信。怎的應二爹來了,一席話說的爹就吃飯了。」金蓮道:「你這賊積年久慣的囚根子!鎮日在外邊替他做牽頭,有個拿不住他性兒的。」玳安道:「從小兒答應主子,不知心腹?」月娘問道:「那幾個在廂房子裡坐著陪他吃飯?」玳安道:「大舅、二舅剛纔來,和溫師父,連應二爹、謝爹、韓夥計、姐夫共爹,八位人哩。」月娘道:「請你姐夫來後邊吃罷了,也擠在上頭?」玳安道:「姐夫坐下了。」月娘分付:「你和小廝往廚房裡拿飯去。你另拿甌兒,拿粥與他吃。清早辰不吃飯。」玳安道:「再有誰?止我在家。都使出報喪、燒布、買東西。王經又使他往張親家爹那裡借雲板去了。」月娘道:「書童那奴才,和他拿去是的。怕打了他紗絹展腳兒!」玳安道:「書童和畫童兩個在靈前,一個打磬,一個伺候焚香燒紙哩。春鴻,爹又使他跟賁四換絹去了;嫌絹不好,要換六錢一疋的絹破孝。」月娘道:「論起來,五錢銀子的也罷。又巴巴兒換去。」又道:「你叫下畫童兒那小奴才,和他快拿去,只顧還挨磨甚麼?」玳安于是和畫童兩個大盤大碗拿到前邊,安放八仙卓席。眾人正吃著飯,只見平安拿進手本來稟:「衙門中夏老爹差寫字的送了三斑軍衛來這裡答應,討回帖。」西門慶看了放下,分付:「討三錢銀子賞他,寫期服生雙回帖兒回你夏老爹,多謝了。」一面吃畢飯,收了家火。只見來保請的畫師韓先生來到。西門慶與他行畢禮,說道:「煩先生揭白傳個神子兒。」那韓先生道:「小人理會得了。」吳大舅道:「動手遲了些,倒只怕面容改了。」韓先生道:「也不妨,就是揭白也傳得。」正吃茶畢,忽見平安來報:「門外花大舅來了。」西門慶陪花子油靈前哭涕了一回,見畢禮數,與眾人一處。因問:「甚麼時候?」西門慶道:「正丑時斷氣,臨死還伶伶俐俐說話兒。剛睡下,丫頭起來瞧,就沒了氣兒。」因見韓先生傍邊小童拿着屏插,袖中取出抹筆顏色來,花子油道:「姐夫如今要傳個神子?」西門慶道:「我心裡疼他,少不的留了個影像兒,早晚看着題念他題兒。」一面分付後邊堂客躲開,掀起帳子,領韓先生和花大舅眾人到根前。這韓先生用手揭起千秋旛,用五輪寶翫着兩點神水,打一觀看。見李瓶兒勒着鴉青手帕,雖故久病,其顏色如生,姿容不改,黃懨懨的,嘴唇兒紅潤可愛。那西門慶由不的掩淚而哭。當下來保與琴童在傍捧着屏插顏色,韓先生一見就知道了。眾人圍着他求畫。應伯爵便道:「先生此是病容。平昔好時,比此還生的面容飽滿,姿容秀麗。」韓先生道:「不須尊長分付,小人知道。不敢就問老爹,此位老夫人前者五月初一日,曾在岳廟裡燒香,親見一面,可是否?」西門慶道:「正是。那時還好哩。先生你用心想着,傳畫一軸大影,一軸半身,靈前供養。我送先生一疋段子上蓋,十兩銀子。」韓先生道:「老爹分付,小人無不用心。」須臾描染出個半身來,端的玉貌幽花秀麗,肌膚嫩玉生香。拿與眾人瞧,就是一幅美人圖兒。西門慶看了,分付玳安:「拿到後邊與你娘每瞧瞧去,看好不好?有那些兒不是,說來好改。」這玳安拿到後邊向月娘道:「爹說交娘每瞧瞧六娘這影,看畫的如何?那些兒不像,說出去教韓先生好改。」月娘道:「成精鼓搗,人也不知死到那裡去了,又描起影來了!畫的那些兒像?」潘金蓮接過來道:「那個是他的兒女,畫下影,傳下神來,好替他磕頭禮拜。到明日六個老婆死了,畫下六個影纔好。」孟玉樓和李嬌兒拿過來觀看,說道:「大娘你來看,李大姐這影倒像似好時那等模樣,打扮的鮮鮮兒,只是嘴唇略匾了些兒。」月娘道:「這左邊額頭略低了些兒。他的眉角,比這眉角兒還灣些。虧這漢子揭白怎的畫來。」玳安道:「他在廟上曾見過六娘一面,剛纔想着就畫到這等模樣。」少頃只見王經進來說道:「娘每看了快教拿出去。喬親家爹來了,等喬親家爹瞧哩。」玳安走到前邊,分付韓先生道:「這裡邊說來,嘴唇略匾了些,左額角稍低,眉還略放灣着些兒。」韓先生道:「這個不打緊。」隨即取描筆改正了,呈與喬爹瞧。喬大戶道:「親家母這幅尊像是畫得通,只是少了口氣兒!」西門慶滿心歡喜,一面遞了三鍾酒與韓先生,管待了酒飯;江漆盤捧出一疋尺頭,十兩白金與韓先生,教他先攢造出半身來,就要挂;大影不誤出殯就是了。俱要用大青大綠,珠翠圍髮冠,大紅通神五彩遍地金袍兒,百花裙,衢花綾裱,象牙軸頭。韓先生道:「不必分付,小人知道。」領了銀子,教小童拿着插屏,拜辭出門。喬大戶與眾人又看了一回做成的棺木,便道:「親家母今日小殮罷了?」西門慶道:「如今仵作行人來就小殮,大殮還等到三日。」喬大戶吃畢茶,就告辭起身去了。不一時仵作行人來伺候,紙劄打捲,鋪下衣衾。西門慶要親與他開光明,強著陳經濟做孝子,與他抿了目。西門慶旋尋出一顆胡珠,安放在他口裡。登時小殮停當,照前停放端正,放下帳子,合家大小哭了一場。來興又早冥衣舖裡,做了四座堆金瀝粉侍奉的捧盆巾盥櫛毛女兒,都是珠子纓絡兒,銀廂墜兒,似真的色綾衣服,一邊兩座擺下。靈前供養彝爐、商瓶、燭臺、香盒、教錫匠打造停當,擺在卓上,耀日爭輝。又兌了十兩銀子,教銀匠打了三付銀爵盞。正在廂房中與應伯爵定管喪禮簿籍,先兌了五百兩銀子,一百弔錢來,委付與韓夥計管帳。賁四與來興兒專管大小買辦,兼管外廚房。應伯爵、謝希大,、溫秀才、甘夥計四人,輪番陪侍往來弔客。崔本專管付孝帳。來保管外庫房,王經管酒房,春鴻與畫童專管靈前伺候。平安逐日與四名排軍,單管人來打雲板,捧香紙。又是一個寫字帶領四名排軍,在大門首記門簿;值念經日期,打傘,相搭挑旛幢,無事把門。都派委已定,寫了告示,貼在影壁上,各遵守去訖。只見皇庄上薛內相差人送了六十根杉條、三十條毛竹、三百領蘆蓆、一百條麻繩,拿帖兒與西門慶瞧。連忙賞了來人五錢銀子,拿期服生回帖兒,打發去了。分付搭採匠,把棚起眷搭大着些,留兩個門走。把影壁夾在中間。前廚房內還搭三間罩棚,大門首紮七間榜棚。請報恩寺十二眾僧人,先念倒頭經。每日兩個茶酒在茶坊內伺候茶水。外廚房兩名廚役,答應各項飯食。花大舅、吳二舅坐了一回,起身去了。西門慶交溫秀才起孝帖兒,要開刊去,令寫:「荊婦奄逝。」悄悄拿與應伯爵看,伯爵道:「這個理上說不通,見有如今吳家嫂子在正室,如何使得?這一個字出去,不被人議論,就是吳大哥心內也不自在。等我慢慢再與他講,你且休要寫着。」陪坐至晚,各散歸家去了。西門慶晚夕也不進後邊去,就在李瓶兒靈傍邊裝起一張涼牀,拿圍屏圍着,鋪陳停當,獨自宿歇。有春鴻、書童兒近前伏侍。天明便往月娘房裡梳洗,裁縫做白唐巾、孝冠、孝衣、白羢襪、白履鞋,絰帶隨身。第二日清辰,夏提刑就來探喪弔問,慰其節哀。西門慶還禮畢,溫秀才相陪,待茶而去。到門首分付寫字的:「好生在此答應。查有不到的排軍,呈來衙門內懲治。」說畢,騎馬往衙門中去了。西門慶令溫秀才發帖兒,差人請各親眷,三日做齋誦經,早來赴會。後晌鋪排來收拾道場,懸掛佛像,不必細說。那日院中吳銀兒打聽得知,坐轎子來靈前哭泣上紙。到後邊,月娘相接引去,吳銀兒與月娘磕頭,哭道:「六娘沒了,我通一字不知。就沒個人兒和我說聲兒,可憐傷感人也!」孟玉樓道:「你是他乾女兒,他不好了這些時,你就不來看他看兒?」吳銀兒道:「好三娘,我但知道,有個不來看的?說句假,就死了。委實不知道。」月娘道:「你不來看你娘,他還掛牽着你,留了件東西兒與你做一念兒,我替你收着哩!」因令小玉:「你取出來與銀姐兒看。」那小玉走到裡間,取出包袱,內包着一套段子衣服、兩根金頭簪兒,一件金花兒。把吳銀兒哭的淚人也相似,說道:「我早知他老人家不好,也來伏侍兩日兒!」說着,一面拜謝了月娘。月月待茶與他吃,留他過了三日去。到三日和尚打起磬子,揚旛,道場誦經,挑出紙錢去。合家大小,都披麻帶孝。陳經濟穿重孝絰巾,佛前拜禮。街坊鄰舍,親朋官長,來弔問上紙祭奠者,不計其數。陰陽徐先生早來伺侯大殮。祭告已畢,擡屍入棺。西門慶交吳月娘又尋出他四套上色衣服來裝在棺內。四角安放了四錠小銀子兒依着。花子油說:「姐夫,倒不消安他在裡面。金銀日久,定要出世,倒非久遠之居。」西門慶不肯,安放如故。放下一七星板,閣上紫蓋。仵作四面用長命丁,一齊釘起來,一家大小放聲號哭。西門慶亦哭的呆了,口口聲聲哭叫:「我的年小的姐姐,再不得見你了!」良久哭畢,管待徐先生齋饌,打發去了。酒花米貼「神燈安真」四個大字在靈前。親朋夥計人等,都是巾帶孝服。行香之時,門首一片皆白。溫秀才舉薦北邊杜中書來題名旌。名子春,號雲野,原侍真宗寧和殿,今坐閑在家。西門慶備金幣請來,在捲棚內備菓盒,西門慶親遞三杯酒。應伯爵與溫秀才相陪,鋪大紅官紵題旌。西門慶要寫:「詔封錦衣西門慶恭人李氏柩」十一字。伯爵再三不肯,說:「見有正室夫人在,如何使得?」杜中書道說:「曾生過子,於禮也無礙。」講了半日,去了「恭」字,改了「室人」。溫秀才道:「恭人係命婦,有爵。室人乃室內之人,只是個渾然通常之稱。」于是用白粉題畢,「詔封」二字貼了金,懸於靈前;又題了神主。叩謝杜中書,管待酒饌,拜辭而去。那日喬大戶、吳大舅、花大舅、門外韓姨夫、沈姨夫各家都是三牲祭卓來燒紙。喬大戶娘子并吳大妗子、二妗子、花大妗子,坐轎子來弔喪,祭祀哭泣。月娘等皆孝髻、頭鬚、繫腰、麻在孝裙出來回禮舉哀,讓後邊待茶擺齋。惟花大妗子與花大舅,便是重孝直身道袍兒,餘者都是輕孝。那日院中李桂姐打聽得知,坐轎子也來上紙。看見吳銀兒在這裡,說道:「你幾時來的?怎的也不會我會兒?好人來,原來只顧你。」吳銀兒道:「我也不知道娘沒了,早知是也來看看兒。」月娘後邊管待,俱不必細說。須臾過了,看看到首七。正是報恩寺十六眾上僧,黃僧官為首座,引領做水陸道場,誦法華經,拜三昧水懺。親朋夥計,無不畢集。那日玉皇廟吳道官,來上紙弔孝,攬二七經。西門慶留在捲棚內,眾人吃齋。忽見小廝來報:「韓先生送半身影來。」眾人觀看,但見:頭戴金翠圍冠雙鳳珠子挑牌,大紅粧花袍兒,白馥馥臉兒,儼然如生時一般。西門慶見了,滿心歡喜,懸掛像材頭上。眾人無不誇獎,只少口氣兒。一面讓捲棚吃齋,囑付大影比長,還要加工夫些。韓先生道:「小人隨筆潤色,豈敢粗心。」西門慶厚賞而去。

午間喬大戶那邊來上祭,豬羊祭品,吃看卓面,高頂簇盤,五老錠勝,方糖樹果,金碟湯飯,五牲看碗,金銀山,段帛綵繒,冥紙炷香,共約五十餘擡,地弔高撬,鑼鼓細樂,吹打纓絡,打挑喧闐而至。官堂客約許多人,陰陽生讀祝。西門慶與陳經濟穿孝衣,在靈前還禮。應伯爵、謝希大與溫秀才、甘夥計等,迎待賓客,那日喬大戶邀了尚舉人、朱堂官、吳大舅、劉學官、花千戶、段親家七八位親朋,各在靈前上香。三獻已畢,俱跪聽讀祝文,曰:

「維政和七年,歲次丁酉,九月庚申朔,越二十二日辛巳,眷生喬洪等,謹以剛鬣柔毛庶羞之奠,致祭于故親家母西門孺人李氏之靈曰:嗚呼,孺人之性,寬裕溫良,治家勤儉,御眾慈祥。克全婦道,譽動鄉邦。閨閫之秀,蘭蕙之芳。夙配君子,效聘鸞凰。撫字子性,以義以方。效顰大德,以柔以良。施懿範於家室,悚和粹於娣障。藍玉已種,浦珠已光。正期諧琴瑟於有永,享彌壽於無疆。胡為一疾,夢斷黃梁,善人之歿,孰不哀傷!弱女襁褓,沐愛姻嬙。不期中道,天不從願,鴛伴失行,恨隔幽冥,莫覩行藏。悠悠情誼,寓此一觴。靈其有知,來格來歆。尚饗!」

官客祭畢,回禮畢,讓捲棚內,自有卓席管待,不在話下。然後喬大戶娘子、崔親家母、朱堂官娘子、尚舉人娘子、段大姐眾堂家女眷祭奠地弔,鑼鼓靈前,弔鬼判隊舞,戟將響樂。吳月娘陪着哭畢,請去後邊待茶設席,三湯五割 ,俱不必細說。西門慶正在捲棚內陪人吃酒,忽聽前邊打的雲板響,答應的荒荒張張進來稟報:「本府胡爺上紙來了,在門首下轎子。」慌的西門慶連忙穿孝衣,靈前伺候。即使溫秀才衣巾素服出迎,前廳伺候換衣裳。左右先捧進香紙,然後胡府尹素服金帶,纔進來,許多官吏圍隨扶衣搊帶,奔走不暇。于是靈前春鴻跪著,捧的香高高的,上了香,展拜兩禮,西門慶便道:「老先生請起,多有勞動!」連忙下來回了禮,胡府尹道:「弔遲、弔遲!令夫人幾時沒了?學生昨日纔知。」西門慶道:「不想粗室一疾不救,辱承老先生枉弔!」溫秀才在傍作揖畢,與西門慶兩邊列坐。待茶一杯,胡府尹起身。溫秀才送出大門,上轎而去。上祭人吃至後晌時分方散。到第二日,院中鄭愛月兒家來上紙。愛月兒下了轎子,穿著白雲絹對衿襖兒,藍羅裙子,頭上勒著珠子箍子,白挑線汗巾子,進至靈前燒了紙。月娘見他抬了八盤餅饊,三牲湯飯來祭奠,連忙討了一疋整絹孝裙與他。吳銀兒與李桂姐都是三錢奠儀,告西門慶說。西門慶道:「值甚麼,每人都與他一疋整絹頭鬚繫腰,後邊房兒裡擺茶管待過夜。」晚夕親朋夥計來伴宿;叫了一起海鹽子弟搬演戲文。李銘、吳惠、鄭奉、鄭春都在這裡答應。晚夕西門慶在大棚內放十五張卓席,為首的就是喬大戶、吳大舅、吳二舅、花大舅、沈姨夫、韓姨夫、倪秀才、溫秀才、任醫官、李智、黃四、應伯爵、謝希太、祝日念、孫寡嘴、白來創、常時節、傅日新、韓道國、甘出身、賁地傳、吳舜臣兩個外甥,還有街坊六七位人,都是十菜五菓開卓兒,點起十數枝高檠大燭來。廳上垂下簾,堂客便在靈前圍著圍屏,放卓席,往外觀戲。當時眾人祭奠畢,西門慶與經濟回畢禮,安席上坐。下邊戲子打動鑼鼓,搬演的是韋臯、玉簫女兩世姻緣玉環記。西門慶分派四名排軍,單管下邊拿盤。琴童、棋童、畫童、來安四個單管下菓兒。李銘、吳惠、鄭奉、鄭春四個小優兒席上斟酒。不一時弔場,生扮韋臯,唱了一回下去。貼旦扮玉簫,又唱了一回下去。廚房裡廚役上湯飯割鵝,應伯爵因使向西門慶說:「我聞的院裡姐兒三個在這裡,何不請出來與喬老親家、老舅席上遞杯酒兒?他到是會看戲,又倒便益了他。」西門慶便使玳安進入說去,請他姐兒三個出來。喬大戶道:「這個都不當,他來弔喪,如何叫他遞起酒來?」伯爵道:「老親家你不知。相這樣小淫婦兒,別要閒著他。快與我牽出來,你說應二爹說,六娘沒了,只當行孝順,也該與俺每人遞杯酒兒。」玳安進去半日說:「聽見應二爹在坐,都不出來哩。」伯爵道:「既恁說,我去罷。」走了兩步,又回坐下。西門慶笑道:「你怎的又回了?」伯爵道:「我有心待要扯那三個小淫婦出來,等我罵兩句,出了我氣我纔去。」落後又使了玳安請了一遍,那三個纔慢條條出來,都一色穿著白綾對衿襖兒,藍段裙子,向席上不端不正拜了拜兒,笑嘻嘻立在傍邊。應伯爵道:「俺每在這裡,你如何只顧推三阻四,不肯出來?」那三個也不答應,向上邊遞了回酒,號設一席坐著。下邊鼓樂響動,關目上來,生扮韋臯,淨扮包知木,同到抅欄裡玉簫家來。那媽兒出來迎接。包知木道:「你去叫那姐兒出來。」媽云:「包官人,你好不著人,俺女兒等閒不便出來,說不的一個請字兒?你如何說叫他出來?」那李桂姐向席上笑道:「這個姓包的就和應花子一般,就是個不知趣的蹇味兒!」伯爵道:「小淫婦!我不知趣,你家媽兒喜歡我?」桂姐道:「他喜歡你?過一邊兒。」西門慶道:「且看戲罷,且說甚麼!再言語,罰一大杯酒。」那伯爵纔不言語了。那戲子又做了一回,並下。這裡廳內左邊弔簾子看戲的,大妗子、二妗子、楊姑娘、潘媽媽、吳大姨、孟大姨、吳舜臣媳婦、鄭三姐、段大姐並本家月娘眾娣妹,右邊弔簾子戲的,是春梅、玉簫、蘭香、迎春、小玉都擠著觀看。那打茶的鄭紀,正拿著一邊菓仁泡茶,從簾下頭過。被春梅叫住問道:「拿茶與誰吃?」鄭紀道:「那邊大妗子娘每要吃。」這春梅取一盞在手。不想小玉聽見下邊扮戲的旦兒名子也叫玉簫,便把玉簫拉著說道:「淫婦,你的孤老漢子來了,鴇子叫你接客哩。你還不出去!」使力往下一推,直推出簾子外。春梅手裡拿著茶,推潑一身。罵玉簫:「怪淫婦,不知甚麼張致,都頑的這等,把人的茶都推潑了。早是沒曾打碎盞兒。」西門慶聽得,使下來安兒來問:「誰在裏面暄嚷?」春梅坐在椅上道:「你去就說玉簫浪淫婦,面見了漢子,這等浪想。」那西門慶問了一回,亂著席上遞酒就罷了。月娘便走過那邊數落小玉:「你出來這一日,也往屋裡瞧瞧去。都在這裡,屋裡有誰?」小玉道:「大姐剛纔後邊去的。兩位師父也在這裡坐著。」月娘道:「教你們賊狗胎在這裡看看,就恁惹是招非的!」春梅見月娘過來,連忙立起身來說道:「娘,你問他,都一個個只像有風出來,狂的通沒些成色兒,嘻嘻哈哈,也不顧人看見。」那月娘數落了一回,仍過那邊去了。那時喬大戶與倪秀才先起身去了。沈姨夫與任醫官、韓姨夫也要起身,被應伯爵攔住道:「東家,你也說聲兒。俺們倒是朋友,不敢散。一個親家都要去?沈姨夫又不隔門,韓姨夫與任大人、花大舅都在門裡,這咱纔三更天氣,門也還未開,慌的甚麼?都來大坐回兒,左右關目還未了哩。」西門慶又令小廝提四罈麻姑酒 ,放在面前說:「列位,只了此四罈酒,我也不留了。」因拿大賞鍾,放在吳大舅面前,說道:「那位離席破坐說起身者,任大人舉罰。」于是眾人又復坐下了。西門慶令書童催促子弟,快弔關目上來,分付揀省熱鬧處唱罷。須臾打動鼓板,扮末的上來。請問西門慶:「小的寄真容的那一摺,唱罷?」西門慶道:「我不管你,只要熱鬧。」貼旦扮玉簫唱了一回。西門慶看唱到「今生難會,固此上寄丹青」一句,忽想起李瓶兒病時模樣,不覺心中感觸起來,止不住眼中淚落,袖中不住取汗巾兒擦拭。又早被潘金蓮在簾內冷眼看見,指與月娘瞧,說道:「大娘你看見他,好個沒來頭的行貨子。如何吃著酒,看見扮戲的哭起來!孟玉樓道:「你聰明一場,這些兒就不知道。樂有悲歡離合,想必看見那一段兒觸著他心,他覷物思人,見鞍思馬,纔落淚來。」金蓮道:「我不信。打啖的弔眼淚,替古人躭憂。這個都是虛,他若唱的我淚出來,我纔算他好戲子。」月娘道:「六姐,悄悄兒咱每聽罷。」玉樓因向大娘子道:「俺六姐不知怎的,只好快說嘴。」那戲子又做了一回,約有五更時分,眾人齊起身。西門慶拿大杯攔門遞酒,款留不住,俱送出門。攪收了家火,留下戲廂,明日有劉公公、薛公公來祭奠,白日坐,還做一日。眾戲子答應,管待了酒飯,歸下處歇去了。李銘等四個亦歸家,不題。西門慶見天色已將曉,就歸後邊歇息去了。正是:

「待多少紅日映窗寒色淺,  淡烟籠竹曙光微。」

畢竟後來如何,且聽下回分解:



\end{showcontents}


