%# -*- coding: utf-8 -*-
% !TeX encoding = UTF-8 Unicode
% !TeX spellcheck = en_US
% !TeX TS-program = xelatex
%~ \XeTeXinputencoding "UTF-8"
% vim:ts=4:sw=4
%
% 以上設定默認使用 XeLaTex 編譯,並指定 Unicode 編碼,供 TeXShop 自動識別

%第十四回
\chapter{花子虛著氣喪身\KG 李瓶兒送奸赴會}

\begin{showcontents}{}



「眼意心期未即休,不堪拈弄玉搔頭,

春回笑臉花含媚,淺感蛾媚柳帶愁;

粉暈桃腮思伉儷,寒生蘭室盼綢繆,

何如得遂相如志,不讓文君詠白頭。」

話說一日吳月娘心中不快,吳大娘子來看。月娘留他住兩日。正陪着在房中坐的,忽見小廝玳安抱進毡包來,說:「爹來家了。」吳大妗子便往李嬌兒房裡去了。少頃,西門慶進來,脫了衣服坐下。小玉拿茶來也不吃。月娘見他面帶幾分憂色,便問:「你今日會茶來家忒早。」西門慶道:「今該常時節會,他家沒地方,請了俺們在門外五里原永福寺去耍子。有花大哥邀了應二哥,俺們四五個往院裡鄭愛香兒家吃酒。正吃在熱鬧處,忽見幾個做公的進來,不由分說,把花二哥拿的去了,把眾人諕的吃了一驚。我便走到李桂姐家躲了半日。不放心,使人打聽,原來是花二哥內臣家房族中花大、花三、花四告家財,在東京開封府遞了狀子。批下來着落本縣拿人。俺每纔放心,各人散歸家來。」月娘聞言便道:「正該!鎮日跟着這夥人喬神道,想着個家?只在外邊胡撞。今日只當丟出事來,纔是個了手。你如今還不心死,到明日不吃人爭鋒廝打,群到那裡,打個爛羊頭,你肯斷絕了這條路兒。正經家裡老婆,好言語說着你肯聽?只是院裡淫婦在你跟前說句話兒,你到着人個驢耳朵聽他。正是:

「人家說著耳邊風,  外人說著金字經。」

西門慶笑道:「誰人敢七個頭八個膽打我?」月娘道:「你這行貨子,只好家裡嘴頭子罷了。若上場兒,諕的看出那嘴舌來了。」正說着,只見玳安走來,說:「隔壁花二娘家使了天福兒來,請爹過那邊去說話。」這西門慶得不的一聲兒,趔趄腳兒就往外走。月娘道:「明日沒的教人扯你把?」西門慶道:「切鄰間不妨事。我去到那裡看他有甚麼話說。」當下走過花子虛家來。李瓶兒使小廝請到後邊說話。只見婦人羅衫不整,粉面慵粧,從房裡出來,臉諕的蠟查也似黃,跪着西門慶,再三哀告道:「大官人,沒耐何,不看僧看佛面。常言道:『家有患難,鄰保相助。』因奴拙夫不聽人言,把着正經家事兒不理,只在外信着人,成日不着家。今日只當吃人暗算,弄出這等事來。着緊這時節,方對小廝說,將來教我尋人情救他。我一個女婦人,沒腳蟹,那裡尋那人情去?發狠起將來。想着他恁不依他說,拿到東京打的他爛爛的不虧。只是難為過世老公公的名字。奴沒奈何,請將大官人來,央及大官人把他不要題起罷。千萬只看奴之薄面,有人情,好歹尋一個兒,只休教他吃凌逼便了。」西門慶見婦人下禮,連忙道:「嫂子請起來不妨。今日我還不知因為了甚勾當?俺每都在鄭家吃酒,只見幾個做公的人,把哥拿的到東京去了。」婦人道:「正是一言難盡。此是俺過世老公公連房大姪兒,花大、花三、花四,與俺家都是叔伯兄弟。大哥喚做花子由,三哥喚花子光,第四個的叫花子華。俺這個名花子虛,卻是老公公嫡親姪兒。雖然老公公掙下這一分家財,見俺這個兒不成器,從廣東回來,把東西只交付與我手裡收着。着緊還打俏棍兒,那別的越發打的不敢上前。去年老公公死了,這花大、花三、花四也分了些床帳家去了。只見一分銀子兒沒曾得,我便說多少與他些也罷了。俺這個成日只在外邊胡幹,把正經事兒通不理一理兒。今日手暗不透風,卻教人弄下來了。」說畢,放聲大哭。西門慶道:「嫂子放心。我只道是甚麼事來,原來是房分中告家財事!這個不打緊處。既是嫂子分付,哥的事兒就是我的事,我的事就如哥的事一般。隨問怎的,我在下謹領。」婦人問道:「官人若肯下顧時,又好了。請問尋分上用多少禮兒?奴好預備。」西門慶道:「也用不多,聞得東京開封府楊府尹,乃蔡太師門生。蔡太師與我這四門家楊提督,都是當朝天子面前說得話的人。拿兩個分上齊對楊府尹說,有個不依的?不拘多大事情也了了。如今倒是蔡太師用些禮物。那提督楊爺,與我舍下有親,他肯受禮?」婦人便往房裡開箱子,搬出六十定大元寶,共計三千兩,教西門慶收去,尋人情上下使用。西門慶道:「只消一半足矣,何消用得許多?」婦人道:「多的大官人收去。奴床後邊,有四口描金箱櫃,蟒衣玉帶,帽頂縧環,提繫條脫,值錢珍寶,玩好之物,亦發大官人替我收去,放在大官人那裡。奴用時取去。趁早奴不思個防身之計,信着他,往後過不出好日子來。眼見得拳迭不得四手,到明日沒的把這些東西兒,吃人暗算奪了去,坑閃得奴三不歸。」西門慶道:「只怕花二哥來家,尋問怎了?」婦人道:「這個都是老公公在時,梯己交與奴收着的之物,他一字不知。官人只顧收去。」西門慶說道:「既是嫂子恁說,我到家叫人來取。」于是一直來家與月娘商議。月娘說:「銀子便用食盒叫小廝抬來。那箱籠東西,若從大門裡來,教兩邊街房看着不惹眼?必須如此如此,夜晚打牆上過來,方隱密些。」西門慶聽言大喜,即令來旺兒、玳安兒、來興、平安四個小廝,兩架食盒,把三千兩金銀,先抬來家。然後到晚夕月上的時分,李瓶兒那邊同兩個丫鬟迎春、秀春,放桌凳,把箱櫃挨到牆上。西門慶這邊,止是月娘、金蓮、春梅用梯子接着。牆頭上舖苫毡條,一個個打發過來,都送到月娘房中去。你說有這等事?要得富,險上做。有詩為證:

「富貴自是福來投,  利名還有利名憂,

命裡有時終須有,  命裡無時莫強求。」

西門慶收下他許多軟細金銀寶物,鄰舍街坊俱不得知道。連夜打馱裝停當,求了他親家陳宅一封書,差家人上東京。一路朝登紫陌,暮踐紅塵。有日到了東京城內,交割楊提督書禮,轉求內閣蔡太師柬帖,下與開封府楊府尹。這府尹名喚楊時,別號龜山,及陝西弘農縣人氏。由癸未進士,陞太理寺卿,今推開封府裡,楊是個清廉的官。況蔡太師是他舊時座主,楊戩又是當道時臣,如何不做分上?這裡西門慶又順星夜稍書花子虛知道說:「人情都到了。等當官問你家財下落,只說都花費無存,止是房產莊田見在。」恰說一日楊府尹陞廳,六房官吏俱都祇候。但見:

「為官清正,作事廉明。每懷惻隱之心,常有仁慈之念。爭田奪地,辨曲直而後施行;鬬毆相爭,審輕重方使決斷。閒則撫琴會客,也應分理民情。雖然京兆宰臣官,果是一邦民父母。」

當日楊府尹陞廳,監中提出花子虛來等一干人上廳跪下,審問他家財下落。那花子虛口口只說:「自從老公公死了,發送念經,都花費了。止有宅舍兩所,莊田一處見在。其餘床帳家火物件,俱被族人分扯一空。」楊府尹道:「你每內官家財,無可稽考,得之易,失之易。既是花費無存,批仰清河縣委官,將花太監住宅二所,莊田一處,估價變賣,分給花子由等三人回繳。」子由等還當廳跪稟,還要監追子虛要別項銀兩下落。被楊府尹大怒都喝下來了。說道:「你這廝少打!當初你那內相一死之時,你每不告,做甚麼來?如今事情已往,又來騷擾,費告我紙筆。」于是把花子虛一下兒也沒打,批了一道公文,押發清河縣前來估計莊宅,不在話下。早有西門慶家人來保,打聽這消息,星夜回來報知西門慶。門慶聽的楊府尹見了分上,放出花子虛來家,滿心歡喜。這裡李瓶兒請過西門慶去計議,要教西門慶:「拿幾兩銀子,買了所住的宅子罷。到明日奴不久也是你的人了。」西門慶歸家,與吳月娘商議。月娘道:「隨他當官估價賣多少,你不可承攬要他這房子。恐怕他漢子一時生起疑心來怎了。」這西門慶聽記在心。那消幾日,花子虛來家,清河縣委下樂縣丞丈估。計太監大宅一所,坐落大街安慶坊,值銀七百兩,賣與王皇親為業;南門外莊田一處,值銀六百五十五兩,賣與守備周秀為業;止有住居小宅,值銀五百四十兩,因在西門慶隔壁,沒人敢買。花子虛再三使人來說,西門慶只推沒銀子,延挨不肯上帳。縣中緊等要回文書。李瓶兒急了,暗暗使過馮媽媽來,對西門慶說:「教拿他寄放的銀子,兌五百四十兩買了罷。」這西門慶方纔依允,當官交兌了銀兩。花大哥都畫了字,連夜做文書回了上司。共該銀二千八百九十五兩,三人均分訖。花子虛打了一場官司出來,沒分的絲毫,把銀兩房舍莊田又沒了,兩箱內三千兩大元寶又不見蹤影,心中甚是焦燥。因問李瓶兒查算西門慶那邊使用銀兩下落:「今剩下多少,還要湊著添買房子。」反吃婦人整罵了四五日,罵道:「呸!魍魎混沌!你成日放着正事兒不理,在外邊眠花臥柳不着家,只當被人所算,弄成圈套,拿在牢裡。使將人來對我說,教我尋人情。奴是個婦人家,大門邊兒也沒走;能走不能飛,曉的甚麼?認的何人?那裡尋人情?渾身是鐵,打得多少釘兒!替你到處求爹爹,告奶奶,甫能得人情平惜不種下,急流之中,誰人來管你?多虧了他隔壁西門慶看日前相交之情,大冷天,刮的那黃風黑風,使了家下人往東京去,替你把事兒幹的停停當當的。你今日了畢官司出來,兩腳踏住平川地,得命思財,瘡好忘痛,來家還問老婆找起後帳兒來了!還說有也沒。你過陰!有你寫來的帖子見在。沒你的手字兒,我擅自拿出你的銀子尋人情,抵盜與人便難了。」花子虛道:「可知是我的帖子來說,實指望還剩下些。咱湊着買房子過日子,往後知數拳兒了。」婦人道:「呸!濁壞料!我不叫罵你的,你早仔細好來!囷兒上下算計,圈底兒下卻算計。千也說使多了,萬也說使多了。你那三千兩銀子,能到的裡?蔡太師、楊提督好小食腸兒?不是恁大人情囑的話,平白拿了你一場,當官蒿條兒也沒曾打在你這王八身上。好好放出來,教你在家裡恁說嘴!人家不囑你管轄不倒,你甚麼着疼的親故?平白怎替你南上北下走跳,使錢救你?你來家該擺席酒兒,請過人來知謝人一知謝兒。還一掃帚掃的人光光的,問人找起後帳兒來了。」幾句連搽帶罵,罵的子虛閉口無言。到次日,西門慶使了玳安送了一分禮來與子虛壓驚。子虛這裡安排了一席,叫了兩個妓者,請西門慶來知謝,就找着問他銀兩下落。依着西門慶這邊,還要找過幾百兩銀子與他湊買房子。李瓶兒不肯,暗地使過馮媽媽子過來,對西門慶說:「休要來吃酒,開送了一篇花帳與他,只說銀子上下打點都使沒了。」花子虛不識時,還使小廝再三邀請。西門慶一徑躲的往院裡去了,只回不在家。花子虛氣的發昏,只是跌腳。看官聽說:大抵只是婦人更變,不與男子漢一心,隨你咬折釘子般剛毅之夫,也難防測其暗地之事。自古男治外而女治內,往往男子之名,都被婦人壞了者。為何?皆由御之不得其道故也。要之,在乎夫唱婦隨,容德相感,緣分相投,男慕乎女,女慕乎男,庶可以保其無咎。稍有微嫌,輒顯厭惡。若似花子虛終日落魄飄風,謾無紀律,而欲其內人不生他意,豈可得乎!正是:

「自意得其墊,  無風可動搖。」

有詩為證:

「功業如將智方求,  當年盜跖卻封侯,

行藏有義真堪羨,  好色無仁豈不羞;

浪蕩貪淫西門子,  背夫水性女嬌流,

子虛氣塞柔腸斷,  他日冥司必報仇。」

話休饒舌。後來子虛只擯湊了二百五十兩銀子,買了獅子街一所房屋居住。得了這口重氣,剛搬到那裡,不幸害了一場傷寒。從十一月初旬睡倒在床上,就不曾起來的。對李瓶兒還請的大街坊胡太醫來看。後來怕使錢,只挨着一日兩,兩日三,挨到三十頭,嗚呼哀哉,斷氣身亡,亡年二十四歲。那手下的大小廝天喜兒從子虛病倒之時,拐了五兩銀子,走了無蹤跡。子虛一倒了頭,李瓶兒就使了馮媽媽請了西門慶過去,與他商議,買棺入殮,念經發送子虛到墳上埋葬。那花大、花三、花四一般兒男婦也都來吊孝。送殯回來,各都散了。西門慶那日也教吳月娘辦了一張桌席,與他山頭祭奠。當日婦人轎子歸家,也回了一個靈位供養在房中。雖是守靈,一心只想着西門慶。從子虛在時,就把兩個丫頭教西門慶要了。子虛死後越發通家往還。一日正月初九日,李瓶兒打聽是潘金蓮生日。未曾過子虛五七,就買禮坐轎子,穿白綾襖兒,藍織金裙,白苧布䯼髻,珠子箍兒,來與金蓮做生日。馮媽媽抱毡包,天福兒跟轎,進門就先與月娘插燭也磕了四個頭,說道:「前日由頭,多勞動大娘受餓,又多謝重禮!」拜了月娘,又請李嬌兒、孟玉樓拜見了。然後潘金蓮來到,說道:「這個就是五娘。」又磕下頭,一口一聲稱呼:「姐姐,請受奴一禮兒!」金蓮那裡肯受,相讓了半日,兩個還平磕了頭。金蓮又謝了他壽禮。又有吳大娘子、潘姥姥,都一同見了李瓶兒,便請西門慶拜見。月娘道:「他今日門外玉皇廟打醮去了。」一面讓坐下,換茶來吃了。良久,只見孫雪娥走過來,李瓶兒見他粧飾少次與眾人,便去起身來問道:「此位是何人?奴不知,不曾請見的。」月娘道:「此是他姑娘哩。」這李瓶兒就要慌忙行禮,月娘道:「不勞起動二娘,只拜平拜兒罷。」于是二人彼此拜畢,月娘就讓到房中,換了衣裳,分付丫鬟明間內放桌兒擺茶。須臾圍爐添炭,酒泛羊羔 ,安排上酒來。當下吳大妗子、潘姥姥、李瓶兒上坐。月娘和李嬌兒主席,孟玉樓和潘金蓮打橫,孫雪娥回廚下照管,不敢久坐。月娘見李瓶兒鍾鍾酒都不辭,于是親自巡了一遍酒。又令李嬌兒眾人各巡酒一遍,頗嘲問他話兒。便說道:「花二娘搬的遠了,俺姊妹們離多會少,好不思想!二娘狠心,就不說來看俺們看兒。」孟玉樓便道:「二娘今日不是因與六姐做生日,還不來哩!」李瓶兒道:「好大娘三娘,蒙眾娘抬舉,奴心裡也要來。一來熱孝在身,二者拙夫死了,家下沒人。昨日纔過了他五七,不是怕五娘怪,還不敢來。」因問:「大娘貴降在幾時?」月娘道:「賤日早哩!」潘金蓮接過來道:「大娘生日八月十五,二娘好歹來走走。」李瓶兒道:「不消說,一定都來。」孟玉樓道:「二娘今日與俺姊姊相伴一夜兒呵,不往家去罷了。」李瓶兒道:「奴可知也和眾位娘敍些話兒。不瞞眾位娘說,小家兒人家,初搬到那裡,自從拙夫沒了,家下沒人。奴那房子後牆,緊靠着喬皇親花園,好不空!晚夕常有狐狸打磚掠瓦,奴又害怕。原是兩個小廝,那個大小廝又走了。正是這個天福兒小廝看守前門,後半截通空落落的,倒虧了這個老馮是奴舊時人,常來與奴漿洗些衣裳,與丫頭做鞋腳累他。」月娘因問:「老馮多大年紀?且是好個恩實媽媽兒,高言兒也沒句兒!」李瓶兒道:「他今年五十六歲,屬狗兒,男兒花女沒有,只靠說媒度日。我這裡常管他些衣裳兒。昨日拙夫死了,叫過他來與奴做伴兒。晚夕同丫頭一炕睡。」潘金蓮嘴快,說道句:「卻又來,既有老馮在家裡看家,二娘在這過一夜兒也罷了。左右那花爹沒了,有誰管着你?」玉樓道:「二娘只依我,教老馮回了轎子不去罷。」那李瓶兒只是笑,不做聲。說話中間,酒過數巡。潘姥姥先起身往前邊去了。潘金蓮隨跟着他娘,往房裡去了。李瓶兒再三辭:「奴的酒勾了。」李嬌兒道:「花二娘怎的在他大娘、三娘手裡吃過酒,偏我遞酒,二娘不肯吃?顯的有厚薄。」于是拿大杯,只顧斟上。李瓶兒道:「好二娘,奴委的吃不去了,豈敢做假?」月娘道:「二娘你吃過此杯,略歇歇兒罷。」那李瓶兒方纔接了,放在面前,只顧與眾人說話。孟玉樓見春梅立在傍邊,便問春梅:「你娘在前邊做甚麼哩?你去連你娘潘姥姥快請來,你說大娘請來陪你花二娘吃酒哩。」春梅去不多時,回來道:「俺姥姥害身上疼,睡哩。俺娘在房裡勻臉,就來。」月娘道:「我倒也沒見,你倒是個主人家,把客人丟下,三不知往房裡去了。俺姐兒一日臉不知勻多少遭數,要便走的勻臉去了。諸般都好,只是有這些孩子氣。」正說着,只見潘金蓮上穿了香色潞紬雁啣蘆花樣對衿襖兒,白綾豎領,粧花眉子,溜金蜂趕菊鈕扣兒;下着一尺寬海馬潮雲,羊皮金沿邊挑線裙子,大紅段子白綾高底鞋,粧花膝褲,青寶石墜子,球子箍,與孟玉樓一樣打扮。惟月娘是大紅段子襖,青素綾披襖,沙綠紬裙頭。上帶着䯼髻貂鼠臥兔兒,玉樓在席上,看見金蓮豔抹濃粧,鬢嘴邊撇着一根金壽字簪兒,從外搖擺將來,戲道:「五丫頭,你好人見,今日是你個驢馬畜,把客人丟在這裡,你躲房裡去了。你可成人養的?」那金蓮笑嘻嘻向他身上打了一下。玉樓道:「好大膽的五丫頭!你還來遞一鍾兒。」李瓶兒道:「奴在三娘手裡吃了好少酒兒,已卻勾了。」,金蓮道:「他的手裡是他手裡帳,我也敢奉二娘一鍾兒。」于是揎起袖子,滿斟一大杯遞與,李瓶兒只顧放着不肯吃。月娘陪吳大妗子從房裡出來,看見金蓮陪着李瓶兒的,問道:「他潘姥姥怎的不來陪花二娘坐?」金蓮道:「俺媽害身上疼,在房裡歪着哩,叫他不肯來。」月娘因看見金鬢上撇着那壽字簪兒,便問:「二娘,你與六姐這對壽字簪兒,是那裡打造的?倒且是好樣兒,倒明日俺每人照樣也配恁一對兒戴。」李瓶兒道:「大娘既要,奴還有幾對兒,到明日每位娘都補奉上一對兒。此是過世公公宮裡御前作帶出來的,外邊那裡有這樣範?」月娘道:「奴取笑鬬二娘要子,俺姊妹們人多,那裡有這些相送?」眾女眷飲酒歡笑,看看日西時分,馮媽媽在後邊雪娥房裡,管待酒,吃的臉紅紅的出來,催逼李瓶兒起身,不起身,好打發轎子回去。月娘道:「二娘不去罷,叫老馮回了轎子家去罷。」李瓶兒只說:「家裡無人,改日再奉看列位娘,有日子住哩。」孟玉樓道:「二娘好執古,俺眾人就沒些分上兒。如今不打發轎子,等住回他爹來,少不的也要留二娘。」自這說話,逼迫的李瓶兒就把房門鑰匙遞與馮媽媽。說道:「既是他眾位娘再三留我,顯的奴不識敬重。分付轎子回去,教他明日來接罷。你和小廝家仔細門戶。」又叫過馮媽媽,附耳低言:「教大丫頭迎春拿鑰匙開我床房裡頭一個箱子,小描金頭面匣兒裡,拿四對金壽字簪兒,你明日早送來,我要送四位娘。」那馮媽媽得了話,拜辭了月娘。月娘道:「吃酒去。」馮媽媽道:「我剛纔在後邊姑娘房裡,酒飯都吃了,明日老身早來罷。」一面千恩萬謝出門,不在話下。少頃,李瓶兒不肯吃酒,月娘請到上房同大妗子一處吃茶坐的。忽見玳安小廝抱進毡包,西門慶來家,掀開簾子進來,說道:「花二娘在這裡?」慌的李瓶兒跳起身來,兩個見了禮坐下。月娘叫玉簫與西門慶接了衣裳。西門慶便對吳大妗子、李瓶兒說道:「今日會門外玉皇廟聖誕打醮,該我年例做會首。要不是,過了午齋,我就來了。因與眾人在吳道官房裡算帳,七擔八柳,纏到這咱晚。」因問二娘:「今日不家去罷了?」玉樓道:「二娘這裡再三不肯,要去。被俺眾姊妹強着留下。」李瓶兒道:「家裡沒人,奴不放心。」西門慶道:「沒的扯淡,這兩日好不巡夜的甚緊,怕怎的?但有些風吹草動,拿我個帖送與周大人點倒奉行。」又道:「二娘怎的冷清清坐着?用了些酒兒不曾?」孟玉樓道:「俺眾人再三奉勸二娘,二娘只是推不肯吃。」西門慶道:「你們不濟,等我奉勸二娘。二娘好小量兒。」李瓶兒口裡雖說奴吃不去了,只不動身。一面分付丫鬟從新房中放桌兒,都是留下伺候西門慶的整下飯菜蔬、細巧果仁,擺了一張桌子。吳大妗子知局,趐趫推不用酒,因往李嬌兒那邊房裡去了。當下李瓶兒上坐,西門慶拿椅子關席。吳月娘在炕上跐着爐壺兒,孟玉樓、潘金蓮兩邊打橫。五人坐定,把酒來斟。也不用小鍾兒,要大銀衢花鍾子。你一杯,我一盞,常言:「風流茶說合,酒是色媒人。」吃來吃去,吃的婦人眉黛低橫,秋波斜視。正是:

「兩朵桃花上臉來,  眉眼施開真色婦。」

月娘見他二人吃的餳成一塊,言頗涉邪,有下上來,往那邊房裡吳大妗坐去了,由着他三個陪着。吃到三更時分,李瓶兒星眼迆斜,身立不住,拉金蓮往後邊淨手。西門慶走到月娘這邊房裡,亦東倒西歪,問月娘打發他那裡歇。月娘道:「他來與那個做生日,就在那個兒房裡歇。」西門慶:「我在那裡歇宿?」月娘道:「隨你那裡歇宿,再不你也跟了他一處去歇罷。」西門慶笑道:「豈有此禮。」因叫小玉來脫衣:「我在這房裡睡了。」月娘道:「就別要汗邪,休惹我那沒好口的罵的出來。你在這裡,他大妗子那裡歇?」西門慶道:「罷罷!我孟三兒房裡歇去罷。」于是往玉樓房中歇了。潘金蓮引着李瓶兒淨了手,同他前邊來,晚夕和姥姥一處歇臥。到次日起來,臨鏡梳頭。春梅與他討洗臉水,打發他梳粧。因見春梅伶變,知是西門慶用過的丫鬟,與了他一付金三事兒,那春梅連忙就對金蓮說了。金蓮謝了又謝,說道:「又勞二娘賞賜他!」李瓶兒道:「不枉了五娘有福,好個姐姐。」早晨金蓮領着他同潘姥姥叫春梅開了花園門,各處遊看了一遍。李瓶兒看見他那邊牆頭開了個便門,通着他那壁,便問:「西門爹幾時起蓋這房子?」金蓮道:「前者央陰陽看來,也只到這二月間典工動土,收起要蓋,把二娘那房子打開通做一處。前面蓋山子捲棚,展一個大花園。後面還蓋三間翫花樓,與奴這三間樓相連做一條邊。」這李瓶兒聽見在心。兩人正說話,只見月娘使了小玉來請後邊吃茶。三人同來到上房,吳月娘、李嬌兒、孟玉樓陪着吳大妗擺下茶等着哩。眾人正吃點心茶湯,只見馮媽媽驀地走來,眾人讓他坐吃茶。馮媽媽向袖中取出一方舊汗巾,包着四對金壽字簪兒,遞與李瓶兒。接過來先奉了一對與月娘,然後李嬌兒、孟玉樓、孫雪娥,每人都是一對。月娘道:「多有破費二娘,這個卻使不得。」李瓶兒笑道:「好大娘,甚麼罕稀之物,胡亂與娘們賞人便了。」月娘眾人拜謝了,方纔各人插在頭上。月娘道:「只說二娘家門首就是燈市,好不熱鬧。到明日俺們看燈去,就到往二娘府上望望,休要推不在家。」李瓶兒道:「奴到那日奉請眾位娘。」金蓮道:「姐姐還不知,奴打聽來,這十五日是二娘生日。」月娘道:「今日說道,若道二娘貴降的日子,俺姊妹一個也不少,來與二娘祝壽去。」李瓶兒笑道:「蝸居小舍,娘們肯下降,奴已定奉請。」不一時吃罷早飯,擺上酒來飲酒。看看留連到日西時分,轎子來接,李瓶兒告辭歸家,眾姊妹款留不住。臨出門請西門慶拜見。月娘道:「他今日早起身出門,與縣丞送行去了。」婦人千恩萬謝,方纔上轎來家。正是:

「合歡核桃真堪笑,  裡許原來別有人。」

畢竟後來何如,且聽下回分解:





\end{showcontents}
