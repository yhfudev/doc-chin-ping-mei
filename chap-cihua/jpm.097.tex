%# -*- coding: utf-8 -*-
%!TEX encoding = UTF-8 Unicode
%!TEX TS-program = xelatex
% vim:ts=4:sw=4
%
% 以上设定默认使用 XeLaTex 编译,并指定 Unicode 编码,供 TeXShop 自动识别

%第九十七回 
\chapter{陳經濟守禦府用事\KG 薛嫂買賣說姻親}


\begin{showcontents}{}




「在世為人保七旬,  何勞日夜弄精神,

世事到頭終有盡,  浮華過眼恐非真;

貧窮富貴天之命,  得失榮枯隙裡塵,

不如且放開懷樂,  莫待無常鬼使侵。」

話說陳經濟到於守備府中下了馬,張勝先進去本稟報春梅。春梅分付,教他在外邊班直房內,用香湯澡盆,沐浴了身體乾淨,後邊使養娘包出一套新衣服靴帽來,與他更換了。張勝把他身上脫下來舊藍縷衣服,捲做一團,閣在班直房內上上吊著,然後稟了春梅。那時守備還未退廳,春梅請經濟到後堂。盛粧打扮,出來相見。這經濟進門,就望春梅拜了四雙八拜:「讓姐姐受禮!」那春梅受了半禮,對面坐下。敘說寒溫離別之情,彼此皆眼中垂淚。春梅恐怕守備退廳進來,見無人在眼前,使眼色與經濟悄悄說:「等住回他若問你,只說是姑表兄弟,我大你一歲,二十五歲了,四月廿五日午時生的。」經濟道:「我知道了。」不一時,丫鬟拏上茶來,兩人吃了茶。春梅便問:「你一向怎麼出了家做了道士?打我這府中出去,守備不知是我的親,錯打了你,悔的要不的!若不是那時就留下你,爭奈有雪娥那賤人在我這裡,不好又安插你的,所以放你去了。落後打發了那賤人,纔使張勝到處尋你不著。誰知你在城外做工,流落至于此地位!」經濟道:「不瞞姐姐說,一言難盡!自從與你相別,要娶六姐。我父親死在東京,來遲了,不曾娶成,被武松殺了。聞得你好心,葬埋了他永福寺,我也到那裡燒布來。在家又把俺娘沒了。剛打發喪事出去,被人坑陷了資本。來家又是大姐死了,被俺丈母那淫婦告了我一狀,床帳粧奩,都搬的去了。打了一場官司。將房兒賣了,弄的我一貧如洗。多虧了俺爹朋友王杏菴賑濟,把我纔送到臨清晏公廟那裡出家。不料又被光棍打了,拴到咱府中,打了十棍出去,投親不理,投友不顧。因此在寺內傭工。多虧姐姐掛心,使張管家尋將我來。見姐姐一面,恩有重報,不敢有忘!」說到傷心處,兩個都哭了。正說話中間,只見守備退廳,進入後邊來。左右掀開簾子,守備進來,這陳經濟向前倒身下拜。慌的守備答禮相還,說:「向日不知是賢弟,被下人隱瞞,有誤衝撞,賢弟休怪!」經濟道:「不才有玷,一向缺禮,有失親近,望乞恕罪!」又磕下頭去。守備一手拉起,讓他上坐。那經濟乖覺,那里肯,務要拉下椅兒旁邊坐了。守備關席,春梅陪他對坐下。須臾換茶上來,吃畢。守備便問:「賢弟貴庚?一向怎的不見?如何出家?」經濟便告說:「小弟虛度二十四歲,俺姐姐長我一歲,是四月二十五日午時生。向因父母雙亡,家業凋喪,妻又沒了,出家在晏公廟。不知家姐嫁在府中。有失探望。」守備道:「自從賢弟那日去後,你令姐晝夜憂心,常時啾啾唧唧不安,直到如今。一向使人找尋賢弟不著。不期今日相會,實乃三生有緣!」一面分付左右放卓兒,安排酒上來。須臾擺設許多盃盤,雞蹄鵝鴨,烹炮蒸煠,湯飯點心,堆滿卓上。銀壼玉盞,酒泛金波。守備相陪敘話,吃至晚來,掌上燈燭方罷。守備分付家人周仁,打掃西書院乾淨,那裡書房床帳都有。春梅拿出兩床鋪蓋衾枕,與他安歇。又撥一個小廝喜兒答應他。又包出兩套紬絹衣服來,與他更換。每日飯食,春梅請進後邊吃。正是:

「一朝時運至,  半點不由人。」

光陰迅速,日月如梭,但見:

「行見梅花肥底,  忽逢元旦新正;

不覺艷杏盈枝,  又早新荷貼水。」

經濟在守備府裡,住了一個月有餘。一日,四月二十五日,春梅的生日。吳月娘那邊買了禮來,一盤壽桃,一盤囊麵,兩隻湯鵝,四隻鮮雞,兩盤果品,一罈南酒 ,玳安穿青衣,挐帖兒送來。守備正在廳上坐的,門上人稟報進去,抬進禮來。玳安遞上帖兒,扒在地下磕頭。守備看了禮帖兒,說道:「多承你奶奶費心,又送禮來。」一面分付家人:「收進禮去,討茶來與大官兒吃。把禮帖教小伴當送與你舅收了。封了一方手帕,三錢銀子,與大官兒;抬盒人錢一百文。拏回帖兒,多上覆。」說畢,守備穿了衣服,就起身出去拜人去了。玳安只顧在廳前伺候討回帖兒。只見一個年小的,戴著瓦楞帽兒,穿著青紗道袍,涼鞋淨襪,從角門裡走出來,手中拿著帖兒賞錢,遞與小伴當,一直往後邊去了。「可要作怪!模樣倒好相陳姐夫一般,他如何卻在這里?」只見小伴當遞與玳安手帕銀錢,打發出門。到于家中,回月娘話。見回帖上寫著:周門龐氏歛袵拜。月娘便問:「你沒見你姐?」玳安道:「姐姐倒沒見,倒見姐夫來!」月娘笑道:「怪囚,你家倒有恁大姐夫!守備好大年紀,你也叫他姐夫?」玳安道:「不是守備,是咱家的陳姐夫!我初進去,周爺正在廳上。我遞上帖兒,與他磕了頭,他說:『又生受你奶奶送重禮來!』分付伴當拿茶與我吃:『把帖兒拏與你舅收了,討一方手帕三錢銀子,與大官兒。抬盒人是一百文錢。』說畢,周爺穿衣服出來上馬,拜人去了。半日,只見他打角門裡出來,遞與伴當回帖賞賜,他就進後邊去了。我就押著盒擔出來。不是他卻是誰?」月娘道:「怪小囚兒,休胡說白道的!那羔子赤道流落在那里討吃,不是凍死就是餓死!他平白在在那府做甚麼?守備認的他甚麼毛片兒,肯招攬下他何用?」玳安道:「奶奶敢和我兩個賭?我看得千真萬真!就燒的成灰骨兒,我也認的!」月娘問:「他穿著甚麼?」玳安告訴:「他戴著新瓦楞帽兒,金簪子,身穿著青紗道袍,涼鞋淨襪,吃的好了!」月娘道:「我不信,不信!」這里說話不題。卻說陳經濟進入後邊,春梅還在房中鏡臺前搽臉,描畫雙蛾。經濟拿吳月娘禮帖兒與他看,因問:「他家如何送禮來與你?是那里緣故?」這春梅便把從前已往,清明郊外永福寺撞遇月娘相見的話,訴說一遍。後來怎生平安兒偷了解當舖頭面,吳巡檢怎生夾打平安兒,追問月娘奸情之事。薛嫂又怎生說人情,守備替他處斷了事。落後他家買禮來相謝,正月裡我往他家與孝哥兒生日,勾搭連環到如今。他許下我生日,買禮來看好一節。經濟聽了,把眼瞅了春梅一眼,說:「姐姐你好沒志氣!想著這賊淫婦,那咱把咱姐兒們生生的拆散開了,又把六姐命喪了。永世千年,門裡門外,不相逢纔好!反替他說人情兒?那怕那吳典恩追拷著平安小廝,供出奸情來,隨他那淫婦,一條繩子拴去出醜見官,管咱每大腿事!他沒和玳安小廝有奸,怎的把丫頭小玉配與他?有我早在這里,我斷不教你替他說人情!他是你我仇人,又和他上門往來做甚麼?六月連陰,想他好晴天兒!」幾句話說得春梅閉口無言。春梅道:「過往勾當也罷了!還是我心好,不念舊仇。」經濟道:「如今人好心不得好報哩!」春梅道:「他既送了禮,莫不白受他的?還等著我這裡人請他去哩!」經濟道:「今後不消理那淫婦了,又請他怎的?」春梅道:「不請他又不好意思的。丟個帖與他,來不來隨他就是了。他若來時,你在那邊書院內,休出來見他。往後咱不招惹他就是了!」經濟惱的一聲兒不言語,走到前邊,寫了帖子。春梅使家人周義,去請吳月娘。月娘打扮出門,教奶子如意見抱著孝哥兒,坐著一頂小轎,玳安跟隨,來到府中。春梅、孫二娘都打扮出來迎接,至後廳相見,敘禮坐下。如意兒抱著孝哥兒相見磕頭畢。經濟躲在那邊書院內,不走出來。由著春梅、孫二娘,在後廳擺茶安席遞酒。叫了兩個妓女,韓玉釧、鄭嬌兒彈唱,俱不必細說。玳安在前邊廂房內管待。只見一個小伴當,打後邊拿出一盤湯飯點心下飯。往西角門書院中走。玳安便問他:「拿與誰吃?」小伴當道:「是與舅吃的。」玳安道:「你舅姓甚麼?」小伴當道:「姓陳。」這玳安賊悄悄後邊跟著他到西書院,小伴當便掀簾子進去。玳安慢慢打紗窗外往裡張看。卻不是陳姐夫?正在書房床上〈扌歪〉著。見拿進湯飯點心來,連忙起來,放卓兒正吃。這玳安悄悄走出外邊來,依舊坐在廂房內。直待天晚,家中燈籠來接,吳月娘轎子起身,到家一五一十,告訴月娘,說:「果然陳姐夫在他家居住。」自從春梅這邊被經濟把攔,兩家都不相往還。正是:

「誰知豎子多間阻,  一念翻成怨恨媒!」

自此經濟在府中,與春梅暗地勾搭,人都不知。或守備不在,春梅就經濟在房中吃飯吃酒,閑時下棋調笑,無所不至。守備在家,便使丫頭小廝,拿飯往書院與他吃,或白日裡,春梅也常往書院內,和他坐半日,方歸後邊來。彼此情熱,俱不必細說。一日,守備領人馬出巡,正值五月端午佳節,春梅在西書院花亭上置了一卓酒席,和孫二娘、陳經濟吃雄黃酒,解粽歡娛。丫鬟侍妾,都兩邊侍奉。當日怎見的蕤賓好景?但見:

「盆栽綠柳,瓶插紅榴,水晶簾捲鍛鬚,雲母屏開孔雀。菖蒲切玉,佳人笑捧紫霞觴;

角黍堆金,侍妾高擎碧玉盞。食烹異品,果獻時新。靈符艾虎簪頭,五色絨繩臂。家家慶賞午節,處處歡飲香醪,遨遊身外醉乾坤,消遣壼中閑日月。得多少珮環聲碎金蓮小,紈扇輕搖玉笋柔。」

春梅令海棠、月桂兩個侍妾,在席前彈唱。當下直吃到炎光西墜,微雨生涼的時分,春梅拏起大金荷花盃來相勸。酒過數巡,孫二娘不勝酒力,起身先往後邊房中看去了。獨落下春梅和經濟在花亭上吃酒,猜枚行令,你一盃,我一盃。不一時,丫鬟掌上紗燈上來,養娘金匱、玉堂打發金哥兒睡去了。經濟輸了,便走出書房內,躲酒不出來。這春梅先使海棠來請。見經濟不去,又使月桂來。分付:「他不來,你好歹與我拉將來,拉不將來,回來把你這賤人打十個嘴巴!」這月桂走至西書房中,推開門,見經濟〈扌歪〉在床上,推打鼾睡不動。月桂說:「奶奶交我來請你老人家。請不去,要打我哩!」那經濟口裡喃喃吶吶說:「打你不干我事,我醉了,吃不的了!」被月桂用手拉將起來,推著他:「我好歹拉你去!拉不將你去,也不算好漢!」推拉的經濟急了,黑影子裡,佯裝著醉,作耍當真,摟了月桂在懷裡,就親個嘴。那月桂亦發上頭上腦說:「人好意叫你,你做大不正,倒做這個營生!」經濟道:「我的兒!你若肯了,那個好意做大不成?」又按著親了個嘴,方走到花亭上。月桂道:「奶奶要打我,還是我把舅拉將來了!」春梅令海棠斟上大鍾,兩個下盤棋,賭酒為樂。當下你一盤,我一盤,熬的丫鬟都打睡去了。春梅又使月桂、海棠後邊取茶去。兩個在花亭上,解珮露相如之玉,朱唇點漢署之香。正是:

「得多少花陰曲檻燈斜照,  旁有墜釵雙鳳翹!」

有詩為證:

「花亭懽洽鬢雲斜,  粉汗凝香沁絳紗;

深院日長人不到,  試看黃鳥啄名花。」

當下兩個正幹得好,忽然丫鬟海棠送茶來:「請奶奶後邊去,金哥睡醒了,哭著尋奶奶哩!」春梅陪經濟又吃了兩鍾酒,用茶漱了口,然後抽身往後邊來。丫鬟收拾了家活,喜兒扶經濟歸書房寢歇,不在話下。一日,朝廷勅旨下來,命守備領本部人馬,會同濟州府知府張叔夜,征勦梁山泊賊王宋江,早晚起身。守備對春梅說:「你在家看好哥兒,叫媒人替你兄弟尋上一門親事。我帶他個名字在軍門,若早僥倖得功,朝廷恩典,陞他一官半職,於你面上也有光輝。」這春梅應諾了。遲了兩三日,守備打點行裝,整率人馬,留下張勝、李安看家。止帶家人周仁跟了去不題。一日春梅叫將薛嫂兒來,如此這般和他說:「他爹臨去,分付替我兄弟尋門親事。你替我尋個門當戶對好女兒,不拘十六七歲的也罷。只要好模樣,腳手兒聰明伶俐些的。他性兒也有些刁厥些兒。」薛嫂兒道:「我不知道他也怎的?要你老人家分付。想著大姐那等的還嫌哩!」春梅道:「若是尋的不好,看我打你耳刮子不打?我要趕著他叫小妗子兒哩,休要當耍子兒!」說畢,春梅令丫鬟擺茶與他吃。只見陳經濟進來吃飯。薛嫂向他道了萬福,說:「姑夫,你老人家一向不見,在那里來?且喜呀!剛纔奶奶分付,交我替你老人家尋個好娘子,你怎麼謝我?」那陳經濟把臉兒蛙著不言語。薛嫂道:「老花子怎的不言語?」春梅道:「你休叫他姑夫,那個已是揭過去的帳了。你只叫他陳舅就是了。」薛嫂道:「只該打我這片子狗嘴!只要叫錯了。往後趕著你只叫舅爺罷。」那陳經濟忍不住撲吃的笑了,說道:「這個纔可到我心上!」那薛嫂撒風撒痴?趕著打了他一下,說道:「你看老花子說的好話兒!我又是你影射的,怎麼可在你心上?」連春梅也笑了。不一時,月桂安排茶食,與薛嫂吃了。提著花箱兒出來,說道:「我替你老人家用心踏看,有人家相應好女孩兒,就來說。」春梅道:「財禮羹果,花紅酒禮,頭面衣服,不少他的,只要好人家好女孩兒,方可進入我門來。」薛嫂道:「我曉得。管情應的你老人家心便了!」良久,經濟吃了飯,往前邊去了。薛嫂兒還坐著,問春梅:「他老人家幾時來的?」春梅便把出家做道士一節說了:「我尋得他來,做我個親人兒。」薛嫂道:「好好,你老人家有後眼!」又道:「前日你老人家好的日子,說那頭他大娘來做生日來?」春梅道:「先送禮來,然後纔使人送帖兒請他坐了一日去了。」薛嫂道:「我那日在一個人家鋪床,整亂了一日,心內要來,急的我要不的!」又問:「他陳舅也見他那頭大娘來?」春梅道:「他肯下氣見他?為請他,好不和我亂成一塊!我與他說,人替他家說人情。說我沒志氣:『那怕吳典恩打著小廝,攀扯他出官纔好,管你腿事!你替他尋分上,想著他昔日好情兒?』」薛嫂道:「他老人家也說的是。及到其間,人不計舊仇。」春梅道:「咱既受了他禮,不請他來坐坐兒又使不的。寧可教他不仁,休要咱不義!」薛嫂道:「怪不的你老人家有恁大福,你的心忒好了!」當下薛嫂兒說了半日話,提著花箱兒拜辭出門。過了兩日,先來說:「城裡朱千戶家小姐,今年十五歲,也好陪嫁。只是沒了娘的兒子。」春梅嫌小,不要。又說:「應伯爵第二個女兒,年二十二歲。」春梅又嫌應伯爵死了,在大爺手內聘嫁,沒甚陪送也不成。都回出婚帖兒來。又遲了幾日,薛嫂兒送花兒來,袖中取出個婚帖兒,大紅段子上寫著:「開段舖葛員外家大女兒,年二十歲,屬雞的,十一月十五日子時生,小字翠屏,生的上畫兒般模樣兒,五短身材,瓜子面皮,溫柔典雅,聰明伶俐。針指女工,自不必說。父母俱在,有萬貫錢財,在大街上開段子舖。走蘇、杭、南京,無比好人家!都是南京床帳箱籠。」春梅道:「既是好,成了這家子的罷。」就交薛嫂兒先通信去。那薛嫂兒連忙說去了。正是:

「欲向繡房求艷質,  須臾紅葉是良媒!」

有詩為證:

「天仙機上繫香羅,  千里姻緣竟足多;

天上牛郎配織女,  人間才子伴嬌娥。」

這里薛嫂通了信來。葛員外家知是守備府裡,情願做親。又使一個張媒人同說媒。春梅這里備了兩抬茶葉,髓餅羹果,教孫二娘坐轎子,往葛員外家插定女兒,帶戒指兒。回來對春梅說:「果然好個女子!生的一表人材,如花似朵,人家又相當。」春梅這里擇定吉日,納採行禮。十六盤羹果茶餅,兩盤上頭面,二盤珠翠,四抬酒,兩牽羊。一頂䯼髻,全付金銀頭面,簪環之類,兩件羅段袍兒,四季衣服。其餘綿花布絹,二十兩禮銀,不必細說。陰陽生擇在六月初八日,准娶過門。春梅先問薛嫂兒:「他家那里有陪床使女沒有?」薛嫂兒道:「床帳粧奩,描金箱廚都有,只沒有使女陪床。」春梅道:「咱這里買一個十三四歲丫頭子,與他房裡使喚,掇桶子倒水方便些。」薛嫂道:「有兩個人家賣的丫頭子,我明日帶一個來。」到次日,果然領了一個丫頭,說:「是商人黃四家兒子房裡使的丫頭,今年纔十三歲。黃四因用下官錢糧,和李三家,還有咱家出去的保官兒,都為錢糧,拏在監裡追賍,監了一年多,家產盡絕,房兒也賣。李三先死,拏兒子李活監著。咱家保官兒那兒子僧寶兒,如今流落在外,與人家跟馬哩!」春梅道:「是來保?」薛嫂道:「他如今不叫來保,改了名字叫湯保了。」春梅道:「這丫頭是黃四家丫頭,要多少銀子?薛嫂道:「只要四兩半銀子,緊等著要交賍去。」春梅道:「甚麼四兩半!與他三兩五錢銀子留下罷。」一面就交了三兩五錢雪花官銀與他,寫了文書,改了名字,喚做金錢兒。話休饒舌。又早到六月初八。春梅打扮珠翠鳳冠,穿通袖大紅袍兒,束金鑲碧玉帶,坐四人大轎,鼓樂燈籠,娶葛家女子,奠雁過門。陳經濟騎大白馬,揀銀鞍轡,青衣軍牢喝道,頭戴儒巾,穿著青段圓領,腳下粉底皂靴,頭上簪著兩枝金花。正是:

「久旱逢甘雨,他鄉遇故知;洞房花燭夜,金榜掛名時。」

一番折洗一番新!到守備府中,新人轎子落下。戴著大紅銷金蓋袱,添粧含飯,抱著寶瓶,進入大門陰陽生引入畫堂,先參拜家堂,然後歸到洞房。春梅安他兩口兒坐帳,然後出來。陰陽生撒帳畢,打發喜錢出門,鼓手都散了。經濟與這葛翠屏小姐,坐了回帳,騎馬打燈籠,往岳丈家謝親,吃的大醉而歸。晚夕女貌郎才,未免燕爾新婚,交姤雲雨。正是:

「得多少春點杏桃紅綻蕊,  風欺楊柳綠翻腰!」

有詩為證:

「近覩多情花月標,  教人無福也難消;

風吹列子歸何處,  夜夜嬋娟在柳梢。」

當夜經濟與這葛翠屏小姐,倒且是合得著。兩個被底鴛鴦,帳中鸞鳳,如魚似水,合巹懽娛。三日完飯,春梅在府廳後堂,張筵掛綵,鼓樂笙歌,請親眷吃會親酒,俱不必細說。每日春梅吃飯,必請他兩口兒,同在房中一處吃。彼此以姑妗稱之,同起同坐。丫頭養娘,家人媳婦,誰敢道個不字,原來春梅收拾西廂房三間,與他做房。裡面鋪著床帳,翻的雪洞般齊整,垂著簾幃。外邊西書院,是他書房,裡面亦有床榻、几席、古書,并守備往來書柬拜帖,并各處遞來手本揭帖,都打他手裡過。或登記簿籍,或御使印信。筆硯文房都有,架閣上堆滿書集。春梅不時常出來書院中,和他閑坐說話。兩個暗地交情,非止一日。正是:

「朝陪金谷宴,  暮伴綺樓娃;

休道歡娛處,  流光逐落霞。」

畢竟未知後來何如,且聽下回分解:





\end{showcontents}


