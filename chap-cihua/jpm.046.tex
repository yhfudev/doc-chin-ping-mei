%# -*- coding: utf-8 -*-
%!TEX encoding = UTF-8 Unicode
%!TEX TS-program = xelatex
% vim:ts=4:sw=4
%
% 以上设定默认使用 XeLaTex 编译,并指定 Unicode 编码,供 TeXShop 自动识别

%第四十六回 
\chapter{元夜遊行遇雪雨\KG 妻妾笑卜龜兒卦}


\begin{showcontents}{}



「帝里元宵,風光好,勝仙島蓬萊。玉塵飛動,車喝綉轂,月照樓臺。三宮此夕歡諧,金蓮萬盞,撒向天街。迓鼓通宵,華燒競起,五夜齊開。」

此隻詞兒,是前人所作。單題這元宵景致,人物繁華。且說西門慶那日打發吳月娘眾人,往吳大妗子家吃酒去了。李智、黃四約坐,伯爵趕送出去,如此這般告訴:「我已替你二公說了,准在明日,還我五百兩銀子。」那李智、黃四向伯爵打了恭,又打恭,到黃昏時分,就告辭去了。廂房中,和謝希大還陪西門慶飲酒。只見李銘掀簾子進來。伯爵看見,便道:「李日新來了。」李銘扒在地下磕頭。西門慶問道:「吳惠怎的不來?」李銘道:「吳惠今日東平府官身也沒去,在家裡害眼。小的叫了王柱來了。」便叫王柱:「進來與爹磕頭。」那王柱掀簾進入房裡,朝上磕了頭,與李銘站立在旁。伯爵道:「你家桂姐剛纔家去了,你不知道?」李銘道:「小的官身,到家洗了洗臉,就來了,並不知道。」伯爵同西門慶說:「他兩個怕不的還沒吃飯哩,哥分付拿飯與他兩個吃。」書童在旁說:「二爹叫他等一等,亦發和吹打的一答裡吃罷。沒也拿飯去了。」怕爵令書童取過一個托盤來,卓上掉了兩碟下飯,一盤燒羊肉 ,遞與李銘等:「拿了飯,你每拿兩碗,在這明間吃罷。」說書童兒:「我那俊侄子,常言道:『方以類聚,物以群分。』你不知他這行人,故雖是當院出身小優兒,比樂工不同,一概看待也罷了,顯的說你我不幫襯了。」被西門慶向伯爵頭上打了一下,笑罵道:「怪不的你這狗材,行記中人,只護行記中人,又知這當差的苦甘!」伯爵道:「俊孩兒,你知道甚麼?你空做子弟一場,連『惜玉憐香』四個字,你還不曉的,甚生說粉頭小優兒,如同鮮花兒!你惜憐他,越發有精神。你但折剉他,敢就八聲甘州『懨懨瘦損』難以存活!」西門慶笑道:「還是我的兒曉的道理。」那李銘、玉柱,須臾吃了飯。應伯爵叫過來,分付:「你兩個會唱『雪月風花共裁剪』不會?」李銘道:「此是黃鍾,小的每記的。」于是拿過箏來,王柱彈琵琶,李銘擽箏,頓開喉音,黃鍾醉花隱:

「雪月風花共裁剪,雲雨夜香嬌玉軟。花正好,月初圓,雪壓風嵌,人比天涯遠。這此時欲寄斷鵬篇,爭奈我無岸的相思,好著我難運轉。」

(喜鶯遷)  「指滄溟為硯,簡城毫逮筆如椽。松烟,將泰山作墨硯。萬里青天為錦箋,都做了草聖傳。一會家書,書不盡心事;一會家訴,訴不盡熬煎。」

(出隊子)  「憶當時初見,見俺風流小業冤,兩心中便結下死生緣。一載門澤如膠漆堅,誰承望半路番騰,倒做了離恨天。二三朝不見,渾如隔了十數年,無一頓茶飯不掛牽,無一刻光陰不唱念,無一個更兒,將他來不夢見。」

(西門子)  「無一個來人行,將他來不問遍;害可人有似風顛,相識每見了重還勸。不由我記掛在心間。思量的跟前活現,作念的口中粘涎。襟領前,袖兒邊,淚痕流遍。想從前我和他,語在前,那時節嬌小當年。論聰明貫世何曾見?他敢真誠處有萬千。」

(刮地風)  「憶咱家為他情無倦,洎江河成春戀。俺也曾坐並著膝,語並著肩。俺也

曾芰荷香,效他交頸鴛。俺也曾把手兒行,共枕眠。天也是我緣薄分淺。」

(水仙子)  「非干是我自專,只不見的鸞膠續斷絃,憶枕上盟言。念神前發願,心堅石也穿。暗暗的禱告青天,若咱家負他前世緣,俏冤家不趁今生願,俺那世裡再團圓。」

〔尾聲〕  「囑付你衷腸莫更變,要相逢除是動載經年。則你那身去遠,莫教心去遠。」

說話唱了,看看晚來,正是:

「金烏漸漸落西山,  玉兔看看上畫闌,

佳人款款來傳報,報道月移花影上紗窗。」

西門慶命收了家火,使人請傅夥計、朝道國、雲主管、賁四、陳經濟,大門首用一架圍屏,圍安放兩張卓席,懸掛兩盞羊角燈,擺設酒筵,堆集許多春檠菓盒,各樣餚饌。西門慶與伯爵、希大,都一代上面坐了。夥計、主管,兩邊打橫。大門首兩邊,一邊十二盞金蓮燈,還有一座小烟火。西門慶分付,等堂客來家時放。先是六個樂工,抬銅鑼銅鼓,在大門首吹打,動起樂來。那一回銅鑼銅鼓又清,吹細樂上來。李銘、王柱兩個小優兒,箏、琵琶上來,彈唱燈詞畫眉序:「花月滿春城」云云。那街上來往圍看的人,莫敢仰視。西門慶帶忠靖冠,絲絨鶴氅,白綾襖子。玳安與平安兩個,一遞一桶放花兒。兩名排軍,各執攬杆,攔擋閑人,不許向前擁擠。不一時碧天雲靜,一輪皓月東升之時,街上遊人,十分熱鬧。但見:

「戶戶嗚鑼擊鼓,家家品竹彈絲:遊人隊隊踏歌聲,士女翩翩垂舞調。鰲山結綵,巍峨百尺矗晴雲:鳳禁縟香,縹緲千層籠綺隊。閑廷內外,溶溶寶月光輝;畫閣高低,燦燦花燈照耀。三市六街人熱鬧,鳳城佳節賞元宵。」

且說後邊春梅、迎春、玉筲、蘭香、小玉眾人,見月娘不在,聽見大門首吹打銅鼓彈唱,又放烟火,都打扮着走來,在圍屏背後扒着望外瞧。書童兒和晝童兒兩個,在圍屏背後火盆上篩酒。原來玉筲和書童舊有私情,兩個常時戲狎;兩個因按在一處,奪瓜子兒磕。不妨火盆上坐着一錫瓶酒,推倒了,那火烘烘望上騰起來,漰了一地灰起去。那玉筲還只顧嘻笑。被西門慶聽見,使下玳安兒來問:「是誰笑?怎的這等灰起?」那日春梅穿着新白綾襖子,大紅遍地金比甲,正坐在一張椅兒上。看見他兩個推倒了酒,一經搗聲罵玉筲:「好個怪浪的淫婦!見了漢子,就邪的不知怎麼樣兒的了!只當兩個把酒推倒了纔罷了,都還嘻嘻哈哈,不知笑的是甚麼?把火也漰死了,平白落了人恁一頭灰!」那玉筲見他罵起來,諕的不敢言語,往後走了。慌的書童兒走上去,回說:「小的火盆上篩酒來,扒倒了錫瓶裡酒了。」那西門慶聽了,更不問其長短,就罷了。先是那日賁四娘子打聽月娘不在,平昔知道春梅、玉筲、迎春、蘭香四個,是西門慶貼身答應,得寵的姐兒,大節下安排了許多菜蔬菓品,使了他女孩兒長兒來,要請他四個去他家裡,散心坐坐。眾人領了來見李嬌兒。嬌兒說:「我燈草拐扙不定,你還請問你爹去!」問雪蛾,雪蛾亦發不敢承攬。看看挨到掌燈已後,賁四娘子又使了長兒來邀四人。蘭香推玉筲,玉筲推迎春,迎春推春梅,要會齊了,往李嬌兒轉央和西門慶說,放他去。那春梅坐着紋絲兒也不動,及罵玉筲等:「都是那沒見食面的行貨子,從沒見酒席,也聞些氣兒來!我就去不成,也不到央及他家去!一個個鬼攛揝的似,不知忙的是甚麼?你教我半個眼兒看的上!」那迎春、玉筲、蘭香都穿上衣裳,打扮的齊齊整整出來,又不敢去。這春梅又只顧坐着不動身。書童見賁四嫂又使了長兒來邀,說道:「我被着爹罵兩句也罷!等我上去替姐們稟稟去。」一直走到西門慶身邊,掩口對耳說道:「賁四嫂家,大節間,要請姐們坐坐。姐教我來稟問爹,去不去?」西門慶聽了,分付:「教你姐每收拾去,早些來,家裡沒人。」這書童連忙走下來,說道:「還虧我到上頭,一言就准了。教你姐快收拾去,早些來。」那春梅慢慢纔往房裡勻施脂粉去了。不一時,四個都一答兒裡出門,書童扯圍屏,掩過半邊來,遮着過去。到了賁四家,賁四娘子見了,如同天上落下來的一般,迎接裡間屋裡。頂槅上點着綉毬紗燈,一張卓兒上整齊菜,春盛堆滿滿的。趕着春梅叫大姑,迎春叫二姑,玉筲是三姑,蘭香是四姑,都見過禮。又請過韓回子娘子來相陪。教下人家,另是一分菜蔬。當下春梅、迎春上坐,玉筲、蘭香對席,賁四嫂與韓回子娘子打橫,長兒往來盪酒拿菜。按下這裡不題。西門慶因叫過樂工來,分付:「你們吹了一套『東風料峭好事近』與我聽。」正值後邊拿上玫瑰元宵來,銀金匙,眾人拿起來同吃。端的香甜美味,入口而化,甚應佳節。李銘、王柱席前又拿樂器,接着彈唱此詞,端的聲慢悠揚,挨徐合節。道:

「東野翠烟,喜遇芳天晴曉。惜花心,惟春來又起得偏早。教人探取間東君,肯與我春多少?見丫鬟笑語回言道:昨夜海棠開了!」

〔千秋歲〕  「杏花稍見著黎花雪,一點梅豆青小,流水橋邊,只聽的賣花人,聲聲頻叫。鞦韆外行人道:我只聽的粉墻內,佳人歡笑,笑道春光好!我把這花籃兒旋簇,食壘高挑。」

〔越恁好〕  「鬧花深處,涌溜溜的酒旗招。牡丹亭佐,倒尋女伴鬬百草。翠巍巍的柳

條,忒楞楞的曉鶯飛過樹梢;撲簌簌亂橫,舞翩翩粉碟兒飛過畫橋。一年景四季中,惟有春光好,向花前暢飲,月下歡笑。」

〔紅綉鞋〕  「聽一派鳳管鸞簫,見一簇翠圍珠繞。捧玉樽醉頻倒,歌金縷,舞甚麼?恁明月上花稍,月上花稍。」

〔尾聲〕  「醉教酩酊眠芳草,高把銀燈花下燒。韶光易老,休把春光虛度了。」

這裡彈唱飯酒不題。且說玳安與陳經濟,袖着許多花炮,又叫兩個排軍,拿着兩個燈籠,竟往吳大妗子家接月娘。眾人,正在明間和吳大姨、吳二妗子、吳舜臣媳婦兒,郁大姐在傍彈唱着。正飲酒,見了陳經濟來,教二舅和姐夫房裡坐:「你大舅今日不在家,衙裡看着造冊哩。」一面放卓兒,拿春盛點心酒菜上來陪經濟。玳安走到上邊,對月娘說:「爹使小的來接娘們來了。請娘早些家去。恐晚夕人亂,和姐夫一答兒來了。」月娘因着頭裡惱他,就一聲兒沒言語答他。吳大妗子便叫來定兒:「拿些甚麼兒與玳安兒吃。」來定兒道:「酒肉湯飯都前頭擺下,和他一處兒吃罷。」吳月娘道:「忙怎的?那裡纔來乍到就與他吃罷。教他前邊站着,我每就起身。」吳大妗子道:「三姑娘,慌怎的?上們兒怪人家?比來大姑娘們在俺這裡,大節下,姊妹間眾位開懷,大坐坐兒。左右家裡有他二娘和他姐在家裡,怕怎的!老早就要家去?是別人家,又是一說。」因叫郁大姐:「你唱個好曲兒伏侍,他眾位娘說你。」孟玉樓道:「他六娘好不惱他哩!不與他做生日。」郁大姐連忙下席來與李瓶兒磕了四個頭,說道:「自從與五娘做了生日!家去就不好起來。昨日妗奶奶這裡接我去,教我纔收拾〈門爭〉〈門坐〉了來。若好時,怎的不與你老人家磕頭?」金蓮道:「郁大姐,你六娘不自在哩!你唱個好的與他聽,他就不惱你了。」那李瓶兒在旁只是笑,不做聲。郁大姐道:「不打緊,拿琵琶過來,等我唱。」大妗子叫吳舜臣媳婦鄭三姐:「你把你三位姑娘和眾位娘的酒兒斟上,這一日還沒上過鍾酒兒。」那郁大姐接琵琶在手,唱一江風道:

「子時那,這淒涼如何過?羅幃錦帳和衣臥。歹哥哥,你許下我子丑時來,不覺寅時錯。疼心腸等他,待如何拋閃了我。愿神靈降與他灾和殃。」

「卯時的,亂挽起島雲髻,羞對菱花鏡。想多情,穿不的錦綉衣裳,戴不起翡翠珍珠,解不開心頭悶。辰時已過了,已時不見影。奴家為你憂成病。」

「午時排,這相思真個害,害的我魂不在。想多才,你記的月下星前,誓海盟山,誰把你輕看待?他若是未時來,也把奴愁懷解。申時買個豬頭兒賽。」

「酉時下,不由人心牽掛,誰說幾句知心話?謊冤家,你在謝館秦樓倚翠偎紅,色胆天來大。戌時點上燭,早晚不見他。亥時去卜個龜兒卦。」

正唱着,月娘便道:「怎的這一回子恁涼淒淒的起來?」來安在旁說道:「外邊天寒下雪哩!」孟玉樓道:「姐姐,你身上穿的不單薄?我倒帶了個綿披襖子來了,咱這一回夜深不冷麼?」月娘道:「見是下雪,叫個小廝,家裡取皮襖來咱們穿。」那來安連忙走下來,對玳安說:「娘分付教人家去取娘們皮襖哩。」那玳安便叫琴童兒:「你取去罷,等我在這裡伺侯。」那琴童也不問,一直家去了。少頃,月娘想起金蓮的皮襖,因問來安兒:「誰取皮襖去了?」來安道:「琴童取去了。」月娘道:「也不問我就去了。」玉樓道:「剛纔短了一句話,就教他拿俺的皮襖。他五娘沒皮襖,只取姐姐的來罷。」月娘道:「怎的家中沒有?還有當的人家一件皮襖,取來與六娘穿就是了。」月娘便問:「玳安那奴才怎的不去,都使這奴才去了?你叫他來。」一面把玳安叫到根前,吃月娘儘力罵了幾句:「好的好奴才!是你怎的不動?又遣將兒,使了那個奴才去了,也不問我聲兒,三不知就去了。但坐壇遣將兒,怪不的你做了大官兒,恐怕打動他展指兒巾,就只遣他去。」玳安道:「娘錯怪了小的,頭裡娘分付教小的去,小的敢不去?若使來安下來,只說教一個家里去。」月娘道:「那來安小奴才,敢分付你?俺們恁大老婆,還不敢使你哩!如今但的你這奴才們,想有些摺兒也怎的!一來主子烟薰的佛像掛在牆上,有恁施主,有恁和尚?你說你恁行動,兩頭戳舌獻動出尖兒,外合裡表,奸懶食纔,奸消流水,背地瞞官作弊,幹的那繭兒,我不知道?頭裡你家主子沒使你送李桂兒家去,你怎的送他?人拿着毡包,你還匹甚手奪過去了。留丫頭不留丫頭不在你,使你進來說,你怎的不進來?你使就恁送他,裡面圖嘴吃去了,都使別人進來。須知我若罵,只罵那個人了,你還說你不久慣牢成?」玳安道:「這個也沒人,就是畫童兒過的舌。爹見他抱着毡包,教我:『你送送你桂姨去罷。』使了他進來時,娘說留丫頭,不留丫頭,不在於小的,小的管他怎的?」月娘大怒罵道:「賊奴才還要說嘴哩!我可不這裡閑着,和你犯牙兒哩!你這奴才脫脖倒坳過颺了。我使着不動,耍嘴兒!我就不信,到明日不對他說,把這欺心奴才打與他個爛羊頭也不筭!」吳大妗子道:「玳安兒,還不快替你娘們取皮襖去!他惱了。」又道:「姐姐,你分付他拿那裡皮襖與五娘穿?」潘金蓮接過來說道:「姐姐不要取去,我不穿皮襖。教他家裡捎了我的披襖子來我穿罷。人家當的赤色好也夕也,黃狗皮也似的,穿在身上教人笑話,也不氣長久,後還贖的去了。」月娘道:「這皮襖纔不是當,倒是當人李智少十六兩銀子,准折的皮襖。當的王招宣府裡那件皮襖,與李嬌兒穿了。」因分付玳安:「皮襖在大櫥裡,教玉筲尋與你,就把大姐的皮襖也帶了來。」那玳安把嘴谷都走出來。陳經濟問道:「你往那去?」玳安道:「精是攘氣的營生!一遍生活兩遍做。這咱晚又往家裡跑一遭。」逕走到家。西門慶還在大門首吃酒,傅夥計、雲主管都去了。還有應伯爵、謝希大、韓道國、賁四眾人吃酒未去。便問玳安:「你娘門來了?」玳安道:「沒來。使小的取皮襖來了。」說畢,便往後走。先是琴童到家。上房裡尋玉筲要皮襖。小玉坐在炕上,正沒好氣,說道:「四個淫婦今日都在賁四老婆家吃酒哩,我不知道皮襖放在那裡?往他家問他要去。」這琴童一直走到賁四家,且不叫,在窗外悄悄覷聽。只有賁四嫂說道:「大姑和二姑,怎的這半日酒也不上,菜兒也不揀一筯兒?嫌俺小家兒人家整治的不好吃也恁的?」春梅道:「四嫂,俺們酒勾了。」賁四嫂道:「耶嚛!沒的說。怎的這等上門兒怪人家?」又叫韓回子老婆:「便是我的切憐,就如東副東一樣,三姑、四姑根前酒,你也替我勸勸兒,怎的單拔?」叫長姐:「篩酒來,斟與三姑吃。你四姑鍾兒斟淺些兒罷。」蘭香道:「我自來吃不的。」賁四娘道:「你姐兒們今日受餓,沒甚麼可口的菜兒管待,休要笑話。今日要叫了先生來唱與姑娘們下酒,又恐怕爹那裡聽着。淺房淺屋,說不的俺小家兒人家的苦。」說着,琴童兒敲了敲門,眾人多不言語了。半日,只聽長兒問:「是誰?」琴童道:「是我,尋姐說話。」一面開了門,那琴童入來。玉筲便問:「娘來了?」那琴童看着待笑,平日不言語。玉筲道:「怪雌牙兒!」因問着:「你看雌的那牙!問着不言語。」琴童道:「娘們還在妗子家吃酒哩。見天陰下雪,使我來家取皮襖來,都教包了去哩。」玉筲道:「皮襖在外描金櫃子裡不是?叫小玉拿與你。」琴童道:「小玉說教我來問你要。」玉筲道:「你信那小淫婦兒。他不知道怎的!」春梅道:「你每有皮襖的,都打發與他。俺娘也沒皮襖,自我不動身。」蘭香對琴童:「你三娘皮襖,問小鸞要。」迎春便向腰裡拿鑰匙與琴童兒:「教綉春開裡間門拿與你。」那琴童兒走到後邊,上房小玉和玉樓房中小鸞都包了皮襖交與他。正拿着往外走,遇見玳安問道:「你來家做甚麼?」玳安道:「你還說哩,為你來了,平白教大娘罵了我一頓好的。又使我來取五娘的皮襖來。」琴童道:「我如今取六娘的皮襖去也。」玳安道:「你取了還在這裡等着,我一答兒裡去。你先去了不打緊,又惹的大娘罵我。」說畢,玳安來到上房,小玉正在炕上籠着爐臺拷火,口中磕瓜子兒。見了玳安問道:「原來你也來了。」玳安道:「你又說哩,受了一肚子氣在這裡。」于是把月娘罵他一節,前後訴說一遍:「着琴童取皮襖,嗔我不來,說我遣將兒。因為五娘沒皮襖,又教我來,去說大櫥裡有李三准折的一領皮襖,教拿與我去哩!」小玉道:「玉筲拿了裡間門上鑰匙,都在賁四家吃酒哩,教他來拿。」玳安道:「琴童往六娘房裡去取皮襖便來也,教他叫去,我且歇歇腿兒,拷拷火兒著。」那小玉便讓炕頭兒,與他並有相挨着向火。小玉道:「壺裡有酒,篩盞子你吃?」玳安道:「可知好哩,看你下顧!」小玉下來,把壺坐在火上,抽開抽梯,拿了一盞子臘鵝肉 ,篩酒與他。無人處,兩個就摟着咂舌親嘴。正吃着酒,只見琴童兒進來。玳安讓他吃了一盞子,便使他叫玉筲姐來,拿皮襖與五娘穿。那琴童把毡包放下,走到賁四家,叫玉筲。玉筲罵道:「賊囚根子,又來做甚麼?」又下來遞與鑰匙,教小玉開門。那小玉開了裡間房門,取了一把鑰匙,通了半日,白通不開,鎖了門。那玉筲道:「不是那個鑰匙,娘櫥裡鑰匙,在牀褥子座下哩。」小玉又罵道:「那淫婦丁子釘在人家不來,兩頭來回,只教使我著。」能開了櫥裡,又沒皮襖。琴童兒又往賁四家問去,來回走的抱怨了:「就死也死三日三夜,以省合氣!又撞者恁瘟死鬼小奶奶兒門,把人瘟也沒出了。」向玳安:「你說此回去,又惹的娘罵。不說屋裡鎖,只怪俺們!」走去又對玉筲說:「裡間娘櫥裡尋,沒有皮襖。」玉筲想了想笑道:「我也忘記,在外間大櫥裡。」到後邊,又被小玉罵道:「淫婦吃那野漢子搗昏了,皮襖在這裡都到處尋。」一面取出來,將皮襖包了,連大姐披襖,都交付與玳安、琴童兩個,拿到吳大妗子家。月娘又罵道:「賊奴才,你說同了,都不來罷了!」那玳安又不敢言語。琴童道:「娘的皮襖都有了,等着姐又尋這件青廂皮襖。」于是打開取出來。吳大妗子燈下觀看,說道:「也好一件皮襖,五娘你怎的說他不好?說是黃狗皮?那裡有恁黃狗皮,與我一件穿也罷了。」月娘道:「新新的皮襖兒,只是面前歇胸舊了些兒。到明日從新換兩遍地金歇胸,穿着就好了。」孟玉樓拿過來,與金蓮戲道:「我兒,你過來,你穿上這黃狗皮,娘與你試試看好不好?」金蓮道:「有本事明日問漢子要一件穿,也不枉的。平白拾了人家舊皮襖,來披在身上做甚麼?」玉樓戲道:「好個不認業的,人家有這一件皮襖,穿在身念佛。」于是替他穿上,見寬寬大大,潘金蓮纔不言語。當下吳月娘是貂鼠皮襖,孟玉樓與李瓶兒俱是貂鼠皮襖,都穿在身上,拜辭吳大妗子、二妗子起身。月娘與了郁大姐一包二錢銀子。吳銀兒道:「我這裡就辭了妗子、列位娘,磕了頭罷。」當下吳大妗子與了一對銀花兒,月娘與李瓶兒每人袖中摘去一兩銀子與他,磕頭謝了。吳大妗子同二妗子、鄭三姐,都還要送月娘眾人,因見天氣落雪,月娘阻回去了。琴童道:「頭裡下的還是雪,這回沾在身都是水珠兒,只怕濕了娘們的衣服。問妗子這裡討把傘打了家去。」吳二連忙取了傘來,琴童兒打着。頭裡兩個排軍打着燈籠,一簇男女跟了,走幾條小巷,到大街上。陳經濟路上放了許多花炮,因叫銀姐:「你家不遠了,俺們送你到家。」月娘便問:「他家去那裡?」經濟道:「這條衚衕內,一直進去,中間一座大門樓,就是他家。」那吳銀兒道:「我這裡就辭了娘們家去。」月娘道:「地下濕,姐家去了罷,頭裡已是見過禮了。我還着小廝送你到家。」因叫過玳安:「你送送銀姐家去。」經濟道:「娘,我與玳安兩個去罷。」月娘道:「也罷,姐夫你與他兩個同送他送。」那經濟得不的一聲,同玳安一路送去了。吳月娘眾人便回家來。潘金蓮路上說:「大姐姐,你原說咱每送他家去,怎的又不去了?」月娘笑道:「你也只是個小孩兒,哄你說着耍了兒,你就信了。皕春院裡那處是那裡?你我送去!」潘金道:「像人家漢子,在院裡嫖院來,家裡老婆沒曾住那裡尋去?尋出沒曾打成一鍋粥。」月娘道:「你來時兒,他爹到明日往院裡去,尋他尋試試;倒沒的丟人家漢子當粉頭拉了去,看你!」那兩個口兒裡說着,看看走東街口上,將近喬大戶門首。只見喬大戶娘子和他外甥媳婦段大姐,在門首站立,遠遠的見月娘這邊一簇男女過來,拉請月娘進去。月娘再三說道:「多謝親家盛情,天晚了,不進去罷!」那喬大戶娘子那裡肯放,說道:「好親家,你怎的上門兒怪人家?」強把月娘眾人拉進去了。客位內掛着燈,擺設酒菓,有兩個女兒彈唱飲酒不題。都說西門慶在家門首,與伯爵眾人飲酒,酒已將闌。先是伯爵與希大二人整吃了一日,頂顙吃不下去。見西門慶在樓子上打盹,趕眼錯把菓碟兒帶減碟都收拾了個淨光,倒在袖子裡,和韓道國就走了。只落下賁四,又不敢往屋裡去;直陪着西門慶打發了樂工酒來吃了,各都與了賞錢,打發出門。看着收了家火,滅息了燈燭,歸後邊去了。只見平安走來賁四家叫道:「姐們還不起身?爹進去了。」那春梅聽見,和迎春、玉筲等,慌的行回不顧,將拜了賁四嫂,辭的一溜烟跑了。只落下蘭香在後邊,別了鞋趕不上,罵道:「你們都搶棺材奔命哩!把人的鞋都別了,白穿不上。」到後邊打聽西門慶在李嬌兒房裡,都來磕頭。大師父見西門慶進入李嬌兒房中,都躲到上房和小玉在一處。玉筲進來,道了萬福。那小玉還說玉筲:「娘那裡使了小廝來要皮襖,你就不來管兒;教我來拿,我又不知那根鑰匙開櫥門。甫能開了,又沒有。落後都在外邊大櫥櫃裡尋出來。你放在裡頭,又昏搶了你不知道?姐姐們都乞勾來了罷,一個也曾見長出塊兒來。」那玉筲倒吃相的臉飛紅,便道:「怪小淫婦兒,如何狗撾了臉似的,人家不請你,怎的和俺每使性兒?」小玉道:「我稀罕那淫婦請!」大師父在傍勸道說:「姐姐們義讓一句兒罷,你爹在屋裡聽着。只怕你娘們來家,頓下些茶兒伺候著。」正說着,只見琴童抱進毡包來。玉筲便問:「娘來了?」琴童道:「娘們來了,又被喬親家娘在門首讓進去吃酒哩!也將好起身。」兩個纔不言語了。不一時,月娘等從喬大戶娘子家出來。到家門首,賁四娘子走出來廝見。陳經濟和賁四一面取出一架小烟火來,在門首又看放了一回烟火,方纔進來。眾人與李嬌兒、大師父道了萬福。雪蛾走來,向月娘根前磕了頭,與玉樓等三人見了禮。月娘因問:「他爹在那裡?」李嬌兒道:「剛纔在我那屋裡,我打發他睡了。」月娘一聲兒沒言語。只見春梅、迎春、玉筲、蘭香進來磕頭。李嬌兒便說:「今日前邊賁四嫂請了四個出去,坐了回兒就來了。」月娘聽了,半日沒言語,罵道:「恁成精狗肉們,平白去做甚麼?誰教他去來?」李嬌兒道:「問過他爹纔去來。」月娘道:「問他好有張主的貨,你家初一十五開的廟門早了,都放出些小鬼來了!」大師父道:「我的奶奶,恁四個上畫兒的姐姐,還說是小鬼?」月娘道:「上畫兒只畫兒半邊兒,平白放出做甚麼?與人家喂眼兒!」孟玉樓見月娘說來的不好,就先走了。落後金蓮見玉樓起身,和李瓶兒、大姐也走了。止落大師父和月娘同在一處睡了。那雪霰直下到四更方止。正是:

「香消燭冷樓臺夜,  挑菜燒燈掃雪天。」

一宿晚景題過。到次日西門慶往衙門中去了。月娘約飯時前後,與孟玉樓、李瓶兒三個,同送大師父家去。因在大門裡首站立,看見一個鄉里卜龜兒卦兒的老婆子,穿着水合襖,藍布裙子,勒黑包頭,背着搭褳,正從街上走來。月娘使小廝叫進來,在二門裡鋪下卦帖,安下靈龜,說道:「你卜卜俺們。」那老婆扒在地下,磕了四個頭:「請問奶奶多大年紀?」月娘道:「你卜個屬龍兒的女命。」那老婆道:「若是大龍兒四十二歲,小龍兒三十歲。」月娘道:「是三十歲了,八月十五日子時生。」那老婆把靈龜一擲,轉了一遭住了。揭起頭一張卦帖兒,上面畫着一個官人,和一位娘子在上面坐;其餘多是侍從人,也有坐的,也有立的,守着一庫金銀財寶。老婆道:「這位當家的奶奶是戊辰生。戊辰巳已大林木,為人一生有仁義,性格寬洪,心慈好善,有經佈施,廣行方便,一生操持,把家做活,替你頂缸受氣,還不道是喜怒有常,主下人不足,正是喜怒起來笑嘻嘻,惱將起來鬧哄哄。人睡到日頭半天還未起,你人早在堂前禁轉,梅香洗銚鐺。雖是一時風火性,轉眼卻無心,就和人說也有,笑也有。只是這位疾厄宮上,着刑星常沾些啾啷。吃了你這心好,濟過來了。往後有七十歲活哩。」孟玉樓道:「你看這位奶奶,命中有子沒有?」婆子道:「休怪婆子說。兒女宮上有些貴,往後只好招得出家的兒子送老罷了;隨你多少,也存不的。」玉樓向李瓶兒笑道:「就是你家吳應元見做道士家名哩。」月娘指着玉樓:「你也叫他卜卜。」玉樓道:「你卜個三十四歲的女命,十一月二十七日寅時生。」那婆子從新撇了卦帖,把靈龜一卜,轉到命宮上住了。揭起第二張卦帖來,上面畫着一個女人,配着三個男人,頭一個小帽商旅打扮,第二個穿紅官人,第三個是個秀才。也守着一庫金銀,有左右侍從人伏侍。婆子道:「這位奶奶,是甲子年生,甲子乙丑海中金,命犯三刑六害,夫主克剋過方可。」玉樓道:「己剋過了。」婆子道:「你為人溫柔和氣,好個性兒。你惱那個人也不知,喜歡那個人也不知,顯不出來。一生上人見喜下欽敬,為夫主寵愛。只一件,你饒與人為美,多不得人心。命中一生替人頂缸受氣,小人駁雜,饒吃了還不道你道你是。你心地好了去了,雖有小人,也拱不動你。」玉樓笑道:「剛纔為小廝討銀子,和爹亂了這回子,亂將出來,自我吃了都是頂缸受氣。」月娘道:「你看這位奶奶,往後有子沒有?」婆子道:「濟得好,見個女兒罷了,子上不敢許。若說壽,倒儘有。」月娘道:「你卜上這位奶奶。李大姐,你與他八字兒。」李瓶兒笑道:「我是屬羊的。」婆子道:「若屬小羊的,今年廿七歲,辛未年生的;生幾月?」李瓶兒道:「正月十五日午時。」那婆子卜轉龜兒,到命宮上矻磴住了。揭起卦帖來,上面畫着兩個娘子,三個官人。頭個官人穿紅,第二個官人穿綠,第三個穿青,懷着個孩兒,守着一庫金銀財寶,傍邊立着個青臉撩牙紅髮的鬼。婆子道:「這位奶奶庚午辛未路傍土,一生榮華富貴,吃也有,穿也有,所招的夫主,都是貴人。為人心地有仁義,金銀財帛不計較。人吃了轉了他的,他喜歡;不好吃不轉,他倒惱。只是吃了比肩不知的虧,凡事恩將仇報。正是比肩刑害亂擾擾,轉眼無情就放刁。寧逢虎摘三生路,休遇人前兩面刀。奶奶你休怪我說,你儘好疋紅羅,只可惜尺頭短了些,氣惱上要忍耐些,就是子上也難為。」李瓶兒道:「今已是寄名,做了道七。」婆子道:「既出了家,無妨了。又一件,你老人家今年計都星照命,主有血光之災。仔細七八月要見哭聲纔好。」說罷,李瓶兒袖中掏出五分一塊銀子,月娘和玉樓每人與錢五十文。剛打發卜龜卦婆子去了。只見潘金蓮和大姐從後邊出來,笑道:「我說後邊不見,原來你們都往前頭來了。」月娘道:「俺們剛纔送大師父出來,卜了這回龜兒卦。你早來一步,也教他與你卜卜兒也罷了。」金蓮拉頭兒道:「我是不卜他,常言:『筭的着命,筭不着行。』想着前日道士打看,說我短命哩!怎的哩?說的人心裡影影的。隨他明日街死街埋,路死路埋,倒在洋溝裡,就是棺材。」說畢,和月娘同歸後邊去了。正是:

「萬事不由人計較,  一生都是命安排。」

有詩為證:

「甘羅發早子牙遲,  彭祖顏回壽不齊;

范單家貧石崇富,  筭來各是只爭時。」

畢竟未知後來何如,且聽下回分解:




\end{showcontents}


