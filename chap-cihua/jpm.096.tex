%# -*- coding: utf-8 -*-
%!TEX encoding = UTF-8 Unicode
%!TEX TS-program = xelatex
% vim:ts=4:sw=4
%
% 以上设定默认使用 XeLaTex 编译,并指定 Unicode 编码,供 TeXShop 自动识别

%第九十六回 
\chapter{春梅遊舊家池館\KG 守備使張勝尋經濟}


\begin{showcontents}{}




「裏虛外實費張羅,  待客酬人使用多,

馬死奴逃難宴集,  臺傾樓倒罷笙歌;

租田稅店歸農主,  玩好金珠托賣婆,

欲向富家權借用,  當人開口奈羞何。」

話說光陰迅速,日月如梭。又早到正月二十一日。春梅和周守備說了,備一張祭卓,四樣羹果,一罈南酒 ,差家人周仁送與吳月娘。一者是西門慶三週年,二者是孝哥兒生日。月娘收了禮物,打發來人帕一方,銀三錢。這邊連忙就使玳安兒穿青衣,具請書兒請去。上寫著:

「重承厚禮,感感。即刻舍具菲酌,奉酬腆儀。仰希高軒俯臨,不外幸甚!

下書西門吳氏端肅拜請

大德周老夫人粧次。」

春梅看了,到日中纔來。戴著滿頭珠翠,金鳳頭面釵梳,胡珠環子,身穿大紅通袖,四獸朝麒麟袍兒,翠藍十樣錦百花裙,玉玎璫禁步,束著金帶腳下大紅繡花白綾高底鞋兒。坐著四人大轎,青段銷金轎衣,軍牢執藤棍喝道,家人伴當跟隨,檯著衣匣。後邊兩頂家人媳婦小轎兒,緊緊跟著大轎。吳月娘這邊請了吳大妗子相陪,又叫了兩個唱的女兒彈唱。聽見春梅來到,月娘亦盛粧縞素打扮,頭上五梁冠兒,戴著稀稀幾件金翠首飾,耳邊二珠環子,金〈扌塞〉領兒,上穿白綾襖,下邊翠藍段子織金拖泥裙。腳下穿玉色段高底鞋兒。與大妗子迎接至前廳。春梅大轎子抬至儀門首,纔落下轎來。兩邊家人圍着,到於廳上敘禮。向月娘插燭也似拜。月娘連忙答禮相見,沒口說道:「向日有累姐姐費心,粗尺頭又不肯受!今又重承厚禮祭卓,感激不盡!」春梅道:「惶恐!家官府沒甚麼。這些薄禮,表意而已!一向要請姥姥過去,家官府不一時出巡,所以不曾請得。」月娘道:「姐姐,你是幾時好日子?我只到那日,買禮看姐姐去罷。」春梅道:「奴賤日是四月廿五日。」月娘道:「奴到那日已定去。」兩個敘畢禮,春梅務要把月娘讓起,受了兩禮。然後吳大妗子相見,亦還下禮去。春梅道:「你看大妗子又沒正經!」一手扶起受禮。大妗子道:「姐姐,你今非昔比,折殺老身!」止受了半禮。面讓上坐,月娘和大妗子主位相陪。然後家人媳婦,丫鬟養娘,都來參見。春梅見了奶子如意兒抱著孝哥兒,吳月娘道:「小大哥,還不來與姐姐磕個頭兒,謝謝姐姐,今日來與你做個生日。」那孝哥兒真個下如意兒身來,扒與春梅唱諾。月娘道:「好小廝,不與姐姐磕頭,只唱諾?」那春梅連忙向袖中掏出一方錦手帕,一付金八吉祥兒,教替他〈扌塞〉帽兒上戴。月娘道:「又教姐姐費心!」又拜謝了。落後小玉,奶子來見,磕頭。春梅與了小玉一對金頭簪子,與了奶子兩枝銀花兒。月娘道:「姐姐你還不知,奶子與了來興兒做了媳婦兒了。來興兒那媳婦,害病沒了!」春梅道:「他一心要在咱家,倒也好!」一面丫鬟拿茶上來,吃了茶。月娘說:「請姐姐後邊明間內坐罷。這客位內冷。」春梅來後邊。西門慶靈前,又早點起燈燭,擺下卓面祭禮。春梅燒了布,落了幾點眼淚。然後周圍設放圍屏,火爐內生起炭火,安放大八仙卓席,擺茶上來。無非是細巧蒸酥,異樣甜食,美口菜蔬,希奇果品,縷金碟,象牙筯,雪錠盤盞兒,絕品芽茶。月娘和大妗子陪著吃了茶,讓春梅進上房裡換衣裳,脫了上面袍兒。家人媳婦,開衣匣取出衣服,更換了一套綠遍地錦粧花襖兒,紫丁香色遍地金裙。在月娘房中坐著,說了一回。月娘因問道:「哥兒好麼?今日怎不帶他來這裡走走?」春梅道:「若不是,也帶他來與姥姥磕頭。他爺說天氣寒冷,怕風冒著他。他又不肯在房裡,只要那當直的抱出來廳上外邊走。這兩日不知怎的,只是哭。」月娘道:「你出來他也不尋你?」春梅道:「左右有兩個奶子輪番看他,也罷了。」月娘道:「他周爺也好大年紀,得你替他養下這點孩子也夠了,也是你裙帶上的福!說他孫二娘還有位姐兒,幾歲兒了?」春梅道:「他二娘養的叫玉姐,今年交生四歲。俺這個叫金哥。」月娘道:「說他周爺身邊,還有兩位房裡姐兒?」春梅道:「是兩個學彈唱的丫頭子,都有十六七歲。成日淘氣在那裡!」月娘道:「他爺也常往他身邊去不去?」春梅道:「奶奶,他那裡得工夫在家?多在外,少在裏。如今四外,好不盜賊生發,朝廷勅書上,又教他兼管許多事情,鎮守地方,巡理河道,提拏盜賊,操練人馬。常不時往外出巡幾遭,好不辛苦哩!」說畢,小玉拿茶來吃了。春梅向月娘說:「姥姥你引我往俺娘那邊花園山子下走走。」月娘道:「我的姐姐,山子花園,還是那咱的山子花園哩!自從你爹下世,沒人收拾他。如今丟搭的破零二落,石頭也倒了,樹木也死了,俺等閒也不去了!」春梅道:「不妨,奴就往俺娘那邊看看去。」這月娘強不過,只得教小玉拿花園門山子門鑰匙,開了門。月娘、大妗子,陪春梅眾人到裡面遊看了半日:

「垣墻欹損,臺榭歪斜。兩邊畫壁長青苔,備地花磚生碧草。山前怪石遭塌毀,不顯嵯峨;亭內涼床被滲漏,已無框檔。石洞口蛛絲結網,魚池內蝦蟆成群。狐狸常睡臥雲亭,黃鼠往來藏春閣。料想經年人不到,也知盡日有雲來。」

春梅看了一回,先走到李瓶兒那邊。見樓上丟著些折卓壞凳破椅子,下邊房都空鎖著。地下草長的荒荒的。方來到他娘這邊,樓上還堆烏生藥香料,下邊他娘房裡,止有兩座廚櫃,床也沒了。因問小玉:「俺娘那張床往那去了?怎的不見?」小玉道:「俺三娘嫁人,賠了俺三娘去了。」月娘走到根前說:「因有你爹在日,將他帶來那張八步床,賠了大姐在陳家。落後他起身,卻把你娘這張床,賠了他嫁人去了。」春梅道:「我聽見大姐死了。對你老人家說把床還抬的來家了。」月娘道:「那床沒錢使,只賣了八兩銀子。打發縣中皂隸都使了。」春梅聽言,點了頭兒。那星眼中,由不的酸酸的。口內不言,心下暗道:「想著俺娘那咱爭強不伏弱的,問爹要買了這張床。我實承望要回了這張床去,也做他人家一念兒!不想又與了人去了!」由不的心下慘切。又問月娘:「俺娘那張螺甸床,怎的不見?」月娘道:「一言難盡。自從你爹下世,日逐只有出去的,沒有進來的。常言:『家無營活計,不怕斗量金!』也是家中沒盤纏,抬出交人賣了。」春梅問:「賣了多少銀子?」月娘道:「止賣了三十五兩銀子。」春梅道:「可惜了的。那張床,當初我聽見爹說,值六十兩多銀子,只賣這些兒!早知你老人家打發,我倒與你老人家三四十兩銀子,我要了也罷!」月娘道:「好姐姐,諸般都有,人沒早知道的!」一面嘆息了半日。只見家人周仁走來接:「爹請奶奶早些家去,哥兒尋奶奶哭哩!」這春梅就抽身往後邊。月娘教小玉鎖了花園門,同來到後邊明間內,又早屏開孔雀,簾控鮫綃,擺下酒筳。兩個妓女,銀箏琵琶,在旁彈唱。吳月娘遞酒安席,不必細說。安春梅上坐,春梅不肯,務必拉大妗子同他一處坐的。月娘主停,筵前遞了酒。湯飯點心,割切上席。春梅教家人周仁賞了廚子三錢銀子。說不盡盤堆異品,酒泛金波。當下傳盃換盞,吃至日色將落時分。只見宅內又差伴當,拏燈籠來接。月娘那裡肯放,教兩個妓女,在根前著彈唱勸酒。分付:「你把好曲兒,孝順你周奶奶一個兒。」一面叫小玉斟上大鍾,放在根前,教春梅吃:「姐姐,你分付個心下愛的曲兒,教他兩個唱與你聽下酒。」春梅道:「姥姥,奴吃不得的,怕孩兒家中尋我。」月娘道:「哥兒尋,左右有奶子看著。天色也還早哩,我曉得你好小量兒!」春梅因問那兩個妓女:「你叫甚名字?是誰家的?」兩個跪下說:「小的一個是韓金釧兒妹子韓玉釧兒,一個是鄭愛香兒姪女鄭嬌兒。」春梅道:「你每會唱懶畫眉不會?」玉釧兒道:「奶奶分付,小的兩個都會。」月娘道:「你兩個既會唱,斟上酒你周奶奶吃,你每慢唱。」小玉在旁連忙斟上酒。兩個妓女一個彈箏,一個琵琶,唱道:

「冤家為你幾時休,捱過春來又到秋。誰人知道我心頭,天害的我伶仃瘦!聽的音書兩淚流,從前已往訴緣由。誰想你無情把我丟!」

那春梅吃過,月娘又令鄭嬌兒遞上一盃酒與春梅。春梅道:「你老人家也陪我一盃。」兩家於是都齊斟上,兩個妓女又唱道:

「冤家為你喊風流,鵲噪簷前不肯休。死聲活氣沒來由,天倒惹的情拖逗!助的凄涼兩淚流,從他去後意無休。誰想你辜恩把我丟!」

春梅說:「姥姥,你也教大妗子吃盃兒。」月娘道:「大妗子吃不的,教他拏小鍾兒陪你罷。」一回令小玉斟上大妗子一小鍾兒酒。兩個妓女又唱道:

「冤家為你惹場憂,坐想行思日夜愁。香肌憔瘦減溫柔,天要見你不能勾!悶的我傷心兩淚流,從前與你共綢繆。誰想你今番把我丟!」

當下春梅見小玉在根前,也斟了一大鍾教小玉吃。月娘道:「姐姐,他吃不的。」春梅道:「姥姥,他也吃兩三鍾兒。我那咱在家裡沒和他吃。」于是斟上,教小玉也吃了一盃。妓女唱道:

「冤家為你惹閒愁,病枕著床無了休。滿懷憂悶鎖眉頭,天忘了還依舊!助的我腮邊兩淚流,從前與你共綢繆。誰想你經年把我丟!」

看官聽說:當時春梅為甚教妓女唱此詞?一向心中牽掛陳經濟在外,不得相會。情種心苗,故有所感,發於吟咏。又見他兩個唱的,好口兒甜乖覺,奶奶長、奶奶短侍奉,心中歡喜。叫家人周仁近前來,拏出兩包兒賞賜來,每人二錢銀子。兩個妓女放下樂器,插燭也似磕頭,謝了賞賜。不一時春梅起身,月娘款留不住,伴當打燈籠,拜辭出門,坐上大轎,家人媳婦都坐上小轎,前後打著四個燈籠,軍牢喝道而去。正是:

「時來頑鐵有光輝,  運去黃金無艷色!」

有詩為證:

「點絳唇紅弄玉嬌,  鳳凰飛下品鸞簫;

堂前高把湘簾捲,  燕子還來續舊巢。」

且說春梅自從來吳月娘家赴席之後,因思想陳經濟,不知流落在何處,歸到府中,終日只是臥床不起,心下沒好氣。守備察知其意,說道:「只怕思念你兄弟不得其所?」一面叫將張勝、李安來分付道:「我一向委你尋你奶奶兄弟,如何不用心找尋?」二人告道:「小的一向找尋來,一地里尋不著下落。已回了奶奶話了。」守備道:「限你二人五日,若找尋不著,討分曉!」這張勝、李安領了鈞語下來,都帶了愁顏,沿街遶巷,各處留心找問不題。話分兩頭,單表陳經濟自從守備府中打了出來,欲投晏公廟。聽見人說:「你師父任道士,因為你宿娼壞事,被人打了,拏在守備府去。查點房中箱籠,東西銀兩沒了。一口重氣,半夜就死了。你還敢進廟中去,眾徒弟就打死你!」這經濟害怕,就不敢進廟來。又沒臉見杏庵玉老。白日裡到處里打油飛,夜晚間還鑽入冷舖中存身。一日也是合當有事。經濟正在街上站立,只見鐵指甲楊大郎頭戴新羅帽兒,身穿白綾襖子,玄色段氅衣,沉香色襪口,光素琴鞋,騎著一疋驢兒,揀銀鞍轡,一個小廝跟隨,正行街心走過來。經濟認的是楊光彥,便向前一把手把嚼環拉住,說道:「楊大哥,一向不見!咱兩個同做朋友,往下江販布。船在清江浦泊著,我在嚴州府探親,吃人陷害,打了一場官司,你就不等我?把我半船貨物,偷拐走的不知去向,我好意往你家問,反吃你兄弟楊二風拏瓦楔礸破頭,趕著打上我家門來!今日弄的我一貧如洗,你是會搖擺受用!」那楊大郎見了經濟討吃,佯佯而笑說:「如今晦氣,出門撞見瘟死鬼!量你這餓不死賊花子,那裡討半船貨,我拐了你的來了,你不撒手,須吃我一頓好馬鞭子!」那經濟便道:「我如今窮了。你有銀子,與我些盤纏。不然,咱到了去處!」楊大郎見他不放,跳下驢來,向他身上也抽了幾鞭子。喝令小廝:「與我撏了這少死的花子去!」那小廝使力把經濟推了一交。楊大郎又向前踢了幾腳,踢打的經濟怪叫。須臾圍了許多人。旁邊閃過一個人來,青高裝帽子,勒著手帕,倒披紫襖,白布〈衤旋〉子,精著兩條腳,數著蒲鞋。生的阿兜眼,掃帚眉,料綽口,三鬚鬍子,面上紫肉橫生,手腕橫勇兢起。吃的楞楞睜睜,提著拳頭,向楊大郎說道:「你此位哥,好不近理!他年少這般貧寒,你只顧打他怎的?自古嗔拳不打笑面!他又不曾傷犯著你,你有錢,看平日相交,與他些。沒錢,罷了。如何只顧打他?自古路見不平,也有向燈向火!」楊大郎說:「你不知,他賴我我拐了他半船貨。量他恁窮嘴臉,有半船貨物?」那人道:「想必他當時也是根基人家娃娃,天生就這般窮來?閣下就到這般有錢?老兄依我,你有銀子,與他盤纏罷。」那楊大郎見那人說了,袖內汗巾兒上,拴著四五錢一塊銀子,解下來遞與經濟。與那人舉一舉手兒,上驢子揚長去了。經濟地下扒起來,抬頭看那人時,不是別人,卻是舊時同在冷舖內,和他一舖睡的土作頭兒飛天鬼侯林兒。近來領著五十名人,在城南水月寺,曉月長老那裡做工,起蓋伽藍殿。因一隻手拉著經濟說道:「兄弟,剛纔若不是我拏幾句言語譏犯他,他肯拏出這五錢銀子與你。他賊都知見範,他若不知範時,好不好吃我一頓好拳頭!你跟著我,咱往酒店內吃酒去。」來到一個食葷小酒店內,案頭上坐下。叫量酒拏四賣嗄飯,兩大壼酒來。不一時,量酒打抹條卓乾淨,擺下小菜嗄飯。四盤四碟,兩大坐壼時興橄欖酒。不用小盃,拏大磁甌子。因問經濟:「兄弟,你吃麵吃飯?」量酒道:「麵是溫淘,飯是白米飯。」經濟道:「我吃麵。」須臾,掉上兩三碗濕麵上來。侯林兒只吃一碗,經濟吃了兩碗,然後吃酒。侯林兒向經濟說:「兄弟,你今日跟我往坊子裡睡一夜。明日我領你城南水月寺曉月長老那裡,修蓋伽藍殿,并兩廊僧房。你哥率領著五十名做工。你到那裡,不要你做重活,只抬幾筐土兒就是了,也算你一工,討四分銀子。我外邊賃著一間廈子,晚夕咱兩個就在那裡歇。做些飯,打發咱的人吃。問你一把鎖鎖了,家都交與你,好不好?強如你在那冷舖中替花子搖鈴打梆子。這個還官樣些!」經濟道:「若是哥哥這般下顧兄弟,可知好哩!不知這工程做的長遠不長遠?」侯林兒道:「纔做了一個月。這工程做到十月裏,不知完不完。」兩個說話之間,你一鍾我一盞,把兩大壼酒都吃了。量酒算帳,該一錢三分半銀子。經濟要會銀子,拏出銀子來秤。侯林兒推過一邊,說:「傻兄弟,莫不教你出錢,哥有銀子在此。」一面扯出包兒來,秤了一錢五分銀子與掌櫃的,還找了一分半錢袖了。搭伏著經濟肩背,同到坊子裏,兩個在一處歇臥。二人都醉了。這侯林兒晚夕幹經濟後庭花,足幹了一夜。親哥,親達達,親漢子,親爺,口裡無般不叫將出來。到天明,城南水月寺,果然寺外侯林兒賃下半間廈子。裡面燒著炕柴皁,也買下許多碗盞家活。早晨上工,叫了名字。眾人看見經濟不上二十四五歲,白臉子,生的眉目清俊,就知是侯林兒兄弟,都亂訝戲他。先問道:「那小夥子兒,你叫甚名字?」陳經濟道:「我叫陳經濟。」那人道:「陳經濟,可不由著你就擠了!」又一人說:「你恁年小小的,原幹的這營生?挨的這大扛頭子?」侯林兒喝開眾人罵:「怪花子,你只顧傒落他怎的?」一面散了鍬鐝筐扛,派眾人抬土的抬土,和泥的和泥,打禡的打禡。原來曉月長老教一個葉頭陀做火頭,造飯與各作匠人吃。這葉頭陀年約五十歲,一個眼瞎,穿著皁直裰,精著腳,腰間束著爛絨縧。也不會看經,只會念佛,善會麻衣神相。眾人都叫他做葉道。一日做了工下來,眾人都吃畢飯,閒坐的、站的,也有蹲著的。只見經濟走向前,問葉頭陀討茶吃。這葉頭陀只顧上上下下看他,內有一人說:「葉道,這個小夥子兒是新來的。你相他一相。」又一人說:「你相他相,倒相個兄弟?」一人說:「倒相個二尾子!」葉頭陀教他近前,端詳了一回,說道:「色怕嫩兮又怕嬌,聲嬌氣嫩不相饒!老年色嫩招辛苦,少年色嫩不堅牢!只吃了你面嫩的虧!一生多得陰人寵愛。八歲十八二十八,下至山根上至髮,有無活計兩頭消,三十印堂莫帶煞!眼光帶秀心中巧,不讀詩書也可人。做作百般人可愛,縱然弄假不成真。休怪我說,一生心伶機巧,常得陰人發跡。你今年多大年紀?」經濟道:「我二十四歲。」葉道道:「虧你前年怎麼打過來!吃了你印堂太窄,子喪妻亡,懸壁昏暗,人亡家破;唇不蓋齒,一生惹是招非;鼻若竈門,家私傾喪。那一年遭官司口舌,傾家喪業,見過不曾?」經濟道:「都見過了。」葉頭陀道:「又一件,你這山根不宜斷絕。麻衣祖師說得兩句好:『山根斷兮早虛花,祖業飄零定破家!』早年父祖丟下家產,不拘多少,到你手裡都了當了!你上停短兮下停長,主多成多敗,錢財使盡又還來。總然你久後營得成家計,猶如烈日照冰霜!你走兩步我瞧。」那經濟真個走了兩步。葉頭陀道:「頭先過步,初主好而晚景貧窮;腳不點地,賣盡田園而走他鄉,一生不守祖業。你往後好有三妻之命,剋過一個妻宮不曾?」經濟道:「已剋過了。」葉頭陀道:「後來還有三妻之會。你面若桃花光焰,雖然子遲,但圖酒色歡娛,但恐美中不美。三十上小人有些不足,花柳中少要行走,還計較些。」一個人說:「葉道,你相差了!他還與人家做老婆,他那有三個妻來?」眾人正笑做一團。只聽得曉月長老打梆子,各人都拏鍬鐝筐扛,上工做活去了。如此者經濟在水月寺也做了約一月光景。一日,三月中旬天氣,經濟正與眾人抬出土來,在寺山門墻下倚著墻根,向日陽蹲踞著,捉身上風蟣。只見一個人頭戴萬字頭巾,腦後撲匾金環,身穿青窄衫,紫裹肚,腰繫纏帶,脚穿【革扁】靴,騎著一疋黃馬,手中提著一籃鮮花兒,見了經濟,猛然跳下馬來,向前深深的唱了喏,便叫:「陳舅,小人那裡沒處尋?你老人家原來在這裡!」倒諕了經濟一跳,連忙還禮不迭。問:「哥哥,你是那裡來的?」那人道:「小人是守備周爺府中親隨張勝。自從舅舅那日府中官事出來,奶奶不好,直到如今。老爺使小人,那裡不曾找尋舅舅,不知在這裡!今早不是俺奶奶使小人往外庄上折取這幾朵芍藥花兒,打這裡經過,怎得看見你老人家在這裡?一來也是你老人家際遇,二者小人有緣!不消猶豫,就騎上馬,跟你老人家往府中去。」那眾做工的人看著,都面面相覷,不敢做聲。這陳經濟把鑰匙遞與侯林兒,騎上馬,張勝緊緊跟隨,逕往守備府中來。正是:

「良人得意正年少,  今夜月明何處樓!」

有詩為證:

「白玉隱於頑石裡,  黃金埋在污泥中;

今朝貴人提拔起,  如立天梯上九重。」

畢竟未知後來如何,且聽下回分解:





\end{showcontents}


