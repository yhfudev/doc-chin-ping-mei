%# -*- coding: utf-8 -*-
%!TEX encoding = UTF-8 Unicode
%!TEX TS-program = xelatex
% vim:ts=4:sw=4
%
% 以上设定默认使用 XeLaTex 编译,并指定 Unicode 编码,供 TeXShop 自动识别

%第八十六回 
\chapter{雪娥唆打陳經濟\KG 王婆售利嫁金蓮}

「人生雖未有十全,  處事規模要放寬,

好事但看君子語,  是非休听小人言;

但看世俗如幻戲,  也畏人心似隔山,

寄與知音女娘道,  莫將苦處認為甜。」

話說潘金蓮自從春梅出去,房中納悶不題。單表陳經濟次日早飯時出去,假作討帳,騎頭口到於薛嫂兒家。薛嫂兒正在屋裡,一面讓進來坐。經濟拴了頭口,進房坐下,點茶吃了。春梅在裡間屋裡,不出來。薛嫂故意問:「姐夫來有何話說?」經濟道:「我往前街討帳,竟到這裡。昨晚小大姐出來了,在你這裡?」薛嫂道:「是在我這裡,還未上主兒哩。」經濟道:「在這裡,我要見他,和他說句話兒。」薛嫂故作喬張致,說:「好姐夫,昨日你家丈母好不分付我。因為你們通同作弊,弄出醜事來,纔被他打發出門。教我防範你們,休要與他會面說話。你還不趁早去哩,只怕他一時使將小廝來看見,到家學了,又是一場兒,倒沒的弄的我也上不的門。」那經濟便笑嘻嘻袖中拏出一兩銀子來:「權作一茶,你且收了,改日還謝你。」那薛嫂見錢眼開,說道:「好姐夫,自恁沒錢使,將來謝我?只是我去年臘月,在你舖子當了人家兩付扣花枕頂,將有一年來,本利該八錢銀子,你討與我罷。」經濟道:「這個不打緊,明日就尋與你。」這薛嫂兒一面請經濟裡間房裡去與春梅廝見,一面叫他媳婦金大姐定菜兒:「我去買茶食點心。」又打了一壺酒,并肉鮓之類,教他二人吃。這春梅看見經濟,說道:「姐夫,你好人兒,就是個弄人的劊子手,把俺娘兒兩個,弄的上不上,下不下,出醜惹人嫌到這步田地!」經濟道:「我的姐姐,你既出了他家門,我在他家也不九了。妻兒趙迎春,各自尋投奔。你教薛媽替你尋個好人家去罷。我醃韮已是入不的畦了!我往東京俺父親那裡去,計較了回來,把他家女兒休了,只要我家寄放的箱子。」說畢,不一時,薛嫂買將茶食酒菜來,放炕卓兒擺了。兩個做一處飲酒敘話。薛嫂也陪他吃了兩盞,一遞一句,說了回月娘心狠:「宅裏恁個出色姐兒出來,通不與一件兒衣服簪環!就是往人家上主兒去,裝門面也不好看。還要舊時原價,就是清水,這碗裏傾倒那碗內,也拋撒些兒!原來這等央腦風!臨時出門,倒虧了小玉丫頭做了個分上,教他娘拏了兩件衣服與他。不是,往人家相去,拏甚麼做上蓋?」比及吃得酒濃時,薛嫂教他媳婦金大姐,抱孩子躲去人家坐的。教他兩個在裏間自在坐個房兒。正是:

「雲淡淡天邊鸞鳳,  水沉沉波底鴛鴦;

寫成今世不休書,  結下來生歡喜帶」

兩個幹訖一度,作別。比時難割難捨。薛嫂恐怕月娘使人來瞧,連忙攛掇經濟出港,騎上頭口來家遲不上兩日,經濟又稍了兩方銷金汗巾,兩雙膝褲與春梅,又尋枕頂出來與薛嫂兒。拏銀子打酒,在薛嫂兒房內,正和春梅吃酒。不想月娘使了來安小廝來,來催薛嫂兒:「怎的還不上主兒?」看見頭口栓在門首,來安兒到家學了舌,說:「姐夫也在那裏來。」這月娘听了,心中大怒。使人一替兩替,叫了薛嫂兒去,儘力數說了一頓:「你領了奴才去,今日推明日,明日推後日,只顧不上緊替我打發,好窩藏着養漢,掙錢兒與你家使。若是你不打發,把丫頭還與我領了來,我另教馮媽媽子賣,你再休上我門來!」這薛嫂兒听了,到底還是媒人的嘴,恨不的生出七八個口來,說道:「天麼!天麼!你老人家怪我差了,我趕着增福神着棍打,你老人家照顧我,怎不打發?昨日也領着走了兩三個主兒,都出不上。你老人家要十六兩原價,俺媒人家那裏有這些銀子賠上?」月娘道:「小廝說陳家種子,今日在你家和丫頭吃酒來?」薛嫂慌道:「耶嚛耶嚛!又是一場兒!還是去年臘月,當了人家兩付枕頂,在咱家獅子街舖內,銀子收了,今日姐夫送枕頭與我,我讓他吃茶,他不吃,忙忙就上頭口來了。幾時進屋裡吃酒來?原來咱家這大官兒,恁快搗謊駕舌!」月娘吃他一篇說的不言語了。說道:「我只怕一時被那種子設念隨邪,差了念頭。」薛嫂道:「我是三歲小孩兒,豈可恁些事兒不知道?你那等分付了我,我長吃好,短吃好!他在那裏,也沒得久停久坐,與了我枕頭,茶也沒吃就來了。幾曾見咱家小大姐見面兒來?萬物也要個真實,你老人家就數落我起來!既是如此,如今守備周爺府中,要他圖生長,只出十二兩銀子。看他若添到十三兩上,我兌了銀子來罷!說起來,守備老爺,前者在咱家酒席上,也曾見過小大姐來。因他會這幾套唱,好模樣兒,纔出這幾兩銀子;又不是女兒,其餘別人出不上。」這薛嫂當下和月娘砧死了價錢。次日早,把春梅收拾打扮粧點起來。戴着圍髮雲髻兒,滿頭珠翠,穿上紅段襖兒,下着藍段裙子,腳上雙彎尖趫趫,一頂轎子,送到守備府中。周守備見了春梅生的模樣兒,比舊時越好,又紅又白,身段兒不短不長,一對小腳兒,滿心歡喜。就兌出五十兩一錠元寶來。這薛嫂兒拏來家,鑿下十三兩銀子,往西門慶家交與月娘。另外又拏出一兩來,說:「是周爺賞我的喜錢。你老人家不與我些兒?」那吳月娘只得免不過,又秤出五錢銀子與他,恰好他還禁了三十七兩五錢銀子。十個九個媒人,都是如此轉錢養家。都表陳經濟見賣了春梅,又不得往金蓮那邊去。見月娘凡事不理他,門戶都嚴緊。到晚夕,親自出來,打燈籠前後照看了,方纔關後邊儀門,夜裏上鎖,方纔睡去。因此弄不得手腳,十分急了,先和西門大姐嚷了兩場,淫婦前淫婦後罵大姐:「我在你家做女婿,不道的雌飯吃吃傷了!你家都收了我許多金銀箱籠,你是我老婆,不顧贍我,反說我雌你家飯吃!我白吃你家飯來?」罵的大姐只是哭涕。十一月廿七日,孟玉樓生日。玉樓安排了幾碟酒菜點心,好意教春鴻拿出前邊舖子,教經濟陪傅夥計吃。月娘便攔說:「他不是材料,休要理他!要與傅夥計,自與傅夥計自家吃就是了,不消叫他。」玉樓不肯。春鴻拿出來,擺在水櫃上,一大壺酒都吃不勾。又使來安兒後邊要去。傅夥計便說:「姐夫,不消要酒去了。這酒勾了。我也不吃了。」經濟不肯,定教來安要去。等了半晌,來安兒出來,回說沒了酒了。這陳經濟也有半酣酒兒在肚內。經濟又使他要去,那來安不動。又另拏錢打了酒來,吃着罵來安兒:「賊小奴才兒,你別要慌!你主子不待見我,連你這奴才們也欺負我起來了!使你使兒不動。我與你家做女婿,不道的酒肉吃傷了。有爹在,怎麼行來?今日爹沒了,就改變了心腸,把我來不理,都亂來擠撮我!我大丈母听信奴才言語,反防範我起來!凡事托奴才不托我。由他,我好耐驚耐怕兒!」傅夥計勸道:「好姐夫,快休舒言。不敬奉姐夫,再敬奉誰?想必後邊忙,怎不與姐夫吃?你罵他不打緊,墻有縫,壁有耳,恰似你醉了一般!」經濟道:「老夥計,你不知道。我酒在肚裏,事在心頭!俺丈母听信小人言語,駕我一篇是非,就算我{入日}了人,人沒{入日}了我!好不好我把這一屋子裏老婆,都刮剌了,到官也只是後丈母通姦,論個不應罪名!如今我先把你家女兒休了,然後一紙狀子告到官;再不,東京萬壽門進一本,你家見收着我家許多金銀箱籠,都是楊戩應沒官賍物。好不好,把你這幾間業房子,都抄沒了,老婆便當官變賣!我不圖打魚,只圖混水耍子!會事的,把俺女婿須收籠著,照舊看待,還是大鳥便益!」傅夥計見他話頭兒來的不好,說道:「姐夫,你原來醉了。王十九自吃酒,且把散話革起。」這經濟睜眼瞅着傅夥計,便罵:「賊老狗,怎的說我散話揭起?我醉了,吃了你家酒了?我才不是他家女婿嬌客,你無故只是他家行財。你也擠撮我起來?我教你這老狗,別要慌,你這幾年轉的俺丈人錢勾了,飯也吃飽了。心裡要打夥兒把我疾發了去,要獨權兒做買賣,好禁錢養家。我明日本狀也帶你一筆,教你跟著打官司!」那傅夥計最是個小膽兒的人,見頭勢不好,穿上衣裳悄悄往家一溜烟走了。小廝收了家活,後邊去了。經濟倒在炕上睡下。一宿晚景題過。次日,傅夥計早辰進後邊見月娘,把前事具訴一遍,哭哭啼啼,要告辭家去,交割帳目,不做買賣了。月娘便勸道:「夥計,你只安心做買賣,休要理那潑才料,如臭屎一般丟着他!當初你家為官事,投到俺家來權住著,有甚金銀財寶?也只是大姐幾件粧奩,隨身箱籠。你家老子便躲上東京去了,教俺家那一個不恐怕小人不足,晝夜耽憂的那心!你來時纔十六七歲,黃毛團兒也一般。也虧在丈人家養活了這幾年,調理的諸般買賣兒都會。今日翅膀毛兒乾了,反恩將仇報,一掃葦掃的光光的!小孩兒家說話欺心,恁沒天理,到明日只天照著他。夥計,你自安心做你買賣,休理他便了!他自然也羞。」一面把傅夥計安撫住了不題。一日,也是合當有事。印子舖擠著一屋子人,贖討東西。只見奶子如意兒抱著李哥兒,送了一壺茶來,與傅夥計吃,放在卓上。孝哥兒在奶子懷裡,哇哇的只管哭。這陳經濟對著那些人,作耍當真說道:「我的哥哥乖乖兒,你休哭了!」向眾人說:「這孩子倒相我養的,依我說話。教他休哭,他就不哭了!」那些人就呆了。如意兒說:「姐夫,你說的好妙話兒,越發叫起兒來了,看我進房裏說不說!」這陳經濟趕上,踢了奶子兩腳,戲罵道:「怪賊邋遢,你說不是?我且踢個響屁兒著。」那奶子抱孩子走到後邊,如此這般,向月娘哭說經濟對眾人將哥兒這般言話發出來。這月娘不听便罷,听了此言,正在鏡臺邊梳著頭,半天說不出話來。往前一撞,就昏倒在地,不省人事。但見:

「荊山玉損,可惜西門慶正室夫妻,寶鑑花殘,枉費九十日東京匹配!花容淹淡,猶如西園芍藥倚朱欄;檀口無言,一似南海觀音來入定。小園昨日春風急,吹折江梅就地托。」

慌了小玉,叫將家中大小,扶起月娘來炕上坐的。孫雪娥跳上炕,獗救了半日,舀姜湯灌下去,半日甦醒過來。月娘氣堵心胸,只是哽咽,哭不出聲來。奶子如意兒,對孟玉樓、孫雪娥說經濟對腳人將哥兒戲言之事,說了一遍:「我好意說他,又趕著我踢了兩腳。把我也氣的發昏在這裏!」雪娥扶著月娘,待的眾人散去,悄悄在房中對月娘說:「娘也不消生氣。氣的你有些好歹,越發不好了!這小廝因賣了春梅,不得與潘家那淫婦弄手腳,纔發出話來。如今一不做二不休,大姐已是嫁出女,如同賣出田一般,咱顧不的他這許多。常言:『養蝦蟆得水蠱兒病。』只顧教那小廝在家裏做甚麼?明日哄賺進後邊,老實打與他一頓,即時趕了離門,教他家去。然後叫將王媽媽子;來是是非人,去時是非者,把那淫婦教他領了去,變賣嫁人。如同狗屎臭尿,掠將出去,一天事都沒了!平空留著他在屋裏做甚麼?到明日沒的把咱們也扯下水去了!」月娘道:「你說的也是。」當下計議已定了。到次日飯時已後,月娘埋伏下丫鬟媳婦七八個人,各拏短棍棒槌,使小廝來安兒誆進陳經濟來後邊,只推說話,把儀門關了,教他當面跪著,問他:「你知罪麼?」那陳經濟也不跪,還似每常臉兒高揚。月娘便道,有長詞為證:

「起初時,月娘不觸犯,龐兒變了。次則陳經濟耐搶白,臉而揚著,不消你枉話兒絮叨叨,須和你討個分曉。月娘道:「此是你丈人深宅院,又不是麗春院,鶯燕巢,你如何把他婦女廝調?他是你丈人愛妾,寡居守孝。你因何把他戲嘲?也有那沒廉恥斜皮,把你刮剌上了。自古母狗不掉尾,公狗不跳槽。都是些污家門罪犯難饒!」陳經濟道:「閃出夥縛鍾馗母妖,你做成這慣打姦夫的圈套。我臀尖難禁這頓拷,梅香休鬧,大娘休焦,險些不大棍無情打折我腰。」月娘道:「賊才料,你還敢嘴兒挑!常言:『冰厚三不是一日惱。最恨無端難恕饒!』虧你呵,再倘著筒兒滿棒剪稻!你再敢不敢,我把你這短命王鸞兒割了,教你直孤到老!」

當下月娘率領雪娥,并來興兒媳婦、來昭妻一丈青、中秋兒、小玉、綉春眾婦人,七手八腳,按下地下,拏棒槌短棍,打了一頓。西門大姐走過一邊,也不來救。打的這小夥兒急了,把褲子脫了,露出那直堅一條棍來,諕的眾婦女看見,都丟下棍棒亂跑了。月娘又是那惱,又是那羞,口裡罵道:「好個沒根基的王八羔子!」經濟口中不言,心中暗道:「若不是我這個好法兒,怎得脫身?」於是扒起來,一手兜著褲子,往前走了。月娘隨令小廝跟隨,教他算帳,交與傅夥計。經濟自然也知立不住,一面收拾衣服舖蓋,也不作辭,使性兒一直出離西門慶家,逕往他母舅張團練住的他舊房子內住去了。正是:

「自古感恩并積恨,  萬年千載不成塵。」

潘金蓮在房中,聽見打了經濟,趕離出門去了。越發憂上加憂,悶上加悶。一日,月娘聽信雪娥之言,使玳安去叫王婆子來。那王婆自從他兒子王潮兒,跟淮上客人,拐了起車的一百兩銀子來家,得其發跡,也不賣茶了。買了兩個驢兒,安了盤磨,一張羅櫃,開起磨房來。聽見西門慶宅裡叫他,連忙穿衣就走。到路上問玳安說:「我的哥哥,幾時沒見你,又早籠起頭去了。有了媳婦兒不曾?」玳安道:「還不曾有哩。」王婆子道:「你爹沒了,你家誰人請我?做甚麼?莫不是你五娘養了兒子了,請我去抱腰?」玳安道:「俺五娘倒沒養兒子,倒養了女婿!俺大娘請你老人家領他出來嫁人。」王婆子道:「天麼,天麼,你看麼!我說這淫婦,死了你爹,原守不住。只當狗改不了吃屎,就弄硶兒來了。就是你家大姐那女婿子?他姓甚麼?」他姓陳,名喚陳經濟。」王婆子道:「想著去年,我為何老九的事,去央煩你爹,到內宅,你爹不在。賊淫婦他就沒留我房裡坐坐兒,折針也迸不出個來!只叫丫頭倒了一鍾清茶我吃了,出來了。我只道千萬歲在他家,如何今日也還出來?好個狼家子淫婦!休說我是你個學生,替你作成了恁好人家。就是世人進去,也不該那等大意!」玳安道:「為他和俺姐夫在家裡毆作攘亂,昨日差些兒沒把俺大娘氣殺了哩!俺姐夫已是打發出去了。只有他老人家,如今教你領他去哩!」王婆子道:「他原是轎兒來,少不得還叫頂轎子。他也有個箱籠來這裡,少不的也與他個箱子兒。」玳安道:「這個少不的暗俺大娘他有個處。」兩個說話中間,到與西門慶門首。進入月娘房裡,道了萬福坐下。丫鬟拿茶吃了。月娘便道:「老王,無事不請你來。」悉把潘金蓮如此這般上項,說了一遍:「今來是是非人,人去是是非者。一客不煩二主,還起動你領他出去。或聘嫁,或打發,教他乞自在飯去罷。我男子漢已是沒了,招攬不過這些人來!說不的,當初死鬼為他丟了許多錢底那話兒,就打他恁個銀人兒也有!如今隨你聘嫁多少兒,教得來,我替他爹念個經兒,也是一場勾當!」王婆道:「你老人家是稀罕這錢的!只要把禍或害離了門就是了!我知道,我也不肯差了。」又道:「今日好日,就出去罷。又一件,他當初有個箱籠兒,有頂轎兒來。也少不的與他頂轎兒坐了去。」月娘道:「箱子與他一個,轎子不容他坐。」小玉道:「俺奶奶氣頭上,便是這等說。到臨岐,少不的顧頂轎兒。不然街坊人家看著拋頭露面的,不乞人笑話?」月娘不言語了。一面使丫鬟綉春,前邊叫金蓮來。這金蓮一見王婆子在房裡,就睜了,向前道了萬福坐下。王婆子開言便道:「你快收拾了,剛纔大娘說,教我今日領你出去呢。」金蓮道:「我漢子死了多少時兒,我為下甚麼非,作下甚麼歹來?如何平空打發我出去?」王婆道:「你休稀里打哄,做啞裝聾。自古蛇瑼鑽窟礲蛇知道,各人幹的事兒,各人心裡明白!金蓮,你休呆裡撒奸,兩頭白面,說長并道短!我手裡使不的你巧語花言,幫閒鑽懶!自古沒個不散的筵席,出頭緣兒先朽爛。人的名兒,樹的影兒,蒼蠅不鑽沒縫兒彈。你休把養漢當飯!我如今要打發你上陽關!」金蓮道:「你打人休打臉?罵人休揭短!常言:『一雞死了一雞鳴。』誰打羅,誰吃飯,誰吃飯,誰人常把鐵箍子哉?那個長捋蓆篾兒支著眼?為人還有相逢處,樹葉兒落還到根邊。你休要把人赤手空拳,往外攢,是非莫聽小人言!」正是:

「女人不穿嫁時衣,男兒不吃分時飯,自有徒牢話歲寒!」

當下金蓮與月娘亂了一回,月娘到他房中打點,與了他兩個箱子,一張抽替兒,四套衣服,幾件釵梳簪環,一床被褥。其餘他穿的鞋腳,都填在箱內。把秋菊叫得後邊來,一把鎖把他房門鎖了。金蓮穿上衣服,拜辭月娘,在西門慶靈前,大哭了一場。又走到孟玉樓房中,也是姊妹相處了一場,一旦分離,兩個落了一回眼淚。玉樓悄瞞著月娘,與了他一對金碗簪子,一套翠藍段襖紅裙子,說道:「六姐,奴與人離多會少了!你看個好人家往前進了罷!自古道:『千里長蓬,也沒個不散的筵席!』你若有了人家,使人來對奴說聲。奴往那裡去,順便到你那里看你去,也是姊妹情腸!」於是灑淚而別。臨出門,小玉送金蓮,悄悄與了金蓮兩根金頭簪兒。金蓮道:「我的姐姐,你倒有一點人心兒在!我上轎子,在大門首。」王婆又早顧人把箱籠卓子,抬的先去了。獨有玉樓、小玉送金蓮到門首,坐上轎子纔回。正是:

「世上萬般哀苦事,  除非死別共生離。」

都說金蓮到王婆家。王婆安插他在裏間,晚夕同他一處睡。他兒子王潮兒,也長成一條大漢,籠起頭來了,還未有妻室,外間支著床子睡。這潘金蓮,次日依舊打扮喬眉喬眼,在簾下看人。無事坐炕炕上,不是描眉畫眼,就是彈弄琵琶。王婆不在,就和王潮兒鬬葉兒下棋。那王婆自去掃麪餵養驢子,不去管他。朝來暮去,又把王潮兒刮剌上了。晚間等的王婆子睡著了,婦人推下炕溺尿,走出外間床子上,和王潮兒兩個幹。搖的床子一片響聲,被王婆子醒來聽見,問:「那裡響?」王潮兒道:「是櫃底下貓捕的老鼠響。」王婆子睡夢中,喃喃吶吶,口裡說道:「只因有這些麩麪在屋裡,引的這扎心的,半夜三更耗爆人,不得睡。」良久,又聽見動彈,搖的床子格支支響。王婆又問:「那裡響?」王潮道:「是貓咬老鼠,鑽在坑洞底下嚼的響。」婆子側耳,果然聽見貓在炕洞裡嚼耗子,方纔不言語了。婦人和小廝幹事,依舊悄悄上炕睡去了。有幾句雙關,說得這老鼠好:

「你身驅兒小膽兒大,嘴兒尖忒潑皮。見了人藏藏躲躲,耳邊廂叫叫唧唧,攪混人半夜三更不睡,不行正人倫,偏好鑽穴隙。更有一庄兒不老實,到底改不了偷饞抹嘴!」

有日,陳經濟打聽得金蓮出來,還在王婆子家聘嫁。提著兩弔銅錢,走到王婆子家來。婆子正在門前掃驢子撒下的糞。這經濟向前深深地唱個喏。婆子問道:「哥哥,你做甚麼?」經濟道:「請借裡邊說話。」王婆便讓進裏面。經濟揭起眼紗,便道:「動問西門大官人宅內,有一位娘子潘六姐在此出嫁?」王婆便道:「你是他甚麼人?」那經濟嘻嘻笑道:「不瞞你老人家說,我是他兄弟,他是我姐姐。」那王婆子眼上眼下,打量他一回,說:「他有甚兄弟我不知道?你休哄我!你莫非不是他家女婿姓陳的,來此處撞蠓子?我老娘手裡放不過。」經濟笑向腰裡解下兩弔銅錢來,放在面前說:「這兩弔錢,權作王奶奶一茶之費,教我且見一面,改日還重謝你老人家。」婆子見錢,越發喬張致起來,便道:「休說謝的話,他家大娘子分付將來,不教閒雜人來看他。咱放倒身說話,你既要見這雌兒一面,與我五兩銀子;見兩面與我十兩;你若娶他,便與我一百兩銀子。我的十兩媒人錢在外,我不管閑帳!你如今兩串錢兒,打水不渾的做甚麼?」經濟見這虔婆口硬不收錢,又向頭上拔下一對金頭銀腳簪子,重五錢,殺雞扯腿跪在地下,說道:「王奶奶,你且收了,容日再補一兩銀子來與你,不敢差了。且容我見他一面,說些話兒則個。」那婆子於是收了他簪子和錢,分付:「你進去見他,說了話就與我出來,不許你涎眉睜目,只顧坐著。所許那一兩頭銀子,明日就送來與家。」於是掀簾放經濟進裡間。婦人正坐在炕邊納鞋,看見經濟,放下鞋扇,會在一處,埋怨經濟:「你好人兒,弄的我前不着村,後不着店;有上稍,沒下稍,出醜惹人嫌!你就影兒不見,不來看我看兒了!我娘兒們好好兒的,拆散開,你東我西,皆因是為誰來?」說著,扯住經濟,只顧哭泣。王婆又嗔哭,恐怕有人聽見。經濟道:「我的姐姐,我為你剮皮割肉,你為我受氣耽羞。怎不來看你?昨日到薛嫂兒家,已知春梅賣在守備府裡去了。又打聽你出離了他家門,在王奶奶這邊聘嫁。今日特來見你一面,和你計議。咱兩個恩情難捨,拆散不開,如之奈何?我如今要把他家女兒休了,問他要我家先前寄放金銀箱籠。他若不與我,我東京萬壽門一本一狀進下來,那時他雙手奉與我,還是遲了!我暗地裡假名托姓,一頂轎子,娶到你家去,咱兩個永遠團圓,做上個夫妻,有何不可?」婦人道:「現今王乾娘要一百兩銀子,你有這些銀子與他?」經濟道:「如何要這許多?」婆子說道:「你家大丈母說,當初你家爹為他打個銀人兒也還多,定要一百兩銀子,少一絲毫也成不的。」經濟道:「實不瞞你老人家說,我與六姐打得熱了拆散不開。看你老人家下顧,退下一般兒來,五六十兩銀子也罷,我往張舅那里,典上兩三間房子,娶了六姐家去,也是春風一度。你老人家少轉些兒罷!」婆子道:「休說五十兩銀子,八十兩也輪不到你手裡了。昨日潮洲販紬絹何官人,出到七十兩。大街坊張二官府如今見在提刑院掌刑,使了兩個節級來,出到八十兩上。拏著兩封銀子來兌,還成不的,都回去了。你這小孩兒家,空口來說空話,倒還敢奚落老娘,老娘不道的吃傷了哩!」當下一陣走出街上,大喓喝說:「誰家女婿,要娶丈母?還來老娘屋裡放屁!」這經濟慌了,一手扯進婆子來,雙膝跪下,央及:「王奶奶噤聲,我依了奶奶價值一百兩銀子罷!爭耐我父親在東京,我明日起身,往東京取銀子去。」婦人道:「你既為我一場,休與乾娘爭執,上緊取去。只恐來遲了,別人娶了奴去了,就不是你的人了!」經濟道:「我僱上頭口,連夜兼程,多則半月,少則十日,就來了。」婆子道:「常言:『先下米,先食飯。』我的十兩銀子在外,休要少了。我的說明白著。」經濟道:「這個不必說,恩有重報,不敢有忘。」說畢,經濟作辭出門,到家收拾行李。次日早僱頭口,上東京取銀子去了。此這去正是:

「青龍與白虎同行,  吉凶事全然未保。」

畢竟未知後來如何,且聽下回分解:

