%# -*- coding: utf-8 -*-
%!TEX encoding = UTF-8 Unicode
%!TEX TS-program = xelatex
% vim:ts=4:sw=4
%
% 以上设定默认使用 XeLaTex 编译,并指定 Unicode 编码,供 TeXShop 自动识别

%第四十四回 
\chapter{吳月娘留宿李桂姐\KG 西門慶醉拶夏花兒}

「窮途日日困泥沙,  上苑年年好物華,

荊林不當馬車道,  管絃長奏絲羅家;

王孫草上悠揚蝶,  少女風前爛漫花,

懶出任從愁子笑,  入門還是舊生涯。」

話說經濟同傅夥計眾人前邊吃酒,吳大妗子轎子來了,收拾要家去。月娘款留再三,說道:「嫂子再住一夜兒,明日去罷。」吳大妗子道:「我連在喬親家那裡,就是三四日了。家裡沒人,你哥衙裡又有事,不得在家,我家去罷。明日請姑娘眾位,好歹往我那裡大節坐坐,晚夕告百備兒來家。」月娘道:「俺們明日只是晚上些去罷了。」吳大妗道:「姑娘早些坐轎子去,晚夕同坐了來家就是了。」說畢,裝了兩個盒子,一盒子元宵,一盒子饅頭,叫來安兒送大妗子到家。李桂姐等四個都磕了頭,拜辭月娘,也要家去。月娘道:「你們慌怎的?也就要去?還等你爹來家着你去。他去分付我留下你們。只怕他還有話和你們說,我是不敢放你去。」桂姐道:「爹去吃酒,到多咱晚來家?俺們原等的他,娘先教我和吳銀兒先去罷。他兩個今日纔來,俺們住了兩日,媽在家裡不知怎麼盼望。」月娘道:「可可的就是你媽盼望,這一夜兒等不的?」李桂姐道:「娘且是說的好。我家裡沒人,俺姐姐又被人包住了。寧可拿器來唱個與娘聽,娘放了奴去罷!」正說着,只見陳經濟走進來交剩下的賞賜與月娘,說道:「喬家并各家貼轎賞一錢,共使了十包,重三兩。還剩下十包在此。」月娘收了。桂姐便道:「我央及姑夫,你看外邊俺們的轎子來了不曾?」經濟道:「只有他兩個的轎子。你和銀姐的轎子沒來。從頭裡不知誰回了去了。」桂姐道:「姑夫,你真個回了?你哄我哩!」那經濟道:「你不信瞧去,不是我哄你。」剛言未罷,只見琴童抱進毡包來說:「爹家來了。」月娘道:「早是你每不去了,這不你爹來了?」不一時,西門慶進來,戴着冠帽,已帶七八分酒了,走入房中,正面坐下。月娘便道:「你董嬌兒、韓玉釧兒二人向前磕頭。」西門慶問道:「人都散了,更已深了,怎的我教他唱?」月娘道:「他們這裡求着我要家去。」且說西門慶向桂姐說:「你和銀兒亦發過了節兒去。且打發他個去罷。」月娘道:「如何?我說你們不信,恰相我哄你一般。」那桂姐把臉兒若低着,不言語。西門慶問玳安:「他兩個轎子在這裡不曾?」玳安道:「只有董嬌兒、韓玉釧兒兩頂轎子伺候着哩。」西門慶道:「我也不吃酒了。你們拿樂器來,唱十段錦兒我聽,打發他兩個先去罷。」當下四個唱的,李桂姐彈琵琶,吳銀兒彈箏,韓玉釧兒撥阮,董嬌兒打着緊急鼓子,一遍一個唱十段錦,二十八半截兒。吳月娘、李嬌兒、孟玉樓、潘金蓮、李瓶兒都在屋裡坐的聽唱。先是桂姐唱:

〔山坡羊〕「俏寃家,生的出類拔萃。翠衾寒,孤殘獨自。自別後,朝思暮想;想寃家,何時得遇?遇見寃家如同往,如同往。」

該吳銀兒唱:

〔金字經〕「惜花人何處,落和春又殘,倚遍危樓十二欄,十二欄。」

韓玉釧唱:

〔駐雲飛〕「悶倚欄杆,燕子鶯兒怕待看。色戒誰曾犯?思病誰經慣?」

董嬌兒唱:

「呀!減盡了花容月貌,重門常是掩。正東風料峭,細雨連纎,落紅千萬點。」

桂姐唱:

〔畫眉序〕「自會俏寃家,銀爭塵鎖怕湯抹。雖然是人離咫尺,如隔天涯。記得百

種恩情,那里計半星兒狂詐。」

吳銀兒唱:

〔紅綉鞋〕「水面上鴛鴦一對,順河岸步步相隨。怎見個打漁船,驚拆在兩下裡飛。」

韓玉釧唱:

〔耍孩兒〕「自從他去添憔瘦,不似今番病久。才郎一去正逢春,急回頭雁過了中秋。」

董嬌兒唱:

〔傍粧臺〕「到如今瑤琴絃斷少知,魯花好時誰共賞?」

桂姐唱:

〔鎖南枝〕「紗窗外月兒斜久,想我人兒,常常不捨你。為我力盡心謁,我為你珠淚偷揩。」

吳銀兒唱:

〔桂枝香〕「楊花心性,隨風不定。他原來假意兒虛名,到我真心陪奉。」

韓玉釧唱:

〔山坡羊〕「惜玉憐香,我和他在芙蓉帳底抵面共。你把衷腸來細講,講離情如何把奴拋棄。氣的我似醉如痴來呵,何必你別心另叙上?知已幾時,得重整佳期;佳期實相逢如同夢裡。」

董嬌兒唱:

〔金字經〕「彈淚痕,羅帕班;江南岸,夕陽山外山。」

李桂姐唱:

〔駐雲飛〕「嗏!書寄兩三番,得見艱難。再猜霜毫,寫下喬公案,滿紙春心墨未乾。」

吳銀兒唱:

〔江兒水〕「香串懶重添,針兒怕待拈。瘦體嵒嵒,鬼病懨懨,俺將這舊情重檢點。愁壓挨兩眉翠尖,空惹的張郎憎厭這些時,鶯花不捲簾。」

韓玉釧唱:

〔畫眉序〕「想在枕上溫存的話,不由人窗顫身麻。」

董嬌兒唱:

〔紅綉鞋〕「一個兒投東去,一個兒向西飛;撇的俺一個兒南來,一個兒北去。」

李桂姐唱:

〔耍孩兒〕「你那裡偎紅倚翠綃金帳,我這裡獨守香閨淚暗流。從記得說來咒,負心的隨燈兒滅,海神廟放著根由。」

吳銀兒唱:

〔傍粧臺〕「美酒兒誰共斟,意散了如瓶兒。難見面,似參辰。從別後幾月深。畫劃兒畫損了掠兒金。」

韓玉釧唱:

〔鎖南枝〕「兩下裡心腸牽罣,誰知道風掃雲開?今宵復顯出團圓月,重令情郎把香羅再解。訴說情誰負誰心,須共你說個明白。」

董嬌兒唱:

〔桂枝香〕「怎忘了舊時,山盟為證,坑人性命。有情人,從此分離了去,何時直得成?」

李桂姐唱:

〔尾聲〕「半叉綉羅鞋,眼兒見了心兒愛。可喜才捨著搶白,忙把這俏身挨。」

唱畢,西門慶與了韓玉釧、董嬌兒兩個唱錢,拜辭出門,留李桂姐、吳銀兒兩個:「這裡歇罷。」;忽聽前邊玳安兒和琴童兩個嚷亂,簇擁定李嬌兒房裡夏花兒,進來稟西門慶說道:「小的剛送兩個唱的出去,打燈籠往馬房裡拌草,牽馬上槽。只見二娘房裡夏花兒躲在馬槽底下,諕了小的一跳,不知甚麼緣故?小的每問着他,又不說。」西門慶聽見,便道:「那奴才在那裡?與我拿來。」就坐出外邊明間穿廊下椅子上坐着,一邊打着,一個簇把那丫頭兒揪着跪下。西門慶問他:「往前邊做甚麼去?」那丫頭不言語。李嬌兒在傍邊說道:「我又不使你,平平白白往馬坊裡做甚麼去?」見他慌做一團,西門慶只說丫頭要走之情,即令小廝:「與我與他搜身上。」他又不容搜。于是琴童把他一拉倒在地,只聽滑浪一聲,沉身從腰裡吊下一件東西來。西門慶問:「是甚麼?」玳安遞上去。可霎作怪,都是一定金子。西門慶燈下看了道:「是頭裡不見了的那定金子。尋不見,原來是你這奴才偷了!」他說是拾的。西門慶問:「是那裡拾的?」他又不言語,西門慶于是心中大怒,令琴童往前邊去取拶子來。須臾,把丫頭拶起來,拶的殺豬也是叫。拶了半日,又敲二十敲。月娘見他有酒了,又不敢勸。那丫頭挨忍不過,方說:「我在六娘房裡地下拾的。」西門慶方命放了拶子,又分付與李嬌兒領到屋裡去:「明日叫媒人,即時與我拉出去賣了。這個奴才,還留着做甚麼?」那李嬌兒沒的話兒說,便道:「恁賊奴才,誰叫你往前頭去來?養在家裡,也問我聲兒,三不知就出去了。你就拾了他屋裡金子,也對我說一聲兒。」那夏花兒只是哭。李嬌兒道:「拶死你這奴才纔好哩,你還哭!」西門慶道:「罷!」把金子交與月娘收了,就往前邊李瓶兒房裡去了。那小廝多出去了。月娘令小玉關上儀門,因叫道玉筲來,問他:「頭裡這丫頭也往前邊去來麼?」小玉道:「二娘、三娘陪大妗子娘兒兩個往六娘那邊去,他也跟了去來。誰知他三不知就偷了他這定金子在手裡。頭裡聽見娘說爹使小廝買狼觔去了,諕的他要不的,在廚房問我:『狼觔是甚麼?』教俺每眾人笑道:『狼觔敢是狼身上的觔,若是那個偷了東西不拿出來,把狼觔抽將起來,就纏在那人身上,抽攢的手腳兒都在一處。』他見咱說想必慌了。到晚夕趕唱的出去,就要走的情。見大門首有人,纔藏入馬坊裡,鑽在槽底下躲着。不想被小廝又看見了,採出來。」月娘道:「那裡看人去?恁小丫頭,原來這等賊頭鼠腦的!到就不是個咍咳的。」且說李嬌兒領夏花兒到房裡,李桂姐晚間甚是說夏花兒:「你原來是個俗孩子,你恁十五六歲,也知道些人事兒,還這等慒懂?要着俺裡邊,纔使不的。這裡沒人,你就拾了些東西,來屋裡悄悄交與你娘。似這等把出來,他在傍邊也好救你。你怎的不望他題一字兒?剛纔這等拶打着,好麼?乾淨俊丫頭!常言道:『穿青衣,抱黑柱。』你不是他這屋裡人,就不管他。剛纔這這等掠掣着你,你娘臉上有光沒光?」又說他姑娘:「你也忒不長俊。要着是我,怎教他把我房裡丫頭對眾拶恁一頓拶子?又不是拉到房裡來,等我打。前邊幾個房裡丫頭怎的不拶,只拶你房裡丫頭?你是好欺負的,就鼻子口裡沒些氣兒?等不到明日真個教他拉出這丫頭去罷,你也就沒句話兒說?你不說,等我說,休教他領出去,教別人好笑話。你看看孟家的和潘家的,兩家一似狐狸一般,你原鬬的過他了?」因叫個夏花兒過來,問他:「你出去不出去?」那丫頭道:「我不出去。」桂姐道:「你不出去,今後要貼你娘的心,凡事要你和他一心一計。不拘拿了甚麼,交付與他,教似元宵一般擡舉你。」那夏花兒說:「姐分付我知道了。」接下這裡教唆夏花兒不題。且說西門慶走到前邊李瓶兒房裡,只見李瓶兒和吳銀兒炕上做一處坐的,心中就要脫衣去睡。李瓶兒道:「銀姐在這裡,也沒地方兒安插,你且過一家兒罷!」西門慶道:「怎的沒地方兒?你娘兒兩個在兩邊,等我在當中睡就是。」李瓶兒便瞅了他眼兒道:「你就說下道兒去了。」西門慶道:「我如今在那里睡?」李瓶兒道:「你過六姐那邊去睡一夜罷!」西門慶坐了一回,起身走了,說道:「也罷!也罷!省的我打攪你娘兒們,我過那邊屋裡睡去罷。」于是一直走過金蓮這邊來。金蓮聽見西門慶進房來,天上落下來一般。向前與他接衣解帶,鋪陳牀鋪乾淨,展放鮫綃,款設珊枕,吃了茶,兩個上牀歇宿不題。李瓶兒這裡打發西門慶出來,和吳銀兒兩個燈下放炕卓兒,撥下黑白棋子,對坐下象棋兒。分付迎春:「定兩盞茶兒,拿個菓盒兒,把這甜金華酒兒 篩一壺兒來,我和銀姐吃。」因問銀姐:「你吃飯?教他盛飯來你吃。」吳銀兒道:「娘,我且不餓,休叫姐盛來。」李瓶兒道:「也罷!銀姐不吃飯,你拿個盒盖兒,我揀粧裡有菓餡餅兒,拾四個兒來,與銀姐吃罷。」須臾,迎春拿了四碟小菜,一碟糟蹄子觔 ,一碟鹹雞,一碟爛雞蛋,一碟炒的荳芽菜拌海蜇 ,一個菓盒,都是細巧菓仁兒,一盒菓餡餅兒,頓備在傍邊。少頃,與吳銀兒下了三盤棋子。篩上酒來,拿銀鐘兒兩個共飲。吳銀兒叫:「迎春姐,你遞過琵琶來,我唱個曲兒與娘聽。」李瓶兒道:「姐姐不唱罷,小大官兒睡着了。他爹那邊又聽着,教他說。咱擲骰子耍耍罷。」于是教迎春遞過色盆來,兩個擲骰兒賭酒為樂。擲了一回,吳銀兒因叫:「迎春姐,你那邊屋裡請過你媽兒來,教他吃鐘酒兒。」迎春道:「他摟着哥兒在那邊炕上睡哩!」李瓶兒道:「教他摟着孩子睡罷,拿了一甌酒送與他吃就是了。你不知俺這小大官,好不伶俐,人只離來開他就醒了。有一日兒,在我這邊炕上睡,他爹這裡敢動一動兒,就睜開眼醒了,恰似知道的一般。教奶子抱了去那邊屋裡,只是哭,只要我摟着他。」吳銀兒笑道:「娘有了哥兒,和爹自在覺兒也不得睡一個兒。爹幾日來這屋裡走一遭兒?」李瓶兒道:「他也不論,遇着一遭也不可止,兩遭也不可止,常進屋裡看他。為這孩子來看他不打緊,教人把肚子也氣破了。相他爹和這孩子,背地咒的白湛湛的;我是不消說的,只與人家墊舌根!誰和他有甚麼大閒事,寧可他不來我這裡還好。第二日教人眉兒眼兒的,只說俺們什麼把攔着漢子。為甚麼剛纔到這屋裡,我就攛掇他出去?銀姐你不知俺這家,人多舌頭多!自今日為不見了這定金子,早是你看着,就有人氣不憤,在後邊調白你大娘,說拿金子進我這屋裡來了,怎的不見了?落後不想是你二娘屋裡丫頭偷了,纔顯出個青紅皂白來。不然,綁着鬼,只是俺這屋裡丫頭和奶子。老馮媽媽急的那哭,只要尋死,說道:『若沒有這金子,我也不家去。』落後見有了金子,那咱纔肯去,還打了燈家去了。」吳銀兒道:「娘,也罷!你看爹的面上,你守着哥兒,慢慢過到那裡是那裡。論起後邊大娘,沒甚言語也罷了。倒只是別人見娘生了哥兒,未免都有些兒氣。爹他老人家有些主就好。」李瓶兒道:「若不是你爹和你大娘看覷,這孩子也活不到如今!」說話之間,你一鍾,我一盞,不覺坐到三更天氣,方纔宿歇。正是:

「得意客來情不厭,  知心人到話相投。」

有詩為證:

「畫樓明日轉窗寮,  相伴嬋娟宿一宵,

玉骨冰肌誰不愛,  一枝梅影夜迢迢。」

畢竟未知後來何如,且聽下回分解:
