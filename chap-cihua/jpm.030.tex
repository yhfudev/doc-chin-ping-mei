%# -*- coding: utf-8 -*-
%!TEX encoding = UTF-8 Unicode
%!TEX TS-program = xelatex
% vim:ts=4:sw=4
%
% 以上设定默认使用 XeLaTex 编译,并指定 Unicode 编码,供 TeXShop 自动识别

%第三十回 
\chapter{來保押送生辰擔\KG 西門慶生子嘉官}


\begin{showcontents}{}



「得失榮枯總是閑,  機關用盡也徒然,

人心不足蛇吞象,  世事到頭螳捕蟬;

無藥可醫卿相壽,  有錢難買子孫賢,

家常寸分隨緣過,  便是消遙自在天。」

話說西門慶與潘金蓮兩個洗畢澡,就睡在房中。春梅坐在穿廊下一張涼椅兒上納鞋。只見琴童兒在角門首探頭舒腦的觀看。春梅問道:「你有甚話說?」那琴童又見秋菊頂着石頭跪在院內,只顧用手往來指。春梅罵道:「怪囚根子!你有甚麼話,說就是了。指手畫腳怎的?」那琴童笑了半日,方纔說:「有看墳的張安兒,在外邊等爹說話哩!」春梅道:「賊囚根子!張安就是了,何必大驚小怪,見鬼也似悄悄兒的!爹和娘在屋裡睡着了,驚醒他你就是死。你且教張安在外邊等等兒。」那琴童兒走出來外邊,約等勾半日,又走來角門首踅探,問:「姐,爹起來了不曾?」春梅道:「怪囚!失張冒勢,恁諕我一跳。有要沒緊兩頭回來遊魂哩!」琴童道:「張安等爹出去見了,說了話,還要趕出門去,怕天晚了。」春梅道:「爹娘正睡的甜甜兒的,誰敢攪擾他。你教張安且等着去。十分晚了,教他明日去罷。」正說着,不想西門慶在房裡聽見,便叫春梅進房。問:「誰說話?」春梅道:「琴童小廝進來說,墳上張安兒在外邊,見爹說話哩。」西門慶道:「拿衣我穿,等我起去。」春梅一面打發西門慶穿衣裳,金蓮便問:「張安來說甚麼話?西門慶道:「張安前日來說,咱家墳隔壁趙寡婦家庄子兒,連地要賣,價錢三百兩銀子。我只還他二百五十兩銀子,教張安和他講去。若成了,我教賁四和陳姐夫去兌銀子。裡面一眼井,四個井圈打水。我買了這庄子,展開合為一處,裡面蓋三間捲棚,三間廳房,叠山子花園,松墻槐樹棚,井亭射箭廳,打毬場耍子去處,破使幾兩銀子,收拾也罷。」婦人道:「也罷,咱買了罷。明日你娘們上墳,到那裡好遊玩耍子。」說畢,西門慶往前邊和張安說話去了。金蓮起來,向鏡臺前重勻粉臉,再整雲鬟,出來院內要打秋菊。那春梅旋去外邊叫了琴童兒來吊板子。金蓮便問道:「教你拿酒,你怎的拿冷酒與你爹吃?原來你家沒大了。說着你,還丁嘴鐵舌兒的!」喝聲叫琴童兒:「與我老實打與這奴才二十板子。」那琴童纔打到十板子上,多虧了李瓶兒笑嘻嘻走過來勸住了,饒了他十板。金蓮教與李瓶兒磕了頭。放他起來,廚下去了。李瓶兒道:「老潘領了個十五歲的丫頭,後邊二姐姐買了房裡使喚,要七兩五錢銀子。請你過去瞧瞧,要送與他去哩。」這金蓮遂與李瓶兒一同後邊去了。李瓶兒果然問了西門慶,用七兩銀子買了丫頭,改名夏花兒,房中使喚,不在話下。安下一頭,卻說一處。單表來保同吳主管押送生辰担,自從離了清河縣,一路朝登紫陌,暮踐紅塵,飢餐渴飲,夜住曉行。正值大暑炎蒸天氣,爍石流金之際,路上十分難行。評話捷說,有日到了東京萬壽門外,尋客店安下。到次日,賫抬馱箱禮物,逕到天漢橋蔡太師府門前伺候。來保教吳主管押着禮物,他穿上青衣,逕向守門官吏唱了個喏。那守門官吏問道:「你是那裡來的?」來保道:「我是山東清河縣西門員外家人,來與老爺進獻生辰禮物。」官吏罵道:「賊少死野囚軍!你那裡便興你東門員外西門員外?俺老爺當今一人之下,萬人之上,不論三台八位,不論公子王孫,誰敢在老爺府前這等稱呼?趁早靠後。」內中有認的來保的,便安來撫來保說道:「此是新參的守門官吏,纔不多幾日,他不認的你,休怪。你要稟見老爺,等我請出翟大叔來。」這來保便向袖中取出一包銀子,重一兩,遞與那人。那人道:「我到不消,你再添一分,與那兩個官吏。休和他一般見識。」來保連忙拿出三包銀子來,每一兩,都打發了。那官吏纔有些笑容兒,說道:「你既是清河縣來的,且略候候。等我領你先見翟管家。老爺纔從上清寶籙宮進了香回來,書房內睡。」良久,請到翟管家出來,穿着涼鞋淨襪,青絲絹道袍。來保見了,先磕下頭去。翟管家答禮相還,說道:「前者累你。你來與老爺進生辰担禮來了。」來保先遞上一封揭帖,腳下人棒着一對南京尺頭,三十兩白金,說道:「家主西門慶多上覆翟爹。無物表情,這些薄禮,與翟爹賞人。前者,鹽客王四之事,多蒙翟爹費心。」翟謙道:「此禮我不當受罷!罷,我且收下。」來保又遞上太師壽禮帖兒看了,還付與來保。吩咐:「把禮抬進來,到二門裡首伺候。」原來二門西首,有三間倒座。來往雜人,都在那裡待茶。須臾,一個小童拿了兩盞茶來,與來保、吳主管吃了。少頃,太師出廳。翟謙先稟知太師。太師然後令來保、吳主管進見,跪於階下。翟謙先把壽禮揭帖,呈遞與太師觀看。來保、吳主管各捧獻禮物。但見黃烘烘金壺玉盞,白晃晃減靸仙人。良工製造費工夫,巧匠鑽鑿人罕見。錦綉蟒衣,五彩奪目;南京紵段,金碧交輝。湯羊美酒,盡貼封皮;異菓時新,高堆盤榼。如何不喜?便道:「這禮物決不好受的,你還將回去。」于是慌了來保等在下叩頭,說道:「小的主人西門慶沒甚孝順,些小微物,進獻老爺賞人便了。」太師道:「既是如此,令左右收了。」傍邊左右祗應人等,把禮物盡行收下去。太師又道:「前日那滄州客人王四等之事,我已差人下書與你巡撫侯爺說了。可見了分上不曾?」來保道:「蒙老爺天恩書到,眾鹽客都牌提到鹽運司,與了勘合,都放出來了。」太師因向來保說道:「禮物我故收了。累次承你主人費心,無物可伸,如何是好?你主人身上可有甚官役?」來保道:「小的主人一介鄉民,有何官役?」太師道:「既無官役,昨日朝廷欽賜了我幾張空名告身劄付,我安你主人在你那山東提刑所做個理刑副千戶,頂補千戶賀金的員缺,好不好?」來保慌的叩道謝道:「蒙老爺莫大之恩,小的家主舉家粉首碎身,莫能報答。」于是喚堂後官,擡書案過來,即時僉押了一道空名告身劄付,把西門慶名字填註上面,列銜金吾衛衣左所副千戶,山東等處提刑所理刑。向來保道:「你二人替我進獻生辰禮物,多有辛苦。」因問:「後邊跪的,是你甚麼人?」來保纔待說是夥計,那吳主管向前道:「小的是西門慶舅子,名喚吳典恩。」太師道:「你既是西門慶舅子,我觀你到好個儀表。」喚堂後官取過一張劄付:「我安你在本處清河縣做個馹丞,倒也去的。」那吳典恩慌的磕頭如搗蒜。又取過一劄付來,把來保名字填寫山東鄆王府,做了一名校尉。俱磕頭謝了,領了劄付。吩咐:「明日早辰,吏兵二部掛號,討勘合,限日上任應役。」又吩咐翟謙:「西廂房管待酒飯。討十兩銀子與他二人做路費。」不在話下。看官聽說:那時徽宗天下失政,奸臣當道,讒侫盈朝。高、楊、童、蔡四個奸黨,在朝中賣官鬻獄,賄賂公行。懸秤陞官,指方補價。夤緣鑽刺者,驟陞美任,賢能廉直者,經歲不除。以致風俗頹敗,賍官污吏遍滿天下。役煩賦重,民窮盜起,天下騷然。不因奸侫居台輔,合是中原血染人。當下翟謙把來保、吳主管邀到廂房管待,廚下大盤大碗,肉賽花糕 ,酒如琥珀,湯飯點心齊上,飽餐了一頓。翟謙向來保說:「我有一件事,央及你爹替我處處,未知你爹肯應承我否?」來保道:「翟爹說那裡話!蒙你老人家這等老爺前扶持看顧。不揀甚事,但肯吩咐,無不奉命。」翟謙道:「不瞞你說,我答應老爺,每日止賤荊一人。我年也將及四十,常有疾病,身邊通無所出。央及你爹,只說你那貴處有好人才女子,不拘十五六上下,替我尋一個送來。該多少財禮,我一一奉過去。」于是一封人事并回書付與來保,又已送二人五兩盤纏。來保再三不肯受,說道:「剛纔老爺上已賞過了。翟爹還收回去。」翟謙道:「那是老爺的。此是我的,不必推辭。」當下吃畢酒飯。翟謙道:「如今我這裡替你差個辦事官,同你到下處。明早好往吏兵二部掛號,就領了勘合好起身。省的你明日又來,途間往返了。我吩咐了去,部裡不敢遲滯了你文書。」那時喚了個辦事官,名喚李中友:「你與二位明日同到部裡,掛了號,討勘合,來回我話。」那員官與來保、吳典恩作辭,出的府門來,到天漢橋街上白酒 店內會話。管待酒飯,又與了李中友三兩銀子。約定明日絕早,先到吏部,然後到兵部,都掛號討了勘合。開得是太師老爺府裡,誰敢遲滯,顛倒奉行?金吾衛太尉朱勔,即時使印僉了票帖,行下頭司,把來保填註在本處山東鄆王府當差。又拿了個拜帖,回翟管家。不消兩日,把事情幹得完備。有日顧頭口起身,星夜回清河縣來報喜。正是:

「富貴心因奸巧得,  功名全仗鄧通成!」

且說一日三伏天氣,十分炎熱。在家中聚景堂中大捲棚內賞玩荷花,避暑飲酒。吳月娘與西門慶居上坐,諸妾與大姐都兩邊列坐。春梅、迎春、玉簫、蘭香一般兒四個家樂,在傍彈唱。怎見的當日酒席?但見:

「盆栽綠草,瓶插紅花。水晶簾捲蝦鬚,雲母屏開孔雀。盤堆麟脯,佳人笑捧紫霞觴;盆浸冰桃,美女高擎碧玉斝。食烹異品,菓獻時新。絃管謳歌,奏一派聲清韻美;綺羅珠翠,擺兩行舞女歌兒。當筵象板撒紅牙,遍體舞裙補錦綉。消遣壺中閒日月,遨遊身外醉乾坤。」

妻妾正飲酒中間,坐間不見了李瓶兒。月娘向綉春說道:「你娘往屋裡做甚麼哩?怎的不來吃酒?」綉春道:「我娘害肚裡疼,屋裡〈扌歪〉着哩,便來也!」月娘道:「還不快對他說去!休要〈扌歪〉着,來這裡坐着聽一回唱罷。」西門慶便問月娘:「怎的?」月娘道:「李大姐忽然害肚裡疼,屋裡倘着哩。我剛纔使小丫頭請他去了。」因向玉樓道:「李大姐七八臨月,只怕攪撒了。」潘金蓮道:「大姐姐,他那裡是這個月,約他是八月裡孩子,還早哩。」西門慶道:「既是早哩,使丫頭請你六娘來聽唱。」不一時,只見李瓶兒來到。月娘道:「只怕你掉了風冷氣,你吃上鍾熱酒,管情就好了。」不一時,各人面前斟滿了酒。西門慶吩咐春梅:「你每唱個『人皆畏夏日』我聽。」那春梅等四個,方纔箏排雁柱,阮跨鮫綃,啟朱唇,露皓齒,唱『人皆畏夏日』云云。那李瓶兒在酒席上,只是把眉頭忔縐着,也沒等的唱完了,回房中去了。月娘聽了詞曲,躭着心。使小玉房中瞧去。回來報說:「六娘害肚裡疼,在炕上打滾哩。」慌了月娘道:「我說是時候,這六姐還強說早哩,還不喚小廝來快請老娘去!」西門慶即令來安兒:「風跑快,請蔡老娘去。」于是連酒也吃不成,都來李瓶兒房中問他。月娘問道:「李大姐,你心裡覺怎的?」李瓶兒回道:「大娘,我只心口連小肚子,往下鱉墜着疼。」月娘道:「你起來,休要睡着,只怕滾壞了胎。老娘請去了,便來也。」少頃,漸漸李瓶兒疼的緊了,月娘又問:「使了誰請老娘去了?這咱還不見來?」玳安道:「爹使了來安去了。」月娘罵道:「這囚根子!你還不快迎迎去?平白沒筭計,使那小奴才去,有緊沒慢的!」西門慶叫玳安:「快騎了騾子趕了去!」月娘道:「一個風火事,還像尋常慢條斯禮兒的!」那潘金蓮見李瓶兒待養孩子,心中未免有幾分氣。在房裡看了一回,把孟玉樓拉出來,兩個站在西稍間簷柱兒底下那裡歇涼,一處說話。說道:「聊嚛嚛!緊着熱剌剌的擠了一屋子裡人,也不是養孩子,都看着下象胆哩!」良久,只見蔡老娘進門,望眾人:「那位主家奶奶?」李嬌兒道:「這位大娘裡。」那蔡老娘倒身磕頭去。月娘道:「姥姥,生受。你怎的這咱纔來?」蔡老娘道:「你老人家聽我告訴:

我做老娘姓蔡,  兩雙腳兒能快。

身穿怪綠喬紅,  各樣䯼髻歪戴。

嵌絲環子鮮明,  閃黃手帕符〈扌寨〉。

入門利市花紅,  坐下就要管待。

不拘貴宅嬌娘,  那管皇親國太。

教他任意端詳,  被他褪衣〈百刂〉劃。

橫生就用刀割,  難產須將拳揣。

不管臍帶包衣,  著忙用手撕壞。

活時來洗三朝,  死了走的偏快。

因此主顧偏多,  請的時常不在。」

月娘道:「你且休閑說。請看這位娘子,敢待生養也?」蔡老娘向牀前摸了摸李瓶兒身上,說道:「是時候了。」問大娘:「預備下綳接草布不曾?」月娘道:「有。」便教小玉:「往我房中快取去。」且說玉樓見老娘進門,便向金蓮說:「蔡老娘來了,咱不往屋裡看看去?」那金蓮一面不是一面說道:「你要看你去,我是看他。他是有孩子的姐姐,又有時運人,怎的不看他?頭裡我自不是,說了句話兒,見他不是這個月的孩子,只怕是八月裡的,教大姐姐白搶白相,我想起來,好沒來由。倒惱了我這半日。」玉樓道:「我也只說他是六月裡孩子。」金蓮道:「這回連你也韶刀了!我和你恁筭他,從去年八月來,又不是黃花女兒,當年懷入門養。一個後婚老婆,漢子不知見過了多少。也一兩個月纔生胎,就認做是咱家孩子,我說差了!若是八月裡孩兒,還有咱家些影兒。若是六月的,踩小板凳兒糊險道神,還差着一帽頭子哩!失迷了家鄉,那裡尋犢兒去?」正說着,只見雪娥後邊和小玉抱着草布綳接并小褥兒來。孟玉樓道:「此是大姐姐預備下,他早晚臨月用的物件兒,今日且借來應急兒。」金蓮道:「一個是大老婆,一個是小老婆,明日兩個對養。十分養不出來,零碎出來也罷。俺每是買了個母雞不下蛋,莫不殺了我不成?」又道:「仰着合着,沒的狗咬尿胞虛喜歡!」玉樓道:「五姐是甚麼話!」以後見他說話兒出來有些不防頭惱,只低着頭弄裙子,並不作聲應答他。潘金蓮用手扶着庭柱兒,一隻腳趾着門檻兒,口裡磕着瓜子兒。只見孫雪蛾聽見李瓶兒前邊養孩子,後邊慌慌張張一步一跌走來觀看。不防黑影裡,被臺基險些不曾絆了一交。金蓮看見,教玉樓:「你看,獻懃的小婦奴才!你慢慢走,慌怎的,搶命哩!黑影子拌倒了,磕了牙,也是錢。姐姐,賣蘿蔔的拉鹽担子攘,鹹嘈心!養下孩子來,明日賞你這小婦一個紗帽戴。」良久,只聽房裡呱的一聲,養下來了。蔡老娘道:「對當家的老爹說,討喜錢,分娩了一位哥兒。」吳月娘報與西門慶。門慶慌的連忙洗手,天地祖先位下,滿爐降香,告許一百二十分清醮,要祈子母平安,臨盆有慶,坐草無虞。這潘金蓮聽見坐下孩子來了,合家歡喜,亂成一塊,越發怒氣,生走去了房裡,向牀上哭去了。時宣和四年,戊申六月廿一日也。正是:

「不如意處常八九,  可與人言無二三。」

這蔡老娘收拾孩兒,咬去臍帶,埋畢衣胞,熬了些定心湯,打發李瓶兒吃了。安頓孩兒停當。月娘讓老娘後邊管待酒飯。臨去,西門慶與了他五兩一定銀子。許洗三朝來還與他一疋段子。這蔡老娘千恩萬謝出門。當日西門慶進房去,見一個滿抱的孩子,生的甚是白淨,心中十分歡喜,合家無不欣悅。晚夕就在李瓶兒牀房中歇了,不住來看孩兒。次日巴天不明早起來,拿十副方盒,使小廝各親戚鄰友處,分投送喜麵。應伯爵、謝希大聽見西門慶生了子送喜麵來,慌的兩步做一步走來賀喜。西門慶留他捲棚內吃麵。剛打發去了,正在廳上亂着,使小廝叫媒人來尋養娘看奶孩兒。忽有薛嫂兒領了個奶子來,原是小人家媳婦兒,年三十歲。新近丟了孩兒,不上一個月,男子漢當軍,過不的。恐出征去,無人養贍,只要六兩銀子要賣他。月娘見他生的乾淨,對西門慶說,兌了六兩銀留下,起名如意兒,教他早晚看奶哥兒。又把老馮叫來暗房中使喚,每月與他五錢銀子,管顧他衣服。正熱鬧,一日忽有平安報:「來保、吳主管在東京回還,見在門首頭口。」不一時,二人進來,見了西門慶報喜。西門慶問:「喜從何來?」二人悉把到東京見蔡太師進禮一節,從頭至尾訴說一遍:「老爺見了禮物,說道:『我累次受你主人禮太多,無可補報』。因問爹:『原祖上有甚差事?』小的說『一介鄉民,並無寸役在身。』太師老爺說:朝廷欽賞了他幾張空名誥身劄付,與了爹一張。填寫爹名姓在上,填註在金吾衛副千戶之職。就委差的在本處提刑所理刑,頂補賀老爹員缺。把小的做了鐵鈴衛校尉,填註鄆王府當差。吳主管陞做本縣馹丞。」于是把一樣三張印信劄付,并吏兵二部勘合,并誥身,都取出來放在桌上,與西門慶觀看。西門慶看見上面銜着許多印信,朝廷欽依事例,果然他是副千戶之職。不覺歡從額角眉尖出,喜向腮邊笑臉生。便把朝廷明降拿到後邊,與吳月娘眾人觀看,說:「太師老爺擡舉我,陞我做金吾衛副千戶,居五品大夫之職。你頂受五花官誥,坐七香車,做了夫人。又把吳主管攜帶做了驛丞,來保做了鄆王府校尉。吳神仙相我不少紗帽戴,有平地登雲之喜。今日果然不上半月,兩樁喜事都應驗了。對月娘說:「李大姐養的這孩兒,甚是腳硬,到三日洗了三,就起名叫做官哥兒罷。」與月娘看了。來保進來與月娘眾人磕頭,說了回話。吩咐:「明日早把文書下到提刑所衙門裡,與夏提刑知會了。」吳主管明日早下文書到本縣,作辭西門慶回家去了。到次日洗三畢,眾親鄰朋友一概都知西門慶第六個娘子新添了娃兒,未過三日,就有如此美事,官祿臨門,平地做千戶之職,誰人不來趨附,送禮慶賀。人來人去,一日不斷頭。常言:「時來誰不來,時不來誰來」正是:

「時來頑鐵有光輝,  運退真金無豔色!」

畢竟未知後來如何,且聽下回分解:






\end{showcontents}


