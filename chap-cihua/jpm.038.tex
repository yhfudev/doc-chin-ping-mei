%# -*- coding: utf-8 -*-
%!TEX encoding = UTF-8 Unicode
%!TEX TS-program = xelatex
% vim:ts=4:sw=4
%
% 以上设定默认使用 XeLaTex 编译,并指定 Unicode 编码,供 TeXShop 自动识别

%第三十八回 
\chapter{西門慶夾打二搗鬼\KG 潘金蓮雪夜弄琵琶}


「麗質溫柔更老成,  玉壺明月適人情,

輕回玉臉花含媚,  淺蹙蛾眉雲髻鬆;

勾引蜂狂桃蕊綻,  潛牽蝶亂柳腰新,

令人心地常相憶,  莫學章臺贈淡情。」

話說馮婆子走到前廳角門首,看見玳安在廳槅子前,拏著茶盤兒伺候。玳安望著媽媽努嘴兒:「你老人家先往那里去?俺爹和應二爹說話哩。說了話,打發去了,就起身。先使棋童兒送酒去了。」那婆子聽見,兩步做一步走的去了。原來應伯爵來說攬頭:「李智、黃四派了年例三萬香蠟等料錢糧下來,該一萬兩銀子,也有許多利息,上完了批,就在東平府見關銀子。來和你計較,做不做?」西門慶道:「我那里做他攬頭,以假充真,買官讓官。我衙門里搭了事件,還要動他,我做他怎的?」伯爵道:「哥若不做,教他另搭別人。在你借二千兩銀子與他,每月五分行利。教他關了銀子還你,你心下如何?計較定了,我對他說,教他兩個明日拏文書來。」西門慶道:「既是你的分上,我挪一千銀子與他罷。如今我庄子收拾,還沒銀子哩。」伯爵見西門慶吐了口兒,說道:「哥若十分沒銀子,看怎麼再撥五百兩銀子貨物兒,湊個千五兒與他罷。他不敢少下你的。」西門慶道:「他少下我的,我有法兒處。又一件,應二哥,銀子便與他,只不教他打著我的旗兒在外邊東馬塀匡西騙。我打聽出來,只怕我衙門監里放不下他。」伯爵道:「哥說的什麼話?典守者不得辭其責。他若在外邊打哥的旗兒,常沒事罷了,若壞了事,要我做什麼?哥,你只顧放心。但有差遲,我就來對哥說。說定了,我明日教他好寫文書。」西門慶道:「明日不教他來,我有勾當。教他後日來。」說畢,伯爵去了。西門慶教玳安伺候馬,帶上眼紗,問:「棋童去沒有?」玳安道:「來了。取挽手兒去了。」不一時,取了挽手兒來,打發西門慶上馬,逕往牛皮巷來。不想韓道國兄弟韓二搗鬼耍錢輸了。吃的光睜睜兒的,走來哥家,問王六兒討酒吃。袖子里掏出一條小腸兒來,說道:「嫂,我哥還沒來哩。我和你吃壺燒酒 。」那婦人恐怕西門慶來,又見老馮在廚下,不去兜攬他,說道:「我是不吃。你要吃,拏過一邊吃去,我那里耐煩!你哥不在家,招是招非的,又來做什麼!」那韓二搗鬼把眼兒涎瞪著,又不去。看見桌底下一罈白泥頭酒,貼著紅紙帖兒,問道:「嫂子,是那里酒?打開篩壺來俺每吃。耶喲!你自受用。」婦人道:「你趁早兒休動,是宅里老爹送來的,你哥還沒見哩!等他來家,有便倒一甌子與你吃。」韓二道:「等什麼哥,就是皇帝爺的,我也吃一鍾兒。」纔待搬泥頭,被婦人劈手一推,奪過酒來,提到屋裡去了。把二搗鬼仰八叉推了一交,半日扒起來,惱羞變成怒,口裡喃喃吶吶罵道:「賊淫婦!我好意帶將兒來,見你獨自一個冷落落,和你吃盃酒。你不理我,倒推我一交。我教你不要慌,你另敘上了有錢的漢子,不理我了,要把我打開,故意的連我囂我,訕我又趨我。休教我撞見,我教你這不值錢的淫婦,白刀子進去,紅刀子出來!」婦人見他的話不防頭,一點紅從耳畔起,須臾紫脹了雙腮。便取棒槌在手,趕著打出來,罵道:「賊餓不死的殺才!倒了你那里〈口床〉醉了,來老娘這里撒野火兒!老娘手裡饒你不過!」那二搗鬼口裡,喇喇哩哩罵淫婦,直罵出門去。不想西門慶正騎馬來,見了他,問是誰。婦人道:「情知是誰!是韓二那廝,見他哥不在家,要便耍錢輸了,吃了酒來毆我。有他哥在家,常時撞見打一頓。」那二搗鬼一溜跑了。西門慶又道:「這少死的花子!等我明日到衙門裡,與他做功德!」婦人道:「又教爹惹惱。」西門慶道:「你不知,休要慣了他。」婦人道:「爹說的是。自古良善被人欺,慈悲生患害。」一面讓西門慶明間內坐。西門慶吩咐棋童回馬家去。叫玳安兒:「你在門首看,但掉著那光棍的影兒,就與我鎖在這里,明日帶衙門裡來。」玳安道:「他的魂兒聽見爹到了,不知走的那里去了!」西門慶坐下,婦人見畢禮,連忙屋裡叫丫鬟錦兒,拏了一盞菓仁茶出來,與西門慶吃,就叫他磕頭。西門慶道:「也罷,倒好個孩子。你且將就使著罷。」又道:「老馮在這里?怎的不替你拏茶?」婦人道:「馮媽媽他老人家,我央及他廚下使著手哩。」西門慶又道:「頭里我使小廝送來的那酒,是個內臣送我的竹葉清酒哩 。裡頭有許多藥味,甚是峻利。我前日見你這里打的酒,道吃不上口,我所以拏的這罈酒來。」婦人又道了萬福,說:「多謝爹的酒!正是這般說,俺每不爭氣,住在這僻巷子里,又沒個好酒店,那里得上樣的酒來吃!只往大街上取去。」西門慶道:「等韓夥計來家,你和他計較。等子獅子街那里,替你破幾兩銀子,買下房子,等你兩口子亦發搬到那里住去罷。舖子里又近,買東西諸事方便。」婦人道:「爹說的是。看你老人家怎的可憐見,離了這塊兒也好。就是你老人家行走,也免了許多小人口嘴。咱行的正,也不怕他。爹心裡要處自情處。他在家和不在家,一個樣兒,也少不的打這條路兒來。」說一回,房里放下卓兒,請西門慶房裡寬了衣服。坐須臾,安排酒菜上來,卓上無非是些雞鴨魚肉嗄飯點心之類。婦人陪定,把酒來斟。不一時,兩個並肩疊股而飲。吃得酒濃時,兩個脫剝上床交歡,自在頑耍。婦人早已床炕上,鋪的厚厚的被褥,被裡薰的噴鼻香。西門慶見婦人好風月,一徑要打動他,家中袖了一個錦包兒來。打開裏面,銀托子、相思套、硫黃圈、藥煮的白綾帶子、懸玉環、封臍膏、勉鈴,一弄兒淫器。那婦人仰臥枕上,玉腿高蹺,口舌內吐,西門慶先把勉鈴教婦人自放牝內,然後將銀托束其根,硫黃圈套其首,臍膏貼于臍上,婦人以手導入牝中,兩相迎湊,漸入大半。婦人呼道:「達達,我只怕你蹲的腿酸,拏過枕頭來,你墊著坐,等我淫婦自家動罷!」又道:「只怕你不自在,你把淫婦腿弔著合,你看好不好?」西門慶真個把他腳帶,解下一條來,拴他一足,弔在床槅子上低著拽,拽的婦人牝中之津,如蝸之吐涎,綿綿不絕,又拽出好些白漿子來。西門慶問道:「你如何流這些白?」纔待要抹之。婦人道:「你休抹,等我吮咂了罷!」于是蹲跪他面前,吮吞數次,嗚咂有聲。咂的西門慶淫心頓起,弔過身子,兩個幹後庭花。龜頭上有硫黃濡研難澀,婦人蹙眉,隱忍半晌,僅沒其稜。西門慶于是頗作抽已,而婦人用手摸之,漸入大半,把屁股坐在西門慶懷里,回首流眸,作顫聲叫:「達達,慢著些!往後越發粗大,教淫婦怎生挨忍?」西門慶且扶起股,觀其出入之勢,因叫婦人小名:「王六兒我的兒,你達不知心里怎的,只好這一庄兒。不想今日遇你,正可我之意。我和你明日生死難開。」婦人道:「達達,只怕後來耍的絮煩了,把奴不理,怎了?」西門慶道:「相交下來,纔見我不是這樣人。」說話之間,兩個幹勾一頓飯時。西門慶令婦人沒高低,淫聲浪語。叫著纔過,婦人在下,一面用手舉股,承受其精,樂極情濃一泄如注。已而拽出那話來,帶著圈子,婦人還替他吮咂淨了。兩個方纔並頭交股而臥。正是:

「一般滋味美,  好耍後庭花。」

有詩為證:

「美寃家,一心愛折後庭花。尋常只在門前里走,又被開路先鋒把住了。放在戶中難禁受,轉絲韁,勒回馬;親得勝。弄的我身上麻。蹴損了奴的粉臉,粉臉那丹霞。」

西門慶與婦人摟抱到二鼓時分,小廝馬來接,方纔起身回家。到次日早,衙門裡,差了兩個緝捕,把二搗鬼拏到提刑院,只當做掏摸土賊,不由分說,一夾二十,打的順腿流血,睡了一個月,險不把命花了。往後嚇了影也再不敢上婦人門纏提了。正是:

「恨小非君子,  無毒不丈夫!」

遲了幾日,來保、韓道國一行人東京回來,備將前事,對西門慶說:「翟管家見了女子,甚是歡喜,說費心。留俺在府裡住了兩日,討了回書,送了爹一匹青馬,封了韓夥計女兒五十兩銀子禮錢,又與了小的二十兩盤纏。」西門慶道:「勾了。」看了回書,書中無非是知感不盡之意。自此兩家都下眷生名字,稱呼親家,不在話下。韓道國與西門慶磕頭,拜謝回家。西門慶道:「韓夥計你還把你女兒這禮錢收去,也是你兩口兒恩養孩兒一場。」韓道國再三不肯收,說道:「蒙老爹厚恩,禮錢已是前日有了。這銀子小人怎好又受得?從前累的老爹好少哩!」西門慶道:「你不依,我就惱了。你將回家,不要花了,我有個處。」那韓道國就磕頭謝了,拜辭回去。老婆見他漢子來家,滿心歡喜。一面接了行李,與他拂了塵土,問他長短,孩子到那里好麼?這道國把往回一路的話,告訴一遍,說:「好人家,孩子到那里,就與了三間房,兩個丫鬟伏侍。衣服頭面是不消說。第二日就領了後邊,見了太太。翟管家甚是歡喜,留俺每住了兩日,酒飯連下人都吃不了。又與了五十兩禮錢,我再三推辭,大官人又不肯,還教我拏回來了。」因把銀子與婦人收了,婦人一塊石頭方落地。因和韓道國說:「咱到明日,還得一兩銀子謝老馮。你不在,虧他常來做伴兒。大官人那里,也與了他一兩。」正說著,只見丫頭過來遞茶。韓道國道:「這個是那里大姐?」婦人道:「這個是咱新買的丫頭,名喚錦兒。過來與你爹磕頭。」磕了頭,丫頭往廚下去了。老婆如此這般,把西門慶勾搭之事,告訴一遍:「自從你去了,來行走了三四遭,纔使四兩銀子買了這個丫頭。但來一遭,帶一二兩銀子來。第二的不知高低,氣不憤,走這里放水,被他撞見了,拏到衙門里打了個臭死,至今再不敢來了。大官人見不方便,許了要替咱每大街上買一所房子,教咱搬到那里住去。」韓道國道:「嗔道他頭里不受這銀子,教我拏回來,休要花了。原來就是這些話了。」婦人道:「這不是有了五十兩銀子,他到明日,一定與咱多添幾兩銀子,看所好房兒。也是我輸了身一場,且落他些好供給穿戴!」韓道國道:「等我明日往舖子裡去了,他若來時,你只推我不知道,休要怠慢了他;凡事奉他些兒。如今好容易撰錢,怎麼趕的這個道路!」老婆笑道:「賊強人,倒路死的!你倒會吃自在飯兒,你還不知老娘怎樣受苦哩!」兩個又笑了一回,打發他吃了晚飯,夫婦收拾歇下。到天明,韓道國宅裡討了鑰匙,開舖子去了;與了老馮一兩銀子謝他,俱不必細說。一日,西門慶同夏提刑衙門回來。夏提刑見西門慶騎著一匹高頭點子青馬,問道:「長官,那疋白馬怎的不騎?又換了這匹馬,到好一匹馬!不知口裡如何?」西門慶道:「那馬在家歇他兩日兒。這馬是昨日東京翟雲峯親家送來的,是西夏劉參將送他的,口裡纔四個牙兒,腳程緊慢,多有他的。只是有些毛病兒,快護糟踅蹬。初時著了路上走,把膘息跌了許多,這兩日,纔吃的好些兒了。」夏提刑道:「這馬甚是會行,只好長騎著,每日蹗街道兒罷了,不可走遠了他。論起在咱這里,也值七八十兩銀子。我學生騎的那馬,昨日又瘸了,今早來衙門裡來,旋拏帖兒問舍親借了這匹馬騎來了,甚是不方便。」西門慶道:「不打緊,長官沒馬,我家中還有一匹黃馬,送與長官罷。」夏提刑舉手道;「長官下顧,學生奉價過來。」西門慶道:「不須計較,學生到家就差人送來。」兩個走到西街口上,西門慶舉手分路來家;到家就使玳安把馬送去。夏提刑見了大喜,賞了玳安一兩銀子,與了回帖兒,說:「多上覆,明日到衙門裡面謝。」過了兩月,乃是十月中旬時分。夏提刑家中做了些菊花酒 ,叫了兩名小優兒,請西門慶一敘,以酬送馬之情。西門慶家中吃了午飯,理了些事務,往夏提刑家飲酒。原來夏提刑備辦一席齊整酒殽,只為西門慶一人而設。見了他來,不勝歡喜,降階迎接,至廳上敘禮。西門慶道:「如何長官這等費心!」夏提刑道:「今年寒家做了些菊花酒 ,閒中屈執事一敘,再不敢他客。」于是見畢禮數,寬去衣服,分賓主而坐。茶罷著棋,就席飲酒敘談。兩個小優兒在旁彈唱。正是得多少:

「金尊進酒浮香蟻,  象板催箏唱鷓鴣。」

不說西門慶在夏提刑家飲酒。單表潘金蓮見西門慶許多時不進他房里來,每日翡翠衾寒,芙蓉帳冷。那一日把角門兒開著,在房內銀燈高點,靠定幃屏,彈弄琵琶。等到二三更,便使春梅瞧數次,不見動靜。正是:

「銀箏夜久慇懃弄,  寂寞空房不忍彈。」

取過琵琶,橫在膝上,低低彈了個二犯江兒水以遣其悶。在床上和衣兒又睡不著,不免:

「悶把幃屏來靠,  和衣強睡倒。」

猛聽的房簷上鐵馬兒一片聲響,只道西門慶來到敲的門環兒響,連忙使春梅去瞧。他回頭:「娘錯了。是外邊風起落雪了。」婦人于是彈唱道:

「聽風聲嘹 ,雪酒窗寮,任水花片片飄。」

一回兒,燈昏香盡,心里欲待去剔續,見西門慶不來,又意兒懶的動旦了。唱道:

「懶把寶燈挑,慵將香篆燒。(只是捱一日似三秋,盼一夜如半夏。)捱過今宵,怕到明朝。細尋思,這煩惱,何日是了?(暗想負心賊,當初說的話兒,心中由不的我傷情兒。)想起來,今夜裡,心兒內焦。誤了我青春年少。(誰想你弄的我三不歸,四捕兒著他。)你撇的人有上稍來沒下稍!」

且說西門慶約一更時分,從夏提刑家吃了酒歸來,一路天氣陰晦,空中半雨半雪下來。落在衣服上,多化了。不免打馬來家。小廝打著燈籠,就不到後邊,逕往李瓶兒房來。李瓶兒迎著,一面替他拂去身上雪霰。西門慶穿著青絨獅子補子,坐馬白綾襖子,忠靖段巾,皂靴棕套,貂鼠風領。李瓶兒替他接了衣服,止穿綾敞衣,坐在床上,就問:「哥兒睡了不曾?」李瓶兒道:「小官兒頑了這回,方睡下了。」西門慶吩咐:「叫孩兒睡罷,休要沉動著,只怕諕醒他。」迎春于是拏茶來吃了。李瓶兒問:「今日吃酒來的早?」西門慶道:「夏龍溪還是前日因我送了他那匹馬,今日全為我費心治了一席酒請我;又叫了兩個小優兒。和他坐了這一回,見天氣下雪,來家早些。」李瓶兒道:「你吃酒?教丫頭篩酒來你吃。大雪裡來家,只怕冷哩。」西門慶道:「還有那葡萄酒 ,你篩來我吃。今日他家吃的是自造的菊花酒 ,我嫌他〈肴欠〉香〈肴欠〉氣的,我沒大好生吃。」于是迎春放下卓兒,就是幾碟醃雞兒嗄飯,細巧菓菜之類。李瓶兒拏杌兒在旁邊坐下,卓下放著一架小火盆兒。這里兩個吃酒,潘金蓮在那邊屋裡冷清清,獨自一個兒坐在床上,懷抱著琵琶,桌上燈昏燭暗。待要睡了,又恐怕西門慶一時來;待要不睡,又是那盹困,又是寒冷。不免除去冠兒,亂挽烏雲,把帳兒放下半邊來,擁衾而坐。正是:

「倦倚綉床愁懶睡,  低垂錦帳綉衾空;

早知薄〈亻辛〉輕摒棄,  辜負奴家一片心。」

又唱道:

「懊恨薄情輕棄,  離愁閒自惱。」

又喚春梅過來:「你去外邊再瞧瞧,你爹來了沒有?快來回我話。」那春梅走去,良久回來,說道:「娘,還認爹沒來哩!爹來家不耐煩了,在六娘屋里吃酒的不是?」這婦人不聽罷了,聽了如同心上戳上幾把刀子一般。罵了幾句負心賊,由不得撲簌簌眼中流下淚來。一逕把那琵琶兒放得高高的,口中又唱道:

「論殺人好恕,情理難饒。負心的,天鑒表!(好好我題起來,又是那疼他,又是那恨他。)心痒痛難掃,愁懷悶自焦。(叫了聲,賊狠心的寃家,我比他何如?鹽也是這般鹽,醋也是這般醋,磚兒能厚,瓦兒能薄,你一旦棄舊憐新!)讓了甜桃,去尋酸棗。(不合今日教你哄了!)奴將你這定盤星兒錯認了。(合)想起來,心兒里焦。誤了我青春年少,你撇的人有上稍來沒下稍!」

「為人莫作婦人身,  百般苦樂由他人,

痴心老婆負心漢,  悔莫當初錯認真。」

「常記的當初相聚,痴心兒望到老。(誰想今日他把心變了,把奴來一旦輕拋不理。正如那日。)被雲遮楚岫,水 籃橋。打拆開鸞鳳走,(到如今當面對語,心隔千山。隔著一堵墻,咫尺不得相見。)心遠路非遙,(意散了,如鹽落水,如水落沙相似了。)情疏魚雁杳。「空教我有情難控訴。」地厚天高,(空教我無夢到陽臺。)夢斷魂勞,俏寃家,這其間,心變了。(合)想起來,心兒裡焦。誤了我青春年少,你撇的人有上稍來沒下稍!」

西門慶正在房中,和李瓶兒吃酒,忽聽見這邊房里,彈的琵琶之聲,便問:「是誰彈琵琶?」迎春答道:「是五娘在那邊彈琵琶響。」李瓶兒道:「原來你五娘還沒睡哩!綉春,你快去請你五娘來吃酒,你說俺娘請哩。」那綉春去了。李瓶兒忙教迎春那邊安下個坐兒,放個鍾筯在面前。良久,繡春走來說:「五娘摘了頭,不來哩。」李瓶兒道:「迎春,你再去請你五娘去。你說娘和爹請五娘哩。」不多時,迎春來說:「五娘把角門兒關了。說吹了燈,睡下了。」西門慶道:「休要信他小淫婦兒。等我和你兩個拉他去,務要把他拉了來,咱和他下盤棋耍子。」于是和李瓶兒,同來打他角門。打了半日,春梅把角門子開了。西門慶拉著李瓶兒,進入他房中,只見婦人坐在帳上,琵琶放在傍邊。西門慶道:「怪小淫婦兒,怎的兩三轉請著你不去?」金蓮坐在床,紋絲兒不動,把臉兒沉著,半日說道:「那沒時運的人兒,丟在這冷屋裡,隨我自生兒由活的,又來揪採我怎的?沒的空費了你這個心留著別處使。」西門慶道:「怪奴才,八十歲媽媽沒牙,有那些唇說的!李大姐那邊請你和他下盤棋兒,只顧等你不去了。」李瓶兒道:「姐姐,可不怎的?我那屋里擺下棋子了,咱每閑著下一盤兒,賭盃酒吃。」金蓮道:「李大姐,你每自去。我摘了頭,你不知我心里不耐煩。我如今睡也比不的你每心寬閑散。我這兩日,只有口遊氣兒。黃湯淡水,誰嚐著來,我成日睜著臉兒過日子哩!」西門慶道:「怪奴才,你好好兒的,怎的不好?你若心內不自在,早對我說,我好請太醫來看你。」金蓮道:「你不信,教春梅拏過我的鏡子來,等我瞧。這兩日瘦的相個人模樣哩!」春梅把鏡子真個遞在婦人手裡,燈下觀看。正是:

「羞對菱花拭粉粧,  為郎憔瘦減容光;

閉門不顧閒風月,  任您梅花自主張。」

「差對菱花來照,蛾眉懶去掃;暗消磨了精神,折損了丰標,瘦伶仃不甚好。」

西門慶拏過鏡子,也照了照,說道:「我怎麼不瘦?」金蓮道:「拏什麼比的你?每日碗酒塊肉,吃的肥胖胖的,專一只奈何人!」被西門慶不由分說,一屁股挨著他坐在床上,摟過脖子來,就親了個嘴。舒手被里,摸見他還沒脫衣裳。兩隻手齊插在他腰里去,說道:「我的兒,真個瘦了些!」金蓮道:「怪行貨子!好冷手,冰的人慌!莫不我哄了你不成?」正是:

「香褪了海棠嬌,  衣惚了楊柳腰。」

說道:「我著香腮,拋下珠淚來。我的苦惱,誰人知道?眼淚打肚里流罷了。」

「悶下無聊,攘攘勞勞,淚珠兒到今滴盡了。(合)想起來,心裡亂焦。誤了我青春年少,撇的人有上稍來沒下稍!」

亂了一回,西門慶還把他強死強活,拉到李瓶兒房內,下了一盤棋,吃了一回酒。臨起身,李瓶兒見他這等臉酸,把西門慶攛掇過他這邊歇了。正是得多少:

「腰瘦故知閒事惱,  淚痕只為別情濃。」

有詩為證:

「自從別後減容光,  萬轉千回懶下床;

虧殺瓶兒成好事,  得教巫女會襄王。」

畢竟未知後來如何,且聽下回分解:
