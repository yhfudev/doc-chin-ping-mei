%# -*- coding: utf-8 -*-
%!TEX encoding = UTF-8 Unicode
%!TEX TS-program = xelatex
% vim:ts=4:sw=4
%
% 以上设定默认使用 XeLaTex 编译,并指定 Unicode 编码,供 TeXShop 自动识别

%第四十三回 
\chapter{為失金西門慶罵金蓮\KG 因結親月娘會喬太太}

「細推今古事堪愁,  貴賤同歸土一丘,

漢武玉堂人豈在?  石家金谷水空流;

光陰自旦還將暮,  草木從春又到秋,

閑事與時俱不了,  且將身入醉鄉遊。」

話說西門慶歸家,已有三更時分。到于後邊,吳月娘還未睡,正和吳大妗子眾人坐着說話。見李瓶兒還伺候着,與他遞酒。大妗子見西門慶來家,就過那邊屋裡去了。月娘見他有酒了,打發他脫了衣裳,只教李瓶兒與他磕了頭,同坐下,問了回今日酒席上話。玉簫點茶來吃,因有大妗子在,就往孟玉樓房中歇了一夜。到次日,廚役早來收拾擡辦酒席。西門慶先到衙門中拜牌,大發放。夏提刑見了,致謝昨房下厚擾之意。西門慶道:「日昨甚是簡慢,恕罪恕罪!」來家,有喬大戶家使了孔嫂兒,引了喬五太太那裡家人,送禮來了,一壜南酒 、四樣殽品。西門慶收了,管待家人酒飯。孔嫂兒進裡邊月娘房裡坐的。吳舜臣媳婦兒鄭三姐轎子先來了,拜了月娘。眾人多陪着孔嫂兒吃茶。正值李智、黃四關了一千兩香蠟銀子。賁四從東平府押了來家。應伯爵打聽得知,亦走來幫扶交與西門慶。令陳經濟拿天平在廳上盤秤,兌明白收了,還久五百兩;又銀一百五十兩利息,息當日黃四拿出四錠金鐲兒來,重三十兩,算一百五十之數。別的搗換了合同。西門慶分付二人:「你等過燈節再來計較,我連日家中有事。」那李智、黃四,老爹長老爹短,千恩萬謝出門。應伯爵因記掛着二人許了他些業障兒,趁此機會好問他。正要跟隨同去,又被西門慶叫住說話。西門慶因問:「昨日你每三個,怎的三不知,不和我說就走了?我使小廝落後趕你不着了。」伯爵道:「昨日甚是深擾哥,本等酒勾多了。我見哥也有酒了。今日嫂子家中擺酒,已定還等哥說話。俺每不走了,還只顧纏到多咱?我猜哥今日也沒得往衙門裡去,本等連日辛苦。」西門慶道:「我昨日來家,已有三更天氣。今日還早到衙門拜了牌,坐廳大發放,理了回公事。如今家中治料堂客之事,今日觀裡打上元醮,拈了香回來,還趕了往周菊軒家吃酒去。不知到多咱纔得來家?」伯爵道:「還是虧哥好神思,你的大福。不是面獎,若是第二個,也成不的!」兩個說了一回,西門慶要留伯爵吃飯。伯爵道:「我不吃飯去罷!」西門慶又問:「嫂子怎的不來?」伯爵道:「房下轎子已叫下來,便來也。」舉手作辭出門,一直趕往李智、黃四去了。正是:

「假饒駕霧騰雲術,  取火鑽冰只要錢。」

都說西門慶,打發伯爵去了,把手中拿着黃烘烘四錠金鐲兒,心中甚是可愛。口中不言,心裡暗道:「李大姐生的這孩子,甚是腳硬。一養下來,我平地就得此官。我今日與喬家結親,又進這許多財。」于是用袖兒抱着那四錠金鐲兒,也不到後邊,徑往花園內李瓶兒房裡來。正往潘金蓮角門首所過,只見金蓮正出來看見,叫住問道:「你手裡托的是什麼東西兒?過來我瞧瞧。」那西門慶道:「等我回來與你瞧。」托着一直往李瓶兒那邊去了。那婦人見叫不回他來,心中就有幾分羞訕,說道:「什麼罕稀貨?忙的這等諕人子刺刺的。不與我瞧罷!賊跌拆腿的三寸貨強盜!正麼剛遂進他門去!正走麼矻齊的把那兩條腿〈扌歪〉拆了,纔見報了我的眼!」都說西門慶拿着金子,走入李瓶兒房裡。見李瓶兒纔梳了頭,奶子正抱着孩子頑耍。西門慶一徑裡把那四個金鐲兒抱着,教他手兒撾弄。李瓶兒道:「是那裡的?只怕冰了他手。」西門慶悉把李智、黃四今日還銀子,推折利錢,約這金子。這李瓶兒生怕冰着他,取了一方通花汗巾兒,與他熱着耍子。只見玳安走來,說道:「雲夥計騎了兩疋馬來,在外邊,請爹出去瞧。」西門慶道:「雲夥計他是那裡的馬?」玳安道:「他說是他哥雲參將邊上稍來的馬,只說會行。」正說着,只見後邊李嬌兒、孟玉樓陪着大妗子、并他媳婦兒鄭三姐,多來李瓶兒房裡看官哥兒。西門慶丟下那四錠金子,就往外邊大門首看馬去了。李瓶兒見眾人來到,只顧與眾人見禮讓坐,也就忘記了孩子拿着這金子,弄來弄去少了一錠。只見奶子如意兒問李瓶兒說道:「娘沒曾收哥兒耍的那錠金子?只三錠,少了一錠了。」李瓶兒道:「我沒曾收,我把汗巾替他裹着哩!」如意兒道:「汗巾子也落在地下了,我料來那裡得那錠金子來?」屋裡就亂起來,奶子問迎春,迎春就問老馮,老馮道:「耶嚛,耶嚛!我老身就瞎了眼,也沒看見。老身在這裡恁幾年,就是折針,我也不敢動。娘他老人家知道我,就是金子我老身也不愛。你每守着哥兒,沒的寃枉起我來了!」李瓶兒笑道:「你看這媽媽子說混話。這裡不見的,不是金子都是什麼?」又罵迎春:「賊臭肉!平白亂的是些什麼?等你爹進來,等我問他。只怕是你爹收了,怎的只收一錠兒?」孟玉樓問道:「是那裡金子?」李瓶兒道:「是他爹外邊拿來的,與孩子耍。誰知道是那裡的!」不想西門慶在門首看了一回馬,眾夥計家人多在跟前。教小廝來回騎溜了兩盪。西門慶:「雖是兩疋東路來的馬,鬃尾醜,不十分會行。論小行也罷!」因問雲夥計道:「此馬你令兄那裡要多少銀子?」雲離守道:「兩疋只要七十兩。」西門慶道:「也不多,只是不會行。你還牽了去。另有好馬騎來,倒不說銀子。」說畢,西門慶進來。只見琴童來請:「六娘房裡請爹哩!」于是走入李瓶兒房裡來。李瓶兒問他:「金子你收了一錠去了?如何只三錠在這裡?」西門慶道:「我丟下就出來了外邊看馬,誰收那錠來?」李瓶兒道:「你沒收,都往那裡去了?尋了這一日沒有。奶子推老馮。急的那老馮賭身罰咒只是哭。」西門慶道:「端的是誰拿了?由他,慢慢兒尋罷!」李瓶兒道:「裡頭要尋,已後邊和大妗子女兒兩個來時亂着,就忘記了。我只說你收了出去,誰知你也沒收,就兩躭了。尋起來,諕的他們多走了。」于是把那三錠,還交與西門慶收了。正值賁四傾了一百兩銀子來交,西門慶往後邊收兌銀子去。且說潘金蓮聽見李瓶兒這邊嚷不見了孩子耍的一錠金鐲子,得不的風兒就是雨兒,就先走來房裡告月娘說:「姐姐,你看三寸貨幹的營生。隨你家怎的有錢,也不該拿金子與孩子耍。」月娘道:「剛纔他每告我說,他房裡好不翻亂,說不見了金鐲子。端的不知那裡的金鐲子?」金蓮道:「誰知他是那裡的?你還沒見他頭裡從外邊拿進來,那等用襖子袖兒托着,恰是八蠻進寶的一般!我問他是什麼?拿過來我瞧瞧。頭兒也不回,一直奔命往屋裡去了。遲了一回,反亂起來,說不見了一錠金子,乾淨就是他。學三寸貨說,不見了由他,慢慢兒尋罷。你家就是王十萬,也使不的!一錠金子,至少重十來兩,也值個五六十兩銀子。平白就罷了!瓮裡走了鱉,左右是他家一窩子。再有誰進他屋裡去?」正說着,只見西門慶進來,兌收賁四傾的銀子。把剩的那三錠金子,交與月娘收了。因告訴月娘:「此是李智、黃四還的這四錠金子,拿到與孩子耍了耍,就不見了一錠。」分付月娘:「你與我把各房裡丫頭,叫出來審問審問。我使小廝街上買狼觔去了。早拿出來便罷,不然,我就教狼觔抽起來」月娘道:「論起來,這金子也不該拿與孩子,沉甸甸冰着他,怕一時砸了他手腳,怎了?」潘金蓮在旁,接過來說道:「不該拿與孩子耍,只恨拿不到他屋哩。頭裡叫着,想回頭也怎的?恰似紅眼軍搶將來的,不教一個人兒知道。這回不見了金子,虧你怎麼有臉兒來對大姐姐說,教大姐姐替你查考各房裡丫頭。教各房裡丫頭,口裡不笑,〈毛皮〉裡笑罷了。」說的西門慶急了,走向前把金蓮按在月娘炕上。提拳來,罵道:「狠殺我罷了!不看世界面上,把你這小歪刺骨兒,就一頓拳頭打死了!單管嘴尖舌快的,不管你事也來插一腳。」那潘金蓮就假做喬張,就哭將起來,說道:「我曉的你倚官仗勢,倚財為主,把心來橫了,只欺貧的是我。你說你,這般把這一個半個人命兒打死了,不放在意裡。那個攔看你手兒哩不成!你打不是?有的是,我隨你怎麼打,難得只打的有這口氣兒在着。若沒了,愁我家那病媽媽子來,不問你要人?隨你家怎麼有錢有勢,和你家一來一狀,你說你是衙門裡千戶便怎的?無遏只是個破砂帽債殼子窮官罷了!能禁的幾個人命耳?就不是,教皇帝敢殺下人也怎的?」幾句說的西門慶反呵呵笑了,說道:「你有這原來小歪刺骨兒,這等刁嘴,我是破紗帽窮官,教丫頭取笑我的紗帽來,我這紗帽那塊兒放着破?這裡清河縣問聲,我少誰家銀子,你說我是債殼子!」金蓮道:「你怎的叫我是歪刺骨來?」因蹺起一隻腳來。「你看老娘這腳!那些兒放着歪?你怎罵我是歪刺骨?那刺骨也不怎的。」月娘在旁笑道:「你兩個銅盆撞了鐵刷帚。常言:『惡人見了惡人磨,見了惡人沒奈何。』自古嘴強的爭一步。六姐,也虧你這個嘴頭子,不然嘴鈍些兒也成不的!」那西門慶見奈何不過他,穿了衣裳往外去了。迎見玳安來,說:「周爹家差人邀來了,備馬了,請問爹先往打醮處去?往周爺家去?」西門慶分付:「打醮處,教你姐夫去罷。到了那裡拈了香,快來家裡看。伺候馬,我往你周爺家吃酒去就是了!」說着。書童兒拿冠帶過來,打發穿了,繫上帶。只見王皇親家,扮戲兩個師父,率眾過來,與西門慶叩頭。西門慶教書童看飯與他吃,說:「今日你等用心唱,伏侍眾奶奶,我自有重賞。休要上邊打箱去。」那師父跪下說道:「小的每若不用心答應,豈敢討賞?」西門慶因分付書童:「他唱了兩日,連賞賜封下五兩銀子賞他。」書童應諾:「小的知道了。」西門慶就上馬往周守備家吃酒去了。單表潘金蓮在上房陪吳妗子坐的。吳月娘便說:「你還不往屋裡,勻勻那臉去?揉的恁紅紅的,等住回人來看着,什麼張致!誰教你惹他來?我倒替你捏兩把汗。若不是我在根前勸着,梆石鬼是也有幾下子打在身上。漢子家,臉上有狗毛,不知好歹,只顧下死手的,和他起來了!不見了金子,隨他不見去,尋不尋不在你。又不在你屋裡不見了,平白扯着脖子和他強怎麼?你也丟了這口氣兒罷!」幾句說的金蓮閉口無言,往屋裡勻臉去了。不一時,只見李瓶兒和吳銀兒多打扮出來,到月娘房裡。月娘問他:「金子怎的不見了?剛纔惹得他爹和六姐兩個在這裡好不辨了這回嘴,差些兒沒曾辨惱了打起來!乞我勸開了。他爹便往人家吃酒去了。分付小廝買狼觔去了。等他晚上來家,要把各房丫頭抽起來。你屋裡丫頭老婆管着那一門兒來?就看着孩子耍,便不見了他一錠金子。是一個半個錢的東西兒也怎的!」李瓶兒道:「平白他爹,拿進四錠金子來與孩子耍,我亂着陪大妗子和鄭三姐並他二娘坐着說話,誰知就不見了一錠。如今丫頭推奶子,奶子推老馮。急的那媽媽哭哭啼啼,只要尋死,無眼難明勾當。如今寃誰的是?」吳銀兒道:「天麼!天麼!早是今日,我在好。每常我還和哥兒耍子,這邊屋裡梳頭,沒曾過去。不然,難為我了!雖然爹娘不言語,你我心上何安?誰人不愛錢?俺裡邊人家,最忌叫這個名聲兒,傳出去醜聽!」正說着,只見韓玉釧兒、董嬌兒兩個提着衣包兒進來,笑嘻嘻先向月娘、大妗子、李瓶兒磕了頭起來。望着吳銀兒拜了一拜,說道:「銀姐昨已來了,沒家去?」吳銀兒道:「你兩個怎的曉得?」董嬌兒道:「昨日俺兩個都在燈巿街房子裡唱來,大爹對俺們說,教俺今日來伏侍奶奶。」一面月娘讓他兩個坐下,須臾,小玉拏了兩盞茶來。那韓玉釧兒、董嬌兒連忙立起身來接茶,還望小玉拜了一拜。吳銀兒因問:「你兩個昨日唱多咱散了?」韓玉釧:「俺們到家,也有二更多了。同你兄弟李銘,都一路去來。」說了一回話,月娘分付玉簫:「早些打發他們吃了茶罷!等住回只怕那邊人來忙了。」一面放下卓兒,兩方春槅,四盒茶食。月娘使小玉:「你二娘房裡請了桂姐來,同吃了茶罷。」不一時,和他姑娘來到,兩個各道了禮數,坐下同吃了茶,收過家活去。忽見迎春打扮着,抱了官哥兒來,頭上戴着金梁段子八吉祥帽兒,身穿大紅氅衣兒,下邊白綾襪兒段子鞋兒,胸前項牌符索,手上小金鐲兒。李瓶兒看見,說道:「小大官兒,沒人請你,來做甚麼?」一面接過來,放在膝蓋上。看見一屋裡子,把眼不住的看了這頭,看那一個。桂姐坐在月娘炕上,笑引鬬他耍子,道:「哥子只看就這裡,想必只要我抱他。」於是用手引了他引兒,那孩子就撲到懷裡教他抱着。吳大妗子笑道:「恁點小孩兒,他也曉的愛好。」月娘接過來說:「他老子是誰?到明日大了,管情也是小嫖頭兒。」孟玉樓道:「若做了小嫖頭兒,教大媽媽就打死了。」那李瓶兒道:「小廝,你姐姐抱,只休溺了你姐姐衣服,我就忙打死了。」那桂姐道:「耶嚛,怕怎麼!溺了也罷,不妨事。我心裡要抱哥兒耍耍兒。」于是與他兩個嘴碰嘴兒耍子。只見孟玉樓也來了,董嬌兒、韓玉釧兒下來行禮畢,坐下說道:「俺兩個來了這一日,還沒曾唱個兒與娘們聽。」因叫:「小玉姐,你取樂器來,等俺唱。」那小玉便取箏和琵琶,遞與他二人。當下韓玉釧兒琵琶,董嬌兒彈箏,吳銀兒也在旁邊陪唱;於是唱了一套「繁華滿月開,金索掛梧桐。」唱出一句來,端的有落塵遶梁之聲,裂石流雲之響。把官哥兒諕的在桂姐懷裡,只磕倒着,再不敢擡頭出氣兒。月娘看見,便叫:「李大姐,你接過孩子來,教迎春抱的屋裡去罷。好箇不長俊的小廝,你看諕的那臉兒!」這李瓶兒連忙接過來,教迎春掩着他耳朵,抱的往那邊房裡去了。於是四個唱的,齊合着聲兒,唱這一套詞道:

「繁花滿月開,錦被空閑在。劣性寃家,誤得我忒毒害!我前生少欠他,今世裡相思債。廢寢忘餐,倚定門兒待。房櫳靜悄如何捱?」

〔罵玉郎〕「冷清清房櫳,靜悄如何捱?獨自把幃屏倚,知他是甚情懷?想當初同行同坐同懽愛,到如今孤另另怎別劃?愁戚戚酒倦釃,羞慘慘花慵戴。」

〔東甌令〕「花慵戴,酒倦釃,如今曾約前期不見來。都應是他在那裡,那裡貪歡愛。物在人何在?空勞魂夢到陽臺,只落得淚盈腮。」

〔感皇恩〕「呀!只落得雨淚盈腮,都應是命裡合該!莫不是你緣薄咱分淺,都應是一般運拙時乖。怎禁那攪閒人是非,施巧計裁排。撕撏碎合歡帶,破分開鸞鳳釵,水淹浸楚陽臺。」

〔針線廂〕「把一床絃索塵埋,兩眉峰不展開。香肌瘦損愁無奈,懶刺繡傍粧臺。舊恨新愁,教我如何捱?我則怕蝶使蜂媒不再來。臨鸞鏡也問道朱顏未改,他又早先改。」

〔採茶歌〕「改朱頻瘦了形骸,冷清清怎生捱?我則怕梁山伯不戀祝英台。他若是背義忘恩尋罪責,我將那盟山誓海說的明白。」

〔解三醒〕「頓忘了盟山誓海,頓忘了音書不寄來。頓忘了枕邊許多恩和愛,頓忘了素體相挨。頓忘了神前雨下千千拜,頓忘了表記香羅紅繡鞋。說起旁人見,珠淚盈腮。」

〔烏夜啼〕「俺如今相離三月如隔數載,要相逢甚日何年再?則我這瘦伶仃形體如柴,甚時節還徹了相思債。又不見青鳥書來,黃犬音乖。每日家病懨懨,懶去傍粧臺。得團圓便把神羊賽,意廝搜心相愛。早成了鸞交鳳友,省的著蝶笑蜂猜。」

〔尾聲〕「把局兒牢鋪擺,情人終久再歸來,美滿夫妻百歲諧。」

四個唱的正唱着,只見玳安進來。月娘便問:「你邀請的眾奶奶們怎的這咱還不見來?」玳安道:「小的到喬親家娘那邊邀來,朱奶奶、尚舉人娘子,都過喬親家娘家來了。只等着喬五太太到了,就往咱這裡來。」月娘分付:「你就說與平安兒小廝,說教他在大門首看着。等奶奶們轎子到了,就先進來說。」玳安道:「大門前邊大廳上,鼓樂迎接哩,娘們都收拾伺候就是了。」月娘分付玳安,後廳明間鋪下錦毯,安放坐位,捲起簾來,金鈎雙控,蘭麝香飄。春梅、迎春、玉簫、蘭香都打扮起來,家人媳婦,都插金戴銀,披紅垂綠,準備迎接新親。只見應伯爵娘子兒應二嫂先到了,應寶跟着轎子。月娘等迎接進來,見了禮數,明間內坐下。向月娘拜了又拜,說:「俺家的常時打擾這裡,多蒙看顧。」良久,只聞喝道之聲漸近,月娘道:「姑娘好說,常時累你二爹。」前廳鼓樂响動,平安兒先進來報道:「喬太太轎子到了。」須臾黑壓壓一群人,跟着五頂大轎,落在門首。惟喬五太太轎子在頭裡。轎上是垂珠銀頂天青重沿綃金走水轎衣,使藤棍唱道。後面家人媳婦,坐小轎跟隨。四名校尉,擡衣箱火爐。兩箇青衣家人,騎着小馬,後面隨從。其餘者就是喬大戶娘子、朱臺官娘子、尚舉人娘子、崔大官媳婦、段大姐、并喬通媳婦,也坐着一頂小轎,跟來收疊衣裳。吳月娘這裡穿大紅五彩遍地錦,白獸朝麒麟段子通袖袍兒,腰束金鑲寶石鬬粧;頭上寶髻巍峨,鳳釵雙插,珠翠堆滿;胸前繡帶垂金,頂牌錯落;裙邊禁步明珠,與李嬌兒、孟玉樓、潘金蓮、李瓶兒,孫雪娥一箇箇打扮的似粉粧玉琢,錦繡耀目,都出二門迎接。只見眾堂客,簇擁着喬五太太進來,生的五短身材,約七旬多年紀,戴着疊翠寶珠冠,身穿大孔宮繡袍兒。近面視之,鬢髮皆白。正是:

「眉分八道雪,  髻綰一窩絲;

眼如秋水微渾,  鬢似楚山雲淡。」

接入後廳,先與吳大妗子叙畢禮數,然後與月娘等廝見。月娘再三請太太受禮,太太不肯。讓了半日,止受了半禮。次與喬大戶娘子,又叙其新親家之禮。彼此道及款曲,謝其厚儀。已畢,然後向,錦屏正面,設放一張錦裀座位,坐了喬五太太。其次坐就讓喬大戶娘子。喬大戶娘子再三辭說:「姪婦不敢與五太太上儹。」讓朱臺官、尚舉人娘子,兩箇又不肯。彼此讓了半日,喬五太太坐了首座,其餘客東主西,兩分頭坐了。當中大方爐火廂籠起火來,堂中氣煖如春。春梅、迎春、玉簫、蘭香,一般兒四箇丫頭,都打扮起來,身上一色都大紅粧花段襖兒,藍織金裙,綠遍地金比甲兒,在根前遞茶。良久,喬五太太對月娘說:「請西門大人出來拜見,叙叙親情之禮。」月娘道:「拙夫今日衙門中理公事去了,還未來家哩。」喬五太太道:「大人居于何官?」月娘道:「乃一介鄉民,蒙朝廷恩例,實授千戶之職,見掌刑名。寒家與親家那邊結親,實是有玷。」喬五太太道:「娘子說那裡話?似大人這等崢榮也夠了!昨日老身聽得舍姪女與府上做親,心中甚喜。今日我來會會,到明日席上好廝見。」月娘道:「只是有玷老太太名目。」喬五太太道:「娘子是甚怎說話?想朝廷不與庶民做親哩!老身說起來話長。如今當今東宮貴妃娘娘,係老身親姪女兒。他父母都沒了,止有老身。老頭兒在時,曾做世襲指揮使。不幸五十歲故了,身邊又無兒孫,輪着別門姪另替了。手裡沒錢,如今倒是做了大戶。我這箇姪兒,雖是差役立身,頗得過的日子,庶不玷污的門戶。」說了一回,吳大妗子對月娘說:「抱孩子出來與老太太看看,討討壽。」李瓶兒慌的走去,到房裡分付奶子抱了官哥來,與太太磕頭。喬太太看了,誇道:「好箇端正的哥哥!」即叫過左右,連忙向毡包內打開,捧過一端宮中紫閃黃錦段,并一付鍍金手鐲與哥兒戴。月娘,連忙下來拜謝了,請去房中換了衣裳。須臾,前邊捲棚內安放四張卓席,擺下茶。每卓四十碟,都是各樣茶菓甜食,美口菜蔬,蒸酥點心,細巧油酥餅饊之數。兩邊家人媳婦丫頭侍奉伏侍,不在話下。吃了茶,月娘就去後邊山子花園中,開了門,遊玩了一回下來。那時陳經濟打醮去,吃了午齋回來了,和書童兒、玳安兒又早在前廳擺放桌席齊整,請眾奶奶們遞酒上席。端的好筵席!但見:

「屏開孔雀,褥隱芙蓉。盤堆異菓奇珍,瓶插金花翠葉。爐焚獸炭,香裊龍涎。器列象州之古玩,簾開合浦之明珠。白玉碟高堆麟脯,紫金壺滿貯瓊槳。煮猩唇,燒豹胎 ,果然下筯了萬錢;烹龍肝,炮鳳髓 ,端的獻時品滿座。梨園子弟,簇捧著鳳管鸞簫;內院歌姬,緊按定銀箏象板。進酒佳人雙洛浦,分香侍女兩嫦娥,正是:兩行珠翠列階前,一派笙歌臨座上。」須臾,吳月娘與李瓶兒遞酒。階下戲子鼓樂嚮罷,喬太太與眾親戚,又親與李瓶兒把盞祝壽。李桂姐、吳銀兒、韓玉釧兒、董嬌兒四個唱的,在席前錦瑟銀箏,玉面琵琶,紅牙象板,彈唱起來,唱了一套壽比南山。下邊鼓樂響動,戲子呈上戲文手本。喬五太太分付下來,教做王日英元夜留鞋記。廚役上來獻小割燒鵝 ,賞了五錢銀子。比及割凡五道,湯陳三獻,戲文四摺下來,天色已晚。堂中畫燭流光者如山疊,各樣花燭都點起來。錦帶飄飄,彩繩低轉。一輪明月,從東而起,照射堂中,燈光掩映。來興媳婦惠秀,與來保媳婦惠祥,每人拏着一方盤菓餡元宵,都是銀鑲茶鍾,金杏葉茶匙放白糖玫瑰 ,馨香美口,走到上邊,春梅、迎春、玉簫、蘭香四人,分頭照席捧遞,甚是禮數周詳,舉止沉穩。階下動樂,琵琶箏阮、笙簫笛管,吹打了一套燈詞畫眉序「花月滿春城」唱畢,喬太太和喬大戶娘子,叫上戲子,賞了兩包一兩銀子。四個唱的,每人二錢。月娘又在後邊明間內,擺設下許多菓碟兒,留後座,四張桌子都堆滿了。唱的唱,彈的彈,又吃了一回酒。喬太太再三說晚了,要起身。月娘眾人款留不住,送在大門首;又攔了遞酒,看放烟火。兩邊街上看的人,鱗次蜂脾一般,平安兒同眾排軍執棍攔擋,再三還湧擠上來。須臾,放了一架烟火,兩邊人散了。喬太太和眾娘子方纔拜辭月娘等起身,上轎去了。那時已有三更天氣。然後又送應二嫂起身。月娘眾姊妹歸到後邊來,分付陳經濟、來興、書童、玳安兒看着廳上收拾家活,管待戲子并兩箇師範酒飯,與了五錢銀子唱錢,打發去了。月娘分付出來剩儹下一桌餚饌,半罈酒,請傳夥計、賁四、陳姐夫,說:「他們管事辛苦,大家吃鐘酒。就在大廳上安放一張桌兒,你爹不知多咱纔回。」許是還有殘燈未盡。當下傳夥計、賁四、經濟、來保上座,來興、書童、玳安、平安打橫,把酒來斟。來保叫:「平安兒,你還委箇人大門首,怕一時爹回,沒人看門。」平安道:「我教畫童看着哩!不妨事。」於是八箇人猜枚飲酒。經濟道:「你每休猜枚,大驚小唱的,惹後邊聽見。咱不如悄悄行令兒耍子。每人要一句,說的出免罰,說不出罰一大盃酒。」該傳夥計先說:「堪笑元宵草物。」賁四道:「人生歡樂有數。」經濟道:「趁此月色燈光。」來保道:「咱且休要辜負。」來興道:「纔約嬌兒不在。」書童道:「又學大娘分付。」玳安道:「雖然剩酒殘燈。」平安道:「也是春風一度。」眾人念畢,呵呵笑了。正是:

「飲罷酒闌人散後,  不知明月轉梅梢。」

畢竟未知後來如何,且聽下回分解:
