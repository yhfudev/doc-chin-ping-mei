%# -*- coding: utf-8 -*-
%!TEX encoding = UTF-8 Unicode
%!TEX TS-program = xelatex
% vim:ts=4:sw=4
%
% 以上设定默认使用 XeLaTex 编译,并指定 Unicode 编码,供 TeXShop 自动识别

%第六十四回 
\chapter{玉簫跪央潘金蓮\KG 合衛官祭富室娘}


\begin{showcontents}{}



「著人情思覺初闌,  失把鮫綃仔細看,

到老春蠶絲乃盡,  成灰蠟燭淚初乾;

鸞交鳳友驚風散,  軟玉嬌香異世問,

西子風流誇未了,  雞鳴殘月五更寒。」

話說眾人散了,已有雞唱時分。西門慶歇息去了。玳安拿了一大壺酒,幾碟下飯,在前邊鋪子裡,還和傅夥計,陳經濟同吃。傅夥計老頭子,熬到這咱,已是不樂。坐搭下鋪,倒在坑上就睡了。因向玳安道:「你自和平安兩個吃罷,陳姐夫想是也不來了。」這玳安櫃上點著夜燭,叫進平安來,兩個把那酒,你一鍾,我一盞都吃了。把家火收過一邊,平安便去門房裡去睡了。玳安一面關上鋪子門,上坑和傅夥計兩個通廝腳兒睡下。傅夥計閒中因話閒話,問起玳安說道:「你六娘沒了,這等樣棺槨,祭祀念經發送,也勾他了!」玳安道:「一來他是福好,只是不長壽。俺爹饒使了這些錢,還使不著俺爹的哩。俺六娘嫁俺爹,瞞不過你老人家,是知道,該帶了多少帶頭來。別人不知道,我知道。把銀子休說,只光金珠玩好玉帶縧環狄髻值錢寶石,還不知有多少。為甚俺爹心裡疼?不是疼人,是疼錢。是便是說起俺這過世的六娘性格兒,這一家子都不如他。又有謙讓,又和氣,見了人只是一面兒笑。俺每下人,自來也不曾呵俺每一呵,並沒失口罵俺每一句奴才,要的誓也沒賭一個。使俺每買東西,只拈塊兒。俺每但說:『娘拿等子你稱稱,俺每好使。』他便笑道:『拿去罷,稱甚麼?你不圖落,圖甚麼來?只要替我買值著。』這一家子,都那個不借他銀使?只有借出來,沒有個不進去的。還也罷,不還也罷。俺大娘和俺三娘使錢也好,只是五娘和二娘慳吝些。他當家,俺每就遭瘟來,會把腿磨細了!會勝買東西,也不與你個足數,綁著鬼一錢銀子,拿出來只稱九分半;著緊只九分,俺每莫不賠出來!」傅夥計道:「就是你大娘還好些。」玳安道:「雖做俺大娘好,毛司火性兒。一回家好,娘兒每親親噠噠說話兒,你只休惱狠著他,不論誰,他也罵你幾句兒。總不如六娘,萬人無怨。又常在爹根前替俺們說方便兒。誰問天來大事,受不的人央。俺們央他央兒對爹說,無有個不依。只是五娘快戳無路兒,行動就說:『你看我對你爹說。』把這『打』只題在口裡。如今春梅姐又是個和氣星,天生的都出在他一屋裡!」傅夥計道:「你五娘來這裡也好幾年了?」玳安道:「你老人家是知道他,想的起那咱來哩?他一個親娘也不認的,來一遭要便像的哭了家去。如今六娘死了,這前邊又是他的世界。那個管打掃花園,又說地不乾淨,一清早辰吃他罵的狗血噴了頭。」兩個說了一回,那傅夥計在枕上齁齁就睡著了。玳安亦有酒了,合上眼,不知天高地下,直至紅日三竿,都還未起來。原來西門慶每常在前邊靈前睡,早辰玉蕭出來收疊牀鋪,西門慶便往後邊梳頭去。書童蓬著頭要便和他兩個在前邊打牙犯嘴,互相嘲鬬,半日才進後邊去。不想今日西門慶歸後邊上歇去,這玉簫趕人沒起來,暗暗走出來與書童遞了眼色,兩個走在花園書房裡幹營生去了。不料潘金蓮起的早,驀地走到廳上,只見靈前燈兒也沒了,大棚裡丟的卓椅橫三豎四,沒一個兒。只見畫童兒正在那裡掃地。金蓮道:「賊囚根,乾淨只你這裡掃地,都往那裡去了?」畫童道:「他每都還沒起來哩。」金蓮道:「你且丟下苕葦,到前邊對你姐夫說,有白絹拿一疋來,你潘姥姥還少一條孝裙子。再拿一副頭鬚繫腰來與他,他今日家去。」畫童道:「怕不俺姐夫還睡哩,等我問他去。」良久回來道:「姐夫說不是他的首尾,書童哥與崔大哥管孝帳,娘問書童哥要就是了。」金蓮道:「知道那奴才往那去了?你去尋他來。」畫童向廂房裡瞧了瞧,說亮:「纔在這裡來,敢往花園書房裡梳頭去了。」金蓮道:「你自在這裡掃完了地,等我自家問這囚根子要去。」于是輕移蓮步,款蹙湘裙,走到花園書房內。偶然聽見裡面有人笑聲,推開門,只見他和玉簫在床上正幹得好哩。便罵道:「好囚根子,你兩個在此幹得好事!」諕得兩個做手腳不迭,齊跪在地下哀告。金蓮道:「賊囚根子,你且拿一疋孝絹,一疋布來,打發你潘姥姥家去。」那書童連忙拿來遞上。金蓮逕歸房來。那玉簫跟到房中打旋磨兒,跪在地下央及:「五娘,千萬休對爹說。」金蓮便問:「賊狗囚,你和我實說,這奴才從前已往偷了幾遭?一字兒休瞞,我便罷。」那玉簫便把和他偷的緣由說了一遍。金蓮道:「既要我饒恕你,你要依我三件事。」玉簫道:「娘饒了我,隨問幾件事我也依娘。」金蓮道:「一件,你娘房裡但凡大小事兒,就來告我說。你不說,我打聽出,定不饒你。第二件,我但問你要甚麼,你就稍出來與我。第三件,你娘向來沒有身孕,如今他怎生便有了?」玉簫道:「不瞞五娘說,俺娘如此這般,吃了薛姑子的衣胞符藥,便有了。」這潘金蓮一一聽記在心,纔不對西門慶說了。那書童見潘金蓮冷笑,領進玉簫去了。知此事有幾分不諧,向書房廚櫃內收拾了許多手帕汗巾,挑牙簪紐,并收的人情,他自己也儹勾十來兩銀子,又到前邊櫃上誆了傅夥計二十兩,只說要買孝絹,逕出城外,顧了長行頭口,到馬頭上,搭在鄉里船上,往蘇州原籍家去了。正是:

「撞碎玉籠飛彩鳳,  頓開金鎖走蛟龍。」

不想那日李桂姐、吳銀兒、鄭愛月都家去了。薛內相、劉內相早辰差了人,擡三牲卓面來,祭奠燒紙;又每人送了一兩銀子,伴宿分資,叫了兩個唱道情的來,白日裡要和西門慶坐坐。緊等著要打發他孝絹,尋書童兒要鑰匙,一地裡尋不著。傅夥計道:「他早辰問我要了櫃上二十兩銀子買孝絹去了。口稱爹分付他孝絹不勾,敢是向門外買去哩?」西門慶道:「我並沒分付他,如何問你要銀子?」一面使人往門外絹鋪找尋他,那裡得來?月娘便向西門慶說:「我猜這奴才有些蹺蹊,不知弄下甚麼硶兒,拐了幾兩銀子走了。你那書房子裡開了門,還大瞧瞧,沒腳蟹的營生,只怕還拿甚麼去了。」西門慶走到兩個書房裡都瞧了,見庫房裡鑰匙掛在牆上,大櫥櫃裡不見了許多汗巾手帕,并書禮銀子,挑牙紐扣之類。西門慶心中大怒,叫該地方的管役來,分付:「各處三瓦兩巷,與我訪緝。」那裡得來,正是:

「不獨懷家歸興急,  五湖烟水正茫茫。」

那時薛內相從响午時就坐轎來了,西門慶請下吳大舅、應伯爵、溫秀才相陪,先到靈前上香打了個問訊,然後與西門慶敘禮,說道:「可傷,可傷,如夫人是甚麼病兒歿了?」西門慶道:「不幸患崩潟之疾,看治不好歿了。又多謝老公公費心。」薛內相道:「沒多兒,將就表意罷了。」因看見掛著影,說道:「好個標致娘子,正好青春享福,只是去世太早些!」溫秀才在傍道:「物之不齊,物之情也。窮通壽夭,自有個定數,雖聖人亦不能強。」薛內相扭回頭來,見溫秀才衣巾穿著素服,說道:「此位老先兒是那學裡的?」溫秀才躬身道:「學生不才,備名府庠。」薛內相道:「我瞧瞧娘子的棺木兒。」西門慶即令左右把兩邊帳子撩起,薛內相進去觀看了一遍,極口稱贊道:「好付板兒,請問多少價買的?」西門慶道:「也是舍親的一付板,學生回了他的來了。」應伯爵道:「請老公公試估估,那裡地道?甚麼名色?」薛內相仔細看了此板:「不是建昌,是付鎮遠。」伯爵道:「就是鎮遠,也值不多。」薛內相道:「最高者必定是楊宣榆。」伯爵道:「楊宣榆單薄短小,怎麼看的過此板?還在楊宣榆之上,名喚做桃花洞,在於湖廣武陵川中。昔日唐漁父入此河洞中,曾見秦時毛女在此避兵,是個人跡罕到之處。此板七尺多長,四寸厚,二尺五寬,還看一半親家分上,要了三百七十兩銀子哩。公公你不曾看見,解開噴鼻香的,裡外俱有花色。」薛內相道:「是娘子這等大福,纔享用了這板。俺每內官家到明日死了,還沒有這等發送哩。」吳大舅道:「老公公好說,與朝廷有分的人,享大爵祿。俺每外官,焉能趕的上?老公公日近清光,代萬歲傳宣金口,見今童老爺加封王爵,子孫皆服蟒腰玉,何所不至哉!」薛內相便道:「此位會說話的兄,請問上姓?」西門慶道:「此是妻兄吳大哥,見居本衛千戶之職。」薛內相道:「就是此娘子的令兄麼?」西門慶道:「不是,乃賤荊之兄。」薛內相復於吳大舅聲諾,說道:「吳大人,失賸。」看了一回,西門慶讓至捲棚內,正面安放一把校椅,薛內相坐下,打茶的拿上茶來吃了。薛內相道:「劉公公怎的這咱還不到?叫我答應的迎迎去。」青衣人跪下稟道:「公公起身時,差小的邀劉公公去。劉公公轎已伺候下了,便來也。」薛內相又問道:「那兩個唱道的來了不曾?」西門慶道:「早上就來了。叫上來。」不一時走來面前磕頭,薛內相道:「你每吃了飯不曾?」那人道:「小的每得了飯了。」薛內相道:「既吃了飯,你每今日用心答應,我重賞你。」西門慶道:「老公公,學生這裡還預備了一起戲子,唱與老公公聽。」薛內相問:「是那裡戲子?」西門慶道:「是一班海鹽戲子。」薛內相道:「那蠻聲哈刺,誰曉的他唱的是甚麼!那酸子每在寒窗之下,三年受苦,九載遨遊,背著個琴劍書箱,來京應舉。怎得了個官,又無妻小在身邊;便希罕他這樣人?你我一個光身漢老內相,要他做甚麼?」溫秀才在傍笑說道:「老公公說話太不近情了。居之齊則齊聲,居之楚則楚聲。老公公處於高堂廣廈,豈無一動其心哉?」這薛內相便拍手笑將起來道:「我就忘了溫先兒在這裡,你每外官原來只護外官。」溫秀才道:「雖是士大夫,也只是秀才做的。老公公砍了一枝損百鄰,兔死狐悲,物傷其類。」薛內相道:「不然,一方之地,有賢有愚。」正說著,忽左右來報劉公公下轎了。吳大舅等出去迎接進來,向靈前作了揖。敘禮已畢,薛內相道:「劉公公你怎的這咱纔來?」劉內相道:「此邊徐同家來拜望,陪他坐了一回,打發去了。」一面分席坐下,左右遞上茶去。因問答應的:「祭奠卓面兒,都擺上了?」下邊人說:「都排停當了。」劉內相道:「咱每去燒了紙罷。」西門慶道:「老公公不消多禮,頭里已是見過禮了。」劉內相道:「此來為何?還當親祭祭。」當下左右接過香來,兩個內相上了香,遞了三鍾酒,拜下去。西門慶道:「老公公請起。」于是拜了兩拜起來。西門慶還了禮,復至捲棚內坐下。然後收拾安席,遞酒上坐。兩位內相分左右坐了。吳大舅、溫秀才、應伯爵從次,西門慶下邊相陪。子弟鼓板响動,遞上關目揭帖。兩位內相看了一回,揀了一段劉智遠紅袍記。唱了還未幾拍摺,心下不耐煩。一面叫上唱道情去,唱個道情兒耍耍到好。于是打起漁鼓,兩個並肩朝上,高聲唱了一套韓文公雪擁藍關故事下去。只見廚役上來磕頭,兩位內相都有賞賜。西門慶預備酒肉,賞賜跟隨人等,不用細說。薛內相便與劉內相兩個席上說說話兒道:「劉哥,你不知道,昨日這八月初十日,下大雨如注,雷電把內裡凝神殿上鴟尾裘碎了,諕死了許多宮人。朝廷大懼,命各官修省,逐日在上清宮宣精靈疏建醮,禁屠十日,法司停刑,百官不許奏事。昨日大金遣使臣進表,要割內地三鎮。依著蔡京老賊就要許他,掣童掌事的兵馬,交都卸史譚積、黃安十大使,節制三邊兵馬,又不肯還,交多官計議。昨日立冬,萬歲出來祭大廟。太常寺一員博士,名喚方軫,早晨直著打掃,看見太廟磚縫出血,殿東北上地陷了一角,寫表奏知萬歲。科道官上本,極言童掌事大了,宦官不可封王。如今馬上差官,拿金牌去取童掌事回京。」劉內相道:「你我如今出來在外做土官,那朝裡事也不干咱每。俗語道,咱過了一日是一日,便塌了天,還有四個大漢。到明日大宋江山,管情被這些酸子弄壞了。王十九,咱每只吃酒!」因與唱道情的上來,分付:「你唱個李白好貪杯的故事。」那人立在席前,打動漁鼓,又唱了一回。直吃至日暮時分,分付下人,看轎起身。西門慶款留不住,送出大門,喝道而去。回來分付點起燭來,把卓席休動,教廚役上來攢整停留,留下吳大舅、應伯爵、溫秀才坐的。又使小而廝請傅夥計、甘夥計、韓道國、賁地傳、崔本和陳經濟復坐,叫上子弟來,分付:「還找著昨日玉環記上來。」因向伯爵道:「內相家不曉的南戲滋味,早知他不聽,我今日不留他。」伯爵道:「哥到辜負的意思。內臣斜局的營生,他只喜藍關記,搗喇小子山歌野調,那裡曉的大關目悲歡離合?」于是下邊打動鼓板,將昨日玉環記做不完的摺數,一一緊做慢唱,都搬演出來。西門慶令小廝席上頻斟美酒,伯爵與西門慶同卓而坐,便問:「他姐兒三個還沒家去,怎的不叫出來遞杯酒兒?」西門慶道:「你還想那一夢兒,他每去的不耐煩了。」伯爵道:「他每在這裡住了有兩三日。」西門慶道:「吳銀兒住的久了。」當日眾人坐到三更時分,搬戲已完,方起身各散。西門慶邀下吳大舅,明日早些來陪上祭官員。與了戲子四兩銀子,打發出門。到次日周守備、荊都監、張團鍊、夏提刑合衛許多官員,都合了分資,辦了一副猪羊吃卓祭奠,有禮生讀祝。西門慶預備酒席,李銘等三個小優兒伺候答應。到向午,只聽鼓响,祭禮到了。吳大舅、應伯爵、溫秀才在門首迎接。只見後擁前呼,眾官員下馬,在前廳換衣服。良久,把祭品擺下。眾官齊到靈前,西門慶與陳經濟伺候還禮。禮生喝禮,三獻畢,跪在傍邊讀祝:

「維政和七年,歲次丁酉,九月庚申朔,越二十五日甲申,寅侍生周秀、荊忠、夏延令、張關、文臣、范勳、吳鎧、徐鳳翔、潘磯等,謹以剛鬣柔毛庶羞之儀,致奠于。故錦衣西門孺人氏之靈曰:維靈秀毓閨閫,善淑女紅。金玉其德,蘭蕙其姿。相內政而有道,主中饋而無闋。重積學而和睦內眷,尊所天而舉案齊眉。人願耆艾,天晞絕奇。正宜同諧鸞琴,何乃嗇後而促其期。噫,修短有數也,天厭善類!珠沈璧碎,雲慘風悲。扣玄扃而莫啟,歎薤露而易晞!秀等忝居僚儕,情重交誼。崇餚於俎,酌酒於巵,庶乎來享,鑒此哀辭,嗚呼尚饗!」

祭畢,西門慶下來謝禮已畢。吳大舅等讓眾官至捲棚內,寬去素服侍茶。小優彈唱起來,安席上坐。手下跟隨之人,自有管待齊整。廚役上來,三道五割,酒餚比前兩日更豐盛。照席還磕了頭。西門慶與吳大舅、應伯爵、溫秀才下席相陪。觥籌交錯,慇懃勸酒。李銘等三個小優兒銀箏象板,朝上彈唱。外邊自有夥計主管,將跟隨祭來各項人役盒擔錢,都照例打發銀子停當。眾官坐到後晌時分,就要起身。西門慶不肯,與吳大舅伯爵等大杯款留。教李銘等彈樂器,唱小曲兒歡飲,直到日暮時分方散。西門慶還要留吳大舅眾人坐,吳大舅道:「各人連日打攪,姐夫也辛苦了。各自歇息去罷。」當時告辭回家。正是:

「天上碧桃和露種,  日邊紅杏倚雲栽,

家中巨富人趨附,  手內多時莫論財。」

畢竟不知後來如何,且聽下回分解:




\end{showcontents}


