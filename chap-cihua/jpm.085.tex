%# -*- coding: utf-8 -*-
%!TEX encoding = UTF-8 Unicode
%!TEX TS-program = xelatex
% vim:ts=4:sw=4
%
% 以上设定默认使用 XeLaTex 编译,并指定 Unicode 编码,供 TeXShop 自动识别

%第八十五回 
\chapter{月娘識破金蓮奸情\KG 薛嫂月夜賣春梅}


\begin{showcontents}{}



「人家養女甚無聊,  倒踏來家更不合,

口稱爹媽虛情意,  權當為兒假做作;

人戶只嫌恩愛少,  出門翻作怨仇多,

若有一些不到處,  一日一場罵老婆。」

話說吳大舅保月娘,有日取路來家不題。單表潘金蓮,自從月娘不在家,和陳經濟兩個,家前院後庭,如雞兒趕彈兒相似,纏做一處,無一日不會合。一日金蓮眉黛低垂,腰肢寬大,終日懨懨思睡,茶飯懶嚥。叫經濟到房中說:「奴有件事告你說,這兩日眼皮兒懶待開,腰肢兒漸漸大,肚腹中捘捘跳,茶飯兒怕待吃,身子好生沉困。有你爹在時,我求薛姑子苻藥衣胞,那等安胎,白沒見個踪影!今日他沒了,和你相交多少時兒,便有了孩子。我從三月內洗換身上,今方六個月,已有半肚身孕。往常時我排磕人,今日都輪到我頭上!你休推睡裡夢裡,趁你大娘還未來家,那裡討貼墜胎的藥,趁早打落了這胎氣離了身,奴走一步也伶俐。不然弄出個怪物來,我就尋了無常罷了!再休想擡頭見人!」經濟聽了便道:「咱家舖中,諸樣藥都有,倒不知那幾庄兒墜胎?又沒方修合。你放心,不打緊處,大街坊胡太醫,他大小方脉、婦人科都善治,常在咱家看病。等我問他那裡贖取兩貼與你吃,下胎便了。」婦人道:「好哥哥,你上緊快去,救奴之命。」這陳經濟包了三錢銀子,逕到胡太醫家叫問。胡太醫正在家,出來相見聲諾,認的經濟西門大官人女婿,讓坐說:「一向稀面,動問到舍,有何見教?」經濟道:「別無干凟。」向袖中取出白金三星:「充藥資之禮,敢求下良劑一二貼,足見盛情!」胡太醫說道:「我家醫道大方脉、婦人科、小兒科、內科、外科、加減十三方、壽域神方、海上方、諸般雜症方,無不通曉。又專治婦人胎前產後。且婦人以血為本,藏于肝,流于臟,上則為乳汁,下則為水月,合精而成胎氣。女子十四而天癸至,任脉通放,月候按時而行。常以三旬一見,則無病。一或血氣不調,則陰陽愆伏。過於陽則經水先期而來,過於陰,則經水後期而至。血性得熱而流,寒則凝滯。過與不及,皆致病也。冷則多白,熱則多赤。冷熱不調,則赤白帶。大抵血氣和平,陰陽調順,其精血聚而包胎成。心腎二脉,應手而動。精盛則為男,血勝則為女,此自然之理也。胎前必須以安胎為本,如無他疾,不可妄服藥餌。待十月分娩之時,尤當謹護。不然,恐生產後諸疾。慎之!慎之!」經濟笑道:「我不要安胎,我今只用墜胎藥。」胡太醫道:「天地之間,以好生為本。人家十個九個只要安胎的要,你何如倒要墜胎?沒有!沒有!」經濟見他掣肘,又添了二錢藥資,說:「你休管他,各家人自有用處。此婦子女生落不順,情願下胎。」這胡太醫接了銀子,說道:「不打緊,我與你一服紅花一埽光,吃下去,如人行五里,其胎自落矣。」有西江月為證:

「牛膝蟹瓜甘遂,定磁大戟芫花,斑毛赭石與碙砂,水銀與芒硝研化。又加桃任通草,麝香文帶凌花。更燕醋煮好紅花,管取孩兒落下。」

經濟於是討了兩貼紅花一埽光,作辭胡太醫,到家遞與婦人,一五一十說了。到晚夕,煎紅花湯吃下去,登時滿肚裡生疼。睡在炕上,教春梅按在身,只情揉揣。可霎作怪,須臾,坐淨桶,把孩子打下來了!只說身上來,令秋菊攪草紙,倒將東淨毛司裡。次日掏坑的漢子,挑出去一個白胖的小廝兒。常言:「好事不出門,惡事傳千里。」不消幾日,家中大小,都知金蓮養女婿,偷出私肚子來了。都說吳月娘有日來家,往回泰安州,去了半個月光景。來時正值十月天氣,家中大小接着,如天上落下來的一般。月娘到家中,先到天地佛前炷了香,然後西門慶靈前拜罷,告訴孟玉樓眾姊妹家中大小,把岱岳廟中及山寨上的,從頭告訴一遍,因大哭一場。合家大小都來參見了。月娘見奶子抱孝哥兒到根前,子母相會在一處,燒布、置酒,管待吳大舅回家。晚夕,眾姊妹與月娘接風,俱不在話下。到第二日,月娘路上風霜跋涉,着了辛苦、又乞了驚怕,身上疼痛沉困,整不好了兩三日。那秋菊在家,把金蓮、經濟兩人幹的勾當,聽的滿耳滿心。要走上房告月娘說,二人怎生偷出私肚子來,傾在毛司裡,乞掏坑的掏出去,何人不看見;又被婦人怎生打罵,含恨正沒發付處。走到上房門首,又被小玉噦罵在臉上,打耳刮子打在臉上,罵道:「賊說舌的奴才,趁早與我走!俺奶奶遠路來家,身子不快活,還未起來,趁早與我走。氣了他,倒值了多少的!」罵的秋菊忍氣吞聲,喏喏而退。一日,也是合有事。經濟進來尋衣裳,婦人又和他在翫花樓上兩個做得好。被秋菊走到後邊,叫了月娘來看,說道:「奴婢兩番三次告大娘說,不信。娘不在,兩個在家明睡到夜,夜到明,偷出私肚子來,與春梅兩個都打成一家。今日兩人,又在樓上幹歹事!不是奴婢說謊,娘快些瞧去!」月娘急忙走到前邊,兩個正幹的好,還未下樓。不想金蓮房簷籠內,馴養得個鸚哥兒會說嘴,高聲叫:「大娘來了!」春梅正在房中,聽見迎出來。見是月娘,比及樓上叫婦人,先是經濟拿衣服下樓往外走,被月娘喝罵了幾句,說:「小孩兒沒記性!有要沒緊,進來撞甚麼?」經濟道:「舖子內人等着,沒人尋衣裳。」月娘道:「我那等分付,教小廝進來取。如何又進來寡婦房裡,有要沒緊做甚麼?沒廉恥!」幾句罵得經濟往外金命水命,走投無命。婦人羞的半日不敢下來。然後下來,被月娘儘力數說了一頓,說道:「六姐,今後再休這般沒廉恥!你我如今是寡婦,比不的有漢子,香噴噴在家裡,臭烘烘在外頭。盆兒罐兒,都有耳躲,你有要沒緊,和這小廝纏甚麼?教奴才們背地排說的硶死了!常言道:『男兒沒性,寸鐵無鋼,女人無性,爛如麻糖。』,『其身正,不令而行;其身不正,雖令不行。』你有長俊正條,肯教奴才排說你?在我跟前說了幾遍,我不信。今日親眼看見,說不的了!我今日說過,要你自家立志,替漢子爭氣。像我進香去,兩番三次被強人擄掠逼勒;若是不正氣的,也來不到家了。」金蓮吃月娘數說,羞的臉上紅一塊白一塊,口裡說一千個沒有。只說:「我在樓上燒香,陳姐夫自去那邊尋衣裳,誰和他說甚話來?」當下月娘亂了一回,歸後邊去了。晚夕西門大姐在房內,又罵經濟:「賊囚根子,敢說又沒真賍實犯拿住你?你還那等嘴巴巴的!今日兩個又在樓上做甚麼?說不的了!兩個弄的好硶兒,只把我合在缸底下一般!那淫婦要了我漢子,還在我根前拿話兒栓縛人。毛司裡磚兒,又臭又硬!恰似強伏着那個一般,他便羊角蔥靠南墻,老辣已定!你還在這屋裡雌飯吃?」經濟罵道:「淫婦,你家收着我銀子,我雌你家飯吃?」使性往前邊來了。自此以後,經濟只在前邊,無事不敢進入後邊來。取東取西,只是玳安、平安兩個往樓上取去。每日飯食,晌午還不拿出來,把傅夥計餓的只拿錢街上盪麵面吃。正是:

「龍鬬虎爭,  苦了小獐。」

各處門戶,日頭半天老早關了。由是,與金蓮兩個恩情又間隔阻了。經濟那邊陳宅房子,一向教他母舅張團練看守居住。張團練革任,在家間住。經濟早晚往那裡吃飯去。月娘亦不追問。兩個隔別約一月不得見面。婦人獨在那邊,挨一日,似三秋,過一宵,如半夏。怎禁這空房寂靜,慾火如蒸?要見他一面,難上之難。兩下音信不通,這經濟無門可入,忽一日,見薛嫂兒打門首經過。有心要托他寄一紙柬兒,到那邊與金蓮,訴其間阻之事,表此肺腑之情。一日,推門外討帳,騎頭口逕到薛嫂家,栓了騾子,掀簾便問:「薛媽在家?」有他兒子薛紀、媳婦兒金大姐,抱孩子在炕上,伴着人家賣的兩個使女。聽見有人叫薛媽,出來問:「是誰?」經濟道:「是我問薛媽在家不在?」金大姐道:「姑夫請家來坐,俺媽往人家兌了頭面,討銀子去了。有甚話說?使人叫去。」連忙點茶與經濟吃。少坐片時,只見薛嫂兒來了。同經濟道了萬福,說:「姑夫那陣風兒吹來我家?」叫金大姐:「倒茶與姑夫吃。」金大姐道:「剛纔吃了茶了。」經濟道:「無事不來。如此這般,與我五娘勾搭日久,今被秋菊丫頭戮舌,把俺兩個姻緣拆散。大娘與大姐甚是疏淡我。我與六姐拆散不開,二人離別日久,音信不通,欲稍寄數字進去與他。無人得到內裡,須央及你,如此這般,通個消息。」向袖中取出一兩銀子來:「這些微禮,權與薛媽買茶吃。」那薛嫂一聞其言,拍手打掌笑起來,說道:「誰家女婿戲丈母?世間那裡有此事!姑夫你實對我說,端的你甚麼得手來?」經濟道:「薛媽禁聲,且休取笑!我有這柬帖封好在此,好歹明日替我送與他去。」薛嫂一手接了,說:「你大娘從進香回來,我還沒看他去。兩當一節,我去走走。」經濟道:「我在那裡討你信?」薛嫂道:「往舖子裡尋你回話。」說畢,經濟騎頭口來家。次日,都說薛嫂提着花箱兒,先進西門慶家上房看月娘。坐了一回,又到孟玉樓房中。然後纔到金蓮這邊。金蓮正放卓兒吃粥。春梅見婦人悶悶不樂,說道:「娘,你老人家也少要憂心。何仙姑人人說:『日日有丈夫,是非來入耳。不聽自然無。』古昔仙人,還有小人不足之處,休說你我。如今爹也沒了,大娘他養出個墓生兒子來,莫不也來路不明?他也難管我你暗地的事。你把心放開,料天塌了,還有撐天大漢哩!人生在世,且風流了一日是一日!」于是篩上酒來,遞一鍾與婦人,說:「娘,且吃一杯兒暖酒,解解愁悶。」因見階下兩雙犬兒交戀在一處,說道:「畜生尚有如此之樂,何況人而反不如此乎?」正飲酒,只見薛嫂來到,向前道了萬福,笑道:「你娘兒兩個好受用。」因觀二犬戀在一處,笑道:「你家好祥瑞!你娘兒們看看,怎不解許多悶?」于是又個萬福。婦人道:「那陣風兒今日刮你來?怎的一向不來走走?」一面讓薛嫂坐。薛嫂兒道:「我鎮日不知幹的甚麼,只是不得閒。大娘頂上進了香,遲看着他,剛纔好不怪我。西房三娘也在根前,留了我兩對翠花,一對大翠圍髮,好快性,就秤了八錢銀子與我。只是後邊住的雪娘,從八月裡要了我二對線花兒,該二錢銀子來,一些沒有支用着,白不與我,好慳吝的人!我對你說,怎的不見你老人家?」婦人道:「我這兩日,身子有些不快,不曾出去走動。」春梅一面篩了一鍾酒,遞與薛嫂兒。薛嫂連忙道萬福,說:「我進門就吃酒。」婦人道:「你到明日,養個好娃娃。」薛嫂兒道:「我養不的,俺家兒子媳婦兒金大姐,到新添了個娃兒,纔兩個月來。」又道:「你老人家沒了爹,終久這般冷清清了。」婦人道:「說不得,有他在好了!如今弄得俺娘兒們,一折一磨的!不瞞老薛說,如今俺家中人多舌頭多。他大娘自從有了這孩兒,把心腸兒也改變了,姊妹不似那咱親熱了。這兩日,一來我心裡不自在,二來因些閒話,沒曾往那邊去。」春梅道:「都是俺房裡秋菊這奴才,大娘不在,霹空架了俺娘一篇是非,把我也扯在裡面,好不亂哩!」薛嫂道:「就是房裡使的那大姐?他怎的倒弄主子?穿青衣,抱黑柱,這個使不的!」婦人使春梅:「你瞧瞧那奴才,只怕他來覷聽。」春梅道:「他在廚房下揀米哩,這破包簍奴才,在這屋,就是走水的槽,單管屋裡事兒,往外學舌。」薛嫂道:「這裡沒人,咱娘兒們說話。直道昨日,陳姐夫到我那裡,如此這般告訴我,乾淨是他戮犯你們的事兒了。陳姐夫說,他大娘數說了他,各處門戶都緊了。不許他進來取衣裳、拿藥材,又把大姐搬進東廂房裡住。每日晌午還不拿飯出去與他吃,餓的他只往他母舅張老爹那裡吃去。一個親女婿不托他,到托小廝,有這個道理?他有好一向沒得見你老人家,巴巴央及我稍了個柬兒,多多拜上你老人家,少要焦心,左又爹也是沒了,爽利放倒身大做一做,怕怎的?點根香怕出烟兒,放把火倒也罷了!」于是取出經濟封的柬帖兒遞與婦人。拆開觀看,別無甚話,上寫紅綉鞋一詞:

「祆廟火,燒皮肉;藍橋水,渰過咽喉。緊按納,風聲滿南州。畢了終是染污,成就了倒是風流。不甚麼,也是有!

六姐妝次

〈下書〉經濟百拜上。」

婦人看畢,收了入袖中。薛嫂兒道:「他教你回個記色,與他寫幾個字兒稍了去,方信我送的有個下落。」婦人教春梅陪著薛嫂吃酒,他進入房。半晌,拿了一方白綾帕,一個金戒子兒,帕兒上也寫着一詞在上,說道:

「我為你耽驚受怕,我為你折挫渾家,我為你脂粉不曾搽,我為你在人前拋了些見識,我為你奴婢上使了些鍬筏;咱兩個一雙憔悴殺!」

婦人寫了,封得停當,交與薛嫂,便說:「你上覆他,教他休要使性兒往他母舅張家那裡吃飯,惹他張舅唇齒。說你在丈人家做買賣,都來我家吃飯,顯得俺們都是沒處活的一般,教他張舅怪或是未有飯吃,教他舖戶裡拿錢,買些點心和夥計吃便了。你使性兒不進來,和誰賭鱉氣哩?都是賊人胆兒虛一般!」薛嫂道:「等我對他說。」婦人又與薛嫂五錢銀子,作別出門。來到前邊舖子裡,尋見經濟,兩個走到僻靜處說話。把封的事物遞與他:「五娘說,教他休使性兒賭鱉氣,教他常進來走走;休往你張舅家吃飯去,惹人家怪。」因拿出五錢銀子與他瞧:「此是裡面與我的。漏眼不藏絲,久後你兩個愁不會在一答裡?對出來,我臉放在那裡?」經濟道:「老薛,多有累你!」深深與他唱喏。那薛嫂走了兩步,又回來,說:「我險些忘了一件事,剛纔我出來,大娘又使丫頭綉春叫進我去,叫我晚上來領春梅,要打發賣他。說他與你們做牽頭,和他娘通同養漢。敢就因這件事?」經濟道:「薛媽,你只個領在家,我改日到你家見他一面,有話問他。」那薛嫂說畢,回家去了。果然到晚夕月上的時分走到,領春梅到月娘房中,月娘開口說:「那咱原是你手裡十六兩銀子買的,你如今拿十六兩銀子來就是了。」分付小玉:「你看着到前邊收拾了,教他罄身兒出去,休要他帶出衣裳去了。」那薛嫂兒到前邊,向婦人如此這般:「他大娘叫我領春梅姐來了。對我說,他與你老人家通同作弊偷養漢子,不管長短,只問我要原價。」婦人聽見說領賣春梅,就睜了眼,半日說不出話來。不覺滿眼落淚,叫道:「薛嫂兒,你看我娘兒兩個沒漢子的好苦也!今日他死了多少時兒,就打發我身邊人!他大娘這般沒人心仁義,自恃他身邊養了個尿胞種,就放人躧到泥裡!李瓶兒孩子週半還死了哩,花巴痘疹未出,赤道天怎麼算計,就心高遮了太陽!」薛嫂道:「孩兒出了痘疹了沒曾?」婦人道:「何曾出來了?還不到一週兒哩!」薛嫂道:「春梅姐,說爹在日,曾收用過他。」婦人道:「收用過二字兒!死鬼把他當心肝肺腸兒一般看待,說一句聽十句,要一奉十。正經成房立紀老婆,且打靠後!他要打那個小廝十棍兒,他爹不敢打五棍兒。」薛嫂道:「可又來大娘差了!爹收用的恁個出色姐兒,打發他,箱籠兒也不與;又不許帶一件衣服兒。只教他罄身兒出去,鄰舍也不好看的!」婦人道:「他對你說,休教帶出衣裳去?」薛嫂道:「大娘分付小玉姐便來,教他看着休教帶衣裳出去。」那春梅在傍邊見打發他,一點眼淚他沒有。見婦人哭,說道:「娘,你哭怎的?奴去了你耐心兒過,休要思慮壞了。你思慮出病來,沒人知你疼熱的。等奴出去,不與衣裳也罷。自古好男不吃分時飯,好女不穿嫁時衣!」正說着,只見小玉進來,說道:「五娘你信我奶奶,倒三顛四的!小大姐扶持你老人家一場,瞞上不瞞下,你老人家拿出他箱子來,揀上色的包與他兩套,教薛嫂兒替他拿了去,做個一念兒,也是他番身一場。」婦人道:「好姐姐,你到有點仁義!」小玉道:「你看誰人保得常無事?蝦墓、促織兒,都是一鍬土上人!兔死狐悲,物傷其類!」一面拿出春梅箱子來,是戴的汗巾兒、翠簪兒,都教他拿去。婦人揀了兩套上色羅段衣服鞋腳,包了一大包;婦人梯已與了他幾件釵梳簪墜戒子,小玉也頭上拔下兩根簪子來,遞與春梅。餘者珠子纓絡、銀絲雲髻、遍地金粧花裙襖,一件兒沒動,都擡到後邊去了。春梅當下拜辭婦人、小玉,灑淚而別。臨出門,婦人還要他拜辭拜辭月娘眾人。只見小玉搖手兒。這春梅跟定薛嫂,頭也不回,揚長決裂,出大門去了。小玉和婦人送出大門回來。小玉到上房回大娘,只說罄身子去了,衣服都留下沒與他。這金蓮歸進房中,往常有春梅娘兒兩個相親相熱,說知心話兒。今日他去了,丟得屋裡冷冷落落,甚是孤恓,不覺放聲大哭。有詩為證:

「耳畔言猶在,  于今恩愛分;

房中人不見,  無語自消魂。」

畢竟未知後來如何,且聽下回分解:





\end{showcontents}


