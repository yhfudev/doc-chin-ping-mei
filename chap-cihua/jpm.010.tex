%# -*- coding: utf-8 -*-
%!TEX encoding = UTF-8 Unicode
%!TEX TS-program = xelatex
% vim:ts=4:sw=4
%
% 以上设定默认使用 XeLaTex 编译,并指定 Unicode 编码,供 TeXShop 自动识别

%第十回 
\chapter{武松充配孟州道\KG 妻妾翫賞芙蓉亭}

\begin{showcontents}{}



「朝看瑜伽經,  暮誦消災咒,

種瓜須得瓜,  種荳須得荳;

經咒本無心,  冤結如何究,

地獄與天堂,  作者還自受。」

話說被地方保甲,拏去縣裡見知縣去了。且表西門慶跳下樓窗,順著房山,扒伏在人家院裡藏了,原來是行醫的胡老人家。只見他家使的一個大胖丫頭走來毛廁裡淨手,蹶著大屁股,猛可見了一個漢子扒伏在院墻下,往前走不迭,大叫:「有賊了!」慌得胡老人急進來看。見認的是西門慶,便道:「大官人,且喜武二尋你不著,把那人打死了,地方拏去縣中見官去了,多已定死罪。大官人歸家去,無事。」這西門慶拜謝了胡老人,搖擺著來家,一五一十,對潘金蓮說。二人拍手喜笑,以為除了患害。婦人叫西門慶:「上下多使些錢,務要結果了他,休要放他出來。」西門慶一面差心腹家人來旺兒,饋送了知縣一副金銀酒器,五十兩雪花銀。上下吏典,也使了許多錢,只要休輕勘了武二。知縣受了西門慶賄賂,到次日早衙陞廳,地方保甲押著武二,并酒保、唱的干證人,在廳前跪下。縣主一夜把臉番了,便叫:「武二,你這廝昨日虛告,如何不遵法度!今又平白打死了人,有何說理?」武二磕頭,告道:「望相公與小人作主。小人本與西門慶執仇廝打,不料撞遇了此人在酒樓上,問道:『西門慶那裡去了?』他不說。小人一時怒起,誤打死了他。」知縣道:「這廝何說,你豈不認的他是縣中皂隸?想必別有緣故!你不實說。」喝令左右:「與我加起刑來!人是苦蟲,不打不成!」兩邊閃出三四個皂隸役卒,抱許多刑具,把武松托翻,雨點般篦板子打將下來。須臾,打了二十板,打得武二口口聲聲叫冤,說道:「小人平日也與相公用力效勞之處,相公豈不憫念?相公休要苦刑小人。」知縣聽了此言,越發惱了:「你這廝親手打死了人,尚還口強抵賴那個!」喝令:「與我好生拶起來!」當下拶了武松一拶,敲了五十杖子。教取面長枷帶子,收在監內,一干人寄監在門房裡。內中縣丞佐貳官,也有和武二好的,念他是個義烈漢子,有心要周旋他;爭奈多受了西門慶賄賂,粘住了口,做不的張主。又見武松只是聲冤,延挨了幾日,只得朦朧取了供招,喚當該吏典,并忤作保甲鄰人等,押到獅子街,檢驗李外傳身屍,填寫屍單格目。委的被武松尋問他,索討分錢不均,酒醉怒起,一時鬬毆拳打腳踢,撞跌身死。左肋、面門、心坎、腎囊,俱有青赤傷痕不等。檢驗明白,回到縣中。一日做了文書申詳,解送東平府來,詳允發落。這東平府府尹,姓陳雙名文昭,乃河南人氏,極是個清廉的官。聽的報來,隨即陞廳。那官人但見:

「天生正直,稟性賢明;幼年向雪案攻書,長大在金鑾對策。常懷忠孝之心,每行仁慈之念。戶口增,錢糧辦,黎民稱頌滿街衢;詞訟減,盜賊休,父老讚歌喧市井。攀轅截鐙,名標書史播千年;勒石鐫碑,聲振黃堂傳萬古。正直清廉民父母,賢良方正號青天。」

這府尹陳文昭已知這事了。便叫押過這一干犯人,就當廳先把清河縣申文看了,又把各人供狀招擬看過,端的上面怎生寫着?文曰:

「東平府清河縣為人命事,呈稱:犯人武松,年二十八歲,係陽谷縣人氏。因有膂力,本縣參做都頭。因公差回還,祭奠亡兄,見嫂潘氏守孝不滿,擅自嫁人。是松在巷口打聽,不合與獅子街王鑾酒樓上,撞遇先不知名,今知名李外傳,因酒醉索討前借錢三百文,外傳不與又不合,因而鬬毆,互相不伏,揪打踢撞,傷重當時身死。比有娼婦牛氏、包氏見證。致被地方保甲捉獲,委官前至屍所,拘集使忤甲鄰人等,檢驗明白,取供具結,填圖解繳,前來覆審,無異同。擬武松合依鬬毆殺人,不問手足他物金刃,律絞。酒保王鸞,并牛氏、包氏,俱供明無罪。今合行申到案發落,請允施行。政和三年八月八日知縣李達夫,縣丞樂和安,主簿華何祿,典史夏恭基,司吏錢勞。」

府尹看了一遍,將武松叫過面前跪下,問道:「你如何打死這李外傳?」那武松只是朝上磕頭,告道:「青天老爺,小的到案下,得見天日!容小的說,小的敢說。」府尹道:「你只顧說來!」武松道:「小的本為哥哥報仇,因尋西門慶,誤打死此人。」把前情訴告了一遍。「委是小的負屈啣冤。西門慶錢大,禁他不得!但只是個小人哥哥武大,含冤地下,枉了性命!」府尹道:「你不消多言,我已盡知了。」因把司吏錢勞叫來,痛責二十板,說道:「你那知縣,也不待做官,何故這等任情賣法?」于是將一干人眾,一一審錄過,用筆將武松供昭都改了。因向佐貳官說道:「此人為兄報仇,誤打死這李外傳,也是個有義的烈漢,比故殺平人不同。」一面打開他長枷,換了一面輕罪枷枷了,下在牢裡,一干人等,都發回本縣聽候。一面行文書着落清河縣,添提豪惡西門慶,并嫂潘氏,王婆、小廝鄆哥,仵作何九,一同從公,根勘明白,奏請施行。武松在東平府監中,人都知道他是屈官司;因此押牢禁子都不要他一文錢,到把酒食與他吃。早有人把這件事報到清河縣,西門慶知到了,慌了手腳。陳文昭是個清廉官,不敢來打點他;走去央求浼親家陳宅心腹,并家人來保星夜來往東京,下書與楊提督。提督轉央內閣蔡大師,大師又恐怕傷了李知縣名節,連忙賷了一封緊要密書帖兒,特來東平府下書與陳文昭,免提西門慶、潘氏。這陳文昭原係大理寺寺正,陞東平府府尹,又係蔡太師門生,又見楊提督乃是朝廷面前說得話的官,以此人情兩盡了。只把武松免死,問了個脊杖四十,刺配二千里充軍。況武大已死,屍傷無存,事涉疑似,勿論。其餘一干人犯,釋放寧家。申詳過省院,文書到日,即便施行。陳文昭從牢中取出武松來,當堂讀了朗廷明降,開了長枷,免不得脊杖四十,取一具七斤半鐵葉團頭枷釘了。臉上刺了兩行金字,迭配孟州牢城,其余發落已完。當堂府尹押行公文,差兩個防送公人,領了武松解赴孟州交割。當日武松與兩個公人,出離東平府,來到本縣家中,將家活多辦買了,打發那兩個公人路上盤費。安撫左鄰姚二郎看管迎兒:「倘遇朝廷恩典,赦放還家,恩有重報,不敢有忘。」那街坊鄰舍,上戶人家,見武二是個有義的漢子,不幸遭此刑。平昔與武二好的,都資助他銀兩,也有送酒食錢米的。武二到下處,問士兵要出行李包裹來,即日離了清河縣上路,迤〈辶里〉往孟州大道而行,正遇着中秋天氣。此這一去,正是:

「若得苟全痴性命,  也甘飢餓過平生。」

有詩為證:

「府尹推詳秉至公,  武松垂死又疏通;

今朝刺配牢城去,  病草萋萋遇煖風。」

這裡武二往孟州充配去了不題。且說西門慶打聽他上路去了,一塊石頭方落地,心中如去了痞一般,十分自在。于是家中分付家人來旺、來保、來興兒,收拾打掃後花園芙蓉亭乾淨,舖設圍屏,懸起金障,安排酒席齊整,叫了一起樂人,吹彈歌舞,請大娘子吳月娘,第二李嬌兒,第三孟玉樓,第四孫雪娥,第五潘金蓮,合家歡喜飲酒。家人媳婦,丫鬟使女,兩邊侍奉。怎見當日好筵蓆?但見:

「香焚寶鼎,花插金瓶;器列象州之古玩,廉開合浦之明珠。水晶盤內,高堆火棗交梨;碧玉盃中,滿泛瓊槳玉液。烹龍肝,炮鳳腑 ,果然下筯了萬錢;黑熊掌 ,紫駝蹄 ,酒後獻來香滿座。更有那軟炊紅蓮香稻 ,細膾通印子魚 。伊魴洛鯉 ,誠然貴似牛羊;龍眼荔枝。信是東南佳味。碾破鳳團,白玉甌中分白浪;斟來瓊液,紫金壺內噴清香。畢竟壓賽孟嘗君,只此敢欺石崇富。」

當下西門慶與吳月娘居上,其餘李嬌兒、孟玉樓、孫雪娥、潘金蓮多兩傍列坐。傳盃弄盞,花簇錦攢飲酒。只見小廝玳安領下一個小廝、一個小女兒,纔頭髮齊眉兒,生的乖覺,拏着兩個盒兒,說道:「隔壁花太監家的,送花兒來與娘們戴。」走到西門慶、月娘眾人跟前,都磕了頭,立在傍邊,說:「俺娘使我送這盒兒點心,并花兒,與西門大娘戴。」揭開簾子看盒兒,一盒是朝廷上用的果餡椒鹽金餅 ,一盒是新摘下來鮮玉簪花兒。月娘滿心歡喜,說道:「又叫你娘費心!」一面看菜兒,打發兩個吃了點心。月娘與了那小丫頭一方汗巾兒,與了小廝一百文錢,說道:

「多上覆你娘,多謝了。」因問小丫頭兒:「你叫什麼名字?」他回言道:「我叫綉春,小廝叫做天福兒。」打發去了。月娘便向西門慶道:「咱這裡間壁住的花家,這娘子兒到且是好;常時使過小廝丫頭,送東西與我,我並不曾回些禮兒與他。」西門慶道:「花二哥他娶了這娘子兒,今不上二年光景;他自說娘子好個性兒。不然,房裡怎生得這兩個好丫頭?」月娘道:「前者六月間,他家老公公死了。出殯時,我在山頭,會他一面;生的五短身材,團面皮,細彎彎兩道眉兒,且自白淨,好個溫克性兒!年紀還小哩,不上二十四五。」西門慶道:「你不知,他原是大名府梁中書妾,晚嫁花家子虛,帶了一分好錢來。」月娘道:「他送盒來親近你我,又在個緊鄰,咱休差了禮數,到明日也送些禮物回答他。」看官聽說:原來花子虛渾家娘家姓李,因正月十五日所生,那日人家送了一對魚瓶兒來,就小字喚做瓶姐。先與大名府梁中書家為妾。梁中書乃東京蔡太史女婿,夫人性甚嫉妒,婢妾打死者,多埋在後花園中。這李氏只在外邊書房內住,有養娘扶侍。只因政和三年正月上元之夜,梁中書同夫人在翠雲樓上,李逵殺了全家大小,梁中書與夫人各自逃生。這李氏帶了一百顆西洋大珠,二兩重一對鴉青寶石,與養娘媽媽走上東京投親。那時花太監由御前班直,陞廣南鎮守。因姪男花子虛沒妻室,就使媒人說親,娶為正室。太監在廣南去,也帶他到廣南住了半年有餘。不幸花太監有病,告老在家,因見清河縣人,在本縣住了。如今花太監死了,一分錢多在子虛手裡,每日同朋友在院中行走,與西門慶都是會中朋友;西門慶是個大哥,第二個姓應雙名伯爵,原是開細絹舖的應員外兒子,沒了本錢,跌落下來,專在本司三院幫嫖貼食,會一腳好氣毬,雙陸棋子,件件皆通。第三個姓謝,名希大,字子純,亦是幫閑勤兒;會一手好琵琶,每日無營運,專在院中吃些風流茶飯。還有個祝日念、孫寡嘴、吳典恩、雲裡手、常時節、卜志道、白來搶共十個朋友。卜志道故了,花子虛補了。每月會在一處,叫兩個唱的,花攢錦簇頑耍。眾人見花子虛乃是內臣家勤兒,手裡使錢撒漫,都亂撮合他在院中請表子,整三五夜不歸家。正是:

「紫陌春光好,  紅樓醉管絃;

人生能有幾,  不樂是徒然!」

此事表過不題。且說當日西門慶率同妻妾,合家歡喜,在芙蓉亭上飲酒,至晚方散;歸到潘金蓮房中,已有半酣。乘着酒興,要和婦人雲雨;婦人連忙薰香打舖,和他解衣上床。西門慶且不與他雲雨,明知婦人第一好品蕭,于是坐在青紗帳內,令婦人馬爬在身邊,雙手輕籠金釵,捧定那話,往口裡吞放。西門慶垂首翫其出入之妙,嗚咂良久,淫興倍增,因呼春梅進來遞茶。婦人恐怕丫頭看見,連忙放下帳子來,西門慶道:「怕怎麼的?」因說起:「隔壁花二哥房裡,到有兩個好丫頭,今日送花來的是小丫頭;還有一個,也有春梅年紀,也是花二哥收過用了。但見他娘在門首站立,他跟出來,見是生的好模樣兒。誰知這花二哥年紀小小的,房裡恁般用人!」婦人聽了,瞅了他一眼,說道:「怪行貨!我不好罵你!你心裡要收這個丫頭,收他便了。如何遠打週折,指山說磨,拏人家來比奴一節。不是那樣人,他又不是我的丫頭。既然如此,明日我往後邊坐,一面騰個空兒,你自在房中叫他來,收他便了。」說畢,當下西門慶品蕭過了,方纔抱頭交股而寢。正是:

「自有內事迎郎意,  慇勤快把紫蕭吹。」

有西江月為證:

「紗帳輕飄蘭麝,娥眉慣把蕭吹;雪白玉體透房幃,禁不住魂飛魂蕩,玉腕款籠金釧,兩情如醉如痴;才郎情動囑奴知,慢慢多咂一會。」

到次日,果然婦人往後邊孟玉樓房中坐了。西門慶叫春梅到房中,春點杏桃紅綻蕊,風欺楊柳綠翻腰;收用了這妮子。婦人自此一力抬舉他起來,不令他上鍋抹灶,只叫他在房中,舖床疊被,遞茶水。衣服首飾,揀心愛的與他,纏的兩隻腳小小的。原來春梅比秋菊不同,性聰慧,喜謔浪,善應對,生的有幾分顏色。西門慶甚是寵他。秋菊為人濁蠢,不任事體,婦人打的是他。正是:

「燕雀池塘語話喧,  皆因仁義說愚賢;

雖然異數同飛鳥,  貴賤高低不一般。」

畢竟未知後來如何,且聽下回分解:




\end{showcontents}
