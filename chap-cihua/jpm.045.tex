%# -*- coding: utf-8 -*-
%!TEX encoding = UTF-8 Unicode
%!TEX TS-program = xelatex
% vim:ts=4:sw=4
%
% 以上设定默认使用 XeLaTex 编译,并指定 Unicode 编码,供 TeXShop 自动识别

%第四十五回 
\chapter{桂姐央留夏花兒\KG 月娘含怒罵玳安}


\begin{showcontents}{}



「佳名號作百花王,  幼出冰肌異眾芳,

映日妖嬈呈素豔,  隨風冷淡散清香;

玉容吳妒啼粧女,  雪臉渾如傳粉郎,

檀板金尊歌勝賞,  何誇魏紫與姚黃。」

話說西門慶因放假,沒往衙門裡去。早辰起來前廳看着差玳安送兩張卓面與喬家去。一張與喬五太太,一張與喬大戶娘子,俱有高頂方糖,時伴樹菓之類。喬五太太賞了玳安兩手帕,三錢銀子;喬大戶娘子是一疋青絹,俱不必細說。原來應伯爵自從與西門慶作別,趕到黃四家,黃四又早夥中封下十兩銀子謝他:「大官人分付教俺過節去,口氣兒只是搗那五百兩銀子文書的情。你我錢糧拿甚麼支持?」應伯爵道:「你如今還得多少纔勾?」黃四道:「李三哥他不知道,只要靠着問那內臣借。一般也是五分行利,不如這裡借著衙門中勢力兒,就是上下使用也省些。如今找著再得出五十個銀子來,把一千兩合用,就是每月也好認利錢。」應伯爵聽了,低了低頭兒,說道:「不打緊。假若我替你說成了,你夥計六人怎生謝我?」黃四道:「我對李三說,夥中再送五兩銀子與你。」伯爵道:「休說五兩的話,要我手段,五兩銀子要不了你的。我只消一言替你每巧一巧兒,就在裡頭了。今日俺房下往他家吃酒,我且不去。明日他請俺每晚夕賞燈,你兩個明日絕早買四樣好下飯,再着上一罈金華酒 ;不要叫唱的,他家裡有李桂兒、吳銀兒還沒去裡。你院裡叫上六名吹打的,等我領着送了去,他就要請你兩個坐。我在傍邊,那消一言半句,管情就替你說成了。找出五百兩銀子來,共搗一千兩文書。一個月滿破認他五十兩銀子,那裡不去了,只當你包了一個月老婆了。常言道,秀才取漆無真;進錢糧之時,香裡頭多上些木頭,蠟裡頭多攙些〈扌臼〉油,那裡查帳去!不圖打點,只圖混水。借着他這名聲兒,纔好行事。」于是計議已定。到是,李三、黃四果然買了酒禮,伯爵領着兩個小廝,擡着送到西門慶家來。西門慶正在前廳打發卓面,只見伯爵來到,作了揖,道及昨日房下在這裡打擾,回家晚了。西門慶道:「我昨日周南軒那裡吃酒,回家也有一更天氣,也不曾見的新親,說老早就去了。今早衙門中放假,也沒去。看着打發了兩張卓面,與喬親家那裡去。」說畢,坐下了。伯爵就喚李錦:「你把禮擡進來。」不一時,兩個抬進儀裡放下。伯爵道:「李三哥、黃四哥再三對我說,受你大恩,節間沒甚麼,買了些微禮來孝順你賞人。」只見兩個小廝向前扒在地下磕頭。西門慶道:「你們又送這禮來做甚麼?我也不好受的,還教他擡回去。」伯爵道:「哥,你不受他的,這一擡出去,就醜死了!他還叫唱的來伏侍,是我阻住他了;只叫了六名吹打的,在外邊伺候。」西門慶即令:「與我叫進來。」不一時,把六名樂工叫至當面跪下。西門慶向伯爵道:「他既是叫將來了,莫不又打發他?不如請他兩個來坐坐罷。」伯爵得不的一聲兒,即叫過李錦來,分付:「到家對你爹說,老爹收了禮了。這裡不着請去了,叫你爹同黃四爹早來這裡坐坐。」那李錦應諾下去。須臾,收進禮去。西門慶令玳安封二錢銀子賞他。磕頭去了。六名吹打的下邊伺候。少頃、棋童兒拿茶上來那裡吃。西門慶陪伯爵吃茶。說道:「有了飯,請問爹。」西門慶讓伯爵西廂房裡坐。因問伯爵:「你今日沒會謝子純?」伯爵道:「我早辰起來時,李三就到我那裡,看着打發了禮來,誰得閑去會他?」西門慶即使棋童兒:「快請你謝爹去。」不一時,書童兒放卓兒擺飯,畫童兒用罩漆方盒兒拿了四碟小菜兒,都是裡外花靠小碟兒:精緻一碟美甘甘十香瓜茄 、一碟甜孜孜五方豆豉 、一碟香噴噴的橘醬 、一碟紅馥馥的糟笋;四大碗下飯:一碗大燎羊頭 、一碗滷燉的炙鴨 、一碗黃芽菜,並〈火川〉的餛飩雞蛋湯、一碗山藥膾的紅肉圓子 ;上下安放了兩雙金筯牙兒。伯爵面前是一盞上新白米飯兒,西門慶面前于是一甌兒香噴噴軟稻粳米粥兒 。兩個同吃了飯,收了家火去,揩抹的卓兒乾淨。西門慶與伯爵兩個坐着賭酒兒打雙陸。伯爵趁謝希大末來,乘先問下西門慶說道:「哥,明日找與李智、黃四多少銀子?」西門慶道:「把舊文書收了,另搗五百兩銀子文書就是了。」伯爵道:「這等也罷了。哥,你總不如再找上一千兩,到明日也好認利錢。我又一句話,那金子你用不着,還筭一百五十兩與他,再找不多兒了。」西門慶聽罷道:「你也說的是。我明日再找三百五十兩與他罷。改一千兩銀子文書就是了。省的金子放在家,也只是閒着。」兩個正打雙陸,忽見玳安兒走來說道:「賁四拿了一座大螺鈿大理右屏風,兩架銅鑼銅鼓,連鐺兒,說是白皇親家的,要當三十兩銀子。爹當與他不當他?」西門慶道:「你教賁四拿進來我瞧。」不一時,賁四同兩個人擡進去,放在廳堂上。西門慶與伯爵放下雙陸,走出來撇看,原來是三尺闊,五尺高,可卓放的螺鈿描金大理石屏風,端的是一樣黑白分明。伯爵觀了一回,悄與西門慶道:「哥,你仔細瞧,恰相好似蹲着個鎮宅獅子一般。」兩架銅鑼銅鼓,都是彩畫生粧雕刻雲頭,十分齊整。在傍一力攛掇說道:「哥,該當下他的。休說兩架銅鼓,只一架屏風,五十兩銀子還沒處尋去。」西門慶道:「不知他明日贖不贖?」伯爵道:「沒的說,贖甚麼?下坡車兒營生,及到三年過來,七八本利相等。」西門慶道:「也罷!教你姐夫前邊鋪子裡,兌三十兩與他罷。」剛打發去了,西門慶把屏風抹乾淨,安在大廳正面,左右看視,金碧彩霞交輝。因問:「吹打樂工吃了飯不曾?」琴童道:「在下邊打發吃飯哩。」西門慶道:「叫他吃了飯來,吹打一回我聽。」于是廳內擡出大鼓來,穿廊下邊一架安放銅鑼銅鼓,吹打起來,端的聲震雲宵,韻驚魚鳥。正吹打着,只見棋童兒請了謝希大到了,進來與二人唱了喏。西門慶道:「謝子純你過來,估估這座屏風兒值多少價?」謝希大近前觀看了半日,口裡只顧誇獎不已,說道:「哥,你這屏風買的巧,也得一百兩銀子與他,少了他不肯。」伯爵道:「你看,連這外邊兩架銅鑼銅鼓,帶鐺鐺兒,通共與了三十兩銀子。」那謝希大拍着手兒叫道:「我的南無耶,那裡尋本兒利兒!休說屏風;三十兩銀子,還攪給不起這兩架銅鑼銅鼓來。你看這兩座架,做的這工夫,硃紅彩漆,都照依官司裡的樣範,少說也有四十觔响銅,該值多少銀子?怪不的一物一主,那裡有哥這等大福,偏這樣巧價兒來尋你的!」說了一回,西門慶請入書房裡坐的。不一時,李智、黃四也到了。西門慶說道:「你兩個如何又費心送禮來?我又不好受你的。」那李智、黃四慌的下了禮說道:「小人惶恐,微物胡亂與爹賞人罷了。蒙老爹呼喚,不敢不來。」于是搬過坐兒來,打橫坐了。須臾,小廝畫童兒拿了五盞茶上來,眾人吃了,收下盞托去。少頃,玳安走上來,請問:「爹在那裡放卓兒?」西門慶令:「擡進卓兒,就在這裡坐罷。」于是玳安與書童兩個,一肩搭擡進一張八仙,瑪瑙籠漆卓兒進來,騎着火盆安放在地平上。伯爵、希大居上,西門慶主位,李智、黃四兩邊打橫坐了。須臾拿上春檠按酒,大盤大碗,湯飯點心,無非鵝、鴨、雞蹄,各樣下飯之類。酒泛羊羔 ,湯浮桃浪,樂工都在窗外吹打。西門慶叫了吳銀兒席上遞酒。這裡前邊飲酒不題。都說李桂姐家保兒,吳銀兒家丫頭蠟梅,都叫了轎子來接他姐姐家去。那桂姐聽保兒來,慌的走到門外,和保兒兩個悄悄說了半日話。回到上房,告辭要回家去。月娘再三留:「俺們如今便都往吳大妗子家去,連你們也帶了去。你越發晚了,從他那裡起身,也不用轎子,伴俺每走百病兒,就往家去便了。」桂姐道:「娘不知我家裡無人,俺姐姐又不在家,有我王姨媽那裡又請了許多人來做盒子會,俺媽不知怎麼盼我。昨日等了我一日。他不急時,不使將保兒來接我。若是閑常日子,隨娘留我幾日,我也住了。」月娘見他不肯,一面教玉筲將他那原來的盒子,裝了一盒元宵,一盒白糖薄脆 ,交與保兒掇着。又與桂姐一兩銀子,打發他早去。這桂姐先辭月娘眾人,然後他姑娘送他到前邊,教畫童替他抱了毡包,竟來書房門首,教玳安請出西門慶來說話。這玳安慢慢掀簾子,進入書房,向西門慶說道:「桂姐家去,請爹說話。」應伯爵道:「李桂兒這小淫婦兒,原來還沒去哩。」西門慶道:「他今日纔家去。」一面走出前邊來,看見李桂姐穿着紫丁香色潞州紬粧花肩子對衿祆兒,白展光五色線挑的寬襴裙子,用青點翠的白綾汗巾兒搭着頭,面前花枝招颭,綉帶飄飄,磕了四個頭,就道:「打攪爹娘這裡。」西門慶道:「你明日家去罷!」桂姐道:「家裡無人,媽使保兒拿轎子來接了。」又道:「我還有一件事對爹說。俺姑娘房裡那孩子,休要領出去罷!俺姑娘昨日晚夕,又打了他幾下。說起來還小哩,恁怎麼不知道。吃我說了他幾句,從今改了,他也再不敢。不爭打發他出去,大節間俺娘房中沒個人使,你心裡不急麼?自古木杓火杖兒短,強如手撥判。爹好歹看我分上,留下這丫頭罷。」西門慶道:「既是你恁說,留下這奴才罷。」一面分付玳安:「你去後邊對你大娘說,休要叫媒人去了。」玳安向畫童兒抱着桂姐毡包,說道:「拿桂姨毡包,等我抱着。教畫童兒後邊說去罷。」那畫童應喏,一直往後邊去了。桂姐與西門慶說畢話,東窗子前揚聲叫道:「應花子,我不拜你了!你娘家去。」伯爵道:「拉回賊小淫婦兒來,休放他去了。叫他唱一套兒,且與我聽聽着。」桂姐道:「等你娘閒了,唱與你聽。」伯爵道:「由他乾乾淨淨自你兩個梯己話兒,就不教我知道了?恁大白日就家去了,便益了賊小淫婦兒了。投到黑,還接好幾個漢子。」桂姐道:「汗邪了你這花子。」一面笑出去。玳安跟着打發他上轎去了。西門慶與桂姐說了話,後邊更衣去了。應伯爵向謝希大說:「李家桂兒這小淫婦兒,就是個真脫牢的強盜,越發賊的疼人子。恁個大節,他肯只顧在人家住?著鴇子來叫他,又不知家裡有甚麼人兒等着他哩?」謝希大道:「你好猜?」悄悄向伯爵耳邊,如此如此,這般這般,說未數句,伯爵道:「悄悄裡說道,哥正不知道哩!」不一時,西門慶走的腳步兒响進來,兩個就不言語了。這應伯爵就把吳銀兒摟在懷裡,和他一遞一口兒吃酒,說道:「是我這乾女兒,又溫柔,又軟款,強如李家狗不要的小淫婦兒一百倍了!」吳銀兒笑道:「二爹好罵,說一個就一個,百個就百個。一般一方之地,也有賢有愚,可可兒一個就比一個來?俺桂姐沒惱着你老人家!」西門慶道:「你問賊狗材!單管只個說白道的!」伯爵道:「你休管他家,等我守着我這乾女兒過日子。乾女兒過來,拿琵琶且先唱個兒我聽。」這吳銀兒不忙不慌,輕舒玉指,款跨鮫綃,把琵琶在于膝上,低低唱了一回柳搖金:

「心中牽掛,飯不飯茶不茶。難割拾我俏寃家,淒涼因為我心上放不下。更不知你在誰家?要離別,與我兩句伶仃話。拋閃殺怒家,閃賺殺奴家,你休要把奴來干罷。」

伯爵吃過酒,又遞謝希大。吳銀兒又唱道:

「常懷憂悶,何時得趁我心?牽掛著我有情人,姊妹們拘管的緊。老尊堂不放鬆,顯的我言兒無信。不愛你寶和金,只愛你生的胖兒俊。我和你做夫妻,死了甘心,教奴和你往來相趁。」

這裡和吳銀兒前邊遞酒彈唱不題。且說畫童兒走到後邊,月娘正和孟玉樓、李瓶兒、大姐、雪蛾、并大師父,都在上房裡坐的。只見畫童兒進來,月娘纔待使他叫老媽來領夏花兒出去,畫童便道:「爹使小的對大娘說,教且不要領他出去罷了!」月娘道:「你爹教賣他,怎的又不賣他了?你實說,是誰對你說,教休要領他出去。」畫童兒道:「剛纔小的抱着桂姨毡包。桂姨臨去對爹說,央及留下了;『且將就使着罷,休領出去了。』爹使玳安進來對娘說。玳安不進來,在爹根前使小的進來了;奪過毡包送桂姨去了。」這月娘聽了,就有幾分惱在心中。罵玳安道:「恁賊兩頭弒番獻勤欺主的奴才!嗔道他頭裡使他教媒人,他就說道:『爹教領出去。』原來都是他弄鬼,如今又幹辦着送他去了。住回等他進後來,我和他答話!」正說着,只見吳銀兒前邊唱了進來。月娘對他說:「你家臘梅接你來了。李家桂兒家去了,你莫不也往家去了罷?」吳銀兒道:「娘既留我,我又家去,顯的不識敬重了!」因問蠟梅:「你來做甚麼?」蠟梅道:「媽使我來瞧瞧你。」吳銀兒問道:「家裡沒甚勾當?」蠟梅道:「沒甚事。」吳銀兒道:「既沒事,你來接我怎的?你家去罷。娘留下我,晚夕還同眾娘每往妗奶奶家走百病兒去。我那裡回來,纔往家去哩。」說畢,蠟梅就要走。月娘道:「你叫他回來,打發他吃些甚麼兒。」吳銀兒道:「你大奶奶賞你東西吃哩!等着就把衣裳包子帶了家去。對媽媽說,休教轎子來,晚夕我走了家去。」因問:「吳惠化怎的不來?」臘梅道:「他在家裡害眼哩。」月娘分付玉筲、臘梅到後邊,拿下兩碗肉,一盤子饅頭,一甌子酒,打發他吃。又拿他原來的盒子,裝了一盒元宵,一盒細茶食,回與他拿去。原來吳銀兒的衣裳包兒,放在李瓶兒房裡。李瓶兒連忙又早尋下一套上色織金段子衣服,兩方銷金汗巾兒,一兩銀子,安放在他毡包內與他。那吳銀兒喜孜孜辭道:「娘,我不要這衣服罷。」又笑嘻嘻道:「實和娘說,我沒個白祆兒穿。娘收了這段子衣服,不拘娘的甚麼舊白綾襖兒,與我一件兒穿罷。」李瓶兒道:「我的白祆子多寬大,你怎的?」于是叫迎春拿鑰匙上大櫥櫃裡,拿一疋整白綾來與銀姐:「對你媽說,教裁縫替你裁兩件好祆兒。」因問:「你要花的要素的?」吳銀兒道:「娘,我要素的罷,圖襯着比甲兒好穿。」笑嘻嘻向迎春說道:「又起動叫姐往樓上走一遭,明日我沒甚麼孝順,只是唱曲兒與姐姐聽罷了。」須臾,迎春從樓上取了一疋松江闊機尖素白綾,下號兒寫着重三十八兩,遞與吳銀兒。銀兒連忙花枝招颭,綉帶飄飄,插燭也是與李瓶兒磕了四個頭,起來又深深拜了迎春八拜。李瓶兒道:「銀姐,你把這段子衣服還包了去,早晚做酒衣兒穿。」吳銀兒道:「娘賞了白綾做祆兒,又包了這衣服去。」于是又磕頭謝了。不一時,臘梅吃了東西,交與盒子毡包,都拿回家去了。月娘便說:「銀姐,你這等我纔喜歡。你休學李桂兒那等喬張致,昨日和今早,只相臥不住虎子一般,留不住的,只要家去,可可兒家裡就忙的恁樣兒?連唱也不用心唱了!見他家人來接,飯也不吃就去了,就不待見了。銀姐,你快休學他!」吳銀兒道:「好娘,這裡一個爹娘宅裡,是那裡去處?就有虛實,放着別處,便敢在這裡使!桂姐年幼,他不知事,俺娘休要惱。」正說着,只見吳大妗子家,使了小廝來定兒來請,說道:「俺娘上覆三姑娘,好歹同眾位娘并桂姐、銀姐請早些過去罷;又請雪姑娘也走走。」月娘道:「你到家對你娘說,俺們如今便收拾去。二娘害腿疼不去,他在家看家哩。你姑夫今日前邊有人吃酒,家裡沒人,後邊姐也不去。李桂姐家去了,連大姐、銀姐和俺每六位去。你家少費心整治甚麼,俺每坐一回,晚上就來。」因問來定兒:「你家叫了誰在那裡唱?」來定兒道:「是郁大姐。」說畢,來定兒先去了。月娘一面同玉樓、金蓮、李瓶兒、大娘并吳銀兒,對西門慶說了,分付奶子在家看哥兒,都穿戴收拾定當,共六頂轎子起身。派定玳安兒、棋童兒、來安兒三個小廝,四名排軍跟轎,往吳大妗子家來。正是:

「萬井風光春落落,  千門燈火夜漫漫;

此生此夜不長見,  明月明年何處看?」

畢竟未知後來何如,且聽下回分解:




\end{showcontents}


