%# -*- coding: utf-8 -*-
%!TEX encoding = UTF-8 Unicode
%!TEX TS-program = xelatex
% vim:ts=4:sw=4
%
% 以上设定默认使用 XeLaTex 编译,并指定 Unicode 编码,供 TeXShop 自动识别

%第九十回 
\chapter{來旺盜拐孫雪娥\KG 雪娥官賣守備府}


\begin{showcontents}{}



「花開花落開又落,  錦衣布衣更換著,

豪家未必常富貴,  貧人未必常寂寞;

扶人未必上青天,  推人未必填溝壑,

勸君凡事莫怨天,  天意與人無厚薄。」

話說吳大舅領着月娘等一簇男女,離了永福寺,順着大樹長堤前來。玳安又早在杏花村酒樓下邊,人烟熱鬧,揀高阜去處,那里幕天席地設下酒殽,等候多時了。遠遠望月娘眾人轎子到了,問道:「如何咱纔來?」月娘又把永福寺中遇見春梅,告訴一遍。不一時,斟上酒來。眾人坐下,正飲酒,只見樓下香車綉轂,往來人烟喧雜,車馬轟雷,笙哥鼎沸。月娘眾人躧着高阜,把眼觀看。看見人山人海圍着,都看教師走馬耍解的。原來是本縣知縣相公兒子李衙內,名喚李拱璧,年約三十餘歲,見為國子上舍。一生風流博浪,懶習詩書,專好鶯犬走馬,打毬蹴踘。常在三瓦兩巷中走,人稱仔為李棍子。那日穿着一弄兒輕羅軟滑衣裳,頭戴金頂纏棕小帽,腳踏乾黃靴,納綉襪口,同廊史何不違帶領二三十好漢,拏彈弓吹筒毬棒,在於杏花庄大酒樓下,看教場李貴走馬賣解,豎肩椿,隔肚帶,輪鎗舞棒,做各樣技藝頑耍。有這許多男女圍着烘笑,那李貴諢名,號為山東夜叉,頭戴萬字中,腦後撲匾金環,身穿紫窄衫,銷金裹肚,腳上耙蹋腿絣,乾黃〈革翁〉靴,五彩飛魚襪口,坐下銀鬃馬,手執朱紅桿明鎗,頭招風令字旗,在街心扳鞍上馬,高聲說念一篇道:

「我做教師世罕有,江湖遠近揚名久。雙拳打下如鎚鑽,兩腳入來如飛走。南北兩京打戲臺,東西兩廣無敵手。分明是個鐵嘴行,自家本事何曾有!少林棍,只好打田雞;董家拳,只好嚇小狗。撞對頭不敢喊一聲,沒人處專會誇大口!騙得銅錢放不牢,一心要折章臺柳。虧了北京李大郎,養我在家為契友。蘸生醬喫了半畦蒜,捲春餅〈口床〉了兩擔韮。小人自來生得饞,寅時吃酒直到酉。牙齒疼,把來剉一剉,肚子脹,將來扭一扭。充饑吃了三斗米飯,點心吃了七石缸酒。多虧了此人未得酬,來世做隻看家狗。若有賊來掘壁洞,把他陰囊咬一口。問君何故咬他囊?!動不的手來只動口!」

當下李衙內一見那長挑身材婦人,不覺心搖目蕩,觀之不足,看之有餘。口中不言,心內暗道:「不知誰家婦女,有男子沒有?」一面叫過手下答應的小張閑架兒來,悄悄分付:「你去那高坡上,打聽那三個穿白的婦人是誰家的?訪得是實,告我知道。」那小張閑掩口應諾,雲飛跑去。不多時,走到跟前,附耳低言,回報說:「如此這般,是縣門前西門慶家妻小。一個年老的姓吳,是他嫂子。一個五短身材,是他大娘子吳月娘。那個長挑身材,有白麻子的,是第三個娘子,姓孟,名喚玉樓。如今都守寡在家。」這李衙內聽了,獨看著孟玉樓,重賞小張閑,不在話下。吳大舅和月娘眾人,觀看了半日,見日色銜山,令玳安收拾了食盒,攛掇月娘上轎回家。一路上得多少錦轡郎袖醉,綺羅人揭綉簾看。有詩為證:

「柳底花陰壓路塵,  一回遊賞一回新;

有緣千里來相會,  無緣對面不相親。」

這月娘眾人回家不題。都說那日孫雪娥與西門大姐在家,午後時分無事,都出大門首站立。也是天假其便,不想一個搖驚閨的過來。那時賣胭脂粉花翠生活磨鏡子,都搖驚閨。大姐說:「我鏡子昏了,使平安兒叫住那人,與我磨磨鏡子。」那人放下擔兒說道:「我不會磨鏡子,我賣些金銀生活,首飾花翠。」站立在門前,只顧眼上眼下看着雪娥。雪娥便道:「那漢子,你不會磨鏡子,去罷,只顧看我怎的?」那人說:「雪姑娘、大姑娘,不認的我了?」大姐道:「眼熟,急忙想不起來。」那人道:「我是爹手裏出去的來旺兒。」雪娥便道:「你這幾年在那里來?怎的不見?出落得恁胖了!」來旺兒道:「我離了爹門,到原籍徐州家裏,閑着沒營生,投跟了個老爹上京來做官。不想到半路里,他老爺兒死了,丁憂家去了。我便投在城內顧銀舖,學會了此銀行手藝,揀鈒大器頭面,各樣生活。這兩日市遲,顧銀舖教我挑副擔兒出來,街上發賣些零碎。看見娘們在門首,不敢來相認,恐怕踅門瞭戶的!今日不是你老人家叫住,還不敢相認!」雪娥道:「原來教我只顧認了半日,白想不起!既是舊兒女,怕怎的?」因問:「你擔兒裏賣的是甚麼生活?挑進裏面,等俺每看一看。」那來旺一面把擔兒挑入裏邊院子里來,打開箱子,用匣兒托出幾件首飾來,金銀箱嵌不等,打造得十分奇巧。但見:

「孤雁啣蘆,雙魚戲藻。牡丹巧嵌碎寒金,貓眼釵頭火焰蠟。也有獅子滾綉球,駱駝獻寶。滿冠擎出廣寒宮,掩鬢鑿成桃源境。左右圍髮,利市相對荔枝叢;前後分心,觀音盤膝蓮花座。也有寒雀爭梅,也有孤鶯戲鳳。正是:絲環平安珇珊綠,帽頂高嵌佛頭青。」

看了一回,問來旺兒:「你還有花翠?拏出來。」那來旺兒又取一盒子各樣大翠鬢花、翠翹滿冠,并零碎草蟲生活來。大姐揀了他兩對鬢花,這孫雪娥便留了他一對翠鳳,一對柳穿金魚兒。大姐便稱出銀子來與他,雪娥兩件生活,欠他一兩二錢銀子,約下他:「明日早來取罷,今日你大娘不在家,同你三娘和哥兒都往坟上與你爹燒紙去了。」來旺道:「我去年在家裏,就聽見人說爹死了,大娘子生了哥兒,怕不的好大了?」雪娥道:「你大娘孩兒如今纔周半兒,一家兒大大小小,如寶上珠一般,全看他過日子哩!」說話中間,來昭妻一丈青出來,傾了盞茶與他吃。那來旺兒接了茶,與他唱了個喏。來昭也在跟前,同敘了回話。分付:「你明日來見見大娘。」那來旺兒挑擔出門。到晚上,月娘眾人轎子來家。雪娥、大姐、眾人丫鬟接着,都磕了頭。玳安跟盒擔走不上,雇了疋驢兒騎來家,打發抬盒入去了。月娘告訴雪娥、大姐,說今日寺裏遇見春梅一節:「原來他把潘家的就葬在寺後首,俺們也不知。他來替他娘燒紙,誤打誤撞遇見他,娘兒們又認了回親。先是寺裏長老擺齋吃了,落後又放下兩張卓席,教伴當擺上他家的四五十攢盒,各樣菜蔬下飯,篩酒上來,通吃不了。他看見哥兒,又與了一對簪兒,好不和氣!起解行三坐五,坐着大轎子,許多跟隨,又且是出落的比舊時長大了好些,越發白胖了。」吳大妗子道:「他倒也不改常忘舊,那時在咱家時,我見他比眾丫鬟行事兒正大,說話兒沉穩,就是個才料兒!你看今日福至心靈,恁般造化。」孟玉樓道:「姐姐沒問他,我問他來,果然半年沒洗換,身上懷着喜事哩!也只是八九月里孩子,守備好不喜歡哩!薛嫂兒說的倒不差。」說了一回。雪娥題起:「今日娘不在,我和大姐在門首,看見來旺兒。原來又在這里學會了銀匠,挑着擔兒賣金銀生活花翠,俺每就不認得他了!買了他幾枝花翠。他問娘來,我說往坟上燒布去了。」月娘道:「你怎的不教他等着我來家?」雪娥道:「俺們叫他明日來。」正坐着說話,只見奶子如意兒向前對月娘說:「哥哥來家,這半日只是昏睡不醒,口中出冷氣,身上湯燒火熱的。」這月娘聽見慌了,向炕上抱起孩兒來,口搵着口兒,果然出冷汗,渾身發熱。罵如意兒:「好淫婦,此是轎子冷了孩兒了!」如意兒道:「我拿小被兒裹的沒沒的,怎得凍着?」月娘道:「再不是,抱了往那死鬼坟上,諕了他來了!那等分付,教你休抱他去,你不依,浪着抱的去了!」如意兒道:「早是小玉姐看着,抱了他到那里,看看就來了。幾時諕着他來?」月娘道:「別要說嘴!看那看兒,便怎的都把他諕了?」即忙叫來安兒:「快請劉婆子去。」不一時,劉婆來到。看了脉息,抹了身上,說:「着了些驚寒,撞見祟禍了。」留了兩服硃砂丸,用姜湯灌下去。分付奶子:「捲着他,熱炕上睡。」到半夜出了些冷汗,身上纔涼了。於是管待劉婆子吃了茶,與了他三錢銀子,叫他明日還來看看。一家子慌的要不的,開門闔戶,整亂了半夜。都說來旺次日依舊挑將生活擔兒,來到西門慶門首,與來昭唱喏,說:「昨日雪姑娘留下我些生活,許下今日教我來取銀子,就見見大娘。」來昭道:「你且去看,改日來。昨日大娘來家,哥兒不好,叫醫婆太醫看下藥,整亂一夜,好不焦心。今日纔好些,那得工夫稱銀子與你?」正說着,只見月娘、玉樓、雪娥送出劉婆子來,到大門首,看見來旺兒。那來旺兒扒在地下,與月娘、玉樓磕了兩個頭。月娘道:「幾時不見你,就不來這里走走?」來旺兒悉將前事說了一遍:「要來不好來的。」月娘道:「舊兒女人家,怕怎的?你爹又沒了。當初只因潘家那淫婦,一頭放火,一頭放水,架的舌,把個好媳婦兒,生逼臨的弔死了!將有作沒,把你墊發了去!今日天也不容他,往那去了!」來旺兒道:「也說不的,只是娘人裏明白就是了!」說了回話,月娘問他:「賣的是甚樣的生活?」拏出來瞧,揀了他幾件首飾,該還他三兩二錢銀子,都用等子稱與他。叫他進入儀門裏面,分付小玉取一壺酒來,又是一般點心,教他吃。那雪娥在廚上,一力攛掇,又熱了一大碗肉出來與他。吃的酒飯飽了,磕頭出門。月娘、玉樓眾人歸到後邊去。雪娥獨自悄悄和他打話:「你常常來走着,怕怎的?奴有話,教來昭嫂子對你說。我明日晚夕,在此儀門裏紫牆兒跟前耳房內等你。」兩個遞了眼色。這來旺兒就知其意,說:「這儀門晚夕關不關?」雪娥道:「如此這般,你先到來昭屋裏。等到晚夕,踩着梯凳,越過牆,順着遮隔,我這邊接你下來。咱二人會合一面,還有底細話與你說。」這來旺得了此話,正是:

「歡從額起,  喜向腮生。」

作辭雪娥,挑擔兒出門。正是:

「不著家神,  弄不得家鬼!」

有詩為證:

「閒來無事倚門闌,  偶遇多情舊日緣;

對人不敢高聲話,  故把秋波送幾番。」

這來旺兒歡喜回家,一宿無話。到次日,也不挑擔兒出來賣生活,慢慢踅來西門慶門首,等來昭出來,與他唱喏。那來昭便說:「旺兒希罕,好些時不見你了!」來旺兒說:「沒事,閒來走走。裡邊雪姑娘少我幾錢生活銀,討討。」來昭道:「既如此,請來屋裡坐。」把來旺兒讓到房里坐下。來旺兒道:「嫂子怎不見?」來昭道:「你嫂子今日後邊上灶哩。」那來旺兒拿出一兩銀子,遞與來昭說:「這幾星銀子,取壺酒來和哥嫂吃。」來昭道:「何消這許多!」即叫他兒子鐵棍兒過來,那鐵棍弔起頭去,十五歲了;拿壺出來,打了一大注酒。使他後邊叫一丈青來。不一時,一丈青蓋了一錫鍋熱飯,一大碗雜熬下飯,兩碟菜蔬,說道:「好呀,旺官兒在這里!」來昭便拿出銀子與一丈青瞧,說:「兄弟破費,也打壺酒咱兩口兒吃。」一丈青笑道:「無功消受,怎生使得?」一回放了炕卓,讓來旺炕上坐。擺下酒菜,把酒來斟。來旺兒先傾頭一盞,遞與來昭,次斟一盞,與一丈青,深深唱喏,說:「一向不見哥嫂,這盞水酒,孝順哥嫂。」一丈青便說:「哥嫂不道酒肉吃傷了?你對真人休說假話!裡邊雪姑娘昨日已央及達知我了。你兩個舊情不斷,托俺每兩口兒,如此這般周全。你每休推睡里夢里!要問山下路,且得過來人!你若入港相會,有東西出來,休要獨吃,須把些汁水,教我呷一呷!俺替你們須躭許多利害!」那來旺便跪下說:「只是望哥嫂周全,並不敢有忘。」說畢,把酒吃了一回。一丈青往後邊和雪娥答了話。出來對他說,約定晚上來來昭屋裏窩藏,待夜裏關上儀門,後邊人歇下,越牆而過,於中取事。有詩為證:

「報應本無私,  影響皆相似;

要知禍福因,  但看所為事。」

這來旺得了此言,回來家,巴不到晚,踅到來昭屋裡,打酒和他兩口兒吃。至更深時分,更無一人覺的,直待的大門關了,後邊儀門上了拴,家中大小歇息定了。彼此都有個暗號兒,只聽牆內雪娥咳嗽之聲。這來旺兒躧着梯凳,黑影中扒過粉牆,順着遮洋〈扌扉〉子,雪娥那邊用凳子接着,兩個在西耳房堆馬鞍子去處,兩個相摟相抱,雲雨做一處。彼此都是曠夫寡女,慾心如火。那來旺兒纓鎗強壯,儘力般弄了一回,樂極精來,一泄如注。事畢,雪娥遞與他一包金銀首飾,幾兩碎銀子,兩件段子衣服,分付:「明日晚夕你再來,我還有些細軟與你,你外邊尋下安身去處。往後這家中過不出好來,不如我和你悄悄出去,外邊尋下房兒,成其夫婦。你又會銀行手藝,愁過不得日子?」來旺兒便說:「如今東門外細米巷,有我個姨娘,有名收生的屈老娘,他那里曲彎小巷倒避眼,咱兩個投奔那里去。遲些時,看無動靜,我帶你往原籍家去,買幾畝地種去也好。」兩個商量已定,這來旺兒作別雪娥,依舊扒過牆來,到來昭屋裏。等至天明,開了大門,挨身出去。到黃昏時分,又來門首,踅入來昭屋裏,晚夕依舊跳過牆去,兩個幹事。朝來暮往,非止一日,也抵盜了許多細軟東西,金銀器皿,衣服之類。來昭兩口子,也得抽分好些肥己,俱不必細說。一日,後邊月娘看孝兒出花心,心中不快,睡得早。這雪娥房中使女中秋兒,原是大姐使的。因李嬌兒房中元宵兒被經濟耍,月娘就把中秋兒與了雪娥,把元宵兒扶侍大姐。那一日,雪娥打發中秋兒睡下。房裏打點一包釵環頭面,裝在一個匣內,用手帕蠻蓋了頭,隨身衣服,約定來旺兒在來昭屋裏等候,兩個要走。這來昭便說:「不爭你走了,我看守大門,管放水鴨兒?若大娘知道,問我要人,怎了?不如你二人打房上去,就躧破些,還有踪跡。」來旺兒道:「哥也說得是。」雪娥又留一個銀折盂、一根金耳幹、一件青綾襖、一條黃綾裙,謝了他兩口兒,直等五更鼓,月黑之時,隔房扒過去。來昭夫婦又篩上兩個大鍾煖酒,與來旺、雪娥吃,說:「吃了好走,路上壯膽些。」吃到五更時分,每人拏着一根香躧着梯子,打發兩個扒上房去,一步一步走,把房上氏也跳破許多。比及扒到房簷跟前,街上人還未行走。聽巡捕的聲音,這來旺兒先跳下去,後都教雪娥躧着他肩背,接摟下來。兩個往前邊走,到十字路口上,被巡捕的攔住,便說:「往那里去的男女?」雪娥便諕慌了手腳。這來旺兒不慌不忙,把手中官香彈了一彈,說道:「俺是夫婦二人,前往城外岳廟裏燒香,起的早了些。長官勿怪。」那人問:「背的包袱內是甚麼?」來旺兒道:「是香燭紙馬。」那人道:「既是兩口兒,岳廟燒香,也是好事,你快去罷。」這來旺得不迭一聲,拉着雪娥往前飛走。走到城下,城門纔開。打人鬧里,挨出城去,轉了幾條街巷。原來細米巷在個僻靜去處,住着不多幾家人家,都是矮房低廈,後邊就是大水穴沿子。到於屈姥姥家,屈姥姥還未開門,叫了半日,屈姥姥纔起來開了門兒,來旺兒領了個婦人來。原來來旺兒本姓鄭,名喚鄭旺。說:「這婦人是我新尋的妻小。姨娘這里有房子,且尋一個寄住些時,再尋房子。」遞與屈姥姥三兩銀子,教買柴米。那屈姥姥見這金銀首飾,來因可疑。他兒子屈鏜,因他娘屈姥姥安歇鄭件夫妻,二人帶此東西,夜晚見財起意,掘開房門,偷盜出來耍錢。致被捉獲,具了事件,拏去本縣見官。李知縣見係賊賍之事,賍身執儀見在,差人押着屈鏜到家,把鄭旺、孫雪娥,一條索子都拴了。那雪娥諕的臉蠟查也似黃了,換了滲淡衣裳,帶着眼紗,把手上戒指都勒下來,打發了公人,押去見官。當下烘動了一街人觀看。有認得的,說:「是西門慶家小老婆,今被這走出去的小廝來旺兒,今改名鄭旺,通姦拐盜財物,走外居住。又被這屈鏜掏摸了。今事發見官。」當下一個傳十,十個傳百個,路人行人口似飛!月娘家中自從雪娥走了,房中中秋兒見廂內細軟首飾都沒了,衣服丟的亂三攪四,報與月娘。月娘吃了一驚,便問中秋兒:「你跟着他睡,走了你豈不知?」中秋兒便說:「他要晚夕,悄悄偷走出外邊,半日方回。不知詳細。」月娘又問來昭:「你看守大門,人出去你怎不曉的?」來昭便說:「大門每日上鎖,莫不他飛出去?」落後看見上瓦躧破許多,方知越房而去了。又不敢使人躧訪,只得按納含忍。不想本縣知縣,當堂問理這件事,先把屈鏜夾了一頓,追出金頭面四件、銀首飾三件、金環一雙、銀鐘二個、碎銀五兩、衣服二件、手帕一個、匣一個;向鄭旺名下,追出銀三十兩、金碗簪一對、金仙子一件、戒指四個;向雪娥名下,追出金挑心一件、銀鐲一付、金鈕五付、銀簪四對、碎銀一包;屈姥姥名下,追出銀三兩。就將來旺兒問擬奴婢因奸盜取財物,屈鏜係竊盜,俱係雜犯死罪,准徒五年,賍物入官。雪娥孫氏,係西門慶妾,與屈姥姥,當下都當官拶了一拶。屈姥姥供明放了。雪娥責令本縣差人到西門慶家,教人遞領狀領孫氏。那吳月娘叫吳大舅來商議:「已是出醜,平白又領了來家做甚麼?沒的玷辱了家門,與死的裝幌子!」打發了公人錢,回了知縣話。知縣拘將官媒人來,當官變賣。都說守備府中春梅,打聽得知,說:「西門慶家中孫雪娥,如此這般,被來旺兒拐出,盜了財物去,在外居住。事發到官,如今當官變賣。」這春梅聽見,要買他來家上灶,要打他嘴,以報平昔之仇。對守備說:「雪娥善能上灶,會做的好菜飯湯水,買來家中伏侍。」這守備即便差張勝、李安,拿帖兒對知縣說。知縣自恁要做分上,只要八兩銀子官價。交完銀子,領到府中,先見了大奶奶,并二奶奶孫氏,次後到房中來見春梅。春梅正在房里縷金床錦帳之中,纔起來。手下丫鬟領雪娥見面。那雪娥見是春梅,不免低身進見,望上倒身下拜,磕了四個頭。這春梅把眼瞪一瞪,喚將當直的家人媳婦上來:「與我把這賤人撮去了䯼髻,剝了上蓋衣裳,打入廚下,與我燒火做飯!」這雪娥聽了,口中只叫苦。自古世間打牆板兒翻上下,掃米都做管倉人!既在他簷下,怎敢不低頭!孫雪娥到此地步,只得摘了髻兒,換了艷服,滿臉悲慟,往廚下去了。有詩為證:

「布袋和尚到明州,  策杖芒鞋任意遊;

饒你化身千百億,  一身還有一身愁。」

畢竟未知後來如何,且聽下回分解:





\end{showcontents}


