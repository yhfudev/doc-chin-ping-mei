%# -*- coding: utf-8 -*-
%!TEX encoding = UTF-8 Unicode
%!TEX TS-program = xelatex
% vim:ts=4:sw=4
%
% 以上设定默认使用 XeLaTex 编译,并指定 Unicode 编码,供 TeXShop 自动识别

%第五十六回 
\chapter{西門慶周濟常時節\KG 應伯爵舉荐水秀才}


\begin{showcontents}{}



「斗積黃金侈素封,  蘧蘧莊蝶夢魂中,

曾聞郿鄔光難駐,  不道銅山運可窮;

此日分籯推鮑子,  當年沉水笑龐公,

悠悠末路誰知己,  惟有夫君尚古風。」

這八句單說人生世上,榮華富貴,不能常守。有朝無常到來,恁地堆金積玉,出落空手歸陰。因此西門慶仗義疎財,救人貧難,人人都是贊嘆他的,這也不在話下。當日西門慶留下兩箇歌童祇候著:「遇有呼喚,不得有違。」兩人應諾去了。隨即打發苗家人回書禮物,又賞了些銀錢。苗實、苗秀磕頭謝了出門。後來兩個歌童,西門慶畢竟用他不著,都送太師府去了。正是:

「千金散盡教歌舞,  留與他人樂少年。」

卻說常時節自那日席上求了西門慶的事情,還不得個到手,房主又日夜催迸了不的。恰遇西門慶自從在東京來家,今日也接風,明日也接風,一連過了十來日,只不得個會面。常言道:「見面情難盡。「一個不見,卻告訴誰?每日央了應伯爵只走到大官人門首,問聲說不在,就空回了。回家又被渾家埋怨道:「你也是男子漢,大丈夫,房子沒間住,吃這般懊惱氣!你平日只認的西門大官,今日求些周濟,也做了瓶落水!」說的常時節有口無言,呆登登不敢做聲。到了明日,早起身尋了應伯爵,來到一個酒店內。只見小小茅簷兒,靠著一灣流水。門前綠樹陰中,露出酒望子來。五七個火家,搬酒搬肉不住的走。店裡橫著一張櫃檯,掛幾樣鮮魚鵝鴨之類,到潔淨可坐。便請伯爵店裡吃三盃去。伯爵道:「這卻不當生受。」常時節拉了到店裡坐下,量酒打上酒來,擺下一盤薰肉,一盤鮮魚。酒過兩巡,常時節道:「小弟向求哥和西門大官人說的事情,這幾日通不能勾會,房子又催迸的緊。昨晚被房下聒絮了半夜,耐不的五更抽身,專求哥趁早大官人還沒出門時,慢慢地候他。不知哥意下如何?」應伯爵道:「受人之托,必當終人之事。我今日好歹要大官人助你些就是了。」兩個人又吃過幾盃。應伯爵便推:「早酒不吃罷。」常時節又勸一盃。等還酒錢,一同出門,逕奔西門慶屋裡來。那時正是新秋時候,金風荐爽。西門慶連醉了幾日,覺精神減了幾分。正遇周內相請酒,便推事故不去。自在花園藏春塢遊玩。原來西門慶后園那藏春塢,有的是菓樹鮮花兒,四季不絕。這時雖是新秋,不知開著多少花朵在園裡。西門慶無事在家,只是和吳月娘、孟玉樓、潘金蓮、李瓶兒五個在花園裡頑耍。只見西門慶頭戴著忠靖冠,身穿柳綠緯羅直身粉頭靴兒。月娘上穿柳綠杭絹對衿襖兒,淺藍水紬裙子,金紅鳳頭高底鞋兒。孟玉樓上穿鴉青段子襖兒,鵝黃紬裙子,桃紅素羅羊皮金滾口高底鞋兒。潘金蓮上穿著銀紅縐紗白絹裏對衿衫子,荳綠沿邊金紅心比甲兒,白杭絹畫拖裙子,粉紅花羅高底鞋兒。只有李瓶兒上穿素青杭絹大衿襖兒,月白熟絹裙子,淺藍玄羅高底鞋兒。四個妖妖嬈嬈,伴著西門慶尋花問柳,好不快活。且說常時節和應伯爵來到廳上,問知大官人在屋裡,懽的坐著等了好半日,卻不見出來。只見門外書童和畫童兩個抬著一隻箱子,都是綾絹衣服,氣吁吁走進門來,亂嚷道:「等了這半日,還只得一半!」就廳上歇下。應伯爵便問:「你爹在那裡?」書童道:「爹在園裡頑耍哩。」伯爵道:「勞你說聲。」兩個依舊抬著進去了。不一時書童出來道:「爹請應二爹、常二叔少待,便出來。」兩人坐著等了一回,西門慶纔走出來。二人作了揖,便請坐地。伯爵道:「連日哥吃酒忙,不得些空。今日卻怎的在家裡?」西門慶道:「自從那日別後,整日被人家請去飲酒,醉的了不的,通沒些精神。今日又有人請酒,我只推有事不去。」伯爵道:「方纔那一箱衣服,是那里抬來的?」西門慶道:「這目下交了秋,大家都要添些秋衣。方纔一箱是你大嫂子的,還做不完,纔勾一半哩。」常時節伸著舌頭道:「六房嫂子就六箱了,好不費事!小戶人家,一疋布也難的。恁做著許多綾絹衣服,哥果是財主哩!」西門慶和應伯爵都笑起來。伯爵道:「這兩日杭州貨船怎地還不見到?不知他買賣貨物何如?前日哥許下李三、黃四的銀子,哥許他待門外徐四銀到手,湊放與他罷!」西門慶道:「貨船不知在那里擔閣著,書也沒稍封寄來。好生放不下。李三、黃四的,我也只得依你了。」應伯爵挨到身邊坐下,乘間便說:「常二哥那一日在哥席上求的事情,一向哥又沒的空,不曾說的。常二哥被房主催迸慌了,每日被嫂子埋怨。二哥只麻作一團,沒個理會。如今又是秋涼了,身上皮襖兒,又當在典舖哩。哥若有好心,常言道:『救人須救時無。』省的他嫂子日夜在屋裡絮絮叨叨。況且尋的房子住著了,人走動也只是哥的體面。因此常二哥央小弟特地來求哥,早些周濟他吧。」西門慶道:「我當先曾許下他來。因為東京去了這番,費的銀子多了。本待等韓夥計到家,和他理會。要房子時,我就替他兌銀子買。如今又恁地要緊?」伯爵道:「不是常二哥要緊,當不的他嫂子聒絮,只得求哥早些便好。」西門慶躊躇了半晌,道:「既這等,也不難。且問你,要多少房子纔勾住了?」伯爵道:「他兩口兒也得一間門面,一間客坐,一間床房,一間廚灶,四間房子是少不得的。論著價銀,也得三四個多銀子。哥只早晚湊些,交他成就了這樁事罷。」西門慶道:「今日先把幾兩碎銀與他拏去。買件衣服,辦些家活,盤攪過來。待尋下房子,我自兌銀與你成交,可好麼?」兩個一齊謝道:「難得哥好心。」西門慶便叫書童:「去對你大娘說,皮匣內一包碎銀取了出來。」書童應諾去了。不一時取了一包銀子出來,遞與西門慶。西門慶對常時節道:「這一包碎銀,是那日東京太師府賞封剩下的十二兩,你拿去好雜用。」打開與常時節看,都是三五錢一塊的零碎紋銀。常時節接過放在衣袖裡,就作揖謝了。西門慶道:「我這幾日不是要遲你,只等你尋下房子,一攪果和你交易。你又沒曾尋的,如今即忙便尋下,待我有銀,一起兌去便了。」常時節又稱謝不迭。三個依舊坐下。伯爵便道:「幾個古人,輕財好施,到後來子孫高大門閭,把祖宗基業一發增的多了。慳吝的積下許多金寶,後來子孫不好,連祖宗墳土也不保。可知天道好還哩!」西門慶道:「兀那東西是好動不喜靜的,曾肯埋沒在一處?也是天生應人用的,一個人堆積,就有一個人缺少了。因此積下財寶,極有罪的。」有詩為證:

「積玉堆金始稱懷,  誰知財寶禍根 ,

一文愛惜如膏血,  仗義翻將笑作呆;

親友人人同陌路,  存形心死定堪哀,

料他也有無常日,  空手俜伶到夜臺。」

正說著,只見書童托出飯來,三人吃了。常時節作謝起身,袖著銀子懽的走到家來。剛剛進門,只見那渾家鬧炒炒嚷將出來,罵道:「梧桐葉落滿身,光棍的行貨子!出去一日,把老婆餓在家裡,尚兀是千懽萬喜到家來,可不害羞哩!房子沒的住,受別人許多酸嘔氣,只教老婆耳躲裡受用。」那常二只是不開口。任老婆罵的完了,輕輕把袖裡銀子摸將出來,放在桌兒上,打開瞧著道:「孔方兄,孔方兄,我瞧你光閃閃响噹噹的無價之寶,滿身通麻了,恨沒口水嚥你下去。你早些來時,不受這淫婦幾場合氣了!」那婦人明明看見包里十二三兩銀子一堆,喜的搶近前來,就想要在老公手裡奪去。常二道:「你生世要罵漢子,見了銀子,就來親近哩!我明日把銀子去買些衣服穿,好自去別處過活,卻再不和你鬼混了。」那婦人陪著笑臉道:「我的哥,端的此是那里來的這些銀子?」常二也不做聲。婦人又問道:「我的哥,難道你便怨了我?我只是要你成家。今番有了銀子,和你商量停當,買房子安身,卻不好?到恁地喬張致!我做老婆的不曾有失花兒,憑你怨我,也是枉了!」常二也不開口。那婦人只顧饒舌,又見常二不揪不採,自家也有幾分慚愧了,禁不的吊下淚來。常二看了,嘆口氣道:「婦人家不耕不織,把老公恁地發作!」那婦人一發吊下淚來。兩個人都閉著口,又沒個人勸解,悶悶的坐著。常二尋思道:「婦人家也是難做。受了辛苦埋怨人,也怪他不的。我今日有了銀子,不採他,人就道我薄情。便大官人知道,也須斷我不是。」就對那婦人笑道:「我自耍你,誰怪你來?只你時常聒噪,我只得忍著出門去了。卻誰怨你來?我明白和你說,這銀子原是早上耐你不的,特地請了應二哥在酒店裡吃了三盃,一同往大官人宅裡等候。恰好大官人正在家,沒曾去吃酒。多虧了應二哥,不知費許多唇舌,纔得這些銀子到手。還許我尋下房子,一頓兌銀與我成交哩!這十二兩是先教我盤攪過日子的。」那婦人道:「原來正是大官人與你的。如今又不要花費開了,尋件衣服過冬,省的耐冷。」常二道:「我正要和你商量,十二兩紋銀買幾件衣服,辦幾件家活在家裡。等有了新房子,搬進去也好看些。只是感不盡大官人恁好情。後日搬了房子,也索請他坐坐是。」婦人道:「且到那時,再作理會。」正是:

「惟有感恩并積恨,  萬年千載不生塵。」

常二與婦人兩個說了一回,那婦人道:「你那里吃飯來沒有?」常二道:「也是大官人屋裡吃來的,你沒曾吃飯,就拿銀子買了米來。」婦人道:「仔細拴著銀子,我等你就來。」常二取栲栳望街上便走。不一時買了米,栲栳上又放著一大塊羊肉兒,笑哈哈跑進門來。那婦人迎門接住道:「這塊羊肉又買他做甚?」常二笑道:「剛纔說了許多辛苦,不爭這一些羊肉,就牛也該宰幾個請你。」那婦人笑指著常二罵道:「狠心的賊,今日便懷恨在心,看你怎的奈何了我?」常二道:「只怕有一日叫我一萬聲親哥,饒我小淫婦罷,我也只不饒你哩!試試手段看。」那婦人聽說,笑的走井邊打水去了。當下婦人做了飯,切了一碗羊肉,擺在卓兒上,便叫:「哥吃飯。」常二道:「我纔在大官人屋裡吃的飯,不要吃了。你餓的慌,自吃些罷。」那婦人便一個自吃了。收了家活,打發常二去買衣服。常二袖著銀子,一直奔到大街上來。看了幾家,都不中意。只買了領青杭絹女襖,一條綠紬裙子,月白雲紬衫兒,紅綾襖子兒,白紬子裙兒,共五件;自家也對身買了件鵝黃綾襖子,丁香色紬直身兒,又有幾件布草衣服。共用去六兩五錢銀子。打做一包,背著來到家中,教婦人打開看看。那婦人忙打開來瞧著,便問:「多少銀子買的?」常二道:「六兩五錢銀子買來。」婦人道:「雖沒的便宜,卻直這些銀子。」一面收拾箱籠放好,明日去買家活。當日婦人懽天喜地過了一日,埋怨的話都吊在東洋大海去了,不在話下。再表應伯爵和西門慶兩個,自打發常時節出門,依舊在廳上坐的。西門慶因說起:「我雖是個武職,恁的一個門面,京城內外也交結的許多官員。近日又拜在太師門下,那些通問的書柬,流水也是往來。我又不得細工夫,多不得料理。我一心要尋個先生們在屋裡,好教他寫寫,省些力氣也好;只沒個有才學的人。你看有時,便對我說。我須尋間空房與他住下,每年算還幾兩束脩與他養家。卻也要是你心腹之友便好。」伯爵道:「哥不說不知。你若要別樣卻有,要這個到難。怎的要這個到沒?第一要才學,第二就要人品了。又要好相處,沒些說是說非,翻唇弄舌,這就好了。若只是平平才學,又做慣搗鬼的,怎用的他?小弟只有祖父相處一個朋友生下來的孫子,他現是本州一個秀才。應舉過幾次,只不得中。他胸中才學,果然班馬之上。就是他人品,也孔孟之流。他和小弟通家兄弟,極有情分的。曾記他十年前應舉,兩道策,那一科試官極口贊他好。卻不想又有一個賽過他的,便不中了。後來連走了幾科不中,禁不的髮白鬢斑。如今他雖是飄零書劍,家裡也還有一百畝田,三四帶房子,整的潔淨住著。」西門慶道:「他家幾口兒也勾用了,卻怎的肯來人家坐館?」應伯爵道:「當先有的田房,都被那些大戶人家買去了。如今只剩得雙手皮哩!」西門慶道:「原來是賣過的田,算甚麼數!」伯爵道:「這果是算不的數了。只他一個渾家,年紀只好二十左右,生的十分美貌。又有兩個孩子纔三四歲。」西門慶道:「他家有了美貌渾家,那肯出來?」伯爵道:「喜的是兩年前,渾家專要偷漢,跟了個人上東京去了。兩個孩子,又出痘死了。如今止存他一口,定然肯出來。」西門慶笑道:「恁地說的他好,都是鬼混!你且說他姓甚麼?」伯爵道:「姓水。他才學果然無比,哥若用他時,管情書柬、詩詞、歌賦,一件件增上哥的光輝哩。人看了時,都道西門大官恁地才學哩!」西門慶道:「你纔說這兩樁,都是吊慌。我卻不信你的吊慌。你有記的他些書柬兒,念來我聽。看好時,我便請他來家,撥間房住下。只一口兒,也好看承的。尋個好日子,便請他也罷。」伯爵道:「曾記得他稍書來,要我替他尋個主兒。這一封書,略記的幾句,念與哥聽〔黃鶯兒〕:

『書寄應哥前,別來思,不待言。滿門兒托賴都康健。舍字在邊傍立著官,有時一定求方便。羨如椽,往來言疏,落筆起雲煙。』

西門慶聽畢,呵呵大笑將起來道:「他滿心正經,要你和他尋個主子,卻怎的不稍封書來。到寫著一隻曲兒?又做的不好,可知道他才學荒疎,人品散彈哩。」伯爵道:「這到不要作准他。只為他與我是三世之交。小弟兩三歲時節,他也纔勾四五歲。那時就同吃糖糕餅果之類,也沒些兒爭論。後來大家長大了,上學堂讀書寫字,先生也道:『應二學生子和水學生子一般的聰明伶俐,後來已定長進。』落後做文字,一樣同做,再沒些妒忌。日裡同行同坐,夜裡有時也同一處歇。到了戴網子,尚兀是相厚的。因此是一個人一般,極好兄弟。故此不拘形跡,便隨意寫個曲兒。我一見了,也有幾分著惱。後想一想,他自托相知,纔敢如此,就不惱罷了。況且那隻曲兒,也到做的有趣。哥卻看不出來。第一句說:『書寄應哥前』是啟口,就如人家寫某人見字一般,卻不好哩?第二句說:『別來思,不待言。』這是敘寒溫了。簡而文,又不好哩?第三句是:『滿門兒托賴都康健』這是說他家沒事故了。後來一發好的緊了!」西門慶道:「第五句是甚麼說話?」伯爵道:「哥不知道,這正是拆白道字,尤人所難。『舍』字在邊旁,立著『官』字,不是個『館』字?若有館時,千萬要舉荐。因此說『有時定要求方便。』『羡如椽』,他說自家一筆如椽。做人家往來的書疏,筆兒落下去,其煙滿紙,因此說:『落筆起雲煙。』哥,你看他詞裡,有一個字兒是閑話麼?只這幾句,穩穩把心窩里事都寫在紙上,可不好哩!」西門慶被伯爵說了他恁地好處,到沒的說了。只得對伯爵道:「你既說他許多好處,且問你有正經的書札,拏些我看看,我就請了他。」伯爵道:「他做的詞賦也有在我處,只是不曾帶得來哥看。我還記的他一篇文字,做得甚好。就念與哥聽著:

『一戴頭巾心甚懽,豈知今日誤儒冠。別人戴你三五載,偏戀我頭三十年。要戴烏紗求閣下,做篇詩句別尊前。此番非是吾情薄,白髮臨期太不堪!今秋若不登高第,踹碎冤家學種田。』

『維歲在大比之期,時到揭曉之候。訴我心事,告汝頭巾。為你青雲利器望榮身,誰知今日白髮盈頭戀故人。嗟乎!憶我初戴頭巾,青青子襟;承汝枉顧,昂昂氣忻。既不許我少年早發,又不許我久屈待伸。上無公卿大夫之職,下非農工商賈之民。年年居白屋,日日走黌門。宗師案臨,膽怯心驚。上司迎接,東走西奔。思量為你,一世驚驚嚇嚇,受了若干辛苦。一年四季,零零碎碎,被人賴了多少束修銀。告狀助貧,分穀五斗,祭下領支肉半斤。官府見了,不覺怒嗔;早快通稱,盡道廣文。東京路上,陪人幾次;兩齋學霸,惟吾獨尊。你看我兩隻皁靴穿到底,一領藍衫剩布筋。埋頭有年,說不盡艱難悽楚;出身何日,空歷過冷淡酸辛。賺盡英雄,一生不得文章力;未沾恩命,數載猶懷霄漢心。嗟乎!哀哉!哀此頭巾!看他形狀,其實可衿。後直前橫,你是何物?七穿八洞,真是禍根。嗚呼!沖霄鳥兮未垂翅,化龍魚兮已失鱗。豈不聞久不飛兮一飛登雲;久不鳴兮一鳴驚人。早求你脫胎換骨,非是我棄舊憐新。斯文名器,想是通神。從茲長別,方感洪恩。短詞薄奠,庶其來歆。理極數窮,不勝具懇。就此拜別,早早請行。』」

伯爵念罷,西門慶拍手大笑道:「應二哥,把這樣才學就做了班揚了。」伯爵道:「他人品比才學又高,如今且說他人品罷。」西門慶道:「你且說來。」伯爵道:「前年他在一個李侍郎府里坐館。那李家有幾十個丫頭,一個個都是美貌俊俏的。又有幾個伏侍的小廝,也一個個都標致龍陽的。那水秀才連住了四五年,再不起一些邪念。後來不想被幾個壞事的丫頭小廝,見是一個聖人一般,歹去日夜括他。那水秀才又極好慈悲的人,便口軟勾搭上了。因此被主人逐出門來,閧動街坊,人人都說他無行。其實水秀才原是坐懷不亂的。若哥請他來家,憑你許多丫頭小廝同眠同宿,你看水秀才亂麼?再不亂的。」西門慶道:「他既前番被主人趕了出門,一定有些不停當哩。二哥雖與我相厚,那樁事不敢領教。前日敝僚友倪桂岩老先生曾說他有個姓溫的秀才。且待他來時再處。」

畢竟未知如何,且聽下回分解:




\end{showcontents}


