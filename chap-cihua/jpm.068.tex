%# -*- coding: utf-8 -*-
%!TEX encoding = UTF-8 Unicode
%!TEX TS-program = xelatex
% vim:ts=4:sw=4
%
% 以上设定默认使用 XeLaTex 编译,并指定 Unicode 编码,供 TeXShop 自动识别

%第六十八回 
\chapter{鄭月兒賣俏透密意\KG 玳安慇懃尋文嫂}


「雪壓殘紅一夜凋,  曉來簾外正飄飄,

數枝翠葉空相對,  萬片香魂不可招;

長樂夢回春寂寂,  武陵人去水迢迢,

欲將玉笛傳遺恨,  若被東風透綺寮。」

話說西門慶與李瓶兒燒布畢,歸潘金蓮房中歇了一夜。到次日,先是應伯爵家送喜麵來,落後黃四領他小舅子孫文相宰了一口豬,一壜酒,兩隻燒鵝,四隻燒雞,兩盒菓子,來與西門慶磕頭。西門慶再三不受,黃四打旋磨兒跪著說:「蒙老爹活命之恩,救出孫文相來,舉家感激不淺。今無甚孝順,些微薄禮,與老爹賞人罷了,如何不受?」推阻了半日,西門慶止受豬酒:「留下送你錢老爹,也是一樣。」黃四道:「既是如此,難為小人一點窮心,無處所盡,只得把羹菓抬回去。又請問老爹,幾時閑暇?小人問了應二叔,裡邊請老爹坐坐。」西門慶道:「你休聽他哄你哩,又費煩你,不如不了。」那黃四和他小舅子,千恩萬謝出門。這裡西門慶賞拾盒錢,打發去訖。到十一月初一日,西門慶往衙門中回來,又往李知縣衙內吃酒去。月娘獨自一人,素粧打扮,坐轎子往喬大戶家,與長姐做生日,都不在家。到後晌,有庵裡薛姑子,聽見月娘許下他到初五日李瓶兒斷七,教他請八眾尼僧來家念經,拜血盆懺。于是悄悄瞞著王姑子,買了兩盒禮物來見月娘。月娘不在家,李嬌兒、孟玉樓留下他,陪他吃茶,說:「大姐姐不在家,往喬親家與長姐做生日去了。你須等他來見他,他還和你說話,好與你寫法銀子。」那薛姑子就坐住了。潘金蓮因想著玉蕭告他說,月娘吃了他的符水藥,纔坐了胎氣。自從李瓶兒死了,又見西門慶在他屋裡,把奶子也要了。恐怕一時奶子養出孩子來,攙奪了他寵愛。于是把薛姑子讓到前邊他房裡無人處,悄悄央薛姑子,與他一兩銀子,替他配坐胎氣符藥吃,尋頭男衣胞,不在話下。到晚夕等的月娘來家,留他住了一夜。次日,問西門慶討了五兩銀子經錢寫法與他。這薛姑子就瞞著王姑子,大師父,不和他說。到初五日,早請八眾女僧,在花園捲棚內建立道場,各門上貼歡門吊子,諷誦華嚴金剛經呪,禮拜血盆寶懺,酒花米,轉念三十五佛明經。晚夕設放焰口施食。那日請了吳大妗子、花大嫂,官客吳大舅、應伯爵、溫秀才吃齋。尼僧也不打動法器,只是敲木魚,擊手磐念經而已。那日伯爵領了黃四家人,具帖,初七日在院中和愛月兒家置酒,請西門慶。門慶見帖兒笑了,說:「我初七日不得閒,張西材家吃生日酒,倒是明日空閒。」問:「還有誰?」伯爵道:「再沒人,只請了我、李三哥相陪。又費事叫了四個女兒唱西廂記。」西門慶分付與黃四家人齋吃了,打發回去。伯爵便問:「黃四那日,買了分甚麼禮來謝你?」西門慶如此這般:「我不受他的,再三磕頭禮拜,我只受了猪、酒、添了兩疋白鷴紵絲、兩疋京段、五十兩銀子,謝了錢龍野先生。」伯爵道:「哥,你不接錢儘勾了,這個是你落得的。少說四疋尺頭值三十兩銀子,那二十兩那裡尋這分上去?便益了他,救了他父子二人性命!」當日坐至晚夕散。西門慶向伯爵說:「你明日還到這邊。」伯爵說:「我知道。」作別去了。八眾尼僧,直亂到一更天時分,方纔道場圓滿,焚燒箱庫散了。至次日,西門慶早往衙門中去了。且說王姑子打聽得知,大清早辰,走來西門慶家,說薛姑子攬了經去,要經錢。月娘怪他:「你怎的昨日不來?他說你往王皇親家做生日去了。」王姑子道:「這個就是薛家老淫婦的鬼。他對著我說,咱家挪了日子,到初六念經。經錢他多拿的去了,一些兒不留下。」月娘道:「這咱裡未曾念經,經錢寫法,都找完了與他了。早是我還與你留下一疋襯錢布在此。」教小玉連忙擺了些昨日剩下的齋食,與他吃了。把與他一疋藍布。這王子姑口裡喃喃吶吶罵道:「我教這老淫婦獨吃,他印造經,轉了六娘許多銀子。原說這個經兒咱兩個使,你又獨自掉攬的去了。」月娘道:「老薛說你接了六娘血盆經五兩銀子,你怎的不替他念?」王姑子道:「他老人家五七時,我在家請了四個師父,念了半個月哩。」月娘道:「你念了,怎的挂口兒不對我題?你就對我說,我還送些襯施兒與你。」那王姑子便一聲兒不言語,訕訕的坐了一回,往薛姑子嚷去了。看官聽說:似這樣緇流之輩,最不該招惹他。臉雖是尼姑臉,心同淫婦心。只是他六根未淨,本性欠明,戒行全無,廉恥已喪。假以慈悲為主,一味利慾是貪。不管墮業輪迴,一味眼下快樂。哄了些小門閨怨女,念了些大戶動情妻。前門接施主檀那,後門丟胎卵濕化。姻緣成好事,到此會佳期。有詩為證:

「佛會僧尼是一家,  法輪常轉度龍華,

此物只好圖生育,  枉使金刀剪落花。」

卻說西門慶從衙門中回來,吃了飯,應伯爵又早到了,盔的新段帽,沉香色〈衤旋〉褶,粉底皂靴,向西門慶聲喏,說:「這天也有晌午,咱也好去了。他那裡使人邀了好幾遍了,休要難為人家。」西門慶道:「咱今邀葵軒走走。」使王經:「往對過請你溫師父來。」王經去不多時,回說:「溫師父不在家,望朋友去了。畫童兒請去了。」伯爵便說:「咱等不的,他秀才家,赤道有要沒緊望朋友,多咱來?倒沒的誤了勾當。」西門慶分付琴童:「備黃馬與應二爹騎。」伯爵道:「我不騎。你依我,省的搖鈴打鼓。我先走一步兒,你坐轎子慢慢來就是了。」西門慶道:「你說的是,你先行罷。」那伯爵舉手先走了。西門慶分付玳安、琴童、四個排軍,收拾下暖轎跟隨。纔待出門,忽平安兒慌慌張張從外拿著雙帖兒來報說:「工部安老爹來拜。先差了個吏送帖兒,後邊走著便來也。」慌的西門慶分付家中廚下備飯,使來興兒買攢盤點心伺候。良久,安郎中來,跟從許多人。西門慶冠冕出來迎接。安郎中穿著粧花雲鷺補子員領,起花萌金帶。進門拜畢,分賓主坐定,左右拿茶上來。茶罷,敘其間闊之情。西門慶道:「老先生榮擢,失賀,心甚缺然。前日蒙賜華扎厚儀,生正值喪事匆匆,未及奉候起居為歉。」安郎中道:「學生有失吊問,罪罪。生到京也曾道達雲峯,未知可有禮到否?」西門慶道:「正是,又承翟親家遠勞致賻。」安郎中道:「四泉已定今歲,恭喜在即。」西門慶道:「在下才微任小。豈敢過于非望?」又說:「老先生此今榮擢美差,足展雄才大略。河治之功,天下所仰。」安郎中道:「蒙四泉過譽。一介寒儒,叨承科甲,處在下僚。若非蔡老先生抬舉,備員冬曹,謬典水利,奔走湖湘之間,一年以來,王事匆匆,不暇安跡。今又承命修理河道,況此民窮財盡之時。前者皇船載運花石,毀閘折壩,所過倒懸,公私困弊之極。而今瓜州、南旺、沽頭、魚臺、徐、沛、呂梁、安陵、濟寧、宿遷、臨清、新河一帶,皆毀壞廢,北南河南陡,淤沙無水。八府之民,皆疲弊之甚。又兼賊盜梗阻,財用匱乏,大覃神輸鬼沒之才,亦無如之何矣!」西門慶道:「老先生自有才猷展布,不日就緒,必大陞擢矣。」因問:「老先生,勅書上有期限否?」安郎中道:「三年欽限。河工完畢,聖上還要差官來祭謝河神。」說話中間,西門慶令放卓兒。安郎中道:「學生實告,還要往黃泰宇那裡拜拜去。」西門慶道:「既如此,少坐片時,教跟從者吃些點心。」不一時,放了卓,就是春盛案酒,一色十六碗,多是頓爛下飯、雞蹄、鵝鴨、鮮魚、羊頭、肚肺、血臟、鮓湯之類。純白上新軟稻粳飯 。用銀廂甌兒盛著,裡面沙糖、榛松、瓜仁拌著飯。又小金鍾暖斟來釀,下人俱有攢盤點心酒肉。安郎中席間,只吃了三鍾,就告辭起身,說:「學生容日再來請教。」西門慶款留不住,送至大門首,上轎而去。回到聽上,解去了冠帶,換了巾幘,止穿紫絨獅補直身,使人問:「溫師父來了不曾?」玳安回說:「溫師父未回家哩,有鄭春和黃四叔家來定兒來邀,在這裡半日了。」西門慶即出門上轎,左右跟隨,逕往院中鄭愛月兒家來。比及進院門,架兒門頭都躲過一邊,只該日俳長兩邊站立,不敢跪接。鄭春與來定兒先通報去了。應伯爵正和李三打雙陸,聽見西門慶來,連忙收拾不及。鄭愛月兒、愛香兒,戴著海獺臥兔兒,一窩來杭州攢翠重梅鈿,見油頭粉面,打扮的花仙也似的,都出來門首迎接。西門慶下了轎,進入客位內。西門慶分付不消吹打,止住鼓樂。先是李三、黃四見畢禮數,然後鄭家鴇子出來拜見了,纔是愛月兒姊妹兩個,插燭也似磕了頭。正面安設兩張交椅,西門慶與應伯爵坐下。李智、黃四與鄭家姊妹兩個打橫,玳安在傍稟問:「轎子在這裡?回了家去?」西門慶令排軍和轎子多回去。分付琴童:「到家看你溫師父家裡來了,拿黃馬接了來。」琴童應喏去了。伯爵因問:「哥,怎的這半日纔來?」西門慶悉把工部安郎中拜留飯之事,說了一遍。須臾,鄭春拿茶上來。愛香兒拿了一盞與伯爵,愛月兒便遞西門慶。那伯爵連忙用手去接,說:「我錯接,只說你遞與我來。」愛月兒道:「我遞與你?沒修這樣福來。」伯爵道:「你看這小淫婦兒,原來只認的他家漢子,倒把客人不著在意裡。」愛月兒笑道:「今日輪不著你做客人,還有客人來。」吃畢茶,收下盞托去。須臾,四個唱西廂,妓女多花枝招颱,綉帶飄飄出來,與西門慶磕頭,一一多問了名姓。西門慶對黃四說:「等住回上來唱,只打鼓兒,不吹打罷。」黃四道:「小人知道。」只見鴇子上來說:「只怕老爹害冷,教鄭春放下暖簾來。」火盆獸炭,頻加蘭麝香霞。只見幾個青衣圓社,聽見西門慶老爹進來在鄭家吃酒,走來門首伺候,探頭舒腦,不敢進去。有認的玳安兒,向玳安打恭,央及作成作成。玳安悄悄進來,替他稟問,被西門慶喝了一聲,諕的眾人一溜煙走了。不一時收拾菓品案酒上來,正面放兩張卓席。西門慶獨自一席,伯爵與溫秀才一席,留空著溫秀才坐位在左首。傍邊一席李三和黃四,右邊是他姊妹二人。端的盤堆異品,花插金瓶。鄭奉、鄭春在傍彈唱。纔遞酒安席坐下,只見溫秀才到了。頭戴過橋巾,身穿綠雲襖,腳穿雲履絨襪,進門作揖。伯爵道:「老先生何來遲也?留席久矣。」溫秀才道:「學生有罪,不知老先生呼喚。適往敝同窗處會書,來遲了一步。」慌的黃四一面安放鍾筯,與伯爵一處坐下。不一時,湯飯上來,黃芽韮燒賣 ,八寶攢湯 ,薑醋碟兒。兩個小優兒彈唱一回下去。端的酒斟綠蟻,詞歌金縷。四個妓女纔上來唱了二摺游藝中原。只見玳安來說:「後邊銀姨那裡,使了吳會和蠟梅送茶來了。」原來吳銀兒就鄭家後邊住,止隔一條巷。聽見西門慶在這裡吃酒,故使送茶。西門慶喚入裡面,吳惠、蠟梅先磕了頭,說:「銀姐使我送茶來與爹吃。」揭開盒兒,斟茶上去,每人一盞爪仁栗絲鹽笋芝麻玫瑰香茶。西門慶問:「銀兒在家做甚麼哩?」蠟梅道:「姐兒今日在家沒出門。」西門慶吃了茶,賞了他兩個三錢銀子。即令玳安同吳惠:「你快請銀姨去。」鄭愛月兒急俐,便就教春春:「你也跟了去,好歹纏了銀姨來。他若不來,你就說我到明日就不和他做夥計了。」應伯爵道:「我倒好笑,你兩個原來是販毛〈毛皮〉的夥計!」溫秀才道:「南老好不近人情。自古同聲相應,同氣相求;本乎天者親上,本乎地者親下。同他做夥計一般了。」愛月兒道:「應花子,你與鄭春他們多是夥計,當差供唱,都在一處。」伯爵道:「傻孩子,我是老王八,那咱和你媽相交,你還在肚子裡。」說笑中間,廚下割獻豕蹄一領,又是四碗下飯,羊蹄、黃芽、臊子韮、肚肺羹 、血臟之類。妓女上來唱了一套半萬賊兵。西門慶叫上唱鶯鶯的韓家女兒,近前問:「你是韓家的?」愛香兒說:「爹,你不認的,他是韓金釧侄女兒,小名消愁兒,今年纔十三歲。」西門慶道:「這孩子到明日成個好婦人兒!舉止伶俐,又唱的好。」因令他上席遞酒。黃四下湯下飯,極盡慇懃。不一時,吳銀兒來到。頭上戴著白縐紗髮髻、珠子箍兒、翠雲鈿兒,周圍撇一溜小簪兒,耳邊戴著金丁香兒。上穿白綾對衿兒粧花眉子。下著紗綠潞紬裙,羊皮金滾邊。腳上墨青素段雲頭鞋兒。笑嘻嘻進門,向西門慶磕了頭,後與溫秀才等各多道了萬福。伯爵道:「我倒好笑了,來到就教我惹氣。俺每是後娘養的,只認的你爹,與他磕頭,望著俺每擩一拜。原來你這麗春院小娘兒,這等欺客。我若有五棍兒衙門,定不饒你!」愛月兒叫:「應花子,好沒羞的孩兒!那裡哥兒,你行頭不仔麼,光一味好撇。」一面安座兒讓銀姐坐。就在西門慶卓邊坐下,連忙放鍾筯。西門慶見了戴著白䯼髻,問:「你戴的誰人孝?」吳銀兒道:「爹故意又問個兒,與娘戴孝一向了。」西門慶一聞與李瓶兒戴孝,不覺滿心歡喜,與他側席而坐,兩個說話。須臾,湯飯上來,愛月兒下來與他遞酒。吳銀兒下席,說:「我還沒見鄭媽哩。」一面走到鴇子房內見了禮出來。鴇子叫:「月兒讓銀姐坐,只怕冷,教丫頭燒個火籠兒與銀姐烤手兒。」隨即添換熱菜,打發上來。吳銀兒在傍,只吃了半個點心,呵了兩口湯,放下筯兒,和西門慶攀話。因拿起鍾兒來說:「爹,這酒寒些,從新折了,另換上暖酒。」鄭春上來把伯爵眾人等酒都斟上,行過一巡。吳銀兒便問:「娘前日斷七念經來?」西門慶道:「五七多謝你每茶。」吳銀姐道:「好說,俺每送了些粗茶,倒教爹又把人情回了,又多謝重禮,教媽惶恐要不的。昨日娘斷七,我會下月姐和桂姐,也要送茶來,又不知宅內念經不念?」西門慶道:「斷七那日,胡亂請了幾眾女僧,在家拜了拜懺,親眷一個都沒請,恐怕費煩。」飲酒說話之間,吳銀兒又問:「家中大娘、眾娘每多好?」西門慶道:「都好。」吳銀兒道:「爹乍沒了娘,到房裡孤孤兒的,心中也想?」西門慶道:「想是不消說。前日在書房中,白日夢見他,哭的我要不的。」吳銀兒道:「熱突突沒了,可知想哩。」伯爵道:「你每說的只情說,把俺每這裡只顧旱著。不說來遞鍾酒,也唱個兒與俺聽。俺每起身去罷。」慌的李三、黃四連忙攛掇他姐兒兩個上來遞酒,安下樂器,吳銀兒也上來。三個粉頭一般兒坐在席傍,躧著火盆,合著聲音,啟朱唇,露皓齒,詞出佳人口,唱了套中呂粉蝶兒三弄梅花,端的有裂石流雲之響。唱畢,西門慶向伯爵說:「你落索他姐兒三個唱,你也下來酬他一杯兒。」伯爵道:「不打緊,死不了人。等我打發他仰靠著,直舒著,側臥著,金雞獨立,隨我受用。又一件,野馬踩場,野狐抽絲,猿猴獻菓,黃狗溺尿,仙人指路,靠背將軍,柱夜對木,伴哥隨他揀著要。」愛香道:「我不好罵出來的,汗邪了你這賊花子,胡說亂道的!」這應伯爵用酒碟安三個鍾兒,說:「我兒,你們在我手裡吃兩鍾;不吃,望身上只一潑。」愛香道:「我今日忌酒。」愛月兒道:「你跪著月姨兒,教我打個嘴巴兒,我纔吃。」伯爵道:「銀姐,你怎的說?」吳銀兒道:「二爹,我今日心內不自在,吃半盞兒罷。」那愛月兒道:「花子,你不跪,我一百年也不吃。」黃四道:「二爺,你不跪,顯的不是趣人;也罷,跪著不打罷。」愛月兒道:「不,他只教我打兩個嘴巴兒,我方吃這鍾酒兒。」伯爵道:「溫老先兒這裡看著,怪小淫婦兒,只顧趕盡殺絕!」于是奈何不過,真個直撅兒跪在地下。那愛月兒輕揎彩袖,款露春纖,罵道:「賊花子,再敢無禮傷犯月姨?再不敢;高聲兒答應,你不答應,我也不吃。」那伯爵無法可處,只得應聲道:「再不敢傷犯月姨了。」這愛月兒一連打了兩個嘴巴,方纔吃那杯酒。伯爵起來道:「好個沒仁義的小淫婦兒,你也剩一口兒我吃。把一鍾酒都吃的淨淨兒的!」愛月兒道:「你跪下,等我賞你一鍾酒。」于是滿滿斟上一杯,笑望伯爵口裡只一灌,伯爵道:「怪小淫婦兒,使促挾灌撒了我一身酒。我老道只這件衣服,新穿了纔頭一日兒,就污濁了我的。我問你家漢子要!」亂了一回,各歸席上坐定。看看天色,掌燭上來。下飯添換,都已上完。下邊玳安、琴童、畫童、應寶都在鴇子房裡放卓兒,有湯飯點心酒餚管待。須臾,拿上各樣菓碟兒來,那伯爵推讓溫秀才,只顧不住手拈放在口裡,一壁又往袖中褪。西門慶分付拿個骰盆兒來,先讓溫秀才。秀才道:「豈有此理?還從老先兒那邊來。」于是西門慶與吳銀兒,用十二個骰兒搶紅。下邊四個妓女,拿樂器彈唱叫呀,酒飲過一巡。吳銀兒卻轉過來與溫秀才、伯爵搶紅。愛香兒卻來西門慶席上遞酒猜枚,須臾過去。愛月兒近前與西門慶搶紅。吳銀兒都往下席遞李三、黃四。原來愛月兒旋往房中新粧打扮出來,上著烟裡火〈廴回〉紋錦對衿襖兒,鵝黃杭絹點翠縷金裙,粧花膝褲,大紅鳳嘴鞋兒。燈下海獺臥兔兒,越顯的粉濃濃雪白的臉兒,猶賽美人兒一般。但見:

「芳姿麗質更妖嬈,  秋水精神瑞雪標,

鳳目半彎藏琥珀,  朱唇一夥點櫻桃;

露來玉笋纖纖細,  行步金蓮步步嬌,

白玉生香花解語,  千金良夜實難消。」

這西門慶一見,如何不愛?吃了幾鍾酒,半酣上來。因想著李瓶兒夢中之言:「少貪在外夜飲。」一面起身,後邊淨手。慌的鴇子連忙叫丫鬟點燈,引到後邊解手出來。愛月隨即也跟來伺候,盆中淨手畢,拉著他手兒同到房中。房中又早月窗半啟,銀燭高燒,氣暖如春,蘭麝馥郁。牀畔則斗帳雲橫,鮫綃霧設。于是脫了上蓋,底下白綾道袍,兩個在牀上,腿壓腿兒做一處。先是愛月兒問:「爹,今日不家去罷了。」西門慶道:「我還去。今日一者銀兒在這裡,不好意思;二者我居著官,今年考察在邇,恐惹是非,只是白日來和你坐坐罷了。」又說:「前日多謝你泡螺兒,你送了去,倒惹的我心酸了半日。當初有世六娘他會揀;他死了,家中再有誰會揀他!」愛月道:「揀他不難,只是要拿的著禁節兒便好。那日我胡亂整治了不多兒,知道爹好吃,教鄭春送來。那瓜仁都是我口裡一個個兒磕的,汗巾兒是我閑著用工夫撮的穗子。瓜仁子,說應花子倒撾了好些吃了。」西門慶道:「你問那訕臉花子頭,我見他早時兩把撾去,喃了好些,只剩下不多些我吃了。」愛月兒道:「倒便益了賊花子,恰好只孝順了他。」又說:「多謝爹的衣梅,媽看見吃了一個兒,喜歡的要不的。他要便痰火發了,晚夕咳嗽,半夜把人聒死了。常時口乾,得恁一個在口內噙著,他倒生好些津液。我和俺姐姐吃了沒多幾個兒,連罐兒他老人家都收了在房內,早晚吃,誰敢動他?」西門慶道:「不打緊,我明日使小廝再送一罐來你吃。」又問:「爹連日會桂姐來沒有?」西門慶道:「自從孝堂裡到如今,誰見他來?」愛月兒道:「六娘五七,他也送茶去來?」西門慶道:「他家使李銘送去來。」愛月道:「我有句兒,只放在爹心裡。」西門慶問:「甚麼話?」那愛月又想了想,說:「我不說罷。若說了,顯得姊妹們恰似我背地說他一般,不好意思的。」西門慶一面摟著他脖子,說:「怪小油嘴兒,甚麼話?說與我,不顯出你來就是了。」兩個正說得入港,猛然應伯爵走入來,大叫一聲:「你兩個好人兒,撇了俺每,走在這裡說梯己話兒。」愛月兒道:「噦!好個不得人意,怪訕臉花子。猛可走來,諕了人恁一跳!」西門慶罵道:「怪狗才,前邊去罷,丟的葵軒和銀姐在那裡,都往後頭來了。」這伯爵一屁股坐在牀上,說:「你拿肐膊來,我且咬口兒我纔去。你兩個在這裡儘著{入日}搗。」于是不由分說,向愛月兒袖口邊,勒出那賽鵝脂雪白的手腕兒來,帶著銀鐲子,猶若美玉,尖溜溜十指春葱手。上籠著金戒指兒,誇道:「我兒,你這兩隻手兒,天生下就是發{髟巳}{髟己}的肥一般。」愛月兒道:「怪刀攘的,我不好罵出來的!」被伯爵拉過來,咬了一口,走了。咬的老婆怪叫,罵:「怪花子,平白進來鬼混人死了!」便叫:「桃花兒,你看他出去了,把籠道子門關一面關上門。」愛月便把李桂姐如今又和王三官兒子女一節,說與西門慶:「怎的有孫寡嘴、祝麻子、小張閑,架兒于寬、孫錫鉞,踢行頭白回子、沙三,日逐嫖著在他家行走。如今丟開齊香兒,又和王家玉芝兒打熱。兩下裡使錢使沒了包了皮祅,當了三十兩銀子,拿著他娘子兒一副金鐲子,放在李桂姐家,算了一個月歇錢。」西門慶聽了,口中罵道:「恁小淫婦兒,我分付和這小廝纏,他不聽,還對著我賭身發呪,恰好只哄我。」愛月兒道:「爹也別要惱。我說與爹個門路兒,管情教王官打了嘴,替爹出氣。」西門慶把他摟在懷裡,用白綾袖子兜著他粉項,搵著他香腮,他便一手拿著銅絲火籠兒,內燒著沉速香餅兒,將袖口籠著燻熱身上,便道:「我說與爹,休教一人知道。就是應花子也休望他題,只怕走了風。」西門慶問:「我的兒,你告我說,我傻了,肯教人知道。端的甚門路兒?」鄭愛月悉把:「王三官娘林太太,今年不上四十歲,生的好不喬樣,描眉畫眼,打扮狐狸也似。他兒子鎮日在院裡,他專在家只送外賣,假托在個姑姑庵兒打齋。但去就他說媒的文嫂兒家落腳。文嫂兒單管與他做牽兒,只說好風月。我說與爹,到明日遇他遇兒也不難。又一個巧宗兒,王三官兒娘子兒,今纔十九歲,是東京六黃太尉姪女兒,上畫般標致,雙陸棋子都會,三官常不在家,他如同守寡一般,好不氣生氣死。為他也上了兩三遭吊,救下來了。爹難得先刮刺上了他娘,不愁媳婦兒不是你的。」當下被他一席話,說的西門慶心邪意亂,摟著粉頭說:「我的親親,我又問你怎的曉的就裡?」這愛月兒就不說常在他家唱,只說我一個熟人兒,如此這般和他娘在其處會過一遍,也是文嫂兒說合。西門慶問:「那人是誰?莫不是大街坊張大戶姪兒張二官兒?」愛月兒道:「那張懋德兒好{入日}的貨!麻著七八個臉彈子,密縫兩個眼,可不砢硶殺我罷了!只好樊家百家奴兒接他,一向董金兒也與他丁八了。」西門慶道:「我猜不著,端的是誰?」愛月兒道:「教爹得知了罷。是原梳籠我的那個南人。他一年來此做買賣兩遭。正經他在裡邊歇不的一兩夜,倒只在外邊常和人家偷貓遞狗,幹此勾當。」這西門慶聽了,見粉頭所事,合著他的板眼,亦發歡喜,說:「我兒,你既貼戀我心,每日我送三十兩銀子與你媽盤纏,也不消接人了,我遇閒就來。」愛月兒道:「爹,你有我心時,甚麼三十兩二十兩,兩日間掠幾兩銀好與媽,我自恁懶待留人,只是伺候爹罷了。」西門慶道:「甚麼話!我決然送三十兩銀子來。」說畢,兩個上牀交歡,牀上鋪的被褥約一尺高,愛月道:「爹脫衣裳不脫?」西門慶道:「咱連衣耍耍罷,只怕他們前邊等咱。」一面扯過夏枕來,粉頭解去下衣,仰臥枕畔,裡面穿著紅潞紬底衣,褪下一隻膝褲腿來。西門慶把他兩隻小小金蓮扛在肩頭上,解開前藍綾褲子,那話使上托子,但見花心輕折,柳腰款擺。正是:

「花嫩不禁揉,春風卒未休。花心猶未足,脉脉情無那。低低喚粉郎,春宵樂未央。」

那當下兩個至精欲洩之際,西門慶幹的氣喘吁吁,粉頭嬌聲不絕,鬢雲拖枕,滿口只教道:「親達達,慢著些兒。」良久,樂極情濃,一泄如注。雲收雨散,各整衣裙,于燈下照鏡理容。西門慶在牀前盆中淨手,著上衣服,兩個携手來到席上。吳銀兒便守著,對愛香兒挨近,葵軒正擲色猜枚,觥籌交錯,要在熱鬧處。眾人見西門慶進入,多立起身來讓坐。伯爵道:「你也一般的把俺每去在這裡,你纔出來。拿酒兒,且扶扶頭著。」西門慶道:「俺每說句話兒,有甚這閑勾當?」伯爵道:「好話,你兩個原來說梯己話兒!」當下伯爵拿大鍾斟上暖酒,眾人陪西門慶吃,四個妓女拿樂器彈唱。玳安在傍掩口說道:「轎子來了。」西門慶弩了個嘴兒與他,那玳安連忙分付排軍打起燈籠,外邊伺候。這西門慶也不坐,陪眾人執杯立飲。分付四個妓女:「你再唱個一見嬌羞我聽。」那韓愁消兒:「俺每會唱。」于是拿起琵琶來,款放嬌聲,拿腔唱道:

「一見嬌羞,雨意雲情,我見他千嬌百媚,萬種妖嬈,一捻溫柔。通書先把話兒勾,傳情暗裡秋波溜。記在心頭,心頭未審,向時成就?」

唱了一個詞兒,吳銀兒遞西門慶酒。鄭香兒便遞伯爵。愛兒奉溫秀才。李智、黃四都斟上。又唱道:

「過爾丫鬟,欲鑄黃金拜將壇。莫通明曉寄與書生,雲雨巫山。重門今夜未曾拴,深閨特把情郎盼夜靜更闌,更闌!偷花妙手今番難按。」

吃畢,西門慶令再斟上,鄭香兒上來遞西門慶,吳銀兒遞溫秀才,愛月兒遞伯爵。鄭春在傍捧著菓菜兒。又唱道:

「夢入高堂,相會風流窈窕娘。我與他同携素手,共入羅幃,永結鸞鳳。靈犀一點透膏肓,鮫綃帳底翻紅浪。粉汗凝香,凝香!今宵一刻,人間天上。」

唱畢又叫呀酒。愛月兒卻轉過捧西門慶酒,吳銀兒遞溫秀才,并李三、黃四,從新斟酒。又唱第四個:

「春暖芙蓉,鬢亂釵橫寶髻鬆。我為他香嬌玉軟,燕侶鶯儔,意美情濃。腰肢無力眼矇朧,深情自把眉兒縱。兩意相同,相同!百年恩愛,和偕鸞鳳。」

唱畢,都飲過,西門慶起身。一面令玳安向書袋內取出大小十一包賞賜來。四個妓女,每人三錢,叫上廚役賞了五錢。吳惠、鄭奉、鄭春,每人三錢,攛掇打茶的,每人二錢。丫頭桃花兒,也與了他三錢。俱磕頭謝了。黃四再三不肯放,道:「應二叔,你老人家說聲,天還早哩。老爹大坐坐,也盡小人之情。如何就要起身?我的月姨兒,你也留留兒!」愛月兒道:「我留他,他白不肯坐。」西門慶道:「你每不知,我明日還有事。」一面向黃四、李三作揖,道:「生受打攪。黃四道:「惶恐!沒的請老爹來受餓。又不肯久坐,還是小人沒敬心。」說著,三個唱的都磕頭,說道:「爹到家,多頂上大娘和眾娘們,俺每閑了,會了銀姐,往宅內看看大娘去。」西門慶道:「你每閒了,去坐上一日來。」一面掌起燈籠,西門慶下臺基,鄭家鴇子迎著道萬福,說道:「老爹,大坐回兒,慌的就起身,嫌俺家東西不美口?還有一道米飯兒未曾上哩。」西門慶道:「勾了。我不是還坐回兒,許多事在身上。明日還要起早,衙門中有勾當。教應二哥他沒事,教他大坐回兒罷。」那伯爵就要跟著起來,被黃四死力攔住,說道:「我的二爺,你若去了,就沒趣死了。」伯爵道:「不是,你休攔我。你把溫老先生有本事留下,我就算你好漢!」那溫秀才奪門就走,被黃家小廝來定兒攔腰抱住。西門慶到了大門首,因問琴童兒:「溫師父有頭口在這裡沒有?」琴童道:「備了驢子在此,畫童兒看著哩。」西門慶向溫秀才道:「既有頭口,也罷,老先兒你陪應二哥再坐坐,我先去罷。」于是多送出門來。那鄭月兒拉著西門手兒,悄悄捏了一把,臉上轉,一徑揚聲說道:「我頭裡說的話,爹你在心些,知道了,法不待六耳。」西門慶道:「知道了。」又道:「鄭春,你送老爹到家,多上覆娘們。」那吳銀兒也說多上覆大娘。伯爵道:「我不好說的,賊小淫婦兒們,都攙行奪市的稍上覆;偏我就沒個人兒上覆。」愛月道:「你這花子過一邊兒!」那吳銀兒就在門首作辭了眾人并鄭家姐兒兩個,吳惠打著燈回家去了。鄭月兒便叫:「銀姐,見了那個流人兒,好歹休要說。」吳銀兒道:「我知道。」眾人回至席上,重添獸炭,再泛流霞。歌舞吹彈,歡娛樂飲,直耍了三更方散。黃四擺了這席酒,也與了他十兩銀子。西門慶賞賜了三四兩,俱不在話下。當日西門慶坐轎子,兩個排軍打著燈,逕出院門,打發鄭春回家。一宿晚景題過。到次日,夏提刑差答應的,來請西門慶早往衙門中審問賊情等事,直問到晌午吃了飯,早是沈姨夫差大官沈定,拿帖兒送了個後生來,在段子舖飯火頭,名喚劉包。西門慶留下了。正在書房中拿帖兒與沈定回家去了。只見玳安在傍邊站立,西門慶便問道:「溫師父昨日多咱來了?」玳安道:「小的舖子裡睡了好一回,只聽見畫童兒打對過門,那咱有三更時分纔來了。我今早辰問溫師父,倒沒酒,應二爹醉了,吐了一地。月姨恐怕夜深了,使鄭春送了他家去了。」西門慶聽了,呵呵笑了,因叫過玳安近前,說道:「舊時與你姐夫說媒的文嫂兒在那裡住?你尋了他來,對門房子裡見我,我和他說話。」玳安道:「小的不認的文嫂兒家,等我問了姐夫去。」西門慶道:「你吃了飯,問了他,快去。」玳安到後邊吃了飯,走到舖子裡問陳經濟。經濟道:「尋他做甚麼?」玳安道:「誰知他做甚麼?猛可教我找尋他去。」經濟道:「出了東大街,一直往南去,過了同仁橋牌坊,轉過往東,打王家巷進去,半中腰裡有個發放巡捕的廳兒,對門有個石橋兒。轉過石橋兒,緊靠著個姑姑庵兒,傍邊有個小衚衕兒,進小衚衕往西走,第三家豆腐舖隔壁上坡兒,有雙扇紅封門兒的,就是他家。你只叫文媽,他就出來答應你。」這玳安聽了說道:「再沒了?小爐匠跟著行香的走,鎖碎一浪湯。你再說一遍我聽,只怕我忘了。」那陳經濟又說了一遍。玳安道:「好近路兒,等我騎了馬去。」一面牽出大白馬來活,搭上替子,兜上嚼環,躧著馬臺,望上一騸,打了一鞭,那馬跑踍跳躍一直去了。出了東大街,逕往南過同仁橋牌坊,由王家巷進去。果然中間有個巡捕廳兒,對門就是座破石橋兒,裡首半戳紅墻,是大悲庵兒,往西是衚衕。北上坡挑著個豆腐牌兒,門首只見一個媽媽晒馬糞。玳安在馬上便問:「老媽媽,這裡有個說媒的文嫂兒?」那媽媽道:「這隔壁封門兒就是。」玳安到他門首,果然是兩扇紅封門兒,連忙跳下馬來,拿鞭兒敲著門兒叫道:「文媽在家不在?」只見他兒子文〈糸堂〉兒開了門,便問道:「是那裡來的?」玳安道:「我是縣門外提刑西門老爹來請,教文媽快去哩。」文〈糸堂〉聽見是提刑西門大官府家來的,便讓家裡坐。那玳安把馬拴住,進入裡面他明間內,見上面供養著利市布,有幾個人在那裡會中倚記罷,進香算帳哩。半日,拿了鍾茶出來,說道:「俺媽不在了。來家說了。明日早去罷。」玳安道:「驢子見在家裡,如何推不在?」側身逕往後走。不料文嫂和他媳婦兒,陪著幾個媽媽子正吃茶,躲不及,被他看見了。說道:「這個不是文媽?剛纔說回我不在家了,教我怎的回俺爹話?惹的不怪我。」文嫂笑哈哈與玳安道了個萬福,說道:「累哥哥你到家回聲兒,我今日家裡會茶。不知老爹呼喚我做什麼?我明日早往宅內去罷。」玳安道:「只分付我來尋你,誰知他做甚麼?原來不知你在這咭溜搭刺兒里住,教我抓尋了個不發心。」文嫂兒道:「他老人家這幾年宅內買使女、說媒、用花兒,自有老馮和薛嫂兒。王媽媽子走跳,希罕俺毋?今日忽刺入又冷鍋中荳兒爆,我猜見你六娘沒了,已定教我去替他打聽親事,要補你六娘的窩兒。」玳安道:「我不知道。你到那裡見了俺爹,他自有話和你說。」文嫂兒道:「哥哥你略坐坐兒,等我打發會茶人去了,同你去。」玳安道:「原來等你會茶?馬在外邊沒人看,俺爹在家緊等的火裡火發,分付又分付,教你快去哩。和你說了話,如今還要往府裡羅同知老爹吃酒去哩。」文嫂道:「也罷,等我拿點心吃了,同你去。」玳安道:「不吃罷。」因問:「你大姐生了孩兒沒有?」玳安道:「還不曾見哩。」這文嫂一面打發玳安吃了點心,穿上衣裳,說道:「你騎馬先行一步兒,我慢慢走。」玳安道:「你老人家放著驢子,怎不備上騎?」文嫂兒道:「我那討個驢子來?那驢子是隔壁豆腐店舖裡驢子,借俺院兒裡喂喂兒,你就當我的驢子?」玳安道:「我記得你老人家騎著匹驢兒來,往那去了?」文嫂兒道:「這咱哩,那一年吊死人家丫頭,打官司,為了場事,把舊房兒也賣了,且說驢子哩。」玳安道:「房子到不打緊處,且留著那驢子和你早晚做伴兒也罷了。別的罷了,我見他常時落下來,好個大鞭子。」那文嫂哈哈笑道:「怪猴兒,短壽命!老娘還只當好話兒,側著耳躲聽,你什麼好物件兒。幾年不見,你也學的恁油嘴滑舌的,到明日還教我尋親事哩。」玳安道:「我的馬走得快,你步行,赤道挨磨到多咱晚,惹的爹說。你上馬,咱兩個疊騎著罷!」文嫂兒道:「怪小短命兒,我又不是你影射的。街上人看著,怪刺刺的。」玳安道:「再不,你備豆腐舖子裡驢

子騎了去。到那裡等我打發他錢就是了。」文嫂兒道:「這等還好說。」一面教文〈糸堂〉將驢子備了,帶上眼紗,騎上。玳安與他同行,逕往西門慶宅中來。正是:

「欲向深閨永艷質,  全憑紅葉是良媒。」

有詩為證:

「誰信桃源有路通,  桃花含露笑春風,

桃源只在山溪裡,  今許漁郎去問津。」

畢竟未知後來如何,且聽下回分解:

