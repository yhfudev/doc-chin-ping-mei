%# -*- coding: utf-8 -*-
%!TEX encoding = UTF-8 Unicode
%!TEX TS-program = xelatex
% vim:ts=4:sw=4
%
% 以上设定默认使用 XeLaTex 编译,并指定 Unicode 编码,供 TeXShop 自动识别

%第四十八回 
\chapter{曾御史參劾提刑官\KG 蔡太師奏行七件事}

格言:

「知危識險,終無羅網之門;譽善薦賢,自有安身之地。施恩布德,乃後代之榮昌;懷妒藏奸,為終身之禍患。損人利己,終非遠大之圖;害眾成家,豈是長久之計?改名異體,皆因巧語而生;訟起傷財,蓋為不仁之召。」

話說安童領著書信,辭了黃通判,往山東大道而來。打聽巡按御史在東昌府察院住劄,姓曾,雙名孝序,乃都御史曾布之子,新中乙未科進士,極是個清廉正氣的官。這安童自思:「我若說下書的,門上人決不肯放。不如我在此等著放告牌出來,我跪門進去,連狀帶書呈上。老爹見了,必然有個決斷。」于是早已把狀子寫下,揣在懷裡,在察院門首,等候多時。只聽裡面打的雲板響,開了大門、二門,曾御史坐廳。頭面牌出來,大書:告親王皇親駙馬勢豪之家;第二面牌出來,告都布按并軍衛有司官吏;第三面牌出來,纔是百姓戶婚田土詞訟之事。這安童就隨狀牌進去。待把一應事情,發放淨了,方走在丹墀上跪下。兩邊左右,問:「是做甚麼的?」這安童方纔把書雙手舉得高高的呈上。只聽公座上曾御史叫:「接上來!」慌的左右吏典下來,把書接上去,安放于事案上。曾公拆開觀看,端的上面寫著甚言詞?書曰:

「寓都下年教生黃美端肅書,奉大柱史少亭曾年兄先生大人門下:違越光儀,倏忽一載,知己難逢,勝游易散。此心耿耿,常在左右,去秋忽報瑤章華扎,開軸啟函,捧誦之間,而神遊恍惚,儼然長安對面時也。每有感愴,輒一歌之,足舒懷抱矣!未幾,年兄省親南旋,復聞德音。知年兄按巡齊、魯,不勝欣慰,叩賀,叩賀!惟年兄忠孝大節,風霜貞操,砥礪其心,耿耿在廊廟,歷歷在士論。今茲出巡,正當摘發官邪,以正風紀之日。區區愛念,尤所不能忘者矣。竊謂年兄平日,抱可為之器,當有為之年;值聖明有道之世,老翁在家康健之時。不乘此大展才猷,以振揚法紀。勿使舞文之吏,以撓其法;而奸頑之徒,以逞其欺。胡乃如東平一府,而有撓大法如苗青者;抱大冤如苗天秀者乎!生不意聖明之世,而有此魍魎!年兄巡歷此方,正當分理冤滯,振刷為之一清可也。去伴安童,持狀告訴,幸垂察。不宜。仲春望後一日具。」

這曾御史覽書已畢,便問:「有狀沒有?」左右慌忙下來問道:「老爺問你有狀沒有?」這安童向懷中取狀遞上。曾公看了,取筆批:「仰東平府府官,從公查明,驗相屍首,連卷詳報。」喝令安童:「東平府伺候。」這安童連忙磕頭起來,從便門放出。這里曾公將批詞連狀裝帶封套內,鈐了關防,差人賫送東平府來。府伊胡師文見了上司批下來,慌得手腳無措。即調委陽谷縣縣丞狄斯彬,本貫河南舞陽人氏,為人剛而且方,不要錢,問事糊突,人都號他做狄混。明文下來,沿河查訪苗天秀屍首下落。也是合當有事,不想這狄縣函率領一行人,巡訪到清河縣城西河邊。正行之際,忽見馬頭前,起一陣旋風,團團不散,只隨著狄公馬走。狄縣丞道:「怪哉!」遂勒住馬,令左右公人:「你去隨此旋風,務要跟尋個下落。」那公人真個跟定旋風而來,七八將近新河口而止。走來回覆了狄公話。狄公即拘了里老來,用鍬掘開岸土深數尺,見一死屍,宛然頸上有一刀痕,命仵作檢視明白,問:「前面是那裡?」公人稟道:「離此不遠,就是慈惠寺。」縣丞即令拘寺中僧行問之。皆言:「去冬十月中,本寺因放水燈兒,見一死屍從上流而來,漂入港裡。長老慈悲,故收而埋之。不知為何而死。」縣丞道:「分明是汝眾僧謀殺此人,埋于此處。想必身上有財帛,故不肯實說。」于是不由分說,先把長老一箍兩拶,一夾一百敲,餘者眾僧都是二十板。俱令收入獄中,回覆曾公,再行報看。各僧皆稱冤不服。曾公尋思:「既是此僧謀死,屍必棄于河中,豈反埋于岸上?」又說:「于礙人眾,此有可疑。」因令將眾僧收監。將近兩月,不想安童來告此狀,即令委官押安童前至屍所,令其認視。這安童見其屍,大哭道:「正是我的主人,被賊人所傷,刀痕尚在。」于是檢驗明白,回報曾公,即把眾僧放回。一面查刷卷宗,復提出陳三、翁八審問。執稱苗青主謀之情。曾公大怒,差人行牌,星夜往揚州提苗青去了。一面寫本參劾提刑院兩員問官,受賍賣法。正是:

「污吏賍官濫國刑,  曾公判刷雪冤情;

雖然號令風霆肅,  萬裡輸贏總未真。」

話分兩頭,都表王六兒自從得了苗青幹事的那一百兩銀子,四套衣服,夜間與他漢子韓道國,就白日不閑,一夜沒的睡,計較著要打頭面,治簪環,喚裁縫來裁衣服,從新抽銀絲䯼髻,用十六兩銀子又買了個丫頭,名喚春香使喚。早晚教韓道國收用不題。一日,西門慶韓道國家,王六兒接著裡面吃茶畢,西門慶往後邊淨手去,看見隔壁月臺,問道:「是誰家的?」王六兒道:「是隔壁樂三家月臺。」西門慶分付王六兒:「你對他說,若不與我即便拆了,如何教他遮住了這邊風水?不然我教地方分付他。」這王六兒與韓道國說:「鄰舍家,怎好與他說的?」韓道國道:「咱不如瞞著老爹,廟上買幾根本植來,咱這邊也搭起個月臺來。上面曬醬,下邊不拘做馬坊,做個東淨,也是好處。」老婆道:「呸!賊沒算計的!此時搭月臺,買些磚瓦來蓋上兩間廈子,都不好?」韓道國道:「蓋兩間廈子倒不好,是東子房子。不如蓋一層兩間小房罷!」于是使了三十兩銀子,又蓋了兩間平房起來。西門慶差玳安抬了許多酒肉燒餅來,與他家犒樂匠人。那條街上,誰人不知。夏提刑得了幾百兩銀子在家,把兒子夏承恩,年十八歲,幹入武學肄業,做了生員。每日邀結師友,習學弓馬。西門慶約會劉、二薛二內相,周守備、荊都監、張團練合衛官員,出人情與他掛軸文慶賀,俱不必細說。西門慶因墳上新蓋了山子捲棚房屋,自從生了官哥,并做了千戶,還沒往墳上祭祖。教陰陽徐先生看了,從新立了一座墳門,砌的明堂神路,門首栽的柳,週圍種松柏,兩邊疊的坡峰。清明日上墳,要更換錦衣牌面,宰豬羊,定卓面。三月初六日清明,預先發柬,請了許多人,推運了東西,酒米下飯菜蔬。叫的樂工雜耍扮戲的:小優兒是李銘、吳惠、王柱、鄭奉,唱的是李桂姐、吳銀兒、韓金釧、董嬌兒。官家請了張團練、喬大戶、吳大舅、花大舅、吳二舅、沈姨夫、應伯爵、謝希大、傅夥計、韓道國、雲離守、賁地傳、并女婿陳經濟等,約二十餘人。堂客請了張團練娘子、張親家母、喬大戶娘子、朱臺官娘子、尚舉人娘子、吳大妗子、二妗子、楊姑娘、潘姥姥、花大妗子、吳大姨、孟大姨、吳舜臣媳婦鄭三姐、崔本妻、段大姐、并家中吳月娘、李嬌兒、孟玉樓、潘金蓮、李瓶兒、孫雪娥、西門大姐、春梅、迎春、玉簫、蘭香,奶子如意兒抱著官哥兒,裏外也有二十四、五頂轎子。先是月娘對西門慶說:「孩子且不消教他往墳上去罷。一來還不曾過一周;二者劉婆子說這孩子腦門還未長滿,胆兒小。這一到墳上路遠,只怕諕著他。依著我,不教他去。」留下奶子和老馮在家,和他做伴兒。只教他娘母子一個去罷。」西門慶不聽,便道:「此來為何?他娘兒兩個不到墳前與祖宗磕個頭兒去?你信那婆子老淫婦胡說,可可就是孩子腦門未長滿!教奶子用被兒裹著在轎子裏。按的孩兒牢牢的,怕怎的?」那月娘便道:「你不聽人說,隨你!」從清早晨,堂客都從家裡取齊,起身上了轎子,無辭出南門到五里外祖墳上,遠遠望見青松鬱鬱,翠柏森森。新蓋的墳門,兩邊坡峰上去,週圍石牆,當中甬路。明堂、神臺、香爐、燭臺,都是白玉石鑿的。墳門上新安的牌面,大書:「錦衣武略將軍西門氏先塋。」墳內正面,土山環抱,林樹交枝。西門慶穿大紅冠帶,擺設豬羊祭品,卓席祭奠。官家祭畢,堂客纔祭。響器鑼鼓,一齊打起來。那官哥兒諕的在奶子懷裏磕伏著,只倒咽氣,不敢動一動兒。月娘便叫:「李大姐,你還不教奶子抱了孩子往後邊去罷哩!你看諕的那腔兒!我說且不教孩兒來罷,恁漒的貨,只當教抱了他來。你看諕的那孩兒這模樣!」李瓶兒連忙下來,分付玳安,且叫把鑼鼓住了。連忙攛掇掩著孩兒耳朵,快抱了後邊去罷。須臾祭畢,徐先生唸了祭文,燒了布。西門慶邀請官客在前客位。月娘邀請堂客,在後邊捲棚內。由花園進去,兩邊松墻普築,竹徑欄杆。週圍花草,一望無際。正是:

「桃紅柳綠鶯梭織,  都是東君造化成。」

當下扮戲的,在捲棚內扮與堂客們瞧。兩個小優兒在前廳官客席前,唱了一回。四個唱的,輪番遞酒。春梅、玉簫、蘭香、迎春四個,都在堂客上邊,執壺斟酒,立在大姐卓頭,同吃湯飯點心。吃了一回,潘金蓮與玉樓、大姐、李桂姐、吳銀兒同往花園裡,打了回鞦韆。原來捲棚後邊,西門慶收拾了一明兩暗三間床炕房兒。裏邊舖陳床帳,擺放卓椅梳籠抿鏡粧臺之類,預備堂客來上墳,在此梳粧歇息。或閒常接了妓者,在此頑耍。糊的猶如雪洞般乾淨,懸掛的書畫琴棋瀟灑。奶子如意兒看守官哥兒,正在那灑金床炕兒,舖著小褥子兒睡。迎春也在傍,和他頑耍。只見潘金蓮獨自從花園驀地走來,手中拈著一枝桃花兒。進屋裡,看見迎春便道:「你原來這一日沒在上邊伺候。」迎春道:「有春梅、蘭香、玉蕭在上邊哩,俺娘教我下邊來看哥兒。拏了兩碟下飯點心,與如意兒吃。」金蓮看見那邊卓上放著一碟子鵝肉,一碟蹄子肉 ,并幾個菓子。奶子見金蓮來,便抱起官哥兒來。金蓮便戲他說道:「小油嘴兒,頭裡見打起鑼鼓來,諕的不則聲,原來這等小胆兒!」于是一面解開藕絲羅襖兒綃金衫兒,接過孩兒,抱在懷裡,與他兩個嘴對嘴親嘴兒。忽有陳經濟掀簾子走入來,看見金蓮鬬孩子頑耍,也鬬那孩子。金蓮道:「小道兒,你也與姐夫個嘴兒。」可霎作怪,那官哥兒便嘻嘻望著他笑。經濟不由分說,把孩子就摟過來,一連親了幾個嘴。金蓮罵道:「怪短命,誰家親孩子,把人的鬢都抓亂了。」經濟笑戲道:「你還說,早時我沒錯親了哩。」金蓮聽了,恐怕婢子瞧科,便戲發訕,將手中拏的扇子,倒過把子來,向他身上打了一下,打的經鯽魚般跳。罵道:「怪短命,誰和你那等調嘴調舌的!」經濟道:「不是你老人家摸量惜些情兒,人身上穿著恁單衣裳。就打恁一下!」金蓮道:「我平白惜甚情兒?今後惹著我。只是一味打。」如意兒見他頑的訕,連忙把官哥兒接過來抱著。金蓮與經濟兩個,還戲謔一處。金蓮將那一枝桃花兒做了一個圈兒,悄悄套在經濟帽子上。走出去,正值孟玉樓和大姐、桂姐三個從那邊來。大姐看見,便問:「是誰幹的營生?」經濟取下來,去了,一聲兒也沒言語。堂客前戲文,扮了四大摺。看看:

「窗外日光彈指過,  席前花影座間移。」

看看天色晚來,西門慶分付賁四,先把擡轎子的,每人一碗酒,四個燒餅,一盤熟肉。俵散停當,然後才把堂客轎子起身。官家騎馬在後。來興兒與廚役慢慢的抬食盒。然後玳安、來安、畫童、棋童兒跟月娘眾人轎子,琴童并四名排軍,跟西門慶馬。奶子如意兒,獨自坐一頂小轎,懷中抱著哥兒,用被裹的緊緊的進城。月娘還不放心,又使回畫童兒來,叫他跟定著奶子轎子,恐怕進城人亂。且說月娘轎子進了城,就與喬家那邊眾堂客轎子分路。來家,先下轎進去。半日,西門慶、陳經濟纔到家下馬。只見平安兒迎門就稟說:「今日掌刑夏老爹親自下馬到廳,問了一遍去了。落後又差人問了兩遍,不知有甚勾當?」西門慶聽了,心中猶豫。到于廳上,只見書童兒在傍接衣服。西門慶因問:「今日你夏老爹來,留下甚麼話來?」書童道:「他也沒說出來,只問:『爹往那去了?使人請去,我有句要緊話兒說。』小的便道:『今日都往墳上燒布去了,至晚纔來。』夏老爹說:『我到午上還來。』落後又差人來問了兩遭,小的說還未來哩。」西門慶心中不足,心下轉道:「都是甚麼?」正疑惑之間,只見平安來報:「夏老爹來了!」那時已有黃昏時分,只見夏提刑便衣坡巾,兩個伴當跟隨,下馬到于廳上敍禮。說道:「長官今日往寶庄去來?」西門慶道:「今日先塋祭掃。不知長官下降,失迎。恕罪恕罪!」敢來有一事,報與長官知道。」因說:「咱每往那邊客位內坐去罷。」西門慶令書童開捲棚門,請往那裡說話。左右都令下去。夏提刑道:「今朝縣中李大人到學生那裡,如此這般,說大巡新近有參本上東京,長官與學生,俱在參例。學生令人抄了個邸報在此,與長官看。」西門慶聽了,大驚失色,急接過邸報來。燈下觀看,端的上面寫著甚言詞?

「巡按山東監察御史曾孝序一本,參劾貪肆不職武官,乞賜罷黜以正法紀事:臣聞巡蒐四方,省察風俗,乃天子巡狩之事也;彈壓官邪,振揚法紀,乃御史糾政之職也。昔春秋載天王巡狩,而萬邦懷保,民風協矣,王道彰矣,四民順矣,聖治明矣。臣自去歲奉命巡按山東齊魯之邦,一年將滿。歷訪方面有司,文武官員賢否,頗得其實。茲當差滿之期,敢不循例甄別,為我皇上陳之。除參劾有司方面官員,另具疏上請參照,山東提刑所掌刑金吾衛正千戶夏延齡,闒茸之材,貪鄙之行;久于物議,有玷班行。昔者典牧皇畿,大大肆科擾,被屬官陰發其私;今省理山東刑獄,復著狼貪,為同僚之箝制。縱子承恩,冒籍武舉,倩心代考,而土風掃地矣!信家人夏壽,監索班錢,被軍騰詈,而政事不可知乎!接物敗奴顏婢膝,時人有『丫頭』之稱;問事則依違兩可,群下有『木偶』之誚。理刑副千戶西門慶,本係市井棍徒,夤緣陞職,濫冒武功。菽麥不知,一丁不識。縱妻妾嬉遊街巷,而帷薄為之不清;携樂婦而酣飲市樓,官箴為之有玷;至于包養韓氏之婦,恣其歡淫,而行檢不修;受苗青夜賂之金,曲為掩飾,而賍跡顯著。此二臣者,皆貪鄙不職,久乖清議,一刻不可居任者也。伏望聖明垂聽,勅下該部,再加詳查。如果臣言不謬,將延齡等亟賜罷斥,則官常有賴,而裨聖德永光矣。」

西門慶看了一遍,諕的面面相覷,默默不言。夏提刑道:「長官,似此如何計較?」西門慶道:常言:『兵來將擋,水來土掩。』事到其間,道在人為,少不的你我打點禮物,早差人上東京,央及毛爺那裡。」于是夏提刑急急作辭,到家拏了二百兩銀子,兩把銀壺。西門慶這裡是金鑲玉寶石鬧粧一條,三百兩銀子。夏家差了家人夏壽,西門慶這裡是來保。將禮物打包端正,西門慶修了一封書與翟管家,兩個早顧了頭口,星夜往東京幹事去了不題。且表官哥兒自從墳上來家,夜間只是驚哭,不肯吃奶,但吃下奶去,就吐了。慌的李瓶兒走來告訴月娘。月娘道:「我那等說,還未到一周的孩子,且休帶他出城門去。獨漒貨,他生死不依。只說:『此來今日墳上祭祖為甚麼來。不教他娘兒兩個走走?』只像那裡攙了分兒一般,睜著眼和我們兩個叫。如今都怎麼好?」李瓶兒正沒法兒擺佈。况西門慶又是因巡按御史參本參了,和夏提刑在前邊說話,往東京打點幹事。心上不遂,家中孩子又不好。月娘使小廝叫劉婆子來看,又請小兒科太醫。開門闔戶,亂了一夜。劉婆看了,說:「哥兒着了些驚氣入肚;又路上撞見五道將軍。不打緊,燒些布兒,退送退送就好了。又留了兩服朱砂丸藥兒,用薄荷燈心湯送下去。那孩兒方纔寧貼。睡了一覺,不驚哭吐奶了,只是身上熱還未退。李瓶兒連忙拏出一兩銀子,教劉婆子備布去。後的帶了他老公,還和一個師婆來,在捲棚內與哥兒燒布跳神。那西門慶早五更打發來保、夏壽起身,就亂著和夏提刑往東平府胡知府那裡,打聽提苗青消息去了。吳月娘聽見劉婆說孩兒路上著了驚氣,甚麼抱怨如意兒,說他不用心看孩兒:「想必路上轎子裡諕了他了。不然,怎的就不好起來?」如意兒道:「我在轎子裡將被兒裹得緊緊的,又沒【石店】着他。娘便回畫童兒來跟著轎子,他還好好的,我按著他睡。只進城內七八到家門首,我只覺他打了個冷戰。到家就不吃奶,哭起來了。」

按下這裡家中燒布與孩子下神。且說來保、夏壽一路儹行,只六日就趕到東京城內。到太師府內見了翟管家,將兩家禮物交割明日,翟謙看了西門慶書信,說道:「曾御史參本還未到哩,你且住兩日。如今老爺新近條陳奏了七信事在這里,旨意還末曾下來。待行下這個本去,曾御史本到,等我對老爺說,要老爺閣中只批與他該部知道;我這里差人再拏我的帖兒,分付兵部余尚書,把他的本只不覆上來。交你老爹只顧放心,管情一些事兒沒有。」于是把二人管待了酒飯,還歸到客店安歇那里。等到一日,蔡太師條陳本,聖旨准下來了。來保央府中門吏抄了個邸報,帶回家與西門慶瞧。端的上面奏行那七件事?

「崇政殿大學士吏部尚書魯國公蔡京一本,陳愚見,竭愚衷,收人才,臻實效,足財用,便民情,以隆聖治事:

第一曰罷科舉取士,悉由學校陞貢:

竊謂教化凌夷,風俗頹敗,皆由取士不得真才,而教化無以仰賴。書曰:『天生斯民,作之君,作之師。』漢學孝廉,唐興學校。我國家始制考貢之法,各執偏陋,以致此輩無真才,而民之司牧何以賴焉?今皇上寤寐求才,宵旰圖治。治在于養賢,養賢莫如學校。今後取士,悉遵古,由學校陞貢。其州縣發解禮闈,一切羅之。每歲考試上舍,則差知貢舉,亦如禮闈之式。仍立八行取士之科。八行者,謂孝、友、睦、婣、任、恤、忠、和也。士有此者,即免試,率相補大學上舍。

二曰罷講議財利司:切惟國初定制,都堂置講議財利司,蓄謂人君節浮費惜民財也。

今陛下即位以來,不寶遠物,不勞逸民,躬行節儉以自奉。蓋天下亦無不可返之俗,亦無不可節之財。惟當事者以俗化為心,以禁令為信。不忽其初不弛其後。治隆俗美,豐亨豫大。又何講議之為哉!悉罷。

三曰更鹽鈔法:切惟鹽鈔乃國家之課,以供邊備者也!今合無遵復祖宗之制,鹽法者,詔雲中、陝西、山西,三邊上納粮草,關領舊鹽鈔,易東南、淮、浙新鹽鈔。每鈔折派三分,舊鈔搭派七分。今商人照所派產鹽之地,下場支鹽,亦如茶法赴官秤騐納息。請批引限日行鹽之處販賣。如遇過限並行拘收。別買新引增販者,俱屬私鹽。如此則國課日增,而邊儲不乏矣。

四曰制錢法:切謂錢貨乃國家之血脈,貴乎流通,而不可淹滯。如扼阻淹滯不行者,則小民何以變通?而國課何以仰賴矣!自晉末鵝眼錢之後,至國初瑣屑不堪,甚至雜以鉛鐵夾錫。邊人販于虜,因而鑄兵器,為害不小。合無一切通行禁之也。以陛下新鑄大錢崇寧大觀通寶,一以當十,庶小民通行,物價不致于踴貴矣。

五曰行結糶俵糴之法:

切惟官糶之法,乃賑恤之義也。近年水旱相仍,民間就食,上始下賑恤之詔。近有戶部侍郎韓侶題覆欽依,將境內所屬州縣,各立社會,行結糶俵糴之法。保之千黨,黨之于里,里之于鄉,倡之結也。每鄉編為三戶。按上上、中中、下下。上戶者納粮,中戶者減半,下戶者遞派。粮數關支,謂之俵糶。如此則斂散便民之法,得以施行。而皇上可廣不費之仁矣!惟責守令,覆切舉行,其關係蓋匪細矣。

六曰詔天下州郡納免夫錢:切惟我國初寇亂未定,悉令天下軍徭丁壯,集于京師,以供運餽,以壯國勢。今承平日久,民各安業。合頒  詔行天下州郡,每歲上納免夫錢。每名折錢三十貫,解赴京師,以資邊餉之用,庶兩得其便矣,而民力少蘇矣!

七曰置提舉御前人舡所:切惟陛下自即位以來,無聲色犬馬之奉。所尚花石,皆山林間物,乃人之所棄者。但有司奉行之過,因而致擾,有傷聖治。陛下節其浮濫,仍請作御前提舉人舡所。凡有用悉出內帑,差官取之。庶無擾于州郡。伏乞聖裁。」

奉聖旨:「鄉言深切時艱,朕心加悅,足見忠猷。都依擬行該部知道。」

來保抄了邸報,等的翟管家,寫了回書,與了五兩盤纏,與夏壽取路回山東清河縣來。有日到家中,西門慶正在家耽心不下。那夏提刑一日一遍來問信。聽見來保二人到了,叫至後邊問他端的。來保對西門慶悉把上項事情訴說一遍:「府中見翟爹,看了爹的書,便說此事不打緊,交你爹放心。見今巡按也滿了,另點新巡按下來了。况他的參本還未到。等他本上時,等我對老爺說了,隨他本上參的怎麼重,只批了該部知道。老爺這里再拏帖兒,分付兵部余尚書,只把他的本立了案,不覆上去。隨他有撥天關本事,也無妨。」西門慶聽了。方纔心中放下。因問:「他的本怎倒還不到?」來保道:「俺每一去時,晝夜馬上行去,只五日就趕到京中,可知在他頭里。俺每回來,見路上一簇響鈴驛馬過,背著黃包袱,插著兩根雉尾,兩面牙旗,怕不就是巡按衙門進送實封纔到了。」西門慶道:「到得他的本上的遲,事情就停當了。我只怕去遲了。」來保道:「爹放心,管情沒事。小的不但幹了這件事的,又打聽的兩樁好事來,報爹知道。」西門慶問道:「端的何事?」來保道:「太師老爺新近條陳了七件事,旨意已是准行。如今老爺親家戶部侍郎韓爺。題准事例,在陝西等三邊,開引種鹽,各府州郡縣設立義倉,官糶糧米。令民間上上之戶,赴倉上米,討倉鈔,派給鹽引支鹽。舊倉鈔七分,新倉鈔三分。咱舊時和喬親家爹高陽關上納的那三萬粮倉鈔,派三萬鹽引,戶部坐派到好,趁著蔡老爹巡鹽下場支種了罷。倒有好些利息。」西門慶聽言,問道:「真個有此事?」來保道:「爹不信,小的抄了個邸報在此。」向書篋中取出來,與西門慶觀看。因見上面許多字樣,前邊叫了陳經濟來,唸與他聽。陳經濟唸到中間,只要結住了,還有幾個眼生字不認的。旋叫了書童兒來唸。那書童到還是門子出身,蕩蕩如流水不差,直唸到底。端的上面奏著那七件事云云。西門慶聽了喜。又看了翟管家書信,已知禮物交得明白,蔡狀元見朝,已點了兩淮巡鹽,心中不勝歡喜。一面打發夏壽回家,報與你老爹知道。一面賞了來保五兩銀子,兩瓶酒,一方肉,回房歇息,不在話下。正是:

「樹大招風風損樹,  人為名高傷喪身。」

有詩為證:

「得失榮枯命里該,  皆因年月日時栽;

胸中有志終須到,  囊內無財莫論才。」

畢竟不知後來如何,且聽下回分解:
