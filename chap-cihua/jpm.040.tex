%# -*- coding: utf-8 -*-
%!TEX encoding = UTF-8 Unicode
%!TEX TS-program = xelatex
% vim:ts=4:sw=4
%
% 以上设定默认使用 XeLaTex 编译,并指定 Unicode 编码,供 TeXShop 自动识别

%第四十回 
\chapter{抱孩童瓶兒希寵\KG 粧丫鬟金蓮市愛}


\begin{showcontents}{}




「善事須好做,  無心近不得,

你若做好事,  別人分不得;

經卷積如山,  無緣看不得,

財錢過壁堆,  臨危將不得;

靈承好供奉,  起來吃不得,

兒孫雖滿堂,  死來替不得。」

話說當夜月娘和王姑子一炕睡。王姑子因問月娘:「你老人家怎的就沒見點喜事兒?」月娘道:「又說喜事哩!前日八月裡,因買了對過喬大戶房子,平白俺每都過去看,上他那樓梯,一腳躡滑了,把個六七個月身扭吊了。至今再誰見什麼孩子來!」王姑子道:「我的奶奶,六、七個月也成形了。」月娘道:「半夜裡吊在榪子裡,我和丫頭點燈撥着瞧,倒是個小廝兒。」王姑子道:「我的奶奶,可惜了,怎麼來扭着了!還是胎氣坐的不牢?」月娘道:「我只上他家樓梯窄趔,不知怎的一腳滑下來!還虧了孟三姐一手扶住我,不然一吊下來了。」王姑子道:「你老人家養出個兒來,強如別人。你看他前邊六娘,進兒多少時兒,倒生了個兒子,何等的好!」月娘道:「他各人的兒女,隨天罷了。」王姑子道:「也不打緊。俺每同行一個薛師父,一紙好符水藥。前年陳郎中娘子,也是中年無子,常時小產了幾胎,白不存。也是吃了薛師父符藥,如今生了,好不醜滿抱的小廝兒!一家兒歡喜的要不得。只是用着一件物件兒難尋。」月娘問道:「什麼物件兒?」王姑子道:「用着頭生孩子的衣胞,拏酒洗了,燒成灰兒,揀着符藥,揀壬子日,人不知,鬼不覺,空心用黃酒 吃了。算定日子兒不錯,至一個月就坐胎氣,好不准!」月娘道:「這師父是男僧女僧?在那裡住?」王姑子道:「他也是俺女僧,也有五十多歲。原在地藏庵兒住來,如今搬在南首裡法華庵兒做首座。好不有道行!他好少經典兒!又會講說金剛科儀,各樣因果寶卷,成月說不了;專在大人家行走。要便接了去,十朝半月不放出來。」月娘道:「你到明日請他來走走。」王姑子道:「我知道。等我替你老人家討了這符藥來着!止是這一件兒難尋。這裡沒尋處,恁般如此,你不如把前頭這孩子的房兒,借情跑出來便了罷。」月娘道:「緣何損別人,安自己的!我與你銀子,你替我慢慢另尋便了。」王姑子道:「這個倒只是問老娘尋他纔有。我替你整治這符水,你老人家吃了,管情就有。難得你明日另養出來,隨他多少,十個明星當不的月!」月娘分付:「你卻休對人說。」王姑子道:「好奶奶,傻了,我肯對人說!」說了一回,各人多睡了。一宿晚景題過。到次日,西門慶打廟裡來家。月娘纔起來梳頭。玉蕭接了衣服坐下。月娘因說:「昨日家裡六姐等你來上壽,怎的就不來了?」西門慶悉把醮事未了,吳親家晚夕費心擺了許多卓席。吳大舅先來了,留住我和花大哥、應二哥、謝希大,兩個小優兒彈唱着,俺每吃了半夜酒。今早我便先進城來了。應二哥他三個還吃酒哩。昨日甚是難為吳親家,破費了許多錢。」告訴了一回。玉蕭遞茶吃了,也沒往衙門裡去,走到前邊書房裡,〈扌歪〉在床上就睡着了。落後潘金蓮、李瓶兒梳了頭,抱着孩子出來,多到上房陪着吃茶。月娘向李瓶兒道:「他爹來了這一日,在前頭哩。我教他吃茶食,他不吃。丫頭有了飯了,你把你家小道士,替他穿上衣裳,抱到前頭與他爹瞧瞧去。」潘金蓮道:「我也去,等我替道士兒穿衣服。」于是戴上綃金道髻兒,穿上道衣,帶了項牌符索,套上小鞋襪兒,金蓮就要奪過去。月娘道:「教他媽媽抱罷,況自你這蜜褐色桃繡裙子,不耐污。撒上點子,臢到了不成!。」于是李瓶兒抱定官哥兒,潘金蓮便跟着,來到前邊西廂房內。書童見他二人掀簾,連忙就躲出來了。金蓮見西門慶臉朝裡睡炕床上,指着孩子說:「老花子,你好睡。小道士兒自家來請你來了。大媽媽房裡擺下飯,教你吃去。你還不快起來?還推睡兒!」那西門慶吃了一夜酒的人,倒去頭,那顧天高地下,鼾睡如雷。金蓮與李瓶兒一邊一個,坐在床上,把孩子放在他面前。怎禁的鬼混,不一時,把西門慶弄醒了。睜開眼看,見官哥兒在面前,頭上戴着綃金道髻兒,身穿小道衣兒,項圍符索,喜歡的眉開眼笑。連忙接過來,抱到懷裡,與他親個嘴兒。金蓮道:「好乾淨嘴頭子,就來親孩兒。小道士兒吳應元,你噦他一口!你說昨日在那裡使牛耕地來?今日乏困的你這樣的!大白日強覺。昨日叫五媽只顧等着你,你恁大膽,不來與五媽磕頭!」西門慶道:「昨日醮事等的晚。晚夕謝將,又整酒吃了一夜。今日到這咱時分,還一頭在這裡。睡回,還要往尚舉人家吃酒去。」金蓮道:「你不吃酒去罷了。」西門慶道:「他家從昨日送了帖兒來,不去惹人家不怪?」金蓮道:「你去,晚夕早些兒來家,我等着你哩。」李瓶兒道:「他大媽媽擺下飯了,又做了些酸筍湯 ,請你吃飯去哩。」西門慶道:「我心裡還不待吃,等我去呵些湯罷。」于是起來往後邊去了。這潘金蓮兒見他去了,一屁股就坐在床上正中間,腳蹬着地爐子,說道:「這原來是個套炕子。」伸手摸了摸褥子裡,說道:「倒且是燒的滾熱的炕兒。」瞧了瞧,旁邊桌上放着個烘硯瓦的銅絲火爐兒。隨手取過來,叫:「李大姐,那邊香几兒上,牙盒裡盛的甜香餅兒,你取些來我。」一面揭開了,拿幾個在火炕內。一面夾在襠裡,拏裙子裏的沿沿的,且薰熱身上。坐了一回,李瓶兒說道:「咱進去罷,只怕他爹吃了飯出來。」金蓮道:「他出來不是,怕他麼?」于是二人抱着官哥兒,進入後邊來。良久,西門慶吃了飯,分付排軍備馬,午後往尚舉人家吃酒去了。潘姥姥先去了。且說晚夕王姑子要家去,月娘悄悄與了他一兩銀子,叫他休對大師父說,好歹往薛姑子帶了符藥來。王姑子接了銀子,和月娘說:「我這一去,只過十六日兒纔來罷。就替你尋了那件東西來。」月娘道:「也罷,你只替我幹的的停當,我還謝你。」于是作辭去了。看官聽說:但凡大人家,似這樣僧尼牙婆,決不可抬舉。在深官大院相伴着婦女,俱以講天堂地獄,談經說典為由。背地裡說釜念款,送煖偷寒,其麼事兒不幹出來!十個九個,都被他送上災厄。有詩為證:

「最有緇流不可言,  深宮大院哄嬋娟,

此輩若皆成佛道,  西方依舊黑漫漫。」

卻說金蓮晚夕走在月娘房裡,陪着眾人坐的。走到鏡臺前,把䯼髻摘了,打了個盤頭揸髻,把臉搽的雪白,抹的嘴唇兒鮮紅,戴着兩個金燈籠墜子,貼着三面花兒,帶着紫銷金箍兒,尋了一套大紅織金襖兒,下着翠藍段子裙,要裝丫頭,哄月娘眾人耍子。叫將李瓶兒來與他瞧,把李瓶兒笑的前仰後合,說道:「姐姐,你裝扮起來,活像個丫頭!等我往後邊去,我那屋裡有紅布手巾,替你蓋着頭。對他們只說他爹又尋了個丫頭,諕他們諕,管定就信了。」春梅打着燈籠,在頭裡走。走到撞見陳經濟,笑道:「我道是誰來?這個就是五娘幹的營生。」李瓶兒叫道:「姐夫,你過來,等我和你說了着。你先進去,見他們只如此如此,這般這般。」經濟道:「我有法兒哄他。」于是先走到上房裡,眾人都在炕上坐着吃茶。經濟道:「娘,你看爹平白裡叫薛嫂兒使了十六兩銀子,買了人家一個二十五歲會彈唱的姐兒,剛纔拏轎子送將來了。」月娘道:「真個?薛嫂怎不先來對我說?」經濟道:「他怕你老人家罵他,送轎子到大門首,他就去了。丫頭便教他每領進來了。」大妗子還不言語。楊姑娘道:「官中有這幾房姐姐勾了,又要他來做什麼?」月娘道:「好奶奶,你禁的!有錢就買一百個,有什麼多?俺每多是老婆當軍,在這屋裡充數兒罷了!」玉簫道:「等我瞧瞧去。」只見月亮地裡,原來春梅打燈籠,叫了來安兒小廝打着,和李瓶兒後邊跟着,搭着蓋頭,穿着紅衣服進來。慌的孟玉樓、李嬌兒都出來看。良久,進入房裡。玉簫挨在月娘邊,說道:「這個是主子,還不磕頭哩!」一面揭了蓋頭。那潘金蓮插燭也似磕下頭去。忍不住撲吃的笑了。玉樓道:「好丫頭,不與你主子磕頭,且笑!」月娘也笑了,說道:「這六姐成精死了罷!把俺每哄的信了。」玉樓道:「大娘,我不信。」楊姑娘道:「姐姐,你怎的見出來不信?」玉樓道:「俺六姐平昔磕頭,也學的那等,磕了頭起來,倒退兩步纔拜。」楊姑娘道:「還是姐姐看的出來,要着老身,就信了。」李嬌兒道:「我也就信了。剛纔不是揭蓋頭,他自家笑,還認不出來。」正說着,只見琴童兒抱進毡包來,說:「爹來家了。」孟玉樓道:「你且藏在明間裡,等爹進來,等我哄他哄。」不一時,西門慶來到。楊姑娘、大妗子出去了。進入房內,椅子上坐下。月娘在旁不言語。玉樓道:「今日薛嫂兒轎子送人家一個二十歲丫頭來,說是你教他送來,要他的。你恁許大年紀,前程也身上,還幹勾當?」西門慶笑道:

「我那裡教他買丫頭來?信那老淫婦哄你哩。」玉樓道:「你問大姐姐不是,丫頭也領在這裡。我不哄你;你不信我,我叫出來你瞧。」于是叫玉簫:「你拉進那新丫頭來見你爹。」那玉簫掩着嘴兒笑,又不敢去拉。前邊走了走兒,又回來了,說道:「他不肯來。」玉樓道:「等我去拉。恁大膽子的奴才,頭兒沒動,就扭主子。也是個不聽指教的。」一面走到明間內,只聽說道:「怪行貨子!我不好罵的。人不進去,只顧拉人,拉的手腳兒不着。」玉樓笑道:「好奴才,誰家使的你恁沒規矩,不進來見你主子磕頭?」一面拉進來。西門慶燈影下睜眼觀看,卻是潘金蓮打着楂䯼裝丫頭,笑的眼沒縫兒。那金蓮就坐在旁邊椅子上。玉樓道:「好大膽丫頭,新來乍到,就恁少條失教的,大刺刺對着主子坐着!道撅臭,與他這個主子兒了。」月娘笑道:「你趁着你主子來家,與他磕個頭兒罷。」那金蓮也不動,走到月娘裡間屋裡,一頓把簪子拔了,戴上䯼髻出來。月娘道:「好淫婦,討了誰上哩話,就戴上髮髻了!」眾人又笑了一回。月娘告訴西門慶說:「今日喬親家那裡,使喬通送了六個帖兒來,請俺每吃看燈酒。咱到明日,不先送些禮兒去?」教玉簫拿帖兒與西門慶瞧。見上面寫着:

「十二日寒舍薄具菲酌,奉屈魚軒。仰冀賁臨,不勝榮幸。右啟大德望西門大親家老夫人粧次,下書眷末喬門鄭氏歛衽拜。」

「到明日咱家發柬,十四日也請他娘子,并周守備娘子,荊都監娘子、夏大人娘子、張親家母,大妗子也不必家去了。教賁四叫將花兒匠來,做幾架煙火;王皇親家一起扮戲的小廝每,來扮西廂記的。你每往院中,再把吳銀兒、李桂兒接了。」西門慶看畢,說道:「明早叫來興兒買四樣餚品,一罈南酒 ,送了去就是了。你每在家看燈吃酒,我和應二哥、謝子純,往獅子街樓上吃酒去。」說畢,不一時放下卓兒,安排酒上來。潘金蓮遞酒,眾姊妹相陪,吃了一回。西門慶困見金蓮裝扮丫頭,燈下艷粧濃抹,不覺淫心蕩漾,不住把眼色遞與他。這金蓮就知其意。行陪着吃酒,就到前邊房裡,去了冠兒,挽着杭州攢,重勻粉面,復點朱唇。原來早在房中,先預備下一桌酒,齊整菓菜,等西門慶進房,婦人還要自己與遞酒。不一時,西門慶果然來到,見婦人還挽起雲髻來,心中喜甚,摟着他坐在椅子上,兩個說笑。不一時,春梅收拾上酒菜來,婦人從新與他遞酒。西門慶道:「小油嘴兒,頭裡已是遞過罷了,又教你費心。」金蓮笑道:「那個大夥裡酒兒不算,這個是奴家業兒,與你遞鐘酒兒,年年累你破費,你休抱怨。」把西門慶笑的沒眼縫兒,連忙接了他酒,摟在懷裡膝蓋兒坐的。春梅斟酒,秋菊拿菜兒。金蓮道:「我問你,到十二日喬家請,俺每多去?只教大姐姐去?」西門慶道:「他既是下帖兒多請你每,如何不去?到明日,叫奶子抱了哥兒也去走走,省的家裡尋他娘哭。」金蓮道:「大姐姐他每多有衣裳穿,我老道只自知數的那幾件子,沒件好當眼的。你把南邊新治來那衣服,一家分散幾件子,裁與俺每穿了罷。只顧放着,怎生小的兒也怎的?到明日咱家擺酒,請眾官娘子,俺每也好見他,不惹人笑話!我長是說着,你把臉兒憨着。」西門慶笑道:「既是恁的,明日叫了趙裁來,與你每裁了罷。」金蓮道:「及至明日叫裁縫做,只差兩日兒,做着還遲了哩。」西門慶道:「對趙裁說,多帶幾個人來,替你每儹造兩三件出來,就勾了。剩下別的,慢慢再做也不遲。」金蓮道:「我早對你說過,好歹揀兩套上色兒的與我。我難向他們多有,我身體沒與我做什麼大衣裳。」西門慶笑道:「賊小油嘴兒,去處搯個尖兒!」兩個說話飲酒,到一更時分,方上床。兩個如被底鴛鴦,帳中鸞鳳,畫樓燕語,不肯即休,覆應即再聚雲情,一時不肯即休,整狂了半夜。到次日,西門慶衙門中回來,開了箱櫃,打開出南邊織造的夾板羅段尺頭來,使小廝叫將趙裁來,每人做件粧花通袖袍兒,一套遍地錦衣服,一套粧花衣服。惟月娘是兩套大紅通袖遍地錦袍兒,四套粧花衣服。在捲棚,一面使琴童兒叫趙裁去。這趙裁正在家中吃飯,聽的西門慶宅中叫,連忙丟下飯碗,帶着剪尺就走。時人有幾句,誇讚這趙裁好處:

「我做裁縫姓趙,  月月主顧來叫,

針線緊緊隨身,  剪尺常掖靴靿;

幅摺趕空走儹,  截彎病除手到,

不論上短下長,  那管襟扭領拗?

每日肉飯三餐,  兩頓酒兒是要,

剪截門首常出,  一月不脫三廟;

有錢老婆嘴光,  無時孩子亂叫,

不拘誰家衣裳,  且交印鋪睡覺。

隨你催討終朝,  只拏口兒支調,

十分要緊騰挪,  又將後來頂倒,

問你有甚高強?  只是一味老落。」

不一時走到,見西門慶坐在上面,連忙磕了頭。桌上鋪着氈條,取出剪尺來,先裁月娘的一件大紅遍地錦五彩粧花通袖襖,獸朝麒麟補子段袍兒,一件玄色五彩金遍邊葫蘆樣鸞鳳穿花羅袍,一套大紅段子遍地金通袖麒麟補子襖兒,翠藍寬拖遍地金裙,一套沉香色粧花補子遍地錦羅襖兒,大紅金皮綠葉百花拖泥裙。其餘李嬌兒、孟玉樓、潘金蓮、李瓶兒四個,多裁了一件大紅五彩通袖粧花錦雞段子袍兒,兩套粧花羅段衣服。孫雪娥只是兩套,就沒與他袍兒。須臾,共裁剪三十件衣服,兌了五兩銀子,與趙裁做工錢。一面叫了十來個裁縫,在家儹造,不在話下。正是:

「金鈴玉墜裝閨女,  錦綺珠翹飾妹娃。」

畢竟未知後來如何,且聽下回分解:




\end{showcontents}


