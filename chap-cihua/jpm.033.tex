%# -*- coding: utf-8 -*-
%!TEX encoding = UTF-8 Unicode
%!TEX TS-program = xelatex
% vim:ts=4:sw=4
%
% 以上设定默认使用 XeLaTex 编译,并指定 Unicode 编码,供 TeXShop 自动识别

%第三十三回 
\chapter{陳經濟失鑰罰唱\KG 韓道國縱婦爭風}

「人生雖未有前知,  富貴功名豈力為,

枉將財帛為根蒂,  豈容人力敵天時;

世俗炎涼空過眼,  塵紛離合漫忘機,

君子行藏須用舍,  不開眉笑待何如。」

話說西門慶衙門中來家,進門就問月娘:「哥兒好些?使小廝請太醫去?」月娘道:「我已叫劉婆子來了。見吃了他藥,孩子如今不洋奶,穩穩睡了這半日,覺好些了。」西門慶道:「信那老淫婦胡針亂炙,還請小兒科太醫看纔好。既好些了罷,若不好,拏到衙門里去拶與老淫婦一拶子!」月娘道:「你枉恁的口拔舌罵人。你家孩兒現吃了他藥好了,還恁舒着嘴子罵人!」說畢,丫鬟擺上飯來。西門慶剛纔吃了飯,只見玳安兒來報:應二爹來了。西門慶教小廝拏茶出去,請應二爹捲棚內坐。向月娘道:「把剛纔我吃飯的菜蔬休動,教小廝拏飯出去,教姐夫陪他吃,我就來。」月娘便問:「你昨日早辰使他往那里去,那咱纔來?」西門慶便告說:「應二哥認的湖州一箇客人何官兒,門外店里堆着五百兩絲線,急等着要起身家去,來對我說,要拆些發脫。我只許他四百五十兩銀子。昨日使他同來保拏了兩錠大銀子作樣銀,已是有了來了,約下今日兌銀子去。我想來獅子街房子空閑,打開門面兩開,倒好收拾開絨線舖子,搭個夥計。況來保已是鄆王府認納官錢,教他與夥計在那里,又看了房兒,又做了買賣。」月娘道:「少不得又尋夥計?」西門慶道:「應二哥說他有一相識,姓韓,原是絨線行,如今沒本錢,閑在家里,說寫算皆精,行止端正,再三保舉。改日領他來見我,寫立合同。」說畢,西門慶在房中兌了四百五十兩銀子,教來保拏出來。陳經濟已是陪應伯爵在捲棚內吃完飯,等的心裡火發。見銀子出來,心中歡喜。與西門慶唱了喏,說道:「昨日打擾哥,到家晚了,今日再扒不起來。」西門慶道:「這銀子我兌了四百五十兩,教來保取搭連眼同裝了。今日好日子,便雇車輛搬了貨來,鎖在那邊房子里就是了。」伯爵道:「哥主張的有理。只怕蠻子停留長智,推進貨來,就完了帳。」于是同來保騎頭口,打着銀子,逕到門外店中,成交易買賣,誰知伯爵背地與何官兒砸殺了,只四百二十兩銀子,打了三十兩背工。對着來保當面只拏出九兩用銀來,二人均分了。雇了車腳,即日推貨進城,堆在獅子街空房內,鎖了門來回西門慶話。西門慶教恁伯爵擇吉日,領韓夥計來見。其人五短身材,三十年紀。言談滾滾,相貌堂堂,滿面春風,一團和氣。西門慶即日與他寫立合同,同來保領本錢雇人染絲,在獅子街開張舖面,發賣各色絨絲。一日也賣數十兩銀子,不在話下。光陰迅速,日月如梭,不覺八月十五日月娘生辰來到。請堂客擺酒,留下吳大妗子、潘姥姥、楊姑娘并兩箇姑子住兩日,晚夕宣誦唱佛曲兒,帶坐到二三更分歇。那日西門慶因上房有吳大妗子在這里不方便,走到前邊李瓶兒房中看官哥兒,心裡要在李瓶兒房裡睡。李瓶兒道:「孩子纔好些兒,我心裡不耐煩,往他五媽媽房裡睡一夜罷。」西門慶笑道:「我不惹你。」于是走過金蓮這邊來。那金蓮聽見漢子進他房來,如同拾了金寶一般,連忙打發他潘姥姥,過李瓶兒這邊宿歇。他便房中高點銀燈,款伸錦被,薰香澡牝,夜間陪西門慶同寢。枕畔之情,百般難述,無非只要牢籠漢子之心,使他不往別人房裡去。正是:

「鼓鬣遊蜂,嫩蕊半勻春蕩漾;餐香粉蝶,花房深宿夜風流。」

李瓶兒見潘姥姥過來,連忙讓在炕上坐的,教迎春安排酒席烙餅 ,晚夕說話,坐半夜纔睡。到次日與了潘姥姥一件葱白綾襖兒,兩雙段子鞋面,二百文錢。把婆子喜歡的屁滾尿流,過這邊來,拏與金蓮瞧,說:「此是那邊姐姐與我的。」金蓮見了,反說他:「娘好恁小眼薄皮的,什麼好的,拏了他來!」潘姥姥道:「好姐姐,人倒可憐見與我,你都說這個話,你肯與我一件兒穿?」金蓮道:「我比不得他有錢的姐姐。我穿的還沒有哩,拏什麼與你?你平白吃了人家的來,等住回,咱整理幾碟子來,篩上壺酒,拏過去還了他就是了。倒明日少不的教人【石店】言試語,我是聽不上。」一面分付春梅定八碟菜蔬,四盒菓子,一錫瓶酒。打聽西門慶不在家,教秋菊用方盒拏到李瓶兒房裡,說:「娘和姥姥過來,無事和六娘吃盃酒。」李瓶兒道:「又教你娘費心。」少頃,金蓮和潘姥姥來,三人坐定,把酒來斟,春梅侍立斟酒。娘兒每說話間,只見秋菊來叫春梅,說:「姐夫在那邊尋衣裳,教你去開外邊樓門哩。」金蓮分付:「叫你姐夫尋了衣裳,來這裡呵甌子酒去!」不一時,經濟尋了幾家衣服,就往外走。春梅進來回說:「他不來。」金蓮道:「好歹拉了他來。」又使出綉春去把經濟請來。潘姥姥在炕上坐,小卓兒擺着菓菜兒,金蓮、李瓶兒陪着吃酒,連忙唱了喏。金蓮說:「我好意教你來吃酒兒,你怎的張致不來?就弔了造化了!」努了個嘴兒,教春梅:「拏寬盃兒來,篩與你姐夫吃。」經濟把尋的衣服,放到炕上,坐下。春梅做定科範,取了箇茶甌子,流沿邊斟上遞與他。慌的經濟說道:「五娘賜我,寧可吃兩小鍾兒罷。外邊舖子裡許多人等着要衣裳。」金蓮道:「教他等着去,我偏教你吃這一大鍾。那小鍾子刁刁的不耐煩!」潘姥姥道:「只教哥哥吃這一鍾罷,只怕他買賣事忙。」金蓮道:「你信他有什麼忙,吃好少酒兒?金漆桶子,吃到第二道箍上」。那經濟笑着,拏酒來剛呷了兩口。潘姥姥叫:「春梅姐姐,你拏盃兒與哥哥,教他吃寡酒。」春梅也不拏筯,故意毆他,向攢盒內取了兩個核桃遞與他。那經濟接過來道:「你敢笑話,我就禁不開他。于是放在牙上只一磕,咬碎了下酒。潘姥姥道:「還是小後生家好口牙。相老身,東西兒硬些,就吃不得。」經濟道:「兒子世上有兩庄兒鵝卵石,牛騎角,吃不得罷了。」金蓮見他吃了那鍾酒,教春梅再斟上一鍾兒,說:「頭一鍾是我的了。你姥姥和六娘不是人麼?也不教你吃多,只吃三甌子,饒了你罷。」經濟道;「五娘,可憐見兒子來!真吃不得了。此這一鍾,恐怕臉紅,惹爹見怪!」金蓮道:「你也怕你爹?我說你不怕他。你爹今日往那里吃酒去了?」經濟道:「後晌往吳驛丞家吃酒;如今在對過喬大戶房子里看收拾哩!」金蓮問:「喬大戶昨日搬了去,咱今日怎不與他送茶?」經濟道:「今早送茶去了。」李瓶兒問:「他家搬到那里住去了?」經濟道:「他在東大街上使了一千二百銀子,買了所好不大的房子,與咱家房子差不多兒,門面七間,到底五層。」說話之間,經濟捏着鼻子,又挨了一鍾,趁金蓮眼錯,得手拏着衣服,往外一溜烟跑了。迎春便道:「娘,你看姐夫,忘記鑰匙去了。」那金蓮取咼來,坐在身底下,向李瓶兒道:「等他來尋,你每且不要說,等我奈何他一回兒,纔與他。」潘姥姥道:「姐姐與他便了,又奈何他怎的?」那經濟走到舖子裡,袖內摸摸不見鑰匙,一直走到李瓶兒房里尋。金蓮道:「誰見你什麼鑰匙。你拏鑰匙,管着什麼來?放在那裡,就不知道。」春梅道:「只怕你鎖在樓上了,頭里我沒見你拏來。」經濟道:「我記的帶出來。」金蓮道:「小孩兒家屁股大,敢弔了心。又不知家裡外頭,什麼人扯落的?你恁有魂沒識,心不在肝上!」經濟道:「有人來贖衣裳,可怎的樣?趁爹不過來,少不得叫個小爐匠來開樓門,纔知有沒?」李瓶兒忍不住,只顧笑。經濟道:「六娘拾了,與了我罷。」金蓮道:「也沒見這李大姐,不知和他笑什麼,恰似俺每拏了他的一般。」急得經濟只是油回磨轉。轉眼看見金蓮身底下,露出鑰匙帶兒來,說道:「這不是鑰匙?」纔待用手去取,被金蓮褪在袖內不與他。說道:「你鑰匙兒,怎落在我手裡?」急得那小夥兒,只是殺雞扯膝。金蓮道:「只說你會唱的好曲兒,倒在外邊舖子里唱與小廝聽,怎的不唱個兒我聽?今日趁着你姥姥和六娘在這裡,只揀眼生好的唱四箇兒,我就與你這鑰匙。不然,隨你就跳上白塔,我也沒有。」經濟道:「這五娘就勒掯出人痞來!誰對你老人家說我會唱的兒?」金蓮道:「你還搞鬼,南京沈萬三,北京枯樹,人的名兒,樹影兒。」那小夥兒吃他奈何不過,說道:「死不了人,等我唱。我肚子裏使心柱肝,要一百個也有!」金蓮罵道:「說嘴的短命!」自把各人面前酒斟上。金蓮道:「你再吃一盃,蓋着臉兒好唱。」經濟道:「我唱了,慢慢吃。我唱菓子花兒,名山坡羊兒你聽:

「初相交,在桃園兒裡結義。相交下來,把你到玉黃李子兒擡舉。人人說你在青翠花家飲酒,氣的我把頻波臉兒,撾的紛紛的碎。我把你賊,你學了虎刺賓了,外實裏虛,氣的我李子眼兒珠淚垂。我使的一對桃奴兒尋你,見你在軟棗兒樹下,就和我別離了去。氣的我鶴頂紅,剪一柳青絲兒來呵!你海東紅,反說我理虧!罵了句牛心紅的強賊,逼的我急了,我在弔枝乾兒上尋個無常,到三秋,我看你倚靠著誰?」

又:

「我聽見金雀兒花,眼前高哨。撇的我鵝毛菊,在斑竹簾兒下喬叫。多虧了二位靈鵲兒報喜。我說是誰來?不想是望江南兒來到。我在水紅花兒下,梳粧未了,狗奶子花迎著門子去咬。我暗使著迎春花兒,遶到處尋你。手搭伏薔薇花,口吐丁香,把我玉簪兒來叫。紅娘子花兒,慢慢把你接進房中來呵!同在碧桃花下鬬了回百草。得了手,我把金盞兒花丟了。曾在轉枝蓮下,纏勾你幾遭。叫了你聲嬌滴滴石榴花兒,你試被九花丫頭傳與十姊妹,什麼張致?可不交人家笑話叉了。」

唱畢,就問金蓮要鑰匙。說道:「五娘,快與了我罷!夥計舖子裡不知怎的等着我哩!只怕一時爹過來。」金蓮道:「你倒自在性兒,說的且是輕巧。等你爹問我,就說你不知在那里吃了酒,把鑰匙不見了,走來俺屋裡尋。」經濟道:「爺嚛!五娘就是弄人的劊子手!」李瓶兒和潘姥姥再三傍邊道:「姐姐與他去罷!」金蓮道:「若不是姥姥和你六娘勸我,定罰教你唱到天晚。頭里騙嘴說一百個二百個。纔唱兩個曲兒,就要騰翅子,我手裡放你不過。」經濟道:「我還有兩個兒看家的,是銀錢名山坡羊,亦發孝順你老人家罷。」于是頓開喉音,唱道:

「冤家你不來,白悶我一月。閃的人反拍著外膛兒,細絲諒不徹。我使獅子頭定兒小廝,拏著黃票兒請你。你在兵部窪兒里,元寶兒家歡娛過夜。我陪銅磐兒家,私為焦心。一旦兒棄捨我,把如同印箝兒印在心里。愁無救解,叫著你,把那挺臉兒高揚著不理。空教我撥著雙火同兒,頓著罐子,等到你更深半夜。氣的奴花銀竹葉臉兒,咬定銀牙來呵!喚官銀,頂上了我房門。隨那潑臉兒冤家,乾敲兒不理。罵了句煎徹了的三傾兒,搗槽斜賊!空把奴一腔子煖汁兒,真心倒與你,只當做熱血!」

又:

「姐姐,你在開元兒家,我和你燃香說誓。我拏著祥道祥元,好黃邊錢也,在你家行三坐四。誰知你香爐拆爪哄我,受不盡你家虔婆鵝眼兒閑氣。你榆葉兒身輕,筆管兒心虛,姐姐你好似古碌錢,身子小,眼兒大,無庄兒可取。自好被那一條棍滑鏝兒油嘴,把你戲耍。脫的你光屁股,把你線邊火漆打硌硌跌澗兒,無所不為。來呵!到明日只弄的倒四顛三,一箇黑沙也是不值。叫了聲二興兒姐姐,你識聽知。可惜我黃鄧鄧的金背,配你這錠難兒一臉褶子。」

經濟唱畢,金蓮纔待叫春梅,斟酒與他。忽有吳月娘從後邊來,見奶子如意兒抱着官哥兒在門首石臺基上坐,便說道:「孩子纔好些,你這狗肉,又抱他在風裡!還不抱進去。」金蓮問:「是誰在說話?」綉春回道:「大娘來了。」經濟慌的拏鑰匙往外走不迭。眾人都下來迎接月娘。月娘便問:「陳姐夫在這里做什麼來?」金蓮道:「李大姐整治些菜請俺娘坐坐。陳姐夫尋衣服,叫他進來吃一盃。姐姐你請坐,好甜酒兒,你吃一盃。」月娘道:「我不吃。後邊他大妗子和楊姑娘要家去。我又記掛着這孩子,逕來看看。李大姐你也不管,又教奶子抱他在風裡坐的。前日劉婆子說他是驚寒,你還不好生看他!」李瓶兒道:「俺每陪着他姥姥吃酒,誰知賊臭人三不知抱他出去了。」月娘坐了半歇,回後邊去了。一回使小玉來請姥姥和五娘、六娘後邊坐。那潘金蓮和李瓶兒勻了臉,伺潘姥姥往後來陪大妗子、楊姑娘吃酒。到日落時分,與月娘送出大門,上轎去了,都在門裡站立。先是孟玉樓說道:「大姐姐,今日他爹不在,往吳驛丞家吃酒去了。咱到好往對門喬大戶家房裡瞧瞧。」月娘問看門的平安兒:「誰拏着那邊鑰匙哩?」平安道:「娘每要過去瞧,開着門哩。來興哥看着兩坌工的在那里做活。」月娘分付:「你教他躲開,等俺每瞧瞧去。」平安兒道:「娘每只顧瞧,不妨事。他每都在第四層大空房撥灰篩土,叫出來就是了。」當下月娘、李嬌兒、孟玉樓、潘金蓮、李瓶兒都用轎子短搬,兩個坌工擡過房子內。進了儀門,就是三間廳,第二層是樓。月娘要上樓去,可是作怪!剛上到樓梯中間,不料梯磴陡趄,只聞月娘哎了一聲,滑下一隻腳來。早是月娘攀住樓梯兩邊欄杆。慌了玉樓,便道:「姐姐怎的?」連忙搊住他一隻胳膊,不曾打下來。女娘乞了一驚,就不上去。眾人扶了下來,諕的臉蠟查兒黃了。玉樓便問:「姐姐,怎麼上來尖了腳,不曾磕着那里?」月娘道:「跌倒不曾跌着,只是扭了腰子,諕的我心跳在口裡。樓梯子趄,我只當咱家裏樓上來,滑了腳,早是攀住欄杆,不然怎了!」李嬌兒道:「你又身上不方便,早知不上樓也罷。」于是眾姊妹,相伴月娘回家。剛到家,叫的應就肚中疼痛。月娘忍不過,趁西門慶不在家,使小廝叫了劉婆子來看。婆子道:「你已是去經事來着傷,多是成不的了。」月娘道:「便是五個多月了,上樓着了扭。」婆子道:「你吃了我這藥,安不住,下來罷了。」月娘道:「下來罷。」婆子于是留了兩大黑丸子藥,教月娘用艾酒吃 。那消半夜,弔下了。在馬桶內,點燈撥看,原來是個男胎,已成形了。正是:「胚胎未能全性命,  真靈先到杳冥天。」幸得那日西門慶來到,沒曾在上房睡,在玉樓房中歇了。到次日,玉樓早辰到上房,問月娘:「身子如何?」月娘告訴:「半夜果然存不住,落下來了,倒是小廝兒。」玉樓道:「可惜了的,他爹不知道?」月娘道:「他爹吃酒來家,到我屋里,纔得脫衣裳,我說你往他每屋里去罷,我心裡不自在。他纔往你這邊來了。我沒對他說。我如今肚裡,還有些隱隱的疼。」玉樓道:「只怕還有些餘血未盡,篩酒吃些鍋臍灰兒,就好了。」又道:「姐姐,你還計較兩日兒。且在屋裡,不可出去,小產比大產還難調理。只怕掉了風寒,難為你的身子。」月娘道:「你沒的說,倒沒的倡揚的一地里知道。平白噪剌剌的抱什麼空窩,惹的人動的唇齒。」以此就沒教西門慶知道此事。表過不題。且說西門慶新搭的開絨線舖夥計,也不是守本分人。姓韓,名道國,字希堯,乃是破落戶韓光頭的兒子。如今跌落下來,替了大爺的差使,亦在鄆王府做校尉。見在縣東街牛皮小巷居住。其人性本虛飄,言過其實,巧于詞色,善于言談。許人錢如捉影捕風;騙人財如探囊取物。因此街上人見他是般說謊,順口叫他做韓盜國。自從西門慶家做了買賣,手裡財帛從容,新做了幾件虼蚫皮,在街上虛飄說詐。掇着肩膊兒,就搖擺起來。人見了,不叫他個韓希堯,只叫他做韓一搖。他渾家乃是宰牲口王屠妹子,排行六姐,生的長挑身材,瓜子面皮,紫膛色,約二十八九年紀。身上有個女孩兒,嫡親三口兒度日。他兄弟韓二,名二搗鬼,是個耍手的搊子,在外另住。舊與這婦人有姦,要使趕韓道國不在家,舖中上宿,他便時常走來,與人吃酒,到晚夕刮涎就不去了。不想街坊有幾個浮浪子弟,見婦人搽脂抹粉,打扮喬樣,常在門首站立睃人。人略鬬他鬬兒,又臭又硬,就張致罵人;因此街坊這些小夥子兒,心中有幾分不憤,暗暗三兩成群,背地講論,看他背地與什麼有首尾。那消半個月,打聽出與他小叔韓二這件事來。原來韓道國在牛皮小巷住着,門面三間,房裡兩邊都是鄰舍,後門通水塘。這夥人單看韓二進去,或倩老嫗灑堂,或夜晚扒在牆上看覷,或白日裡暗使小猴子,在後堂推道捉蛾兒,單等捉姦。不想那日,二鬼打聽他哥不在,大白日裝酒,和婦人吃醉了,倒插了門在房裡幹事。不防眾人睃見蹤跡,小猴子扒過來,把後門開了。眾人一齊進去,掇開房門。韓二奪門就走,被一少年一拳打倒拏住。老婆還在炕上慌衣不迭,一人進去,先把褲子撾在手裡,都一條繩子拴出來。須臾,圍了一門首人,跟到牛皮街廂舖里,就哄動了那一條街巷。這一個來問,那一個來瞧,都說韓道國婦人與小叔犯姦。內中見男婦二人拴做一處,便問左右站的人:「此是為什麼事的?」旁邊有多口的道:「你老人家不知,此是小叔姦嫂子的。那老者點了點頭兒,說道:「可傷!原來小叔兒要嫂子的。到官,叔嫂通姦,兩個都是絞罪。」那旁多口的,認的他有名叫做陶扒灰,一連娶三個媳婦,都吃他扒了。因此插口說道:「你老人家深通條律,相這小叔嫂子的,便是絞罪;若是公公養媳婦的,都論什麼罪?」那老者見不是話,低着頭,一聲兒沒言語走了。正是:

「各人自掃簷前雪,  莫管他家屋上霜。」

這里二搗鬼與婦人被捉不題。單表那日韓道國舖子里不該上宿,來家早。八月中旬天氣,身上穿着一套兒輕紗軟絹衣服,新盔的一頂帽兒,細網巾圈,玄色段子履鞋,清水絨襪兒,搖着扇兒,在街上闊行大步,搖擺走着。但遇着人,或坐或立,口若懸河,滔滔不絕,就是一回。內中遇着他兩個相熟的人,一個是開紙舖的張二哥,一個是開銀舖的白四哥,慌作揖舉手。張好問便道:「韓老兄連日少見,聞得恭喜在西門大官府上開寶舖做買賣,我等缺禮失賀,休怪,休怪!」一面讓他坐下。那韓道國坐在凳上,把臉兒揚着,手中搖着扇兒,說道:「學生不才,仗賴列位餘光,在我恩主西門大官人做夥計,三七分錢。掌巨萬之財,督數處之舖。其蒙敬重,比他人不同。」有謝汝謊道:「聞老兄在他門下做,只做線舖生意?」韓道國笑道:「二兄不知。線舖生意,只是名目而已。今他府上大小買賣,出入貲本,那些兒不是學生算帳?言聽計從,禍福共知。通沒我,一時兒也成不得。初大官人每日衙門中來家擺飯,常請去陪侍。沒我便吃不下飯去;俺兩個在他小書房裡,閑中吃菓子說話兒。常坐半夜,他方進後邊去。昨日他家大夫人生日,房下坐轎子,行人情,他夫人留飲至二更方回。彼此通家,再無忌憚。不可對兄說,就是背地他房中話兒,也常和學生計較。學生先一個行端莊,立心不苟,與財主興利除害,拯溺救焚。凡百財上分明,取之有道,就是傅自新,也怕我幾分。不是我自己誇獎,大官人正喜我這一件兒。」剛說在鬧熱處,忽見一人慌慌張張,走向前,叫道:「韓大哥,你還在這裡說什麼?教我舖子裡尋你不着?」拉到僻靜處告他說:「你家中如此如此,這般這般。大嫂和二哥,被街坊眾人撮弄兒,拴到舖裡,明早要解縣見官去。你還不早尋人情,理會此事?」這韓道國聽了,大驚失色,只中只咂嘴,下邊頓足,就要翅趫走。被張好問叫道:「韓老兄,你話還未盡,如何就去了?」這韓道國舉手道:「學生家有小事,不及奉陪。」慌忙而去。正是:

「誰人挽得西江水,  難洗今朝一面羞。」

畢竟未知後來何如,且聽下回分解:


