%# -*- coding: utf-8 -*-
% !TeX encoding = UTF-8 Unicode
% !TeX spellcheck = en_US
% !TeX TS-program = xelatex
%~ \XeTeXinputencoding "UTF-8"
% vim:ts=4:sw=4
%
% 以上設定默認使用 XeLaTex 編譯,並指定 Unicode 編碼,供 TeXShop 自動識別

%第十二回
\chapter{潘金蓮私僕受辱\KG 劉理星魘勝貪財}

\begin{showcontents}{}



「堪笑西門暴富,  有錢便有主顧,

一家歪斯胡纏,  那討綱常禮數;

狎客日日來往,  紅粉夜夜陪宿,

不是常久夫妻,  也筭春風一度。」

話說西門慶在院中,貪戀住桂姐姿色,約半月不曾來家。吳月娘使小廝一連拏馬接了數次,李家把西門慶衣帽都藏過一邊,不放他起身。丟的家中這些婦人,都閑靜了。到別人猶可,惟有潘金蓮這婦人,青春未及三十歲,慾火難禁一丈高。每日和孟玉樓兩箇,打扮粉粧玉琢,皓齒朱唇,無一日不走在大門首倚門而望,等到黃昏時分。到晚來,歸入房中粲枕孤幃,鳳臺無伴,睡不着,走來花園中,款步花苔,月洋水底。猶恐西門慶心性難拏,怪玳瑁貓兒交懽,鬬的我芳心迷亂。當時玉樓帶來一箇小廝,名喚琴童,年約十六歲,纔留起頭髮。生的眉目清秀,乖滑伶俐。西門慶教他拿鑰匙看管花園打掃,晚夕就在花園門前一間小耳房內歇。潘金蓮和孟玉樓白日裡常在花園中亭子上坐在一處做針指,或下棋。這小廝專一道小慇懃,常觀見西門慶來,就先來告報。以此婦人喜他,常叫他入房,賞酒與他吃。兩箇朝朝暮暮,眉來眼去,都有意了。不想將近七月廿八日,西門慶生日來到。吳月娘見西門慶在院中留戀烟花,不想回家,一面使小廝玳安拏馬往院中接西門慶。這潘金蓮暗暗修了一柬帖,交付玳安,教:「悄悄遞與你爹,說五娘請爹早些家去罷。」這玳安不敢怠慢,騎馬一直到构攔李家。只見應伯覺、謝希大、祝日念、孫寡嘴、常時節眾人,正在那裡相伴着西門慶,摟着粉頭,花攢錦簇,懽樂飲酒。西門慶看見玳安來到,便問:「你來怎麼?家中沒事?」玳安道:「家中沒事。」西門慶道:「前邊各項銀子,叫傅二叔討討,等我到家算帳。」玳安道:「這兩日傅二叔討了許多,等爹到家上帳。」西門慶道:「你桂姨那一套衣服,稍來不曾?」玳安道:「已稍在此。」便向毡包內取出一套紅衫藍裙,遞與桂姐。桂姐、桂卿道了萬福收了。連忙分付下邊,管待玳安酒飯。那小廝吃了酒飯,復走來上邊伺候。悄悄向西門慶耳邊附耳低言,說道:「家中五娘,使我稍了箇帖兒在此,請爹早些家去。」西門慶纔待用手去接,早被李桂姐看見。只道是西門慶前邊那表子寄來的情書,一手撾過來,拆開觀看,卻是一幅迴文邊錦箋,上寫着幾行墨跡。桂姐遞與祝日念,教念與他聽。這祝日念見上面寫詞一首,名落梅風,對眾朗誦了一遍:

「黃昏想,白日想,盼殺人多情不至。因他為他憔悴死,可憐也繡衾獨自!燈將殘,人睡也,空留得半窗的月。 孤眠衾硬渾似鐵,這淒涼怎捱今夜?下書愛妾潘六兒拜。」

那桂姐聽畢,撇了酒席,走入房中,倒在床上,面朝裡邊睡了。且說西門慶,見桂姐惱了,把帖子扯的稀爛。眾人前把玳安踢了兩靴腳,請桂姐兩遍不來,慌的西門慶親自進房內,抱出他來。到酒席上,說道:「吩咐帶馬回去,家中那箇淫婦使你來,我這一到家都打箇臭死!」不說玳安含淚回家。西門慶道:「桂姐,你休惱,這帖子不是別人的,乃是舍下第五箇小妾頭,寄請我到家,有些事兒計較,再無別故。」祝日念在旁,又戲道:「桂姐,你休聽他哄你哩!這個潘六兒,乃是那邊院裡新敍的一箇表子,生的一表人物,你休放他去。」西門慶笑趕着打,說道:「你這賊天殺的!單管弄死了人!緊着他恁麻犯人,你又胡說!」李桂卿道:「姐夫差了,既然家中有人拘管,就不消在外面梳攏人家粉頭,自守着家裡的便了。纔相伴了多少時,那人兒便就要拋離了去!」應伯爵插口道:「說的有理。」便道:「大官人你依我,你也不消家去;桂姐也不必惱。今日說過,那箇再恁惱了,每人罰二兩銀子,買酒肉咱大家吃。」到是這四五箇敗客,說的說,笑的笑,在席上猜枚行令,頑耍飲酒,把桂姐窩盤住了。西門慶把桂姐摟在懷中倍笑,一遞一口兒飲酒,只見少頃,鮮紅漆丹盤拿了七鍾茶來。雪綻般茶盞,杏葉茶匙兒,鹽筍芝蔴木樨泡茶 ,馨香可掬,每人面前一盞。應伯爵道:「我有箇朝天子兒,單道這茶好處!」:

「這細茶嫩芽,生長在春風下,不揪不採葉兒楂;但煮著顏色大。絕品清奇,難畫。口兒裡常時呷,醉了時想他,醒來時愛他。原來一簍兒千金價!」

謝希大笑道:「大官人使錢費物,不圖這一摟兒,卻圖些甚的?如今每人有詞的唱詞,不會詞,每人說箇笑話兒,與桂姐下酒。」該謝希大先說:「有一箇泥水匠,在院中謾地;老媽兒怠慢着他些兒,他暗暗把陰溝內堵上箇磚。落後天下雨,積的滿院子都是水;老媽慌了,尋的他來,多與他酒飯,還秤了一錢銀子,央他打水平。那泥水匠吃了酒飯,悄悄去陰溝內,把那箇磚拿出,把水登時出的罄盡。老媽便問作頭:『此是那裡的病?』泥水匠回道:『這病與你老人家病一樣,有錢便流,無錢不流。』」原來把桂姐家來傷了,桂姐道:「我也有箇笑話,回奉列位。有一孫真人,擺着筵席請人,卻教座下老虎去請,那老虎把客人一箇箇都路上吃了,真人等至天晚,不見一客到。人都說你那老虎,都把客人路上吃了。不一時,老虎來,真人便問:『你請的客人都往那裡去了?』老虎口吐人言:『告師父得知,我從來不曉得請人,只會白嚼人,就是一能。』」當下把眾人都傷了。應伯爵道:「可見的俺每,只自白嚼你家孤老,就還不起箇東道。」于是,向頭上拔下一根鬧銀耳幹兒來,重一錢,謝希大一對鍍金網巾圈,秤了秤,只九分半,祝日念袖中掏出一方舊汗巾兒,算二百文長錢;孫寡嘴腰間解下一條白布男裙,當兩壺半壜酒;常時節無以為敬,問西門慶借了一錢成色銀子;都遞與桂卿置辦東道,請西門慶和桂姐。那桂卿將銀錢都付與保兒,買了一錢螃蟹,打了一錢銀子豬肉,宰了一隻雞,自家又賠出些小菜兒來。廚下安排停當,大盤小碗拿上來。眾人坐下,說了一聲動筯吃時,說時遲,那時快,但見:

「人人動嘴,箇箇低頭。遮天映日,猶如蝗喃一起來;擠眼裰肩,好似餓牢纔打出。這箇搶風膀臂,如經未見酒和餚;那箇連二快子,成歲不逢筵與席。一箇汗流滿面,恰似與雞骨朵有冤仇;一箇油抹唇邊,把豬毛皮連唾嚥。吃片時,盃盤狼藉;啖良久,筯子縱橫。盃盤狼藉,如水洗之光滑;筯子縱橫,似打磨之乾淨。這箇稱為食王元帥,那箇號作淨盤將軍。酒壺番晒又重斟,盤饌已無還去探。正是:珍羞百味片時休,果然都送入五臟廟。」

當下眾人吃了箇淨光王佛。西門慶與桂姐吃不上兩鍾酒,揀了些菜蔬,還被這夥人吃的去了。那日把席上椅子坐折了兩張,前邊跟馬的那小廝,不得上來掉嘴吃。把門前供養的土地,翻倒來使位恰【扌利】了一泡【禾囗也】谷都的熱屎。臨出門來,孫寡嘴把李家明間內供養的鍍金銅佛,塞在褲腰裡。應伯爵推鬬桂姐親嘴,把頭上金啄針兒戲了。謝希大把西門慶川扇兒藏了。祝日念走到桂卿房裡照臉,溜了他一面水銀鏡子。常時節借的西門慶一錢八成銀子,竟是寫在敗帳上了。原來這起人,只伴着西門慶頑耍,好不快活。有詩為證:

「构欄妓者媚如揉,  只堪乘興暫時留;

若要死貪無足厭,  家中金鑰教誰收。」

按下這裡眾人簇擁着西門慶歡樂飲酒。單表玳安小廝回馬到家,吳月娘和孟玉樓、潘金蓮在房坐的,見了玳安,便問:「你接了爹來了不曾?」玳安哭的兩眼紅紅的,如此這般:「被爹踢罵了小的來了!說道那箇再使人接,來家都要罵。」月娘便道:「你看,不合理!不來便了,如何去罵小廝來?如何狐迷變心這等的!」孟玉樓道:「你踢將小廝便罷了,如何連俺們都罵將來?」潘金蓮道:「十箇九箇院中淫婦,和你有甚情實?常言說的好:『船載的金銀,填不滿烟花寨。』」金蓮只知說出來,不防路上說話,草裡有人。李嬌兒從玳安自院中來家時分,走來窗下潛聽。見潘金蓮對着月娘罵他家千淫婦,萬淫婦,暗暗懷恨在心。從此二人結仇,不在話下。正是:

「甜言美語三冬煖,  惡語傷人六月寒;

金蓮只曉爭先話,  那料旁人起禍端。」

不說李嬌兒與金蓮結仇。單表金蓮這婦人歸到房中,捱一刻似三秋,盼一時如半夏。知道西門慶不來家,把兩箇丫頭打發睡了。推往花園中遊翫,將琴童叫進房,與他酒吃,把小廝灌醉了,掩閉了房門,褪衣解帶,兩箇就幹做在一處。正是:

「色膽如天怕甚事,  鴛幃雲雨百年情。」

但見:

「一箇不顧綱常貴賤,一箇那分上下高低。一箇色膽歪邪,管甚丈夫利害;一箇淫心蕩漾,從他律犯明條。一箇氣暗眼瞪,好似牛吼柳影;一箇言嬌語澀,渾如鶯囀花間。一箇耳畔許雨意雲情,一箇枕邊說山盟海誓。百花園內,翻為快活排場;主母房中,變作行樂世界。霎時一滴驢精髓,傾在金蓮玉體中。」

自此為始,每夜婦人便叫這小廝進房中如此。未到天明,就打發出來。背地把金裹頭簪子兩三根,帶在頭上,又把裙邊帶的錦香囊股子葫蘆兒,也與了他,繫在身底下。豈知這小廝不守本分,常常和同行小廝在街吃酒耍錢,頗露出圭角。常言:「若要人不知,除非己莫為。」有一日,風聲吹到孫雪娥、李嬌兒耳朵內,說道:「賊淫婦!往常言語假撇清,如何今日也做出來了!偷養小廝!」齊來告月娘。月娘再三不信,說道:「不爭你們和他合氣,惹的孟三姐不怪,只說你們擠撮他的小廝。」說的二人無言而退。落後,婦人夜間和小廝在房中行事,忘記關廚房門,不想被丫頭秋菊出來淨手,看見了。次日傳與後邊小玉,小玉對雪娥說,雪娥同李嬌兒,又來告訴月娘,正值七月廿七日,西門慶上壽,從院中來家。二人如此這般:「他屋裡丫頭,親口說出來,又不是俺們葬送他。大娘不說,俺們對他爹說;若是饒了這箇淫婦,自除非饒了蝎子娘是的!」月娘道:「他纔來家,又是他好日子,你每不依我,只顧說去。等住回亂將起來,我不管你。」二人不聽月娘之言,約的西門慶進入房中,齊來告訴,說金蓮在家養小廝一節。這西門慶,不聽萬事皆休,聽了怒從心上起,惡向膽邊生。走到前邊坐下,一片聲叫琴童兒。早有人報與潘金蓮,金蓮慌了手腳,使春梅忙叫小廝到房中,囑咐:「千萬不要說出來!」把頭上簪子都要過來收了。着了慌,就忘下解了香囊葫蘆下來。被西門慶叫到前廳跪下,吩咐三四箇小廝,選大板子伺候。西門慶道:「問賊奴才!你知罪麼?」那琴童半日不敢言語。西門慶令左右:「除了帽子,拔下他簪子來我瞧!」見撇着兩根金裹頭銀簪子,因問:「你戴的金裹頭銀簪子往那裡去了?」琴童道:「小的並沒甚銀簪子。」西門慶道:「奴才!還搗鬼,與我旋剝了衣服,拿板子打!」當下兩三箇小廝扶侍,一箇剝去他衣服,扯了褲子,見他身底下穿着玉色絹〈衤旋〉兒,〈衤旋〉兒帶上,露出錦香囊葫蘆兒。西門慶一眼就看見,便叫:「拏上來我瞧!」認的是潘金蓮裙邊帶的物件,不覺心中大怒,就問他:「此物從那裡得來?你實說,是誰與你的?」諕的小廝半日開口不得,說道:「這是小的某日打掃花園,在花園內拾的,並不曾有人與我。」西門慶越怒切齒,喝令:「與我綑起,着實打。」當下把琴童兒綳子綳着,雨點般欖杆打將下來。須臾打了三十大棍,打得皮開肉綻,鮮血順腿淋漓,又教大家人來保:「把奴才兩箇鬢與我撏了,趕將出去,再不許進門!」那琴童磕了頭,哭哭啼啼出門去了。這小廝只因:

「昨夜與玉皇殿上掌書仙子廝調戲,  今日罪犯天條貶下方。」

有詩為證:

「虎有倀弓鳥有媒,  金蓮未必守空閨;

不堪今日私奴僕,  自此遭愆更莫追。」

當下西門慶打畢琴童,趕出去了。潘金蓮在房中聽見,如提冷水盆內一般。不一時,西門慶進房來,諕的戰戰兢兢,渾身無了脈息,小心在旁扶侍接衣服,被西門慶兜臉打了箇耳刮子,把婦人打了一交。吩咐春梅:「把前後角門頂了,不放一箇人進來!」拿張小椅兒坐在院內花架兒底下,取了一根馬鞭子,拏在手裡,喝令:「淫婦,脫了衣裳跪着!」那婦人自知理虧,不敢不跪。到是真箇脫去了上下衣服,跪在前面,低垂粉面,不敢出一聲兒。西門慶便問:「賊淫婦,你休推睡裡夢裡,奴才我纔已審問明白,他一一都供出來了!你實說,我不在家,你與他偷了幾遭?」婦人便哭道:「天麼,天麼!可不冤屈殺了我罷了!自從你不在家,半箇來月,奴白日裡只和孟三姐做一處做針指。到晚夕早關了房門就睡了,沒勾當不敢出這角門邊兒來。你不信,只問春梅便了。有甚和鹽和醋,他有箇不知道的?」因叫春梅來:「姐姐你過來,親對你爹說。」西門慶罵道:「賊淫婦!有人說你把頭上金裹頭簪子兩三根,都偷與了小廝,你如何不認?」婦人道:「就屈殺了奴罷了!是那箇不逢好死的,嚼舌根的淫婦,嚼他那旺跳的身子!見你常時進奴這屋裡來歇,非都氣不憤,拏這有天沒日頭的事壓枉奴!就是你與的簪子,都有數兒,一五一十都在,你查不是?我平日想起甚麼來,與那奴才?好成楫的奴才不枉說的。行一箇尿不出來的毛奴才,平空把我纂一篇舌頭!」西門慶道:「簪子有沒罷了。」向袖中取出琴童那香囊來,說道:「這箇是你的物件兒,如何打小廝身底下捏出來?你還口漒甚麼?」說着紛紛的惱了,向他白馥馥香肌上,颼的一馬鞭子來,打的婦人疼痛難忍!眼噙粉淚,沒口子叫道:「好爹爹,你饒了奴罷!你容奴說,奴便說。不容奴說,你就打死奴,也只臭烟了這塊地。這箇香囊葫蘆兒,你不在家,奴那日同孟三姐在花園裡做生活,因從木香欄下所過,帶繫兒不牢,就抓落在地。我那裡沒尋,誰知這奴才拾了,奴並不曾與他。」只這一句,就合着剛纔琴童前廳上供稱,在花園內拾的一樣的話。又見婦人脫的光赤條條,花朵兒般身子,嬌啼嫩語,跪在地下,那怒氣早已鑽入瓜哇國去了。把心已回動了八九分,因叫過春梅,摟在懷中,問他:「淫婦果然與小廝有首尾沒有?你說饒了淫婦,我就饒了罷!」那春梅撒嬌撒痴,坐在西門慶懷裡。說道:「這箇,爹,你好沒的說!和娘成日唇不離腮,娘肯與那奴才?這箇都是人氣不憤俺娘兒們,作做出這樣事來。爹你也要個主張,好把醜名兒頂在頭上,傳出外邊去好聽。」幾句把西門慶說的一聲兒不言語,丟了馬鞭子,一面教金蓮起來穿上衣服,吩咐秋菊看菜兒,放桌兒吃酒。這婦人當下滿斟了一盃酒,雙手遞上去。花枝招颭,繡帶飄飄,跪在地下,等他鍾兒。西門慶吩咐道:「我今日饒了你,我若但凡不在家,要你洗心改正,早關了門戶,不許你胡思亂想。我若知道,定不饒你!」婦人道:「你吩咐,奴知道了。」到是插燭也似與西門慶磕了四箇頭,方纔安座兒,在旁陪坐飲酒。正是:

「為人莫作婦人身,百年苦樂由他人。」

潘金蓮這婦人,平日被西門慶寵的狂了,今日討得這場羞辱在身上。有詩為證:

「金蓮容貌更溫柔,  恃寵爭妍惹寇仇;

不是春梅當日勸,  父孃皮肉怎禁抽。」

西門慶正在金蓮房中飲酒,忽聽小廝打門,說:「前邊有吳大舅、吳二舅、傅夥計女兒、女婿、眾親戚,送禮來祝壽。」方纔撇了金蓮,整衣出來前邊陪待賓客。那時應伯爵、謝希大等眾人,都有人情。院中李桂姐家,亦使保兒送禮來。西門慶前邊亂着,收人家禮物,發柬請人,不在話下。且說孟玉樓打聽金蓮受辱,約的西門慶不在家裡,瞞着李嬌兒、孫雪娥走來看望金蓮。見金蓮睡在床上,因問道:「六姐,你端的怎麼緣故?告我說則箇。」那金蓮滿眼流淚,哭道:「三姐,你看小淫婦,今日在背地裡白唆調漢子,打了我恁一頓,我到明日和這兩箇淫婦,冤仇結的有海深!」玉樓道:「你便與他有瑕玷,如何做作着把我的小廝弄出去了?六姐,你休煩惱,莫不漢子就不聽俺每說句話兒?若明日他不進我房裡來便罷,但到我房裡來,等我慢慢勸他。」金蓮道:「多謝姐姐費心。」一面叫春梅看茶來吃,坐着說了回話。玉樓告辭回房去了。至晚,西門慶因上房吳大娘子來了,走到玉樓房中宿歇。玉樓因說道:「你休枉了六姐心,六姐並無此事。都是日前和李嬌兒、孫雪娥兩箇有言語,平白把我的小廝扎罰子。你不問了青紅皂白,就把他屈了。你休怪六姐,卻不難為六姐了。我就替他賭了大誓,若果有此事,大姐姐有箇不先說的?」西門慶道:「我問春梅,他也是般說。」玉樓道:「他今在房中不好哩!你不去看他看去。」西門慶道:「我知道,明日到他房中去。」當晚無話。到第二日,西門慶正生日,有周守備、夏提刑、張團練、吳大舅許多官客飲酒。拏轎子接了李桂姐,并兩箇唱的,唱了一日。李嬌兒見他姪女兒來,引着拜見月娘眾人,在上房裡坐吃茶。請潘金蓮見,連使丫頭請了兩遍,金蓮不出來,只說心中不好。到晚夕桂姐臨家去,拜辭月娘,月娘與他一件雲絹比甲兒、汗巾花翠之類,同李嬌兒送出到門首。桂姐又親自到他花園角門首:「好歹見見五娘。」那金蓮聽見他來,使春梅把角門關閉,煉鐵桶相似,就是樊噲也叫不開。說道:「我不開!」這花娘遂羞訕滿面而回。正是:

「廣行方便,  為人何處不相逢?

多結冤仇,  路逢狹處難回避。」

不題李桂姐回家去了。單表西門慶至晚進入金蓮房內來,那金蓮把雲鬟不整,花容倦淡,迎接進房,替他脫衣解帶,伺候茶湯腳水,百般慇懃扶侍,把小意定貼戀。到夜裡,枕蓆魚水歡愉,屈身忍辱,無所不至。說道:「我的哥哥,這一家都誰是疼你的?都是露水夫妻,再醮貨兒!惟有奴知道你的心,你知道奴的意。旁人見你這般疼奴,在奴身邊去的多,都氣不憤。背地裡架舌頭,在你根前唆調。我的傻冤家,你想起甚麼來!中了人的拖刀之計,把你心愛的人兒,這等下無情折剉!常言道:『家雞打的團團轉,野雞打的貼天飛。』你就把奴打死了,也只在這屋裡,敢往那裡去?就是前日你在院裡,踢罵了小廝來,早時有上房大姐姐、孟三姐在根前,我是不是說了一聲也是好的。恐怕他家裡粉頭,淘淥壞了你身子。院中唱的,只是一味愛錢。你有甚情節,誰人疼你?誰知被有心的人聽見,兩箇背地伯成一幫兒算計我。自古人害人不死,天害人纔害死了!往後久而自明。只要你與奴做箇主兒便了。」于是幾句把西門慶說的窩盤住了,是夜與他淫慾無度。到次日,西門慶備馬,玳安、平安兩箇小廝跟隨,往院中來。卻說李桂姐正打扮着陪人坐的,聽見他來,連忙走進房去,洗了濃粧,除了簪環,倒在床上,裹衾而臥。西門慶走到,坐了半日,還沒一箇出來陪侍。只見老媽出來,道了萬福,讓西門慶坐下。虔婆便問:「怎的姐夫,連日不進來走走?」西門慶道:「正是因賤日窮冗,家中無人。」虔婆道:「姐兒那日打擾!」西門慶道:「怎的那日姐姐桂卿不來走走?」虔婆道:「桂卿不在家,被客人接去店裡,這幾日還不放了來。」說了半日話,小頂人拿茶來,陪着吃了。西門慶便問:「怎的不見桂姐?」虔婆道:「姐夫還不知哩!小孩兒家不知怎的那日着了惱來家,就不好起來,睡倒了。房門兒也不出,直到今日。姐夫好狠心,也不來看看姐兒!」西門慶道:「真箇?我通不知。」因問:「在那邊房裡?我看看去。」虔婆道:「在他後邊臥房裡睡。」慌忙令丫鬟掀簾子,西門慶走到他房中,只見粉頭烏雲散亂,粉面慵粧,裹被便坐在那床上,面朝裡。見了西門慶,不動一動兒。便問道:「你那日來家,怎的不好?」也不答應。又問:「你着了誰人惱?你告我說。」問了半日,那桂姐方開言說,說道:「左右是你家五娘子!你家中既有恁好的,迎歡買俏,又來稀罕俺們這樣淫婦做甚麼?俺們雖是門戶中出身,蹺起腳兒,比外邊良人家不成的貨兒高好些!我前日又不是供唱,我也送人情去。大娘倒見我甚是親熱,又那兩箇與我許多花翠衣服。待要不請你見,又說俺院中沒禮法。只聞知人說你家有的了五娘子,當能請你拜見,又不出來。家來,同俺姑娘又辭你去,你使丫頭把房門關了。端的好不識人敬重!」西門慶道:「你倒休怪他!他那日本等心中不自在。他若好時,有箇不出來見你的?這個淫婦,我幾次因他再三咬裙兒口嘴傷人,也要打他哩!」這桂姐兒反手向西門慶一掃,說道:「沒羞的哥兒,你就打他!」西門慶道:「你還不知我手段。除了俺家房下,家中這幾箇老婆丫頭,但打起來,也不善着。緊二三十馬鞭子,還打不下來,好不好還把頭髮都剪了。」桂姐道:「我見砍頭的,沒見砍嘴的!你打三箇官兒唱兩箇喏,誰見來?你若有本事到家裡,只剪下一料子頭髮,拏來我瞧,我方信你是本司三院,有名的好子弟!」西門慶道:「你敢與我排手?」那桂姐道:「我和你排一百箇手!」當日西門慶在院中歇了一夜。到次日黃昏時分,辭了桂姐,上馬回家。桂姐道:「我在這裡眼望旌節旗,耳聽好消息。哥兒你這一去,沒有這物件,就休要見我!」這西門慶吃他激怒了幾句話,歸家已是酒酣。不往別房裡去,逕到前邊潘金蓮房來。婦人見他有酒了,加意用心伏侍。問他酒飯,都不吃。吩咐春梅把床上拭抹涼蓆乾淨,帶上門出去,他便坐在床,令婦人脫靴,那婦人不敢不脫。須臾脫了靴,打發他上床。西門慶且不睡,坐在一隻枕頭上,令婦人褪了衣服,地下跪着。那婦人諕的捏兩把汗,又不知因為甚麼,于是跪在地下,柔聲大哭道:「我的爹爹,你透與奴箇伶俐說話,奴死也甘心!饒奴終夕恁提心吊膽,陪着一千箇小心,還投不着你的機會。只拏鈍刀子鋸處我,教奴怎生吃受?」西門慶罵道:「賊淫婦!你真箇不脫衣裳,我就沒好意了!」因叫春梅:「門背後有馬鞭子,與我取了來!」那春梅只顧不進房來。叫了半日,纔慢條斯禮,推開房門進來。看見婦人跪在床地平上,向燈前倒着桌兒下了油,西門慶使他,只不動身,婦人叫道:「春梅,我的姐姐!你救我救兒!他如今要打我。」西門慶道:「小油嘴兒!你不要管他。你只遞馬鞭子與我,打這淫婦!」春梅道:「爹你怎的恁沒羞!娘幹壞了你的甚麼事兒?你信淫婦言語來?平地裡起風波。要便搜尋娘,還教人和你一心一計哩!你教人有刺眼兒看得上你,倒是也不依他!」拽上房門,走在前邊去了。那西門慶無法可處,反呵呵笑了,向金蓮道:「我且不打你,你上來。我問你要樁物兒,你與我不與我?」婦人道:「好親親,奴一身都骨朵肉兒,都屬了你。隨要甚麼,奴無有不依隨的。不知你心裡要甚麼兒?」西門慶道:「我心要你頂上一柳兒好頭髮。」婦人道:「好心肝,淫婦的身上,隨你怎的揀着燒遍了也依,這箇剪頭髮卻成不的,可不諕死了我罷了!奴出娘胞兒,活了二十六歲,從沒幹這營生,打緊我頂上這頭髮,近來又脫了奴好些,只當可憐見我罷!」西門慶道:「你只嗔我惱,我說的你就不依我。」婦人道:「我不依你再依誰?」因文問:「你實對奴說,要奴這頭髮做甚麼去?」西門慶道:「我要做網巾。」婦人道:「你要做網巾,我就與你做。休要拏與淫婦,教他好壓鎮我。」西門慶道:「我不與人便了,要你髮兒做頂線兒。」婦人道:「你既要做頂線,待奴剪與你。」當下婦人分開頭髮,西門慶拏剪刀,按婦人當頂上,齊臻臻剪下一大梆來,用紙包放在順袋內。婦人便倒在西門慶懷中,嬌聲哭道:「奴凡事依你,只願你休忘了心腸。隨你前邊和人好,只休拋閃了奴家。」是夜,與他歡會異常。到次日,西門慶起身,婦人打發他吃了飯出門,騎馬逕到院裡。桂姐便問:「你剪的他頭髮在那裡?」西門慶道:「有,在此。」便向茄袋內取出,遞與桂姐。打開觀看,果然黑油也一般好頭髮,就收在袖中。西門慶道:「你看了還與我,他昨日為剪這頭髮,好不費難。吃我變了臉惱了,他纔容我剪下這一梆子來。我哄他只說要做網巾頂線兒,逕拏進來與你瞧,可見我不失信。」桂姐道:「甚麼稀罕貨!慌的你恁箇腔兒。等你家去,我還與你,比是你恁怕他,就不消剪他的來了!」西門慶笑道:「那裡是怕他的,我語言不的了。」桂姐一面教桂卿陪著他吃酒,走到背地裡,把婦人頭髮早絮在鞋底下,每日躧踏,不在話下。到是把西門慶纏住,連過了數日,不放來家。金蓮自從頭髮剪下之後,覺意心中不快。每日房門不出,茶飯慵餐。吳月娘使小廝請了家中常走看的那劉婆子看視,說:「娘子着了些暗氣,惱在心中,不能回轉。頭疼惡心,飲食不進。」一面打開藥包來,留了兩服黑丸子藥兒:「晚上用薑湯 吃。」又說:「我明日叫俺老公來,替你老人家看看今歲流年,有災沒有?」金蓮道:「原來你家老公,也會算命?」劉婆道:「他雖是箇瞽目人,到會兩三樁本事:第一,善陰陽講命,與人家禳保;第二,會針炙收瘡;第三樁兒不可說,單管與人家回背。」婦人問道:「怎麼是回背?」劉婆子道:「如何有父子不和,兄弟不睦,大妻小妻爭鬬,教了俺老公去說了,替他用鎮物安鎮,鎮書符水,與他吃了,不消三日,教他父子親熱,兄弟和睦,妻妾不爭。若人家買賣不順溜,田宅不興旺者,常與人開財門、發利市、治病洒掃、禳星告斗都會。因此人都叫他做劉理星。也是一家子新娶箇媳婦兒,是小人家女兒,有些手腳兒不穩,常偷盜婆婆家東西,往娘家去。丈夫知道,常被責打。俺老公與他回背,書了二道符,燒灰放在水缸下埋着。渾家大小吃了缸內水,眼看着媳婦偷盜,只相沒看見一般。又放一件鎮物在枕頭,男子漢睡了那枕頭,也好似手封住了的,再不打他了。」那潘金蓮聽見,遂留心,便叫丫頭打發茶湯點心與劉婆吃了。臨去包了三錢藥錢,另外又秤了五錢,教買紙劄信物,明日早飯時,叫劉瞎來燒神紙,那劉婆子作辭回家。到次日,果然大清早晨,領賊瞎逕進大門,往裡走。那日西門慶還在院中未來。看門小廝便問:「瞎子往那裡走?」劉婆道:「今日與裡邊五娘燒紙。」小廝道:「既是與五娘燒紙,老劉你領進去,仔細看狗。」這婆子領定,逕到潘金蓮臥房明間內。等到半日,婦人纔出來,瞎子見了禮,坐下。婦人說與他八字,賊瞎子用手搯了搯,說道:「娘子庚辰年庚寅月乙亥日,巳丑時,初八日立春,已交正月算命。依子平正論,娘子這八字中,雖故清奇,一生不得夫星濟。子上有些妨礙,亥中一木,生到正月間,不作身旺論,不尅當自焚。又兩重庚金羊刃,大重。夫星難為,尅過兩箇纔好。」婦人道:「已尅過了。」賊瞎子道:「娘子這命中,休怪小人說,子平雖取煞印格。只吃了亥中有癸水,庚中又有癸水。水太多了,冲動了,只一重巳土,關煞混雜。論來男人煞重掌威權,女子煞重必刑夫。所以主為人聰明機變,得人之寵辱。只有一件,今歲流年甲辰,歲運併臨災殃,必命中又犯小耗勾絞。兩位星辰打攪,雖不能傷,只是主有比肩不和,小人嘴舌,常沾些啾唧不寧之狀。」婦人聽了,說道:「累先生仔細用心,與我回背回背。我這裡一兩銀子,相謝先生買一盞茶吃。奴不求別的,只願得小人離退,夫主愛敬便了。」一面轉入房中,拔了兩件首飾,遞與賊瞎。賊瞎接了,放入袖中,說道:「既要小人回背,用柳木一塊,刻兩箇男女人形像,書着娘子與夫主生時八字。用七七四十九根紅線,扎在一處。上用紅紗一片,蒙在男子眼。中用艾塞其心,用針釘其手。下用膠粘其足,暗暗埋在睡的枕頭內。又朱砂書符一道,燒火灰,暗暗攪在豔茶內。若得夫主吃了茶,到晚夕睡了枕頭,不過三日,自然有驗。」婦人道:「請問先生,這四樁兒是恁的說?」賊瞎道:「好教娘子得知。用紗蒙眼,使夫主見你一似西施一般嬌豔。用艾塞心,使他心愛到你。用針釘手,隨你怎的不是,使他再不敢動手打你;着緊還跪着你。用膠粘足者,使他再不往那裡胡行。」婦人聽言有這等事,滿心歡喜。當下備了香燭紙馬,替婦人燒了紙,到次日,使劉婆送了符水鎮物與婦人,如法常頓停當。將符燒灰,頓下好茶。待的西門慶家來,婦人叫春梅遞茶與他吃,到晚夕與他共枕同床。過了一日兩,兩日三,似水如魚,歡會如常。看官聽說:但凡大小人家,師尼僧道,乳母牙婆,切記休招惹他。背地甚麼事不幹出來?古人有四句格言說得好:

「堂前切莫走三婆,  後門常鎖莫通和;

院內有井防小口,  便是禍少福星多。」

畢竟未知後來如何,且聽下回分解:




\end{showcontents}
