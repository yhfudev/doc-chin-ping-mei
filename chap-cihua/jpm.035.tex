%# -*- coding: utf-8 -*-
%!TEX encoding = UTF-8 Unicode
%!TEX TS-program = xelatex
% vim:ts=4:sw=4
%
% 以上设定默认使用 XeLaTex 编译,并指定 Unicode 编码,供 TeXShop 自动识别

%第三十五回 
\chapter{西門慶挾恨責平安\KG 書童兒粧旦勸狎客}


\begin{showcontents}{}




「莫入州衙與縣衙,  勸君勤謹作生涯,

池塘積水須防旱,  買賣辛勤是養家;

教子教孫并教藝,  栽桑栽棗莫栽花,

閑是閑非休要管,  渴飲清泉悶煮茶。」

此八句單說為人之父母,必須自幼訓教子孫,讀書學禮,知孝順父母,尊敬長上,和睦鄉里,各安生理;切不可縱容他。少年驕惰放肆,三五成群,遊手好閑,張弓挾矢,籠養飛鳥,蹴掬打毬,飲酒賭博,飄風宿娼,無所不為。將來必然招事惹非,敗壞家門。似此人家,使子陷于官司,大則身亡家破,小則吃打受牢,財入公門,政出吏口,連累父兄,惹悔耽憂,有何益哉!話說西門慶早到衙門,先退廳與夏提刑說:「此四人再三尋人情來說,交將就他。」夏提刑道:「也有人到學生那邊,不好對長官說。既是這等,如今提出來,戒飭他一番,放了罷。」西門慶道:「長官見得有理。」即陞廳,令左右提出車淡等犯人跪下。生怕又打,只顧磕頭。西門慶也不等夏提刑開言:「我把你這起光棍,如何尋這許多人情來說?本當都送問,且饒你這遭。若犯子我手裡,都活監死。出去罷!」連韓二都喝出來了,往外金命水命,走投無命。這里處斷公事不題,且說應伯爵拏着五兩銀子,尋書童兒問他討話,悄悄遞與他銀子。書童接的袖了。那平安兒在門首,拏眼兒睃着他。書童于是如此這般,勸住時說:「昨日已對爹說了。今日往衙門裡發落去了。」伯爵道:「他四個父兄再三說,恐怕又責罰他。」書童道:「你老人家只顧放心去,管情兒一下不打他。」那伯爵得了這消息,急急走去,回他每話去了。到日飯時分,四家人都到家,個個撲着父兄家屬放聲大哭。每人去了百十兩銀子,落了兩腿瘡,再不敢妄生事了。正是:

「禍患每從勉強得,  煩惱皆因不忍生。」

卻說那日西門慶未來家時,書童兒在書房內叫來安兒掃地,向食盒揭了,把人家送的卓面上晌糖與他吃。那小廝千不合,萬不合,叫:「書童哥,我有句話兒告你說。昨日俺平安哥接五娘轎子,在路上好不學舌,說哥的過犯。」書童問道:「他說我什麼來?」來安兒道:「他說哥攬的人家幾兩銀子,大膽買了酒肉,送在六娘房裡,吃了半日出來。又在前邊舖子里吃,不與他吃。又說你在書房裡,和爹幹什麼營生。」這書童不聽便罷,聽了暗記在心,過了一日,也不題起。到次日,西門慶早辰約會了,不往衙門裡去,都往門外永福寺,置酒與須坐營送行去了。直到下午時分,纔來家。下馬就分付平安:「但有人來,只說還沒來家。」說畢,進到廳上,書童兒接了衣裳。西門慶因問:「今日沒人來?」書童道:「沒有。管屯的徐老爹送了兩包螃蟹,十斤鮮魚。小的拏回帖打發去了,與了來人二錢銀子。又有吳大舅送了六個帖兒,明日請娘每吃三日。」原來吳大舅兒子吳舜臣,娶了喬大戶娘子姪女兒鄭三姐做媳婦兒。西門慶早送了茶去,他那里來請。西門慶到後邊,月娘拏帖兒與他瞧。說道:「明日你每都收拾了去。」說畢,出來到書房裡坐下。書童連忙拏炭火爐內燒甜香餅兒,雙手遞茶上去。西門慶擎茶在手,他慢慢挨近,站立在卓頭邊。良久,西門慶努了個嘴兒,使他把門關上。用手摟在懷里,一手捧着他的臉兒。西門慶吐舌頭,那小郎口裡叱看鳳香餅兒,遞與他。下邊又替他弄玉莖。西門慶問道:「我兒,外邊沒人欺負你?」那小廝乘機就說:「小的有樁事,不是爹問,小的不敢說。」西門慶道:「你說不妨。」書童就把平安一節,告說一遍:「前日爹叫小的在屋裏,他和畫童在窗外聽覷。小的出來舀水與爹洗手,親自看見他。又在外邊對着人罵小的蠻奴才,百般欺負小的」西門慶聽了,心中大怒,發狠說道:「我若不把奴才腿卸下來,也不算!」這里書房中說話不題。平昔平安兒專一打聽這件事,三不知走去房中報與金蓮。金蓮使春梅前邊來請西門慶說話,剛轉過松牆,只見畫童兒在那里弄松虎兒。便道:「姐來做什麼?爹在書房里。」被春梅頭上鑿了一下。西門慶在裡面聽見裙子響,就知有人來,連忙推開小廝,走在床上睡着。那書童在卓上弄筆硯。春梅推門進來,見了西門慶,咂嘴兒說道:「你每悄悄的在屋裡把門兒關着,敢守親哩!娘請你說話。」西門慶仰睡在枕頭上,便道:「小油嘴兒,他請我說什麼話?你先行,等我略倘倘兒就去。」那春梅那里容他,說道:「你不去,我就拉起你來。」西門慶怎禁他死拉活拉,拉到金蓮房中,金蓮問他:「在前頭做什麼?」春梅道:「他和小廝兩個在書房里,把門兒插着,捏殺蠅子兒是的,赤道幹的什麼繭兒,恰似守親的一般。我進去,小廝在卓子根前推寫字兒了。我眼張大個的,他便倘刺在床上,拉着再不肯來。」潘金蓮道:「他進來我這屋裡,只怕有鍋鑊吃了他是的。賤沒廉恥的貨!你想有個廉恥!大白日和那奴才平白兩個關着門,在屋里做什麼來?左右是奴才臭屁股門子,鑽了,到晚夕還進屋裏,還和俺每沾身睡,好乾淨兒!」西門慶道:「你信小油嘴兒胡說?我那里有此勾當。我看着他寫禮帖兒來,我便歪在床上。」金蓮道:「巴巴的關着門兒寫禮帖,什麼機密謠言,什麼三隻腿的金剛,兩個鯨角的象,怕人瞧見?明日吳大妗子家做三日,掠個帖子兒來,不長不短的,也尋什麼件子與我做拜錢!你不與,莫不問我和野漢子要?大姐姐是一套衣裳五錢銀子,別人也有簪子的,也有花的,只我沒有。我就不去了。」西門慶道:「前邊廚櫃內,拏一疋紅紗來與你做拜錢罷。」金蓮道:「我就去不成,也不要那囂紗片子,拏出去倒沒的教人笑話。」西門慶道:「你休亂,等我往那邊樓上尋一件什麼與他便了。如今往東京這賀禮,也要幾疋尺頭,一套兒尋下來罷。」于是走到李瓶兒那邊樓上,尋了兩疋玄色織金麒麟補子尺頭,兩疋南京色段,一疋大紅斗牛紵絲,一疋翠藍雲段。因對李瓶兒說:「尋一件雲絹衫,與蓮做拜錢,如無,拏帖去段子舖討去罷。」李瓶兒道:。「你不要舖子里取去。我有一件織金雲絹衣服罷,大紅衫兒藍裙,留下一件,也不中用。俺兩個都做了拜錢罷。」一面向箱中取出來,李瓶兒親自拏與金蓮瞧:「隨姐姐揀衫兒也得,裙兒也得。咱兩個一事包了做拜錢倒好。省得人取去。」金蓮道:「你的,我怎好要你的?」李瓶道:「好姐姐,怎生恁說話?」推了半日,金蓮方纔肯了。又出去教陳經濟換了腰封,寫了二人名字在上。這里西門慶後邊揀尺頭不題。且說平安兒正在大門首,只見西門慶朋友白來搶走來,問道:「大官人在家麼?」平安道:「俺爹不在家了。」那白來搶不信,逕入裏面廳上,見槅子關着,說道:「果然不在家,往那里去了?」平安道:「今日門外送行去了,還沒來。」白來搶道:「既是送行,這咱晚也來家了。」平安道:「白大叔,有甚說話說下,待爹來家,小的稟就是了。」白來搶道:「沒什麼話,只是許多時沒見,閑來望望。既不在,我等等罷。」平安道:「只怕來晚了,你老人家等不得。」白來搶不依,把槅子推開,進入廳內,在椅子上就坐了。眾小廝也不理他,由他坐去。不想天假其便,西門慶教迎春抱着尺頭從後邊走來。剛轉過軟壁,頂頭就撞見白來搶在廳上坐着。迎春兒丟下段子,往後走不迭。白來搶道:「這不是哥在家!」一面走下來唱喏。這西門慶見了,推辭不得,須索讓坐。睃見白來搶頭帶着一頂出洗覆盔過的,恰如太山遊到嶺的舊羅帽兒,身穿着一件壞領磨襟救火的硬漿白布衫,腳下靸着一雙乍板唱曲兒,前後彎絕戶綻的古銅木耳兒皂靴,裏邊插着一雙碌子繩子,打不到黃絲轉香馬凳襪子,坐下也不叫茶。只見琴童在旁伺候。西門慶分付:「把尺頭抱到客房裡,教你姐夫封去。」那琴童應諾,抱尺頭往廂房裡去了。白來搶舉手道:「一向久情,沒來望的哥。」西門慶道:「多謝掛意。我也常不在家,日逐衙門中有事。」白來搶道:「哥,這衙門中也日日去麼?」西門慶道:「日日去兩次,每日坐廳問事。到朔望日子,還要拜牌,畫公座,大發放,地方保甲番役打卯。歸家便有許多窮冗,無片時閒暇。今日門外去,因須南溪陞了,新陞了新平寨坐營,眾人和他送行,只剛到家。明日管皇庄薛公公家請吃酒,路遠去不成。後日又要打聽接新巡按。又是東京太師老爺四公子又選了駙馬,蕭茂德帝姬,童太尉姪男童天新選上大堂陞指揮使僉書管事,兩三層都要賀禮。自這連日,通辛苦的了不得。」說了半日話,來安兒纔拏上茶來。白來搶纔拏在手裡,呷了一口,只見玳安拏着大紅帖兒,往後飛跑,報道:「掌刑的夏老爹來了,外邊下馬了!」西門慶就往後邊穿衣服去了。白來搶躲在西廂房內,打簾裡望外張看。良久,夏提刑進來,穿着黑青水緯羅五彩洒線猱頭金獅補子圓領,翠藍羅襯衣,腰繫合香嵌金帶,腳下皂朝靴,身邊帶鑰匙,黑壓壓跟着許多人進到廳上。西門慶冠帶從後邊迎將來。兩個敘禮畢,分賓主坐下。不一時,棋童兒雲南瑪瑙雕漆方盤拏了兩盞茶來,銀鑲竹絲茶鍾,金杏葉茶匙,木樨青荳泡茶 吃了。夏提刑道:「昨日所言接大巡的事,今日學生差人打聽,姓曾,乙未進士,牌已行到東昌地方。他列位每都明日起身遠接,你我雖是武官,係領敕衙門,提點刑獄,比軍衛有司不同。咱後日起身,離城十里,尋個去所,預備一頓飯,那里接見罷。」西門慶道:「長官所言甚妙。也不消長官費心,學生這里着人尋個庵觀寺院,或是人家庄園,亦好教個廚役早去整理。」夏提刑謝道:「這等又教長官費心。」說畢,又吃了一道茶,夏提刑起身去了。西門慶送了進來,寬去衣裳。那白來搶還不去,走到廳上又坐下了,對西門慶說:「自從哥這兩個月沒往會里去,把會來就散了,老孫雖年紀大,主不得事。應二哥又不管。昨日七月內,玉皇廟打中元醮,連我只三四個人兒到,沒個人拏出錢來,都打撒手兒。難為吳道官晚夕謝將,又叫了個說書的,甚是破費他。他雖故不言語,各人心上不安。不如那咱哥做會首時,還有個張主。不久還要請哥上會去。」西門慶道:「你沒的說,散便散了罷。我那里得工夫幹此事?遇閒時,在吳先生那里一年打上個醮,答報答報天地就是了。隨你每會不會,不消來對我說。」幾句搶的白來搶沒言語了。又坐了一回,西門慶見他不去,只得喚琴童兒廂房內放卓兒,拏了四碟小菜,帶葷連素,一碟煎麵觔 ,一碟燒肉 ,西門慶陪他吃了飯;篩酒上來。西門慶後邊討副銀鑲大鍾來,斟與他吃了幾鍾,白來搶纔起身。西門慶送他二門首,說道:「你休怪我不送你。我帶着小帽,不好出去得。」那白來搶告辭去了。西門慶回到廳上,拉了把椅子來,就一片聲的叫平安兒。那平安兒走到跟前,西門慶罵道:「賊奴才!還站着!」叫:「答應的!」就是三四個排軍,在旁伺候。那平安不知什麼緣故,諕的臉蠟查黃,跪下了。西門慶道:「我進門就分付你,但有人來,答應不在,你如何不聽?」平安道:「白大叔來時,小的回說爹往門外送行去了,沒來家。他不信,強着進來了。小的就跟進來,問他:『白大叔有話說下,待爹來家,小的稟就是了。』他又不言語,自家推開廳上槅子坐下了。落後,不想出來就撞見了。」西門慶罵道:「你這奴子,不要說嘴。你好小膽子兒!人進來,你在那里耍錢吃酒去來?不在大們首守着。」令左右:「你聞他口裡。」那排軍聞了一聞,稟道:「沒酒氣。」西門慶分付:「叫兩個會動刑的上來,與我着實拶這奴才!」當下兩個伏侍一個,套上拶指,只雇檠起來,拶的平安疼痛難忍,叫道:「小的委的回爹不在,他強着進來。」那排軍拶上,把繩子綰住,跪下稟道:「拶上了。」西門慶令:「與我敲五十敲。」旁邊數着,敲到五十上,住了手。西門慶分付:「打二十棍。」須臾,打了二十,打的皮開肉綻,滿腿杖痕。西門慶喝令:「與我放了。」兩個排軍,向前解了拶子,解的直聲呼喚。西門慶罵道:「我把你這賊奴才!你說你在大門首,想說要人家錢兒,在外邊壞我的事,休吹到我耳垛內,把你這奴才腿卸下來!」那平安磕頭了起來,提着褲子往外去了。西門慶看見畫童兒在旁邊,說道:「把這小奴才拏下去,也拶他一拶子。」一面拶的小廝殺豬兒似怪叫。這里西門慶在前廳拶人不題。單說潘金蓮從房里出來,往後走。剛走到大廳後儀門首,只見孟玉樓獨自一個在軟壁後廳覷。金蓮便問:「你在此聽什麼兒哩?」玉樓道:「我在這里聽他爹打平安兒,連畫童小奴才也拶了一拶子。不知為什麼?」一回棋童兒過來,玉樓叫住問他:「為什麼打平安兒?」棋童道:「爹嗔他放進白來搶來了。」金蓮接過來道:「也不是為放進白來搶來,敢是為他打了象牙梳?不是打了象牙,平白為什麼打得小廝這樣的!賊沒廉恥的貨!亦發臉做了主了,想有些廉恥兒,也怎的!」那棋童就走了。玉樓便問金蓮:「怎的打了象牙?」金蓮道:「我要告訴你,還沒告訴你。我前日去俺媽家做生日去了,不在家。學說蠻秫秫小廝,攬了人家說事幾兩銀子,買嗄飯在前邊治了兩方盒,又是一罈金華酒 ,掇到李瓶兒房裡,和小廝吃了半日酒,小廝纔出來。沒廉恥貨來家,學說也不言語,還和小廝在花園書房里插着門兒,兩個不知幹着什麼營生!平安這小廝,拏着人家帖子進去,見門關着,就在窗下站着了。蠻小廝開門看見了,想是學與賊沒廉恥的貨,今日挾仇,打這小廝,打的膫子成!那怕蠻奴才,到明日把一家子都收拾了,管人弔腳兒事!」玉樓笑道:「好說,雖是一家子,有賢有愚,莫不都心邪了罷?」金蓮道:「不是這般說,等我告訴你。如今這家中,他心肝肐蒂兒事,偏歡喜的這兩個人,一個在裏,一個在外,成日把魂恰似落在他身上一般。見了說也有,笑也有。俺每是沒時運的,行動就相鳥眼雞一般!賊不逢好死變心的強盜!通把心狐迷住了,更變的如今相他哩!三姐,你聽着,到明日弄出什麼八怪七喇出來!今日為拜錢,又和他合了回氣。但來家,不是在他房裡,就在書房裡,不知幹的什麼事!我今日使春梅:『你看他在那里?叫他來。』誰知他大白日裡,和賊蠻奴才關着門兒,在書房裡。春梅推門入去,諕的一個眼張失道的。到屋裡教我儘力數罵了幾句,他只雇左遮右掩的。先拏一疋紅紗與我做拜錢,我不要。落後往李瓶兒那邊樓上尋去。賊人膽兒虛,自知理虧,拏了他廂人一套織金衣服來,親自來儘我,說道:「姐姐,你看這衣服好不好?省的拆開了,咱兩個拏去都做了拜錢罷。』我便說:『你的東西兒,我如何要你的?教爹舖子裡取去。』他慌了,說:『姐姐,怎的這般計較?姐姐揀衫兒也得,裙兒也得。看了好,拏到前邊教陳姐夫封寫去。』儘了半日,我纔吐了口兒。他讓我要了衫子。」玉樓道:「這也罷了。也是他的儘讓之情。」金蓮道:「你不知道,不要讓了他。如今年世,只怕睜着眼兒的金剛,不怕閉着眼兒的佛。老婆漢子,你若放些鬆兒與他,王兵馬的皂隸,還把你不當{入日}的!」玉樓戲道:「六丫頭,你是屬麵觔的,倒且是有靳道!」說着,兩個笑了。只見小玉來,請三娘、五娘:「後邊吃螃蟹哩!我去請六娘和大姑娘去。」兩個手拉着手兒進來。月娘和李嬌兒正在上房那門穿廊下坐,說道:「你兩個笑什麼兒?」金蓮道:「我笑他爹打平安兒。」月娘道:「嗔他恁亂蝍蟆叫喊的,只道打什麼人,原來打他!為什麼來?」金蓮道:「為他打折了象牙了。」月娘老實,便問:「象牙放在那里來?怎的教他打折了?」那潘金蓮和孟玉樓兩個嘻嘻哈哈,只雇笑成一塊。月娘道:「不知你每笑什麼?不對我說。」玉樓道:「姐姐,你不知道。爹打平安,為放進白來搶來了。」月娘道:「放進白來搶便罷了,怎麼說道打了象牙?也沒見這般沒稍幹的人,在家閉着膫子坐,平白有要沒緊,來人家撞些什麼!」來安道:「他來望爹來了。」月娘道:「那個弔下炕來了,望沒的扯臊淡!不說來挄嘴吃罷了!」良久,李瓶兒和大姐來到。眾人圍遶吃螃蟹。月娘分付小玉:「屋裡還有些葡萄酒 ,篩來與你娘每吃。」金蓮快嘴,說道:「吃螃蟹,得些金華酒 吃纔好。」又道:「只剛一味螃蟹就着酒吃,得隻燒鴨 兒撕了來下酒。」月娘道:「這咱晚那里買燒鴨子去 。」那席上李瓶兒聽了,把臉飛紅了。正是:

「話頭兒包含著深意,  題目兒里暗蓄著留心。」

那月娘是個誠實的人,怎曉的話中之話。這里吃螃蟹不題。且說平安兒被責,來到外邊,打內刺扒着腿兒,走那屋裡,拶的把人揸沙着。賁四、來興眾人都亂來問:「平官兒,爹為什麼打你?」平安哭道:「我知為什麼?」來興兒道:「爹嗔他放進白搶來了。」平安道:「早是頭里你看着,我那等攔了他兩次兒,說爹不在家。他強着進去了。到廳上槅子門裡,我說:『你老人家,有什麼說,說下罷。爹門外送行去了。不知多咱來,只怕等不得。』他說:『我等等兒。』話又不說,坐住了。不想爹從後邊出來,撞見了。又沒甚話;『我閑來望望兒。』吃了茶,再不起身。只見夏老爹來了,我說他去了。他還躲在廂房里,又不去。爹沒法兒,少不的留他坐。人家知慚愧的,略坐一回兒就去。他直等拏酒來吃了纔去,倒惹的進來打我這一頓。說我不在門首看,放進人來了。你說我不造化低?我沒攔他,又說我沒攔他;他強自進來坐着,不虧了管我腿事,打我!教那個賊天殺男盜女娼的狗骨禿,吃了俺家這東西,打背梁春下過!」來興兒道:「爛折春梁骨的,倒好了他往下撞。」平安道:「教他生噎食病,把顙根軸子爛弔了」平安道:「天下有沒廉恥皮臉的,不相這狗骨禿沒廉恥,來我家闖的狗也不咬,賊雌飯吃花子{入日}的!再不,爛了賊亡八的屁股門子!」來興笑道:「爛了屁股,門上人不知道,只說是臊的。」眾人都笑了。平安道:「想必是家裡沒晚米做飯,老婆不知餓得怎麼樣的。閑的沒的幹,來人家抹嘴吃,圖家裡省了一頓,也不是常法兒。不如教老婆養漢,做了忘八,倒硬朗些,不教下人唾罵。」正是:

「外頭擺浪子,  家里老婆啃家子。」

玳安在舖子裡篦頭,篦了,打發那人錢去了。走出來說:「平安兒,我不言語,鱉的我慌,虧你還答應主子,當家的性格,你還不知道。你怎怪人?常言:『養兒不要倚金溺銀,只要見景生情。』比不的應二叔和謝叔來,答應在家不在家,他彼此都是心甜厚間便罷了。以下的人,他又分付你答應不在家,你怎的放人來?不打你卻打誰?」賁四戲道:「平安兒從新做了小孩兒,纔學閑閑。他又會頑,成日只踢毬兒耍子。」眾人又笑了一回。賁四道:「他便為放進人來。這畫童兒卻為什麼也陪拶了一拶子?是好吃的菓子兒,陪吃個兒。吃酒吃肉,也有個陪客。十個指頭套在拶子上,也有個陪的來!」那畫童兒揉着手,只是哭。玳安戲道:「我兒少哭,你娘養的你忒嬌。把饊子兒拏繩兒拴在你手兒上,你還不吃。」這里前邊小廝熱亂亂不題。西門慶在廂房中,看着陳經濟、書童封了禮物尺頭,寫了揭帖,次日早打發人上東京,送蔡駙馬童堂上禮,不在話下。到次日,西門慶往衙門裡去了。吳月娘與眾房共五頂轎子,頭帶珠翠冠,身穿錦綉袍,來興媳婦一頂小轎跟隨,往吳大妗家做三日去了。止留下孫雪娥在家中,和西門大姐看家。早間韓道國送禮相謝,一罈金華酒 ,一隻水晶鵝 ,一副蹄子,四隻燒鴨 ,四尾鰣魚 。帖子上寫着:「晚生韓道國頓首拜。」書童沒人在家,不敢收,連盒擔留下。待的西門慶衙門中回來,拏與西門慶瞧。西門慶使琴童兒舖子里旋叫了韓夥計,甚是說他:「沒分曉,又買這禮來做什麼?我決然不受。」那韓道國拜說:「老爹,小人蒙老爹莫大之恩,可憐見與小人出了氣。小人舉家感激不盡。無甚微物,表一點窮心,望乞老爹好歹笑納!」西門慶道:「這個使不得。你是我門下夥計,如同一家,我如何受你的禮?即令原人與我擡回去。」韓道國慌了,央說了半日。西門慶分付左右,只受了鵝酒,別的禮都令擡回去了。教小廝拏帖兒,請應二爹和謝爹去。對韓道國說:「你後晌叫來保看着舖子,你來坐坐。」韓道國說:「禮物不受,又教老爹費心!」應諾去了。西門慶家中,又添買了許多菜蔬,後晌時分,在花園中翡翠軒捲棚內,放下一張八仙卓兒。應伯爵、謝希大先到了。西門慶告他說:「韓夥計費心,買禮來謝我。我再三不受他,他只雇死活央告,只留了他鵝酒,我怎好獨享?請你二位陪他坐坐。」伯爵道:「他和我計較來,要買禮謝。我說你大官府里那里,稀罕你的?休要費心。你就送去,他決然不受。如何?我恰似打你肚子裡鑽過一遭的,果然不受他的。」說畢,吃了茶,兩個打雙陸。不一時,韓道國到了,二人敘禮畢,坐下。應伯爵、謝希大居上,西門慶關席,韓道國打橫。登時四盤四碗拏來,卓上擺了許多嗄飯,吃不了,又是兩大盤玉米麵鵝酒蒸餅兒 堆集的。把金華酒 分付來安兒,就在旁邊打開,用銅甑兒篩熱了拏來,教書童斟酒,畫童兒單管後邊拏菓拏菜去。酒斟上來,伯爵分付書童兒:「後邊對你大娘房裡說,怎的不拏出螃蟹來與應二爹吃?你去說,我要螃蟹吃哩。」西門慶道:「傻狗材,那里有一個螃蟹?實和你說,管屯的徐大人送了我兩包螃蟹,到如今娘每都吃了,剩下醃了幾個。」分付小廝:「把醃螃蟹 〈扌扉〉幾個來。今日娘每都不在,往吳大妗子家做三日去了。」不一時,畫童拏了兩盤子醃蟹上來。那應伯爵和謝希大兩個,搶着吃的淨光。因見書童兒斟酒,說道:「你應二爹一生不吃啞酒。自誇你會唱的南曲,我不曾聽見,今日你好歹唱個兒,我纔吃這鍾酒。」那書童纔待拍手着唱,伯爵道:「這個唱一萬個,也不算。你裝龍似龍,裝虎似虎,下邊搽畫粧扮起來,相個旦兒的模樣纔好。」那書童在席上,把眼只看西門慶的聲色兒。西門慶笑罵伯爵:「你這狗材!專一歪斯纏人。」因向書童道:「既是他索落你,教玳安兒前邊問你姐要了衣服,下邊粧扮了來。」玳安先走到前邊金蓮房裡,問春梅要,春梅不與。旋往後,問上房玉簫要了四根銀簪子,一個梳背兒,面前一件仙子兒,一金鑲假青石頭墜子,大紅對衿絹衫兒,綠重絹裙子,紫綃金箍兒;要了些脂粉,在書房裡搽抹起來,儼然就是個女子,打扮的甚是嬌娜。走在席邊,雙手先遞上一盃與應伯爵。頓開喉音,在旁唱玉芙蓉,道:

「殘紅水上飄,梅子枝頭小,這些時淡了眉兒誰描?因春帶得愁來到,春去緣何愁未消?人別後山遙水遙,我為你數盡歸期,畫損了掠兒稍。」

伯爵聽了,誇獎不已。說道:「相這大官兒,不枉了與他碗飯吃。你看他這喉音,就是一管簫。說那院裡小娘兒便怎的!那套唱都聽的熱了,怎生如他那等滋潤?哥,不是俺每面獎,似他這般的人兒在你身邊,你不喜歡?」西門慶笑了。伯爵道:「哥,你怎的笑?我倒說的正經話。你休虧了這孩子,凡事衣類兒上,另着個眼兒看他。難為李大人送了他來,也是他的盛情。」西門慶道:「正是,如今我不在家,書房中一應大小事,收禮帖兒,封書柬答應,都是他和小婿。小婿又要舖子裡兼看看。」應伯爵飲過,又斟雙盃。伯爵道:「你替我吃些兒。」書童道:「小的不敢吃,不會吃。」伯爵道:「你不吃,我就惱了。我賞你,怕怎的?」書童只顧把眼看西門慶。西門慶道:「也罷,應二爹賞你,你吃了。」那小廝打了個僉兒,慢慢低垂粉頭,呷了一口。餘下半鍾殘酒,用手擎着與伯爵吃了,方纔轉過身來,遞謝希大酒。又唱個前腔兒:

「新荷池內翻,過雨瓊珠濺,對南薰燕侶鶯儔心煩。啼痕界破殘粧面,瘦對腰肢憶小蠻。從別後千難萬難,我為你盼歸期,靠損了玉欄杆。」

謝希大問西門慶道:「哥,書官兒青春多少?」西門慶道:「他今年纔交十六歲。」問道:「你也會多少南曲?」書童道:「小的也記不多,幾個曲子,胡亂席上答應爹每罷了。」希大道:「好個乖覺孩子!」亦照前遞了酒。下來遞韓道國。道國道:「老爹在上,小的怎敢欺心!」西門慶道:「今日你是客。」韓道國道:「豈有此理?還是從老爹上來,次後纔是小人吃酒。」書童下席來,遞西門慶酒。又唱第三個前腔兒:

「東籬菊綻開,金井梧桐敗,聽南樓塞雁聲哀傷懷。春情欲寄梅花信,鴻雁來時人未來。從別後音乖信乖,我為你恨歸期,跌綻了綉羅鞋。」

西門慶吃畢,到韓道國跟前。那韓道國慌的連忙立起身來接酒。伯爵道:「你坐着,教他好唱。」那韓道國方纔坐下。書童又唱個第四個前腔兒:

「漫空柳絮飛,亂舞蜂蝶翅,嶺頭梅開了南枝。折梅須寄皇華使,幾度停針長歎時。從別後朝思暮思,我為你數歸期,掐破了指尖兒。」

那韓道國未等詞終,連忙一飲而盡。正飲酒中間,只見玳安來說:「賁四叔來了,請爹說話。」西門慶道:「你叫他來這里說罷。」不一時,賁四身穿青絹褶子,單穗縧兒,粉底皂靴,向前作了揖,旁邊安頓坐了。玳安連忙取一雙鍾筯放下。西門慶令玳安後邊取菜蔬去了。西門慶因問他:「庄子上收拾怎的樣子了?」賁四道:「前一層纔蓋瓦,後邊捲棚昨日纔打的基。還有兩邊廂房,與後一層住房的料沒有。還少客位與捲棚。漫地尺二方磚,還得五百;那舊的都使不得。砌牆的大城角多沒了。墊地腳帶山子上土,也添勾一百多車子。灰還得二十兩銀子。」西門慶道:「那灰不打緊,我明日衙門里,分付灰戶,教他送去。昨日你磚廠劉公公說,送我些磚兒,你開個數兒,封幾兩銀子送與他;須是一半人情兒回去。只少這木植。」賁四道:「昨日老爹分付門外看那庄子。人今早到賃上同張安兒到那家庄子上,原來是向皇親家庄子,大皇親沒了,如今向五要賣神路明堂。咱每不是要他的,講過只拆他三間廳,六間廂房,一層群房就勾了,他口氣要五百兩。到跟前,拏銀子和他講,三百五十兩上也該拆他的。休說木植木料,光磚瓦連土也值一二百兩銀子。」應伯爵道:「我道是誰來?是向五的那庄子。向五被人告爭地土,告在屯田兵備道打官司,使了好多銀子;又在院裡包着羅存兒。如今手裡弄的沒錢了。你若要,與他三百兩銀子,他也罷了。冷手撾不着熱饅頭,在那壇兒哩念佛麼!」西門慶分付賁四:「你明日拏兩錠大銀子同張安兒和他講去,若三百兩銀子肯,拆了來罷。」賁四道:「小人理會。」良久,後邊拏了一碗湯,一盤蒸餅上來。賁四吃了。斟上陪眾人吃酒。書童唱了一遍下去了。應伯爵道:「這等吃的酒沒趣,取箇骰盆兒,俺每行個令兒吃纔好。」西門慶令玳安:「就在前邊六娘屋裡,取個骰盆來。」不一時,玳安取了來,放在伯爵跟前,悄悄走到西門慶耳邊,掩口說:「六娘房裡哥哭哩。迎春姐教爹陪着個人兒,接接六娘去。」西門慶道:「你放下壺,快教個小廝拏燈籠接去。」因問:「那兩個小廝那里?」玳安道:「琴童與棋童兒先拏兩個燈籠接去了。」伯爵見盆內放着六個骰兒,伯爵即用手拈着一個,說:「我擲着點兒,各人要骨牌名一句,見合着點數兒。如說不過來,罰一大盃酒,下家唱曲兒。不會唱曲兒,說笑話兒。兩樁兒不會,定罰一大盃。西門慶道:「怪狗材,忒韶刀了。」伯爵道:「令官放個屁,也欽此欽遵,你管我怎的?」叫來安:「你且先斟一盃罰了爹,然後好行令。」西門慶笑而飲之。伯爵道:「眾人聽着,我起令了。說差了,也罰一盃。」說道:「張生醉倒在西廂,吃了多少酒,一大壺,兩小壺。」果然是個么。西門慶教書童兒上來斟酒,該下家謝希大唱。布大拍着手兒:「我唱了個折桂令兒,你聽罷。」唱道:

「可人心二八嬌娃,百件風流所事慷達。眉蹙春山,眼橫秋水,鬢綰著烏鴉,乾相思,撇不下,一時半霎,咫尺間,如隔著海角天涯。瘦也因他,病也因他。誰與做個成就了姻緣,便是那救苦難菩薩。」

伯爵吃過酒,過盆與謝希大該擲,擲輪着西門慶唱。謝希大拏過骰兒來說:「多謝紅兒扶上床。什麼時候?三更四點。」可煞作怪,擲出個四來。伯爵道:「謝子純該吃四盃。」希大道:「折兩盃罷,我吃不得。」書童兒滿斟了兩盃。先吃了頭一盃,等他唱。席上伯爵二個,把一碟子荸薺 都吃了。西門慶道:「我不會唱,說了笑話兒罷。」說道:「一個人到菓子舖,問:『可有榧子 麼』?那人說:『有。』取來看。那買菓子的不住的往口裡放。賣菓子的說:『你不買,如何只顧吃?』那人道:『我圖他潤肺。』那賣的說:『你便潤了肺,我卻心疼。』」眾人多笑了。伯爵道:「你若心疼,再拏兩碟子來。我媒人婆拾馬糞,越發越晒。」謝希大吃了。第三說西門慶擲,說:「留下金釵與表記,多少重?五六七錢。」西門慶拈起骰兒來,擲了個五。書童兒道:「再斟上兩鍾半酒?」謝希大道:「哥大量,也吃兩鍾兒?沒這個理。哥吃四鍾罷,只當俺一家孝順一鍾兒。」該韓夥計唱。韓道國讓賁四哥年長。賁四道:「我不會唱,說個笑話兒罷。」西門慶吃過兩鍾,賁四說道:「一官問姦情事,問:『你當初如何姦他來?』那男子說:『頭朝東,腳也朝東姦來。』官云:『胡說!那里有個曲着行房的道理?』旁邊一個人走來,跪下說道:『告稟,若缺刑房,待小的補了罷。』」應伯爵道:「好賁四哥,你便益不失當家,你大官府又不老,別的還可說,你怎麼一個行房,你也補他的?」賁四聽見他此言,諕的把臉通紅了,說道:「二叔什麼話,小人出于無心!」伯爵道:「什麼話?檀木靶;沒了刀兒,只有刀鞘兒了。」那賁四在席上終是坐不住,去又不好去,如坐針氈相似。西門慶于是飲畢四鍾酒,就輪該賁四擲。賁四纔待拏起骰子來,只見來安兒來請:「賁四叔,外邊有人尋你。我問他,說是窰上人。」這賁四巴不得要去,聽見這一聲,一個金蟬脫殼走了。西門慶道:「他去了,韓夥計你擲罷。」韓道國舉起骰兒道:「小人遵令了。」說道:「夫人將棒打紅娘,打多少?八九十下。」伯爵道:「該我唱,我不唱罷。我也說個笑話兒。」教書童:「合席都篩上酒,連你爹也篩上,聽我這個笑話:一個道士,師徒二人往人家送疏。行到施主門,徒弟把縧兒鬆了些,垂下來。師父說:『你看那樣,倒相沒屁股的。』徒弟回頭答道:『我沒屁股,師父你一日也成不得!』」西門慶罵道:「你這歪狗材!狗口裏吐出什麼象牙來!」這里飲酒不題。且說玳安先到前邊,又叫了畫童,拏着燈籠來吳大妗子家接李瓶兒。瓶兒聽見說家里孩子哭,也等不得上拜,留下拜錢就要告辭來家。吳大妗、二妗子那里肯放,好歹等他兩口兒上了拜兒。月娘道:「大妗子,你不知道,倒教他家去罷。家裡沒人,孩子好不尋他哭哩。庵每多坐回兒,不妨事。」那吳大妗子纔放李瓶兒出門。玳安丟下畫童,和琴童兒兩個,隨着轎子,跟了先來家了。落後上了拜,堂客散時,月娘和四位轎子,只打着一個燈籠,況是八月二十四日,月黑的時分。月娘問:「別的燈在那里?如何只一個?」棋童道:「小的原拏了兩個來,玳安要了一個,和琴童先跟六娘家去了。」月娘冷帳更不問,就罷了。潘金蓮有心,便問棋童:「你每頭里拏幾個來?」棋童道:「小的和琴童拏了兩個來接娘每,落後玳安與畫童又要了一個去,把畫童換下,和琴童先跟了六娘去了。」金蓮道:「玳安那囚根子,他沒拏燈來?」畫童道:「我和他又拏一個燈籠來了。」金蓮道:「既是有一個就罷了,怎的又問你要這個?」棋童道:「我那們說,他強着奪去了。」金蓮便叫吳月娘:「姐姐,你看!玳安恁賊獻懃的奴才,等到家裡和他答話!」月娘道:「奈煩,孩子家裡緊等着,叫他打了來罷了。怎的?」金蓮道:「姐姐,不是這等說。俺便罷了,你是個大娘子,沒些家法兒,晴天還好,這等月黑,四頂轎子只點着一個燈籠,雇那些兒的是!」說着轎子到門首。月娘、李嬌兒便往後邊去了。金蓮和孟玉樓一答兒下轎,進門就問:「玳安兒在那里?」平安道:「在後邊伺候哩。」剛說着,玳安出來,被金蓮罵了幾句:「我把你獻勤的囚根子!明日你只認起了,單揀着有時運的跟,只休要把腳兒錫錫兒!有一個燈籠打着罷了,信那斜汀世界一般,又奪了個來,又把小廝也換了來。他一頂轎子倒占了兩個燈籠,俺每四頂轎子反打着一個燈籠。俺每不是爹的老婆?」玳安道:「娘錯怪小的了。爹見哥兒哭,教小的快打燈籠接你六娘先來家罷,恐怕哭壞了哥兒。莫不爹不使我,我好幹着接去來?」金蓮道:「你這囚根子,不要說嘴!他教你接去,沒教你把燈籠都拏了來。哥哥,你的雀兒只揀旺處飛。休要認着了,冷灶上着一把兒,熱灶上着一把兒纔好。俺每天生就是沒時運的來!」玳安道:「娘說的什麼話!小的但有這心,騎馬把脯子骨撞折了!」金蓮道:「你這欺心的囚根子!不要慌,我洗淨眼兒看着你哩!」說着,和玉樓往後邊去了。那玳安對着眾人說:「我精攘氣的營生!平白的爹使我接的去,教五娘罵了我恁一頓!」玉樓、金蓮二人到儀門首,撞見來安兒,問:「你爹在那里坐着哩?」來安道:「爹和應二爹、謝爹、韓大叔還在槅捲內吃酒。書童哥裝了個唱的,在那里唱哩。娘每瞧瞧去。」金蓮拉玉樓:「咱瞧瞧去。」二人同走到捲棚槅子外,往裏觀看,只見應伯爵在上坐着,把帽兒歪挺着,醉的只相線兒提的。謝希大醉的把眼兒通睜不開。書童便粧扮在旁邊斟酒唱南曲。西門慶悄悄使琴童兒抹了伯爵一臉粉,又拏草圈兒悄悄兒從後邊作戲,弄在他頭上。把金蓮和玉樓在外邊忍不住,只是笑的不了,罵:「賊囚根子!到明日死了也沒罪了,把醜卻教他出盡了。」西門慶聽見外邊笑,使小廝出來問是誰,二人纔往後邊去了。散時已一更天氣了。西門慶那日,往李瓶兒房里睡去了。金蓮歸房,因問春梅:「李瓶兒來家,說什麼話來?」春梅道:「沒說什麼。」又問:「你沒廉恥貨,進他屋裡去來沒有?」春梅道:「六娘來家,爹往他房里還走了兩遭。」金蓮道:「真個是因孩子哭,接他來?」春梅道:「孩子後晌好不怪哭的,抱着也哭,放下也哭,沒法處。」又問:「書童那奴才,穿的誰的衣服?」春梅道:「先來問我要,教我罵了玳安出去,落後和上房玉簫借了。前邊對爹說了,纔使小廝接去。」金蓮道:「若是這等的也罷了,我說又是沒廉恥的貨,三等兒九般使了接去。」金蓮道:「衣有來,休要與秫秫奴才穿。」說畢,見西門慶不進來,使性兒關了門睡了。且說應伯爵見賁四管工,在庄子上撰錢。明日又拏銀子買向五皇親房子,少說也有幾兩銀子背。又行令之間,可可見賁四不防頭,說出這個笑話兒來。伯爵因此錯他這一錯,使他知道。賁四果然害怕,次日封了三兩銀子,親到伯爵家磕頭。伯爵反打張驚兒,說道:「我沒曾在你面上盡得心,何故行此事?」賁四道:「小人一向缺禮,早晚只望二叔在老爹面前扶持一二,足感不盡。」伯爵于是把銀子收了,待了一鍾茶,打發賁四出門。拏銀子到房中與他娘子兒說:「老兒不發狠,婆兒沒布裙。賁四這狗啃的,我舉保他一場,他得了買賣,扒自飯碗兒,就不用着我了。大官人教他在庄子上管工,明日又托他拏銀子成向五家庄子,一向撰的錢也勾了。我昨日在酒席上拏言語錯了他錯兒。他慌了,不怕他今日不來求我,送了我這三兩銀子。我且買幾疋布,勾孩子每冬衣了。」正是:

「恨小非君子,  無毒不丈夫。」

畢竟未知後來何,且聽下回分解,正是:

「袛恨閑愁成懊惱,  始知伶俐不如痴。」





\end{showcontents}


