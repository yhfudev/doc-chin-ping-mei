%# -*- coding: utf-8 -*-
%!TEX encoding = UTF-8 Unicode
%!TEX TS-program = xelatex
% vim:ts=4:sw=4
%
% 以上设定默认使用 XeLaTex 编译,并指定 Unicode 编码,供 TeXShop 自动识别

%第四十七回 
\chapter{王六兒說事圖財\KG 西門慶受贓枉法}

「風擁狂瀾浪正顛,  孤舟斜泊抱愁眠,

離嗚叫切寒雲外,  驛鼓清分旅夢邊,

詩思有添池草綠,  河船天約晚潮昇,

憑虛細數誰知己,  惟有故人月在天。」

此一首詩,單題塞北以車馬為常,江南以舟楫為便。南人乘舟,北人乘馬,蓋可信也。話說江南楊州廣陵城內,有一苗員外,名喚苗天秀。家有萬貫資財,頗好詩禮。年四十歲,身邊無子。止有一女,尚未出嫁。其妻李氏,身染痼疾在牀。家事盡托與寵妾刁氏,名喚刁七兒,原是楊州大馬頭娼妓出身。天秀用銀三百兩,娶來家納為側室,寵嬖無比。忽一日,有一老僧在門首化緣,自稱是東京報恩寺僧,因為堂中缺少一尊鍍金銅羅漢,故雲遊在此,訪善紀錄。天秀問之不吝,即施銀五十兩與那僧人。僧人道:「不消許多,一半足以完備此像。」天秀道:「吾師休嫌少,除完佛像,餘剩可作齋供。」那僧人問訊致謝,臨行向天秀說道:「員外左眼眶下有一道白氣,乃是死氣,主不出此年,當有大災殃;你有如此善緣與我,貧僧焉乃不預先說與你知?今後隨有甚事,切勿出境。戒之!戒之!」言畢,作辭天秀而去。那消半月,天秀偶遊後園,見其家人苗青,平白是個浪子,正與刁氏在亭側相倚私語,不意天秀卒至,躲避不及。看見不由分說,將苗青痛打一頓,誓欲逐之。苗青恐懼,轉央親鄰,再三勸留得免,終是切恨在心。不期有天秀表兄黃美,原是楊州人氏,乃舉人出身,在東京開封府做通判,亦是博學廣識之人也。一日差人寄了一封書來楊州與天秀,要請天秀上東京,一則遊玩,二者為謀其前程。苗天秀得書不勝歡喜,因向其妻妾說道:「東京乃輦轂之地,景物繁華所萃。吾心久欲遊覽,無由得便。今不期表兄書來相招,實有以大慰平生之意。」其妻李氏便說:「前日僧人相你面上有災厄,囑付不可出門。且此去京都甚遠,況你家私沉重,拋下幼女病妻在家,未審此去前程如何,不如勿往為善。」天秀不聽,反加怒叱,說道:「大丈夫生于天地之間,桑弧逢失,不能遨遊天下,觀國之光,徒老死牖下無益矣!況吾胸中有物,囊有餘資,何愁功名之不到手?此去表兄必有美事于我,切勿多言!」三人于是分付家人苗青,收拾行李衣裝,多打點兩廂金銀,載一船貨物,帶了兩個家童并苗青來上東京,取功名如拾芥,得美職猶唾手。囑付妻妄守家值日。起行正值秋末冬初之時,從楊州馬頭上船,行了數日到徐州洪,但見一派水光,十分陰惡:

「萬里長洪水似傾,  東流海島若雷鳴;

滔滔雪浪令人怕,  客旅逢之誰不驚!」

前過地名陝灣,苗員外見看天晚,命舟人泊住船隻。也是天數將盡,合當有事,不料搭的船隻,卻是賊船,兩個艄子皆是不善之徒。一個姓陳,名喚陳三,一個姓翁,乃是翁八。常言道:「不着家人,弄不得家鬼。」這苗青深恨家主苗天秀。日前被責之仇,一向要報無由。口中不言,心內暗道:「不如我如此如此,這般這般,與兩個艄子做一路,拿得將家主害了性命,推在水內,盡分其財物。我這一回去,再把病婦謀死。這分家私,連刁氏都是我情愛的。」正是:

「花枝葉下猶藏剌,  人心怎保不懷毒!」

這苗青由是與兩個艄子密密商量說道:「我家主皮廂中還有一千兩金銀,二千兩段疋,衣服之類極廣。汝二人若能謀之,願將此物均分。」陳三、翁八笑道:「汝若不言,我等不瞞你說,亦有此意久矣!」是夜天氣陰黑,苗天秀與安童在中艙睡,苗青在艙後。將近三鼓時分,那苗青故意連叫有賊。苗天秀從夢中驚醒,便探頭出艙外觀看。被陳三手持利刀,一下剌中脖下,推在洪波蕩裡。那安童正要走時,乞翁八一悶棍打落于水中。三人一面在船艙內,打開廂籠,取出一應財帛金銀并其段貨衣服,點數均分。二艄便說:「我哥若留此貨物,必然有犯。你是他手下家人,載此貨物到於市店上發賣,沒人相疑。」因此二艄盡把皮廂中一千兩金銀,并苗員外衣服之類分乞,依前撑船回去了。這苗青另搭了船隻,載至臨青馬頭上,鈔關上過了,裝到清河縣城外官店內卸下。見了楊州故舊商家,只說家主在後船便來也。這個苗青在店發賣貨物不題。常言:「人便如此如此,天裡未然未然。」可憐苗員外平昔良善,一旦遭其從僕人之害,不得好死。雖則不納忠言之勸,其亦大數難逃。不想安童被艄一棍打昏,雖落水中,幸得不死,浮沒蘆港,得上岸來,在於堤邊號泣連聲。看看天色微明之時,忽見上流有一隻漁船,撑將下來。船上坐着個老翁,頭頂箬笠,身披短簑。只聽得岸邊蘆荻深處有啼哭,移船過來看時,都是一個十七八歲小廝,滿身是水。問其始末情由,都是楊州苗員外家童,在洪上被劫之事。這漁翁帶下船,撑回家中,取衣服與他換了,給以飲食。因問他:「你要回去乎?都同我在此過活?」安童哭道:「主人遭難,不見下落,如何回得家去?愿隨公公在此。」漁翁道:「也罷,你且隨我在此,等我慢慢替你訪此賊人是誰,再作理會。」安童拜謝公公,遂在此翁家過其日月。一日,也是合當有事,年除歲末,漁翁忽帶安童正出河口賣魚,正撞見陳三、翁八在船上飲酒,穿着他主人衣服,上岸來買魚。安童認得,即密與漁翁說道:「主人之冤當雪矣!」漁翁道:「如何不具狀官司處告理?」當下安童將情具告到巡河周守備府內,守備見沒賍證,不接狀子。又告到提刑院,夏提刑見是強盜劫殺人等事,把狀批行了。從正月十四日,差緝捕公人,押安童下來拿人。前至新河口,把陳三、翁八獲住到於案,責問了口詞。二艄見安童在傍執證,也沒得動刑,一一招承了,供稱:「下手之時,還有他家人苗青同謀,殺其家主,分賍而去。」這裡把三人監下,又差人訪拿苗青,拿到一起定罪。因節間放假,提刑官吏一連兩日沒來衙門中問事。早有衙門首透信兒的人,悄悄報與苗青,苗青把這件事兒慌了,把店門鎖了,暗暗躲在經紀樂三家。這樂三就在獅子街石橋西首,韓道國家隔壁,門面一間,到底三層房兒居住。他渾家樂三嫂,與王六兒所交敬厚,常過王六兒這邊來做伴兒坐。王六兒無事,也常往他家行走,彼此打的熱鬧。這樂三見苗青面帶憂容,問其所以。說道:「不打緊,間壁韓家,就是提刑西門老爹的外室,又是他家夥計,和俺家交往的甚好,凡事百依百隨;若要保得你無事,破多少東西,教俺家過去和他家說說。」這苗青聽了,連忙就下跪說道:「但得除割了我身上沒事,恩有重報,不敢有忘!」于是寫了說帖,封下五十兩銀子,兩套粧花段子衣服,樂三教他老婆拿過去,如此這般,對王六兒說。喜歡的要不的,把衣服和銀子并說帖都收下,單等西門慶,不見來。到七十日,日西時分,只見玳安夾着毡包,騎着頭口,從街心裡來。王六兒在門首叫下來問道:「你往那裡去來?」玳安道:「我跟了爹走了遠差,往東平府送禮去來。」王六兒道:「你爹如今在那裡?來了不曾?」玳安道:「爹和賁四先往家去了。」王六兒便叫進去,和他如此這般說話,拿帖兒與他瞧。玳安道:「韓大嬸管他這事?休要把事輕看了。如今衙門裡監着那兩個船家供着,只要他哩。拿過幾兩銀子來,也不勾打發腳下人的哩。我不管別的帳,韓大嬸和他說,只與我二十兩銀子罷!我請將俺爹來,隨你老人家與俺爹說就是了。」王六兒笑道:「怪油嘴兒,要飯吃休要惡了火頭!事成了,你的事甚麼打緊?寧可我們不要,也少不得了你的。」玳安道:「韓大嬸,不是這等說。常言:『君子不羞,當面先斷,過後商量。」王六兒當下備幾樣菜,留玳安吃酒。玳安道:「吃得的紅頭紅臉,咱家爹問,都怎的回爹?」王六兒道:「怕怎的?你就說在我這裡來。」于是玳安只吃了一甌子就走了。王六兒道:「你到好歹累你說,我這裡等着哩。」玳安一直上了頭口來家,交進毡包,後邊立等的。西門慶房中睡了一覺出來,在廂房中坐的。這玳安慢慢走到根前,無得說:「小的回來,韓大嬸叫住小的,要請爹快些過去,有句要緊話和爹說。」西門慶說:「甚麼話?我知道了。」說畢,正值劉學官來借銀子,打發劉學官去了,西門慶騎馬,帶着眼紗小帽,便叫玳安、琴童兩個跟隨,來到王六兒家,下馬進去,到明間客位坐下。王六兒出來拜見了。那日韓道國因來前邊舖子裡該上宿,沒來家。老婆買了許多東西,叫老馮廚下整治,等候西門慶。一面丫鬟錦兒拿茶上來,婦人遞了茶。西門慶分付琴童把馬送到對門房子裡去,把大門關上。婦人且不敢就題此事,先只說:「爹家中連日擺酒辛苦,我聞得說哥家中定了親事,你老人家喜呀!」西門慶道:「只因舍親吳大嫂那裡說起,和喬家做了這門親事。他家也只這一個女孩兒。論起來也還不敢陪,胡亂親上做親罷了。」王六兒道:「就是和他做親也好,只是爹如今居着恁大官,會在一處不好意思的。」西門慶道:「說甚麼哩!」說了一回,老婆道:「只怕爹寒冷,往房裡坐去罷。」一面讓至房中,一面安着一張椅兒,籠着火盆。西門慶坐下,婦人慢慢先把苗青揭帖拿與西門慶看,說:「他央了間壁經紀樂三娘子過來對我說,這苗青是他店裡客人,如此這般,被兩個船家拽扯,只望除豁了他這名字,免提他。他備了些禮兒在此謝我,好歹望老爹怎的將就他罷。」西門慶看了帖了,因問:「他拿甚禮物謝你?」王六兒向廂中,取出五十兩銀子來與西門慶瞧,說道:「明日事成,還許兩套衣裳。」西門慶看了笑道:「這些東西兒,平白你要他做甚麼?你不知道,這苗青乃揚州苗員外家人,因為在船上與兩個船家商議,殺害家主,攛在河裡,圖財謀命。如今見打撈不着屍首;又當官兩個船家招尋他。原跟來的一個小廝安童,又當官三口執證着要他這一個過去,穩定是個凌遲罪名。那兩個都是真犯斬罪。兩個船家見供他有二千兩銀貨在身上。拿這些銀子來做甚麼?還不快送與他去。」這王六兒一面到廚下使了丫頭錦兒,把樂三娘子兒叫了來,將原禮交付與他,如此這般對他說了去。那苗青不聽便罷,聽他說了,猶如一桶水,頂門上直灌到腳底下。正是:

「驚駭六葉連肝膽,  唬壞三魂七魄心。」

即請樂三一處商議道:「寧可把二千貨銀都使了,只要救得性命家去。」樂三道:「如今老爹上邊即發此言,一些半些,恆屬打不動兩位官府,頂得湊一千貨物與他。其餘節級、原解、緝捕,再得一半,纔得勾用。」苗青道:「况我貨物未賣,那討銀子來?」因使過樂三娘來,和王六兒說:「老爹就要貨物,發一千兩銀子貨與老爹。如不要,伏望老爹再寬限兩三日,等我倒下價錢,將貨物賣了,親往老爹宅裡進禮去。」王六兒拿禮帖復到房裡,西門慶瞧。西門慶道:「既是恁般,我分付原解且寬限他幾日拿他。教他即便進禮。」當下樂三娘子得此口詞,回報苗青。苗青滿心歡喜。西門慶見間壁有人,也不敢久坐,吃了幾鍾酒,與老婆坐了一回,見馬來接,就起身家去了。次日,到衙門早發放,也不題問這件事。分付緝捕:「你休捉這苗青。」就托經紀樂三,連夜替他會了人,攛掇貨物出去。那消三日,都發盡了,共賣了一千七百兩銀子。把原與王六兒的不動,另的五十兩銀子,又另送他四套上色衣服,且說十九日,苗青打點一千兩銀子,裝在四個酒壜內,又宰一口豬,約掌燈已後時分,抬送到西門慶門首。手下人都是知道的。玳安、平安、書童、琴童四個禁子,與了十兩銀子纔罷。玳安在王六兒這邊,梯已又要十兩銀子。須臾,西門慶出來,捲棚內坐的,也不掌燈。月色朦朧纔上來,抬至當面,苗青穿着青衣,望西門慶只顧磕着頭,說道:「小人蒙老爹超拔之恩,粉身碎骨,死生難報!」西門慶道:「你這件事情,我也還沒好審問哩。那兩個船家甚是攀你。你若出官,也有老大一個罪名。即是人說,我饒了你一死。此禮我若不受你的,你也不放心。我還把一半送你掌刑夏老爹,同做分上。你不可久住,即便星夜回去。」因問:「你在楊州那裡?」苗青磕頭道:「小的在楊州城內住。」西門慶分付後邊拿了茶來,那苗青在松樹下立着吃了,磕頭告辭回去。又叫回來問:「下邊原解的,你都與他說了不曾說?」苗青道:「小的外邊人說停當了。」西門慶分付:「既是說了,你即回家。」那苗青出門,走到樂三家收拾行李,還剩一百五十兩銀子。苗青拿出五十兩來并餘下幾疋段子,都謝了樂三夫婦。五更替他僱長行牲口,起身往楊州去了。正是:

「忙忙如喪家之狗,  急急似漏網之魚。」

不說苗青逃出性命不題,單表西門慶、夏提刑從衙門中散了出來,並馬而行。走到大街口上,夏提刑要作辭分路。西門慶在馬上舉着馬鞭兒說道:「長官不棄,降到舍下一敍。」把夏提刑邀到家來。門首同下了馬,進到廳上敍禮。請入捲棚內,寬了衣服,左右拿茶上來吃了。書童、玳安走上,安放卓席擺設。夏提刑道:「不當閑來打攪長官。」西門慶道:「豈有此理!」須臾兩個小廝,用方盒拿了小菜,就在傍邊擺下,各樣雞啼、鵝、鴨、鮮魚下飯,就是十六碗。吃了飯,收了家火去,就是吃酒的各樣菜蔬出來。小金把鍾兒、銀臺盤兒,金鑲象牙筯兒。飲酒中間,西門慶慢慢題起苗青的事來:「這廝昨日央及了個士夫,再三來對學生說,又餽送了些禮在此。學生不敢自專,今日請長官來,與長官計議。」于是把禮帖遞與夏提刑。夏提刑看了,便道:「恁憑長官尊意裁處。」西門慶道:「依着學生,明日只把那個賊人真賍送過去罷,也不消要這苗青。那個原告小廝安童,便收領在外。待有了苗天秀屍首,歸給未遲。禮還送到長官處。」夏提刑道:「長官這些意就不是了。長官見得極是,此是長官費心一場,何得見讓於我?斷然使不得!」彼此推辭了半日,西門慶不得已,還把禮物兩家平分了,裝了五百兩在食盒內。夏提刑下席來,也作揖謝說道:「既是長官見愛,我學生再辭,顯的迂闊了。盛情感激不盡,實為多愧!」又領了幾盃酒,方纔告辭起身。這裡西門慶隨即就差玳安拿了盒,還當酒抬送到夏提刑家。夏提刑親在門上收了。拿回帖,又賞了玳安二兩銀子,兩名排軍四錢,俱不在話下。常言道:「火到豬頭爛,錢到公事辦。」且說西門慶、夏提刑已是會定了,次日到衙門裡陞廳,那提孔節級,并緝捕觀察,都被樂三替苗青上下打點停當了。擺設下刑具,監中提出陳三、翁八,審問情由,只是供稱:「跟伊家人苗青同謀。」西門慶大怒,喝令:「左右與我用起刑來!你兩個賊人,專一積年在江河中,假以舟緝裝載為名,實是劫幫鑿漏,邀截客旅,圖財致命。見有這個小廝供稱是你等持刀戮死苗天秀波中,又將棍打傷他落水。見有他主人衣服存證,你如何抵頭賴別人?」因把安童提上來,問道:「是誰剌死你主人,推在水中來?」安童道:「某日夜至三更時分,先是苗青叫有賊,小的主人出船艙觀看,被陳三一刀戮死,推在水中。小的便被翁八一棍打落水中,纔得逃出性命。苗青並不知下落。」西門慶道:「據這小廝所言,就是實話。汝等如何展轉得過?」於是每人兩夾棍、三十根頭,打的脛骨皆碎,殺豬也似叫動。他一千兩賍貨,已追出大半。餘者花費無存。這裡提刑連夜做了文書,歇過賍貨,申詳東平府。府尹胡師文,又與西門慶相交,照依原行文書,叠成案卷,將陳三、翁八問成強盜殺人斬罪。只把安童保領在外聽候。有日走到東京投到開封府黃判通衙內,具訴苗青情:「奪了主人家事,使錢提刑,除了他名字出來。主人冤仇,何時得報?」黃通判聽了,連夜修書,并他訴狀封在一處,與他盤費,就着他往巡按山東察院裡投下。這一來,管教苗青之禍,從頭上起,西門慶往時做過事,今朝沒興一齊來!有詩為證:

「善惡從來畢有因,  吉凶禍福並肩行;

平生不作虧心事,  夜半敲門不乞驚!」

畢竟未知後來何如,且聽下回分解:
