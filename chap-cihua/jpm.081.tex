%# -*- coding: utf-8 -*-
%!TEX encoding = UTF-8 Unicode
%!TEX TS-program = xelatex
% vim:ts=4:sw=4
%
% 以上设定默认使用 XeLaTex 编译,并指定 Unicode 编码,供 TeXShop 自动识别

%第八十一回 
\chapter{韓道國拐財倚勢\KG 湯來保欺主背恩}

「萬事從天莫強尋,  天公報應自分明,

貪淫縱意奸人婦,  背主侵財被不仁;

莫道身亡人弄鬼,  由來勢敗僕忘恩,

堪嘆西門成甚業,  贏得奸徒富半生。」

話說韓道國與來保兩個,自從西門慶將二千兩銀子,打發他在江南等處置買貨物。一路餐風宿水,夜住曉行。到于揚州去處,抓尋苗青家內宿歇。苗青見了西門慶手札,想他活命之恩,儘力趨奉。他兩個成尋花問柳,飲酒取樂。一日初冬天氣,寒雲淡淡,哀雁凄凄,樹木彫零,景物蕭瑟,不勝旅思。于是兩人連忙將銀往各處置了布疋,裝在楊州苗青家安下,待貨物買完起身。先是韓道國舊日請的表子楊州舊院王玉枝兒,來保便請了林彩虹妹子小紅,日逐請楊州鹽客王海峯和苗青遊寶應湖。遊了一日,歸到院中。玉枝兒鴇子生日,這韓道國又邀請眾人擺酒,與鴇子王一媽做生日。使後生胡秀置辦酒肴果菜,又使他請客商汪東橋與錢晴川兩個,又不見到,想他就同王海峯來了。至日落時分,胡秀纔來,被韓道國帶酒罵了幾句,說:「這廝不知在那〈口床〉酒,〈口床〉得這咱纔來!口裏噴出來酒氣!客人也先來了已半日,你不知那裡來?我到明日定弄你出去!」那胡秀把眼瞅着他,走到下邊,口裏喃喃吶吶說:「你罵我?你家老婆在家裡仰搧着掙,你在這裡合蓬着丟!宅裡老爹,包着你家老婆,{入日}的不值了,纔交你領本錢出來做買賣!你在這裡快活,你老婆不知怎麼受苦哩!得人不化,白出你來?你落得為人!」對玉枝兒鴇子只顧說鴇子。便拉出他院子裡,說:「胡官人,你醉了,你往房裡睡去罷!」那胡秀大腰小喝,白不進房來。不料韓道國正陪眾客商在席上吃酒,身穿着白綾道袍,線絨氅衣,毡鞋羢襪,聽見胡秀口內放屁辣臊,心中大怒,走出來。踹了兩腳,罵道:「賊野囚奴,我有了五分銀子雇你一日,怕尋不出人來!」即時趕他去。那胡秀那裡肯出門,在院子內聲叫起來,說道:「你如何趕我?我沒壞了管帳事。你倒養老婆,倒攆我?看我到家說不說!」被來保勸住韓道國,手拉他過一邊,說道:「你這狗骨頭,原來這等酒硬?」那胡秀道:「保叔,你老人家休管他!我吃甚麼酒來?我和他做一做!」被來保推他往屋裡挺覺去了。正是:

「酒不醉人人自醉,  色不迷人人自迷!」

來保打發胡秀房裡睡去不題。韓道國恐眾怕眾客商耻笑,和來保席上觥籌交錯,遞酒開笑。林彩虹、小紅姊妹二人并王玉枝兒,三個唱的,彈唱歌舞。花攢錦簇,行令猜枚,吃至三更方散。次日,韓道國要打胡秀。胡秀說:「小的道不曉一字!」被來保、苗小湖做好做歹,勸住了。話休饒舌,有日貨物置完,打包裝載上船。苗青打點人事禮物,抄寫書帳,打發二人。并胡秀起身。王玉枝并林彩虹姊妹,少不的置酒馬頭,作別餞行。從正月初十日起身,一路無詞。一月,前臨行閘上,這韓道國正在船頭上站立,忽見街坊嚴四郎從上流坐船而來,往臨江接官去。看見韓道國舉手說:「韓四橋,你家老爹從正月間沒了!」說畢,船行得快,就過去了。這韓道國聽了此言,遂安心在懷,瞞著來保,不對他說。不想那時河南山東大旱,赤地千里,田蚕荒蕪不收,棉花布價,一時踊貴,每疋布帛,加三利息,各處鄉販,都打着銀兩遠接,在臨清一帶馬頭,迎着客貨而買。韓道國便與來保商議:「船上布貨,約四千餘兩。見今加三利息,不如且賣一半,便益鈔關納稅。就到家發賣,也不過如此。遇行市不賣,誠為可惜!」來保道:「夥計所言雖是,誠恐賣了一時到家,惹當家財主見怪,如之奈何?」韓道國便說:「老爹見怪,都在我身上。」來保只得強不過他,在馬頭上發賣了一千兩布貨。韓道國說:「雙橋你和何秀在船上等着納稅。我打旱路,同小郎王漢,打着這一千兩銀子,裝成馱垛,先行一步家去,報老爹知道。」來保道:「你到家,好歹討老爹一封書來。下與鈔關錢老爹,少納稅錢,先放船行。」韓道國應諾,同小郎王漢裝成馱垛,往清河縣家中來,不在言表。有日進城,在甕城南門裏,日色漸落。不想路上撞遇西門慶家看墳的張安,推着車輛酒米食盒,正出南門。看見韓道國便叫:「韓大叔,你來家了!」韓道國看見他帶着孝,問其故。張安說:「老爹死了,明日三月初九日是斷七,大嫂交我拏此酒米食盒往墳上去,明日墳上與老爹燒布去也。」這韓道國聽了,說:「可傷,可傷!果然路上行人口似碑,話不虛傳。」打頭口逕進城中,那時天已漸晚。但見:

「十字街熒煌燈火,九曜廟香靄鐘聲。一輪明月掛疏林,幾點疏星明碧落。六軍營內,嗚嗚畫角頻吹;五鼓樓頭,點點銅壺雙滴。四邊宿霧,昏昏罩舞榭歌臺;三市沉烟,隱隱閉綠窗朱戶。兩兩佳人歸綉 ,紛紛仕子捲書幃。」

這韓道國進城來,到十字街上,心中算計:「且住;有心要往西門慶家去,況令他已死了,天色又晚,不如且歸家,停宿一宵,和渾家商議了,明日再去不遲。」于是和王漢打着頭口,逕到獅子街家中。二人下了頭口,打發趕腳人回去。叫開門,王漢搬行李馱垛進來。有丫鬟看見,報與王六兒說:「爹來家了。」老婆一面迎接入門。拜了佛祖,拂去塵土,馱垜搭連放在堂中。王六兒替他脫衣坐下,丫鬟點茶吃。韓道國先告訴往回一路之事:「我在路上撞遇嚴四哥,說老爹死了。剛纔來到城外,又撞見墳頭張安推酒米往墳上去,說明日是斷七,果不虛傳。端的好好的怎的死了?」王六兒道:「天有不測風雲,人有旦時禍福!誰人保得無常?」韓道國一面把馱垛打開,裏面是他江南置的衣裳,細軟貨物,兩條搭連內,倒中那一千兩銀子,一封一封倒在坑上。打開都是白光光雪花銀兩。對老婆說:「此是我路上賣了這一千兩銀子先來了。」又是兩包梯已銀子一百兩:「今日晚了,明日早送與他家去罷。」因問老婆:「我去後,家中他先看顧你不曾?」王六兒道:「他在時倒也罷了!如今你這銀,還送與他家去?」韓道國道:「正是要和你商議,咱留下些,把一半與他如何?」老婆道:「呸!你這傻才,這遭再休要傻了!如今他已是死了,這裡無人,咱和他有甚瓜葛?不爭你送與他一半,交他招韶道兒,問你下落!到不如一狠二狠,把他這一千兩咱顧了頭口,拐了上東京,投奔咱孩兒那裡。愁咱親家太師爺府中,招放不下你我?」韓道國說:「丟下這房子,急切打發不出去,怎了?」老婆道:「你看沒才料!何不叫將第二個來,留幾兩銀子與他,就交他看守便了。等西門慶家人來尋你,只說東京咱孩兒叫了兩口去了。莫不他七個頭八個膽,敢往太師府中尋咱們去?就尋去,你我也不怕他!」韓道國說:「爭奈我受大官人好處,怎好變心的?沒天理了!」老婆道:「自古有天理,到沒飯吃哩!他占用着老娘,使他這幾兩銀子,不差甚麼!想着他孝堂,我到好意備了一張插卓三牲,往他家燒布。他家大老婆,那不賢良的淫婦,半日不出來,在屋裡罵的我好訕的!我出又出不來,坐又坐不住。落後他第三個老婆出來,陪我坐;我不去坐,坐轎子來家。想着他這個情兒,我也該使他這幾兩銀子!」一席話,說得韓道國不言語了。夫妻二人,晚夕計議已定。到次日五更,叫將他兄弟韓二來,如此這般,交他看守房子。又把與他一二十兩銀子盤纏。那二搗鬼千肯萬肯說:「哥嫂只顧去,等我打發他。」這韓道國就把王漢小郎,并兩個丫頭,也跟他帶上東京去;僱了二輛大車,把箱籠細軟之物,都裝在車上,投天明出西門,逕上東京去了。正是:

「撞碎玉籠飛彩鳳,  頓斷金鎖走蛟龍。」

這裡韓道國夫妻東京去不題。單表吳月娘次日帶孝哥兒,同孟玉樓、潘金蓮、西門大姐、奶子如意兒、女婿陳經濟,往墳上與西門慶燒布。墳頭告訴月娘把昨日撞見韓大叔來家一節。月娘道:「他來了,怎的不到家裡來?只怕他今日來。」在墳上剛燒了布,坐了沒多回,老早就趕了來家。使陳經濟往他家叫韓夥計去,問他船到那裡了。初時叫著,不聞人言。次則韓二出來,說:「俺姪女兒東京叫了哥嫂去了。船不知在那裡!」這陳經濟回月娘,月娘不放心,使陳經濟騎頭口,往河下尋舟去了。三日到臨清馬頭船上,尋著來保船隻。來保問:「韓夥計先打了一千兩銀子家去了?」經濟道:「誰見他來?張安看見他進城,次日墳上來家,大娘使我問他去。他兩口子絜家連銀子,都拐的上東京去了。如今爹死了,斷七過了。大娘不放心,使我來找尋船隻。」這來保口中不言,心內暗道:「這天殺,原來連我也瞞了!嗔道路上賣了這一千兩銀子,乾淨要起毛心!正是人面咫尺,心隔千里!」當下這來保見西門慶已死,也安心要和他一路。把經濟小夥兒引誘在馬頭上各唱店中,歌樓上飲酒,請表子頑耍。暗暗船上搬了八百兩貨物,卸在店家房內,封記了。一日鈔關上納了稅,放船過來,在新河口起腳裝車,往清河縣城裡來,家中東廂房卸下。那時自從西門慶死了,獅子街絲綿舖已關了。對門段舖,甘夥計、崔本賣貨銀兩,都交付明白,各辭歸家去了;房子也賣了。止有門首解當生藥舖,經濟與傅夥計開着。這來保妻惠祥,有個五歲兒子,名僧寶兒;韓道國老婆王六兒,有個姪女兒四歲,二人割衿,做了親家。家中月娘通不知道。這來保交卸了貨物,就一口把事情都推在韓道國身上,說他先賣了二千兩銀子來家。那月娘再三使他上東京,問韓道國銀子下落,被他一頓話,說:「咱早休去!一個太師老爺府中,誰人敢到?沒的招是惹非!得他不來尋趁,咱家念佛;到沒的招惹虱子頭上撓!」月娘道:「翟親家也虧咱家替他保親,莫不看些分上兒?」來保道:「他家女兒見在他家得時,他敢只護他娘老子,莫不護咱不成?此話只好在家對我說罷了;外人知道,傳出去,到不好了!這幾兩銀子罷,更休題了。」月娘交他會買頭,發賣布貨。他甫會了主兒,月娘交陳經濟兌銀講價錢。主兒都不服,拏銀出去了。來保便說:「姐夫,你不知買賣甘苦,俺在江湖上走的多,曉的行情。寧可賣了悔,休要悔了賣!這貨來家,得此價錢就勾了。你十分把弓兒拽滿,迸了主兒,顯得不會做生意!我不是托大說話,你年少不知事體!我莫不胳膊兒外撇?不如賣弔了是一場事!」那經濟聽了,使性兒不管了。他不等月娘分付,匹手奪過算盤來,邀回主兒來,把銀子兌了二千餘兩,一件件交付與經濟經手,交進月娘收了,推貨出門。月娘與了陳經濟二三十兩銀子房中盤纏。他便故意兒昂昂大意不收,說道:「你老人家還收了。死了爹,你老人家死水兌自家盤纏,又與俺們做甚?你收了去,我決不要!」一日晚夕,外邊吃的醉醉兒,走進月娘房中,搭伏着護炕,說念月娘:「你老人家青春少小,沒了爹,你自家守着這點孩兒子,不害孤另麼?」月娘一聲兒沒言語。一日東京翟管家寄書來,知道西門慶死了,聽見韓道國說他家中有四個彈唱出色女子,該多價錢,說了去,兌銀子來,要載到京答應老太太。月娘見書,慌了手腳,叫將來保來計議:「與他去好,不與他去好?」來保進入房中,也不叫娘,只說:「你娘子人家不知事!不與他去,就惹下禍了!這個都是過世老頭兒惹的,恰似賣富一般,但擺酒請人,就交家樂出去,有個不傳出去的?何況韓夥計女兒,又在府中答應老太太,有個不說的?我前日怎麼說來,今果然有此勾當鑽出來!你不與他,他裁派府縣差人坐名兒來要,不怕你不雙手兒奉與他,還是遲了!不如今日,難說四個都與他,胡亂打發兩個與他,還做面皮!」這月娘沉吟半晌,孟玉樓房中蘭香,與金蓮房中春梅,都不好打發。綉春又要看哥兒,不出門,問他房中玉簫與迎春,情願要去。以此就差來保僱車輛,裝載兩個女子,出門往東京太師府中來。不料來保這廝,在路上把這兩個女子都姦了。有日到東京,會見韓道國夫婦,把前後事都說了:「若不是親家看顧我,在家阻住;我雖然不怕他,也不敢來東京尋我。」翟謙看見兩個女子迎春、玉簫都生的好模樣兒,一個會箏,一個會絃子,都不上十七、八歲;進入府中伏侍老太太,賞出兩錠元寶來。這來保還尅了一錠,到家只拏出一錠元寶來與月娘,還將言語恐嚇月娘:「若不是我去,還不得他這錠元寶拏家來。你還不知韓夥計兩口兒,在那府中好不受用富貴!獨自着住一所宅子,呼奴使婢,坐五行三,翟管家以老爺呼之!他家女孩兒韓愛姐,日逐上去答應老太太,寸步不離,要一奉十,揀口兒吃用,換套穿衣。如今又會寫又會算,福至心靈,出落得好長大身材,姿容美貌!前日出來見我,打扮的如瓊林玉樹一般,百伶百俐,一口一聲,叫我保叔。如今咱家這兩個家樂,到那裡,還在他手裡討針線哩!」說畢,月娘還甚是知感他不盡,打發他酒饌吃了。與他銀子,又不受;拏了一疋段子,與他妻惠祥做衣服穿,不在話下。這來保一日同他妻弟劉倉往臨清馬頭上,將封寄店內布貨,盡行賣了八百兩銀子,暗買下一所房子在外邊,就來劉倉右邊門首,開雜貨舖兒。他便日逐隨倚祀會茶。他老婆惠祥,要便對月娘說,假推往娘家去,到房子裡從新換了頭面衣服珠子箍兒,插金戴銀,往王六兒娘家王母猪家,扳親家,行人情,坐轎看他家女兒去。來到房子裡,依舊換了慘淡衣裳,纔往西門慶家中來。只瞞過月娘一人不知。來保這廝,常時吃醉了,來月娘房中嘲話調戲,兩番三次。不是月娘為人正大,也被他說念的心邪,上了道兒!又有一般家奴院公,在月娘根前,說他媳婦子在外與王母猪作親家,插金戴銀,行三坐五。潘金蓮他也對月娘說了幾次,月娘不信。惠祥聽見此言,在廚房中罵大罵小;他便裝胖學蠢,自己誇獎說眾人:「你每只好在家裡說炕頭子上嘴罷了!相我,水皮子上顧瞻將家中這許多銀子貨物來家!若不是我,都乞韓夥計老牛箝嘴,拐了往東京去。只呀的一聲,乾丟在水裡也不响!如今還不得俺每一個是,說俺轉了主子的錢了,架俺一篇是非!正是割股也不知,撚香的也不知!自古信人調,丟了瓢!」他媳婦子惠祥便罵:「賊嚼舌根的淫婦!說俺兩口子轉的錢大了,在外行三坐五,扳親家!老道出門,問我姊那裡借的衣裳,幾件子首飾,就說是俺落得主子銀子治的!要擠撮俺兩口子出門,也不打緊,等俺每出去!料莫天也不着餓老鴉兒吃草!我洗淨着眼兒,看你這些淫婦奴才,在西門慶家裡住牢着!」月娘見他罵大罵小,尋由頭兒和人嚷鬧上弔;漢子又兩番三次無人處在根前無禮,心裡也氣得沒入腳處,只得交他兩口子搬離了家門。這來保就大利利和他舅子開起個布鋪來,發賣各色細布。日逐會倚祀,行人情,不在話下。正是:

「勢敗奴欺主,  時衰鬼弄人!」

有詩為證:

「我勸世間人,  切莫把心欺,

欺心即欺天,  莫道天不知,

天只在頭上,  昭然不可欺。」

畢竟未知後來何如,且聽下回分解:

