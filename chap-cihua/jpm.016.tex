%# -*- coding: utf-8 -*-
%!TEX encoding = UTF-8 Unicode
%!TEX TS-program = xelatex
% vim:ts=4:sw=4
%
% 以上设定默认使用 XeLaTex 编译,并指定 Unicode 编码,供 TeXShop 自动识别

%第十六回 
\chapter{西門慶謀財娶婦\KG 應伯爵喜慶追歡}

「傾城傾國莫相疑,  巫水巫雲夢亦癡,

紅粉情多銷駿骨,  金蘭誼薄惜蛾眉;

溫柔鄉裡精神健,  窈窕風前意態奇,

村子不知春寂寂,  千金此夕故踟躕」

話說當日西門慶出離院門,玳安跟隨打馬,逕到獅子街李瓶兒家門首下馬。見大門關的緊緊的,就知堂客轎子家去了。一面叫玳安問馮媽媽開門,西門慶進來。李瓶兒堂中秉燭,花冠齊整,素服輕盈,正倚簾櫳,口中磕瓜子兒。見西門慶來,忙輕移蓮步,款蹙湘裙,下階迎接,笑道:「你早來些兒,他三娘、五娘還在這裡。只剛纔轎子起身,往家裡去了。今日他大娘去的早,說你不在家。那裡去了?」西門慶道:「今日我和應二哥、謝子純早晨看燈,打你門首過去來。不想又撞見兩個朋友,都拉去院裡家走,撞到這咱晚。我又恐怕你這裡等候,小廝去時,教我推淨手打後門跑了。不然,必吃他們掛住了,休想來的成。」李瓶兒道:「適問多謝官人重禮。他娘每又不肯坐只說家裡沒人。教奴到沒意思的。」于是重篩美酒,再設佳餚。堂中把花燈都點上,放下暖簾來。金爐添獸炭,寶篆熱龍涎。春臺上高堆異品,看杯中香醪滿泛。婦人遞與西門慶酒,磕下頭去,說道:「拙夫已故,舉眼無親。今日此杯酒只靠官人與奴作個主兒。休要嫌奴醜陋,奴情願與官人鋪床疊被,與眾位娘子作個姊妹,奴死也甘心。不知官人心下如何?」說着滿眼落淚。西門慶一壁接酒,一壁笑道:「你請起來!既蒙你厚愛,我西門慶銘刻于心。待你孝服滿時,我自有處,不勞你費心。今日是你的好日子,咱每且吃酒。」西門慶于是吃畢,亦滿斟了一杯,回奉婦人,安他上席坐下。馮媽媽單管廚下看菜兒。須臾拿麵上來吃。西門慶因問李瓶兒:「今日是董嬌兒、韓金釧兒兩個在這裡。臨晚送他三娘、五娘家中討花兒去了。」西門慶坐席左,兩個在席上交杯換盞飲酒。迎春、秀春兩個丫鬟在傍,斟酒下菜伏侍。只見玳安上來,扒在地下,與李瓶兒磕頭拜壽。李瓶兒連忙起身,還了萬福。分付迎春:「教老馮廚下看壽麵點心下飯,拿一壺酒與玳安吃。」西門慶分付:「吃了早些回馬家去罷。」李瓶兒道:「到家裡你娘問,只休說你爹在這裡。」玳安道:「小的知道。只說爹在裏邊過夜,明日早來接爹就是了。」西門慶便點了點頭兒。當下把李瓶兒喜歡的要不的,說道:「好個乖孩子!眼裡說話!」即令迎春拿二錢銀子,節間叫買瓜子兒磕:「明日你拿個樣兒來,我替你做雙好鞋兒穿。」那玳安連忙磕頭,說:「小的怎麼敢?」走到下邊,比了酒飯,帶馬出門。馮媽媽把大門上了拴。李瓶兒同西門慶猜枚吃了一回,又拿一副三十二扇象牙牌兒,卓上鋪茜紅毡條,兩個燈下抹牌飲酒。吃一回,分付迎春房裡秉燭。原來花子虛死了,迎春、秀春都已被西門慶要了,以此凡事不避他。教他收拾牀鋪,拿菓盒杯酒。又在牀上紫錦帳中,婦人露着粉般身子,西門慶香肩相並,玉體廝挨。兩個看牌,拿大鍾飲酒。因問西門慶:「你那邊房子幾時收拾?」西門慶道:「且待二月間興工動土。連你這邊一所,通身打開,與那邊花園取齊。前邊起蓋山子捲棚,花園耍子去處。還蓋三間玩花樓。」婦人因指道:「奴這牀後茶葉箱內,還藏着四十斤沉香、二百斤白蠟、兩罐子水銀、八十斤胡糊椒。你明日都搬出來,替我賣了銀子,湊着你蓋房子使。你若不嫌奴醜陋,到家好歹對大娘說,奴情願只要與娘們做個姊妹,隨問把我做第幾個的也罷。親親,奴捨不的你!」說着,眼淚紛紛的落將下來。西門慶慌把汗巾替他抹拭,說道:「你的情意,我知道。也待你這邊孝服滿,我那邊房子蓋了纔好。不然娶你過去,沒有住房。」婦人道:「既有實心取奴家去,到明好歹把奴的房,蓋的與他五娘在一處。奴捨不的他,好個人兒!與後邊孟家三娘,見了奴且親熱。兩個天生的,打扮也不相兩個姊妹,只相一個娘兒生的一般。惟有他大娘性兒不是好的,快眉眼裡掃人。」西門慶道:「俺吳家的這個拙荊,他到好性兒哩。不然,手下怎生容得這些人?明日這邊與那邊,一樣蓋三間樓,與你居住,安兩個角門兒出入。你心下何如?」婦人道:「我的哥哥!這等纔可奴之意。」于是兩個顛鸞倒鳳,淫慾無度。狂到四更時分,方纔就寢。枕上並肩交股,直睡到次日飯時不起來。婦人且不梳頭,迎春拿進粥來,只陪着西門慶吃了上半盞粥兒。又拿酒來,二人又吃。原來李瓶兒好馬爬著,教西門慶坐在枕上,他倒插花,往來自動。兩個正在美處,只見玳安兒外邊打門,騎馬來接。西門慶喚他在窗下問他話。玳安說:「家中有三個川廣客人,在家中坐着。有許多細貨,要科兌與傅二叔,只要一百兩銀子押合同,其餘八月中旬找完銀子。大娘使小的來請爹家去,理會此事。」西門慶道:「你沒說我在這裡?」玳安道:「小的只說爹在裡邊桂姨家,沒說在這裡。」西門慶道:「你看不曉事!教把傅二叔打發他便了,又來請我怎的?」玳安道:「傅二叔講來,客人不肯,直等我爹去,方纔批合同。」李瓶兒道:「既是家中使了孩子來請,買賣要緊。你不去,惹的他大娘不怪麼?」西門慶道:「你不知賊蠻奴才行巿,連貨物沒處發脫,纔來上門脫與人,遲半年三個月找銀子。若快時,他就張致了。滿清河縣,除了我家舖子大,發貨多,隨問多少時,不怕他不來尋我。」婦人道:「買賣不與道路為仇。只依奴到家,打發了再來也。往後日子,多如柳葉兒哩。」西門慶于是依聽李瓶兒之言,慢慢起來,梳頭淨面,戴網巾,穿衣服。李瓶兒收拾飯與他吃。西門慶一直帶着個眼紗,騎馬來家。舖子裡有四五個客人,等候秤貨兌銀。批了合同,打發去了。走到潘金蓮房中。便問:「你昨日往那裡去來?實說便罷,不然我就嚷的塵鄧鄧的。」西門慶道:「你們都在花家吃酒,我和他每燈巿裡走了回來,同往裡邊吃酒過一夜。今日小廝接去,我纔來家。」金蓮道:「我知小廝去接,那院裡有你那魂兒罷麼?賊負心!你還哄我哩!那淫婦昨日打發俺每來了,弄神弄鬼的,晚夕叫了你去{入日}搗了一夜。{入日}搗的了,纔放來了。玳安這賊囚根子,久慣兒牢成!對着他大娘,又一樣話兒,對着我又是一樣話兒。先是他回馬來家,他大娘又是問他:『你爹怎的不來家?在誰家吃酒哩?』他回話:『和應二叔眾人,看了燈回來,都在院裡李桂姨家吃酒,教我明早接去哩。』落後我叫了問他,他笑不言語。問的急了,纔說:『爹在獅子街花二娘那裡哩。』賊囚根!他怎的就知我何你一心一計?想必你叫他話來?」西門慶哄道:「我那裡教他。」于是隱瞞不住,方纔把:「李瓶兒晚夕請我去到那裡與我遞酒,說要過你每來了。又哭哭啼啼告訴我說,他沒人手,後半截空,晚夕害怕。一心要教我取他。問幾時收拾這房子。他還有些香蠟細貨,也直幾百兩銀子,教我會經紀,替他打發銀子。教我收湊着蓋房子,上緊修蓋。他要和你一處住,與你做了姊妹,恐怕你不肯。」婦人道:「我也不多着個影兒在這裡,巴不的來總好。我這裡也空落落的,得他來與老娘做伴兒。自古船多不礙港,卓多不礙路。我不肯招他,當初那個怎麼招我來?攙奴甚麼分兒也怎的?倒只怕人心不似奴心。你還問聲大姐姐去。」西門慶道:「雖故是恁說,他孝服還未滿哩。」說畢,婦人與西門慶盡脫白綾襖,袖子裡滑浪一聲吊出個物件兒來。拿在手內,沉甸甸的紹彈子大,認了半日,竟不知甚麼東西。但見:

「原是番兵出產,逢人荐轉在京。身軀瘦小內玲瓏,得人輕借力,展轉作蟬鳴。解使佳人心膽,慣能助腎威風。號稱金面勇先鋒,戰降功第一,揚名勉子鈴。」

婦人認了半日,問道:「是甚麼東西兒?怎的把人半邊胳膊都麻了?」西門慶笑道:「這物件你就不知道了,名喚做勉鈴,南方勉甸國出產的。好的也值四五兩銀子。」婦人道:「此物使到那裡?」西門慶道:「先把他放入爐內,然後行事,妙不可言。」婦人道:「你與李瓶兒也幹來?」西門慶于是把晚間之事,從頭告訴一遍。說得金蓮淫心頓起,兩個白日裡,掩上房門,解衣上牀交歡。正是:

「不知子晉緣何事?  纔學吹簫便作仙。」

話休饒舌。一日西門慶會了經紀,把李瓶兒牀後茶葉箱內堆放的香蠟等物,都秤了斤兩,共賣了三百八十兩銀子。李瓶兒只留下一百八十兩盤纏,其餘都付與西門慶收了,湊着蓋房。便教陰陽擇用二月初八日,興工動土。五百兩銀子委付大家人來招,并主管賁四,卸磚瓦木石,管工計帳。這賁四名喚賁地傳,年少生的百浪囂虛,百能百巧。原是內相勤兒出身,因不守本分,打出吊入滑流水,被趕來。初時跟着人做兄弟兒來,次後投入大人家做家人,把人家奶子拐出來做了渾家。都在故衣做經紀,琵琶簫管都會。西門慶見他這般本事,常照顧他在生藥舖中秤貨,討中人錢使。以此凡大小事情,少他不得。當日賁地傳與來招,督管各作匠人興工。先拆毀花家那邊舊房,打開牆垣,築起地腳,蓋起捲棚山子,各亭臺耍子去處,非止一日,不必盡說。光陰迅速,日月如梭。西門慶在家看管起蓋花園,約有一個月有餘。都在三月上旬,乃花子虛百日。李瓶兒預先請過西門慶去和他計議,要把花子虛靈燒了:「房子賣的賣。不的,你着人來看守。你早把奴取過去罷,省的奴在這裡,晚夕空落落的,我害怕,常有狐狸鬼混的慌,你到家對大娘說,只當可憐見奴的性命罷。隨你把奴做第幾個,奴情願伏侍你,鋪牀疊被,也無抱怨。」說着,淚如雨下。西門慶道:「你休煩惱。前日我把你這話,到家對房下和潘五姐也說過了,直待與你把房蓋得完,那時你孝服將滿,取你過門不遲。」李瓶兒道:「好好。你既有真心取奴,先早把奴房攛掇蓋了,取過奴去。到你家住一日,死也甘心。省的奴在這裡度日如年。」西門慶道:「你的話,我知道了。」李瓶兒道:「再不的,房子蓋完,我燒了靈,搬在五姐那邊樓上住兩日。等你蓋了新房子,搬移不遲。你好歹到家和五姐說,我還等你的話。這三月初十日,是他百日,我好念經燒靈。」西門慶應諾,與婦人歇了一夜,到次日,一五一十,對潘金蓮說了。金蓮道:「可知好哩!奴巴不的騰兩間房與他住,只怕別人,你還問聲大姐姐去。我落得河水不洗船,看大姐姐怎麼說。」這西門慶一直走到月娘房裡來,月娘正梳頭。西門慶把李瓶兒要嫁一節,從頭至尾聽說一遍。月娘道:「你不好取他的休。他頭一件,孝服不滿;第二件,你當初和他男子漢相交;第三件,你又和他老婆有連手,買了他房子,收着他寄放的許多東西。常言:『機兒不快,梭兒快。』我聞得人說,他家房族中花大,是個刁徒潑皮的人。倘或一時有些聲口,倒沒的惹虱子頭上撓。奴說的是好話,趙錢孫李,你依不依隨你。」幾句說的西門慶閉口無言。走出前廳來,自己坐在椅子上沉吟。又不好回李瓶兒話,又不好不去的。尋思了半日,還進入金蓮房裡來,金蓮問道:「你到大姐姐房裡,大姐姐怎麼說?」西門慶把月娘的話,告訴了一遍。金蓮道:「大姐不肯,論他也說的是。你又買了他房子,又取他老婆,當初又與他漢子相交了一世,方纔好。我又是一說,既做朋友,沒絲也有寸交,官兒也看喬了。」西門慶道:「這個也罷了。倒只怕花大那廝,沒圈子跳,知道,挾制他孝服不滿,在中間鬼混,怎生計較?我如今又不好回他的。」金蓮道:「呸!有甚難處事?我問你,今日回他去,明日回他去?」西門慶道:「他教我今日回他聲去。」金蓮道:「你今日到那裡,恁對他說。你說:『我到家對五姐說來,他的樓上堆着許多藥料,你這家火去,到那裡沒處堆放。亦發再寬待些時,你這邊房子七八也待蓋了,攛掇匠人,早些裝修油漆停當。你這邊孝服也將滿,那時取你過去,都不齊備些?強似搬在五姐樓上,葷不葷,素不素,擠在一處甚麼樣子?』管情他也罷了。」西門慶聽言大喜,那裡等的時分,走到李瓶兒家。婦人便問:「你到家所言之事如何?」西門慶道:「五姐說來,一發等收拾油漆你新房子,你搬去不遲。如今他那邊樓上,堆的破零三亂。你這些東西過去,那裡堆放?只有一件打攪,只怕你家大伯子,說你孝服不滿,如之奈何?」婦人道:「他不敢管我的事。休說各衣另飯,當官寫立分單,已倒斷開了的勾當。只我先嫁由爹娘,後嫁由自己。自古嫂兒不通問,大伯管不的我暗地裡事。我如今見過不的日子,他顧不的我。他若但放出個屁來,我教那賊花子坐着死,不敢睡着死。大官人你放心,他不敢惹我。」因問:「你這房子,也得幾時方收拾完備?」西門慶道:「我如今分付匠人,先替你蓋出這三間樓來,及到油漆了,也到五月頭上。」婦人道:「我的哥哥,你上緊些。奴情願等着到那時候也罷。」說畢,丫鬟擺上酒,兩個歡娛飲酒過夜,西門慶自此,沒三五日不來,俱不必細說。光陰迅速,西門慶家中已蓋了兩月房屋。三間玩花樓,裝修將完;只少捲棚還未安磉。一日,五月蕤賓佳節,家家門插艾葉,處處戶掛靈符。李瓶兒治了一席酒,請過西門慶來。一者解粽,二者商議過門之日。擇五月十五日,先請僧人念經燒靈,然後西門慶這邊擇取婦人過門。西門慶因問李瓶兒道:「你燒靈那日,花大、花三、花四請他不請?」婦人道:「我每人把個帖子,隨他來不來。」當下計議已定。單等五月十五日,婦人請了報恩寺十二眾僧人,在家念經除靈。西門慶那日封了三錢銀子人情,與應伯爵做生日。早辰拿了五兩銀子與玳安,教他買辦雞鴨置酒。晚夕李瓶兒除服,都教平安、畫童兩個跟馬,約午後時分,往應伯爵家來。那日在席前者,謝希大、祝日念、孫天化、吳典恩、雲離守、常時節,連新上會賁地傳,十個朋友一個不少。又叫了兩個小優兒彈唱。遞畢酒,上坐之時,西門慶叫過兩優兒,認的頭一個是吳銀兒兄弟,名喚吳惠。那一個不認的,跪下說道:「小的是鄭愛香兒的哥,叫鄭奉。」西門慶坐首席,每人賞二錢銀子。吃到日西時分,只見玳安拿馬來接。正上席來,向西門慶耳邊悄悄說道:「娘請爹早些去罷。」西門慶與了他個眼色,就往下走!被應伯爵叫住問道:「賊狗骨頭兒!你過來實說。若不實說,我把你小耳朵擰過一邊來。你應爹一年有幾個生日?恁日頭半天裡,就拿馬來接了你爹,往那裡去?端的誰使了你來?或者是你家中那娘使了你來?或是裡邊十八子那裡?你若不說過,一百年也不對你爹說替你這小狗禿兒娶老婆。」那玳安只是說道:「委的沒人使小的。小的恐怕夜緊,爹要起身,早拿馬來伺侯。」那應伯爵奈何了他一回,見不說,便道:「你不說,我明日打聽出來,和你這小油嘴兒算帳。」于是又斟了一鍾酒,拿了半碟點心,與玳安下邊吃去。良久,西門慶下來,東淨裡更衣。叫玳安道,到僻靜處問他話:「今日花家那有誰來?」玳安道:「花三往鄉裡去了。花四家裡瞎眼,都沒人來。只有花大家兩口子來,吃了一日齋飯,他漢子先家去了。只有他老婆臨去,二娘叫到房裡去了,與了他十兩銀子,兩套衣服,還與二娘磕了頭。」西門慶道:「他沒說甚麼?」玳安道:「他一字通沒敢題甚麼,只說到明日二娘過來,他三日要來爹家走走。」西門慶道:「他真個說此話來?」玳安道:「小的怎敢說謊!」這西門慶聽了,滿心歡喜。又問:「齋供了畢不曾?」玳安道:「和尚老早就去了,靈位也燒了。二娘說請爹早些過去。」西門慶道:「我知道了,你外邊看馬去。」這玳安正往外走,不想應伯爵在過道內聽,猛可叫了一聲,把玳安諕了一跳。伯爵罵道:「賊小狗骨頭兒!你不告我說,我就的也聽見了。原來你爹兒們幹的好繭兒!」西門慶道:「怪狗才!休要唱揚一地裡知道。」伯爵道:「你央及我央兒,我不說便了。」于是走到席上,如此這般,對眾人說了一回。把西門慶拉着說道:「哥,你可成個人?有這等事,就掛口不對兄弟們說聲兒。就是花大有些甚話說,哥只分付俺每一聲,等俺每和他說,不怕他不依。他若敢道個不是,俺每就與他結一個大胳膊。端的不知哥這親事成了不曾?哥一一告訴俺們。比來相交朋友做甚麼?哥若有使令俺們處,兄弟情厚,火裡火去,水裡水去;願不求同日生,只求各自死。弟兄每這等待你,哥你不說個道理,還只顧瞞着不說。」謝希大接過說道:「哥如若不說,俺每明日唱揚的裡邊李桂姐、吳銀兒那裡知道了,大家都不好意思的。」西門慶笑道:「我教眾位得知罷。親事已都停當了。」應伯爵問道:「取行禮過門,還未定日子?」謝希大道:「哥到明日取嫂子過門,俺每賀哥去。哥好歹叫上四個唱的,請俺每吃喜酒。」西門慶道:「這個不消說,一定奉請列位兄弟。」祝日念道:「比時明日與哥慶喜,不如咱如今替哥把一杯酒兒,先慶了喜罷。」于是叫伯爵把酒,謝希大執壺,祝日念捧茶,其餘都陪跪。把兩個小優兒也叫來,跪着彈唱一套十三腔喜遇吉日,一連把西門慶灌了三四鍾酒。祝日念道:「哥,那日請俺每吃酒,也不要少了鄭奉、吳惠他兩個。」因定下:「你二人好歹去。」鄭奉掩口道:「小的們已定早去宅裡伺侯。」須臾,遞畢酒,各歸席座下,又吃了一回。看看天晚,那西門慶那埋坐的住,趕眼錯起身走了。應伯爵還要攔門不放,謝希大道:「應二哥,你放哥去罷。休要誤了他的事,教嫂子見怪。」那西門慶得手上馬,一直走了。到了獅子街,李瓶兒摘去孝䯼髻,換了一身豔服。堂中燈燭熒煌,預備下一桌齊整酒餚。上面獨獨安一張交椅,讓西門慶上坐,方打開一壜酒篩來,丫鬟執壺,李瓶兒滿斟一杯遞上去,插燭也似磕了四個頭,說道:「今日拙夫靈已燒了。蒙大官人不棄,奴家得奉巾櫛之歡,以遂于飛之願。」行畢禮起來。西門慶下席來,亦回遞婦人一杯,方纔坐下。因問:「今日花大兩口子,沒說甚麼?」李瓶兒道:「奴午齋後,叫進他到房中,就說大官人這邊做親之事。他滿口說好,一句閑話也無。只說明日三日哩,教他娘子兒來咱家走走。奴與他十兩銀子,兩套衣服,兩口子喜歡的要不的。臨出門,謝了又謝。」西門慶道:「他既恁說,我容他上門走走也不差甚麼。但有一句閑話,我不饒他。」李瓶兒道:「他就放屁辣騷,奴也不放過他。」于是湯水嗄飯,老媽廚下一齊拏上。李瓶兒親自洗手剔甲,做了些葱花羊肉,一寸的匾食兒。銀鑲鍾兒盛着南酒 。秀春斟了兩盃,李瓶兒陪西門慶吃。西門慶止吃了上半甌,就把下半甌送與李瓶兒吃。一往一來,迭連吃上幾甌。真個是:

「年隨情少,  酒因境多。」

李瓶兒因過門日子近了,比常時益發喜歡得了不的。臉上堆下笑來,對西門慶道:「方纔你在應家吃酒,奴已候得久了。又恐怕你醉了,叫玳安來請你早些歸來,不知那邊可有人覺道麼?」西門慶道:「又被應花子猜着,逼勒小廝說了幾句,鬧混了一場。諸弟兄要與我賀喜,喚唱的,做東道。又齊攢的幫襯,灌上我幾盃。我趕眼錯就走出來,還要攔阻,又說好說歹,放了我來。」李瓶兒就道:「他每放了你,也還解趣哩。」西門慶看他醉態顛狂,情眸眷戀,一霎的不禁胡亂兩個口吐丁香,臉偎仙杏。李瓶兒把西門慶抱在懷裡叫道:「我的親哥!你既真心要娶我,可趁早些。你又往來不便,休丟我在這裡日夜懸望。」說畢,翻來倒去,攪做一團,真個是:

「傾國傾城漢武帝,  為雲為雨楚襄王。」

有詩為證:

「情濃胸緊湊,  欵洽臂輕籠;

賸把銀缸照,  猶疑是夢中;」

畢竟未知後來如何,且聽下回分解:
