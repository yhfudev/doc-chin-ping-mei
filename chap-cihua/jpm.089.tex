%# -*- coding: utf-8 -*-
%!TEX encoding = UTF-8 Unicode
%!TEX TS-program = xelatex
% vim:ts=4:sw=4
%
% 以上设定默认使用 XeLaTex 编译,并指定 Unicode 编码,供 TeXShop 自动识别

%第八十九回 
\chapter{清明節寡婦上新墳\KG 吳月娘誤入永福寺}

「風拂烟籠錦旆揚,  太平時節日初長,

多添壯士英雄胆,  善解佳人愁悶腸;

三尺繞垂楊柳岸,  一竿斜插杏花旁,

男兒未遂平生志,  且樂高哥入醉鄉。」

話說吳月娘次日備辦了一張祭卓,猪首、三牲、羹飯、冥紙之類,封了一疋尺頭,交大姐收拾一身縞素衣服,坐轎子,薛嫂兒押着祭禮先行,來到陳宅門首。只見陳經濟正在門首站立,那薛嫂把祭禮交人抬進去。經濟便問:「那里的?」薛嫂道了萬福,說:「姐夫,你休推不知。你丈母家來與你爹燒紙,送大姐來了。」經濟便道:「我{髟巳}{髟八}{入日}的纔是丈母。正月十六貼門神,遲了半月!人也入了土,纔來上祭。」薛嫂道:「好姐夫,你丈母說,寡婦人,沒腳蟹,不知你這里親家靈櫃來家,遲了一步,休怪!」正說着,只見大姐轎子落在門首,經濟問:「是誰?」薛嫂道:「再有誰?你丈母心內不好,一者送大姐來家,二者敬與你爹燒紙。」經濟罵道:「趁早把這淫婦抬回去,好的死了萬萬千千,我要他做甚?」薛嫂道:「常言道,嫁夫着主,你怎的說這個話?」經濟道:「我不要這淫婦了,還不與我走?」那抬轎的只是顧站立不動。被經濟向前踢了兩腳,罵道:「還不與我抬了去,我把花子腿砸了,把淫婦鬢毛都蒿淨了!」那抬轎子的見他踢起來,只得抬轎子往家中走不迭。比及薛嫂叫出他娘張氏來,轎子已抬的去了。薛嫂兒沒奈何,收下祭禮,走來回覆吳月娘。把吳月娘氣的一個發昏,說道:「恁個沒天理的短命囚根子!當初你為了官事,躲來丈人家居住,養活了這幾年,今日反恩將仇報起來了!恨起死鬼,當初攪下的好貨在家里,弄出事來,到今日交我做臭老鼠,交他這等放屁辣臊。」對着大姐說:「孩兒,你是眼見的,丈人丈母,那些兒虧了他來?你活是他家人,死是他家鬼,我家里也難以留你。你明日還去,休要怕他。料他挾不到你井里!他好胆子,恒是殺不了人,難道世間沒王法管他也怎的!」當晚不題。到之日,一頂轎子,交玳安兒跟隨着,把大姐又送到陳經濟家來。不想陳經濟不在家,往坟上替他父親添上叠山子去了。張氏知禮,把大姐留下,對着玳安說:「大官到家,多多上覆親家,多謝祭禮,休要和他一般兒見識!他昨日已有酒了,故此這般。等我慢慢說他。」一面管待玳安兒,安撫來家。至晚陳經濟坟上回來,看見了大姐,就行踢打,罵道:「淫婦你又來做甚麼?還是說我在你家雌飯吃!你家收着俺許多箱籠,因此起的這大產業,不道的白養了女婿!好的死了萬千,我要你這淫婦人?」這大姐亦罵:「沒廉耻的囚根子!沒天理的囚根子!淫婦出去吃人殺了,沒的禁拿我煞氣!」被經濟抹過頂髮,儘力打了幾拳頭。他娘走來解勸,把他娘推了一交。他娘叫罵哭喊說:「好囚根子,紅了眼,連我也不認的了!」到晚上,一頂轎子把大姐又送將來。分付道:「不討將寄放粧奩箱籠來家,我把你這淫婦活殺了!」這大姐害怕,躲在家中居住,再不敢去了。有詩為證:

「相識當初信有疑,  心情還似永無涯;

誰知好事多更變,  一念翻成怨恨媒。」

這里西門大姐在家躲住,不敢去了。一日,三月清明佳節,吳月娘備辦香燭、金錢、冥布、三牲祭物、酒肴之類,抬了兩大食盒,要往城外五里新坟上,與西門慶上新坟祭掃。留下孫雪娥和着大姐、眾丫頭看家。帶了孟玉樓和小玉,并奶子如意兒,抱着孝哥兒,都坐轎子,往坟上去。又請了吳大舅和大妗子老公母二人同去。出了城門,只見那郊原野曠,景身芳菲,花紅柳綠,仕女遊人不斷頭的走的。一年四季,無過春天,最好景致。日謂之麗日,風謂之和風,吹柳眼,綻花心,拂香塵。天色暖謂之暄,天色寒謂之料峭。騎的馬謂之寶馬,坐的轎謂之香車,行的路謂之香徑。地下飛的土來謂之香塵。千花發蕊,萬草生芽,謂之春信。韶光淡蕩,淑景融和。小桃深粧臉妖嬈,嫩柳嬝宮腰細膩。百囀黃鸝,驚回午夢;數聲紫燕,說破春愁。日舒長煖藻鵝黃,水渺茫浮香鴨綠。隔水不知誰院落,鞦韆高掛綠楊烟。端的春景,果然是好!到的春來,那府州縣道,與各處村鎮鄉市,都有遊玩去處。有詩為證:

清明何處不生烟,  郊外微風掛紙錢,

人笑人歌芳草地,  乍晴乍雨杏花天;

海棠枝上綿鶯語,  楊柳堤邊醉客眠,

紅粉佳人爭盡技,  綵繩搖洩學飛仙。」

都說吳月娘等轎子到五里原坟上,玳安押着食盒,又早先到廚下,生起火來。廚役落作整理不題。月娘與玉樓、小玉、奶子如意抱着孝哥兒,到於庄院客坐內,坐下吃茶。等着吳大妗子,不見到。玳安向西門慶坟上祭臺上,擺設卓面三牲,羹飯祭物,列下布錢。只等吳大妗子,因顧不出轎子來,約巳牌時分,纔同吳大舅顧了兩個驢兒騎將來。月娘便說:「大妗子顧不出轎子來,果然沒有轎子。」一面吃了茶,換了衣服,走來西門慶坟前祭掃。那月娘手拈着五根香、一根香。他拿在手內,一根香遞與玉樓,一根遞與奶子如意兒,抱着孝哥兒,那兩根遞與吳大舅、大妗子。月娘插在香爐內,深深拜下去,說道:「我的哥哥,你活時為人,死後為神,今日三月清明佳節,你的孝妻吳氏三姐、孟三姐,同你周歲孩童孝哥兒,敬來與你坟前燒一百錢布。你保佑他長命百歲,替你做坟前拜掃之人。我的哥哥,我和你做夫婦一場,想起你那模樣兒,并說的話來,是好傷感人也!」玳安把布錢點着,有哭山坡羊為證:

「燒罷布,小腳兒連跺;奴與你做夫婦一場,並沒個言差語錯!實指望同諧到老,誰知你半路將奴拋卻。當初人情看望,全然是我;今丟下銅斗兒家緣,孩兒又小,撇的俺子母孤孀,怎生遣過!恰便似中途遇雨,半路裡遭風來呵!折散了鴛鴦,生揪斷異果!叫了聲好性兒的哥哥,想起你那動影行藏,可不嗟嘆我!」

〔帶步步嬌〕「燒的布灰兒團團轉,不見我兒夫面。哭了聲年少夫,撇下嬌兒,閃的奴孤單!咱兩無緣,怎得和你重相見!」

玉樓向前插上香,深深拜上,哭唱前腔:

「燒罷紙,滿眼淚墮。叫了聲,人也天也,丟個奴無有個下落!實承望和你白頭廝守,誰知道,半路花殘月沒!大姐姐有兒童,他房里還好。閃的奴樹倒無陰,跟著誰過?獨守孤幃,怎生奈何!恰便似前不著店,後不著村里來呵!那是我葉落歸根,收園結果。叫了聲,年小的哥哥,要見你,只非夢兒里相逢,卻不想念殺了我!」

〔帶步步嬌〕哭來哭去,哭的奴痴呆了!你一去了無消耗,思量好無下稍,無下稍!你正青春,奴又多嬌,好心焦,清減了花容月貌!」

玉樓上了香,奶子如意抱着哥兒,也跪下上香,磕了頭。吳大舅、大妗子都炷了香,行畢禮數,同讓到庄上捲棚內,放卓席擺飯,收拾飲酒。月娘讓吳大舅、大妗子上坐,月娘與玉樓打橫。小玉和奶子如意兒,同大妗子家使的老姐蘭花,那兩邊打橫列坐,把酒來斟。按下這里吃酒不題。都表那日周守備府里也上坟。先是春梅隔夜和守備睡,假推做夢,睡夢中哭醒了。守備慌的問:「你怎的哭?」春梅便說:「我夢見我娘向我哭泣,說養我一場,怎地不與他清明寒食燒布兒?因此哭醒了。」守備道:「這個也是養女一場,你的一點孝心。不知你娘坟在何處?」春梅道:「在南門外,永福寺後面便是。」守備說:「不打緊,永福寺是我家香火院,明日咱家上坟,你教伴當抬些祭物,往那里與你娘燒分布錢,也是好處。」至此日,守備令家人收拾食盒酒果祭品,逕往城南祖坟上,那里有大庄院、廳堂、花園去處,那里有享堂、祭臺。大奶奶、孫二娘并春梅,都坐四人轎,排軍喝路,上坟耍子去了。都說吳月娘和大舅、大妗子吃了回酒,恐怕晚來,分付玳安、來安兒,收拾了食盒酒菓,先往那十里長隄杏花村酒樓下,揀高阜去處,人烟熱鬧那里,設放卓席等候。又見大妗子沒轎子,都把轎子抬着,後面跟隨不坐。領定一簇男女,吳大舅牽着驢兒壓後同行,踏青遊玩。三里抹過桃花店,五里望見杏花村,只見那隨路上坟遊玩的王孫士女,花紅柳綠,鬧鬧喧喧,不斷頭的走。偏襯着日煖風和,尋芳問景,不知又多少。正走之間,也是合當有事,遠遠望見綠槐影里,一座菴院,蓋造得十分齊整。但見:

「山門高聳,梵宇清幽。當頭敕額字分明,兩下金剛形勢猛。五間大殿,龍鱗瓦砌碧成

行;兩廊僧房,龜背磨磚花嵌縫。前殿塑風調雨順,後殿供過去未來。鐘鼓樓森立,藏經閣巍峨,旛竿高峻接青雲,寶塔依稀侵碧漢。木魚橫掛,雲板高懸。佛前燈燭熒煌,爐內香烟繚繞。幢幡不斷,觀音殿接祖師堂。寶蓋相連,鬼母位通羅漢院。時時護法諸天降,歲歲降魔尊者來。」

吳月娘便問:「這座寺叫着甚麼寺?」吳大舅便說:「此是周秀老爺香火院,名喚永福禪林。前日姐夫在日,曾捨幾十兩銀子在寺中,重修佛殿,方是這般新鮮。」月娘向大妗子說:「咱也到這寺中看一看。」於是領着一簇男女,進入寺中來。不一時,小沙彌看見,報於長老知道,見有許多男女,便出方丈來,迎請施主菩薩隨喜。但見這長老,怎生模樣:

「一個青旋旋光頭新剃,把麝香松勻搽。黃烘烘直裰初縫,使沉速箋檀濃染。山根鞋履,是福州染到深青;九縷絲縧,係西地買來真紫。那和尚光溜溜一雙賊眼,單睃趂施主嬌娘;這秃厮美甘甘滿口甜言,專說誘喪家少婦。淫情動處,草菴中去覓尼姑;色胆發時,方丈內來尋行者。仰觀神女思同寢,每見嫦娥要講歡。」

這長老見吳大舅、吳月娘,向前合掌道了問訊,連忙喚小和尚開了佛殿,請施主菩薩隨喜遊玩,小僧看茶。那小沙彌開了殿門,領月娘一簇男女,前後兩廊參拜。觀看子一回,然後到長老方丈。長老連忙點上茶來,雪錠般盞兒,甜水好茶。吳大舅請問長老道號。那和尚笑嘻嘻說:「小僧法名道堅,這寺是恩主帥府周爺香火院。小僧忝在本寺長老,廊下管百上眾僧。後邊禪堂中,還有許多雲遊僧,行常串座禪,與西方檀越,答報功德。一面方丈中擺齋,讓月娘:「眾菩薩請坐,小僧一茶而已。」月娘道:「不當打攪長老寶剎。」一面拿出五兩銀子,交大舅遞與長老,「佛前請燒香。」那和尚笑吟吟打問訊謝了,說道:「小僧無甚管待施主菩薩,少坐略備一茶而已,何勞費心賜與希施?」不一時,小和尚放了卓兒,拿上素菜齋食、餅饊上來,那和尚在旁陪坐。舉筯兒,纔待讓月娘眾人吃時,忽見兩個青衣漢子走的氣喘吁吁,暴雷也一般,報與長老說道:「長老還不快出來迎接,府中小奶奶來祭祀來了。」慌的長老披袈裟,戴僧帽不迭。分付小沙彌,連忙收了家活:「請列位菩薩且在小房避避,打發小夫人燒了紙,祭畢去了,再款坐一坐不遲。」吳大舅告辭,和尚死活留住,又不肯放。那和尚慌的鳴起鐘鼓來,出山門迎接,遠遠在馬路口上等候。只見一簇青衣人,圍着一乘大轎,從東雲飛般來。轎夫走的個個汗流滿面,衣衫皆濕。那長老躬身合掌說道:「小僧不知小奶奶前來,理合遠接;接待遲了,勿蒙見罪。」這春梅在簾內答道:「起動長老!」那手下伴當,又早向寺後金蓮坟上,抬將祭卓來,擺設已久,紙錢列下,春梅轎子來到,也不到寺,逕入寺白楊樹下金蓮坟前下了轎子。兩邊青衣人伺候。這春梅不慌不忙,來到坟前插了香,拜了四拜,說道:「我的娘,今日龐大姐特來與你燒陌紙錢。你好處生天。苦處用錢!早知你死在仇人之手,奴隨問怎的,也娶來府中,和奴做一處。還是奴躭誤了你,悔已是遲了!」說畢,令左右把紙錢燒了。這春梅向前放聲大哭,有哭山坡羊為證:

「燒罷紙,把鳳頭鞋跌綻。叫了聲娘,把我肝腸兒叫斷!自因你逞風流,人多惱你,疾發你出去,被仇人纔把你命兒坑陷!奴在深宅,怎得個自然?又無親,誰把你掛牽?實指望你同床兒共枕,怎知道你命短無常,死的好可憐!叫了聲不睜眼的青天,常言道:好物道全,紅羅尺短!」

這里春梅在金蓮坟上祭祀哭泣不題。都說吳月娘在僧房內只知有宅內小夫人來到,長老出去山門迎接,又不見進來。問小和尚,和尚說:「這寺後有小奶奶的一個姐姐,新近葬下,今日清明節,特來祭掃燒紙。」孟玉樓便道:「怕不就是春梅來了?也不止的。」月娘道:「他又那得個姐來,死了葬在此處?」又問小和尚:「這府里小夫人姓甚麼?」小和尚道:「姓龐氏,前日與了長老四五兩經錢教替他姐姐念經,薦拔生天。」玉樓道:「我聽見爹說,春梅娘家姓龐,叫龐大姐,莫不是他?」正說話,只見長老先生走來,分付小沙彌,快看好茶。不一時,轎子抬進方丈二門里,纔下轎。月娘和玉樓眾人,打僧房簾內,望外張看怎樣的小夫人?守睛仔細看時,都是春梅。但比昔時出落長大身材,面如滿月,打扮的淡粧玉琢。頭上戴着冠兒,珠翠堆滿,鳳釵半卸,穿大紅粧花襖兒,下着翠藍縷金寬襴裙子,帶着玎璫禁步,比昔不同許多!但見:

「寶髻巍峨,鳳釵半卸。胡珠環耳邊低掛,金挑鳳鬢後雙插。紅綉姖襖偏襯玉香肌,翠紋裙下映金蓮小。行動處,胸前搖響玉玎璫;坐下時,一陣麝蘭香噴鼻。膩粉粧成脖頸,花鈿巧貼眉尖。舉止驚人,貌比幽花殊麗;姿容閒雅,性如蘭蕙溫柔。若非綺閣生成,定是蘭房長就。儼若紫府瓊姬離碧漢,蕊宮仙子下塵寰。」

那長老一面掀簾子,請小夫人方丈明間內,上面獨獨安放一張公座椅兒。春梅坐下,長老參見已畢,小沙彌拿上茶。長老遞茶上去,說道:「今日小僧不知宅內上玟,小奶奶來這里祭祀,有失迎接,恕罪小僧!」春梅道:「外日多有起動長老誦經追薦!」那和尚沒口子說:「小僧豈敢!有甚殷勤補報恩主?多蒙小奶奶賜了許多經錢襯施,小僧請了八眾禪僧,整做道場,看經禮懺一日。晚夕又多與他老人家,裝些廂庫焚化。道場圓滿,纔打發三位管家進城,宅里回小奶奶話。」春梅吃了茶,小和尚接下鐘盞來。長老只顧在旁,一遞一句與春梅說話,把吳月娘眾人攔阻在內,又不好出來的。月娘恐怕天晚,使小和尚請下長老來要起身。那長老又不肯放,走來方丈稟春梅說:「小僧有件事,稟知小奶奶。」春梅道:「長老有話,但說無妨。」長老道:「適間有幾位遊玩娘子,在寺中隨喜,不知小奶奶來。如今他要回去,未知小奶奶尊意如何?」春梅道:「長老何不請來相見?」那長老慌的來請,吳月娘又不肯出來。只說:「長老不見罷,天色晚了,俺每告辭去罷。」長老見收了他布施,又沒管待。又意不過,只顧再三催促。吳月娘與孟玉樓、吳大妗子推阻不過,只得出來,春梅一見,便道:「原來是二位娘與大妗子。」於是先讓大妗子轉上,花枝招颭,磕下頭去。慌的大妗子還禮不迭,說道:「姐姐今非昔日比,折殺老身!」春梅道:「好大妗子,如何說這話?奴不是那樣人,尊卑上下,自然之理。」拜了大妗子,然後向月娘、孟玉樓,插燭也似磕頭去。月娘、玉樓,亦欲還禮。春梅那里肯,扶起,磕了四個頭,說:「不知娘們在這里,早知也請出來見。」月娘道:「姐姐,你自從出了家門,在府中,一向奴多缺禮,沒曾看你,你休怪!」春梅道:「好奶奶,奴那里出身,豈敢說怪?」因見奶子如意兒抱着孝哥兒,說道:「哥哥也長的恁大了!」月娘說:「你和小玉過來,與姐姐磕個頭兒。」那如意兒和小玉二人,笑嘻嘻過來,亦與春梅都半磕了頭。月娘道:「姐姐,你受他兩個一禮兒。」春梅向頭上拔下一對金頭銀簪兒來,插在孝哥兒帽上,月娘說:「多謝姐姐簪兒,還不與姐姐唱個喏兒?」如意兒抱着哥兒,真個與春梅道了唱個喏,把月娘喜歡的要不得。玉樓說:「姐姐,你今日不到寺中,咱娘兒們怎得遇在一處相見?」春梅道:「便是因俺娘他老人家,新埋葬在這寺後,奴在他手裡一場,他又無親無故,奴不記掛着替他燒張紙兒,怎生過得去?」月娘說:「我記的你娘沒了好幾年,不知葬在這里。」孟玉樓道:「大娘,還不知龐大姐說話?說的潘六姐死了,多虧姐姐,如今把他埋在這里。」月娘聽了,就不言語了。吳大妗子道:「誰似姐姐這等有恩!不肯忘舊,還葬埋了。你逢節令,題念他來,替他燒錢化紙。」春梅道:「好奶奶,想着他怎生抬舉我來!今日他死的苦,是這般拋露丟下,怎不埋葬他!」說畢,長老教小和尚放卓兒,擺齋上來,兩張大八仙卓子,蒸酥煠餅饊點心,各樣素饌菜蔬,堆滿春臺,絕細金芽雀舌甜水好茶,眾人吃了,收下家活去。吳大舅自有僧房管待,不在話下。孟玉樓起身,心里要往金蓮坟上看看,替他燒張布,也是姊妹一場。見月娘不動身,拿出五分銀子,教小沙彌買布去。長老道:「娘子不消去買,我這里有金銀紙,拿幾分燒去。」玉樓把銀子遞與長老,使小沙彌領到後邊白楊樹下金蓮坟上。見三尺坟堆,一堆黃土,數柳青蒿,上了根香,把紙錢點着,拜了一拜,說道:「六姐,不知你在這里,今日孟三姐誤到寺中,與你燒陌錢布!你好處生天,苦處用錢!」一面取出汗巾兒來,放聲大哭。有哭山坡羊為證:

「燒罷布,淚珠兒亂滴。叫六姐一聲,哭的奴一絲兒雨氣!想當初,咱二人不分個彼此。做姊妹一場,並無面紅耳赤。你性兒強,我常常兒讓你,一面兒不見,不是你尋我,我就尋你。恰便像比目魚,雙雙熱粘在一處,忽被一陣風,咱分開來呵!共樹同栖,一旦各自去飛!叫了聲六姐,你試聽知,可惜你一段兒聰明,今日埋在土裡!」

那奶子如意兒見玉樓往後邊,也抱了孝哥兒來看一看。月娘在方丈內和春梅說話,教奶子:「休抱了孩子去,只怕諕了他。」如意兒道:「奶奶不妨事,我知道。」徑抱到坟上,看玉樓燒布哭罷回來。春梅和月娘勻了臉,換了衣裳。分付小伴當將食盒打開,將各樣細菓甜食餚品點心攢盒,擺下兩卓子,布甑內篩上酒來,銀鐘牙筯,請大妗子、月娘、玉樓上坐,他便主位相陪。奶子、小玉、老姐,兩邊打橫。吳大舅另放一張卓子在僧房內。正飲酒中間,忽見兩個青衣伴當,走來跪下稟道:「老爺在新庄,差小的來請小奶奶看雜耍調百戲的。大奶奶、二奶奶都去了。請奶奶快去哩。」這春梅不慌不忙,說:「你回去,知道了。」那二人應諾下來,又不敢去,在下邊等候,且待他陪大妗子、月娘,便要起身,說:「姐姐,不可打攪。天色晚了,你也有事,俺每去罷。」那春梅那里肯放,只顧令左右將大鐘來勸道:「咱娘兒們會少離多,彼此都見長着,休要斷了這們親路。奴也沒親沒故,到明日娘好的日子,奴往家里走走去。」月娘道:「我的姐姐,說一聲兒就勾了,怎敢起動你!容一日,奴去看姐姐去。」飲過一杯,月娘說:「我酒勾子,你大妗子沒轎子,十分晚了,不好行的。」春梅道:「大妗子沒轎子,我這里有跟隨小馬兒,撥一疋與妗子騎,送了家去。」一面收拾起身。春梅叫道那長老來,令小伴當拿出一疋大布、五錢銀子與長老。長老拜謝了。送出山門。春梅與月娘拜別,看着月娘、玉樓眾人上了轎子,他也坐轎子,兩下分路,一簇人跟隨,喝着道往新庄上去了。正是:

「樹葉還有相逢處,  豈可人無得運時!」

畢竟未知後來如何,且聽下回分解:

