%# -*- coding: utf-8 -*-
%!TEX encoding = UTF-8 Unicode
%!TEX TS-program = xelatex
% vim:ts=4:sw=4
%
% 以上设定默认使用 XeLaTex 编译,并指定 Unicode 编码,供 TeXShop 自动识别

%第九十四回 
\chapter{劉二醉毆陳經濟\KG 酒家店雪娥為娼}


\begin{showcontents}{}



「花開不擇貧家地,  月照山河到處明,

世間只有人心歹,  萬事還教天養人;

癡聾瘖瘂家豪富,  伶俐聰明卻受貧,

年月日時該載定,  筭來由命不由人。」

話說陳經濟自從陳三兒引到謝家大酒樓上見了馮金寶,兩箇又抅搭上前情,往後沒三日不和他相會。或一日經濟廟中有事不去,金寶就使陳三兒稍寄物事,或寫情書來叫他去。一次或五錢,或一兩。以後日間供其柴米,納其房錢,歸到廟中,便臉紅。任道士問他何處吃酒來?經濟只說:「在米舖和夥計暢飲三盃解辛苦來。」他師兄金宗明又替他遮掩,晚夕和他一處盤弄那勾當,是不必說。朝來暮往,把任道士囊篋中細軟的本也抵盜出大半,花費了不知覺。一日,也是合當有事。這酒家店的劉二,有名坐地虎。他是帥府周守備府中親隨張勝的小舅子。專一在馬頭上開娼店,倚強凌弱,舉放私債,與巢窩中各娼使錢,加三討利。有一不給,搗換文書,將利作本,利上加利。嗜酒行兇,人不敢惹他。就是打粉頭的班頭,欺酒客的領袖。因見陳經濟是晏廟任道士的徒弟,白臉小廝,在謝三家大酒樓上,把粉頭鄭金寶兒包占住了。吃的楞楞睜睜,提著碗來大小拳頭,走來謝家樓下,問:「金寶在那裡?」慌的謝三郎連忙聲諾說道:「劉二叔,他在樓上,第二箇閣兒裡便是。」這劉二大扠步上樓來。經濟正與金寶在閣兒裡面,兩箇飲酒,做一處快活。只把房門關閉,外邊簾子大掛著。被劉二一把手扯下簾子,大叫:「金寶兒出來!」諕的陳經濟鼻口內氣兒也不敢出,這劉二用腳把門跺開,金寶兒只得出來相見,說:「劉二叔叔,有何說話?」劉二罵道:「淫婦,你少我三箇月房錢,卻躲在這裡就不去了!」金寶笑嘻嘻說道:「二叔叔你家去,我使媽媽就送房錢來。」被劉二只摟心一拳,打了老婆一交,把頭顱搶在階沿下磕破,血流滿地。罵道:「賊淫婦,還等甚送來,我如今就要!」看見陳經濟在裡面,走向前把桌子只一掀,碟兒打得粉碎。那經濟便道:「阿呀!你是甚麼人?走來撒野!」劉二罵道:「我{入日}你道士秫秫娘!」手採過頭髮來,按在地下,拳踵腳踢無數。那樓上吃酒的人看著,都立睜了。店主人謝三郎初時見劉二醉了,不敢惹他。次後見打得人不像模樣,上樓來解勸說道:「劉二叔,你老人家息怒。他不曉得你老人家大名,誤言沖撞,休要和他一般見識。看小人人薄面,饒他去罷!」這劉二那裡依從,儘力把經濟打了發昏章第十一。叫將地方保甲,一條繩子,連粉頭都拴在一處墩鎖。分付:「天明早解到老爺府裡去!」原來守備勑書上,命他保障地方,巡捕盜賊,兼管河道。這裡拏了經濟,任道士廟中,還尚不知;只說他晚夕米鋪中上宿未回。卻說次日,地方保甲,巡河快手,押解經濟、金寶,顧頭回騎上,趕清晨,早到府前伺候,先遞手本與兩箇管事張勝、李安看看,說是劉二叔地方喧鬧一起,晏公廟道士一名陳經濟,娼婦鄭金寶。眾軍牢都問他要錢,說道:「俺們是廳上動刑的,一班十二人隨你罷;正景兩位管事的,你倒不可輕視了他!」經濟道:「身邊銀錢倒,有都被夜晚劉二打我時,被人掏摸的去了!身上衣服都扯碎了,那得錢來?止有頭上關頂一根銀簪兒,拔下來與二位管事的罷?」眾牢子拏著那根簪子,走來對張勝、李安如此這般:「他一個錢兒不拏出來,止與了這根簪兒,還是鬧銀的。」張勝道:「你叫他近,前等我審問他。」眾軍牢不一時,推擁他到根前跪下,問:「你是任道士第幾箇徒弟?」經濟道:「第三箇徒弟。」又問:「你今年多大年紀?」經濟道:「廿四歲了。」張勝道:「你這等年少,只該在廟中做道士,習學經典。許你在外宿娼飲酒喧嚷?你把俺老爺帥府衙門,當甚麼些小衙門,不拏了錢兒來?這根簪子打水不渾,要他做甚!」還掠與他去。分付牢子:「等住回老爺升廳,把他放在頭一起,眼看這狗男女道士,就是箇佞錢的!只許你白要四方施主錢糧?休說你為官事,你就來吃酒赴席,也帶方汗巾兒揩嘴!等動刑時,著實加力拶打這廝!」又把鄭金寶叫上去。鄭家王八跟著,上下打發了三四兩銀子。張勝說:「你係娼門,不過趁熟。不趁些衣飯為生,沒甚大事。看老爺喜怒不同;看惱,只是一兩拶子;若喜歡,只恁放出來也不止。」旁邊那箇牢子說:「你再把與我一錢銀子,等若拶你,待我饒你兩箇大指頭!」李安分付:「你帶他遠些伺候,老爺將次出廳。」不一時,只見裡面雲板響,守備升廳,兩邊療掾軍牢森列,甚是齊整!但見:

「緋羅繳壁,紫綬卓圍。當廳額掛茜羅,四下簾垂翡翠。勘官守正,戒石上刻卸製四行;人從謹廉,鹿角旁插令旗兩面。軍牢沈重,僚掾威儀。執大棍授事立階前,挾文書廳旁廳發放。雖然一路帥臣。果是滿堂神道!」

當時沒巧不成話。也是五日劫冤家聚會,姻緣合當湊著。春梅在府中,從去歲八月間,已生了箇哥兒小衙內,今方半歲光景。貌如冠玉,唇若塗硃。守備喜似席上之珍,過如無價之寶。未幾大奶奶下世,守備就把春梅冊正做了夫人,就住著五間正房。買了兩箇養娘抱奶哥兒,一名玉堂,一名金匱;兩個小丫鬟伏侍,一箇名喚翠花,一個名喚蘭花。又有兩箇身邊得寵彈唱的姐兒,都十六七歲,一名海棠,一名月桂,都在春梅房中侍奉。那孫二娘房中,止使著一箇丫鬟,名喚荷花兒,不在話下。比的小衙內,只要張勝懷中抱他外邊頑耍。遇著守備升廳,在旁邊觀看。當日守備升廳坐下,放了告牌出去,各地方解進人來。頭一起正叫上陳經濟,并娼婦鄭金寶兒去。守備看了呈狀,又見經濟面上帶傷,說道:「你這廝是箇道士,不守那清規,如何宿娼飲酒,騷擾我地方?行止有虧!左右拏下去打二十棍,追了度牒還俗。那娼婦鄭氏,拶一拶,敲五十敲,責令歸院當差。」兩邊軍牢向前,纔待扯翻經濟,攤去衣服,用繩索綁起,轉起棍來,兩邊招呼打時,可要作怪,張勝抱著小衙內,正在廳前月台上站立觀看。那小衙內看見走過來打經濟,在懷裡攔不住,撲著要經濟抱。張勝恐怕守備看見,走過來,亦發大器起來,直哭到後邊春梅根前。春梅問:「他怎的哭?」張勝便說:「老爺廳上發放事,打那晏公廟道士,姓陳,他就撲著他抱。小的走下來,他就哭了。」這春梅聽見是姓陳的,不免輕移蓮步,款蹙湘裙,走到軟屏後面,探頭觀覷。廳下打的那人,聲音模樣,倒好似陳姐夫一般。「他因何出家做了道士?」又叫過張勝問他:「此人姓甚名誰?」張勝道:「這道士共狀上年廿四歲,俗名叫陳經濟。」春梅暗道:「正是他了!」一面使張勝:「請下你老爺來。」這守備廳上打經濟,纔打到十棍,一邊還拶著娼的。忽聽後邊夫人有請,分付牢子,把棍且閣住休打。一面走下廳來,春梅說道:「你打的那道士是我姑表兄弟,看奴面上,饒了他罷!」守備道:「夫人不早說,我已打了他十棍,怎生奈何?」一面出來分付牢子:「都與我放了。」娼的便歸院去了。守備悄悄使張勝:「叫那道士回來。且休去,問了你奶奶,請他相見。」這春梅纔待使張勝請他到後堂相見,忽然想起一件事來,口中不言,心內暗道:「剜去眼前瘡,安上心頭肉;眼前瘡不去,心頭肉如何安得上?」于是分付張勝:「你且叫那人去著,等我慢慢再叫他。」度牒也不曾追,這陳經濟打了十棍,出離了守備府,還奔來晏公廟。不想任道士聽見人來說:「你那徒弟陳宗美在大酒樓上包著娼的鄭金寶兒,惹了酒家店坐地虎劉二,打得臭死,連老婆都拴了解到守備府裡去了。行止有虧,便差軍牢來拏你去審問,追度牒還官!」這任道士聽了,一者年老的著了驚怕,二來身體胖大,因打開囊篋內,又沒了細軟東西,著了口重氣,心中痰疾湧上來,昏倒在地。眾徒弟慌忙向前扶救,請將醫者來灌下藥去,通不省人事。到半夜,嗚呼斷氣身亡!亡年六十三歲。第二日陳經濟來到,左邊鄰人說:「你還敢廟裡去?你師父因為你,如此這般得了口重氣,昨夜三更鼓死了!」這陳經濟聽了,諕的忙忙似喪家之犬,急急如漏網之魚!復回清河縣城中來。正是:

「鹿隨鄭相應難辨,  蝶化莊周未可知!」

話分兩頭,卻把春梅一見經濟,方待留他,忽然心上想起一件事來,還使出張勝來,教經濟且去罷。走歸房中,摘了冠兒,脫了綉服,倒在床上,一面捫心撾被,聲疼叫喚起來。諕的合宅大小都慌了。下房孫二娘來問道:「大奶奶行好好的,怎的來就不好起來?」春梅說:「你每且去,休管我。」落後守備退廳進來,見他倘在床上叫一番也慌了。扯着他手兒問道:「你心裡怎的來?」也不言語。又問:「那箇惹著你來?」也不做聲。守備道:「不剛纔兒我打了你兄弟,你心內惱麼?」亦不應答。這守備無計奈何,自出外邊,麻犯起張勝、李安來了:「你那兩箇,早知他是你奶奶兄弟,如何不早對我說?卻教我打了他十下,惹的你奶奶心中不自在起來!我曾教你留下他,請你奶奶相見。你如何又放他去了?你這廝每卻討分曉!」張勝說:「小的曾稟過奶奶來,奶奶說且教他去看。小的纔放他去了。」一面走入房中,哭啼哀告春梅:「望乞奶奶在爺前方便一言,不然,爺要見責小的每哩!」這春梅睜圓星眼,剔起蛾眉,叫過守備進前說:「我自心中不好,干他們甚事?那廝他不守本分,在外邊做道士,且崇他些時,等我慢慢招認他!」這守備纔不麻犯張勝、李安了。守備見他只歷聲喚,又使張勝請下醫官來看脉說:「老夫人染了六慾七情之病,著了重氣在心。」討將藥來,又不吃,都放冷了。丫頭每都不敢向前說話。請將守備來看著吃藥,只呷了一口,就不吃了。守備出去了,大丫鬟月桂拏過藥來:「請奶奶吃藥。」被春梅拏過來匹臉只一潑,罵道:「賊浪奴才,你只顧拏這苦水來灌我怎的!我肚子裡有甚麼!」教他跪在面前。孫二娘走來問道:「月桂怎的?奶奶教他跪著。」海棠道:「奶奶因他拏藥與奶奶吃來!奶奶說我肚子裡有甚麼?拏這來灌我?教他跪著。」孫二娘道:「奶奶你委的今一日沒曾吃甚麼,這月桂他不曉得。奶奶休打他,看我面上,饒他這遭罷!」分付海棠:「你往廚下熬些粥兒來,與你奶奶吃口兒。」春梅于是把月桂放起來。那海棠走到廚下,用心用意熬了一小鍋粳小米濃濃的粥兒,定了四碟小菜兒,用甌兒盛著,象牙快兒,熱烘烘拏到房中。春梅倘在床上,面朝裡睡,又不敢叫。直待他番身,方纔請他:「有箇粥兒在此,請奶奶吃粥。」春梅把眼合著,不言語。海棠又叫道:「粥【日亮】泠了,請奶奶起來吃粥。」孫二娘在旁說道:「大奶奶,你這半日沒吃甚麼。這會你覺好些,且起來吃些箇,有柱戧些!」那春梅一〈石古〉碌子扒起來,教奶子拏過燈來,取粥在手,只呷了一口,往地下只一推,早是不曾把家伙打碎,被奶子按住了。就大吆喝起來,向孫二娘說:「你平白叫我起來吃粥,你看賊奴才熬的好粥!我又不坐月子,熬這照面湯來與我吃怎麼?」分付奶子金匱:「你與我把這奴才臉上,把與他四箇嘴巴!」當下真箇把海棠打了四箇嘴巴。孫二娘便道:「奶奶你不吃粥,卻吃些甚麼兒?卻不餓著你?」春梅道:「你教我吃,我心內攔著吃不下去。」良久,叫過小丫鬟蘭花兒來,分付道:「我心內想些雞尖湯兒吃。你去廚房內,對著淫婦奴才,教他洗手做碗好雞尖湯兒與我吃口兒。教他多有著些酸笋,做的酸酸辣辣的我吃。」孫二娘便說:「奶奶分付他,教雪娥做去。你心下想吃的,就是藥。」這蘭花不敢怠慢,走到廚下對雪娥說:「奶奶教你做雞尖湯,快些做,等著要吃哩!」原來這雞尖湯,是雛雞脯翅的尖兒,碎切的做成湯。這雪娥一面洗手剔甲,旋宰了兩隻小雞,退刷乾淨,剔選翅尖,用快刀碎切成絲,加上椒料 蔥花芫荽酸笋油醬之顏,揭成清湯。盛了兩甌兒,用紅漆盤兒,熱騰騰蘭花拏到房中。春梅燈下看了,呷了一口,怪叫大罵起來:「你對那淫婦奴才說去,做的甚麼湯!精水寡淡,有些甚味?你們只教我吃,平白教我惹氣!」慌的蘭花生怕打,連忙走到廚下,對雪娥說:「奶奶嫌湯淡,好不罵哩!」這雪娥一聲兒不言語,忍氣吞聲,從新坐鍋,又做了一碗。多加了些椒料,香噴噴教蘭花拏到房裡來。春梅又嫌忒鹹了,拏起來照地下只一潑,早是蘭花躲得快,險些兒潑了一身。罵道:「你對那奴才說去,他不憤氣做與我吃!這遭做的不好,教他討分曉哩!」這雪娥聽見,千不合萬不合,悄悄說了一句:「姐姐幾時這般大了,就抖摟起人來!」不想蘭花回到房裡,告春梅說了。這春梅不聽便罷,聽了此言,登時柳眉剔豎,星眼圓睜,咬碎銀牙,通紅了紛面,大叫:「與我採將那淫婦奴才來!」須臾使了養娘丫鬟三四箇,登時把雪娥拉到房中。春梅氣狠狠的,一手扯住他頭髮,把頭上冠子跺了,罵道:「淫婦奴才!你怎的說幾時這般大?不是你西門慶家抬舉的我這般大!我買將你來伏侍我,你不憤氣!教你做口子湯,不是精淡,就是苦丁子鹹!你倒還對著丫頭說我幾時恁般大起來?抖摟索落!我要你何用?」一面請將守備來:「採雪娥出去,當天井跪著!前邊叫將張勝、李安,旋剝褪去衣裳,打三十大棍!」兩邊家人點起明晃晃燈籠,張勝李安各執大棍伺候。那雪娥只是不肯脫衣裳。守備恐怕氣了他,在根前不敢言語。孫二娘在旁邊再三勸道:「隨大奶奶分付打他多少,免褪他小衣罷!不爭對著下人脫去他衣裳,他爺體面上不好看的!只望奶奶高抬貴手,委的他的不是了!」春梅不肯,定要去他衣服打,說道:「那箇攔我,我把孩子先摔殺了!然後我也一條繩子吊死就是了!留著他便是了!」于是也不打了,一頭撞倒在地,就直挺挺的昏迷,不省人事。守備諕的連忙扶起說道:「隨你打罷,沒的氣著你!」當下可憐,把這孫雪娥拖番在地,褪去衣服,打了三十大棍,打的皮開肉綻。一面使小牢子半夜叫將薛嫂兒來,即時罄身領出去辦賣。春梅把薛嫂兒叫在背地分付:「我只要八兩銀子,將這淫婦奴才,好歹與我賣在娼門!隨你轉多少,我不管你。你若賣在別處,我打聽出來,只休要見我!」那薛嫂兒道:「我靠那裡過日子?卻不依你說!」當夜領了雪娥來家。那雪娥悲悲切切,整哭到天明。薛嫂便勸道:「你休哭了。也是你的晦氣,冤家撞在一處!老爺見你到罷了,只恨你與他有些舊仇舊恨,折挫你,那老爺也做不得主兒!見他有孩子,須也依隨他。正景下邊孫二娘,不讓他幾分?常言:「拐米倒做了倉官」,說不的了!你休氣哭。」雪娥收淚謝薛嫂:「只望早晚尋箇好頭腦,我去自有飯吃罷!」薛嫂道:「他千萬分付,只教我把你送在娼門。我養兒養女,也要天理!等我替你尋箇單夫獨妻,或嫁箇小本經紀人家,養活得你來也!」那雪娥千恩萬福,謝了薛嫂。過了兩日,只見鄰住一箇開店張媽走來,叫:「薛媽,你這壁廂有甚娘子?怎的哭的悲切?」薛嫂便道:「張媽請進來坐。」說道:「便是這位娘子。他是大人家出來的。因和大娘子合不著,打發出來,在我這裡嫁人。情愿尋箇單夫獨妻,免得惹氣!」張媽媽道:「我那邊下著一個山東賣綿花客人,姓潘,排行第五,年三十七歲。幾車花果,常在老身家安下。前日說他家有箇老母有病,七十多歲;死了渾家半年光景,沒人扶侍。再三和我說,替他保頭親事,並無相巧的。我看來,這位娘子年紀到相當,嫁與他做箇娘子罷!」薛嫂道:「不瞞你老人家說,這位娘子大人出身,不拘粗細都做的。針指女工,鍋頭灶腦,自不必說,又做的好湯水。今纔三十五歲。本家只要三十兩銀子,倒好保與他罷。」張媽媽道:「有箱籠沒有?」薛嫂道:「止是他隨身衣服簪環之類,並無箱籠。」張媽媽道:「既是如此,老身回去對那人說,教他自家來看一看。」說畢,吃茶坐回去了。晚夕對那人說了。次日飯罷以後,果然領那人來相看。一看見了雪娥,好模樣兒,年小,一口氣就還了二十五兩,另外與薛嫂一兩媒人錢。薛嫂也沒爭兢,就兌了銀子,寫了文書,晚夕過去。次日就上車起身。薛嫂叫人改換了文書,只兌了八兩銀子,交到府中春梅收了,只說賣與娼門去了。那人娶雪娥到張媽家,止過了一夜。到第二日五更時分,謝了張媽媽,作別上了車,逕到臨清去了。此是六月天氣,日子長。到馬頭上,纔日西時分。到于酒家店,那裡有百十間房子,都下著各處遠方來的窠子行院娼的。這雪娥一領進入一箇門戶,半間房子裡面,打著土炕,炕上坐著箇五六十歲的婆子,還有箇十七八頂老丫頭,打著盤頭揸頭,抹著鉛粉紅唇,穿著一弄兒軟絹衣服,在炕邊上彈弄琨琶。這雪娥看見,只叫得苦!纔知道那漢子潘五是箇水客,買他來做粉頭,起了他箇名兒叫玉兒。這小妮子名喚金兒,每日拏廝鑼兒出去,酒樓上接客供唱,做這道路營生。這潘五進門,不問長短,把雪娥先打了一頓,睡了兩日,只與他兩碗飯吃。教他樂器學彈唱;學不會又打。打得身青紅遍了,引上道兒,方與他好衣穿,粧點打扮,門前站立,倚門獻笑,眉目嘲人。正是:

「遺蹤堪入時人眼,  不買胭脂畫丹青!」

有詩為證:

「窮途無奔更無投,  南去北來休便休;

一夜彩雲何處散,  夢隨明月到青樓。」

這雪娥在酒家店,也是天假其便。一日,張勝被守備差遣,往河下買幾十石酒麯。這酒家店坐地虎劉二,看見他姐夫來,連忙打掃酒樓乾淨,在上等閣兒裡安排酒殽盃盤,各樣時新果品,好酒活魚,請張勝坐在上面飲酒,酒博士保兒篩酒,近前跪下:「稟問二叔,下邊叫那幾箇唱的上來遞酒?」劉二分付:「叫王家老姐兒,趙家嬌兒,潘家金兒、玉兒四箇,上來伏侍你張姑夫。」酒博士保兒應諾下樓。不多時,只聽得胡梯畔笑聲兒,一般兒四箇唱的頂老,打扮得如花似朵,都穿著輕紗軟絹衣裳,上的樓來,望下一面花枝招颭,繡帶飄飄,拜了四拜,立在旁邊。這張勝猛睜眼觀看,內中一個粉頭,可霎作怪:「到相老爺宅裡小奶奶打發發出來廚下做飯的那雪娥娘子,他如何做這道路在這裡?」那雪娥亦眉眼掃見是張勝,都不做聲。這張勝便問劉二:「那箇粉頭是誰家的?」劉二道:「不瞞姐夫,他是潘五屋裏玉兒、金兒,這箇是王老姐。一箇是趙嬌兒。」張勝道:「王老姐兒我我認的。這潘家玉兒我有些眼熟。」因叫他近前,悄悄問他:「你莫不是老爺宅裡雪姑娘麼?怎生到于此處?」那雪娥聽見他問,便簇地兩行淚下,便道:「一言難盡!」如此這般,具說一遍:「被薛嫂攛瞞,把我賣了二十五兩銀子,賣在這裡供筵習唱,接客迎人!」這張勝平昔見他生的好,纔是懷心。這雪娥席前慇懃勸酒。兩箇說得入港,雪娥和金兒不免拏過琵琶來,唱了箇詞兒,與張勝下酒,名四塊金:

「前生想著少久下他相思債。中途洋卻綰不住同心帶。說著教我淚滿腮,悶來愁似海。萬誓千盟,到今何在?不良才,怎生消磨了我許多時恩愛!」

當下唱畢,彼此穿盃換盞,倚翠偎紅。吃得酒濃時,常言:「世財紅粉歌樓酒,誰為三般事不迷!」這張勝就把雪娥來愛了。兩箇晚夕留在閣兒裡,就一處睡了。這雪娥枕邊風月,耳畔山盟,和張勝儘力盤桓,如魚似水,百般難述。次日起來,梳洗了頭面,劉二又早安排酒餚上來,與他姐夫扶頭。大盤大碗,饕食一頓。收起行裝,餵飽頭口,裝載米麵,伴當跟隨,臨出門與了雪娥三兩銀子。分付劉二:「好生看顧他,休教人欺負!」自此以後,張勝但來河下,就在酒家店與雪娥相會。往後走來走去,每月與潘五幾兩銀子,就包住了他,不許接人。那劉二自恁要圖他姐夫歡喜,連房錢也不問他要了。各窠窩刮刷將來,替張勝出包錢,包定雪娥柴米來。有詩為證:

「豈料當年縱意為,  貪淫倚勢把心欺;

禍不尋人人自取,  色不迷人人自迷。」

畢竟未知後來如何,且聽下回分解:





\end{showcontents}


