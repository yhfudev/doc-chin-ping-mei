%# -*- coding: utf-8 -*-
%!TEX encoding = UTF-8 Unicode
%!TEX TS-program = xelatex
% vim:ts=4:sw=4
%
% 以上设定默认使用 XeLaTex 编译,并指定 Unicode 编码,供 TeXShop 自动识别

%第六十一回 
\chapter{韓道國筵請西門慶\KG 李瓶兒苦痛宴重陽}


\begin{showcontents}{}



「去年九日愁何限,  重上心來益斷腸,

秋色夕陽俱淡薄,  淚痕離思共淒涼;

征鴻有隊全無信,  黃菊無情卻有香,

自覺近來消瘦了,  頻將鸞鏡照容光。」

話說一日,韓道國晚夕舖中散了,回家睡到半夜,他老婆王六兒與他商議:「你我被他照顧此遭,掙了恁些錢,就不擺席酒兒請他來坐坐兒?休說他又丟了孩兒,只當與他釋悶,也請他坐半日。他能吃多少?彼此好看些。就是後生小郎看着,到明日就到南邊去,也知財主和你我親厚,比別人不同。」韓道國道:「我心裡也是這等說。明日是初五日,月忌不好。到初六日,叫了廚子,安排酒席,叫兩個唱的,具個柬帖,等我親自到宅內請老爹散悶坐坐。我晚夕便往舖子裡睡去。」王六兒道:「平白又叫甚麼唱的?只怕他酒後要來這屋裡坐坐,不方便。隔壁樂三姨家常走一個女兒申二姐,年紀小小兒的,打扮又風流,又會唱時興的小曲兒。倒請將他來唱。等晚夕闌上來,老爹若進這屋裡來,打發他過去就是了。」韓道國道:「你說的是。」一宿晚景題過。到次日,這韓道國走到舖子裡,央及溫秀才寫了個請柬兒。走到對門宅內,親見西門慶。聲喏畢,說道:「老爹明日沒事,小人家里治了一杯水酒,無事請老爹貴步下臨,散悶坐一日。」因把請柬遞上去。西門慶看了,說道:「你如何又費此心?我明白倒沒事,衙門中回家就去。」那韓道國作辭出門,來到舖子做買賣。拏銀子叫後生胡秀拏籃子,往街買雞蹄、鵝鴨、鮮魚、嗄飯菜蔬。一面叫廚子在家整理割切,使小廝早拏轎子接了申二姐來。王六兒同丫鬟伺候下好茶好水,客座內打掃收拾桌椅乾淨,單等西門慶來到。等到午後,只見琴童兒先送了一壜葡萄酒 來。然後西門慶着坐涼轎,玳安、王經跟隨,到門首下轎。頭戴忠靖冠,身穿青水緯羅直身,粉頭皂靴。韓道國至,迎入內。見畢禮數,說道:「又多謝老爹賜將來酒!」正面獨獨安放一張校椅,西門慶坐下。不一時,王六兒打扮出來,頭上銀絲{髟狄}髻,翠藍縐紗羊皮金滾邊的箍兒。週圍插碎金草蟲啄針兒,白杭娟對衿兒,玉色水緯羅比甲兒,鵝黃挑線裙子。腳上老鴉青光素段子高底鞋兒,羊皮金緝的雲頭兒。耳邊金丁香兒。打扮的十分精緻,與西門慶插燭也似磕了四個頭兒,回後邊看茶去了。須臾王經紅漆描金託子,拿了兩盞八寶青荳木樨泡茶,韓道國先取一盞,舉的高高,奉與西門慶,然後自取一盞,旁邊相陪。吃畢,王經接了茶盞下去。韓道國便開言說道:「小人承老爹莫大之恩,一向在外,家中小媳婦蒙老爹看顧。王經又蒙擡舉,叫在宅中答應,感恩不淺。今日與媳婦兒商議,無甚孝順,治了一杯水酒兒,請老爹過來坐坐。前日因哥兒沒了,雖然小人在那裡,媳婦兒因感了些風寒,不曾往宅裡弔問的,恐怕老爹惱。今日一者請老爹解解悶,二者就恕俺兩口兒罪。」西門慶道:「無事又教你兩口兒費心。」說着,只見王六兒也在旁邊小杌兒坐下。因向韓道國道:「你和老爹說了不曾?」道國道:「我還不曾說哩。」西門慶問道:「是甚麼?」王六兒道:「他今日心裡要內邊請兩位姐兒來伏侍老爹。恐怕老爹計較,又不敢請。隔壁樂家常走的一個女兒,姓申,名喚申二姐,諸般大小時樣曲兒連數落都會唱。我前日在宅裡見那一位郁大姐,唱的也中中的,還不如這申二姐唱的好。教我今日請了他來唱與爹聽。未知你老人家心下何如?若好,到明日叫了宅裡去,唱與他娘每聽。他也常在各人家走。若叫他,預先兩日定下他,他並不敢誤了。」西門慶道:「既是有女兒,亦發好了。你請出來我看看。」不一時,韓道國教玳安上來,替老爹寬去衣服。一面安放卓席,胡秀拿菓菜案酒上來。無非是鴨腊、蝦米 、海味、燒骨秃之類。當下王六兒把酒打開,燙熱了,在旁執壺。道國把盞,與西門慶安席坐下。然後纔叫上申二姐來。西門慶睜眼觀看他,高髻雲鬟,插着幾枝稀稀花翠。□□□□,□□□□淡淡釵梳,綠衫紅裙,顯一對金蓮趫趫。枕腮粉臉,抽兩道細細春山。青石墜子耳邊垂,糯米銀牙噙口內。望上花枝招颭,與西門慶磕了四個頭。西門慶便道:「請起,你今青春多少?」申二姐道:「小的二十一歲了。」又問:「你記得多少小唱?」申二姐道:「小的大小也記百十套曲子。」西門慶令韓道國旁邊安下個坐兒與他坐。那申二姐向前行畢禮,方纔坐下。先拏箏來,唱了一套秋香亭。然後吃了湯飯,添換上來,又唱了一套半萬賊兵,落後酒闌上來。西門慶分付:「把箏拏過去,取琵琶與他。等他唱小詞兒我聽罷。」那申二姐一逕要施逞他能彈接唱,一面輕搖羅袖,款跨鮫綃,頓開喉音,把絃兒放得低低的,彈了個四不應山坡羊:

「一向來,不曾和冤家面會。肺腑情,難稍難寄。我的心誠想著你,你為我懸心掛意。咱兩個相交,不分個彼此。山盟海誓,心中牢記。你比鶯鶯重生而再有,可惜不在那蒲東寺。不由人一見了眼角留情來呵,玉貌生春,你花容無比。聽了聲嬌姿,好教人目斷東墻,把西樓倦倚。」

「意中人,兩人裡懸心掛意。意兒里,不得和你兩個眉來眼去。去了時,強挨孤枕。枕兒寒,衾兒剩,瑤琴獨對,病體如柴,瘦損了腰肢。知道你夫人行應難離,倒等的我寸心如醉。最關心伴著這一盞寒燈來呵,又被風弄竹聲,只道多情到矣。急忙出離了書幃,不想是花影輕搖,月明如水。」

唱了兩個山坡羊,叫了斟酒。那韓道國教渾家篩酒上來,滿斟一盞,遞與西門慶。因說:「申二姐,你還有好鎖南枝,唱兩個兒與老爹聽。」那申二姐改了調兒,唱鎖南枝道:

「初相會,可意人,年少青春,不上二旬。黑鬖鬖兩朵烏雲,紅馥馥一點朱唇,臉賽夭桃如嫩笋。若生在畫閣蘭堂,端的也有個夫人分。可惜在章臺,出落做下品。但能勾改嫁從良,勝強似棄舊迎新。」

「初相會,可意嬌,月貌花容,風塵中最少。瘦腰肢一捻堪描,俏心腸百事難學。恨只恨,和他相逢不早。常則願席上樽前,淺斟低唱相偎抱,一覷一個真,一看一個飽。雖然是半霎歡娛,權且將悶減愁消。」

西門慶聽了這兩個鎖南枝,正打着他初請了鄭月兒那一節事來,心中甚喜。又見他叫了個賞音。王六兒在旁滿滿的又斟上一盞,笑嘻嘻說道:「爹,你慢慢兒的消飲。申二姐這個纔是零頭兒,他還記得好些小令兒哩。到明日閒了,拏轎子接了,唱與他娘每聽。」又說:「宅中那位唱姐兒?」西門慶道:「那個是常在我家走的郁大姐,這好些年代了。」王六兒道:「管情申二姐到宅裡,比他唱的高。爹到明日呼喚他,早些兒來對我說。我使孩子早拏轎子去接他,送到宅內去。」西門慶因說:「申二姐,我重陽那日使人來接你,去不去?」申二姐道:「老爹說那裡話,但呼喚小的,怎敢違阻?」西門慶聽見他說話兒,心中大喜。不一時,交杯換盞之間,王六兒恐席間說話不方便,教他唱了幾套。悄悄向韓道國說:「教小廝招弟兒送過他那邊樂三嫂家歇去罷。」臨去,拜辭西門慶。門慶向袖中掏出一包兒三錢,賞賜與他買絃。那申二姐連忙花枝招颭,向西門慶磕頭謝了。西門慶約下:「我初八日使人請你去。」那王六兒道:「爹只教王經來對我說,等這裡教小廝送他去。」那申二姐拜辭了韓道國夫婦,招弟領着往隔壁去了。那韓道國打發申二姐去了,與老婆說知,就往舖子裡睡去了。只落下老婆在席上,陪西門慶擲骰飲酒。吃了一回,兩個看看吃的涎將上來,西門慶推起身往後邊更衣,就走入婦人房裡,兩個頂門頑耍。王經便把燈燭拏出來,在前半間內,和玳安、琴童兒三個,做一處飲酒。那後生胡秀,不知道多咱時分,在後邊廚下偷吃多幾碗酒,打發廚子去了,走在王六兒隔壁半間供養佛祖先堂兒內地下,鋪着一領蓆就睡着了。睡了一覺起來,原來與那邊臥房,止隔着一層板壁兒。忽聽婦人房裡聲喚起來,這胡秀只見板壁縫兒,透過燈喨兒來。只道西門慶去了,韓道國在房中宿歇。暗暗用頭上簪子,取下來,刺破透板縫中糊的紙,打一往那邊張看。見那邊房中,喨騰騰點着燈燭。不想西門慶和老婆在屋裡,兩個正幹得好伶伶俐俐。看見把老婆兩隻腿,都是用腳帶弔在床頂上。西門慶上身止着一件綾襖兒,下身赤露,就在床沿上。兩個一來一往,一動一靜,搧打的連身响喨。老婆口裡百般言語都叫將出來,淫聲豔語,通做成一塊。良久,只聽老婆說:「我的親達,你要燒淫婦,隨你心裡揀着那塊,只顧燒,淫婦不敢攔你。左右淫婦的身子屬了你,顧的那些兒了。」西門慶道:「只怕你家裡的嗔是的!」老婆道:「那忘八七個頭八個膽,他敢嗔?他靠着那裡過日子哩!」西門慶道:「你既是一心在我身上。到明日等賣下銀子,這遭打發他和來保起身,亦發留他長遠在南邊立庄,做個買手。家中已有甘夥計發賣,那裡只是缺少個買手看着置貨。」老婆道:「等走過兩遭兒回來,都教他去。省的閒著在家,做甚麼?他說道:『倒在外邊走慣了,一心只要外邊去。』他江湖從小兒走過,甚麼買賣客貨中事兒不知道?你若下顧他,可知好哩。等他回來,我房裡替他尋下一個。我也不要他,一心撲在你身上,隨你把我安插在那裡就是了。我若說一句假,把淫婦不值錢身子就爛化了。」西門慶道:「我兒,你快休賭誓!」這裡兩個一動一靜,都被這胡秀聽了個不亦樂乎。那韓道國先在家中不見胡秀,只說往舖子裡睡去了。走到段子舖裡,問王顯、榮海,說他沒來。韓道國一面又走回家,叫開門,前後尋胡秀,那裡得來?只見王經陪玳安、琴童,三個在前邊吃酒。這胡秀聽見他的語音來家,連忙倒在蓆上,又推睡了。不一時,韓道國點燈尋到佛堂地下,看見他鼻口內打鼾睡,用腳踢醒,罵道:「賊野狗死囚,還不起來!我只說先往舖子裡睡去,你原來在這裡挺的好覺兒。還不起來跟我去!」那胡秀起來,推揉了揉眼,楞楞睜睜,跟道國往舖子裡去了。西門慶弄老婆,直弄勾有一個時辰,方纔了事。燒了王六兒心口裡,并〈毛皮〉蓋子上,尾停骨兒上,共三處香。老婆起來,穿了衣服,教丫鬟打發舀水淨了手。重篩煖酒,再上佳希,情話攀盤。又吃了幾鍾,方纔起身上馬。玳安、王經、琴童三個跟着,到家中已有二更天氣。走到李瓶兒房中。李瓶兒睡在床上,見他吃的酣酣兒的進來,說道:「你今日在誰家吃酒來?」西門慶悉把韓道國請我,見我丟了孩子,與我釋悶。他家叫了個女先生申二姐來,年經小小,好不會唱。又不說郁大姐,等到明日重陽,使小廝拏轎子,接他來家唱兩日你每聽,就與你解解悶。你緊心裡不好,休要只顧思想他了。說着,就要叫迎春來脫衣裳,和李瓶兒睡。李瓶兒道:「你沒的說,我下邊不住的長流,丫頭火上替我煎藥哩。你往別人屋裡睡去罷!你看着我成日好模樣兒罷了,只有一口遊氣兒在這裡邊,來纏我起來。」西門慶道:「我的心肝!我心裡捨不的你,只要和你睡,如之奈何?」李瓶兒瞟了他一眼,笑了笑兒:「誰信你那虛嘴掠舌的?我到明日死了,你也捨不的我罷!」又道:「亦發等我好好兒,你再進來和我睡,也是不遲。」那西門慶坐了一回,說道:「罷罷!你不留我,等我往潘六兒那邊睡去罷。」李瓶兒道:「着來你去,省的屈着你那心腸兒。他那裡正等的你火裡火發,你不去,都忙惚兒來我這屋裡纏。」西門慶道:「你恁說,我又不去了。」那李瓶兒微笑道:「我哄你哩,你去麼?」于是打發西門慶過去了。這李瓶兒起來,坐在床上,迎春伺候他吃藥。拏起那藥來,止不住撲簌簌從香腮邊滾下淚來,長吁了一口氣,方纔吃那盞藥。正是:

「心中無限傷心事,  付與黃鸝叫幾聲。」

不說李瓶兒吃藥睡了。單表西門慶到于潘金蓮房裡。金蓮纔教春梅罩了燈,上床睡下。忽見西門慶推開門進來,便道:「我兒,又早睡了。」金蓮道:「稀倖,那陣風兒刮你到我這屋裡來?」因問:「你今日往誰家吃酒去來?」西門慶道:「韓夥計打南邊來,見我沒了孩子,一者與我釋悶,二者照顧了他外邊走了這遭,請我坐坐。」金蓮道:「他便在外邊。你在家,都照顧了他老婆了。」西門慶道:「夥計家,那裡有這道理!」婦人道:「夥計家有這個道理?齊腰拴着根線兒,只怕{入日}過界兒去了!你還搗鬼哄俺每哩,俺每知道的不耐煩了!你生日時,賊淫婦他沒在這裡?你悄悄把李瓶兒壽字簪子,黃貓黑尾偷與他,都教他戴了來這裡施展。大娘、孟三兒這一家子,那個沒看見?乞我相問着他,那臉兒上紅了。他沒告訴你?今日又摸到那裡去了,賊沒廉耻的貨!你家外頭還少哩,也不知怎的一個大摔瓜長淫婦,喬眉喬樣,抽的那水鬢長長的,搽的那嘴唇鮮紅的,倒人家那血〈毛皮〉,甚麼好老婆?一個大紫腔色黑淫婦!我不知你喜歡他那些兒?嗔道把忘八舅子也招惹將來,都一早一晚教他好往回傳梢話兒。」那西門慶堅執不認,笑道:「怪小奴才兒,單管只胡說!那裡有此勾當?今日他男子漢陪我坐,他又沒出來。」婦人道:「你拏這個話兒哄我,誰不知他漢子是個明忘八!又放羊,又拾柴。一徑把老婆丟與你,圖你家買賣做,要撰你的錢使。你這傻行貨子,是好四十里聽銃响罷了!」見西門慶脫了衣裳,坐在床沿上。婦人探出手來,把褲子扯開,摸見那話軟叮噹的,托子還帶在上面,說道:「可又來,你臘鴨子煮到鍋裡,身子兒爛了,嘴頭兒還硬。見放着不語先生,在這裡強道,和那淫婦怎麼弄聳,聳到這咱晚纔來家?弄的恁軟如鼻涕濃瓜醬的,嘴頭兒還強哩!你賭幾個誓,我教春梅舀一瓶子涼水,你只吃了,我就算你好膽子。論起來,鹽也是這般鹹,酸也是這般酸,禿子包網巾,饒這一抿子兒也罷了!若是信着你意兒,把天下老婆都耍遍了罷。賊沒羞的貨,一個大眼裡火行貨子!你早是個漢子,若是個老婆,就養遍街,{入日}遍巷,屬皮匠的,逢着的就上。」幾句說的西門慶睜睜的。上的床來,教春梅篩熱了燒酒 ,把金穿心盒兒內,拈了一粒,放在口裡嚥下去。仰臥在枕上。令婦人:「我兒,你下去替你達品品,起來是你造化。」那婦人一徑做喬張智,便道:「好乾淨兒,你在那淫婦窟礲子裡鑽了來,教我替你咂,可不硶殺了我!」西門慶道:「怪小淫婦兒,單管胡說白道的!那裡有此勾當?」婦人道:「那裡有此勾當,你指着肉身子賭個誓麼?」亂了一回,教西門慶下去使水,西門慶不肯下去。婦人旋向袖子裡掏出通花汗巾來,將那話抹展了一回,方纔用朱唇裹沒。嗚咂半响,登時咂弄的那話奢稜跳腦,暴怒起來,乃騎在婦人身上,縱塵柄自後插入牝中,兩手兜其股,蹲踞而擺之,肆行搧打,連聲响喨。燈光之下,窺翫其出入之勢,婦人倒伏在枕畔,舉股迎湊者。久之,西門慶興猶不愜,將婦人仰臥朝上,那話上使了粉紅藥兒,頂入去,執其雙足,又舉腰沒稜露腦,抓騰者將二三百度。婦人禁受不的,瞑目顫聲,沒口子叫:「達達,你這遭兒只當將就我,不使上他也罷了!」西門慶口中呼叫道:「小淫婦兒,你怕我不怕?再敢無禮不敢?」婦人道:「我的達達,罷麼。你將就我些兒,我再不敢了。達達慢慢提,看提撒了我的頭髮。」兩個顛鸞倒鳳,又狂了半夜,方纔體倦而寢。話休饒舌。又早到重陽令節。西門慶對吳月娘說:「韓夥計家前日請我,席上唱的一個申二姐,生的人材又好,又會唱,琵琶箏都會。我使小廝接他去。等接了他來,留他兩日,教他唱與你每聽。」于是分付廚下,收拾酒菓肴饌。在花園大捲棚聚景堂內,安放大八仙卓席,放下簾來,合家宅眷,在那裡飲酒,慶賞重陽佳節。不一時,王經轎子接的申二姐到了。入到後邊,與月娘眾人磕了頭。月娘見他年小,生的好模樣兒,問他套數,倒會不多。若題諸般小曲兒,山坡羊、鎖南枝兼數落,倒記的有十來個。一面打發他吃了茶食,先教在後邊唱了兩套。然後花園擺設下酒席。那日西門慶不曾往衙門中去,在家看着栽了菊花,請了月娘、李嬌兒、孟玉樓、潘金蓮、李瓶兒、孫雪娥并大姐,都在席上坐的。春梅、玉簫、迎春、蘭香在旁斟酒伏侍。申二姐先拏琵琶在旁彈唱。那李瓶兒在房中身上不方便,請了半日,纔請了來,恰似風兒刮倒的一般,強打着精神,陪西門慶坐。眾人讓他酒兒也不大好生吃。西門慶和月娘他面帶憂容,眉頭不展,說道:「李大姐,你把心放開,教申二姐唱個曲兒你聽。」玉樓道:「你說與他,教他唱甚麼曲兒,他好唱。」那李瓶兒只顧不說。正飲酒中間,忽見王經走來,說道:「應二爹、常二叔來了。」西門慶道:「請你應二爹、常二叔在小捲棚裡坐,我就來。」王經道:「常二叔教人拏了兩個盒子在外頭。」西門慶向月娘道:「此是他成了房子,買了些禮來謝我的意思。」月娘道:「少不的安排些甚麼管待他,怎好空了他去?你陪他坐去,我這裡分付看菜兒。」西門慶臨出來,又叫申二姐:「你好歹唱個好曲兒與他六娘聽。」一直往前邊去了。金蓮道:「也沒見這李大姐,隨你心裡說個甚麼曲兒,教申二姐唱個你聽就是了!辜負他爹的心。此來為你叫將他來,你又不言語的。」于是催逼的李瓶兒急了,半日纔說出來:「你唱個『紫陌紅徑』俺每聽罷。」那申二姐道:「這個不打緊,我有。」于是取過箏來,排開鴈柱,調定冰絃,頓開喉音,唱折腰一枝花:

「紫陌紅徑,丹青妙手難畫成。觸目繁華如鋪錦,料應是春負我,非是辜負了春。為著我心上人,對景越添愁悶。」

〔東甌令〕  「花零亂,柳成陰,蝶因蜂迷鶯倦吟。方纔眼睜,心兒裡忘了。想啾啾唧唧呢喃燕,重將舊恨,舊恨又題醒,撲簌簌淚珠兒暗傾。」

〔滿園春〕  「悄悄庭院深,默默的情掛心。涼亭水閣,果是堪宜宴飲。不見我情人,和誰兩個問樽?把絲絃再理,將琵琶自撥,是奴欲歇悶情,怎如倦聽?」

〔東甌令〕  「榴如火,簇紅錦,有焰無烟,燒碎我心。懷著向前,欲待要摘一朵,觸觸拈拈不堪口。怕奴家花貌,不似舊時人。伶伶仃仃,怎宜樣簪?」

〔梧桐樹〕  「梧葉兒飄,金風動,漸漸害相思,落入深深井。一旦夜長,難捱孤枕。懶上危樓,望我情人,未必薄情,與奴心相應。他在那裡,那裡貪歡戀飲?」

〔東甌令〕  「菊花綻,桂花零,如今露冷風寒,愁意漸深。驀聽的窗兒外幾聲,幾聲孤鴈,悲悲切切,如人訴。最嫌花下砌畔小蛩吟。咭咭咶咶,惱碎奴心。」

〔浣溪沙〕  「風漸急,寒威凜,害相思,最恐怕黃昏。沒情沒緒,對著一盞孤燈。兒眼數,教還再輪。畫角悠悠聲透耳,一聲聲哽咽難聽。愁來別酒強重斟,酒入悶懷珠淚傾。」

〔東甌令〕 「長吁氣,兩三聲,斜倚定幃屏兒,思量那個人。一心指望夢兒裡,略略重相見。撲撲簌簌雪兒下,風吹簷馬,把奴夢魂驚。叮叮噹噹,攪碎了奴心。」

〔尾聲〕  「為多情,牽掛心。朝思暮想淚珠傾,恨殺多才不見影。」

唱畢,吳月娘道:「李大姐,你好甜酒兒,吃上一鍾兒。」那李瓶兒又不敢違阻了月娘,拏起鍾兒來。咽了一口兒,又放下了。強打着精神兒,與眾人坐的。坐不多時,下邊一陣熱熱的來,又往屋裡去了。不說這裡內眷。單表西門慶到于小捲棚翡翠軒,只見應伯爵與常時節在松墻下正看菊花。原來松墻兩邊,擺放二十盆,都是七尺高各樣有名的菊花。也有大紅袍、狀元紅、紫袍金帶、白粉西、黃粉西、滿天星、醉楊妃、王牡丹、鵝毛葡、鴛鴦花之類。西門慶出來,二人向前作揖。常時節即喚跟來人,把盒兒掇進來。西門慶一見便問:「又是甚麼?」伯爵道:「常二哥蒙你厚情,成了房子。無甚麼酬答,教他娘子製造了這螃蟹鮮,并兩雙爐燒鴨兒 ,邀我來同和哥坐坐。」西門慶道:「常二哥,你又費這個心做甚麼?你令正病纔好些,你又禁害他。」伯爵道:「我也是恁說!他說道:『別的東兒來,恐怕哥不稀罕。』」西門慶令左右打開盒兒觀看,四十個大螃蟹,都是剔剝淨了的。裏邊釀着肉,外用椒料 薑蒜米兒團粉裹就,香油 堞醬油醋造過,香噴噴酥脆好食。又是兩大隻院中爐燒熟鴨。西門慶看了,即令春鴻、王經掇進去。分付:「拏五十文錢,賞拏盒人。」因向常時節謝畢。琴童在旁,掀簾請入翡翠坐的。伯爵只顧誇獎不盡:「好菊花。」問:「哥是那裡尋的?」西門慶道:「是管磚廠劉太監送我這二十盒。」伯爵道:「連這盒?」西門慶道:「就連這盒,都送與我了。」伯爵道:「花到不打緊!這盆正是官窰雙箍鄧漿盆,又吃年代,又禁水漫,都是用絹羅打,用腳跐過泥,纔燒造這個物兒。與蘇州鄧漿磚一個樣兒做法,如今那裡尋去?」誇了一回,西門慶喚茶來吃了。因問:「常二哥,幾時搬過去?」伯爵道:「從兌了銀子,三日就搬過去了。那家子已是尋下房子,兩三日就搬了。昨見好日子,買刮了些雜貨兒,門首把舖兒也開了。就是常二嫂兄弟,替他在舖兒裡看銀子兒。」西門慶道:「俺每幾時買些禮來,休要人多了,再邀謝子純你三四位。我家裡整理菜兒擡了去,休費煩常二哥一些東西兒。叫兩個妓者,咱每替他煖煖房,耍一日。」常時節道:「小弟有心,也要請哥坐坐。算計來,不敢請。地方兒窄狹,恐怕哥受屈馳。」西門慶道:「沒的扯淡!那裡又費他的事起來?如今使小廝請將謝子純來,和他說說。」即令琴童兒:「快請你謝爹去。」伯爵因問:「哥,你那日叫那兩個去?」西門慶笑道:「叫你鄭月娘和洪四兒去。洪四兒令打掇鼓兒,唱慢山坡羊兒。」伯爵道:「哥,你是個人。你請他就不對我說聲,我怎的也知道了?比李桂兒風月如何?」西門慶道:「通色絲子女不可言。」伯爵道:「他怎的前日你生日時,那等不言語扭扭的?也是個肉佞賊小淫婦兒!」西門慶道:「等我到幾時再去着,也攜帶你走走。你月娘兒會打的好雙陸,你和他打兩貼雙陸。」伯爵道:「等我去混那小淫婦兒,休要慣了他!」西門慶道:「你這歪狗材,不要惡識他便好!」正說着,謝希大到了。聲喏畢,坐下。西門慶道:「常二哥如此這般,新有了華居,瞞着俺每已搬過去了。咱每人隨意出些分資,休要費煩他絲毫。我這裡整治停當,教小廝擡了他府上。我還助兩個妓者,咱耍一日何如?」謝希大道:「哥分付每人出多少分資,俺每都送哥這裡來就是了。還有那幾位?」西門慶道:「再沒人,只這三四個兒。每人二星銀子就勾了。」伯爵道:「十分人多了,他那裡沒地方兒。」正說着,只見琴童來說:「吳大舅來了。」西門慶道:「請你大舅這裡來坐。」不一時,吳大舅進入軒內。先與三人作了揖,然後與西門慶敘禮坐下。小廝拿茶上來,同吃了茶。吳大舅起身說道:「請姐夫到後邊說句話兒。」西門慶連忙讓大舅到于後邊月娘房裡。月娘還在捲棚內,與眾姊妹吃酒聽唱。聽見小廝說:「大舅來了,爹陪着在後邊坐着坐說話哩。」一面走到上房見大舅,道了萬福,叫小玉遞上茶來。大舅向袖中取出十兩銀子,遞與月娘,說道:「昨日府上纔領了三錠銀子。姐夫且收了這十兩。餘者待後次再送來。」西門慶道:「大舅,你怎的這般計較?且使着,慌怎的?」大舅道:「我恐怕遲了姐夫的。」西門慶因問:「倉廒修理的也將完了?」大舅道:「還得一個月將完。」西門慶道:「工完之時,一定撫按有些獎勵。」大舅道:「今年考選軍政在邇,還望姐夫扶持,大巡上替我說說。」西門慶道:「大舅之事,都在于我。」說畢話,月娘道:「請大舅來前邊坐。」大舅道:「我去罷。只怕他三位來有甚話說。」西門慶道:「沒甚麼話。常二哥新近問我借了幾兩銀子,買下了兩間房子,已搬過去了。今日買了些禮兒來謝我。節間留他每坐坐,不想大舅來的正好。」于是讓至前邊坐下。月娘連忙教廚下打發菜兒上去。琴童兒與王經先安放八仙卓席端正,拿了小菜菓酒上去。西門慶旋教開庫房,拏去一罈夏提刑家送的菊花酒來 。打開,碧靛清,噴鼻香。未曾篩,先攙一瓶涼水,以去其蓼辣之性。然後貯于布甑內,篩出來,醇厚好吃,又不說葡萄酒 。教王經用小金鍾兒斟一杯兒,先與吳大舅嘗了。然後伯爵等每人都嘗訖,極口稱羨不已。須臾,大盤大碗嚘飯肴品擺將上來,堆滿卓上。先拿了兩大盤玫瑰菓餡蒸糕,蘸着白砂糖,眾人乘熱一搶着吃了一頓。然後纔拿上釀螃蟹 ,并兩盤燒鴨子 來。伯爵讓大舅吃。連謝希大也不知是甚麼做的,這般有味酥脆好吃。西門慶道:「此是常二哥家送來的。」大舅道:「我空癡長了五十二歲,並不知螃蟹這般造作,委的好吃!」伯爵又問道:「後邊嫂子都嘗了嘗兒不曾?」西門慶道:「房下每都有了。」伯爵道:「也難為我這常嫂,也這般好手段兒。」常時節笑道:「賤累還恐整理的不堪口,教列位哥笑話。」吃畢螃蟹,左右上來斟酒。西門慶令春鴻和書童兩個在旁,一遞一個歌唱南曲。應伯爵忽聽大捲棚內彈箏歌唱之聲,便問道:「哥,今日有李桂姐在這裡?不然,如何這等音樂之聲?」西門慶道:「你再聽着,是不是?」伯爵道:「李桂姐不是,就是吳銀兒。」西門慶道:「你這花子,單管只瞎謅。倒是個女先生。」伯爵道:「不是郁大姐?」西門慶道:「不是他。這個姓申二姐,年小哩,好個人材,又會唱。」伯爵道:「真個這等好?哥怎的不牽出來,俺每瞧瞧。又唱個兒俺每聽。」西門慶道:「今日你眾娘每,大節間叫他來賞重陽頑耍。偏你這狗材耳朵內聽的見。」伯爵道:「我便是千里眼,順風耳。隨他四十里有蜜蜂兒叫,我也聽見了。」謝希大道:「你這花子兩耳朵似竹簽兒也似,愁聽不見。」兩個又頑笑了一回。伯爵道:「哥,你好歹叫他出來,俺每見兒。俺每不打緊,教他只當唱個兒與老舅聽也罷了,休要就古執了。」西門慶乞他逼迫不過,一面使王經領申二姐出來,唱與大舅聽。不一時,申二姐來,望上磕了頭起來,旁邊安放校床兒,與他坐下。伯爵問申二姐:「青春多少?」申二姐回道:「屬牛的,二十一歲了。」又問:「會多少小唱?」申二姐道:「琵琶箏上套數小唱,也會百十來個。」伯爵道:「你會許多唱也勾了。」西門慶道:「申二姐,你拏琵琶唱小詞兒罷!省的勞動了你。說你會唱四夢八空,你唱與大舅聽!」分付王經、書童兒席問斟上酒。那申二姐款跨鮫綃,微開檀口,唱羅江怨道:

「懨懨病轉濃,甚日消融?春思夏想秋又冬。滿懷愁悶。訴與天公也。天有知呵,怎不把恩情送?恩多也是個空,情多也是個空,都做了南柯夢。」

「伊西我在東,何日再逢?花箋慢寫封又封,叮嚀囑付,與鱗鴻也。他也不忠,不把我這音書送。思量他也是空,埋怨他也是空,都做了巫山夢。」

「恩情逐曉風,心意懶慵,伊家做作無始終。山盟海誓,一似耳邊風也。不記當時,多少恩情重。虧心也是空,痴心也是空,都做了蝴蝶夢。」

「惺惺似懞懂,落伊套中。無言暗把珠淚湧。口心誰想,不相同也。一片真心,將我廝調弄。得便宜也是空,失便宜也是空,都做了陽臺夢。」

不說前邊彈唱飲酒。且說李瓶兒歸到房中,坐淨桶,下邊似尿也一般,只顧流將起來,登時流的眼黑了。起來穿裙子,忽然一陣旋暈的,向前一頭搭倒在地。饒是迎春在旁搊扶着,還把額角上磕傷了皮。和奶子搊到炕上,半日不省人事。慌了迎春使綉春連忙快對大娘說去。那綉春走到席上,報與月娘眾人:「俺娘在房中暈倒了。」這月娘撇了酒席,與眾姊妹慌忙走來看視。見迎春、奶子兩個搊扶着他,坐在炕上,不省人事。便問他:「好好的進屋裡,端的怎麼來就不好了?」迎春揭開淨桶與月娘瞧,把月娘諕了一跳,說道:「此是他剛纔吃了酒,助趕的他這血旺了,流了這些。」玉樓、金蓮都說:「他幾曾大好生吃酒來?」一面煎燈心薑湯 灌他。半晌甦着過來,纔說出話兒來了。月娘問:「李大姐,你怎的來?」李瓶兒道:「我不怎的。坐下桶子,起來穿裙子,只見眼面前黑黑的一塊子,就不覺天旋地轉起來,由不的身子就倒了。」月娘便要使來安兒:「請你爹進來對他說,教他請任醫官來看你。」那李瓶兒又嗔教請去:「休要大驚小怪,打攪了他吃酒。」月娘分付迎春:「打舖教你娘睡罷。」月娘于是也就吃不成酒了。分付收拾了家火,都歸後邊去了。西門慶陪侍吳大舅眾人,至晚歸到後邊月娘房中。月娘于是也就吃不成酒了。分付收拾了家火,都歸後邊去了。西門慶陪侍吳大舅眾人,至晚歸到後邊月娘房中。月娘告訴李瓶兒跌倒之事。西門慶慌走到前邊來看視。見李瓶兒睡在炕上,面色臘查黃了,扯着西門慶衣袖哭泣。西門慶問其所以。李瓶兒道:「我到屋裡坐榪子。不知怎的,下邊只顧似尿也一般流起來。不覺眼前一塊黑黑的,起來穿裙子,天旋地轉,就跌倒了。恁甚麼就顧不的了!」西門慶見他額上磕傷一道油皮,說道:「丫頭都在那裡,不看你?怎的跌傷了面貌?」李瓶兒道:「還虧大丫頭都在根前,和奶子搊扶着我。不然還不知跌得怎樣的。」西門慶道:「我明日還早使小廝請任醫官來看你看。」當夜就在李瓶兒對面床上,睡了一夜。次日早辰,投往衙門裡去,旋使琴童騎頭口請任醫官去了。直到晌午纔來。西門慶先在大廳上陪吃了茶,使小廝說進去。李瓶兒房裡收拾乾淨,薰下香,然後請任醫官到房中。診畢脉,走出外邊廳上,對西門慶說:「老夫人脉息,比前番甚加沉重些。七情感傷,肝肺火太盛,以致木旺土虛,血熱妄行,猶如山崩而不能節制。復使大官兒後邊問去,若所下的血紫者,猶可以調理。若鮮紅者,乃新血也。學生撮過藥來,若稍止則可有望,不然難為矣!」西門慶道:「望乞老先生留神加減,學生必當重謝!」任醫官道:「是何言語?你我厚間,又是明川情分,學生無不盡心。西門慶待畢茶,送出門。隨即具一疋杭絹、二兩白金,使琴童兒討將藥來,名日歸脾湯,乘熱而吃下去,其血越流之不止。西門慶越發慌了。又請大街胡太醫來瞧。胡太醫說:「是氣沖血管,熱入血室。」亦取將藥來吃下去,如石沉大海一般。月娘見前邊亂着請太醫,只留申二姐住了一夜,與了他五錢銀子,一件雲絹比甲兒并花翠裝了個盒子,打發他坐轎子去了。花子由自從開張那日吃了酒去,聽見李瓶兒不好,至是使了花大嫂買了兩物禮來看他。見他瘦的黃懨懨兒,不比往時,兩個在屋裡大哭了一回。月娘後邊擺茶,請他吃了。韓道國說:「東門外住的一個看婦人科的趙太醫,指下明白,極看得好。前歲小姪媳婦月經不通,是他看來。老爹這裡差人請他來看看六娘,管情就好!」西門慶於是就使玳安同王經兩個,疊騎着頭口,往門外請趙太醫去了。西門慶請了應伯爵來,在廂房坐的,和他商議:「第六個房下,甚是不好的重,如之奈何?」伯爵失驚道:「這個嫂子貴恙,說好些,怎的又不好起來?」西門慶道:「自從小兒沒了,一向着了憂慼,把病來又犯了。昨日重陽,我說接了申二姐,節間你每打夥兒散悶頑耍。他又沒大好生吃酒。誰知走到屋中,就不好暈起來,一交跌倒在地,把臉都磕破了。請任醫官來看,說脉息比前沉重。吃了藥,倒越發血盛了。」伯爵道:「哥,你請胡太醫來看,怎的說?」西門慶道:「胡太醫說是氣沖了血管,吃了他的,也不見動靜。今日韓夥計說,門外一個趙太醫,名喚趙龍崗,專科看婦女。我使小廝騎頭口請。去了一回,把我焦愁的了不得!生生為這孩子不好,是白日黑夜,思慮起這病來了。婦女人家,又不知個回轉,勸着他,又不依你,教我無法可處!」正說着,平安來報:「喬親家爹來了。」西門慶一面讓進廳上坐。敘禮已畢,坐下。喬大戶道:「聞得六親家母有些不安,昨日舍甥到家,請房下便來奉看。西門慶道:「便是一向因小兒沒了,他着了憂慼,身上原有些不調,又感發起來了。蒙親家掛心。」喬大戶道:「也曾請人來看不曾?」西門慶道:「常吃任後溪的藥。昨日又請大街胡先生來看,吃藥越發轉盛,今日又請門外專看婦人科趙龍崗去了。」喬大戶道:「咱縣門前住的行醫何老人,大小脉方俱精。他兒子何歧軒,見今上了個冠帶醫士。親家何不請他來看看親家母?」西門慶道:「既是好,等小价請了趙龍崗來看了脉息,看怎的說,再請他來不遲。」喬大戶道:「親家依我愚見,如今請了何老人來看了親家母脉息,講說停當,安在廂房內坐的。待盛价門外請將趙龍崗來,看他診了脉怎麼說,教他兩個細講一講,就論出病原來了。然後下藥,無有個不效之理。」西門慶道:「親家說的是。」一面使玳安:「拏我拜帖兒,和喬通去請縣門前行醫何老人來。」玳安等應諾去了。西門慶請伯爵到廳上,與喬大戶相見,同坐一處吃茶。那消片晌之間,何老人到來。進門與西門慶、喬大戶等作了揖,讓于上面坐下。西門慶舉手道:「數年不見,你老人家不覺越發蒼髯皓首。」喬大戶又問:「令郎先生肆業盛行?」何老人道:「他逐日縣中迎送,也不得閒。倒是老拙常出來看病。」伯爵道:「你老人家高壽了?還這等健朗!」何老人道:「老拙今年痴長八十一歲。」敘畢話,看茶上來吃了,小廝說進去。須臾請至房中,就床看李瓶兒脉息,旋搊扶起來,坐在炕上,挽着香雲,阻隔三焦,形容瘦的十分狼狽了。但見他:

「面如金紙,體似銀條,看看減褪丰標,漸漸消磨精彩。胸中氣急,連朝水米怕沾唇,五臟膨脝,盡日藥丸難下腹。隱隱耳虛聞磐响,昏昏眼暗覺螢飛。六脉細沉,東岳判官催命去;一靈縹緲,西方佛子喚同行。喪門弔客已臨身,扁鵲慮醫難下手。」

那何老人看了脉息,出來外邊廳上,向西門慶、喬大戶說道:「這位娘子乃是精沖了血管起來,然後着了氣惱。氣與血相博則血如崩。細思當初起將病之由,看是也不是?」西門慶道:「你老人家如何治療?」正相論間,忽報:「琴童和王經,門外請了趙先生來了。」何老人便問:「是何人?」西門慶道:「也是夥計舉來一醫者。你老人家只推不知。待他看了脉息出來,你老人家和他兩個相講一講,好下藥。」不一時,從外而入。西門慶與他敘禮畢,然後與眾人相見。何、喬二老居中,讓他在左,應伯爵在右,西門慶主位相陪。來安兒拿上茶來吃了,收下盞託去。此人便問:「二位尊長貴姓?」喬大戶道:「俺二人一位姓何,一位姓喬。」伯爵道:「在下姓應。敢問先生高姓,尊寓何處,治何生理?」其人答道:「不敢,在下小子,家居東門外頭條巷二郎廟三轉橋四眼井住的,有名趙搗鬼便是。平生以醫為業。家祖見為太醫院院判,家父見充汝府良醫。祖傳三葷,習學醫術。每日攻習王叔和、東垣勿聽子藥性賦,黃帝素問、難經,活人書,丹溪纂要,丹溪心法,潔古老脉訣,加減十三方,千金奇效良方,壽域神方,海上方,無書不讀,無書不看。藥用胸中活法,脉明指下玄機。六氣四時,辨陰陽之標格;七表八裡,定關格之沉浮。風虛寒熱之症候,一覽無餘;弦洪芤石之脉理,莫不通曉。小人拙口鈍脗,不能細陳。聊有幾句,道其梗概。」便道:

「我做太醫姓趙,  門前常有人叫。

只會賣杖搖鈴,  那有真材實料。

行醫不按良方,  看脉全憑嘴調。

撮藥治病無能,  下手取積而妙。

頭疼須用繩箍,  害眼全憑艾醮。

心疼定敢刀剜,  耳聾宜將針套。

得錢一昧胡醫,  圖利不圖見效。

尋我的少吉多凶,  到人家有哭無笑。」

正是:

「半積陰功半養身,  古來醫道通仙道。」

眾人聽了,都呵呵笑了。何老人道:「你門裡出身,門外出身?」趙太醫道:「門裡出身怎的說?門外出身怎的說?」何老人道:「你門裡出身,有父待子接脉理之良法。若是門外出身,只可問病下藥而已。」趙太醫道:「老先生你就不知道,古人云:『望聞問切,神聖功巧。』學生三輩門裡出身,先問病,後看脉,還要觀其氣色。就如同子平兼五星,還要觀手相貌,纔看得准,庶乎不差!」何老人道:「既是如此,請先生進看去。」西門慶即令琴童後邊說去:「又請了趙先生來了。」不一時,西門慶陪他進入李瓶兒房中。那李瓶兒方纔睡下,安逸一回,又搊扶起來,靠着枕褥坐着。這趙太醫先診其左手,次診右手。便教老夫人抬起頭來,看看氣色。那李瓶兒真個把頭兒揚起來。趙太醫教西門慶:「老爹,你問聲老夫人,我是誰?」西門慶便問李瓶兒:「你看這位是誰?」那李瓶兒擡頭看了一眼,便低聲說道:「他敢是太醫?」趙先生道:「老爹不妨事,死不成。還認的人哩!」西門慶笑道:「趙先生你用心看,我重謝你。」一面看視了半日,說道:「老夫人此病,休怪我說。據看其面色,又診其脉息,非傷寒則為雜症,不是產後,定然胎前。」西門慶道:「不是此疾,先生你再仔細診一診。」先生道:「敢是飽悶傷食,飲饌多了?」西門慶道:「他連日飯食,通不十分進。」趙先生又道:「莫不是黃病?」西門慶道:「不是。」趙先生道:「不是,如何面色這等黃?」又道:「多管是脾虛泄瀉。」西門慶道:「也不是泄疾。」趙先生道:「不泄瀉,都是甚麼?怎生的害個病,也教人摸不着頭腦?」坐想了半日,說道:「我想起來了。不是便毒魚口,定然是經水不調勻。」西門慶道:「女婦人,那裡便毒魚口來?你說這經事不調,倒有些近理。」趙先生道:「南無佛耶,小人可怎的也猜着一庄兒了!」西門慶問:「如何經事不調勻?」趙先生道:「不是乾血癆,就是血山崩。」西門慶道:「實說與先生,房下如此這般,下邊月水淋漓不止,所以身上都瘦弱了。你有甚急方?合些好藥與他吃,我重重謝你。」趙先生道:「不打緊處,小人有藥。等我到前邊寫出個方來,好配藥去。」西門慶一面同他來到前廳。喬大戶,何老人還未去,問他:「甚麼病源?」趙先生道:「依小人講,只是經水淋漓。」何老人道:「當用何藥以治之?」趙先生道:「我有一妙方,用着這幾味藥材,吃下去,管情就好。」聽我說:

「甘草甘逐與碙砂,藜蘆巴豆與芫花。人言調著生半夏,用烏頭杏仁天麻。這幾味兒齊加,蔥蜜和丸只一撾。清辰用燒酒 送下。」

何老人聽了,便道:「這等藥吃了,不藥殺人了?」趙先生道:「自古毒藥苦口利于病。若早得摔手伶俐,強如只顧牽經。」西門慶道:「這廝俱是胡說。」教小廝:「與我扠出去。」喬大戶道:「夥計既舉保來一場,醫家休要空了他。」西門慶道:「既是恁說,前邊舖子裡稱二錢銀子,打發他去罷。」那趙太醫得二錢銀子往家,一心忙似箭,兩家走如飛。西門慶見打發趙太醫去了,因向喬大戶說:「此人原來不知甚麼。」何老人道:「老拙適纔不敢說。此人東門外有名的趙搗鬼,專一在街上賣杖搖鈴,哄過往之人。他那裡曉的甚脉息病源?」因說:「老夫人此疾,老拙到家撮兩貼藥來,遇緣看服畢,經水少減,胸口稍開,就好用藥。只怕下邊不止,飲食再不進,就難為矣!」說畢起身。西門慶這裡封白金一兩,使玳安拏盒兒討將藥,晚夕與李瓶兒吃了。並不見其分毫動靜。吳月娘道:「你也省可里與他藥吃。他飲食先阻住了,肚腹中有甚麼兒?只顧拿藥陶碌他。前者那吳神仙算他二十七歲有血光之災,今年都不整廿七歲了?你還使人尋這吳神仙去,教替他打算算,這祿馬數上,看如何?只怕犯着甚麼星辰,替他禳保禳保。」西門慶這裡旋差人拏帖兒往周守備府裡問去。那裡說:「吳神仙雲遊之人,來去不定。但來,只在城南土地廟下。今歲從四月裡往武當山去了。要打數算命,真武廟外有個黃先生,打的好數。一數只要三錢銀子,不上人家門去。一生別後事,都如眼見。」西門慶隨即使陳經濟拏三錢銀子,逕到北邊真武廟門首抄尋。有黃先生家門上,貼着「抄算先天易數,每命卦金三星。」陳經濟向前作揖,奉上卦金,說道:「有一命,煩先生推算。」說與他八字,女命,年二七歲,正月十五日午時。這黃先生把算子一打,就說:「這女命辛未年,庚寅月,辛卯日,壬午時,理取印綬之格,借四歲行運。四歲已未,十四歲戊午,廿四歲丁巳,三十四歲丙辰。今年流年丁酉,此肩用事,歲傷日干,計都星照命,又犯喪門五鬼,災殺作抄。夫計都者,乃陰晦之星也。其像猶如亂絲而無頭,變異無常。人運逢之,多主暗昧之事,引惹疾病。主正、二、三、七、九月病災,有損暗傷財物,小口凶殃。小人所算,口舌是非,主失財物。若是陰人,大為不利。」斷云:

「計都流年臨照,  命逢陸地行舟,  必然家主皺眉頭。

靜裡躊躇無奈,  閒中悲慟無休,  女人犯此問根由。

必似亂絲不久,  切記胎前產後。」

其數曰:

「莫道成家在晚時,  止緣父母早先離,

芳姿嬌媚年來美,  百計俱全更有思;

傅揚伉儷當龍至,  榮合屠羊看虎威,

可憐情熟恩情失,  命入雞宮葉落裡。」

抄畢數,封付與經濟拏來家。西門慶和應伯爵、溫秀才坐的,見經濟抄了數來,拏到後邊,解說與月娘聽,命中多凶少吉。西門慶不聽便罷,聽了眉頭搭上三黃鎖,腹內包藏萬斛愁。正是:

「高貴青春遭大喪,  伶俐醒然卻受貧,

年月日時該定載,  算來由命不由人。」

畢竟未知後來如何,且聽下回分解:





\end{showcontents}


