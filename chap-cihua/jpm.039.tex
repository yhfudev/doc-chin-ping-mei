%# -*- coding: utf-8 -*-
%!TEX encoding = UTF-8 Unicode
%!TEX TS-program = xelatex
% vim:ts=4:sw=4
%
% 以上设定默认使用 XeLaTex 编译,并指定 Unicode 编码,供 TeXShop 自动识别

%第三十九回 
\chapter{西門慶玉皇廟打醮\KG 吳月娘聽尼僧說經}


「漢武清齋夜築壇,  自斟明水醮仙宮,

殿前玉女移香案,  雲際金人捧露盤;

絳節幾時還人夢,  碧桃何處更驂鸞,

茂陵煙雨埋弓劍,  石馬無聲蔓草寒。」

話說當日西門慶在潘金蓮房中,歇了一夜。那婦人恨不的鑽入他腹中,在枕畔千般貼戀,萬種牢籠,淚搵鮫鮹,語言溫順,實指望買住漢子心。不料西門慶外邊又刮刺上了韓道國老婆王六兒,替他獅子街石橋東邊,使了一百廿兩銀子,買了一所門面兩間,倒底四層房屋居住。除了過道,第二層間半客位,第三層除了半間供養佛像祖先,一間做住房。裡面依舊廂着炕床,對面又是燒煤火炕,收拾糊的乾淨。第四層除了一間廚房,半間盛煤炭,後邊還有一塊做坑廁。俱不必細說。自從搬過來,那左近街坊鄰舍,都知他是西門慶夥計,又見他穿着一套兒齊整絹帛衣服,在街上搖擺,他老婆常插戴的頭上黃熀熀打扮模樣,在門前站立。這等行景,不敢怠慢,都送茶盒與他。又出人情慶賀。那中等人家,稱他做韓大哥、韓大嫂。以下者,趕着以叔嬸呼之。西門慶但來他家,韓道國就在舖子裡上宿,教老婆陪他自在頑耍。朝來暮往,街坊人家也多知道這件事。懼怕西門慶有錢有勢,誰敢惹他!見一月之間,西門慶也來行走三四次,與王六兒打的一似火炭般熱,穿着器用的,比前日不同。看看臘月時分,西門慶在家亂着送東京并府縣軍衛本衛衙門中節禮。有玉皇廟吳道官使徒弟送了四盒禮物,一盒肉,一盒銀魚,兩盒菓餡蒸酥;并天地疏,新春符,謝灶誥。西門慶正在上房吃飯,玳安兒拏進帖來,上寫着:「玉皇廟小道吳宗嘉頓首拜。」西門慶揭開盒兒看了,說道:「出家人,又教他費心送這厚禮來!」分付玳安,連忙教書童兒封一兩銀子拿回帖與他。月娘在旁,因話題起:「一個出家人,你要使的頭節尾常受他的禮,到把前日李大姐生孩兒時,你說許了多少願醮,就教他打了罷。」西門慶道:「早是你題起來,我許下一佰廿分醮,我就忘死了!」月娘道:「原來你這個大謅答子貨,誰家願心是忘記的!你便有口無心許下,神明都記着。嗔道孩子成日恁啾啾唧唧的,原來都這願心壓的他!此是你幹的營生?」西門慶道:「既恁說,正月裡就把這醮願在吳道官這廟裡還了罷。」月娘道:「昨日李大姐說,這孩子有些病痛兒的,要問那裡討個外名。」西門慶道:「又往那裡討外名?就寄名在吳道官這廟裡罷。」因問玳安:「他廟裡有誰在這裡?」玳安道:「是他第二個徒弟應春跟了禮來。」西門慶一面走出外邊來,那應春兒連忙跨馬磕頭,說:「家師父多拜上老爹,沒什麼孝順,使小徒來送這天地疏,并些微禮兒,與老爹賞人。」西門慶止還了半禮,說道:「多謝你師父厚禮。」讓他坐。說道:「小道怎麼敢坐?」西門慶道:「你坐,我有話和你說。」那道士頭戴小帽,身穿青布直掇,下邊履鞋淨襪,謙遜數次,方纔把椅兒挪到旁另坐下。西門慶換茶來吃了,說道:「老爹有甚鈞語吩咐?」西門慶道:「正月裡,我有些醮願,要煩你師父替我還還兒,在你本院,也是那日就送小兒寄名。不知你師父閑不閑?」徒弟連忙立起身來,說道:「老爹分付,隨問有甚人家經事,不敢應承。請問老爹,訂在正月幾時?」西門慶道:「就訂在初九爺旦日那個日子罷。」徒弟道:「此日又是天誕。玉匣記上,我請律爹交慶,五福駢臻,修齋建醮甚好。那日開大殿與老爹鋪壇。請問老爹,多少醮款?」西門慶道:「也是今歲七月,為生小兒,許了一百廿分清醮。一向不得個心淨,趁着正月裡還了罷!就把小兒送與你師父,向三寶座下討個外名。」徒弟又問:「請問那日,延請多少道眾?」西門慶道:「教你師父請十六眾罷。」說畢,左右放卓兒待茶,先封十五兩經錢,另外又封了一兩酬答他的節禮。又說:「道眾的襯施,你師父不消備辦。我這裡連阡張香燭,一事帶去。」喜歡的道士屁滾尿流,臨出門謝了又謝,磕了頭兒又磕。到正月初八,先使玳安兒送了一石白米,一擔阡張,十斤官燭,五斤沉檀馬牙香,十二疋生眼布做襯施;又送了一對京段,兩罈南酒 ,四隻鮮鵝,四隻鮮雞,一對豚蹄 ,一腳羊肉,十兩銀子,與官哥兒寄名之禮。西門慶預先發帖兒,請下吳大舅、花大舅、應伯爵、謝希大四位相陪。陳經濟騎頭口先到廟中,替西門慶瞻拜。到初九日,西門慶也沒往衙中去,絕早冠帶,騎大白馬,僕從跟隨,前呼後擁,送出東門,往玉皇廟來。遠遠望見結綵的寶旛,過街榜棚,進約不上五里之地,就是玉皇廟。至山門前下馬,睜眼觀看,果然好座廟宇,天宮般蓋造。但見:

「青松鬱鬱,翠柏森森;金釘朱戶,玉橋低影。軒宮碧瓦雕簷,繡 高懸寶檻。七間大殿,中懸勅額金書;兩廡長廊,彩畫天神帥將。祥雲影裡,流星門高接青霄;瑞霞光中,鬱羅臺直侵碧漢。黃金殿上,列天帝三十二尊;白玉京中,現臺光百千萬億。三天門外,離婁與師曠猙獰,左右階前,白虎與青龍猛勇。寶殿前仙妃玉女,霞帔曾獻御香花;玉陛下四相九卿,朱履肅朝丹鳳闕。九龍床上,坐著個不壞金身,萬天教主玉皇張大帝。頭戴十一冕旒,身披袞龍青袍。腰繫藍田帶,按八卦九宮;手執白玉圭,聽三皈五戒。金鐘撞處,三千世界盡皈依;玉磬鳴時,萬象森羅皆拱極。朝天閣上,天風吹下步虛聲;演法壇中,夜月常聞仙佩響。只此便為真紫府,更于何處覓蓬萊!」

西門慶由正門而入,見頭一座流星門上,七尺高朱紅牌架,列着兩行門對,大書:

「黃道天開,祥啟九天之闐闔,迓金輿翠蓋以延恩;

玄壇日麗,光臨萬聖之 幢,誦寶芨瑤章而闡化。」

到了寶殿上,懸着二十六齋題,大書着:

「靈寶答天謝地,報國酬恩,九轉玉樞,盟寄名,吉祥普滿齋壇。」

兩邊一聯:

「先天立極,仰大道之巍巍,庸申至悃;

昊帝尊居,鑒清修之翼翼,上報洪恩。」

西門慶進入壇中香案前,旁邊一小童捧盆巾灌手畢,鋪排跪請上香,鋪毡褥行禮叩壇畢。原來吳道官諱宗嘉,法名道真,生的魁偉身材,一臉鬍鬚,襟懷洒落,廣結交,好施捨。見作本宮住持,以此高貴達官,多往投之,做醮席設甚齊整,迎賓待客,一團和氣。手下也有三五個徒弟徒孫,一呼百諾。西門慶會中,常在建醮,每生辰節令,疏禮不缺。何況西門慶又做了刑名官,來此做好事,送公子寄名,受其大禮,如何不敬?那日就是他做齋功主行法事,頭戴玉環九陽雷巾,身披天青二十四宿大袖鶴氅,腰繫絲帶,忙下經筵來與西門慶稽首:「小道蒙老爹錯愛,迭受重禮,使小道卻之不恭,受之有愧!就是哥兒寄名,小道禮當叩祝三寶,保安增延壽命,尚不能以報老爹大恩;何以又叨受老爹厚賞許多厚禮,誠有媿赧!經襯又且過厚,令小道愈不安。」西門慶道:「厚勞費心辛苦,無物可酬,薄禮表情而已!」敘禮畢,兩邊道眾齊來稽首。一面請去外方丈三間廠廳,名曰松鶴軒,多是朱紅亮槅,那裡自在坐處待茶。西門慶四面粉牆,擺設湖山瀟洒,堂中椅卓光鮮,左壁掛:「黃鶴樓白日飛昇」;右壁懸:「洞庭湖三番渡過」。正面有兩幅吊屏,草書一聯:「引兩袖清風舞鶴,對一方明月談經。」西門慶剛坐下,就令小廝棋童兒:「拏馬接你應二爹去,只怕他沒馬,如何這咱還沒來?」玳安道:「有姐夫騎的驢子,還在這裡。」西門慶道:「也罷。」分付棋童:「快騎接去。」那棋童從山門裡面,牽出來騎了,一直去了。吳道官誦畢經,下來遞茶,陪西門慶坐敘話:「老爹敬神,一點誠心,小道怎敢惹罪?各道多從四更起來,到壇諷誦諸品仙經,并玉皇恭行醮經。今日三朝九轉玉樞法事,多是整做。將官兒的生月八字,另具一字文書,奏名于三寶面前,起名叫做吳應元。太乙司命桃延合康壽齡,永保富貴遐昌。小道這裡又添了二十四分答謝天地,十二分慶讚上帝,二十四分薦亡,共列一百五十八分醮款。」西門慶道:「多有費心!」不一時,打動法鼓,請西門慶到壇看文書。西門慶從新換了大紅五彩獅補吉服,腰繫蒙金犀角帶,到壇,有絳衣表白在,方先宣念齋意:

「大宋國山東清河縣縣牌坊居住,奉道祈恩,酬醮保安。信官西門慶,本命丙寅年七月廿八日子時建生,同妻吳氏,本命戊辰年八月十五日子時建生。」表白道:「還有寶眷,小道未曾添上。」西門慶道:「你只添上個李氏,辛未年正月十五日申時建生,同男官哥兒,丙申年七月廿三日申時建生。」「領家眷等,即日投誠,拜干洪造。言念慶一介微生,三才末品。出入起居,每感龍天之護佑;迭遷寒暑,常蒙神聖以匡扶。職列武班,叨承禁衛。沐恩光之寵渥,享符祿之豐盈。蒞任刑名,每思圖報。恭逢盛世,仰賴帡幪。是以修設清醮,共廿四分位,答報天地之洪恩,酬祝皇王之巨澤。又修設清醮十二分位,茲逢天誕,慶讚帝真。介五福以遐昌,迓諸天而下邁。良願于去歲七月二十三日,因為側室李氏生男官哥兒是慶,要祈坐蓐無虞,臨盆有慶。恭對將男官兒寄宇三寶殿下,賜名吳應元。期在出幼圓滿,另行請祈天地位下,告許清醮一百廿分位,續箕裘之行嗣,保壽命之延長。附薦西門氏門中,三代宗親等魂:祖西門京良,祖妣李氏,先考西門達,妣夏氏,故室人陳氏,及前亡後化,昇墜罔知。是以修設淨醮十二分位,恩資道力,均證生方。共列仙醮一百八十分位,仰干化覃,俯賜勾銷。謹以宣和三年正月初九日,天誕良辰,特就大慈玉皇殿仗延官道,修建靈寶,答天謝地,報國酬盟,慶神保安,寄名轉經,吉祥普滿大齋一晝夜。延三境之司尊,迓萬天之帝駕。日近清光,出入金門而有喜;時加美秩,褒封紫誥以增榮。一門長叨均安,四序公和迪吉。公于道力,今滿方來。謹意。」

宣畢齋意,鋪設下許多文書符命,表白一一請看。揭開第一張說道:「此是棄世功果影發文書,申請三天三境上帝,十極高真,三官四聖,泰玄都省,及天曹大皇萬滿真君,天曹掌醮司真君,天曹降聖司真君,到壇證監功德的奏收。」又揭起第二張:「此是申請東岳天齊大生神聖帝,子孫娘娘,監生衛房聖母元君,并當時許還願日受禱之神。今日勾銷頃願典者,祠家侍奉長生香火,三教明神,勾銷老爹昔日許的願款。及行下七十五司地府真官案吏主者,到壇來受追薦,護送亡人生天。此一票,是玉女靈官,天神帥將,功曹符使,土地等神,捧奏三天門運遞關文。此一張,王清總召萬靈真符,高功發遣公文,受事官符。此一張,是召九斗陽芒流星火全紾大將,開天門的符命。」看畢此處,又到一張卓上,揭起頭一張來:「此是早朝開啟請無佞太保康元帥,九天靈符監齋使者,嚴禁齋儀,監臨廚所。此一張,是請正法馬、趙、溫、關四大元帥,崔、盧、竇、鄧四大天君,監臨壇監門。及玄壇四靈神君,九鳳破機大將軍,淨壇蕩穢,以格高真。此一字是早朝啟五師箋文,晚朝謝五師箋文。此一字是開闢二代捲簾化壇真符。此一字是請神霄辟非大將軍,鳴金鍾陽牒;神雷禁壇大將軍,擊玉磬陰牒。此一字是安鎮五方真人雲象,東方九炁鎮天玉字真文,南三炁鎮天玉字真文,西方七炁鎮天玉字真文,北方五炁鎮天玉字真文,中央一炁鎮天玉字真文,請五老上帝安鎮壇垠證監功德。」俱是五方顏色彩畫的。「此一字早朝頭一遍,轉經高上神霄,玉真王南極長生大帝;第二遍轉經高上碧霄,東極青華生大帝;第三遍轉經高上青雷九天應元雷聲。普化天尊;午朝第四遍,轉經高上玉霄九天雷祖大帝;第六遍,轉經高上泰霄六天洞淵大帝;晚朝第七遍,轉經高上紫霄深波天主帝君;第八遍轉經高上景霄,青城益算可幹司丈人真君;第九遍轉經高上絳霄九天採訪使真君。九道表箋,掠剩報應,幽枉積逮,起四司謝四司箋。此又一字,是午朝高功捧奏拜進二天玉陛,黃素朱衣,并遣旨介真符醮吏者,當同日受事功曹,護送章表殿遞云盤關文。一字是三天持寶籙大將軍,并金龍茭龍騎吏火府,賫簡童子,靈寶諸符命,不可細數。此一字是晚朝謝恩誠詞都疏,及一百八十表醮,經醮雲鶴馬子俵分錢馬滿散關文。」又一卓案上:「此是哥兒三寶蔭下寄名外,一家文書符索牒劄。」其餘不暇細覽。「請謝高功老爹今日十分費心。」西門慶于是洞案前炷了香,畫了文書,左右捧一疋尺頭與吳道官畫字。固辭再三,方令小童收了。然後一個道士,向殿角頭〈石古〉碌碌。擂動法鼓,有若春雷相似。合堂諸眾,一派音樂響起。吳道官身披大紅五彩雲織法氅,腳穿雲根飛舄朱履,手執牙笏,關發文書,發壇召將,兩邊鳴起鍾來。鋪排引西門慶進壇裏,向三寶案左右兩邊上香。西門慶于是睜眼觀看,果然鋪設齋壇齊整。但見:

「位按五方,壇分八級,上層供三清四御,八極九霄,十極高真,雲宮列聖;中層山川嶽瀆,社會隍司,福地洞天,方輿博厚;下層冥官幽壤,地府羅郡,江河湖海之神,水國泉扃之眾。兩班醮筵森列,合殿官將威儀。香騰瑞靄,千枝畫燭流光;花簇錦筵,百盞銀燈散彩。天地亭,左右金童玉女,對對高張羽蓋;玉帝堂,兩邊執盂捧劍,重重密布幢旛。風清三界步虛聲,月冷九天乘沆瀣。金鐘撞處,高功來進奏虛皇;玉佩鳴時,多講登壇朝玉帝。絳綃衣,星辰燦爛;美蒙冠,金碧交加。監壇神將猙獰,直日功曹猛勇。道眾齊宣寶懺,上瑤臺酌水獻花;真人密誦靈章,按法劍踏罡步斗。青龍隱隱來黃道,白鶴翩翩下紫宸。」

西門慶剛遶壇拈香下來,被左右就請到松鶴軒閣兒裡,地鋪錦毯,爐焚獸炭,那裡坐去了。不一時,應伯爵、謝希大來到。唱畢喏,每人封了一星折茶銀子,說道:「實告,要送些茶兒來。路遠,這些微意,權為一茶之需。」西門慶也不接,說道:「奈煩!自恁請你來陪我坐坐,又幹這營生做什麼?吳親家這裡點茶,我一總多有了,不消拏出來了。」那應伯爵連忙又唱喏說:「哥,真個俺每還收了罷?」因望着謝希大說道:「都是你幹這營生,我說哥不受,拏出來倒惹他訕兩句好的!」良久,吳大舅、花子油都到了,每人兩盒細茶食來點茶。西門慶都令吳道官收了。

吃畢茶,一同擺齋,放了兩張卓。卓上堆的鹹食齋饌,點心湯飯,甚是豐潔。西門慶寬去衣服,同吃了早齋。原來吳道官叫了個說書的,說西漢評話鴻門會。吳道官發了文書,走來陪坐,問:「哥兒今日來不來?」西門慶道:「正是小頑還小哩,房下恐怕路遠諕着他,來不的。到午間,拿他穿的衣服來,三寶面前攝受過,就是一般。」吳道官道:「小道也是這般計較最好。」西門慶道:「別的倒也罷了,他是有些小膽兒。家裡三四個丫鬟,連養娘輪流看視,只是害怕,貓狗都不敢到他根前。」吳大舅道:「孩兒們好容易養活大!」正說着,只見玳安進來說:「裡邊桂姨、銀姨使了李銘、吳惠送茶來了。」西門慶道:「叫他進來。」李銘、吳惠兩個拿着兩個盒子,跪下。揭開,都是頂皮餅 、松花餅 、白糖萬壽糕 、玫瑰搽穰捲兒 ,西門慶俱令吳道官收了。因問李銘:「你每怎得知道,今日我在這裡打醮?」李銘道:「小的今早辰路見陳姑夫騎頭口,問來,纔知道爹今日在此做好事。歸家告訴桂姐,三媽說:『還不快買禮去!』旋約了吳銀姐纔來了。多上覆爹,本當親來,不好來得。這盒粗茶兒與爹賞人罷了。」西門慶分付:「你兩個等着吃齋。」吳道官一面讓他二人下去,自有坐處,連手下人多飽食一頓。話休饒舌,到了午朝拜表畢,吳道官預備了一張大插卓,簇盤定勝 ,高頂方糖菓品,各樣托葷蒸碟鹹食素饌,點心湯飯,又有四十碟碗;又是一罈金華酒 ,哥兒的一頂黑青段子綃金道髻,一件玄色紵絲道衣,一件綠雲段小襯衣,一雙白綾小襪,一雙青潞紬納臉小履鞋,一根黃絨線縧,一道三寶位下的黃線索,一道子孫娘娘面前紫線索,一付銀項圈條脫,刻着「金玉滿堂,長命富貴。」一道朱書辟非黃綾符,上書着「太乙司命,桃延合唐。」八字,就扎在黃線索上,都用方盤盛着。又是四盤美菓,擺在卓上。差小童經袱內包着宛紅布經疏,將三朝做過法事,一一開載節次,請西門慶過了目,方纔裝入盒擔內,共約八抬,送到西門慶家。西門慶甚是歡喜,快使棋童兒家去,賞了道童兩方手帕,一兩銀子。且說那日是潘金蓮生日,有吳大妗子、潘姥姥、楊姑娘、郁大姐,都在月娘上房坐的。見廟裡送了齋來,又是許多羹菓,插卓禮物,擺了四張卓子還擺不下,都亂出來觀看。金蓮便道:「李大姐,你還不快出來看哩,你家兒子師父廟裡送來了。又有許多他的小道冠髻,道衣兒;噫!你看,又是小履鞋兒。」孟玉樓又走向前,拿起來手中看,說道:「大姐姐,你看道士家也精細的!這小履鞋,白綾底兒,都是倒扣針兒,方勝兒,綃的這雲兒又且是好。我說他敢有老婆?不然,怎的扣捺的恁好針腳兒?」吳月娘道:「沒的說,他出家人那裡有老婆?想必是顧人做的。」潘金蓮接過來,說:「道士有老婆!相王師父和大師父會挑的好汗巾兒,莫不是也有漢子?」王姑子道:「道士家掩上個帽子,那裡不去了?似俺這僧家,行動就認出來。」金蓮說道:「我聽得說,你住的觀音寺,背後就是玄明觀。常言道:『男僧寺,對着女僧寺,沒事也有事!』」月娘道:「這六姐好恁囉說白道的!」金蓮道:「這個是他師父與他娘娘寄名的紫線瑣,又是這個銀脖項符牌兒,上面銀打的八個字,帶着且是好看。背面墜着他名字,吳什麼元?」棋童道:「此是他師父起的法名,吳應元。」金蓮道:「這是個『應』字。」叫道:「大姐姐,道士無禮!怎的把孩子改了他姓了?」月娘道:「你看不知禮!」因使李瓶兒:「你去抱了你兒子來,穿上這道衣,俺每瞧瞧好不好?」李瓶兒道:「他纔睡下,又抱他出來?」金蓮道:「不妨事,你揉醒他。」那李瓶兒真個去了。這潘金蓮識字,取過紅布袋兒,扯出送來的經疏,看上面西門慶底下,同室人吳氏,傍邊只有李氏,再沒別人,心中就有幾分不忿,拏與眾人瞧:「你說,賊三等兒九格的強人!你說他偏心不偏心?這上頭只寫着生孩子的,把俺每都是不在數的,都打到贅字號裡去了!」孟玉樓問道:「有大姐姐沒有?」金蓮道:「沒有大姐姐,倒好笑!」月娘道:「也罷了,有了一個,也多是一般。莫不你家有一隊伍人,也多寫上,惹的道士不笑話麼?」金蓮道:「俺每都是劉湛兒鬼兒麼?比那個不出材的?那個不是十個月養的哩!」正說着,李瓶兒從前邊抱了官哥兒,李嬌兒道:「拿過衣服來,等我替哥哥穿。」李瓶兒抱着,孟玉樓替他戴上道髻兒,套上頂牌,和兩道索。諕的那孩子只把眼兒閉着,半日不敢出氣兒。王樓把道衣替他穿上。吳月娘分付李瓶兒:「你把這經疏納個阡張頭兒,親往後邊佛堂中,自家燒了罷。」那李瓶兒去了。金蓮見玉樓抱弄孩子,說道:「穿着這衣服,就是個小道士兒。」金蓮接過來說道:「什麼小道士兒,倒好相個小太乙兒!」被月娘正色說了兩句,便道:「六姐,你這個什麼話!孩兒們上,快休恁的!」那金蓮訕訕的不言語了一回。那孩子穿着衣服害怕,就哭起來。李瓶兒走來連忙接過來,替他脫衣裳時,就扯了一抱裙奶屎。孟玉樓笑道:「好個吳應元,原來拉屎也有一托盤!」月娘進忙教小玉拿草紙替他抹。不一時,那孩子就磕伏在李瓶兒懷裡睡着了。李瓶兒道:「小大哥原來困了,媽媽送你到前邊睡去罷。」吳月娘一面把卓面多散了,請大妗子、楊姑娘、潘姥姥眾人出來吃齋。看看晚來。原來初八日,西門慶因打醮,不用葷酒,潘金蓮晚夕就沒曾上的壽。直到今晚來家,就與他遞酒。來到大門站立。不想等到日落時分,只見陳經濟和玳安自騎頭口來家。潘金蓮問:「你爹來了?」經濟道:「爹怕來不成了。我來時,醮事還未了,纔拜懺,怕不弄到起更。道士有個輕饒素放的?還要謝將吃酒。」金蓮聽了,一聲兒沒言語,使性子回到上房裡,對月娘說:「賈瞎子傳揉,乾起了個五更;隔牆掠肝,能死心塌地?兜肚斷了帶子,沒得絆了!剛纔在門首站了一回,只見陳姐夫騎了頭口來了;說爹不來了,醮事還沒了,先打發他來家。」月娘道:「他不來罷,咱每自在。晚夕聽大師父、王師父說因果唱佛曲兒。」正說着,只見陳經濟掀簾進來,已帶半酣兒,說:「我來與五娘磕頭。」問大姐:「有鍾兒?尋個兒,篩酒與五娘遞一鍾兒。」大姐道:「那裡尋鍾兒去?只恁與五娘磕個頭兒,到這回等我遞罷。你看他醉腔兒!恰好今日打醮,只好了你,吃的恁憨憨的來家!」月娘便問道:「你爹真個不來了?玳安那奴才沒來?」陳經濟道:「爹見醮事還沒了,恐怕家裡沒人,先打發我來了。留下玳安在那裡答應哩。道士再三不肯放我,強死強活,拉着吃了兩三大鍾酒纔來了。月娘問:「今日有那幾個在那裡?」經濟道:「今日有大舅和門外花大舅、應二叔和謝三叔、李銘,又有吳惠兩個小優兒。夜黑不知纏到多咱晚。今日只吳大舅來了,門外花大舅教爹留住了,也是過夜的數。」金蓮沒見李瓶兒在根前,便道:「陳姐夫,連你也叫起花大舅來,是那們兒親?死了的知道罷了!你叫他李大舅纔是,怎叫他花大舅?」經濟道:「五娘,你老人家,鄉里姐姐嫁鄭恩,睜着個眼兒,閉着個眼兒。早出兒子,不知他什麼帳兒,只是夥裡分錢就是了。」大姐道:「賊囚根子!快磕了頭,趁早與我外頭挺去,又口裡恁汗邪胡說了!」陳經濟于是請金蓮轉上,踉踉蹌蹌磕了四個頭,往前邊去了。不一時,房中掌上燈燭,放下卓兒,擺上菜兒,請潘姥姥、楊姑娘、大妗子與眾人來了。金蓮遞了酒,打發坐下,吃了麵。吃到酒闌,收了家活,抬了卓出去。月娘分付小玉把儀門關了,炕上放下小卓兒。眾人圍定,兩個姑子在正中間,焚下香,秉着一對蠟燭,都聽他說因果。先是大師父說道:

「蓋聞大藏經中講說一段佛法,乃是西天第三十二祖下界,降生東土傳佛心印。昔日唐高宗天子咸亨三年,中夏記是不題。卻說嶺南鄉泡渡村有一張員外,家豪大富,廣有金銀,呼奴使婢。員外所取八個夫人,朝朝快樂,日日奢華。貪戀風流,不思善事。忽的一日出門遊翫,見一夥善人,馱載香油 細米等物,人人稱念佛號。向前便問:『你這些善人何往?』內中一人答曰:『一者打齋,二者聽經。』員外又問:『你等打齋聽經,有何功德?』眾人言說:『人生在世,佛法難聞,人身難得。法華經云說的好,若人有福,曾供養佛。今生不捨,來生榮華富貴。從何而來?古人云:龍聽法而悟道,蟒聞懺以生天。何況人乎?』張員外到家,便叫安童:『去後房請出你八個奶奶來。』不一時,都到堂前。員外說:『婆婆,我今黃梅寺修行去,把家財分作八分,各人過其日月。想你我如今只顧眼前快樂,不知身後如何?若不修行,求出火炕,定落三塗五苦。』有夫人聽說,便道:「員外,你八寶羅漢之體,有甚業障?比不的俺女流之輩,生男長女,觸犯神祇,俺每業重。你在家裡修行,等俺八個替你耽罪。你休要去罷!』」正是:「婆婆將言勸夫身,  員外冷笑兩三聲。」

大師父說了一回,該王姑子接偈。月娘、李嬌兒、孟玉樓、潘金蓮、孫雪娥、李瓶兒、西門大姐并玉簫多齊聲接佛。王姑子念道:

「說八個眾夫人要留員外,  告丈夫休遠去在家修行,

你如今下狠心撇下妻子,  痛哭殺兒和女你也心疼!

閃得俺姊妹們無處歸落,  好教我一個個怎過光陰?

從小兒做夫妻相隨到老,  半路里丟下俺倚靠何人?

兒扯爺女扯娘搥胸跌腳,  一家兒大共小痛哭傷情。」

〔金字經〕

「夫人聽說淚不乾,苦勸員外莫歸山。顧家園,兒女永團圓;休遠去,在家修行都一般。」

(白文)

「員外便說:『多謝你八個夫人,我明白死在陰司,你們替我耽罪。我今與你們遞一鍾酒,明日好在閻王面前承當。』飲酒中間,員外設了一計:『夫人與我把燈剔一剔。』員外哄的夫人剔燈,一口把燈吹死。諕的八個夫人失色,連忙叫梅香:『快點燈來!』員外取出鋼刀劍,諕殺八個眾夫人。」

又偈:

「老員外喚梅香把燈點起,  將鋼刀拿在手指定夫人,

那一個把明燈一口吹死,  圖家財害我命改嫁別人,

若不說一劍去這頭落地,  一個個心害怕倒在埃塵。

有八個老夫人慌忙跪下,  告員外你息怒饒俺殘生,

你分明一口氣把燈吹死,  吃幾鍾紅面酒拏劍殺人,

你若還殺了俺八個夫人,  到陰司告閻君取你真魂。」

「員外冷笑,便叫八個夫人:『你哄我當身吹燈不認,如何認我陰司耽罪?八個女流之輩,倒哄男身笑殺年高有德人。』說的八個夫人閉口無言。員外想人生富貴,都是前生修來,便叫安童:『連忙與我裝載數車香油 米麵,各樣菜蔬錢財等物,我往黃梅山裡打齋聽經去也。』」

〔金字經〕

「夫人聽我說根源,梵王天子棄江山。不貪戀要結萬人緣;多全捨,萬古標名在世間。」

「員外今日修行去,  親戚鄰人送起程。」

念了一回,吳月娘道:「師父餓了,且把經請過,吃些甚麼?」一面令小玉安排了四碟素菜兒,兩碟鹹食兒,四碟兒糖,薄脆蒸酥,菊花餅,扳搭饊子,請大妗子、楊姑娘、潘姥姥陪着二位師父用一個兒。大妗子說:「俺每不當家的,都剛吃的飽。教楊姑娘陪個兒罷。他老人家又吃着個齋。」月娘連忙用小描金碟兒,每樣揀了個點心,放在碟兒裡,先遞與兩位師父,然後遞與楊姑娘,說道:「你老人家陪二位請些兒。」婆子道:「我的佛爺,不當家!老身吃的可勾了。」又道:「這碟兒裡是燒骨禿 ,姐姐你拿過去。只怕錯揀到口裡。」把眾人笑的了不得。月娘道:「奶奶,這個是頭裡廟上送來的,托葷鹹食,你老人家只顧用,不妨事。」楊姑娘道:「既是素的,等老身吃。老身乾淨眼花了,只當做葷的來!」正吃着,只見來興兒媳婦子惠香走來。月娘道:「賊臭肉,你也來做什麼?」惠香道:「我也來聽唱曲兒。」月娘道:「儀門關着,你打那裡進來了?」玉簫道:「他在廚房封火來。」月娘道:「嗔道恁王小的鼻兒烏嘴兒黑的,成精鼓搗來聽什麼經!」當下眾丫鬟婦女圍定兩個姑子,吃了茶食,收過家活去,搽抹經卓乾淨。月娘從新剔起燈燭來,炷了香。兩個姑子打動擊子兒,又高念起來:從張員外在黃梅山寺中修行,白日長跪聽經,夜晚參禪打坐。四祖禪師觀見他不是凡人,定是個真僧出世,問其鄉貫、住處,姓甚名誰?員外具說前因一遍:弟子把家財妻子棄了,實為生死出家。四祖收留座下,做了徒弟。白日教他栽樹,夜晚樁米。六年苦行已滿,驚動護法韋馱尊天驚覺四祖,教他尋安身立命之處,與了他三座寶貝,斗蓬、簑衣、灣棗棍往南去濁河邊投胎奪舍,尋房兒居住,三百六十日經果圓成。你如今年紀高大,房兒壞了,傳不得真妙法,度脫不得眾生。直說到千金小姐、姑嫂兩個,在濁河邊洗濯衣裳,見一僧人借房住,不合答了他一聲,那老人就跳下河去了。潘金蓮熬的磕困上來,就往房裡睡去了。少頃,李瓶兒房中綉春來叫,說:「官哥兒醒了。」也去了。只剩下李嬌兒、孟玉樓、潘姥姥、孫雪娥、楊姑娘、大妗子,守着聽到河中漂過一夥大鱗桃來,小姐不合吃了,歸家有孕,懷胎十月。王姑子唱了一個耍孩兒:

「一靈真性投肚內,這個消息誰得知?人人不識西來意,呀的一聲孕男女。認的娘生鐵面皮,纔得見光明際。崑崙頂上轉大千沙界,古彌陀分南北東西。」

說:「千金小姐來到嫂子房中,『吃咱兩個曾在濁河邊洗衣見了那老人,問咱借房兒住,他如何跳在河內,諕的我心中驚怕。又吃了一個仙桃,我如今心頭膨悶,好生疑悔腹中成其身孕!』正是:

「十月腹中母懷胎,  千金小姐淚盈腮。」

「千金說在綉房成其身孕,  心中悔無可奈忍氣吞聲,

一個月懷胎著如同露水,  兩個月懷胎著纔卻朦朧,

三個月懷胎著纔成血餅,  四個月懷胎著骨節纔成,

五個月懷胎著纔分男女,  六個月懷胎著長出六根,

七個月懷胎著生長七竅,  八個月懷胎著著相成人,

九個月懷胎著看看大滿,  十個月母腹中准備降生。」

「五祖投胎在母腹中,因為度眾生,裟婆男女不肯回心。古佛下界轉凡身,借胎出殼,久後度母到天宮。」

「五祖一佛性,  投胎在腹中;

權住十個月,  轉凡度眾生。」

念到此處,月娘見大姐也睡去了,大妗子〈扌歪〉在月娘裡間床上睡着了,楊姑娘也打起欠呵來,卓上蠟燭也點盡了兩根。問小玉:「這天有多咱晚了?」小玉道:「已是四更天氣,雞鳴叫。」月娘方令兩位師父收拾經卷。楊姑娘便往玉樓房裡去了。郁大姐在後邊雪娥房裡宿歇。只有兩個姑子,月娘打發大師父和李嬌兒一處睡去了。王姑子和月娘在炕上睡。兩個還等着小玉頓了一甌子茶吃了,纔睡。大妗子在裡間床上,和玉簫睡。月娘因問王姑:「後來這五祖長大了,怎生成了正果?」王姑子道:「這裡爺娘見他有身孕,教他哥哥祝虎把千金小姐趕將出去,要行殺害。多虧祝龍慈心,放他逃生,走在垂楊樹下自縊。驚動天上太白李金星,教他尋茶討飯,隨緣度日。不覺十月滿足,來到仙人庄神廟裡,降生下五祖。紫霧紅光罩滿了廟堂。小姐見孩兒生下,就盤膝端坐,心中害怕,不比尋常。後又到天喜村王員外家場裡宿歇。場中火起,拏起見員外。見小姐顏色,就要留下做小。子母兩個下拜,登時把員外、夫人多拜死了。家奴院公,拏住子母。後員外甦省過,說道:『只怕是好人。』留在家中,養活六歲,五祖方說話。不由為母的,一直走到濁河邊枯樹,取了三庄寶貝,逕往黃梅寺聽四祖說法,遂成正果。後邊度脫母親生天。」月娘聽

了,越發好信佛法了,有詩為證:

「聽法聞經怕無常,  紅蓮舌上放毫光;

何人留下禪空話,  留取尼僧化稻糧。」

畢竟未知後來如何,且聽下回分解:

