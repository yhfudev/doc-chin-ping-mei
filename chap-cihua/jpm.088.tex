%# -*- coding: utf-8 -*-
%!TEX encoding = UTF-8 Unicode
%!TEX TS-program = xelatex
% vim:ts=4:sw=4
%
% 以上设定默认使用 XeLaTex 编译,并指定 Unicode 编码,供 TeXShop 自动识别

%第八十八回 
\chapter{潘金蓮托夢守備府\KG 吳月娘佈福募緣僧}

「上臨之以天鑒,  下察之以地祗,

明有王法相制,  暗有鬼神相隨,

忠直可存於心,  喜怒戒之在氣;

為不節而忘家,  因不廉而失位,

勸君自警平生,  可笑可驚可畏。」

話說武松殺了婦人、王婆,劫去財物,逃上梁山為盜去了。都表王潮兒去街上叫保甲,見武松家前後門都不開。又王婆家被劫去財物,房中衣服丟的地下橫三豎四,就知是武松殺死二命,劫取財物而去。未免打開前後門,見血瀝瀝兩個死屍倒在地下,婦人心肝五臟,用刀插在後樓房簷下。迎兒倒扣在房中。問其故,只是哭泣。次日早衙,呈報到本縣。殺人兇刃,都拿放在面前。本縣新任知縣也姓李,雙名昌期,乃河北真定府棗強縣人氏。聽見殺人公事,即委差當該吏典,拘集兩鄰保甲,并兩家苦主王潮、迎兒,眼同招出,當街如法檢驗。生前委被武松因忿帶酒,殺潘氏、王婆二命。叠成交案,就委地方保甲瘞埋看守。掛出榜文,四廂差人跟尋,訪拿正犯武松。有人首告者,官給賞銀五十兩。守備府中張勝、李安,打着一百兩銀子到王婆家,看見王婆、婦人,俱已被武松殺死,縣中差人檢屍,捉拿兇犯。二人回報到府中。春梅聽見婦人死了,整哭了兩三日,茶飯都不吃。慌了守備,使人門前叫了調百戲的貨郎兒進去,要與他觀看,只是不喜歡。日逐使張勝、李安打聽拿住武松正犯,告報府中知道,不在話下。按下一頭,卻表陳經濟前往東京取銀子,一心要贖金蓮,成其夫婦。不想走到半路,撞見家人陳定從東京來,告說家爺病重之事:「奶奶使我來請大叔往家去,囑托後事。」這經濟一聞其言,兩程做一程,路上儹行。有日到東京,他姑夫張世廉家。張世廉已死,止有姑娘見在。他父親陳洪,已是沒了三日光景,滿家帶孝。經濟參見他父親靈座,與他母親張氏并姑娘磕頭。張氏見他長成人,母子哭做一處,通同商議:「如今一則以喜,一則以憂。」經濟便道:「如何是喜?如何是憂?」張氏道:「喜者,如今且喜朝廷冊立東宮,郊天大赦;憂則不想你爹爹得病,死在這裡,你姑夫又沒了,姑娘守寡,這裡住着,不是常法。方使陳定叫將你來,和你打發你爹爹靈柩回去,葬埋鄉井,也是好處。」這經濟聽了,心內暗道:「這一會發送裝戴靈柩,家小粗重上車,少說也得許多日期躭擱,卻不誤了娶六姐?不如此這般,先誆了兩車細軟箱籠家去,待娶了六姐,再來搬取靈柩不遲。」一面對張氏說道:「如今隨路盜賊,十分難走。假如靈柩家小箱籠,一同起身,若說數輛車馱,未免起眼。倘遇小嘍囉怎了?寧可躭遲不躭錯。我先押兩車細軟箱籠家去,收拾房屋。母親後和陳定家眷,跟父親靈柩,過年正月間起身回家,寄在城外寺院,然後做齋念經,入墳安葬,也是不遲。」張氏終是婦人家,不合一時聽信經濟巧言念轉,先打點細軟箱籠,裝戴兩大車,上插旗號,扮做香車,從臘月初一日東京起身,不上數日,到了山東清河縣家門首,對他母舅張團練說:「父親已死,母親押靈車不久就到。我押了兩車行李,先來收拾打掃房屋。」他母舅聽說:「既然如此,我須搬回家便了。」一面就令家人搬家活,騰出房子來。這經濟見母舅搬去,滿心歡喜說:「且得冤家離眼前,落得我娶六姐來家,自在受用。我父親已死,我娘又疼我,先休了那個淫婦,然後一紙狀子,把俺丈母告到官,追要我寄放東西,誰敢道個不字?又挾制俺家充軍人數不成?」正是:

「人便如此如此,  天理不然不然。」

這經濟早攛掇他母舅出來,然後打了一百兩銀子在腰裡,另外又袖着十兩謝王婆。來到紫石街王婆門首。可霎作怪,只見門前街旁,埋着兩個屍首,上面兩桿鎗交叉,上面挑着個燈籠。門首掛着一張手榜,上書:「本縣為人命事,兇犯武松殺死潘氏、王婆二命,有人捕獲首告官司者,官給賞銀五十兩。」這經濟仰頭還大看了,只見從窩舖中鑽出兩個人來,喝聲道:「甚麼人?看此榜文做甚?見今正身兇犯捉拿不着,你是何人?」大扠步便來捉獲。這經濟慌的奔走不迭,恰然走到石橋下酒樓邊,只見一個人頭戴萬字巾,身穿青衲襖,隨後趕到橋下,說道:「哥哥,你好大膽,平白在此看他怎的?」這經濟扭回頭看時,卻是一個識熟朋友,鐵指甲楊二郎。二人聲喏。楊二哥道:「哥哥,一向不見,那裡去來?」經濟便把東京父死往回之事,告說一遍:「恰才這殺死婦人,是我丈人的小潘氏,不知他被人殺了。適纔見了榜文,方知其故。」楊二郎告道:「是他小叔武松,充配在外,遇赦回還。不知因甚殺了婦人,連王婆子也不饒。他家還有個女孩兒,在我姑夫姚二郎家養活了三四年,昨日他叔叔殺了人,走的不知下落。我姑夫將此女縣中領出,嫁與人為妻小去了。見今這兩個屍首,日久只顧埋着,只是苦了地方保甲看守,更不知何年月日纔拿住兇犯武松!」說畢,楊二郎招了經濟上酒樓飲酒:「與哥哥拂塵。」這經濟見那人已死,心中轉痛不下,那裡吃得下酒?約莫飲勾三盃,就起身下樓,作別來家。到晚夕,買了一陌錢布,在紫石街離王婆門首遠遠的石橋邊,題着婦人:「潘六姐,我小兄弟陳經濟,今日替你燒陌錢布。皆因我來遲了一步,誤了你性命!你活時為人,死後為神。早保佑捉獲住仇人武松,替你報仇雪恨!我在法場上,看着剮他,方趁我平生之志!」說畢哭泣,燒化了錢布。經濟回家,關了門戶,走歸房中,恰纔睡着,似睡不睡,夢見金蓮身穿素服,一身帶血,向經濟哭道:「我的哥哥,我死的好苦也!實指望與你相處在一處,不期等你不來,被武松那廝害了性命。如今陰司不收,我白日遊遊蕩蕩,夜歸向各處尋討漿水。適間蒙你送了一陌錢布與我。但只是仇人未獲,我的屍首埋在當街。你可念舊日之情,買具棺材盛了葬埋,免得日久暴露。」經濟哭道:「我的姐姐,我可知要葬埋你,但恐西門慶家中,我丈母那無仁義的淫婦知道,他自恁賴我,倒趁了他機會。姐姐,你須往守備府中對春梅說知,教他葬埋你身屍便了。」婦人道:「剛纔奴到守備府中,又被那門神戶尉攔攩不放。奴須慢慢再哀告他則個。」經濟哭着,還要拉着他說話,被他身上一陣血腥氣,撒手掙脫,卻是南柯一夢。枕上聽那更鼓時,正打三更二點,說道:「怪哉!我剛纔分明夢見六姐向我訴告衷腸,教我葬埋之意,又不知甚年何日拿住武松,是好傷感人也!」正是:

「夢中無限傷心事,  獨坐空房哭到明!」

不說經濟這裡也打聽武松不題。卻表縣人訪拿武松,約兩個月有餘,捕獲不着。已知逃遁梁山為盜,地方保甲鄰佑,呈報到官,所有兩座屍首,相應責令家屬領埋。王婆屍首,便有他兒子王潮,領的埋葬。止有婦人身屍,無人來領。卻說府中春梅,兩三日一遍,使張勝、李安來縣中打聽,回去只說:「兇犯還未拿住,屍首照舊埋葬,地方看守,無人敢動。」直挨過年,正月初旬時節,忽一日晚間,春梅作一夢,恍恍惚惚,夢見金蓮雲髻蓬鬆,渾身是血,叫道:「龐大姐,我的好姐姐,奴死的好苦也!好容易來見你一面,又被門神把住,嗔喝不敢進來。今仇人武松已是逃走脫了。所有奴的屍首,在街暴露日久,風吹雨洒,雞犬作踐,無人領埋。奴舉眼無親,你若念舊日母子之情,買具棺材把奴埋在一處。奴死在陰司,口眼皆閉。」說畢,大哭不止。春梅扯住他,還要再問他別的話。被他睜開,撒手驚覺,卻是南柯一夢。從睡夢中直哭醒來,心內猶疑不定。次日,叫進張勝、李安,分付:「你二人去縣前打聽,那埋的婦人婆子屍首,還有無有?」張勝、李安諾去了。不多時,走來回報:「正犯兇身,已逃走脫了。所有殺死身屍,地方看守,日久不便,相應責令各人家屬領埋。那婆子屍首,他兒子招領的去了。還有那婦人,無人來領,還埋在街心。」春梅道:「既然如此,我有庄事兒累你二人,替我幹得來,我還重賞你。」二人跪下:「小夫人說那裡話!若肯在老爺前抬舉小人一二,就消受不了。雖赴湯跳火,敢說不去?」春梅走到房中,拿出十兩銀子、兩疋大布,委付二人:「這死的婦人,是我一個嫡親姐姐,嫁在西門慶家。今日出來,被人殺死。你二人休教你老爺知道,拿這銀子替我買一具棺材,把他裝殮了,抬出城外,擇方便地方,埋葬停當,我還重賞你。」二人道:「這個不打緊,小人就去。」李安說:「只怕縣中不教你我領屍怎了?須拿老爺個帖兒,下到縣官纔好。」張勝道:「只說小夫人是他妹子,嫁在府中,那縣官不敢不依,何消帖子?」於是領了銀子,來到班房內。張勝便向李安說:「想必這死的婦人,與小夫人曾在西門慶家做一處,相結的好,今日方立這等為他費心。相着死了時,整哭了三四日不吃飯,直教老爺門前叫了調百戲貨郎兒,調與他觀看,還不喜歡。今日他無親人領去,小夫人豈肯不葬埋他?咱每若替他幹得此事停當,早晚他在老爺跟前,只方便你我,就是一點福星!見今老爺百依百隨,聽他說話。正經大奶奶、二奶奶,且打靠他!」說畢,二人拿銀子到縣前,遞了領狀,就說他妹子在老爺府中,來領屍首。使了六兩銀子,合了一具棺木,把婦人屍首掘出,把心肝填在肚內。頭用線縫上,用布裝殮停當,裝入材內。張勝說:「就埋在老爺香火院城南永福寺裡,那裡有空閑地。」葬埋了,回小夫人話去,叫了兩名伴當,抬到永福寺,對長老說:「宅內小夫人親。」長老不敢怠慢,就在寺後揀一塊空白楊樹下,那裡葬埋已畢,走來宅內回春梅話說:「除買棺材裝殮,還剩四兩銀子。」交割明白。春梅分付:「多有起動你二人,將這四兩銀子,拿二兩與長老道堅,教他早晚替他念些經懺,超度他生天。」又拿出一大瓶酒、一腿豬肉、一腿羊肉,「這二兩銀子,你每人將一兩家中盤纏。」二人跪下,那裡敢接。只說:「小夫人若肯在老爺面前抬舉,小人消受不了。這些小勞,豈敢接受銀兩?」春梅道:「我賞你不收,我就惱了。」二人只得磕頭領了出來,兩個班房吃酒,甚是稱念小夫人好處。次日,張勝送銀子與長老念經。春梅又與五兩銀子,買布與金蓮燒,俱不在話下。卻說陳定從東京戴靈柩家眷,到清河縣城外,把靈柩寄在永福寺,待的念經發送歸葬墳內。經濟在家,聽見母親張氏家小車輛到了,父親靈柩寄停在城外永福寺,收卸行李已畢,與張氏磕了頭。張氏怪他:「就不去接我一接。」經濟只說:「心中不快,家裡無人看守。」張氏便問:「你舅舅怎的不見?」經濟道:「他見母親到了,連忙搬回家去了。」張氏道:「且教你舅舅住着,慌搬去怎的?」一面他母舅張團練來看他姐姐。姊妹抱頭而哭,置酒敘話,不必細說。次日,他娘張氏,早使經濟拿五兩銀子,幾陌金銀錢布,往門外與長老,替他父親念經。正騎頭口街上走,忽撞遇他兩個朋友。陸大郎、楊大郎,下頭口聲喏。二人問道:「哥哥往那裡去?」經濟悉言:「先父靈柩,寄在門外寺裡。明日廿日是終七,家母使我送銀子與長老做齋念經。」二人道:「兄弟不知老父靈柩到了,有失弔問。」因問:「幾時發引安葬?」經濟道:「也只在一二日之間,念畢經,入墳安葬。」說罷,二人舉手作別。這經濟又叫住,因問楊大郎:「縣前我丈人的小,那潘氏屍首怎不見?被甚人領的去了?」楊大郎便道:「半月前,地方因捉不着武松,稟了本縣相公,令各家領去葬埋。王婆是他兒子領去。上有婦人屍首,丟了三四日,被守備府中買了一口棺木,差人抬出城外永福寺那裡葬去了。」經濟聽了,就知是春梅在府中收葬了他屍首。因問二郎:「城外有幾個永福寺?」二郎道:「本自南門外只一個永福寺,是周秀老爺香火院。那裡有幾個永福寺來?」經濟聽了暗喜:「就是這個永福寺,也是緣法湊巧,喜得六姐亦葬在此處。」一面作別二人,打頭口出城,逕到永福寺。見到長老,且不說念經之事,就先問長老道堅:「此處有守備府中新近葬的一個婦人,在那裡?」長老道:「就在寺後白楊樹下,說是宅內小夫人的姐姐。」這經濟且不參見他父親靈柩,先拿錢布祭物,到於金蓮墓上與他祭了,燒化錢布,哭道:「我的六姐,你兄弟陳經濟敬來與你燒一陌錢布。你好處安身,苦處用錢。」祭畢,然後纔到方丈內他父親靈柩跟前,燒布祭祀。遞與長老經錢,教他二十日請八眾禪僧,念斷七經。長老接了經襯,備辦齋供。經濟來家,回了張氏話。二十日都去寺中拈香,擇吉發引,把父親靈柩,歸到祖塋。安葬已畢,來家母子過日不題。卻表吳月娘,一日二月初旬,天氣融和。孟玉樓、孫雪娥、西門大姐、小玉,出來大門首站立,觀看來往車馬,人烟熱鬧。忽見一簇男女,跟着個和尚,生的十分胖大。頭頂三尊銅佛,身上抅着數枝燈樹,杏黃架娑風兜袖,赤腳行來泥沒踝。自言說是五臺山戒壇上下來的行腳僧,雲遊到此,要化錢糧,蓋造佛殿。當時古人,有幾句讚的這行腳僧好處:

「打坐參禪,講經說法。鋪眉苦眼,習成佛祖家風;賴教求食,立起法門規矩。白日裡賣杖搖鈴,黑夜間舞鎗弄棒。有時門首磕光頭,餓了街前打响嘴。空色色空,誰見眾生離下土;去來來去,何曾接引到西方!」

那和尚見月娘眾婦女在門首,向前道了個問訊,說道:「在家老菩薩施主,既生在深宅大院,都是龍華一會上人。貧僧是五臺山下來的,結化善緣,蓋造十王功德三寶佛殿。仰賴十方施主菩薩,廣種福田,捨資財共成勝事,修來生功果。貧僧只是挑腳漢。」月娘聽了他這般言語,便喚小玉往房中取一頂僧帽、一雙僧鞋、一弔銅錢、一斗白米。原來月娘平昔好齋僧布施,常時閒中,發心做下僧帽、僧鞋,預備布施。這小玉取出來,月娘分付:「你叫那師父近前來,布施與他。」這小玉故做嬌態,高聲叫道:「那變驢的和尚,還不過來。俺奶奶布施與你這許多東西,還不磕頭哩!」月娘便罵道:「怪墮業的小臭肉兒,一個僧家,是佛家弟子。你有要沒緊,恁謗他怎的?不當家化化的,你這小淫婦兒,到明日不知墮多少罪業!」小玉笑道:「奶奶,這賊和尚,我叫他,他怎的把那一雙賊眼,眼上眼下打量我?」那和尚雙手接了鞋、帽、錢、米,打問訊說道:「多謝施主老菩薩布施布施!」小玉道:「這禿廝好無禮,這些人站着,只打兩個問訊兒,就不與我打一個兒?」月娘道:「小肉兒,還恁說白道黑。他一個佛家之子,你也消受不的他這個問訊!」小玉道:「奶奶,他是佛爺兒子,誰是佛爺女兒?」月娘道:「相這比丘尼姑僧,是佛的女兒。」小玉道:「譬若說相薛姑子、王姑子、太師父,都是佛爺的女兒。誰是佛爺的女婿?」月娘忍不住笑罵道:「這賊小淫婦兒,學的油嘴滑舌,見見就說下道兒去了!」小玉道:「奶奶只罵我。本等這禿和尚,賊眉豎眼的只看我。」孟玉樓道:「他看你,想必認得的,要度脫你去。」小玉道:「他若度我,我就去。」說着,眾婦女笑了一回,月娘喝道:「你這小淫婦兒,專一毀僧謗佛!」那和尚得了布施,頂着三尊佛,揚長去了。小玉道:「奶奶還嗔我罵他,你看這賊禿,臨去還看了我一眼纔去了。」有詩單道月娘修善施僧好處:

「守寡看經歲月深,  私邪空色久違心;

奴身好似天邊月,  不許浮雲半點侵。」

月娘眾人正在門首說話,忽見薛嫂兒提着花箱兒,從街上過來。見月娘眾人,道了萬福。月娘問:「你往那裡去來?怎的影跡兒不來我這裡走走?」薛嫂兒道:「不知我終日窮忙的是些甚麼!這兩日,大街上掌刑張二老爹家,與他兒子娶親,和北邊徐公公做親,娶了他姪兒,也是我和文嫂兒說的親事。昨日三日,擺大酒席,忙的連守備府裡咱家小大姐那裡叫,我也沒去。不知怎麼惱我哩!」月娘問道:「你如今往那裡去?」薛嫂道:「我有庄事,敬來和你老人家說來。」月娘道:「你有話進來說。」一面讓薛嫂兒到後邊上房裡坐下,吃了茶,薛嫂道:「你老人家還不知道,你陳親家從去年在東京得病沒了。親家母叫了姐夫去搬取家小靈柩,從正月來家,已是念經發送墳上安葬畢。我只說你老人家這邊知道,怎不去燒張布兒探望探望?」月娘道:「你不來說,俺這裡怎得曉的?又無人打聽。倒自知道潘家的,吃他小叔兒殺了,和王婆都埋在一處。卻不知如今怎麼了?」薛嫂兒道:「自古生有地兒死有處。五娘他老人家,不因那些事出去了,卻不好來?平日不守本分,幹出醜事來出去了。若在咱家裡,他小叔怎得殺了他?還是仇有頭,債有主!倒還虧了咱家小大姐春梅,越不過娘兒們情腸,差人買了口棺材,領了他屍首葬埋了。不然,只顧暴露着,又拿不着小叔子,誰去管他?」孫雪娥在旁說:「春梅賣在守備府裡,多少時兒,就這等大了!手裡拿出銀子,替他買棺材埋葬。那守備也不嗔,當他甚麼人?」薛嫂道:「耶嚛,你還不知,守備好不喜他!每日只在他房裡歇臥,說一句,依十句。一娶了他,生的好模樣兒,乖覺伶俐,就與他西廂房三間房住,撥了個使女伏侍他。老爺一連在他房裡歇了三夜,替他裁四季衣服。上頭三日吃酒,賞了我一兩銀子,一疋段子。他大奶奶五十歲,雙目不明,吃長齋,不管事。東廂孫二娘,生了小姐,雖故當家,撾着個孩子。如今大小庫房鑰匙,倒都是他拿着。守備好不聽他說話哩!且說銀子,手裡拿不出來?」幾句說的月娘、雪娥都不言了。坐了一回,薛嫂起身。月娘分付:「你明日來我這裡,備一張祭卓、一疋尺頭、一分冥布,你來送大姐與他公公燒布去。」薛嫂兒道:「你老人家不去?」月娘道:「你只說我心中不好,改日望親家去罷。」那薛嫂約定:「你教大姐收拾下等着我,飯罷時候。」月娘道:「你如今到那裡去?守備府中,不去也罷。」薛嫂道:「不去,就惹他怪死了!他使小伴當叫了我好幾遍。」月娘道:「他叫你做甚麼?」薛嫂道:「奶奶,你不知,他如今有四五個月身孕了。老爺好不喜歡,叫了我去,已定賞我。」提着花箱作辭去了。雪娥便說:「老淫婦說沒個行款兒!他賣守備家多少時?就有了半肚子?那守備身邊少說也有幾房頭,莫就興起他來,這等大道!」月娘道:「他還有正景大奶奶,房裡還有一個生小姐的娘子兒哩!」雪娥道:「可又來,到底還是媒人嘴;一尺水,十丈波的!」不因今日雪娥說話,正是:

「從天降下鈎和線,  就地引起是非來。」

有詩為證:

「曾記當年侍主傍,  誰知今日變風光;

世間萬事皆前定,  莫笑浮生空自忙。」

畢竟未知後來如何,且聽下回分解:

