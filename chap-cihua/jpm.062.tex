%# -*- coding: utf-8 -*-
%!TEX encoding = UTF-8 Unicode
%!TEX TS-program = xelatex
% vim:ts=4:sw=4
%
% 以上设定默认使用 XeLaTex 编译,并指定 Unicode 编码,供 TeXShop 自动识别

%第六十二回 
\chapter{潘道士解禳祭燈法\KG 西門慶大哭李瓶兒}


\begin{showcontents}{}



「行藏虛實自家知,  禍福因由更問誰,

善惡到頭終有報,  只爭來早與來遲;

閒中點檢平生事,  靜裡思量日所為,

常把一心行正道,  自然天理不相虧。」

話說西門慶見李瓶兒服藥百般醫治無效,求神、問卜、發課,皆有凶無吉,無法可處。初時李瓶兒還〈門乍〉〈門爭〉著梳頭洗臉,還自己下炕來坐淨桶。次後漸漸飲食減少,形容消瘦,下邊流之不止。那消幾時,把個花朵朵人兒,瘦弱的不好看,也不着的炕了,只在裀褥上舖墊草布,恐怕人進來嫌穢惡,教丫頭燒著下些香在房中。西門慶見他胳膊兒瘦的銀條兒相似,守著在房內哭泣。衙門中隔日去走一走。李瓶兒便道:「我的哥,你還往衙門中去,只怕誤了你公事。我不妨事,只吃下邊流的虧。若得止住不流了,再把口裏放開,吃下些飲食兒就好了。你男子漢,常絆住你在房中守著甚麼?」西慶哭道:「我的姐姐,我見你不好,心中捨不的你。」李瓶兒道:「好傻子!只不死,將來你攔的住那些?」又道:「我要對你說,也沒與你說。我不知怎的,但沒人在房裡,心中只害怕。恰似影影綽綽,有人在我跟前一般。夜裡要便夢見他,恰似好時的拏刀弄扙,和我廝嚷;孩子也在他懷裡,我去奪,反被他推我一交。說他那里又買了房子,來纏了好幾遍,只叫我去。只不好對你說。」西門慶聽了說道:「人死如燈滅。這幾年知道他往那里去了?此是你病的久了,下邊流的你這神虛氣弱了。那裡有甚麼邪魔魍魎,家親外祟?我明日往吳道官廟裡討兩道符來,貼在這房門上,看有邪祟沒有!」說話中間,走到前邊,即差玳安騎頭口往玉皇廟討符去。走到路上,迎見應伯爵和謝希大,下頭口,因問:「你爹在家裡?」玳安道:「爹在家裡。」又問:「你往那裡去?」玳安道:「小的往玉皇廟討符去。」伯爵與謝希大到西門慶家,因說道:「謝子純聽見嫂子不好,諕了一跳,敬來問安。」西門慶道:「這兩日較好些。告訴身上瘦的,通不相模樣了。丟的我上不上,下不下,都怎生樣的!孩子死了,隨他罷了,成夜只是哭,生生憂慮出病兒來了。勸著又不依你,教我有甚法兒處!」伯爵道:「哥,你又使玳安往廟裡做甚麼去?」西門慶悉把李瓶兒房中無人害怕之事,告訴一遍:「只恐有邪祟,教小廝問吳道官那里討兩道符來,貼在房中,鎮壓鎮壓。」謝希大道:「哥,此是嫂子神氣虛弱,那里有甚麼邪祟魍魎來?」伯爵道:「哥若遣邪也不難,門外五岳觀潘道士,他受的是天心五雷法,極遣的好邪,有名喚做潘捉鬼,常將符水救人。哥,你差人請請他來,看看嫂子房裡有甚邪祟,他就知道。你就教他治病,他也治得。」西門慶道:「等討了吳道官符來看。在那里住?沒奈何,你就領小廝騎了頭口,請了他來。」伯爵道:「不打緊,等我去。天可憐見嫂子好了,我就頭著地也走。」說了一回話,伯爵和希大吃了茶,起身自勾當去了。玳安兒討了符來,貼在房中。晚間,李瓶兒還害怕,對西門慶說:「死了的,他剛纔和兩個人來拏我。見你進來,躲出去了。」西門慶道:「你休信邪,不防事。昨日應二哥說,此是你虛極了。他說門外五岳觀有個潘道士,好符水治病,又遣的好邪。我明日早教應二哥去請他來,看你有甚邪祟,教他遣遣。」李瓶兒道:「我的哥哥,你請他早早來。那廝他剛纔發恨而去,明日還來拏我哩!你快些使人請去!」西門慶道:「你若害怕,我使小廝拏轎子接了吳銀兒和你做兩日伴兒。」李瓶兒搖頭兒說:「你不要叫他,只怕誤了他家里勾當。」西門慶道:「叫老馮來伏侍你兩日兒如何?」李瓶兒點頭兒。這西門慶一面使來安往那邊房子裡叫馮媽媽,又不在,鎖了門出去了。對與一丈青說下:「等他來,好歹教他快來宅內,六娘叫他哩。」西門慶一面又差下玳安,明日早起,你和應伯爵往門外五岳觀請潘道士去了。俱不在話下。次日,只見觀音庵王姑子跨著一盒兒粳米 、二十塊大乳餅 、一小盒兒十香瓜茄 來看。李瓶兒見他來,連忙教迎春搊扶起來坐的。王姑子道了問訊,李瓶兒請他坐下,道:「王師父,你自印經時去了,影邊兒通不見你。我恁不好,你就不來看我看兒?」王姑子道:「我的奶奶,我通不知你不好。昨日他大娘使了大官兒到庵里,我纔曉得的。又說印經來,你不知道,我和薛姑子老淫婦,合了一場好氣!與你老人家印了一場經,只替他趕了網兒。背地裡和印經家打了一兩銀子夾帳,我通沒見一個錢兒!你老人家作福,這老淫婦到明日墮阿鼻地獄!為他氣的我不好了,把大娘的壽日都誤了,沒曾來。」李瓶兒道:「他各人作業,隨他罷,你休與他爭執了。」王姑子道:「誰和他爭執甚麼!」李瓶兒道:「大娘好不惱你哩,說你把他受生的經都誤了。」王姑子道:「我的菩薩,我雖不好,敢誤了他的經?在家整誦了一個月受生,昨日纔圓滿了。今日纔來,先到後邊見了他,把我這些屈氣告訴了他一遍。我說不知他六娘不好,沒甚麼,這盒粳米 和些十香瓜,幾塊乳餅 ,與老人家吃粥兒。大娘纔教小玉姐領我來看你老人家。」小玉打開盒兒,與李瓶兒看了,說道:「多謝你費心。」王姑子道:「迎春姐,你把這乳餅 就蒸兩塊兒來,我親看你娘吃些粥兒。」那迎春一面收下去了。李瓶兒分付迎春擺茶來與王師父吃。王姑子道:「我剛纔後邊大娘屋裡吃了些茶。煎些粥米,我看著你吃些粥兒。」不一時,迎春安放卓兒,擺了四樣茶食,打發王姑子吃了。然後拿上李瓶兒粥來,一碟十香甜醬瓜茄 ,一碟蒸的黃霜霜乳餅 ,兩盞粳米粥 ,一雙小牙快,迎春拏着。奶子如意兒在旁拏著甌兒,喂了半日,只呷了兩三口粥兒,咬了一些乳餅兒,就搖頭兒不吃了,教拏過去罷。王姑子道:「人以水食為命。恁煎的好粥兒,你再吃些兒不是?」李瓶兒道:「也得我吃的下去是怎的。」迎春便把吃茶的卓兒掇過去。王姑子揭開被,看李瓶兒身上肌體,都瘦的沒了,諕了一跳,說道:「我的奶奶,我去時你好些了,如何又不好了,就瘦得恁樣的了!」如意兒道:「可知好了哩,娘原是氣惱上起的病。爹請了太醫來看,每日服藥,已是好到七八分了。只因八月內哥兒着了驚諕不好,娘晝夜憂慼,那樣勞碌,連睡也不得睡。實指望哥兒好了,不想沒了。成日着了那哭,又着了那暗氣暗惱在心裡,就是鐵石人也禁不的,怎的不把病又犯了!是人家有些氣惱兒,對人前分解分解也還好。娘又不出語,着緊問還不說哩!」王姑子道:「那討氣來?你爹又疼他,你大娘又散他。在右是五六位娘,端的誰氣著他?」奶子道:「王爺,你不知道誰氣着他?」因使綉春:「外邊瞧瞧看,關著門不曾?路上說話,草里有人不備。俺娘都因為著了那邊五娘一口氣,他那邊貓撾了哥兒手,生生的諕出風來。爹來家那等問著娘,只是不說。落後大娘說了,纔把那貓來摔殺了。他還不承認,拏俺每煞氣!八月裡哥兒死了,他每日那邊指桑樹罵槐樹,百般稱快。俺娘這屋裡,分明聽見,有個不惱的?左右背地裡氣,只是無眼淚!因此這樣暗氣暗惱,纔致了這一場病。天知道罷了!娘可是好性兒,好也在心裡,歹也在心裡。姊妹之間,自來沒有個面紅面赤。有件稱心的衣裳,不等的別人有了,他還不穿出來。這一家子那個不叨貼他娘些兒?可是說的,饒叨貼了娘的,還背他不道是。」王姑子道:「怎的不道是?」如意兒道:「相五娘那邊潘姥姥來一遭,遇著爹在那邊歇,就過來這屋裡和娘做伴兒。臨去,娘與他鞋面、衣服、銀子,甚麼不與他?五娘還不道是!」李瓶兒聽見,便嗔如意兒:「你這老婆,平白只顧說他怎的?我已是死去的人了,隨他罷了!天不言而自高,地不言而自卑。」王姑子道:「我的佛爺,誰知道你老人家這等好心!天也有眼望下看著哩,你老人家往後來還有好處。」李瓶兒道:「王師父,還有甚麼好處!一個孩兒也存不住去了。我如今又不得命,身底下弄這等疾,就是做鬼,走一步也不得個伶俐!我心裡還要與王師父些銀子兒,望你到明日我死了,你替我在家請幾位師父,多誦些血盆經懺,我這罪業還不知墮多少罪業哩!」王姑子道:「我的菩薩,你老人家忒多慮了!天可憐見,到明日假若好了是的。你好心人,龍天自有加護。」正說著,只見琴童兒進來對迎春說:「爹分付把房內收拾收拾,花大舅便進來看娘,在前邊坐著哩。」王姑子便起身說道:「我且往後邊走走去。」李瓶兒道:「王師父你休要去了,與我做兩日伴兒,我還和你說話哩。」王姑子道:「我的奶奶,我不去。」不一時,西門慶陪花大舅進來看問,見李瓶兒睡在炕上不言語,花子油道:「我不知道,昨日聽見這邊大官兒去說,纔曉的。明日你嫂子來看你。」那李瓶兒只說了一聲:「多有起動!」就把面朝裡去了。花子油坐了一回,起身到前邊,向西門慶說道:「俺過世公公老爹,在廣南鎮守,帶的那三七藥,曾吃來不曾?不拘婦女甚崩漏之疾,用酒調五分末兒,吃下去即止。大姐他手裡有收下此藥,何不服之?」西門慶道:「這藥也吃過了。昨日本府胡大尹來拜,我因說起此疾,他也得了個方兒,棕灰與白雞冠花,煎酒服之,只止了一日。到第二日,流的比常更多了。」花子油道:「這個就難為了。姐夫,你早替他看下副板兒預備他罷,明日教嫂子來看他。」說畢起身,西門慶再三款留不住,作辭去了。奶子與迎春正與李瓶兒墊草布在身底下,只見馮媽媽來到,向前道了萬福。如意兒道:「馮媽媽貴人,怎的不來看看娘?昨日爹使來安兒叫你去來,說你鎖著門,往那里去來?」馮婆子道:「說不得我這苦,成日往廟裡修法。早辰出去了,是也直到黑,不是也直到黑,來家倘有那些張和尚、李和尚、王和尚。」如意兒道:「你老人家,怎的這些和尚?早時沒王師父在這裡!」那李瓶兒聽了,微笑了一笑兒,說道:「這媽媽子,單管只撒風!」和意兒道:「馮媽媽,叫著你還不來。娘這幾日粥兒也不吃,只是心內不耐煩。你剛纔來到,就引的娘笑了一笑兒。你老人家伏侍娘兩日,管情娘這病就好了。」馮媽媽道:「我是你娘退灾的博士。」又笑了一回。因向被窩裡摸了摸他身上,說道:「我的娘,你好些兒也罷了!」又問:「坐榪子還下的來?」迎春道:「下得來倒好,前兩遭娘還〈門乍〉〈門爭〉俺每搊扶著下來。這兩日通只在炕上舖墊草布,一日回兩三遍。」如意兒道:「本等沒吃甚麼大食力,怎禁的這等流!」正說著,只見西門慶進來,看見馮媽媽,說道:「老馮,你也常來這邊瞧瞧,怎的去了就不來?」婆子道:「我的爺,我怎不來?這兩日醃菜的時候,掙兩個錢兒醃些菜在屋裡,遇著人家領的業障,好與他吃。不然我那討閒錢買菜兒與他吃?」西門慶道:「你不對我說,昨日俺庄子上起菜,撥兩三畦與你也勾了。」婆子道:「又敢纏你老人家?」說畢,老馮過那邊屋裡去了。西門慶便坐在炕沿上,迎春在旁薰熱芸香。西門慶便問:「你今日心裡覺怎樣?」又問迎春:「你娘早辰吃了些粥兒不曾?」迎春道:「吃的倒好,王師父送了乳餅 蒸來,娘只咬了一些兒,呷了不上兩口粥湯,就丟下了。」西門慶道:「剛纔應二哥小廝門外請那潘道士,又不在了。明日我教來保騎頭口再請去。」李瓶兒道:「你上緊著人請去。那廝但合上眼,只在我根前纏。」西門慶道:「此是你神弱了。只把心放正著,休要疑影他。管情請了他替你把這那祟遣遣,再服他些藥兒,管情你就好了。」李瓶兒道:「我的哥哥,奴已是得了這個拙病,那裡好甚麼?若好,只除非再與兩世人是的。奴今日無人處,和你說些話兒。奴指望在你身邊團圓幾年,死了也是做夫妻一場!誰知到今二十七歲,先把冤家死了。奴又沒造化,這般不得命,拋閃了你去了。若得再和你相逢,只除非在鬼門關上罷了!」說著,一把拉著西門慶手,兩眼落淚哽咽,再哭不出聲來。那西門慶亦悲慟不勝,哭道:「我的好姐姐,你有甚話,只顧說。」兩個正在屋裡哭,忽見琴童兒進來,說:「答應的稟爹,明日十五衙門裡拜牌,畫公座,大發放,爹去不去?班頭好伺候。」西門慶道:「我明日不得去。拏我帖兒回你夏老爹,自家拜了牌罷。」琴童應諾去了。李瓶兒道:「我的哥哥,你依我還往衙門去,休要誤了你公事要緊。我知道幾時死,還早哩。」西門慶道:「我在家守你兩日兒,其心安忍!你把心來放開,不要只管多慮了。剛纔他花大舅和我說,教我早與你看下副壽木,沖你沖,管情你就好了。」李瓶兒點頭兒,便道:「也罷,你休要信著人,使那憨錢。將就使十來兩銀子,買副熟料材兒,把我埋在先頭大娘墳旁,只休把我燒化了,就是夫妻之情。早晚我就搶些漿水,也方便些。你惹多人口,往後還要過日子哩!」這西門慶不聽便罷,聽了如刀剜肝膽,劍挫身心相似,哭道:「我的姐姐,你說的是那裡話?」我西門慶就窮死了,也不肯虧負了你!」正說著,只見月娘親自拏著一小盒兒鮮蘋菠進來,說道:「李大姐,他大娘子那裡,送蘋菠兒來與你吃。」因令迎春:「你洗淨了,拏刀兒切塊來你娘吃。」李瓶兒道:「又多謝他大娘子掛心!」不一時迎春旋去皮兒切了,用甌兒盛貯,西門慶與月娘在旁看著,拈喂了一塊,與他放在口內,只嚼了些味兒,還吐出來了。月娘恐怕勞碌他,安頓他面朝裡就睡了。西門慶與月娘都出來外邊商議。月娘便道:「李大姐,我看他有些沉重。你不早早與他看一副材板兒來預備他,直到那臨時到節熱亂,又亂不出甚麼好板來。馬捉老鼠一般,不是那幹營生的道理。」西門慶道:「今日花大哥也是這般說。適纔我略與他題了題兒,他分付:『休要使多了錢,將就擡副熟板兒罷。你惹多人口,往後還要過日子!』倒把我傷心了這一會。我說亦發請潘道士來看了他,看板去罷。」月娘道:「你看沒分曉,一個人的形也脫了,關口都鎖住,勺水也不進來,還妄想指望好!咱一壁打鼓,一壁磨旗。幸的他若好了,把棺材就捨與人,也不值甚麼!」西門慶道:「既是恁說。」同月娘到後邊使小廝叫將賁四來,在廳上問他:「誰家有好材板?你和姐夫兩個拏銀子看一副來。」賁四道:「大街上陳千戶家,新到了幾副好板。」西門慶道:「既有好板。」即令陳經濟:「你後邊問你娘要四錠大銀子來,你兩個看去。」那陳經濟少頃取了五錠元寶出來,同賁地傳去了。直到後晌纔來回話。西門慶問:「怎的這咱纔來?」他二人回說:「到陳千戶家看了幾副板,都中等,又價錢不合。回來到路上,撞見喬親家爹,說尚舉人家有一副好板。原是尚舉人父親在四川成都府做推官時帶來,預備他老人的兩副桃花洞。他使了一副,只剩下這一副。墻磕底蓋堵頭俱全,共大小五塊,定要三百七十兩銀子,喬親家爹同俺每過去看了板,是無比的好板。喬親家與做舉人的講了半日,只退了五十兩銀子。不是明年上京會試用這幾兩銀子,便也還捨不得賣。這副板還看咱這裡要,別人家定要三百五十兩。」西門慶道:「既是你喬親家爹主張,兌三百二十兩擡了來罷,休要只顧搖鈴打鼓的了。」陳經濟道:「他那裡叫了咱二十五十兩,還找與他七十兩銀子就是了。」一面問月娘又要出七十兩雪花銀子,二人去了。比及黃昏時分,只見許多閒漢,用大紅氈條裹著,擡板進門,放在前廳天井內。打開西門慶觀看,果然好板。隨即叫匠人來鋸開,裡面噴香,每塊五寸厚,二尺五寸寬,七尺五寸長,與伯爵觀看,滿心歡喜。向伯爵道:「這板也看得過了。」伯爵不住口只顧喝采不已。說道:「原說是姻緣板。大抵一物各還有一主。嫂子嫁哥一場,今日暗受這副材板勾了!」分付匠人:「你用心,只要做的好,你老爹賞你五兩銀子。」匠人道:「小人知道。」一面在前廳七手八腳,連夜攢造棺槨不題。伯爵囑來保:「明日早五更去請潘道士,他若來,就同他一答兒來,不可遲滯。」說畢,陪西門慶晚夕在前廳看著做材。到一更時分,纔家去了。西門慶道:「明日早些來,只怕潘道士來的早。」伯爵道:「我知道。」作辭出門去了。都說老馮與王姑子,晚夕都在李瓶兒屋里相伴。只見西門慶前邊散了,進來看視,要在屋裡睡。李瓶兒不肯,說道:「沒的這屋裡齷齷齪齪的,他每都在這裡,不方便。你往別處睡去罷。」西門慶又見王姑子都在這裡,遂過那邊金蓮房中去了。李瓶兒教迎春把角門關了,上了栓。教迎春點著燈,打開箱子,取出幾件衣服銀飾來,放在旁邊。先叫過王姑子來,與了他五兩一錠銀子、一疋紬子:「等我死後,你好歹請幾位師父,與我誦血盆經懺。」王姑子道:「我的奶奶,你忒多慮了!天可憐見你只怕好了。」李瓶兒道:「你只收著,不要對大良說我與你銀子,只說我與了你這疋紬子做經錢。」王姑子道:「我理會了。」于是把銀子和紬子接過來了。又喚過馮媽媽來,向枕頭邊也拏過四兩銀子,一件白綾襖,黃綾裙,一根銀掠兒遞與他,說道:「老馮,你是個舊人,我從小兒你跟我到如今。我如今死了去也,甚麼這一套衣服,并這件首飾兒,與你做一念兒。這銀子你收著,到明日做個棺材本兒。你放心那房子,等我對你爹說,你只顧住著,只當替他看房兒,他莫不就攆你不成!」馮媽媽一手接了銀子和衣服,倒身下拜,哭的說道:「老身沒造化了!有你老人家在一日,與老身做一日主兒。你老人家若有些好歹,那裡歸著!」李瓶兒又叫過奶子如意兒,與了他一襲紫紬子襖兒藍紬裙,一件舊綾披襖兒,兩根金頭簪子,一件銀滿冠兒,說道:「也是你奶哥兒一場。哥兒死了,我原說的教你休撅上奶去,實指望我在一日,占用你一日。不想我又死去了!我還對你爹和你大娘說,到明日我死了,你大娘生了哥兒,也不打發你出去了,就教接你的奶兒罷。這些衣物,與你做一念兒,你休要抱怨。」那奶子跪在地下,磕著頭,哭道:「小媳婦實指望伏侍娘到頭,娘自來沒曾大氣兒呵著小媳婦。還是小媳婦沒造化,哥兒死了,娘又這般病的不得命!好歹對大娘說,小媳婦男子漢又沒了,死活只在爹娘這裡答應了,出去投奔那里?」說畢,接了衣服首飾,磕了頭起來,立在旁邊,只顧揩眼淚。李瓶兒一面叫過迎春、綉春來跪下,囑付道:「你兩個,也是你從小兒在我手裡答應一場。我今死去,也顧不得你每了。你每衣服都是有的,不消與你了。我每人與你這兩對金裹頭簪兒、兩枝金花兒,做一念兒。那大丫頭迎春,已是他爹收用過的,出不去了,我教與你大娘房裡拘管著。這小丫頭綉春,我教你大娘尋家兒人家,你出身去罷,省的觀眉說眼,在這屋裏教人罵沒主子的奴才!我死了,就見出樣兒來了,你伏侍別人,還相在我手裡那等撒嬌撇癡,好也罷,歹也罷了,誰人容的你?」那綉春跪在地下哭道:「我娘,我就死也不出這個門!」李瓶兒道:「你看傻丫頭!我死了,你在這屋裡伏侍誰?」綉春道:「我守著娘的靈。」李瓶兒道:「就是我的靈,供養不久,也有個燒的日子。你少不的也還出去。」綉春道:「我和迎春都答應大娘。」李瓶兒道:「這個也罷了。」這綉春還不知甚麼。那迎春聽見瓶兒囑付他,接了首飾,一面哭的言語說不出來。正是:

「流淚眼觀流淚眼,  斷腸人送斷腸人。」

當夜李瓶兒都把各人囑付了,到天明,西門慶走進房來。李瓶兒問:「買了我的棺材來了沒有?」西門慶道:「從昨日就擡了板來,在前邊做材哩,且沖你沖。你若好了,情愿捨與人罷。」李瓶兒因問:「是多少銀子買的?休要使那枉錢,往後不過日子哩!」西門慶道:「沒多,只給了百十兩來銀子。」李瓶兒道:「也還多了,預備下與我放著。」那西門慶說了回出來,前邊看著做材去了。只見吳月娘和李嬌兒先進房來,看見他十分沉重,便問道:「李大姐,你心裡都怎樣的?」李瓶兒揝著月娘手,哭道:「大娘,我好不成了。」月娘亦哭道:「李大姐,你有甚麼話兒?二娘也在這裡,你和俺兩個說。」李瓶兒道:「奴有甚話說?奴與娘做姊妹這幾年,又沒曾虧了我。實承望和娘相守到白頭,不想我的命苦,先把個冤家沒了。如今不幸我又得了這個拙病死去了!我死之後,房裡這兩個丫頭無人收拘。那大丫頭已是他爹收用過的,教他往娘房裡伏侍娘。小丫頭,娘若要使喚,留下;不然,尋個單夫獨妻,與小人家做媳婦兒去罷,省的教人罵沒主子的奴才!也是他優待奴一場。奴就死,口眼也閉!又奶子如意兒,再三不肯出去。大娘也看着奴分上,也是他奶孩兒一場,明日娘十月已滿生下哥兒,就教接他奶兒罷。」月娘道:「李大姐,你放寬心,都在俺兩個身上。說凶得吉,你若有些山高水低,迎春教他伏侍我,綉春教他伏侍二娘罷。如今二娘房裡丫頭,不老實做活,早晚要打發出去,教綉春伏侍他罷。奶子如意兒,既是你說他沒頭奔,咱家那裡占用不下他來?就是我有孩子沒孩子,到明日配上個小廝與他做房家人媳婦也罷了。」李嬌兒在旁便道:「李大姐,你休只要顧慮,一切事都在俺兩個身上。綉春到明日過了你的事,我收拾房內伏侍我,等我擡舉他就是了。」李瓶兒一面教奶子和兩個丫頭過來,與二人磕頭。那月娘由不得眼淚出。不一時,孟玉樓、潘金蓮、孫雪娥都進來看他。李瓶兒都留了幾句姊妹仁義之言,不必細記。落後待的李嬌兒、玉樓、金蓮眾人都出去了,獨月娘在屋裡守著他。李瓶兒悄悄向月娘哭泣說道:「娘到明日,好生看養著,與他爹做個根蒂兒,休要似奴心粗,吃人暗算了!」月娘道:「姐姐,我知道。」看官聽說:自這一句話,就感觸月娘的心來。後次西門慶死了,金蓮就在家中。住不牢者,就是想著李瓶兒臨終這句話。正是:

「惟有感恩并積恨,  千年萬載不成塵。」

正說話中間,只見琴童分付房中收拾焚下香,五岳觀請了潘法官來了。月娘一面看著,教丫頭收拾房中乾淨,伺候淨茶淨水,焚下百合真合。月娘與眾婦女,都藏在那邊床屋裡聽覷。不一時,只見西門慶領了那潘道士進來。怎生形相?但見:

「頭戴雲霞五岳冠,身穿皂布短褐袍。腰繫雜色綵絲縧,背上橫紋古銅劍。兩隻腳穿雙耳麻鞋,手執五明降鬼扇。八字眉,兩個杏子眼,四方口,一道落腮鬍。威儀凜凜,相貌堂堂。若非霞外雲遊客,定是蓬萊玉府人。」

只見進入角門,剛轉過影壁,恰走到李瓶兒房穿廊臺基下。那道士往後退訖兩步,似有呵叱之狀。爾語數四,方纔左右揭簾進入房中,向病榻而至。運雙睛努力,似慧通神目一視。仗劍手內,搯指步罡,念念有辭,早知其意。走出明間,朝外設下香案。西門慶焚了香。這潘道士焚符喝道:「直日神將,不來等甚!」噀了一口法水去。見一陣狂風所過,一黃巾力士現于面前,但見:

「黃羅抹額,紫綉羅袍。獅蠻帶緊束狼腰,豹皮被牢栓虎體。常遊雲路,每歷罡風。洞天福地片時過,岳瀆酆都撚指到。業龍作孽,向海底以擒來;妖魅為殃,劈山穴而提出。玉皇殿上,稱為符使之名;北極車前,立有天丁之號。常在壇前護法,每來世上降魔。胸懸雷部赤銅牌,手執宣花金蘸斧。」

那位神將,拱立階前。大言:「召吾神,那廂使令?」潘道士便道:「西門氏門中李氏陰人不安,投告于我案下。汝即與我拘當坊土地,本家六神,查考有何邪祟,即與我擒來,毋得遲滯!」言訖,其神不見。須臾,潘道士瞑目變神,端坐于位上。據案擊令牌,恰似問事之狀,久久乃止。出來,西門慶讓至前邊捲棚內,問其所以。潘道士便說:「此位娘子,惜乎為宿世冤愆所訴于陰曹,非邪祟也,不可擒之。」西門慶道:「法官可解禳得麼?」潘道士道:「冤家債主,須得本人。可捨則捨之,雖陰官亦不能強。」因見西門慶禮貌虔切,便問:「娘子年命若干?」西門慶道:「屬羊的,二十七歲。」潘道士道:「也罷,等我與他祭祭本命星壇,看他命燈何如?」西門慶問:「幾時祭?用何香布祭物?」潘道士道:「就是今晚五更正子時,用白灰界畫,建立燈壇。以黃絹圍之,鎮以生辰壇斗,祭以五穀棗湯。不用酒脯,只用本命燈二十七盞,上浮以華蓋之儀,餘無他物。壇內俯伏行禮,貧道祭之,雞犬皆關去,不可入來打攪。可齋戒青衣在內。」這西門慶都一一備辦停當,就不敢進入。在書房中,沐浴齋戒,換了淨衣。那日留應伯爵也不家去了,陪潘道士吃齋饌。到三更天氣,建立燈壇完備。潘道士高坐在上,下面就是燈壇。按青龍、白虎、朱雀、玄武,上建三台華蓋,周列十二官辰,下首纔是本命燈,共合二十七盞。先宣念了投詞。西門慶穿青衣,俯伏階下。左右盡皆屏去,再無一人在左右。燈燭熒煌,一齊點將起來。那潘道士在法座上披下髮來,仗劍,口中念念有詞,望天罡取真炁,布步訣躡瑤壇。正是:

「三信焚香三界合,  一聲令下一聲雷。」

但見晴天星明朗燦,忽然一陣地黑天昏。捲棚四下皆垂著簾幙,須臾,起一陣怪風所過,正是:

「非干虎嘯,豈是龍吟。彷彿入戶穿簾,定是摧花落葉。推雲出岫,送雨歸川。雁迷失伴作哀鳴,鷗鷺驚群尋樹杪。嫦娥急把蟾官閉,列子空中叫故人。」

大風所過三次,一陣冷氣來,把李瓶兒二十七盞本命燈,盡皆刮盡。惟有一盞復明。那潘道士明明在法座上,見一個白衣人領著兩個青衣人,從外進來。手裡持著一紙文書,呈在法案下。潘道士觀看,都是地府勾批,上面有三顆印信。諕的慌忙下法座來,向前喚起西門慶來,如此這般說道:「官人,請起來罷。娘子已是獲罪于天,無所禱也。本命燈已滅,豈可復救乎?只在旦夕之間而已了。」那西門慶聽了,低首無語,滿眼落淚,哭泣哀告:「萬望法師搭救則個!」潘道士道:「定數難逃,難以搭救了!」就要告辭。西門慶再三款留:「等天明早行罷。」潘道士道:「出家人草行露宿,山栖廟止,自然之道。」西門慶不復強之,因令左右捧出布一疋,白金三兩,作經襯錢。潘道士道:「貧道奉行皇天至道,對天盟誓,不敢貪受世財,取罪不便。」推讓再四,只令小童收了布疋作道袍穿,就作辭而行。囑付西門慶:「今晚官人都忌不可往病人房裡去,恐禍及汝身。慎之,慎之!」言畢,送出大門,拂袖而去。西慶歸到捲棚內,看著收拾燈壇。見沒救星,心中甚慟。向伯爵坐的,不覺眼淚出。伯爵道:「此乃各人稟的壽數,到此地位,強求不得,哥也少要煩惱。」因打四更時分,說道:「哥,你也辛苦了,安歇安歇罷。我且家去,明日再來。」西門慶道:「教小廝拏燈籠送你去。」即令來安取了燈,送伯爵出去,關上門進來。那西門慶獨自一個坐在書房內,掌著一枝蠟燭,心中哀慟,口裡只長吁氣。尋思道:「法官戒我休往房里去,我怎坐忍得!寧可我死了也罷,須得廝守著,和他說句話兒。」于是進入房中,見李瓶兒面朝裡睡。聽見西門慶進來,翻過身來,便道:「我的哥哥,你怎的就不進來了?」因問:「那道士點的燈怎麼說?」西門慶道:「你放心,燈上不妨事。」李瓶兒道:「我的哥哥,你還哄我哩。剛纔那廝領著兩個人,又來在我根前鬧了一回,說道:『你請法師來遣我,我已告准在陰司,決不容你!』發恨而去,明日便來拏我也。」西門慶聽了,兩淚交流,放聲大哭道:「我的姐姐,你把心來放正著,休要理他。我實指望和你相伴幾年,誰知你又拋閃了我去了,寧教我西門慶口眼閉了,倒也沒這等割肚牽腸!」那李瓶兒雙手摟抱著西門慶脖子,嗚嗚咽咽,悲哭半日,哭不出聲,說道:「我的哥哥,奴承望和你並頭相守,誰知奴家今日死去也!趁奴不閉眼,我和你說幾句話兒。你家事大,孤身無靠,又沒幫手,凡事斟酌,休要那一沖性兒。大娘等,你也少要虧了他的。他身上不方便,早晚替你生下個根絆兒,庶不散了你家事。你又居著個官,今後也少要往那裡去吃酒,早些兒來家,你家事要緊。比不的有奴在,還早晚勸你。奴若死了,誰肯只顧的苦口說你?」西門慶聽了,如刀剜心肝相似,哭道:「我的姐姐,你所言我知道。你休掛慮我了。我西門慶那世裡絕緣短倖,今世裡與你夫妻不到頭?疼殺我也!天殺我也!」李瓶兒又說:「迎春、綉春之事,奴已和他大娘說來,到明日我死,把迎春伏侍他大娘,那小丫頭,他二娘已承攬他。房內無人,便教伏侍二娘罷。」西門慶道:「我的姐姐,你沒的說。你死了,誰人敢分散你丫頭?奶子也不打發他出去,都教他守你的靈。李瓶兒道:「甚麼靈!回個神主子,過五七兒燒了罷了。」西門慶道:「我的姐姐,你不要管他。有我西門慶在一日,供養你一日。」兩個說話之間,李瓶兒催促道:「你睡去罷,這咱晚了!」西門慶道:「我不睡了,在這屋裡守你守兒。」李瓶兒道:「我死還早哩!這屋裡穢惡,薰的你慌。他每伏侍我不方便。」西門慶不得已,分付丫頭:「仔細看守你娘。」往後邊上房裡,對月娘說,悉把祭燈不濟之事,告訴一遍:「剛纔我到他房中,我觀他說話兒還伶俐。天可憐,只怕還熬出來了也不見得!」月娘道:「眼眶兒也塌了,嘴唇兒也乾了,耳輪兒也焦了,還好甚麼?也只在早晚間了。他這個病,是恁伶俐,臨斷氣還說話兒!」西門慶道:「他來了咱家這幾年,大大小小,沒曾惹了一個人。且是又好個性格兒,又不出語,你教我捨得他那些兒!」題起來,又哭了。月娘亦止不住落淚。不說西門慶與月娘說話。且說李瓶兒喚迎春、奶子:「你扶我面朝裡略倒倒兒。」因問道:「天有多咱時分了?」奶子道:「雞還未叫,有四更天了。」叫迎春替他鋪墊了身底下草布,搊他朝裡蓋被停當睡了,眾人都熬了一夜,沒曾睡。老馮與王姑子都已先睡了。那邊屋裡鎖著。迎春與綉春在面前地坪上搭著鋪,那裡剛睡倒。沒半個時辰,正在睡思昏沉之際,夢見李瓶兒下炕來,推了迎春一推,囑付:「你每看家,我去也。」忽然驚醒,見卓上燈尚未滅。向床上視之,還面朝裡。摸了摸,口內已無氣矣!不知多咱時分,嗚呼哀哉,斷氣身亡!可惜一個美色佳人,都化作一場春夢!正是:

「閻王叫你三更死,  怎敢留人到五更。」

迎春慌忙推醒眾人,點燈來照。果然見沒了氣兒,身底下流血一窪,慌了手腳。走去後邊,報知西門慶。西門慶聽見李瓶兒死了,和吳月娘兩步做一步奔到前邊,揭起被,但見面容不改,體尚微溫,脫然而逝。身上止著一件紅綾抹胸兒。這西門慶也不顧的甚麼身底下血漬,兩隻手抱著他香腮親著,口口聲聲只叫:「我的沒救的姐姐,有仁義好性兒的姐姐,你怎的閃了我去了!寧可教我西門慶死了罷,我也不久活于世了,平日活著做甚麼!」在房裡離地跳的有三尺高,大放聲號哭。吳月娘亦搵滾哭涕不止。落後李嬌兒、孟玉樓、潘金蓮、孫雪娥合家大小丫鬟養娘,都擡起房子來也一般,哀聲動地哭起來。月娘向李嬌兒、孟玉樓道:「不知晚夕多咱死了,恰好衣服兒也不曾得穿一件在身上。」玉樓道:「娘,我摸他身上還溫溫兒的,也纔去了不多回兒。咱不趁熱腳兒不替他穿上衣裳,還等甚麼?」月娘因見西門慶磕伏在他身上,撾臉兒那等哭,只叫:「天殺了我西門慶了!姐姐,你在我家三年光景,一日好日子沒過,都是我坑陷了你了!」月娘聽了,心中就有些不耐煩了。說道:「你看韶刀,哭兩聲兒去開手罷了!一個死人身上,也沒個忌諱,就臉撾著臉兒哭。倘忽口裡惡氣,撲著你是的!他沒過好日子,誰過好日子來?人死如燈滅。半晌時,不借留的住他倒好。各人壽數到了,誰人不打這條路兒來!」因令李嬌兒、孟玉樓:「你兩個拏鑰匙,那邊屋裡尋他裝防的衣服出來,咱與他眼看著,與他穿上。」叫:「六姐,咱兩個把這頭來整理整理。」西門慶又向月娘說:「多尋出兩套他心愛的好衣服,與他穿了去。」月娘分付李嬌兒、玉樓:「你尋他新裁的大紅段遍地錦襖兒,柳黃遍地金裙,併他今年喬親家去那套丁香色雲紬粧花衫,翠藍寬拖子裙;并新做的白綾襖,黃紬子裙出來罷。」當下迎春拏著燈,孟玉樓拏鑰匙,開了床屋裡門,拔步床上的第二個描金箱子裡,都是新做的衣服。揭開箱蓋,玉樓、李嬌兒尋了半日,尋出三套衣裳來。又尋出件綁襯身紫綾小襖兒一件,白紬子裙一件,大紅小衣兒,白綾女襪兒,粧花膝庫腿兒。李嬌兒抱過這邊屋裡,與月娘瞧。月娘正與金蓮燈下替他整理頭髻,用四根金簪兒,綰一方大鴉青手帕,旋轉停當。李嬌兒因問:「尋雙甚麼顏色鞋,與他穿了去?」潘金蓮道:「姐姐,他心裡只愛穿那雙大紅遍地金鸚鵡摘桃白綾高底鞋兒,只穿了沒多兩遭兒,倒尋那雙鞋出來,與他穿了去罷。」吳月娘道:「不好,倒沒的穿上陰司裡,好教他跳火炕。你把前日門外往他嫂子家去,穿的那雙紫羅遍地金高底鞋,也是扣的鸚鵡摘桃鞋,尋出來與他裝綁了去罷。」這李嬌兒聽了,走來向他盛鞋的四個小描金箱兒約百十雙鞋,翻遍了都沒有。迎春說:「俺娘穿了來,只放在這裡。怎的沒有?」走來廚下問綉春。綉春道:「我看見娘包放在箱坐廚里。」扯開坐廚子尋,還有一大包,都是新鞋。尋出來了。眾人七手八腳都裝綁停當。西門慶率領眾小廝,在大廳上,收捲書畫,圍上幃屏。把李瓶兒用板門擡出,停于正寢。下鋪錦褥,上覆紙被。安放几筵香案,點起一盞隨身燈來。專委兩個小廝在旁侍奉,一個打磬,一個炷布。一面使玳安:「快請陰陽徐先生來看時批書。」月娘打點出裝綁衣服來,就把李瓶兒床房門鎖了。只留炕屋裡,交付與丫頭養娘。那馮媽媽見沒了主兒,哭的三個鼻頭兩個眼淚。王姑子且口裡喃喃吶吶,替李瓶兒念密多心經、藥師經、解冤經、楞嚴經并大悲中道神咒,請引路王菩薩,與他接引冥途。西門慶在前廳,手拘著胸膛,由不的撫尸大慟,哭了又哭,把聲都呼啞了。口口聲聲,只叫我的好性兒有仁義的姐姐不住。比及亂著,雞就叫了。玳安請了徐先生來,向西門慶施禮,說道:「老爹煩惱!奶奶沒了在于甚時候?」西門慶道:「因此時候不真。睡下之時,已打四更。房中人都困倦睡熟了,不知多咱時分沒了。」徐先生道:「此是第幾位奶奶?」西門慶道:「乃是第六的小妾,生了個拙病,淹淹纏纏,也這些時了!」徐先生道:「不打緊。」因令左右掌起燈來,廳上揭開紙被觀看,手搯五更。說道:「正當五更二點徹,還屬丑時斷氣。」西門慶即令取筆硯,請徐先生批書。這徐先生向燈下打開青囊,取出萬年曆通書來觀看,問了姓氏并生時八字,批將下來:「一故錦衣西門夫人李氏之喪,生于元祐辛未正月十五日午時,卒于政和丁酉九月十七日丑時。今日丙子,月令戊戌,犯天地往亡日,重喪之日,煞高一丈,向西南方而去。遇太歲煞沖迎斬之局。避本家,忌哭聲,成服後無妨。入殮之時,忌龍、虎、雞、蛇四生人,外親人不避。」吳月娘使出玳安來,教徐先生看看黑書上,往那方去了。這徐先生一面打開陰陽秘書觀看,說道:「今日丙子日,乃是巳丑時。死者上應寶瓶宮,下臨齊地。前生曾在濱州王家作男子,打死懷胎母羊。今世為女人屬羊。稟性柔婉,自幼陰謀之事。父母雙亡,六親無靠,先與人家作妾,受大娘子氣。及至有夫主,又不相投,犯三刑六害。中年雖招貴夫,常有疾病,比肩不和,生子夭亡。主生氣疾,肚腹流血而死。前九日魂去,托生河南汴梁開封府袁指揮家為女,艱難不能度日。後躭擱至二十歲,嫁一富家,老少不對。中年享福,壽至四十二歲,得氣而終。」看畢黑書,眾婦女聽了,皆各嘆息。西門慶教徐先生看破土安葬日期。徐先生請問:「老爹停放幾時?」西門慶哭道:「熱突突,怎麼就打發出去的!須放過五七纔好。」徐先生道:「五七裡沒有安葬日期。倒是四七裡,宜擇十月初八日丁酉午時破土,十二月辛丑巳時安葬。合家六位本命都不犯。」西門慶道:「也罷。到十月十二日發引,再沒那移了。」徐先生當寫殄榜,蓋伏死者身上,向西門慶道:「十九日辰時大殮,一應之物,老爹這里備下。」于是剛打發徐先生出了門,天已發曉。西門慶使琴童兒騎頭口往門外請花大舅,然後分班差家下入各親眷處報喪。又使人往衙門中給假,在家整理喪事。使玳安往獅子街取了二十桶瀼紗漂白,三十桶生眼布來,教趙裁顧了許多裁縫,在西廂房先顧人造幃幕帳子卓圍,并入殮衣衾纏帶,各房裡女人衫裙。外邊小廝伴當,每人都是白唐巾,一件白直裰。又兌了一百兩銀子,教賁四往門外店裡摧了三十桶魁光麻布,二百疋黃絲孝絹。一面又教搭匠在大天井內搭五間大棚。西門慶因想起李瓶兒動止行藏模檥兒來,心中忽然想起忘了與他傳神,叫過來保來問:「那裡有寫真好畫師,尋一個傳神。我就把這件事忘了!」來保道:「舊時與咱家畫圍屏的韓先兒,他原是宣和殿上的畫士,革退來家。他傳的好神。」西門慶道:「他在那里住?快與我請來。」這來保應諾去了。西門慶熬了一夜沒睡的人,前後又亂了一五更,心中已著了悲慟,神思恍亂,只是沒好氣,罵丫頭、踢小廝,守著李瓶兒屍首,由不的放聲哭叫。那玳安在傍亦哭的言不的語不的。吳月娘正和李嬌兒、孟玉樓、潘金蓮在帳子後,打夥兒分散各房裡丫頭并家人媳婦。看見西門慶只顧哭起來,把喉音也叫啞了,問他與茶也不吃,只顧沒好氣。月娘便道:「你看恁勞叨!死也死了,你沒的哭的他活!哭兩聲丟開手罷了,只顧扯長絆兒哭起來了!三兩夜沒睡,頭也沒梳,臉也還沒洗,亂了恁五更,黃湯辣水還沒嚐著,就是鐵人也禁不的。把頭梳了,出來吃些甚麼,還有個主張。好小身子,一時摔倒了,都怎樣兒的!」玉樓道:「他原來還沒梳頭洗臉哩。」月娘道:「洗了臉倒好。我頭裡使小廝請他後邊洗臉,他把小廝踢進來,誰再問他來!」金蓮接過來道:「你還沒見頭裡進他屋裡尋衣裳,教我是不是倒好意說他,都相恁一個死了,你恁般起來,把骨禿肉兒也沒了。你在屋裡吃些甚麼兒,出去再亂也不遲。他倒把眼睜紅了的,罵我:『狗攮的淫婦,管你甚麼事!』我如今鎮日不教狗攮,卻教誰攮哩!恁不合理的行貨子,只說人和他合氣!」月娘道:「熱突突死了,怎麼不疼?你就疼,也還放心裡。那裡就這般顯出來!人也死了,不管那有惡氣沒惡氣,就口撾著那口那等叫喚,不知甚麼張致!吃我說了兩句。他可可兒來三年,沒過一日好日子?鎮日教他挑水挨磨來?」孟玉樓道:「娘不是這等說。李大姐倒也罷了,沒甚麼,倒吃了他爹恁三等九格的!」金蓮道:「他得過好日子,那個偏受用著甚麼哩?都是一個跳板兒上人。」正說著,只見陳經濟手裡拿著九疋水光絹:「爹說教娘每剪各房裡手帕,剩下的與娘每做裙子。」月娘收了娟,便道:「姐夫去請你爹進來扒口子飯,這咱七八待晌午,他茶水還沒嚐著哩!」經濟道:「我是不敢請他,裡頭小廝請他吃飯,差些沒一腳踢殺了。我又惹他做甚麼?」月娘道:「你不請他,等我另使人請他來吃飯。」良久叫過玳安來,說道:「你爹還沒吃飯,哭這一日了。你拿上飯去,趁溫先生在,陪他吃些兒。」玳安道:「請應二爹和謝爹去了,等他來時,娘這裡使人拿飯上去,消不的他幾句言語兒,管情爹就吃了飯。」月娘道:「硶說嘴的囚根子!你是你爹肚裡蛔虫?俺每這幾個老婆倒不如你了!你怎的就知道他兩個來纔吃飯?」玳安道:「娘每不知,爹的好朋友大小酒席兒,那遭少了他兩個?爹三錢,他也是三錢,爹二星,他也是二星。爹隨問怎的著了惱,只他到略說兩句話兒,爹就眉花眼笑的。」說了一回,棋童兒請了應伯爵、謝希大二人來到,進門撲倒靈前地下,哭了半日,只哭:「我的有仁義的嫂子!」被金蓮和玉樓道:「賊油嘴的囚根子!俺每都是沒仁義的。」二人哭畢,扒起來。西門慶與他回禮,兩個又哭了,說道:「哥煩惱煩惱!」一面讓至廂房內與溫秀才敘禮坐下。先是伯爵問道:「嫂子甚時候沒了?」西門慶道:「正丑時斷氣。」伯爵道:「我到家已是四更多了。房下問我,我說:『看陰騭,嫂子這病已在七八了。』不想剛睡就做了一夢,夢見哥使大官兒來請我,說家裡吃慶官酒,教我急急來到。見哥兒穿著一身大紅衣服,向袖中取出兩根玉簪兒與我瞧,說:『一根折了。』教我瞧了半日,對哥說:『可惜了,這折了是玉的,完全的倒是硝子石。』哥說:『兩根都是玉的。』俺兩個正睡著,我就醒了。教我說此夢做的不好,房下見我只顧咂嘴,便問:『你和誰說話?』我道:『你不知,等我到天曉告訴你。』等到天明,只見大官兒到了,戴著白,教我只顧跌腳。果然哥有孝服!」西門慶道:「我前夜也做了恁個夢,和你這個一樣兒。夢見東京翟親家那裡寄送了六根簪兒,內有一根〈石否〉折了。我說:『可惜兒的!』教我夜裡告訴房下。不想前邊斷了氣,好不睜眼的天,撇的我真好苦!寧可教我西門慶死了,眼不見就罷了。到明日一時半霎想起來,你教我怎不心疼?平時我又沒曾虧欠了人,天何今日奪吾所愛之甚也!先是一個孩兒也沒了,今日他又長伸腳子去了,我還活在世上做甚麼!雖有錢過北斗,成何大用!」伯爵道:「哥,你這話就不是了。我這嫂子與你是那樣夫妻,熱突突死了,怎的不心疼?爭耐你惹大的家事,又居著前程,這一家大小太山也似靠著你。你若有好歹,怎麼了得?就是這些嫂子都沒主兒。常言:『一在三在,一亡三亡。』哥你聰明,你伶俐,何消兄弟每說。就是嫂子他青春年少,你疼不過越不過他的情,成服令僧道念幾卷經,大發送葬,埋在墳裡,哥的心也盡了,也是嫂子一場的事,再還要怎樣的?哥,你且把心放開。」當時被伯爵一席話,說的西門慶心地透澈,茅塞頓開,也不哭了。須臾拿上茶來吃了,便喚玳安:「後邊說去,看飯來,我和你應二爹、溫師父、謝爹吃。」伯爵道:「哥原來還未吃飯哩。」西門慶道:「自後你去了,亂了一夜,到如今誰嘗甚麼兒來!」伯爵道:「哥,你還不吃飯,這個就糊突了。常言道:『寧可折本,休要饑損。』孝經上不說的:『教民無以死傷生,毀不滅性。』死的自死了,存者還要過日子。哥要做個張主!」正是:

「數語撥開君子路,  片言題醒夢中人。」

畢竟未知後來如何,且聽下回分解:





\end{showcontents}


