%# -*- coding: utf-8 -*-
%!TEX encoding = UTF-8 Unicode
%!TEX TS-program = xelatex
% vim:ts=4:sw=4
%
% 以上设定默认使用 XeLaTex 编译,并指定 Unicode 编码,供 TeXShop 自动识别

%第七十回 
\chapter{西門慶工完陞級\KG 群僚廷參朱太尉}


\begin{showcontents}{}



「昨夜西風鼓角喧,  曉來隆凍怯寒毡,

茫茫一片渾無地,  浩浩四方俱丹天;

綺壁凄涼宜未守,  霸陵豪傑且停鞭,

陽春有腳恩如海,  願借餘溫到客邊。」

話說西門慶自此與李桂姐斷絕不題。都說走差人到懷慶府林千戶處打聽消息。林千戶將陞官邸報,封付與來人,又賞了五錢銀子,連夜來遞與提刑兩位官府。當廳,夏提刑拆開,同西門慶先觀本衛行來考察官員照會,其略曰:

「兵部一本,尊明旨,嚴考覈,以昭勸懲,以光聖治事。先該金吾衛提督官校太尉太保兼太保朱題前事,考察禁衛官員,除堂上官自陳外,其餘兩廂,詔獄輯捕,捉察、稽察察、觀察,典牧皇畿,內外提刑所指揮千百戶,鎮撫等官,各按冊籍祖職世襲轉陞功陞蔭陞納級等項,各挨次格,從公舉劾。甄別賢否,具題上請。當下該部詳議,黜陟陞調降革等因素奉聖旨兵部知道,欽此欽遵,抄出到科,按行到部。看得太尉朱題前事,遵奉舊例,委的本官殫力致忠,公于考覈。委所同并,內外屬官,各據冊籍,博協輿論,甄別賢否。皆出聞見之實,而無偏執之私。足見本官,仰拔天顏之咫尺,而存體國之忠謀也。分別等第獎勵,淑慝井井有條。足以勵人心,而孚公議,無容臣等再喙。但恩威賞罰,出自朝廷。合候命下之日,一體照例施行等因。庶考覈明而人心服,冒濫革而官箴肅矣。奉欽此欽依擬行。內開:山東提刑所正千戶夏廷齡,資望既久,才練老成。昔視典牧,而坊隅安靜;今理齊刑而綽有政聲。宜加獎勵,以異甄陞,可備鹵簿之選者也。貼刑副千戶西門慶,才幹有為,英偉素著,家稱殷實,而在任不貪;國事克勤,而臺工有績。翌神運而分毫不索,司法令而齊民戴仰。宜加轉正,以掌刑名者也。懷慶提刑千戶所正千戶林承勳,年清優學,占籍武科。繼祖等,抱負不凡,提刑獄,詳明有法。幹濟有法,泰嚴亡度。可加薦獎勵簡任者也。副千戶謝恩,年齒既殘,昔在行伍,猶有可觀;今任理刑,罹軟尤甚。可宜罷黜革任者也。」

西門慶看了他轉正千戶當刑,心中大悅。夏提刑見他陞指揮管鹵簿,大半日無言,面容失色。于是又展開工部工完的本觀看,上面寫道:

「工部一本,神運屇京,天人胥慶。懇乞天恩,俯加偓典,以蘇民困,以廣聖澤事。奉聖旨,這神運春迎大內,奠安銀嶽。以豕天眷,朕心加悅。你每既效有勤勞,副朕事玄至意,所經過地方,委的小民困苦。著行撫按衙門,查勘明白,行蠲免今歲田租之半。所毀埧閘,你部裡差官,會同巡按御史,即行修理。完日,還差內侍孟昌齡,前去致祭。蔡京、李邦彥、王煒、鄭居中、高俅輔弼朕躬,直贊內庭,勳勞茂著;京加太師,邦彥加柱國太子太師,王煒太傳,鄭居中、高俅太保,各賞銀五十兩,四表裡。蔡攸還蔭一子為殿中監。國師林靈素,胡知朕叩宣,佑國宣化,遠致神運。北伐虜謀,實與天通,加封忠孝伯,食祿一千石,賜坐龍衣一襲,肩輿入內,賜號玉真教主,加淵澄玄妙廣德真人,金門羽客,真達靈玄妙先生。朱勔、黃經臣督理神運,忠勤可加。勔加太傳兼太太子太傳,經臣加殿前都太尉,提督御前人船,各蔭一子為金吾衛正千戶。內侍李彥、孟昌齡、賈祥、何沂、藍從熙,著直延福五位官近侍,各賜蟒衣玉帶,仍蔭弟姪一人為副千戶,俱見任管事。禮部尚書張邦昌,左侍郎兼學士蔡攸,右侍郎白時中,兵部尚書余深,工部尚書林攄,俱加太子太保,各賞銀四十兩,彩段二表裡。巡撫兩浙僉都御史張閣,陞工部右侍郎。巡撫山東都御史侯蒙,陞太常正卿。巡撫兩浙山東監察御史尹大諒、朱喬年,都水司郎中安忱、伍訓,各陞俸一級,賞銀二十兩。祇迎神運千戶魏承勳、徐相、楊廷珮、司鳳儀、趙友蘭、扶天澤、西門慶、田九皋等各陞一級。內侍宋推等,營將王佑等,尚各賞銀十兩。所官薛顥忠等,各賞五兩。校尉昌玉等,絹二疋。該衙門知道。」

夏提刑與西門慶看畢,各散衙回家。後晌時分,有王三官差永定同文嫂拿著請書盒兒來,內安泥金摺,初十日請西門慶往他府中赴席,少罄謝私之意。西門慶收下,不勝歡喜,以為其妻指日在于掌握。不期到初十日晚夕,東本東衛經歷司,差人行照會到,曉諭各省提刑官員知悉,火速赴京,赶冬至令節見朝引奏謝恩,毋得違悮,取罪不便。西門慶看了,到次日衙門中會了夏提刑,回手本打發來人回去,不在話下。各人到家,收拾行裝,備辦贄見禮物,不日約令起程。西門慶使玳安叫了文嫂兒,教他回王三官,十一日不得來赴席,如此這般,上京見朝謝恩去也。王三官道:「既是老伯有事,待容回來,潔誠具請。」西門慶一面叫將賁四,分付教他跟了去。與他五兩銀子家中盤纏。留下春鴻看家,帶了玳安、王經跟隨答應。又問周守備討了四名巡捕軍人,四匹小馬,打點馱裝暖轎,馬排軍抬扛。夏提刑那邊,夏壽跟隨。兩家有二十餘人跟從。十二日起身,離了清河縣,冬天易晚,晝夜趲行。到了懷西懷慶府,會林千戶。千戶已上東京去了。一路天寒坐轎,天暖乘馬。朝登紫陌紅塵,夜宿郵亭旅邸。正是:

「意急款搖青毡,  心忙斝碎紫絲鞭。」

評話捷說,到了東京,進得萬壽門來。依著西門慶分別,他主意要往相國寺下。夏提刑不肯,堅執要請往他令親崔中書家投下。西門慶不免先具拜帖拜見。正值崔中書在家,即出迎接。至廳敘禮相見,道及寒喧契闊之情。拂去塵土,坐下,茶湯已畢,拱手問西門慶尊號。西門慶道:「賤號四泉。」因問:「老先生尊號?」崔中書道:「學生性最愚朴,名閑林下,賤名守愚,拙號遜齋。」因說道:「舍親龍溪,久稱盛德。全仗扶持,同心協恭,莫此為厚!」西門慶道:「不敢。在下常領教誨,今又為堂尊,受益恒多,可幸可幸!」夏提刑道:「長官如何這等稱呼?雖有鎡基,不如待時。」崔中書道:「四泉說的也名分使然,不得不早。」言畢,彼此笑了。不一時,收拾了行李,天晚了。崔中書分付童僕放卓擺飯,無非是菓酌餚饌之類,不必細說。當日二人在崔中書宿歇不題。到次日各備禮物拜帖,家人跟隨,早往蔡太師府中叩見。那日太師在內閣還未出來,府前官吏人等,如蜂屯蟻聚,通擠匝不開。西門慶與夏提刑與了門上官吏兩包銀子,拿揭帖稟進去。翟管家見了,即出來相見,讓他到外邊私宅。先是夏提刑相見畢,然後西門慶敘禮,彼此道及往還酬答之意,各分賓位坐下。夏提刑先遞上禮帖,兩疋雲鶴金段,兩疋色段,翟管家的是十兩銀子。西門慶禮帖上是一疋大紅絨綵蟒,一疋玄色粧花斗牛補子員領,兩疋京段;另外梯已送翟管家一疋黑綠雲絨,三十兩銀子。翟謙分付左右:「把老爺禮,都交收進府中去上簿籍。」他只受了西門慶那疋雲絨,將三十兩銀子,連那夏提刑的十兩銀子,都不受。說道:「豈有此理?若如此,不見至交親情!」一面令左右放卓兒擺飯,說道:「今日聖上奉銀嶽,新蓋上清寶籙宮,奉安牌扁,該老爺主祭,直到午後纔散。到家周李爺又往鄭皇親家吃酒,只怕親家和龍溪等不的,誤了你每勾當。遇老爺閒,等我替二位稟,就是一般。」西門慶道:「蒙親家費心,若是這等又好了!」因問:「親家那裡住?」西門慶就把夏龍溪令親家下歇說了。不一時,安放卓席端正,就是大盤大碗,湯飯點心,一齊拿上來,都是光祿烹炮美味,極品無加。每人金爵飲酒三杯,就要告辭起身。翟謙于是款留,令左右再篩上一杯。西門慶因問:「親家,俺每幾時見朝?」翟謙道:「親家,你同不得夏大人。大人如今京堂官,不在此例。你與本衛新陞的副千戶何太監姪兒何永壽,他便貼刑,你便掌刑,與他作同僚了。他先謝了恩,只等著你見朝引奏畢,一同好領劄付。你凡事只會他去。」夏提刑聽了,一聲兒不言語。西門慶道:「請問親家,你曉的我還等冬至郊天畢回來,見朝如何?」翟謙道:「親家你等不的。冬至聖上郊天回來,那日天下官員上表朝賀畢,還要排慶成宴,你每原等的?不如你今日先鴻臚寺報了名,明日早朝謝了恩,直到那日堂上官引奏畢,領劄付起身就是了。」西門慶道:「蒙親家指教,何以克當!」臨起身,翟謙又拉西門慶道側淨處說話,甚是埋怨西門慶,說:「親家,前日我的書去,那等囑了,大凡事謹密,不可使同僚每知道;親家如何對夏大人說了,教他央了林真人帖子來,立逼著朱太尉,太尉來對老爺說,要將他情愿不管鹵簿。仍以指揮職啣,在任所掌刑三年情況。何太監又在內廷,轉央朝廷所寵安妃劉娘娘的分上,便也傳旨出來,親對太爺和朱太尉說了,要安他姪兒何永壽在山東理刑。兩下人情阻住了,教老爺好不作難。不是我再三在老爺根前維持,回倒了林真人,把親家不撑下去了?」慌得西門慶連忙打躬,說道:「多承親家盛情!我並不曾對一人說,此公何以知之?」翟謙道:「自古機事不密則害成,今後親家凡事謹慎些便了。」這西門慶千恩萬謝,與夏提刑作辭出門,來到崔中書家。一面差賁四鴻臚寺報了名,次日見朝,青衣冠帶,同夏提刑進內,不想只在午門前謝了恩出來,剛轉過西闕門來,只見一個青衣人走向前問道:「那位是山東提刑西門慶老爹?」賁四問道:「你是那裡的?」那人道:「我是內府匠作監何公公來請老爹說話。」言未畢,只見一個太監,身穿大紅蟒衣,頭戴三山帽,腳下粉底皂靴,縱御街定聲叫道:「西門大人請了。」西門慶遂與夏大人分別,被這太監用手一把拉在傍邊一所直房內,都是明窗亮槅,裡面籠的火暖烘烘的,卓上陳設的許多卓盒。一面相見,作了揖,慌得西門慶倒身還禮不迭。說道:「大人,你不認的我,在下是內府匠作太監何沂,見在延寧第四官端妃馬娘娘位下近侍。昨日內工完了,蒙萬歲爺爺恩典,將姪男何永壽陞授金吾衛左所副千戶,見在貴處提刑所理刑管事,與老大人作同僚。」西門慶道:「原來是何老公公,學生不知,恕罪恕罪!」一面又作揖,說道:「此禁地不敢行禮,容日到老太監外宅進拜。」于是敘禮畢,讓坐,家人捧茶,金漆硃紅盤托盞遞上茶去吃了。茶畢,就揭卓盒蓋兒。卓上許多湯飯餚品,拿盞筯兒來安下。何太監道:「不消小杯了,我曉的大人朝下來,天氣寒冷,拿個小盞來。沒甚麼餚,褻瀆大人,且吃個頭腦兒罷。」西門慶道:「不敢當擾。」何太監于是滿斟上一大杯,遞與西門慶。門慶道:「老太監承賜,學生領下。只是出去還要貝官拜部,若吃得面紅,不成道理。」何太監道:「吃兩盞兒盪寒,何害?」因說道:「舍姪兒年幼,不知刑名。望乞大人看我面上,同僚之間,凡事教導他教導。」西門慶道:「豈敢!老太監勿得太謙!令姪長官雖是年幼,居氣養體,自然福至心靈。」何太監道:「大人好道。常言:『學到老,不會到老。』天下事如牛毛,孔夫子也識得一腿。恐有不知到處,大人好歹說與他。」西門慶道:「學生謹領。」因問:「老太監外宅在何處?學生好去奉拜長官。」何太監道:「舍下在天漢橋東文華坊,雙獅馬台就是。」亦問:「大人下處在那裡?「我教做官的先去叩拜。」西門慶道:「學生暫借崔中書家下。」彼此問了住處,西門慶吃了一大杯就起身。何太監送出門,拱著手說道:「適間所言,大人凡事看顧看顧,他還等著你會同一答兒引奏,當堂上作主進了禮,好領劄付。」西門慶道:「老太監不消分付,學生知道。」于是出朝門,又到兵部。又遇見了夏提刑,同拜了部官來。比及到本衛參見朱太尉,遞履歷手本,繳劄付,又拜經歷司并本所官員,已是申刻時分。夏提刑改換指揮服色,另具手本,參見了朱太尉,免行跪禮,擇日南衙到任。剛出衙門,西門慶還等著,遂不敢與他同行,讓他先上馬。夏延齡那裡肯,定要同行。西門慶趕著他呼堂尊。夏指揮道:「四泉,你我同僚在先,為何如此稱呼?」西門慶道:「名分已定,自然之道,何故太謙?」因問:「堂尊高陞美任,不還山東去了。寶眷幾時搬取?」夏延齡道:「欲待搬來,那邊房舍無人看守。如今且在舍親這邊權住,直待過年差人取家小罷了。日逐望長官早晚家中看顧一二!房子若有人要,就央長官替我打發,自當感謝。」西門慶道:「學生謹領。請問府上那房價值若干?」夏延齡道:「舍下此房,原是一千三百兩買的徐內相房子,後邊又蓋了一層,收拾使了二百兩。如今賣原價也罷了。」西門慶道:「堂尊說與我,有人問,我好回答,庶不誤了。」夏延齡道:「只是有累長官費心。」二人歸到崔宅,王經向前稟說:「新陞何老爹來拜,下馬到廳,小的回部中還未來家。何老爹說多拜上,還與夏老爹、崔老爹都投下帖。午間差人送了兩疋金段來。」宛紅帖兒,拿與西門慶看。上寫著:「謹具段帕二端,奉引贄敬。寅侍教生何永壽頓首拜。」西門慶看了,連忙差王經封了兩疋南京五彩獅補員領,寫了禮帖,吃了飯,連忙往何家回拜去。到于廳上,何千戶忙整衣迎接出來,穿著五彩粧花玄色雲絨獅補員領,烏紗皂履,腰繫玳帽蒙金帶,年紀不上二十歲。生的面如傅粉,眉目清秀,唇若塗朱,趨下階來,揖讓退遜,謙恭特甚。西門慶陞階,左右忙去掀簾。呼喚一聲,奔走後先應諾。二人到廳上敘禮,西門慶令玳安揭開段盒,捧上贄見之禮,拜下去。說道:「適承光顧,兼領厚儀,所失迎迓。今早又蒙老公公直房賜饌,威德不盡!」何千戶忙頂頭還禮,說:「小弟叨受微職,忝與長官同例,早晚得領教益,為三生有幸!適間進拜不遇,又承垂愛,蓬蓽生光!」令左右收下去。一面扯公座椅兒,都是塵皮坐褥,分賓主坐下。左右捧上茶來,何千戶躬身捧茶,遞與西門慶。門慶亦離席交換。吃茶之間,彼此問號,西門慶道:「學生賤號四泉。」何千戶道:「學生賤號天泉。」又問:「長官今日拜畢部堂了?」西門慶道:「從內裡蒙公公賜酒出來,拜畢部,又到本衙門見堂,繳了劄付,拜了所司,出來見長官尊帖下顧,失迎,不勝惶恐!」何千戶道:「不知長官到,學生拜遲。」因問:「長官今日與夏公都見朝來?」西門慶道:「龍溪今已陞了指揮直駕,今日都見朝謝恩在一處。只到衙門見堂之時,他另具手本參見。」問畢,何千戶道:「今日與長官計議了,咱每幾時與本主老爹見禮領劄付?」西門慶道:「依著舍親說,咱每先在衛主宅進了禮,然後大朝引奏,還在本衙門到堂,同眾領劄付。」何千戶道:「既是長官如此說,咱每明日早備禮進了罷。」于是都會下各人禮數,何千戶是兩疋蟒衣,一束玉帶。西門慶是一疋大紅麒麟金段,一疋青絨蟒衣,一柄金廂玉縧環;各金華酒四罈 ,明早在朱太尉宅前取齊。約會已定,茶湯兩換,西門慶告辭而回,並不與夏延齡題此事。一宿晚景題過。到次日早,到何千戶家。何千戶又是預備飯食頭腦小席,大盤大碗,齊齊整整。連手下人飽餐一頓,然後同往太尉宅門前來。賁四同何家人,又早押著禮物伺候已久。那時正值朱太尉新加太保,徽宗天子又差遣往南壇視往未回。各家餽送賀禮,伺候參見官吏人等,黑壓壓在門首等的鐵桶相似。何千戶下了馬,在左近一相識家坐的,差人打聽老爺道午响,就來通報。一等等到午後時分,忽見一人飛馬而來,傳報道:「老爺視往回來,進南薰門了。」分付閒雜人打開。不一時,騎報回來,傳:「老爺過天漢橋了。」頭一廚役跟隨,茶盒攢盒到了。半日纔遠遠牌兒馬到了。眾官都頭帶勇字鎖鐵盔,身穿摟掭紫花甲,青紵絲團花窄袖衲祅,紅綃裹肚,緑麂皮挑線海獸戰裙。脚下四縫着腿黑靴,弓彎雀畫,箭插雕翎金袋。肩上橫擔銷金令字藍旗。端的人如猛虎,馬賽飛龍。須臾,一對藍旗過來,夾著一對青衣節級上,一個個長長大大,搊搊搜搜。頭帶黑青巾,身穿皂直裰,腳上乾黃皮底靴,腰間懸繫虎頭牌,騎在馬上,端的威風凜凜,相貌堂堂。須臾,三隊牌兒馬過畢,只聞一片喝聲傳來。那傳道者都是金吾衛士,直場排軍,身長七尺,腰闊三停。人人青巾桶帽,個個腿纏黑靴。左手執著藤棍,右手潑步撩衣。長聲道了一聲喝道而來,下路端的嚇魄消魂,陡然市衢澄靜。頭道過畢,又是二道摔手。摔手過後,兩邊雁翎排列。二十名青衣緝捕,皆身腰長大,都是寬腰大肚之輩,金眼黃鬚之徒,個個貪殘類虎,人人那有慈悲。十對青衣後面,轎是八擡八簇肩輿明轎,轎上坐著朱太尉。頭戴烏紗,身穿猩紅斗牛絨袍,腰橫四指荊山白玉鈴瓏帶,腳靸皂靴,腰懸太保牙牌,黃金魚鑰,頭帶貂蟬,腳登虎皮,搭擡那轎的離地約有三尺高。前面一邊一個相抱角帶,身穿青紵絲,家人跟著,轎後又是一斑兒六面牌兒馬,六面令字旗,緊緊圍護,以聽號令。後約有數十人,都騎著寶鞍駿馬,玉勒金〈革登〉,都是官家親隨掌案書辦書吏人等,都出于袴養時話,驕自已好色貪財,那曉王章國法。登時一隊隊都到宅門首,一字兒擺下。喝的人靜迴避,無一人聲嗽。那來見的官吏人等,黑壓壓一群,跪在街前。良久太尉轎到根前,左右喝聲:「起來伺候!」那眾人一齊應諾,誠然聲震雲霄。只聽東邊鼕鼕來响動,原來本尉八員太尉堂官兒,見太尉新加光祿大夫太保,又蔭一子為千戶,都各備大禮在此,治具酒筵來此慶賀。故此有許多教坊伶官,在此動樂。太尉纔下轎,樂就止了。各項官吏人等,預備進見。忽然一聲道子响,一青衣承差,手拿兩個紅拜帖,飛走而來,遞與門上人,說:「禮部張爺與學士蔡大爺來拜。」連忙稟報進去。須臾,轎在門首,尚書張邦昌與侍郎蔡攸,都是紅吉服孔雀補子,一個犀帶,一個金帶。進去拜畢,待茶畢,送出來。又是吏部尚書王祖道與左侍郎韓侶,右侍郎尹京,也來拜,朱太尉都待茶送了。又是皇親喜國公,樞密使鄭居中,駙馬掌宗人府王晉卿,都是紫花玉帶來拜。惟鄭居中坐轎,這兩個都騎馬。送出去,方是本衙堂上六員太尉到了,呵殿宣儀,行仗羅列。頭一位是提督管兩廂捉察使孫榮,第二位管機察梁應龍,第三管內外觀察典牧畿童太尉姪兒童天胤,第四提督京城十三門巡察使,第五管京營衛緝察皇城使竇監,第六督管京城內外巡捕使陳宗善,都穿大紅,頭帶貂蟬。惟榮是太子太保,玉帶,餘者都是金帶。下馬進去,各家都有金幣尺頭禮物。少頃,裡面樂聲响動,眾太尉插金花,拿玉帶,與朱太尉把盞遞酒。階下一派簫韶盈耳,兩行絲竹和鳴。端的食前方丈,花簇錦筵。怎見得太尉的富貴?但見:

「官居一品,位列三台。赫赫公堂,晝長鈴索靜;潭潭相府,漏定戟杖齊。林花散彩賽長春,簾影垂虹光不夜。芬芬馥馥,獺髓新調百和香;隱隱層層,龍紋大篆千金鼎。貪擁半牀翡翠,枕歌八寶珊瑚。時間浪珮玉叮咚,特看傳燈金錯落。虎符玉節,門庭甲仗生寒;象板銀箏,磈礧排場熱鬧。終朝謁見,無非公子王孫;逐歲逭遊,盡是侯門戚里。雪兒歌發,驚聞麗曲三千;雲母屏開,忽見金釵十二。鋪荷芰,遊魚沼內不驚人;高挂籠,嬌鳥簾前能對語。那裡解調和爕理,一昧趨諂逢迎。端的笑談起干戈,吹噓驚海岳。假旨令,八位大臣拱手;巧辭使,九重天子點頭。督擇花石,江南淮北盡灾殃;進獻黃楊,國庫民財皆匱竭。當朝無不心寒,列士為之屏息。正是:輦下權豪第一,人間富貴無雙。」

須臾遞畢,安席坐下。一斑兒五個俳優,朝上箏阮琵琶,方响箜篌,紅牙象板,唱了一套正官端正好。端的餘音遶梁,聲清韻美。唱道:

「享富貴,受皇恩,起寒賤,居高位,秉權衡威振京畿。惟君恃寵,把君王媚,全不想存仁義。」

〔滾綉球〕  「起官夫,造水池。與兒孫,買田基。圖求謀,多只為一身之計。縱奸貪,那裡管越瘦吳肥。趨附的,身即榮;觸忤的,令必危。妒量才喜親小輩,只想著復私仇,公道全虧。你將九重天子深瞞眛,致四海生民總亂離,更不道天綱恢恢!」

〔倘秀才〕  「巧言詞,取君王一時笑喜。那裡肯效忠良,使萬國雍熙。你只待顛倒豪傑把世迷,隔靴空癢揉,久症卻行醫,減絕了天理!」

〔滾綉球〕  「你有秦趙事,指鹿心;屠岸賈,縱犬機;待學漢王莽,不臣之意;欺君的董卓燃臍,但行動絃管隨,出門時兵仗圍,入朝中百官悚畏。仗一人假虎張威,望塵有客趨奸黨,借劍無人斬腰賊,一任的忒狂為!」

〔尾聲〕  「金甌底下無名姓,青史編中有是非。你那知爕理陰陽調元氣,那知盜賣江山結外夷。枉辱了玉帶金魚挂蟒衣,受祿無功愧寢食。權方在手人皆懼,禍到臨頭悔後遲。南山竹罄難書罪,東海波乾臭未遺,萬古流傳,教人唾罵你!」

當時酒進三巡,歌吟一套,六員太尉起身,朱太尉親送出來,回到廳,樂聲暫止。管家稟事,各處官員進見。朱太尉令左右擡公案,就在當廳一張虎皮校椅上坐下。分付出來,先令各勳戚中貴仕宦家人吏書人等送禮的進去。須臾打發出來,纔是本衛紀事,南北衙兩廂五所七司,提察譏察,觀察巡察,典牧直駕,提牢指揮,千百戶等官,各有首領,具手本呈遞。然後纔傳出來,叫兩淮、兩浙、山東、山西、關東、關西、河東、河北、福建、廣南、四川十三省提刑官,挨次進見。西門慶與何千戶在第五起上,擡進禮物去。管家又早將何太監拜帖,鋪在書案上,二人立在階下,等上邊叫名字。這西門慶擡頭,見正面五間皆廠廳,歇山轉角,滴水重簷,珠簾高捲上,週圍都是綠欄杆。上面朱紅牌扁,懸著黴宗皇帝御筆,欽賜「執金吾堂」斗大小四個金字,乃是官家耳目牙爪所家輯訪密之所,常人到此者處斬。兩邊六間廂房,階墀寬廣,院宇深沉。朱太尉身著太紅,在上面坐著。須臾,叫到根前,二人應諾陞階,到滴水簷前,躬身參謁,四拜一跪,聽發放。朱太尉道:「那兩員千戶,怎的又叫你家太監送禮來?」令左右收了,分付:「在地方謹慎做官,我這裡自有公道。伺候大朝引奏畢,來衙門中領劄赴任。」二人齊聲應諾。左右喝起去,由左角門出來。剛出大門來,尋見賁四等擡擔出來。正要走,忽聽一人飛馬報來,拿宛紅拜帖來報,說道:「王爺、高爺來了。」西門慶與何千戶閃在人家門裡觀看。須臾,軍牢喝道,人馬圍隨,填街塞巷。只見總督京營八十萬禁軍隴西公王燁,同提督神策御林軍總兵官太尉高俅,俱大紅玉帶,坐轎而至。那各省參見官員,都一湧出來,又不得見了。西門慶與何千戶,良久等了賁四盒擔出來,到于僻處,呼跟隨人拉過馬來,二人方纔騎上馬回寓。正是:

「不因奸佞居台鼎,  那得中原血染衣!」

看官聽說:妾婦索家,小人亂國,自然之道。識者以為將來,數賊必覆天下。果到宣和三年,徽欽北狩,高宗南遷,而天下為虜,有可深痛哉!史官意不盡,有詩為證:

「權姦誤國禍機深,  開國承家戒小人;

六賊深誅何足道,  奈何二聖遠蒙塵。」

畢竟未知後來如何,且聽下回分解:





\end{showcontents}


