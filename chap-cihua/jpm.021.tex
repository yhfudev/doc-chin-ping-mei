%# -*- coding: utf-8 -*-
%!TEX encoding = UTF-8 Unicode
%!TEX TS-program = xelatex
% vim:ts=4:sw=4
%
% 以上设定默认使用 XeLaTex 编译,并指定 Unicode 编码,供 TeXShop 自动识别

%第二十一回 
\chapter{吳月娘掃雪烹茶\KG 應伯爵替花勾使}

「脉脉傷心只自言, 好姻緣化惡姻緣,

回頭恨罵章臺柳, 赧面羞看玉井蓮;

只為春光輕易泄, 遂教鸞鳳等閒遷,

誰人為挽天河水, 一洗前非共往愆。」

話說西門慶從院中歸家,已一更天氣。到家門首,小廝叫開門,下馬,踏著那亂瓊碎玉,到於後邊儀門首。只見儀門半掩半開,院內悄無人聲。西門慶口中不言,心內暗道:「此必有蹺蹊。」于是潛身立於儀門內粉壁前,悄悄試聽覷。只見小玉出來,穿廊下放桌兒。原來吳月娘自從西門慶與他反目,不說話以來,每月吃齋三次,逢七拜斗焚香,夜杳祝禱穹蒼,保估夫主早早回心,齊理家事;早生一子,以為終身之計。西門慶還不知。只見丫鬟小玉放畢香桌兒,少頃,月娘整衣出房,向天井內滿爐炷了香,望空深深禮拜,祝道:「妾身吳氏,作配西門。奈因夫主流戀烟花,中年無子。妾等妻妾六人,俱無所出,缺少墳前拜掃之人;妄夙夜憂心,恐無所托。是以瞞著兒夫,發心每逢夜于星月之下,祝贊三光,要祈保佑兒夫早早回心,棄都繁華,齊心家事。不拘妾等六人之中,早見嗣息,以為終身之計,乃妾之素愿也!」正是:

「私出房櫳夜氣清, 滿庭香霧月微明;

拜天盡訴衷腸事, 那怕傍人隔院聽。」

這西門慶不聽便罷,聽了月娘這一篇言語,口中不言,心內暗道:「原來一向我錯惱了他,原來他一篇都為我的心,倒還是正經夫妻。」一面從粉壁前,扠步走來,抱住月娘。月娘恰燒畢了香,不防是他大雪裡走來來,倒諕一跳,就往屋裡走。被西門慶雙關抱住,說道:「我的姐姐!我西門慶死不曉你,你一片都是為我的;一向錯見了,丟冷了你的心,到今悔之晚矣!」月娘道:「大雪裡,你錯走了門兒了,敢不是這屋裡?你也就差了!我是那不賢良的淫婦,和你有甚情節?那討為你的來!你平白又來理我怎的?咱兩個永世千年,休要見面!」那西門慶把月娘一手拖進房來。燈前看見他家常穿著;大紅潞紬,對衿祆兒,軟黃裙子。頭上戴著貂鼠臥兔兒,金滿池嬌分心,越顯出他;粉粧玉琢銀盆臉,蟬髻鴉鬟楚岫雲。那西門慶如何不愛?連忙與月娘的根前,深深作了個揖,說道:「我西門慶一時昏昧,不聽你之良言,辜負你的好意。正是:『有眼不識荊山玉,拿著頑石一樣看;過後知君子,方纔識好人。』千萬作饒恕我則個!」月娘道:「我又不是你那心上的人兒,凡事投不著你的機會,有甚良言勸你?隨我在這屋裡自生由活,你休要要理他。我這屋裡也難擡放你,趁早與我出去,我不著丫頭攆你!」西門慶首:「我今日平白惹一肚子氣,大雪來家,逕來告訴你。」月娘道:「作氣不作氣,休對我說!我不管你,望著管的你人去說。」西門慶見月娘臉兒不瞧一面,折跌腿裝矮子,跪在地下,殺雞扯脖,口裡姐姐長,姐姐短。月娘看不上,說道:「你真個恁涎臉涎皮的!我叫丫頭進來。」一面叫小玉。那西門慶見那小玉進來,連忙立起來;無計支他出去,說道:「外邊下雪了,一香桌兒,還不收進來罷?」小玉道:「香桌兒頭裡已收進來了。」月娘忍不住笑道:「沒羞的貨!丫頭根前也調個謊兒。」小玉出去,那西門慶又跪下央及。月娘道:「不看世界面上,一百年不理纔好!」說畢,方纔和他坐的一處,教玉筲來捧與他吃了。那西門慶因把今日常家會茶散後,同邀伯爵同到李家,如此這般嚷鬧,告訴一遍:「我叫小廝打了李家一場,被眾人拉勸開了;賭了誓,再不踏院門了。」月娘道:「你躧不躧,不在於我,我是不管你傻才料。你拿晌金白銀包著他,你不去,可知他另接了別的漢子?養漢老婆的營生,你拴住他身,拴不住他心,你長拿封皮封著他也怎的?」西門慶道:「你說的是。」於是脫衣,打發丫鬟出去,要與月娘上床宿歇求歡。月娘道:「教你上炕就撈定兒吃,今日只容你在我床上就勾了;要思想別的事,都不能勾。」那西門慶把那話露將出來,向月娘戲道:「都是你氣的他,中風不語了。」月娘道:「怎的中風不語?」西門慶道:「他既不中風不語,如何大睜著眼說不出話來?」月娘罵道:「好個汗邪的貨!教我有半個眼兒看的上你?」西門慶不由分說,把月娘兩隻白生生腿,扛在肩膊上,那話插入牝中,一任其鶯恣蝶探,殢雨尤雲,未肯即休。正是:

「得多少海棠枝上鶯梭急, 翡翠梁間燕語頻。」

不覺到靈犀一點,美愛無加之處,麝蘭半吐,脂香滿唇。西門慶情極,低聲求月娘叫達達;月娘亦低聲幃暱,枕態有余,妍口呼親親不絕。是夜,兩人雨意雲情,並頭交頸於帳內。正是:

「意恰尚忘垂綉帶, 興狂不管墜金釵。」

有詩為證:「鸞亂釵橫與已曉, 情濃尤復厭通霄;

晚來獨向粧臺立, 淡淡春山不用描。」

當晚夫妻幽歡不題。都表次日大清早晨,孟玉樓走到潘金蓮房中,未曾進門,先叫道:「六丫頭,起來不曾?」春梅道:「俺娘纔起來梳頭哩,三娘進屋裡坐。」玉樓進來,只見金蓮正在粧臺前整掠香雲。因說道:「我有庄事兒來告訴你,你知道不知?」金蓮道:「我在這背哈喇子,誰曉得?」因問:「端得甚麼事?」玉樓道:「他爹昨日二更來家,走到上房裡,和吳家的好了,在他房裡歇了一夜。」金蓮道:「俺每那等勸著,他說一百年,二百年。又知怎的?平白浪〈扌扉〉著自家又好了,又沒人勸他!」玉樓道:「今早我纔知道,俺大丫頭蘭香在廚房內,聽見小廝每說,昨日他爹和應二在院裡李桂兒家吃酒,看出淫婦家甚麼破綻,把淫婦每門窗戶壁都打了。大雪裡著惱來家,進儀門,看見上房燒夜香,想必聽見些甚麼話兒,兩個纔到一答裡。丫頭學說兩個說了一夜話:他爹怎的跪著上房的,叫媽媽,上房的又怎的聲喚擺話的,硶死了!相他這等,就沒的話說,若是別人,又不知怎的說浪!」金蓮接過來說道:「早時與人家做大老婆,還不知怎樣久慣鬼牢成?一個燒夜香,只該默默禱祝,誰家一徑倡揚,使漢子知道了;有這個道理來?又沒人勸,自家暗裡又和漢子好了;硬到底纔好,乾淨假撇清!」玉樓道:「他不是假撇清,他有心也要和,只是不好說出來的。他說他是風老婆不下氣,倒教俺每做分上,怕俺每久後玷言玷語說他,敢說你兩口子話差也,虧俺每說和。那個因院裡著了氣來家,這個正燒夜香,湊了這個巧兒,正是:我親不用媒和證,暗把同心帶結成。如今你我這等較論,休教他買了乖兒去了。你快梳了頭自過去,和李瓶兒說去,咱兩個人,每人出五錢銀子,教李瓶兒拿出一兩來,原為他廢事起來,今日安排一席酒,一者與他兩個把一杯,二者當家兒只當賞雪,耍戲一日,有何不可?」金蓮道:「你說的是,不知他爹,今日有個勾當沒有?」玉樓道:「大雪裡有甚勾當?我來時兩口子還不見動靜,上房門兒纔開,小玉拿水進去了。」這金蓮慌忙梳頭畢,和玉樓同過李瓶兒這邊來。李瓶兒還睡在床上,迎春說:「三娘、五娘來了!」玉樓、金蓮進來,說道:「李大姐,好自在!這咱時還睡,懶龍纔伸腰兒!」金蓮就舒進手去被窩裡摸,見薰被的銀香球,說道:「李大姐生了彈!」這裡掀開被,見他一身白肉,那李瓶兒連忙穿衣不迭。玉樓道:「五姐,休鬼混他。李大姐,你快起來,俺每有庄事來對你說。如此這般,他爹昨日和大姐姐好了,咱每人五錢銀子,你便多出些兒,當初因為你起來。今日大雪裡,只當賞雪,咱安排一席酒兒,請他爹和大姐姐坐坐,好不好?」李瓶兒道:「隨姐姐教我出多少,奴出便了!」金蓮道:「你將就只出一兩兒罷。你秤出來,俺好往後邊,問李嬌兒、孫雪蛾要去。」這李瓶兒一面穿衣纔腳,叫迎春開廂子,拿出銀子,拿了一塊,金蓮上等子秤,重一兩二錢五分。玉樓教金蓮伴著李瓶兒梳頭:「等我後往後邊問李嬌兒孫雪蛾要銀子去。」金蓮看著李瓶兒梳頭洗面。約一個時辰,見玉樓從後邊來說道:「我早知也不幹這個營生!大家的事,相白要他的!小淫婦說:『我是沒時運的人,漢子不再進我屋裡來。我那討銀子?』要著一個錢兒不拿出來!求了半日,只拿出這根銀簪子來,你秤秤,重多少?」金蓮取過等子來秤,只重三錢七分。因問:「李嬌兒怎的?」玉樓道:「李嬌兒初時只說:『沒有,雖是日逐錢打我手裡使,都是扣數的。使多少,交多少,那裡有富餘錢?』教我說了半日:『你當家還說沒錢,俺每那個是有的?六月日頭,沒打你門前過也怎的?大家的事,你不出罷?』教我使性子走出來了,他慌了,使丫頭叫我回去,纔拿出這銀子與我。沒來由,教我恁惹氣剌剌的!」金蓮拿過李嬌兒銀子來,秤了秤,只四錢八分。因罵道:「好個奸倭的淫婦!隨問怎的綁著鬼,也不與人家足數,好歹短幾分,」玉樓道:「只許他家拿黃桿等子秤人的;人問他要,只相打骨禿出來一般,不知叫人罵多少!」一面連玉樓、金蓮共湊了三兩一錢;一面使綉春叫了玳安來。金蓮先問他:「你昨日跟了你爹去,在李家為甚麼著了惱來?」玳安悉把在常時節家會茶,起散的早,邀應二爹和謝爹,同到李家。他鴇子回說不在家,往五姨媽家做生日去了。不想落後爹淨手到後邊,看見粉頭和一個蠻子吃酒不出來,爹就惱了。不由分說,叫俺眾人,把淫婦家門窗戶壁,儘力打了一頓,只要把蠻子粉頭墩鎖在門上。多虧應二爹眾人,再三勸住,爹使性步馬回家;路上發狠,到明日還要擺布淫婦哩!」金蓮道:「賊淫婦!我只道蜜罐兒,長連拿的牢牢的,如何今日也打了?」又問玳安:「你爹真個恁說來?」玳安道:「莫不小的敢哄娘?」金蓮道:「賊囚根子!他不揪不採,也是你爹的表子,許你罵他!想著迎頭兒俺每使著你,只推不得閒,『爹使我往桂姨家送銀子去哩。』叫的桂姨那甜!如今他敗落下來,你主子惱了,連你也叫起他淫婦來了!看我到明日對你爹說,不對你爹說?」玳安道:「耶躒,五娘!這回日頭打西出來,從新又護起他家來了!莫不爹不在路上罵他淫婦,小的敢罵他?」金蓮道:「許你爹罵他便了,原來也許你罵他?」玳安道:「早知五娘麻犯小的,小的也不對娘說。」玉樓便道:「小囚兒,你別要說嘴。這裡三兩一錢銀子,你快和來興兒替我買東西去,如此這般,今日俺每請你爹和你大娘賞雪飲酒。你將就少落我們些兒罷,我教你五娘不告你爹說罷。」玳安道:「娘使小的,小的敢落錢?于是拿了銀子,同來興兒買東西去了。且說西門慶起來,正在上房梳洗。只見大雪裡,來興買了雞鵝下飯,逕往廚房裡去了;玳安便提了一罐金華酒 進來。便問玉筲:「小廝的東西,是那裡的?」玉筲回道:「今日眾娘置酒,請爹娘賞雪。」西門慶道:「金華酒 是那裡的?」玳安道:「是三娘與小的銀子買的。」西門慶道:「阿呀!家裡見放著酒,又去買!」分付玳安:「拿鑰匙,前邊廂房,有雙料茉莉酒 ,提兩壜攙著些這酒吃。」于是在後廳明間內,設石崇錦帳圍屏,放下軸紙梅花暖簾來。爐安獸炭,擺列酒筵。不一時,廚下整理停當,李嬌兒、孟玉樓、潘金蓮、李瓶兒來到,請西門慶、月娘出來。當下李嬌兒把盞,孟玉樓執壺,潘金蓮捧菜,李瓶兒陪跪。頭一鍾先遞了與西門慶,西門慶接酒在手,笑道:「我兒,多有起動,孝順我老人家,長禮兒罷!」那潘金蓮嘴快,插口道:「好老氣的孩兒!誰在這裡替你磕頭哩?俺每磕著你,你站著。楊角葱靠南墻,越發老辣。已定還不跪下哩!也折你的萬年草料,若不是大姐姐帶携你,俺每今日與你磕頭!」于是遞了西門慶,賴了鍾兒。從新又滿滿斟了盞,請月娘轉上,遞與月娘。月娘道:「你每也不和我說,誰和你每平白又費這個心。」玉樓笑道:「沒甚麼。俺每胡亂置了杯水酒兒,大雪與你老公婆兩個散悶而已。姐姐請坐,受俺每一禮兒。」月娘不肯,亦平還下禮去。玉樓道:「姐姐不坐,我每也不起來了。」相讓了平日,月娘纔受了半禮。金蓮戲道:「對姐姐說過,今日姐姐有俺每面上,寬恕了他;下次再無禮,沖撞了姐姐,俺每不管他來!」望西門慶說道:「你裝憨打勢,還在上坐著,還不快下來,與姐姐遞個鍾兒,陪不是哩!」那西門慶只是笑,不動身。良久遞畢,月娘轉下來,令玉筲執壺,亦斟酒與眾姐妹回酒。惟孫雪蛾跪著接酒;其餘都平敍姐妹之情。于是西門慶與月娘居上坐,其余李嬌兒、孟玉樓、潘金蓮、李瓶兒、孫雪蛾并西門大姐,那兩邊打橫。金蓮便道:「李大姐,你也該梯已與大姐姐遞杯酒兒,當初因為你的事起來,你做了老林,怎麼,還恁木木的!」那李瓶兒真個就走下席來,要遞酒。被西門慶攔住,說道:「你休聽那小淫婦兒,他哄你,已是遞過一遍酒罷了;遞幾遍兒?」那李瓶兒方不動了。當下春梅、迎春、玉筲、蘭香,一般兒四個家樂,琵琶、箏、絃子、月琴,一面彈唱起來,唱了一套南石榴花「佳期重會」云云。西門慶聽了,便問:「誰教他唱道一套詞來?」玉筲道:「是五娘分付唱來。」西門慶就看著潘金蓮說道:「你這小淫婦!單管胡枝扯葉的。」金蓮道:「誰教他唱他來?沒的又來纏我。」月娘便道:「怎的不請陳姐夫來坐坐?」一面使小廝前邊請去。不一時,經濟來到,向席上都作了揖,就在大姐下邊坐了。月娘令小玉安放了鍾筯,合家金爐添獸炭,美酒泛羊羔 。正飲酒來,西門慶把眼觀看簾前,那雪如撏綿扯絮,亂舞梨花,下的大了,端的好雪!但見:

「初如柳絮,漸似鵝毛;刷刷似數蟹行沙上,紛紛如亂瓊堆砌間。但行動衣沾六出,頃刻拂滿蜂鬚。似飛還止,龍公試手於起舞之間。新陽力玉女,尚喜於團風之際。襯瑤臺,似玉龍鱗甲遠空飛;飄粉額,如白鶴羽毛接地落。正是:凍合玉樓寒起粟,光搖銀海燭生花。」

吳月娘見雪下在粉壁前太湖石上,甚厚。下席來,教小玉拿著茶罐,親自掃雪,烹江南鳳團雀舌牙茶 ,與眾人吃。正是:

「白玉壺中翻碧浪, 紫金壺內噴清香。」

正吃茶中間,只見玳安進來,報道:「李銘來了,在前邊伺候。」西門慶道:「教他進來。」不一時,李銘朝上向眾人磕下頭去。又打了個軟腿兒,走在傍邊,把兩隻腳兒並立。西門慶便道:「你來得正好,往那裡去來?」李銘道:「小的沒往那去,北邊酒醋門劉公公那裡,教了些孩子,小的瞧了瞧。計掛著爹宅內姐兒每,還有幾段唱未合拍,來伺候。」西門慶就將手內吃的那一盞木穉金燈茶,遞與他吃。說道:「你吃了休去,且唱一套我聽。」李銘道:「小的知道。」一面下邊吃了茶,上來把箏絃調定,頓開喉音,並足朝上,唱了一套冬景絳都春,「寒風布野」云云。唱畢,西門慶令李銘近前,賞酒與他吃。教小玉拿團靶勾頭雞膆壺,滿斟窩兒酒,傾在銀法郎桃兒鍾內;那李銘跪在地下,滿飲三杯。西門慶又在桌上,拿一碟鼓蓬蓬白麵蒸餅 ,一碗韮菜酸笋蛤蜊湯 ,一盤子肥肥的大片水晶鵝 ,一碟香噴噴曬乾的巴子肉 ,一碟子柳蒸的勒養魚,一碟奶罐子酪酥伴的鴿子錐兒,用盤子托著與李銘。那李銘走到下邊,三扒兩咽,吞到肚內,舔的盤兒乾乾淨淨,用絹兒把嘴兒抹了,走到上邊,把身子直豎豎的靠著槅子站立。西門慶因把昨日桂姐家之事,告訴一遍。李銘道:「小的並不知道一字。一向也不過那邊去;論起來不干桂姐事,都是俺三媽幹的營生。爹也別要惱他,等小的見他說他便了。」當日飲酒到一更時分,妻妾俱合家歡樂。先是陳經濟、大姐徑往前邊去了。落後酒闌,西門慶又賞李銘酒,打發出門,分付;「你到那邊,休說今日在我這裡。」李銘道:「爹分付,小的知道。」西門慶令左右送他出門,關上大門,于是妻妾各散。西門慶還在月娘上房歇了。有詩為證:

「赤繩緣分莫疑猜, 扊扅夫妻共此懷,

魚水相逢從此始, 兩情愿保百年諧。」

都說次日雪晴,應伯爵、謝希大受了李家燒鵝 瓶酒,恐怕西門慶動意擺布他家,敬來邀請西門慶進裡邊陪禮。月娘早晨梳粧畢,正和西門慶在房中吃餅,只見小廝玳安來說:「應二爹和謝爹來了,在前廳上坐著哩。」西門慶放下餅,就要往前走。月娘:「兩個勾使鬼,又不知來做甚麼?你亦發吃了出去,教他外頭挨着去。慌的恁沒命的一般,往外走怎的?大雪裡又不知勾了那去?」西門慶道:「你教小廝把餅拿了前邊,我和他兩個吃罷。」說著,起身往外來。月娘分付:「你和他吃了,別要信着,又勾引的往那去了。大雪裡家裡坐着罷,今日孟三姐晚夕上壽哩。」西門慶道:「我知道。」于是與應、謝二人,相見聲諾,說道:「哥昨日着惱家來了,俺每甚是怪他家:『從前已往,哥在你家使錢費物,雖故一時不來,休要改了腔兒纔好,許你家粉頭背地偷接蠻子。冤家路兒窄,又被他親眼看見,他怎的不惱!休說哥惱,俺每心裡也看不過!』儘力說了他娘兒幾句,他也甚是都沒意思。今日早請了俺兩個到他家,娘兒每哭哭啼啼跪着,恐怕你動意,置了一杯水酒兒,好歹請你進去,陪個不是。」西門慶道:「我也不動意。我再也不進去了。」伯爵道:「哥惱有理,但說起來,也不干桂姐事;這個丁二官兒,原先是他姐姐桂卿的孤老,也沒說要請桂姐。只因他父親債船,搭在他鄉里陳監生船上,纔到了不多兩日,這陳監生號兩淮,乃是秘山省陳參政的兒子;丁二官見拿了十兩銀子,在他家擺酒請陳監生。纔送這銀子來,不想你我到了他家,就慌了,躲不及,把個蠻子藏在後邊,被你看見了;實告,不曾和桂姐沾身。今日他娘兒每賭身發呪,磕頭禮拜,央俺二人,好歹請哥到那裡,把這委曲情由,也對哥表出,也把惱解了一半。」西門慶道:「我已下對房下賭誓,再也不去,又惱甚麼?你上覆他家,到不消費心。我家中今日有些小事,委的不得去。」慌的二人一齊跪下,說道:「哥甚麼話?不爭你不去,既他央了俺兩個一場,顯的我每請哥不的。哥去到那裡,略坐坐兒,就來也罷!」當下二人,死告活央,說的西門慶肯了。不一時,放桌兒,留兩人吃餅。須更,吃畢,令玳安取衣服去。月娘正和孟玉樓坐著,便問玳安:「你爹要往那裡去?」玳安道:「小的不知,爹只教小的取衣服。」月娘罵道:「賊囚根子!你還瞞着我不說,你爹但來晚了,都在你身上,等我和你答話!今日你三娘上壽哩。不教他早些來,休要那等到那黑天暗地的,我自打你這賊囚根子。」玳安道:「娘打小的,管小的甚事?」月娘道:「不知怎的,聽見他這老子每來,恰似奔命的一般,行吃着飯,丟下飯碗,往外不迭。又不知勾引遊營撞屍,撞到多咱纔來!」那時十一月廿六日,就是孟玉樓壽日,家中置酒等候不題。且說西門慶被兩個邀請到院裡,李家又早堂中置了一席齊整酒餚,叫了兩個妓女彈唱。李桂姐與桂卿兩個,打扮迎接,老虔婆出來,跪着陪禮,姐兒兩個遞酒。應伯爵、謝希大在傍打諢要笑,說砂磴語兒,向桂姐道:「還虧我把嘴頭上皮也磨了半邊去,請了你家漢子來。就不用着人兒,連酒兒也不替我遞一杯兒,自認你家漢子!剛才若他撅不來,休說你哭瞎了你眼,唱門詞兒,到明日諸人不要你。只我好說話兒,將就罷了。」桂姐罵道:「怪應花子,汗邪了你!我不好罵出來的。可可兒的我唱門詞兒來?」應伯爵道:「你看賊小淫婦兒!念了經,打和尚。往後不省人了!他不來!慌的那腔兒;這回就翅膀毛兒乾了!你過來,且與我個嘴溫溫寒着!」于是不由分說,摟過脖子來,就親了個嘴。桂姐笑道:「怪攘刀子的!看推撒了酒在爹身上!」伯爵道:「小淫婦兒,會喬張致的,這回就疼漢子。『看撒了爹身上酒!』叫的爹那甜;我是後娘養的?怎的不叫我一聲兒?」桂姐道:「我叫你是我的孩子兒!」伯爵道:「你過來,我說個笑話兒你聽:一個螃蟹,與田雞結為弟兄,賭跳過水溝兒去便是大哥;田雞幾跳,跳過去了;螃蟹方欲跳,撞遇兩個女子來汲水,用草繩兒把他拴住,要打了水,帶回去,臨行忘記了,不將去;田雞見他不來。過來看他,說道:『你怎的就不過去了?』蟹云:『我過的去,倒不吃兩個小淫婦捩的恁樣了!』于是,兩過一齊趕着打,把西門慶笑的要不的。不說這裡花攢錦簇,調笑頑耍不題。且說家中吳月娘一者置酒回席,二者又是玉樓上壽,吳大妗、楊姑娘,并兩個姑子,都在上房裡坐的。看看等到日落時分,不見西門慶來家,急的月娘要不的。只見金蓮拉着李瓶兒,笑嘻嘻向月娘說道:「大姐姐,他這咱不來,俺每往門首,瞧他瞧去。」月娘道:「耐煩瞧他怎的?」金蓮又拉玉樓說:「咱三個打夥兒走走去。」玉樓道:「我這裡聽大師父說笑話兒哩,等聽說完了這個笑話兒咱去。」那金蓮方住了腳,圍住兩個姑子,聽說笑話兒哩,說:「俺每只好葷笑話兒,素的休要打發出來。」月娘道:「你每由他說,別要搜求他。」金蓮道:「大姐姐,你不知大師父會好說笑話兒!前者那一遭來,俺每在後邊,奈何着他,說了好些笑話兒。」因說道:「大師父,你有快些說。」那王姑子,不慌不忙,坐在炕上,說:「一個人走至中途,撞見一個老虎,要吃他。此人云:『望你饒我一命,家中只有八十歲老母,無人養活。不然向我家去,有一豬與你吃罷!』那老虎果饒他,隨他到家,與母親說;母親正磨豆腐,捨不的那豬,對兒子:『把幾塊豆腐與他吃罷!』兒子云:『娘娘,你不知他,平日不吃素的。』」金蓮道:「這個不好,俺每耳朵內不好聽素,只好聽葷的。」王姑子又道:「一家三個媳婦兒,與公公上壽。先該大媳婦遞酒,說:『公公好相一員官。』公公云:『我如何相官?』媳婦云:『坐在上面,家中大小都怕你,如何不相官?』次該二媳婦上來遞酒,說:『公公相虎威皂隸。』公公曰:『我如何相虎威皂隸?』媳婦云:『你喝一聲,家中大小都吃一驚,怎的不相皂隸?』公公道:『你說的我好。』該第三媳婦遞酒,上來說:『公公也不相官,也不相皂隸。』公公道:『都相甚麼?』媳婦道:『公公相個外郎!』公公道:『我如何相外郎?』媳婦云:『不相外郎,如何六房裡都串到?』」把眾人都笑了。金蓮道:「好禿子!把俺每都說在裡頭。那個外郎,敢恁大胆?許他在各房裡串。俺每就打斷他那狗禿的下截來!」說罷,金蓮、玉樓、李瓶兒同來到前邊大門首瞧西門慶,不見到。玉樓問道:「今日他爹大雪裡不在家,那裡去了?」金蓮道:「我猜他一定往院中李桂兒那淫婦家去了。」玉樓道:「他打了一場,和他惱了;賭了誓,再不去了。如何又去?咱每賭甚麼?管情不在他家。」金蓮道:「李大姐做證見,你敢和我拍手麼?我說今日往他家去了。前日打了淫婦家,昨日李銘那王八,先來打探子兒;今日應二和姓謝的,大清早晨,勾使鬼走來勾了他去了;我猜老虔婆和淫婦,舖謀定計叫了去。不知怎的撮弄,陪着不是,還要回爐復帳。不知涎纏到多咱時候,有個來的成來不成?大姐姐還只顧等着他!」玉樓道:「就不來,小廝他該來家回一聲兒。」正說着,只見賣瓜子的過來,兩個且在門首買瓜子兒磕。忽見西門慶從東來了,三個往後跑不迭。西門慶在馬上,教玳安先頭裡走:「你瞧是誰在大門首?」玳安走了兩步,說道:「是三娘、五娘、六娘,在門首買瓜子哩。」良久,西門慶到家下馬,進入後邊儀門首。玉樓、李瓶兒先去上房,報月娘去了;獨有金蓮藏在粉壁背後黑影裡。西門慶撞見,諕了一跳,說道:「怪小淫婦兒,猛可諕我一跳!你每在門首做甚麼來?」金蓮道:「你還敢說哩!你在那裡,這時纔來,教娘每只顧在門首等着你。」良久,西門慶在房中,月娘安酒餚,端端整整,擺在桌上。教玉筲執壺,大姐遞酒,先遞了西門慶酒,然後眾姐妹都遞酒完了,安席坐下。春梅、迎春,下邊彈唱。吃了一回,都收下去。從新擺上玉樓上壽的酒餚,并四十樣細巧各樣的菓碟兒上來。壺斟美釀,盞泛流霞。讓吳大妗子上坐。吃到起更時分,大妗子吃不多酒,歸後邊去了。止是吳月娘同眾姐妹,陪西門慶擲骰,猜枚行令。輪到月娘根前,月娘道:「既要我行令,照依牌譜上飲酒。一個牌兒名,兩個骨牌,合西廂一句。」月娘先說個:「擲個六娘子醉楊妃,落了八珠環,遊絲兒抓住荼蘪架。」不犯。該西門慶擲:「我虞美人見楚漢爭鋒,傷了正馬軍,只聽見耳邊金鼓連天震。」果然是個正馬軍,吃了一杯。該李嬌兒,說:「水仙因二士入桃源,驚散了花開蝶滿枝,只做了落紅滿地,胭脂冷。」不遇。次該金蓮擲,說道:「鮑老兒臨老入花叢,壞了三綱五常,問他個非奸做賊拿。」果然是個三綱五常,吃了一杯酒。輪該李瓶兒擲,說:「端正好,搭梯望月,等到春分晝夜停,那時節隔墻兒險化做望夫山。」不遇。該孫雪蛾說:「麻郎兒見群鴉打鳳,絆住了折腳雁,好教我兩下裡做人難。」不遇。落後該玉樓完令,說道:「念奴嬌醉扶定四紅沉,拖着錦裙襴,得多少春風夜月銷金帳。」正擲了四紅沉。月娘滿,令小玉:「斟酒與你三娘吃。」說道:「你吃三大杯才好!今晚你該伴新郎宿歇。」因對李嬌兒、金蓮眾人說:「吃畢酒,咱送他兩個歸房去。」金蓮道:「姐姐嚴令,豈敢不依!」把玉樓羞的要不的。少頃,酒闌,月娘等相送西門慶到玉樓房門首方回。玉樓讓眾人坐,都不坐。金蓮便戲玉樓道:「我兒,兩口兒好好睡罷!你娘明日來看你,休要淘氣!」因向月娘道:「親家,孩兒小哩!看我面上,凡是耽待些兒罷!」玉樓道:「六丫頭!你老米醋,挨着做。我明日和你答話!」金蓮道:「我媒人婆上樓子。老娘好耐驚怕兒!」玉樓道:「我的兒,你再坐回兒不是?」金蓮道:「俺每是外四家兒的門兒的外頭的人家。」於是和李瓶兒、西門大姐,一路去了。剛走到儀門首,不想李瓶兒被地滑了一交。這金蓮遂怪喬叫起來,說道:「這個李大姐,只相個瞎子,行動一磨趄子就倒了;我搊你去,倒把我一隻腳⅚塀茶在雪裡,把人的鞋也⅚塀柴泥了!月娘聽見,說道:「就是儀門首那堆子雪,我分付了小廝兩遍,賊奴才,白不肯擡,只當還滑倒了。」因叫小玉:「你打個燈籠,送送五娘、六娘去。」西門慶在房裡向玉樓道:「你看賊小淫婦兒!躧在泥裡,把人絆了一交。他還說人跳泥了他的鞋;恰是那一個兒,就沒些嘴抹兒。恁一個小淫婦!昨日教丫頭每平白唱佳期重會,我就猜是他幹的營生。」玉樓道:「佳期重會,是怎的說?」西門慶道:「他說吳家的不是正經相會,是私下相會。恰似燒夜香有意等着我一般!」玉樓道:「六姐他諸般曲兒倒都知道,俺每都不曉的。」西門慶道:「你不知道這淫婦,單管咬群兒。」不說西門慶在玉樓房中宿歇不題。單表潘金蓮、李瓶兒兩個走着說話,行叫李大姐、花大姐一路兒走到儀門,大姐便歸前邊廂房中去了;小玉打着燈籠,送二人到花園內。金蓮已帶半酣,接着李瓶兒:「二娘,我今日有酒了,你好歹送到我房裡。」李瓶兒道:「姐姐你不醉。」須臾,送到金蓮房內。打發小玉回後邊,留李瓶兒坐吃茶。金蓮又道:「你說你那咱不得來,虧了誰?誰想今日咱姐妹在一個跳板兒上走,不知替你頂了多少瞎缸,教人背地好不說我!奴只行好心,自有天知道罷了!」李瓶兒道:「奴知道姐姐費心,恩當重報,不敢有忘!」金蓮道:「得你知道,纔說話了。」不一時,春梅拿茶來吃了,李瓶兒告辭歸房,金蓮獨自歇宿,不在話下。正是:

「若得始終無悔吝, 纔生枝節便多端。」

畢竟未知後來何如,且聽下回分解:

