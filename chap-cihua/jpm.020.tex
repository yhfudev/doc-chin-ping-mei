%# -*- coding: utf-8 -*-
%!TEX encoding = UTF-8 Unicode
%!TEX TS-program = xelatex
% vim:ts=4:sw=4
%
% 以上设定默认使用 XeLaTex 编译,并指定 Unicode 编码,供 TeXShop 自动识别

%第二十回 
\chapter{孟玉樓義勸吳月娘\KG 西門慶大鬧麗春院}

\begin{showcontents}{}



「在世為人保七旬,  何勞日夜弄精神,

世事到頭終有悔,  浮華過眼恐非真;

貧窮富貴天之命,  得失榮華隙里塵,

不如且放開懷樂,  莫使蒼然兩鬢侵。」

話說西門慶在房中,被李瓶兒幾句柔情軟話。感觸的回嗔作喜,拉他起來,穿上衣裳;兩個相摟相抱,極盡綢繆。一面令春梅進房放桌兒,往後邊取酒去。且說金蓮和孟玉樓從西門慶進他房中去,站在角門首,打聽消息。他這邊門又閉著,止是春梅一人在院子裡伺候。金蓮拉玉樓兩個,打門縫兒望裡張覷,只見房中掌著燈燭,裡邊說話,都聽不見。金蓮道:「俺不如春梅賊小肉兒,他倒聽得伶俐。」那春梅便在穿下潛聽一回。春梅走過來,金蓮悄問他房中怎的動靜?這春梅聽了,便隔門告訴與二人說:俺爹怎的教他脫衣裳跪著。他不脫,爹惱了,抽了他幾馬鞭子。金蓮問道:「打了他,他脫了不曾?」春梅道:「他見爹惱了,纔慌了,就脫了衣裳,跪在地平上。爹如今問他話哩!」玉樓恐怕西門慶聽見,便道:「五姐,咱過那邊去罷。」拉金蓮來西角門首站立。那時八月二十頭,月色纔上來。站在黑裡頭,金蓮吃瓜子兒,兩個一處說話,等著春梅出來問他話。潘金蓮便向玉樓道:「我的姐姐,說好食菓子,一心只要來這裡。頭兒沒動,下馬威討了這幾下在身上!俺這個好不順臉的貨兒,你著他順順兒,他倒罷了。屬扭孤兒糖 的,你扭扭兒也是錢,不扭也是錢。想著先前乞小婦奴才壓柱造舌我那一行院,我陪下十二分小心,還乞他奈何的我那等哭哩!姐姐你來了幾時,還不知他性格兒哩。」二人正說話之間,少頃只聽開的角門響,春梅出來,一直逕往後邊走。不妨他娘站在黑影處叫他,問道:「小肉兒,那去?」那春梅笑著,只顧走。那金蓮道:「怪小肉兒,你過來,我問你話。慌走怎的?」那春梅方纔立住了腳,方說:「如此這般,他哭著,對俺爹說了許多話說哩。爹喜歡抱起來,令他穿上衣裳,教我放了桌兒,如今往後邊取酒去。」金蓮聽了,便向玉樓說道:「賊沒廉恥的貨!頭裡那等雷聲大雨點小,打哩亂哩。及到其間,也不怎麼的。我猜也沒的想,管情取了酒來,教他遞!賊小肉兒,沒他房裡丫頭,你替他取酒去。到後邊,又叫雪娥那小婦奴才,〈毛皮〉聲浪顙,我又聽不上!」春梅道:「爹使我,管我事!」于是笑嘻嘻去了。金蓮道:「俺的小肉兒,正經使著他,死了一般懶待動彈;不知怎的?聽見幹貓兒頭差事,鑽頭覓縫,幹辨了要去,去的那快!見他房裡兩個丫頭,你替他走,管你腿事!賣蘿蔔的跟著鹽擔子走,好個閒嘈心的小肉兒!」玉樓道:「可不怎的!俺大丫頭蘭香,我正使他做活兒,他想伏實只不;他爹使他行鬼頭兒,聽人的話兒,你看他的,走的那快!」正說著,只見玉筲自後邊驀地走來,便道:「三娘還在這裡?我來接你來了。」玉樓:「怪狗肉,諕我一跳!」因問:「你娘知道你來不曾?」玉筲道:「我打發娘睡下這一日了,我來前邊瞧瞧,剛纔看見春梅後邊要酒果去了。」因問:「俺爹到他屋裡,怎樣個動靜兒?」金蓮接過來道:「進他屋裡去,尖頭醜婦,蹦到毛司牆上,齊頭故事。」玉筲又問玉樓,玉樓便一一告他說。玉筲道:「三娘,真個教他脫了衣裳跪著,打了他五馬鞭子來?」玉樓道:「你爹因他不跪,纔打他。」玉筲道:「帶著衣服打來?去了衣裳打來?虧他瑩白的皮肉兒上,怎麼挨得?」玉樓笑道:「怪小狗肉兒!你倒替古人躭憂!」正說著,只見春梅和小玉取了酒菜來。春梅拿著酒,小玉拿著方盒,逕往李瓶兒那邊去。金蓮道:「賊小肉兒,不知怎的,聽見幹恁個勾當兒,雲端裡老鼠,天生的耗!」分付:「快送了來,教他家丫頭伺候去。你不要管他,我要使你哩!」那春梅笑嘻嘻,同小玉進去了,一面把酒菜擺在桌上,這春梅和小玉就出來了。只是迎春、綉春在房答應。玉樓、金蓮問了他話。玉筲道:「三娘,咱後邊去罷。」二人一路去了。金蓮教春梅關上角門,歸進房來,獨自宿歇,不在話下。正是:「可惜團圞今夜月,  情光咫尺別人圓。」不說金蓮獨宿,單表西門慶與李瓶兒兩個,相憐相愛,飲酒說話到半夜,方纔被伸翡翠,枕設鴛鴦,上牀就寢。燈光掩映,不啻鏡中之鸞鳳和鳴;香氣薰籠,好似花間之蝴蝶對舞。正是:

「今宵勝把銀缸照,  祇恐相逢是夢中。」

有詞為證:

「淡畫眉兒斜插梳,不忻拈弄倩工夫,雲窗霧閣深深許,蕙性蘭心款款呼。  相憐愛態情人扶,神仙標格世間無,從今罷卻相思調,美滿恩情錦不如。」

兩個睡到次日飯時,李瓶兒恰待起來,臨鏡梳頭。只見迎春後邊拿將來,四小碟瓶甜醬瓜茄 ,細巧菜蔬,一鷗頓爛鴿子鶵兒,一甌黃韭乳餅 ,并醋燒白菜 ,一碟火燻肉 ,一碟紅糟鰣魚 ,兩銀廂甌兒,白生生軟香稻粳米飯兒 ,兩雙牙筯。婦人先漱了口,陪西門慶吃了上半盞兒。就教迎春:

「昨日剩的銀壺裡金華酒 篩來。」拿甌子陪著西門慶,每人吃了兩甌子,方纔洗臉梳粧。一面開箱子,打點細軟首飾衣服,與西門慶過目。拿出一百顆西洋珠子與西門慶看,原是昔日梁中書家帶來之物。又拿出一件金廂鴉青帽頂子,說是過世老公公的。起下來上等子秤,四錢八分重;李瓶兒教西門慶拿與銀匠,替他做一對墜子。又拿出一頂金絲䯼髻,重九兩。因問西門慶:「上房他大娘眾人,有這䯼髻沒有?」西門慶道:「他每銀絲䯼髻倒有兩三頂;只沒編這䯼髻。」婦人道:「我不好帶出來的,你替我拿到銀匠家毀了,打一件金丸鳳墊根兒,每個鳳嘴啣一掛珠兒;剩下的再替我打一件,照依他大娘,正面戴金廂玉觀音,滿池嬌分心。」西門慶收了,一面梳頭洗臉,穿了衣服出門。李瓶兒分付:「那邊房子裡沒人,你好歹過去看看,委付個人兒看守,替了小廝天福兒來家使喚。那老馮老行貨子,啻啻磕磕的,獨自在那裡,我又不放心!」西門慶道:「你分付,我知道了。」袖著䯼髻和帽頂子出門,一直往外走。不防金蓮鬅著頭,還未梳洗,站在東角門首,叫道:「哥,你往那裡去?這咱纔出來,看雀兒撞兒眼!」那西門慶道:「我有勾當去。」婦人道:「怪行貨子!你還來,慌走怎的?我和你說話。」那西門慶見他叫的緊,只得回來。被婦人引到房中,婦人便坐在椅子上,把他兩隻手拉,說道:「我不好罵出來的!怪火燎腿三寸貨,那個拿長鍋鑊吃了你?慌往外搶的是些甚的?你過來,我且問你。」西門慶道:「罷麼!小淫婦兒!只顧問甚麼!我有勾當哩,等我回來說。」說著,往外走。婦人摸見他袖子裡重重的,道:「是甚麼?拿出來我瞧瞧!」西門慶道:「是我的銀子包。」婦人不信。伸手進去袖子裡就掏,掏出一頂金絲䯼髻來。說道:「這是他的䯼髻,你拿那去?」西門慶道:「他問我,你每沒有這䯼髻;到銀匠家替他毀了,打兩件頭面戴。」金蓮問道:「這䯼髻多少重?他要打甚麼?」西門慶道:「這䯼髻重九兩,他要打一件九鳳甸兒,一件照依上房戴的。正面那一件玉觀音,滿池嬌分心。」金蓮道:「一件九鳳甸兒,滿破使了三兩五六錢金子勾了;大姐姐那件分心,我秤只重一兩六錢;把剩的,好歹你替我照依他,也打一件九鳳甸兒。」西門慶道:「滿池嬌他要揭實枝梗的。」金蓮道:「就是揭實枝梗,使了三兩金子滿篡,綁著鬼,還落他二三兩金子,勾打個甸兒了。」西門慶笑罵道:「你這小淫婦兒!單管愛小便益兒,隨處也搯個尖兒。」金蓮道:「我兒,娘說的話,你好歹記著。你不替我打將來,我和你答話!」那西門慶袖了䯼髻,笑著出門。金蓮戲道:「哥兒,你幹上了。」西門慶道:「我怎的幹上了?」金蓮道:「你既不幹,昨日那等雷聲大雨點小,要打著教他上吊。今日拿出一頂䯼髻來,使的你狗油嘴鬼推磨,不怕你不走!」西門慶笑道:「這小淫婦兒,單只管胡說!」說著往外走了。都說吳月娘和孟玉樓、李嬌兒在房中坐的,忽聽見外邊小廝一片聲尋來旺兒,尋不著。只見平安來掀簾子,月娘便問:「尋他做甚麼?」平安道:「爹緊等著哩。」月娘半日纔說:「我使了他有勾當去了。」原來月娘早辰分付下他,往王姑子庵裡,送香油 白米去了。平安道:「小的回爹,只說娘使了他有勾當去了。」月娘罵道:「怪奴才!隨你怎麼回去!」平安諕的不敢言語一聲兒,往外走了。月娘便向玉樓眾人說道:「我開口,又說我多管;不言語,我又鱉的慌!一個人也拉剌將來了,那房子賣吊了就是了。平日扯淡,搖鈴打鼓的,看守甚麼?左右有他家馮媽媽子在那裡,再派一個沒老婆的小廝,晚夕同在那裡上宿睡就是了。怕走了那房子也怎的?作養娘抱,巴巴叫來旺兩口子去!自他媳婦子七病八病,一時病倒了在那裡,上床誰扶持他?」玉樓便道:「姐姐在上,不該我說。你是個一家之主,不爭你與他爹兩個不說話,就是俺每不好張主的,下邊孩子們也沒投奔。他爹這兩日隔二騙三的,也甚是沒意思!看姐姐恁的,依俺每一句話兒,與他爹笑開了罷。」月娘道:「孟三姐,你休要起這個意。我又不曾和他兩個嚷鬧,他平日的使性兒,那怕他使的那臉格,休想我正眼看他一眼兒!他背地對人罵我不賢良的淫婦,我怎的不賢良的來?如今聳六十個在屋裡,纔知道我不賢良!自古道:『順情說好話,幹直惹人嫌。』我當初大說攔你,也只為好來。你既收了他許多東西,又買了房子,今日又圖謀他老婆,就著官兒,也看喬了;何況他孝服不滿,你不好娶他的。誰知道人在背地裡,把圈套做的成成的,每日行茶過水,自瞞我一個兒,把我合在缸底下。今日也推在院裡歇,明日也推在院裡歇,誰想他只當把個人兒歇了。家裡來端的好在院裡歇!他自吃人在他根前那等花麗狐哨,喬龍盡虎的,兩面刀哄他,就是千好萬好了。似俺每這等依老實,苦口良言,著他理你理兒!你倒如今,反被為仇。正是:前車倒了千千輛,後車倒了亦如然,分明指與平川路,錯把忠言當惡言!你不理我,我想求你?一日不少我三頓飯。我只當沒漢子,守寡在這屋裡。隨我去,你每不要管他。」幾句話,說的玉樓眾人訕訕的。良久,只見李瓶兒梳粧打扮,上穿大紅遍地金對衿羅衫兒,翠藍拖泥粧花羅裙。迎春抱著銀湯瓶,綉春拿著茶盒,走來上房,與月娘眾人遞茶。月娘叫小玉安放座兒與他坐。落後孫雪娥也來到,都遞了茶,一處坐的。潘金蓮嘴快,便叫道:「李大姐,你過來,與大姐下個禮兒。實和你說了罷,大姐姐和他爹,那些時兩個不說話,因為你來!俺們剛纔替你勸了恁一日;你改日安排一席酒兒,央及央及大姐姐,教他兩個老公婆笑開了罷。」李瓶兒道:「姐姐分付,奴知道。」于是向月娘面前,花枝招展,綉帶飄票,插燭也似磕了四個頭。月娘:「李大姐,他哄你哩!」又道:「五姐,你每不要來攛掇。我已是賭下誓,就是一百年,也不和他在一答兒哩!」以此眾人再不敢復言。金蓮在傍把拿抿子,與李瓶兒抿頭。見他頭上戴著一付金玲瓏草虫兒頭面,并金纍絲松竹梅歲寒三友梳背兒。因說道:「李大姐,你不該打這碎草虫頭面,只是有些抓住了頭髮。不如大姐姐頭上戴的這金觀音滿池嬌,是揭實枝梗的好。」這李瓶兒老實,就說道:「奴也照樣兒要教銀匠打恁一件哩!」落後小玉、玉筲來根前遞茶,都亂戲他。先是玉筲問道:「六娘你家老公公,當初在皇城內那衙門來?」李瓶兒道:「先生惜薪司掌廠,御前班直,後陞廣南鎮守。」玉筲笑道:「嗔道你老人家昨日挨的好柴!」小玉又道:「去年城外落鄉,許多里長老人好不尋你,教你往東京去。」婦人不知道甚麼,說道:「他尋我怎的?」小玉笑道:「他說你老人家會告的好水災。」玉筲又道:「你老人家鄉里媽媽拜千佛,昨日磕頭磕勾了。」小玉又說道:」朝廷昨日差了四個夜不收,請你老人家往口外和番,端的有這話麼?」李瓶兒道:「我不知道。」小玉笑道:「說你老人家會叫的好達達!」把玉樓、金蓮笑的不了。月娘便道:「怪臭肉,每幹你那營生去,只顧奚落他怎的?」于是把個李瓶兒羞的臉上一塊紅,一塊白,站又站不得,坐又坐不住,半日回房去了。良久,西門慶進房來,回他顧銀匠家打造生活。就與他計較,明日發柬,二十五日請官客吃會親酒,少不的拿帖兒請請花大哥。李瓶兒道:「他娘子三日來,再三說了。也罷,你請他請罷。」李瓶兒又說:「那邊房子左右有老馮看守,你這裡再叫一個,和天福兒輪著晚夕上宿就是,不消教旺官去罷,上房姐姐說他媳婦兒有病,去不的。」西門慶道:「我不知道。」即叫平安近前分付:「你和天福兒兩個輪,一遞一日,獅子街房子裡上宿。」不在言表。話休饒舌,不覺到二十五日,西門慶家中吃會親酒,插花筵席,四個唱的,一起雜耍步戲。頭一席花大舅、吳大舅;第二席是吳二舅、沈姨夫;第三席,應伯爵、謝希大;第四席,祝日念、孫天化;第五席,常時節、吳典恩;第六席,雲離守、白來創;西門慶主位,其餘傅自新、賁地傳、女婿陳經濟,兩邊列位。先是李桂姐、吳銀兒、董玉仙、韓金釧兒從晌午時分,坐轎子就來了,在月娘上房裡坐。官客在新蓋捲棚內坐的吃茶,然後到齊了。大廳上,坐席上都有桌面,某人居上,某人居下。先吃小割海青捲兒 ,八寶攢湯 ,頭一道割燒鵝 大下飯。樂人撮撮弄雜耍回數,就是笑樂院本,下去。李銘、吳惠兩個小優,上來彈唱,問省清吹,下去。四個唱的出來,筵外遞酒。應伯爵在席上,先開言說道:「今日哥的喜酒,是兄弟不當斗膽。請新嫂子出來,拜見拜見,足見親厚之情。俺每不打緊,花太尊親,并二位老舅,沈姨丈在上,今日為何來?」西門慶道:「小妾醜陋,不堪拜見,免了罷。」謝希大道:「哥,你這話難說,當初已言在先,不為嫂子,俺每怎麼見來?何況這個嫂子,見有我尊親花大哥在上,先做友,後做親,又不同別人。請出來見見,怕怎的?」那西門慶笑,不動身。應伯爵道:「哥,你不要笑,俺每都拿著拜見錢在這裡,不白教他出來見。」西門慶道:「你這狗材,單管胡說!」乞他再三逼迫不過,叫過玳安來,教他後邊說去。半日,玳安出來回說:「六娘道,免了罷。」應伯爵道:「就是你這小狗骨禿兒的鬼!你幾時往後邊去,就來哄我?賭幾個誓,真個我就後邊去了!」玳安道:「小的莫不哄應二爹,二爹進去問不是?」伯爵道:「你量我不敢進去?左右花園中熟景,好不好我走進去,連你那幾位娘,都拉了出來。」玳安道:「俺家那大揉廝狗,好不利害!倒沒的把應二爹下半截撕下來。」怕爵故意下席,趕著玳安踢兩腳,笑道:「好小狗骨禿兒!你傷的我好!趁早與我後邊請去。請不將來,可二十欄杆。」把眾人、四個唱的都笑了。那玳安到下邊,又走來立著,把眼看著他爹不動身。西門慶無法可處,只得叫過玳安,近前分付:「對你六娘說,收拾了出來見見罷!」那玳安去了半日出來,復請了西門慶進去。然後纔把腳下人趕出去,關上儀門。四個唱的,都往後邊彈樂器,簇擁婦人上拜。孟玉樓、潘金蓮,百方攛掇,替他抿頭戴花翠,打發他出來。廳上又早鋪下錦毡綉毯,麝蘭靉靆。絲竹和鳴,四個唱的,導引前行。婦人身穿大紅五彩通袖羅袍兒,下著金枝線葉沙綠百花裙。腰裡束著碧玉女帶,腕上籠著金壓袖。胸前項牌纓落,裙邊環珮玎璫,頭上珠翠堆盈,鬢畔寶釵半卸。紫瑛金環,耳邊低掛;珠子挑鳳,髻上雙插。粉面宜貼翠花鈿,湘裙越顯紅鴛小。恍似嫦娥離月殿,猶如神女到筵前。四個唱的,琵琶箏絃,簇擁婦人,花枝招颭,綉帶飄飄,望上朝拜。慌的眾人都下席來還禮不迭。都說孟玉樓、潘金蓮、李嬌兒,簇擁著月娘,都在大廳軟壁後聽覷,聽見唱喜得功名完,遂唱到天之配合「一對兒如鸞似鳳,夫共妻。」直到「笑吟吟慶喜。高擎著鳳凰杯,象板銀箏間玉笛。列杯盤,水陸排佳會。」直到「永團圓,世世夫妻。」根前金蓮向月娘說道:「大姐姐,你聽唱的,小老婆今日不該唱這一套,他做了一對魚水團圓,世世夫妻,把姐姐放到那裡?」那月娘雖故好性兒,聽了這兩句,未免有幾分動意,惱在心中。又見應伯爵、謝希大這夥人,見李瓶兒出來上拜,恨不的生出幾個口來誇獎奉承,說道:「我這嫂子,端的寰中少有,蓋世無雙!休說德性溫良,舉止沉重;自這一表人物,普天之下,也尋不出來。那裡有哥這樣大福?俺每今日得見嫂子一面,明日死也得好處!」因喚玳安兒:「快請你娘回房裡,只怕勞動著,倒值了多的。」吳月娘眾人聽了,罵扯淡輕嘴的囚根子不絕。良久,李瓶兒下來。四個唱的見他手有錢,都亂趨捧著他,娘長娘短,替他拾花翠,疊衣服,無所不至。月娘歸房,甚是悒怏不樂。只見玳安、平安接了許多拜錢,也有尺頭衣服,并人情禮,盤子盛著,拿到月娘房裡。月娘正眼也不看,罵道:「賊囚根子!拿送到前頭就是了,平白拿進我屋裡來做甚麼?」玳安道:「爹分付拿到娘房裡來。」月娘教玉筲接了,掠在牀上去。不一時,吳大舅吃了第二道湯飯,走進後邊來見月娘。月娘見他哥進房來,連忙花枝招颭,與他哥哥行禮畢,坐下。吳大舅道:「昨日你嫂子在這裡打攬,又多謝姐夫送了桌面去。到家對我說,你與姐夫兩個不說話。我執著要來勸你,不想姐夫今日請。姐姐,你若這等把你從前一場好都沒了;自古痴人畏婦,賢女畏夫,三從四德,乃婦道之常。今後姐姐,他行的事,你休要攔他,料姐夫他也不肯差了,落得你不做好好先生,纔顯出你賢德來。」月娘道:「早賢德好來,不教人這般憎嫌。他有了富貴的姐姐,把俺這窮官兒家丫頭,只當亡故了的算帳。你也不要管他,左右是我,隨他把我怎麼的罷!賊強人,從幾時這等變心來?」說著,月娘就哭了。吳大舅道:「姐姐,你這個就差了。你我不是那等人家,快休如此。你兩口兒好好的,俺每走來也有光輝些!」勸月娘一回,小玉拿了茶來,吃畢茶,分付放桌兒,留吳大舅房裡吃酒。吳大舅道:「姐姐沒的說。我適纔席上,酒飯都吃的飽飽的,來看姐姐。」坐了一回,只見前邊使小廝來請,吳大舅便作辭月娘出來。當下眾人吃到掌燈以後,就起身散了。那日四個唱的,李瓶兒每人都是一方綃金汗巾兒,五錢銀子,歡喜回家。自此西門慶一連在瓶兒房裡歇了數夜。別人都罷了,只是潘金蓮惱的要不的,替他唆調吳月娘與李瓶兒合氣。對李瓶兒,又說月娘許多不是,說月娘容不的人。李瓶兒尚不知墮他計中,每以姐姐呼之,與他親厚尤密。正是:

「逢人且說三分話,  未可全拋一片心。」

西門慶自從娶李瓶兒過門,又兼得了兩三場橫財,家道營盛,外庄內宅,煥然一新。米麥陳倉,騾馬成群,奴僕成行。把李瓶兒帶來小廝天福兒,改名琴童,又買了兩個小廝,一名來安兒,一名棋童兒。把金蓮房中春梅,上房玉筲,李瓶兒房中迎春,玉樓房中蘭香,一般兒四個丫鬟,衣服首飾,粧束出來,在前廳西廂房,教李嬌兒兄弟樂工李銘來家,教演習學彈唱。春梅琵琶,玉筲學箏,迎春學絃子,蘭香學胡琴。每日三茶三飯,管待李銘,一月與他五兩銀子。又打開門面二間,脫出二千兩銀子來,委付夥計、賁地傳,開解當舖。女婿經濟只要掌鑰匙,出入尋討,不拘藥材。賁地傳只是寫帳目,秤發貨物。傅夥計便督理生藥,解當兩個舖子,看銀色,做買賣。潘金蓮這樓上,堆放生藥;李瓶兒那邊樓上,廂成架子,閣解當庫,衣服,首飾、古董、書畫,玩好之物。一日也嘗當許多銀子出門。陳經濟每日起早遲睡,帶著鑰匙,同夥計查點出入銀錢,收放寫算皆精;西門慶見了,喜歡的要不的。一日,在前廳與他同桌兒吃飯,說道:「姐夫,你在我家這等會做買賣,就是你父親在東京知道,他也心安,我也得托了。常言道:『有兒靠兒,無兒靠婿。』姐夫是何人?我家姐姐是何人?我若久後沒出,這分兒家當,都是你兩口兒的。」那陳經濟說道:「兒子不幸,家遭官事,父母遠離,投在爹娘這裡;蒙爹娘抬舉,莫大之恩,生死難報!只是兒子年幼,不知好歹,望爹娘耽待便了,豈敢非望!」這西門慶聽見他說話兒,聰明乖覺,越發滿心歡喜。但凡家中大小事務,出入書柬禮帖,都教他寫;但凡人客到,必請他席側相陪。吃茶吃飯,一時也少不的他。誰知這小夥兒,綿裡之針,肉裡之剌,常向綉簾窺賈玉,每從綺閣竊韓香。有詩為證:

「東牀嬌婿實堪憐,  況遇青春美少年,

待客每令席側坐,  尋常只在便門穿;

家前院後明嘲戲,  呆裡撒乖暗做奸,     空在人前稱半子,  從來骨肉不牽連。」

光陰似箭,日月如梭。又見中秋賞月;忽然菊綻東籬。空中寒雁向南飛,不覺雪花滿地。一日,十一月下旬天氣,西門慶在友人常時節家,會答飲酒。散的早,未等掌燈時分就起身,同應伯爵、謝希大、祝日念三個,並馬而行。剛出了常時節門,只見天上彤雲密布,又早紛紛揚揚,飄下一天大雪花兒來。應伯爵便說道:「哥,咱這時候就家去,家裡也不收。我每知你許久不曾進裡邊看看桂姐,今日趁著天氣落雪,只當孟浩然踏雪尋梅,咱望他望去。」祝日念道:「應二哥說的是。你每月風雨不阻,出二十兩銀子包錢包著他,你不去,落得他自在。」西門慶于是吃三人你一言,我一句,說的把馬逕往東街构攔那條路來了。來到李桂姐家,已是天氣將晚。只見客位裡掌起燈燭,丫頭正掃地不迭。老馮并李桂卿出來見畢,上面列四張交椅,四人坐下。老虔婆便道:「前者桂姐在宅裡來晚了,多有打攪;又多謝六娘賞汗巾花翠。」西門慶道:「那日空過他,我恐怕晚了,他每客人散了,就打發他來了。」說著,虎婆一面看茶吃了,丫鬟就安放桌兒,設放案酒。西門慶道:「怎麼桂姐不見?」虎婆道:「桂姐連日在家伺候姐夫,不見姐夫來到。不想今日他五姨媽生日,拿轎子接了,與他五姨媽做生日去了。」看官聽說;原來世上,惟有和尚、道士并唱的人家,這三行人,不見錢眼不開;嫌貧取富,不說謊調詖也成不了的。原來李桂姐也不曾往五姨家做生日去。近日見西門慶不來,又接了杭州販紬絹的丁相公兒子丁二官人,號丁雙橋。販了千兩銀子紬絹,在客店裡安下。瞞著他父親來院中敲嫖,頭上拿十兩銀子、兩套杭州重絹衣服,請李桂姐一連歇了兩夜。適纔正和桂姐在房中吃酒,不想西門慶到。老虔婆教桂姐連忙陪他後邊第三層一間僻淨小房,那裡坐去了。當下西門慶聽信虔婆之言,便道:「既是桂姐不在,老媽快看酒來,俺每慢慢等他。」這老虔婆在下邊一力攛掇,酒餚菜蔬齊上,須臾,堆滿桌席。李桂卿不免箏排雁柱,歌按新腔。眾人席上猜枚行令,正飲酒在熱鬧處,不防西門慶往後邊更衣去。也是合當有事,忽聽東耳房有人笑聲。西門慶更畢衣,走到窗下偷眼觀覷,正見李桂兒在房內,陪著一個戴方巾的南蠻子飲酒。由不的心頭火起,走到前邊,一手把吃酒桌子掀倒,碟兒盞兒打的粉碎。喝令跟馬的平安、玳安、畫童、琴童四個小廝上來,不由分說,把李家門窗戶壁牀帳都打碎了。應伯爵、謝希大、祝日念,向前拉勸不住。西門慶口口聲聲,只要採出蠻囚來,和粉頭一條繩子,墩鎖在門房內。那丁二官兒,又是個小膽之人,外邊嚷鬧起來,諕的藏在裡間牀底下,只叫:「桂姐救命!」桂姐道:「呸!好不好,就有媽哩!不妨事。隨他發作怎的叫嚷,你休要出來!」且說老虔婆兒見西門慶打的不相模樣,不慌不忙,拄枴而出,說了幾句閑話。西門慶心中越怒起來,指著罵道,有滿庭芳為證:

「虔婆你不良,迎新送舊,靠色為娼;巧言詞,將咱誑,說短論長。我在你家使勾,有黃金千兩,怎禁賣狗懸羊?我罵你句真伎倆,媚人狐黨,衝一片假心腸!」虔婆亦答道:「官人聽知:你若不來,我接下別的。一家兒指望他為活計,吃飯穿衣,全憑他供柴糴米。沒來由暴叫如雷,你怪俺全無意。不思量自己,不是你憑媒娶的妻!」

西門慶聽了,心中越怒,險些不曾把李老媽媽打起來。多虧了應伯爵、謝希大、祝日念三個死勸,活喇喇拉開了手,西門慶大鬧了一場,賭誓再不踏他門來,大雪裡上馬回家。正是:

「宿盡閑花萬萬千,  不如歸去伴妻眠,     雖然枕上無情趣,  睡到天明不要錢。」

又曰:

「女不織兮男不耕,  全憑賣俏做營生,

任君斗量并車載,  難滿虔婆無底坑。」

又曰:

「假意虛脾恰似真,  花言巧語弄精神,

幾多伶俐遭他陷,  死後應知拔舌根。」

畢竟未知後來何如,且聽下回分解:




\end{showcontents}
