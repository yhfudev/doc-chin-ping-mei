%# -*- coding: utf-8 -*-
%!TEX encoding = UTF-8 Unicode
%!TEX TS-program = xelatex
% vim:ts=4:sw=4
%
% 以上设定默认使用 XeLaTex 编译,并指定 Unicode 编码,供 TeXShop 自动识别

%第七十九回 
\chapter{西門慶貪慾得病\KG 吳月娘墓生產子}

「仁者難逢思有常,  閑居慎勻恃無傷,

爭先徑路機關惡,  近後語言滋味長;

夾口物多終倣病,  快心事過必為殃,

與其病後能求藥,  不若病前能自防。」

此八句詩,乃邵堯夫所作,皆言天道福善,鬼神惡盈。作善降之百祥,作不善降之百殃。西門慶自知淫人妻子,而不知死之將至。當日在夾道內姦耍了來爵老婆,走到捲棚內陪吳大舅、應伯爵、謝希大、常時節飲酒。荊統制娘子、張團練娘子、喬親家母、崔親家母、吳大姨、吳大妗子、段大姐坐了好一回,上罷元宵圓子,方纔起身,告辭上轎家去了。大妗子那日,同吳舜臣媳婦,都家去了。陳經濟打發了王皇親戲子二兩銀子唱錢,酒食管待出門。只見四個唱的并小優,還在捲棚內彈唱遞酒。伯爵向西門慶說道:「明日花大哥生日,哥,你送了禮去不曾?」西門慶說道:「我早辰送過去了。」玳安道:「花大哥那里頭里使來定兒送請帖兒來了。」伯爵道:「哥,你明日去不去?我好來會你。」西門慶道:「到明日看,再不你先去罷,我慢慢兒去遞盃酒。」四個唱的後邊去了,李銘等上來彈唱。西門慶不住只是在椅子上打睡。吳大舅道:「姐夫連日辛苦了,罷罷,咱每告辭罷。」於是起身。那西門慶又不肯,只顧攔着留坐,到二更時分纔散。西門慶先發四個唱的轎子去了,拏大鍾賞李銘等三人,每人兩鍾酒,與了六錢唱錢。臨出門,叫回李銘分付:「我十五日要請你周爺和你荊爺、何老爹眾位,你早替叫下四個唱的,休要誤了。」李銘跪下,稟問:「爹叫那四個?」西門慶道:「樊百家奴兒、秦玉芝兒,前日何老爹那里唱的一個馮金寶兒,并呂賽兒,好歹叫了來。」李銘應諾:「小的知道了。」磕了頭去了。西門慶歸後邊月娘房里來,月娘告訴:「今日林太太在席,與荊大人娘子,好不喜歡!坐到那咱晚纔去了。酒席上謝我老爹扶持,但得好處,不敢有忘!也在出月,往淮上催攢粮運去也。又說何大人娘子,今日也吃了好些酒,喜歡六姐;又引到那邊花園山子上瞧了瞧;今日各項也賞唱的許多東西。」說畢,西門慶就在上房歇了。到了半夜,月娘做了一夢,天明告訴西門慶說道:「敢是我日裡看見他林太太穿着大紅絨袍兒,我黑夜就夢見你李大姐廂子內,尋出一件大紅絨袍兒,與我穿在身。被潘六姐匹手奪了去,披在他身上;教我就惱了,說道:『他的皮襖你要的去穿了罷了,這件袍兒你又來奪!』他使性兒,把袍兒上扯了一道大口子。吃我大喓喝,和他罵嚷,嚷嚷著,就醒了,不想都是南柯一夢!」西門慶道:「你從睡夢中,只顧氣罵不止,不打緊,我到明日替你尋一件穿就是了。自古夢是心頭想。」到次日起來,頭沉,懶待往衙門中去。梳頭淨面,穿上衣裳,走來前邊書房中籠上火,那里坐的。只見玉簫早辰來如意兒房中,擠了半甌子奶,逕到廂房與西門慶吃藥。見西門慶倚靠床上,有王經替他打腿。王經見玉簫來,就出去了。打發他吃藥,西門慶使他拏了一對金裹頭簪兒,四個烏銀戒指兒,教他送到來爵媳婦子房里去。那玉簫听見主子使他幹此營生,又似來旺媳婦子那一本帳,連忙鑽頭覓縫袖的去了。送到了物事,還走來回西門慶話,說道:「收了,改日與爹磕頭」。拏回空甌子兒到上房。月娘問他:「你爹吃了藥了?在廂房內做甚麼哩?」玉簫道:「沒言語。」月娘道:「你替他熬粥下來。」約莫等飯時前後,還不見進來。原來王經稍帶了他姐姐王六兒一包兒物事,遞與西門慶瞧,就請西門慶往他家去。西門慶打開紙包兒,都是老婆剪下一柳黑臻臻光油油的青絲,用五色絨纏就的一個同心結托兒,用兩根錦帶兒拴著,安放在塵柄下,做的十分細巧工夫。那一件是兩個的鴛鴦紫遍地金順袋兒,都緝着迴絞綿綉,里邊盛着瓜穰兒。西門慶觀翫良久,滿心歡喜。遂把順袋放在書廚內,錦托兒褪於袖中。正在凝思之際,忽見吳月娘驀地走來,掀開簾子,見倘在床上,王經扒着替他打腿。便說道:「你怎的只顧在前頭,就沒進去了?屋里擺下粥了。你告我說,你心里怎的?只是恁沒精神!」西門慶道:「不知怎的,心中只是不耐煩,害腿疼。」月娘道:「想必是春氣起了。你吃了藥也,等慢慢來。」一面請到房中,打發他吃了粥。因說道:「大節下,你也打起精神兒來。今日門外花大舅生日,請你往那里走走去,再不叫將應二哥;他也不在了,與花大舅做生日去了。你整治下酒菜兒,我往燈巿舖子內,和他二舅吃回酒坐坐罷。」月娘道:「你備馬去,我叫丫鬟整理。」這西門慶一面分付玳安備馬,王經跟隨,穿上衣裳,逕到獅子街燈巿里來。但見燈巿中車馬轟雷,燈毬燦綵,遊人如蟻,十分熱鬧。

「大平時序好風催,  羅綺爭馳鬬錦〈廴回〉;

鰲山高聳青雲上,  何處遊人不看來。」

西門慶看了回燈,到獅子街房子門首下馬,進入里面坐下。慌的吳二舅、賁四都來聲喏。門首買賣,甚是興勝。來昭妻一丈青,又早書房內籠下火,拿茶吃了。不一時,家中吳月娘使琴童兒、來安兒,拿了兩方盒點心上嗄飯菜蔬,鋪內有南邊帶來豆酒,打開一罎,擺在樓上,坐着炭火,請吳二舅與賁四輪番吃酒。樓窗外就着見燈巿往來人烟不斷,諸行貨殖如山。吃至飯後的時分,西門慶使王經對王六兒說去。王六兒听見西門慶來,家中又整治下春臺果盒酒肴等候。西門慶分付來昭:「將這一卓酒菜晚夕留着,與二舅、賁四在此上宿吃,不消拏回家去了。」又教琴童提送一罎酒過王六兒這邊來。西門慶於是騎馬,逕到他家。婦人打扮迎接,到明間內,插燭也似磕了四個頭。說道:「迭承你厚禮,怎的兩次請你不去?」王六兒道:「爹倒說的好,我家中再有誰?不知怎的這兩日只是心里不好,茶飯兒也懶吃,做事沒入腳處!」西門慶道:「敢是想你家老公?」婦人道:「我那里想他,倒是見爹這一向不來,不知怎的怠慢着爹了,爹把我網巾圈兒打靠後了,只怕另個有心上的人兒了!」西門慶笑道:「那里有這個道理?倒因家中節間擺酒,忙了兩日。」婦人道:「說昨日爹家中請堂客來?」西門慶道:「便是你大娘吃過人家兩席節酒,須得請人回席。」婦人道:「請了那幾位堂客?」西門慶便說某人某人,從頭訴說一遍。婦人道:「看燈酒兒,只請要緊的,就不請俺每請兒了?」西門慶道:「不打緊,到明日正月十六日,還有一席,可請你每眾夥計娘子走走去。是必到根前,又推故不去着?」婦人道:「娘若賞個帖兒來,怎敢不去?不是因前日他小大姐罵了申二姐,教他好不抱怨說俺每。他那日要不去來,倒是俺每攛掇了他去了。落後罵了來,好不在這里哭。俺每到沒意思剌剌的!落後又教爹娘費心,送了盒子,并那一兩銀子來安撫了他,纔罷了。不知原來家中小大姐這等等藻暴性子,就是打狗,也看主人面!」西門慶道:「你不知這小油嘴,他好不兜胆的性,着緊把我也擦扛的眼直直的!也見他教你唱,唱個兒與他听罷了。誰教你不唱,又說他來?」婦人道:「耶嚛,嚛嚛!他對我說,他凡時說他來!走來指着臉子,就罵他起身;罵的他來,在我這里好不醜的三行鼻涕兩行眼淚的哭!我這里留他住了一夜,纔打發他去了。」說了一回,丫鬟拿茶吃了。小廝進財兒買了點心鮮魚嗄飯來,老馮婆子在廚下整理,又走來上邊與西門慶磕頭。西門慶與了他約三四錢一塊銀子,說道:「從你娘沒了,就不常往我那里走走去?」婦人道:「沒他主兒,那里着落?倒常時來我這邊和我做伴兒。」不一時,房中收拾乾淨,婦人請西門慶房中坐的,問:「爹用了午飯不曾?」西門慶道:「我早辰家中吃了些粥,剛纔陪你二舅又吃了兩個點心,且不吃些甚麼理。」一面放卓兒,設擺春臺,安排上酒來。卓上無非是節食美饌,佳殽菓菜之類。婦人令王經打開荳酒,篩將上來,陪西門慶做一處飲酒。婦人問道:「我稍來的那物件兒,爹看見來?都是奴旋剪下頂中一柳頭髮親手做的。管情爹見了愛。」西門慶道:「多謝你厚情。」飲至半酣,見房內無人,西門慶袖中取出來套在龜身下。兩根錦帶兒,扎在腰間,龜頭又帶着景東人事,用酒服下胡僧藥下去。那婦人用手搏弄,弄的那話登時奢稜跳腦,橫觔皆現,色若紫肝,比銀托子和白綾帶子又不同。西門慶摟婦人坐在懷內,那話插進牝中,在上面兩個一遞一口飲酒咂舌頭。婦人把菓仁兒用舌尖哺與西門慶吃,直頑笑吃至掌燈。馮媽媽廚下做了猪肉韮菜餅兒拿上來,婦人陪西門慶每人吃了兩個,丫鬟收下去。兩個在里間廂成的煖炕上,撩開錦幔,二人解衣就寢。婦人知道西門慶好點着燈行房,把燈臺移在明間炕邊一張卓上安放,一面將紙門關上,澡牝乾淨,換了一雙大紅潞紬白綾平底鞋兒,穿在腳上,脫了褲兒,鑽在被窩里與西門慶做一處,相摟相抱睡了一回。原來西門慶心中,只想着何千戶娘子藍氏,慾情如火,那話十分堅硬,先令婦人馬伏在下,那話放入庭花內,極力〈扌扉〉硼了,約二三百度,〈扌扉〉硼的屁股連聲響喨,婦人用手在下,操着秘心子,口中叫達達如流水。於是心中還不美意,起來披上白綾小襖,坐在一隻枕頭上,婦人仰臥,尋出兩條腳帶,把婦人兩隻腳拴在兩邊護炕柱兒上,賣了個金龍探爪,將那話放入牝中。少時,沒稜露腦,淺抽深送;次後,半出半入,纔進長驅。恐其婦人害冷,亦取紅綾短襦,蓋在他身上。這西門慶乘其酒興,把燈光挪近根前,垂首翫其出入之勢,抽徹至首,復送至根,又數百回。婦人口中百般柔聲顫語,都叫將出來,西門慶又取粉的膏子藥,塗在龜頭上攮進去,婦人陰中麻癢不能當,急令深入,兩相迎就。這西門慶故作逗遛,戲將龜頭濡〈扌晃〉其牝口,又挑弄其花心,不肯深入。急的婦人淫津流出,如蝸之吐涎,往來滯的牝戶,翻覆可愛。燈光影里,見他兩隻腿兒,穿着大紅鞋兒,白生生腿兒,蹺在兩邊,吊的高高的,一往一來,一衝一撞,其興不可遏。因口呼道:「淫婦,你想我不想?」婦人道:「我怎不想達達,只要你松栢兒冬夏長青,便好。休要日遠日疎,頑耍膩了,把奴來也不理,奴就想死了罷了!敢和誰說?有誰知道?必是俺那王八來家,我也不和他說。想他恁在外邊做買賣,有錢不養老婆的,他肯掛念我?」西門慶道:「我的兒,你若一心在我身上,等他來家,我爽利替他另娶一個,你只長遠等着我便了。」婦人道:「我達達,等他來家,好歹替他娶了一個罷!或把我放在外頭,或是招我到家去,隨你心里。淫婦爽利把不值錢的身子,拚與達達罷,無有個不依你的。」西門慶道:「我知道。」兩個說話之間,又幹勾兩頓飯時,方纔精拽。解卸下婦人腳帶來,摟在被窩內,並頭交股,醉眼朦朧,一覺直睡到三更天氣方醒。西門慶起來穿衣淨手,婦人開了房門,叫丫鬟進來,再添美饌,復飲香醪,滿斟暖酒,又陪西門慶吃了十數盃,不覺醉上來,纔點茶來漱了口,向袖中掏出一紙帖兒,遞與婦人:「問甘夥計鋪子里取一套衣服你穿,隨你要甚花樣。」那婦人萬福謝了,送出門。王經打着燈籠,玳安、琴童籠着馬,打發上了馬,婦人方纔關門。這西門慶身穿紫羊絨褶子,圍着風領,騎在馬上。那時也有三更時分,天氣有些陰雲,昏昏慘慘的月色,街巿上靜悄悄,九衢澄淨,嗚柝唱號提鈴。打馬正過之次,剛走到西首那石橋兒根前,忽然見一個黑影子,從橋底下鑽出來,向西門慶一拾,那馬見了只一驚躲,西門慶在馬上打了着個冷戰,醉中把馬加了一鞭,那馬搖了搖鬃,玳安、琴童兩個用力拉着嚼環,收熬不住,雲飛般望家奔將來,直跑到家門首方止。王經打着燈籠,後邊跟不上。西門慶下馬,腿軟了,被左右扶進,逕在前邊潘金蓮房中來。此這一不來倒好;正是:

「失脫人家逄五道,  濱冷餓餽撞鍾馗。」

原來金蓮從後邊來,還沒睡,渾衣倒在炕上,等待西門慶。听見來了,慌的砧碌扒起來,向前替他接衣服。見他吃的酩酊大醉,也不敢問他。這西門慶隻手搭伏着他肩膊上,摟在懷里,口中喃喃吶吶說道:「小淫婦兒,你達達今日醉了,收拾鋪我睡也。」那婦人扶他上炕,打發他歇下。那西門慶丟倒頭在枕頭上,鼾睡如雷,再搖也搖不醒。然後婦人脫了衣裳,鑽在被窩內,慢慢用手腰里摸他那話,猶如綿軟,再沒些硬朗氣,更不知在誰家來?翻來覆去,怎禁那慾火燒身,淫心蕩意。不住用手只顧捏弄,蹲下身子,被窩內替他百計品咂,只是不起。急的婦人要不的。因問西門慶:「和尚藥在那里放着哩?」推了半日,推醒了。西門慶酩酊里罵道:「怪小淫婦,只顧問怎的!你又教達達擺布你?你達今日懶待動彈。藥在我袖中金穿心盒兒內,你拏來吃了,有本事品弄的他起來,是你造化。」那婦人便去袖內摸出穿心盒,打開里面,只剩下三四丸藥兒。這婦人取過燒酒壺來。斟了一鍾酒,自己吃了一丸。還剩三丸,恐怕力不效,千不合,萬不合,拏燒酒 都送到西門慶口內。醉了的人,曉的甚麼,合着眼只顧吃下去。那消一盞熱茶時,藥力發作起來,婦人將白綾帶子拴在根上,那話躍然而起。但見裂瓜頭凹眼圓睜,落腮鬍挺身直竪,婦人見他共顧睡,於是騎在他身上,又取膏子藥安放馬眼內,頂入牝中。只顧揉搓,那話直抵苞花窩里,覺翕翕然渾身酥麻,暢美不可言。又兩手據按,舉股一起一坐那話沒稜露腦,約一二百回,初時澀滯,次後淫水浸出,稍沾滑落,西門慶由着他掇弄,只是不理。婦人情不能當,以舌親於西門慶口中,兩手摟着他脖項,極力揉搓,左右偎擦,塵柄盡沒至根,止剩二卵在外。用手摸之,美不可言,淫水隨拭隨出,比三鼓,凡五換巾帕,婦人一連丟了兩次。西門慶只是不泄,龜頭越發脹的色若紫肝,橫觔皆現,猶如火熱。一回,害箍脹的慌,令婦人把根下帶子去了,還發脹不已。令婦人用口吮之,這婦人扒伏在他身上,用朱唇吞裹其龜頭,只顧往來不已。又勒勾約一頓飯時,那管中之精,猛然一股,邈將出來,猶水銀之瀉筒中相似,忙用口接嚥不及,只顧流將起來。初時還是精液,往後盡是血水出來,再無個收救。西門慶已昏迷去,四肢不收。婦人也慌了,急取紅棘與他吃下去。精盡繼之以血,血盡出其冷氣而已。良久方止。婦人慌做一團,便摟着西門慶,問道:「我的哥哥,你心里覺怎麼的?」西門慶甦省了一回,方言:「我頭目森森然,莫知所矣。」「你今日怎的流出恁許多?」更不說他用的藥多了。看官听說:「一已精神有限,天下色慾無窮。」又曰:「嗜慾深者,其天機淺。」西門慶只知貪淫樂色,更不知油枯燈盡,髓竭人亡。原來這女色坑陷得人有成時必有敗。古人有幾句格言道得好:

「花面金剛,玉體魔王,綺羅粧做豺狼。法場斗帳,獄牢牙床,柳眉刀,星眼劍,絳唇鎗。口美舌香,蛇蝎心腸,共他者無不遭殃!纖塵入水,片雪投湯。秦楚強,吳越壯,為他亡!早知色是傷人劍,殺盡世人人不防!」

「二八佳人體似酥,  腰間仗劍斬愚夫;

雖然不見人頭落,  暗里教君骨髓枯。」

一宿晚景題過。到次日清早晨,西門慶起來梳頭,忽然一陣暈起來,望前一頭搶將去。早被春梅雙手扶住,不曾跌着,磕傷了頭臉。在椅子上坐了半日,方纔回過來。慌的金蓮連忙問道:「只怕你空心虛弱,且坐着吃些甚麼兒着出去也不遲。」一面使秋菊:「後邊取粥來,與你爹吃。」那秋菊走到後邊廚下,問雪娥:「熬的粥怎麼了?爹如此這般,今早起來害頭暈,跌了一交,如今要吃粥哩!」不想被月娘听見,叫了秋菊,問其端的。秋菊悉把西門慶梳頭,頭暈跌倒之事,告訴一遍。月娘不听便了,听了魂飛天外,魄散九霄。一面分付雪娥快熬粥,一面走來金蓮房中看視。見西門慶坐在椅子上,問道:「你今日怎的頭暈?」西門慶道:「我不知怎的?剛纔就頭暈起來。」金蓮道:「早時我和春梅在根前扶住了;不然,好輕身子兒,這一交和你善哩!」月娘道:「敢是你昨日來家晚了,酒多了頭沉?」金蓮道:「昨日往誰家吃酒,這咱晚纔來?」月娘道:「他昨日和他二舅在鋪子里吃酒來。」不一時,雪娥熬了粥,教秋菊拿着,打發西門慶吃。那西門慶拏起粥來,只吃了半甌兒,懶待吃,就放下了。月娘道:「你心里覺怎的?」西門慶道:「我不怎麼,只是身子虛飄飄的,懶待動彈。」月娘道:「你今日不往衙門中去罷?」西門慶道:「我不去了。消一回,我往前邊看着姐夫寫了帖兒,發帖兒去,十五日請周菊軒、荊南崗、何大人他每眾官吃酒。」月娘道:「你今日還沒吃藥,取奶來,把那藥你再吃上一服。是你連日張羅的,你有着辛苦勞碌了。」一面教春梅問如意兒擠了奶來,用盞兒盛着,教西門慶吃了藥,起身往前邊去。春梅扶着,剛走到花園角門首,覺眼便黑了,身子晃晃蕩蕩,做不的主兒,只要倒。春梅又扶回來了。月娘道:「依我,且歇兩日兒,請人也罷了。那里在乎在這一時上!今日在屋里將息兩日兒,不出去罷。」因說:「你心里要吃甚麼?我往後邊教丫鬟做來與你吃。」西門慶道:「我心里不想吃。」月娘到後邊,從新又審問金蓮:「他昨日來家不醉?再沒曾吃酒?與你行甚麼事?」那金蓮听了,恨不的生出幾個口來,說一千個沒有:「姐姐,你沒的說。他那咱晚來了,醉的行禮兒也不顧的,還問我燒酒 吃。教我拏茶當酒與他吃,只說沒了酒,好好打發他睡了。自從姐姐那等說了,誰和他有甚事來!倒沒的羞人子刺刺的!倒只怕外邊別處有了事來?俺每不知。若說家里,可是沒絲亳事兒!」月娘一面和玉樓都坐在一處,叫了玳安、琴童兩個,到根前軫問他:「你爹昨日在那里吃酒來?你實說便罷,不然有一差二錯,就在你這兩個囚根子身上!」那玳安咬定牙,只說獅子街和二舅、賁四吃酒,再沒往那里去。落後叫將吳二舅來問他,二舅道:「姐夫只陪俺每吃了沒多大回酒,就起身往別處去了。」這吳月娘听了,心中大怒,待二舅去了,把玳安、琴童儘力數罵了一頓,要打他二人。二人慌了,方纔說出昨日在韓道國老婆家吃酒來。那潘金蓮得不的一聲,就來了,說道:「姐姐剛纔埋怨起俺每來,正是冤殺旁人笑殺賊!俺每人人有面,樹樹有皮。姐姐那等說來,莫不俺每成日把這件事放在頭里!」又道:「姐姐,你再問這兩個囚根子,前日你往何千戶家吃酒,他爹也是那咱時分纔來,不知在誰家來?誰家一個拜年,拜到那咱晚!」玳安又生恐琴童說出來,隱瞞不住,遂把私通林太太之事,且說一遍。月娘方纔信乎,說道:「嗔道教我拏帖兒請他!我還說人生面不熟,他不肯來。怎知和他有連手!我說恁大年紀,描眉畫鬢兒的,搽的那臉倒相膩抹兒抹的一般,乾淨是個老浪貨!」玉樓道:「姐姐,沒見一個兒子也長恁大,大兒大婦,還幹這個營生!忍不住嫁了個漢子。」金蓮道:「那老淫婦有甚麼廉耻!也休要出這個醜!」月娘道:「我說只怕他不來,誰想他浪〈扌扉〉着來了!」金蓮道:「這個姐姐纔顯出個皂白來了,相韓道國家這個淫婦,姐姐還嗔我罵他罷?乾淨一家子都養漢,是個明王八,把個王八花子也裁派將來,早晚好做勾使鬼!」月娘道:「王三官兒娘,你還罵他老淫婦;他說你從小兒在他家使喚來!」那金蓮不聽便罷,聽了把臉掣耳朵帶脖子紅了,便罵道:「汗邪了那賊老淫婦!我平白在他家做甚麼?還是我姨娘在他家緊隔壁住。他家有個花園,俺每小時在俺姨娘家住,常過去和他家伴姑兒耍去,就說我在他家來,我認的他甚麼?是個張眼露睛的老淫婦!」月娘道:「你看那嘴頭子,人和你說話,你罵他!」那金蓮一聲兒不言語了。月娘主張雪娥做了些水角兒 ,拿了前邊與西門慶吃。正走到儀門首,只見平安兒逕直往花園中走,被月娘叫住,問道:「你做什麼?」平安兒道:「李銘叫了四個唱的,十五日擺酒用來回話,問擺的成,擺不成?我說還沒發帖兒哩。他不信,教我進來稟爹。」月娘罵道:「怪賊奴才,還擺甚麼酒?問甚麼?還不回那王八去哩,還來稟爹娘哩!」把平安兒罵的往外金命水命,走投無命!月娘走到金蓮房中,看着西門慶只吃了三四個水角兒。就不吃了。因說道:「李銘來回唱的,教我回倒他酒且擺不成,改了日子了,他去了。」西門慶點頭兒。西門慶自知一兩日好些出來。誰知過了一夜,到次日,下邊虛陽腫脹,不便處發出紅暈來了,連腎囊都腫的明滴溜如茄子大。但溺尿,尿管中猶如刀子犂的一般。溺一遭,疼一遭,外邊排軍當備下馬伺候,還等西門慶往衙門里大發放。不想又添出這樣症候來!月娘道:「你依我,拏帖兒回了何大人,在家調理兩日兒,不去罷。你身子恁虛弱,趁早使小廝請了任醫官教瞧瞧。你吃他兩貼藥過來,休要只顧躭著,不是事!你惹大的身量,兩日通沒大好吃甚麼兒,如何禁的?」那西門慶只是不肯吐口兒請太醫,只說:「我不妨事。過兩日兒好了,我還出去。」雖故差人拏帖兒,送假牌往衙門里去,在床上睡着,只是急藻沒好氣。應伯爵打聽得知走來看他。西門慶請至金蓮房中坐的。伯爵聲喏道:「前日打攪哥,不知哥哥心中不好。嗔道花大舅那里不去。」西門慶道:「我心中若好時,我去了。不知怎的懶待動彈!」伯爵道:「哥,你如今心內怎樣的?」西門慶道:「不怎的,只是有些頭暈起來,身子軟,走不的。」伯爵道:「我見你面容發紅色,只怕是火。教人看來不曾?」西門慶道:「房下說請任后溪來看我,我說又沒甚麼病,怎好請他的?」伯爵道:「哥你這個就差了。還請他來,看看怎的說。吃兩貼藥,散開這火就好了。春氣起,人都是這等,痰火舉發舉發。昨日李銘撞見我,說你使他叫唱的,今日請人擺酒;說你心中不好,改了日子。把我諕了一跳,教我今日早來看看哥。」西門慶道:「我今日連衙門中拜牌也沒去,送假牌去了。」伯爵道:「可知去不的。大調理兩個日兒出門。」吃畢茶,道:「我去罷,再來看哥。李桂姐會了吳銀兒,也要來看你哩。」西門慶道:「你吃了飯去。」伯爵道:「我一些不吃。」揚長出去了。西門慶於是使琴童兒往門外請了任醫官來,進房中診了脉,說道:「老先生此貴恙,乃虛火上炎,腎水下竭,不能既濟,乃是脫陽之症。須是補其陰虛,方纔好得。」封了五星銀了,討將藥來吃了,止住了頭暈,身子依舊還軟,起不來;下邊腎曩,越發腫痛,溺尿甚難。說畢,作辭起身去了。到後晌時分,李桂姐、吳銀兒坐轎子來看;每人兩個盒子,一盒菓餡餅兒,一盒玫瑰金餅,一副蹄,兩隻燒鴨 ,進房與西門慶磕頭,說道:「爹怎的心里不自在?」西門慶道:「你姐兒兩個自恁來看看便了,如何又費心買禮兒?」因說道:「我今年不知怎的,痰火發的重些。」桂姐道:「還是爹這節間酒吃的多了,清潔他兩日就好了。」坐了一回,走去李瓶兒那邊屋里,與月娘眾了見節。請到後邊,擺茶畢,又走來前邊,陪西門慶坐的說話兒。只見伯爵又陪了謝希大、常時節來望。西門慶教玉筲搊扶他起來坐的,留他三在房內放卓兒吃酒。謝希大道:「哥用了些粥不曾?」玉筲把頭扭着不答應。西門慶道:「我還沒吃粥,嚥不下去。」希大道:「拏粥,等俺每陪哥吃些粥兒還好。」不一時,拿將粥來。玉筲拏盞兒伺候,眾人陪着吃點心下飯。西門慶拿起粥來,只扒了半盞兒,就不吃下去。月娘和李桂姐、吳銀兒都在李瓶兒那邊坐的管待。伯爵問道:「李桂姐與銀姐來了,怎的不見?」西門慶道:「在那邊坐的。」伯爵因令來安兒:「你請過來唱一套兒,與你爹聽。」那吳月娘恐怕西門慶不耐煩,攔着,只說吃酒哩,不教過來。眾人吃了一回酒,說道:「哥,你陪着俺每坐,只怕勞碌着你。俺每去了,你自在側側兒罷。」西門慶道:「起動列位掛心!」三人於是作辭去了。應伯爵走出小院門,叫玳安過來分付:「你對你大娘說,你就說應二爹說來,你爹面上變色,有些滯氣,不好。早尋人看,他大街上胡太醫,最治的好痰火,何不使人請他看看?休要躭遲了!」玳安不敢怠慢,走來告訴月娘。月娘慌進房來,對西門慶說:「方纔應二哥對小廝說,大街上胡太醫看的痰火,你何不請他來看看來?」西門慶道:「胡太醫前番看李大姐不濟,又請他?」月娘道:「藥醫不死病,佛度有緣人。看他不濟,只怕有緣,吃了他的藥兒,好了是的。」西門慶道:「也罷,你請他去。」不一時,使棋童兒請了胡太醫來。適有吳大舅來看,陪他到房中看了脉,對吳大舅、陳經濟說:「老爹是個下部蘊毒,若久而不治,卒成溺血淋之疾,乃是忍便行房。」又封了五星藥金,討將藥來吃下去,如石沉大海一般,反溺不出來。月娘慌了,打發桂姐、吳銀兒去了。又請何老人兒子何春泉來看,又說是癃閉便毒,一團膀胱邪火,趕到這邊下來;四肢經絡中,又有濕痰流聚,以致心腎不交。封了五錢藥金,討將藥來,越發弄的虛陽舉發,塵柄如鐵,晝夜不倒。潘金蓮晚夕不知好歹,還騎在他上邊,倒澆燭掇弄,死而復甦者數次。到次日,何千戶要來望,先使人來說。月娘便對西門慶道:「何大人便來看你,我扶住你往後邊去罷。這邊隔二偏三,不是個待人的。」那西門慶點頭兒,於是月娘替他穿上煖衣,同金蓮肩搭搊扶著,往離了金蓮房,往後邊上房,鋪下被褥高枕,安頓他在明間炕上坐的。房中收拾乾淨,焚下香。不一時何千戶來到,陳經濟請他到於後邊臥房,看見西門慶坐在病榻上,說道:「長官,我不敢作揖。」因問:「貴恙覺好些?」西門慶告訴:「上邊火倒退下了,只是下卵腫毒當不的。」何千戶道:「此係便毒。我學生有一相識,在東昌府探親。昨日新到舍下,有一封書下。乃是山西汾州人氏,姓劉號橘齋,年半百,極看的好瘡毒。我就使人請他來看看長官貴恙。」西門慶道:「多承長官費心,我這里就差人請去。」何千戶吃畢茶,說道:「長官你耐煩保重。衙門中事,我每日委答應的,遞事件與你,不消掛意。」西門慶舉手道:「只是有勞長官了。」作辭出門。西門慶這里隨即差玳安拏帖兒,同何家人請了這劉橘齋來。看了脉,并不便處。連忙上了藥,又封一貼煎藥來。西門慶答賀了一疋杭州絹,一兩銀子,吃了他頭一盞藥,還不見動靜。那日不想鄭愛月兒送了一盒鴿子雛兒 ,一盒菓餅頂皮酥,坐轎子來看西門慶。進門花枝招颭,綉帶飄飄,與西門慶磕着頭,說道:「不知道爹不好,桂姐和銀姐好人兒,不對我說聲兒,兩個就先來了。看的爹遲了,休怪!」西門慶道:「不遲,又起動你媽費心,又買禮來!」愛月兒笑道:「甚麼大禮!惶恐的要不的。」因說:「爹清減的恁樣的!每月飲饌,也用些兒?」月娘道:「用的倒好了,吃不多兒。今日早辰,只吃了些粥湯兒,還沒些吃甚麼兒。剛纔太醫看了去了。」愛月兒道:「娘,你分付姐把鴿子雛兒頓爛一個兒來 ,等我勸爹進些粥兒。你老人家不吃,恁惹大身量,一家子金山也似靠着你,都怎麼樣兒的!」月娘道:「他只害心口內攔着,吃不下去。」愛月兒道:「爹你依我說,把這飲饌兒逐日就懶待吃,須也強吃些兒,怕怎的?人無根本,水食為命。終須但用的,有枉【扌戧】些兒。不然,越發淘淥的身子空虛了!」不一時,頓爛了鴿子雛兒,小玉拿粥上來,十香甜醬瓜,茄粳粟米粥兒。這鄭月兒跳上炕去,用盞兒托着,跪在西門慶身邊,一口口喂他。強打着精神,只吃上上半盞兒,揀了兩筯兒鴿子雛兒在口內,就搖頭兒不吃了。愛月兒道:「一來也是藥,二來還虧我勸爹,都怎的也進了些飲饌兒。」玉筲道:「爹每常也吃,不似今日月姐來勸着吃的多些。」月娘一面擺茶與愛月兒吃,臨晚管待酒饌,與了他五錢銀子,打發他家去。愛月兒臨出門,又與西門慶磕頭,說道:「爹你耐心兒將息兩日兒,我再來看你。」比及到晚夕,西門慶又吃了劉橘齋第二貼藥,遍身痛,叫喚了一夜。到五更時分,那不便腎囊腫脹破了,流了一灘鮮血,龜頭上又生出疳瘡來,流黃水不止。西門慶不覺昏迷過去。月娘眾人慌了,都守着看視。見吃藥不效,一面請了劉婆子,在前邊捲棚內與西門慶點人燈跳神。一面又使小廝往周守禦家內,訪問吳神仙在那里,請他來看西門慶;他原相他,今年有嘔血流膿之灾,骨瘦形衰之病。賁四說:「也不消問周老爹宅內去;如今吳神仙見在門外土地廟前,出着個卦肆兒,又行醫又賣卦,人請他,不爭利物,就去看治。」月娘連忙就使琴童把這吳神仙請將來。進房看了西門慶,不似往時,形容消減,病體懨懨,勒着手帕,在於臥榻。先診了脉息,說道:「官人乃是酒色過度,腎水竭虛;是太極邪火,聚於慾海。病在膏肓,難以治療。吾有詩八句,說與你聽。只因他:

「醉飽行房戀女娥,  精神血脉暗消磨,

遺精溺血流白濁,  燈盡油乾腎水枯;

當時祇恨歡娛少,  今日翻為疾病多,

玉山自倒非人力,  總是盧醫怎奈何!」

月娘見他治不的了,說道:「既下藥不好,先生看他命運如何?」吳神仙搯指尋紋,打算西門慶八字,說道:「屬虎的,丙寅年,戌申月,壬午日,丙辰時,今年戊戌流年,三十三歲算命,見行癸亥運,雖然是火土傷官,今年戊土來剋壬水,歲傷旱,正月又是戊寅月,三戊沖辰,怎麼當的?雖發財發福,難保壽源!有四句斷語不好。」說道:

「命犯災星必主低,  身輕煞重有灾危;

時日若逄真太歲,  就是神仙也縐眉!」

月娘道:「命中既不好,先生你替他演演禽星如何?」這吳神仙鋪下禽遁干支,他說道:

「心月狐狸角木蛟,  絳幃深處不相饒,

常在月宮飛玉露,  慣從月下奪金標;

樂處化為真雞子,  死時還想爛甜桃,

天罡地煞皆無救,  就是王禪也徒勞。」

月娘道:「禽上不好,請先生替我圓圓夢罷。」神仙道:「請娘子說來,貧道圓。」月娘道:「我夢見大廈將頹,紅衣罩體,攧拆了碧玉簪,跌破了菱花鏡。」神仙道:「娘子莫怪我說,大廈將頹,夫君有厄;紅衣罩體,孝孝服臨身;攧拆了碧玉簪,姊妹一時失散;跌破了菱花鏡,夫妻指日分離。此夢猶然不好,不好!」月娘道:「問先生有解麼?」神仙道:「白虎當頭攔路,喪門魁在生災。神仙也無解,太歲也難推。造物已定,神鬼莫移!」月娘見命中無有救星,於是拏了一疋布謝了神仙,打發出門,不在話下。正是:

「卦裡陰陽仔細尋,  無端閑事莫閑心;

平生作善天加慶,  心不欺貧禍不侵。」

月娘見求神問卜,皆有凶無吉,心中慌了。到晚夕天井內焚香,對天發願,許下兒夫好了,要往泰安州頂上與娘娘進香掛袍三年。孟玉樓又許下逄七拜斗。獨金蓮與李嬌兒不許願心。西門慶自覺身體沉重,要便發昏過去,眼前看見花子虛、武大在他根前站立,問他討債。又不肯告人說,只教人廝守着他。見月娘不在根前,一手拉着潘金蓮,心中捨不的他,滿眼落淚,說道:「我的冤家,我死後,你姊妹們好好守着我的靈,休要失散了。」那金蓮亦悲不自勝,說道:「我的哥哥,只怕人不肯容我。」西門慶道:「等他來,等我和他說。」不一時吳月娘進來,見他二人哭的眼紅紅的,便道:「我的哥哥,你有甚話?對奴說幾句兒,也是奴和你做夫妻一場!」西門慶聽了,不覺哽咽,哭不出聲來,說道:「我覺自家好生不濟,有兩句遺言和你說。我死後你若生下一男半女,你姊妹好生待著,一處居住,休要失散了,惹人家笑話!」指着金蓮說:「六兒他從前的事,你躭待他罷!」說畢,那月娘不覺桃花臉上滾下珍珠來,放聲大哭,悲慟不止。西門慶道:「你休哭,聽我囑付你。」有駐馬聽為證:

「賢妻休悲,我有衷情告你知。妻你腹中是男是女,養下來看大成人,守我的家私。三賢九烈要貞心,一妻四妾,携帶着住。彼此光輝光輝!我死在九泉之下,口眼皆閉!」

月娘聽了,亦回答道:

「多謝兒夫,遺後良言教道奴。夫我本女流之輩,四德三從,與你那樣夫妻,平生作事不糢糊。守貞肯把夫名污?生死同途,一鞍一馬,不須分付!」

囑付了吳月娘,又把陳經濟叫到根前,說道:「姐夫,我養兒靠兒,無兒靠婿;姐夫就是我的親兒一般。我若有些山高水低,你發送了我入土,好歹一家一計,幫扶着你娘兒們過日子,休要教人笑話!」又分付:「我死後,段子鋪是五萬銀子本錢,有你喬親家爹。那邊多少本利,那找與他。教傅夥計把貨賣一宗交一宗,休要開了。賁四絨線鋪,本銀六千五百兩;吳二舅紬絨鋪是五千兩,都賣盡了貨物,收了來家。又李三討了批來,也不消做了,教你應二叔拏了別人家做去罷。李三、黃四身上還欠五百兩本錢,一百五十兩利錢未算,討來發送我。你只和傅夥計,守着家門這兩個鋪子罷!段子鋪占用銀二萬兩,生藥鋪五千兩。韓夥計、來保松江船上四千兩。開了河,你早起身往下邊接船去,接了來家,賣了銀子,交進來你娘兒們盤纏。前邊劉學官還少我二百兩;華主簿少我五十兩;門外徐四鋪內,還本利久我三百四十兩,都有合同見在,上緊使人催去。到日后,對門并獅子街兩處房子,都賣了罷,只怕你娘兒們顧攬不過來。」說畢。哽哽咽咽的哭了。陳經濟道:「爹囑付兒子,都知道了。」不徐顧,且守着月娘,拏榪子伺候。見月娘看看疼的緊了,不一時蔡老娘到了,登時生下一個孩子來。這屋裏裝柳西門慶停當,口內纔沒了氣兒,合家大小,放聲號哭起來。蔡老娘收裹孩兒,剪去臍帶,煎定心湯與月娘吃了,扶月娘煖炕上坐的。月娘與了蔡老娘三兩銀子,蔡老娘嫌少,說道:「養那位哥兒賞了我多少,還與我多少便了。休說這位哥兒,是大娘生養的。」月娘道:「比不的那時,有當家的老爹在此。如今沒了老爹,將就收了罷。待洗三來,再與你一兩就是了。」那蔡老娘道:「還賞我一套衣服罷。」拜謝去了。月娘甦省過來,看見廂子大開着,便罵玉筲道:「賊臭肉,我便昏了,你也昏了!廂子大開着,恁亂烘烘人走,就不說鎖鎖兒!」玉筲說:「我只說娘鎖了廂子,就不曾看見。」於是取鎖來搯。玉樓見月娘多心,就不肯在他屋里。走出對着金蓮說:「原來大姐姐恁樣的,死了漢子頭一日,就防範起人來了!」殊不知李嬌兒已偷了五定元寶往屋里去了。當下吳二舅、賁四往尚推官家買了一付棺材板來,教匠人解鋸成槨。眾小廝把西門慶抬出,停當在大廳上,請了陰陽徐先生來批書。不一時,吳大舅也來了。吳二舅、眾夥計,都在前廳熱亂,收燈捲畫,蓋上紙被,設放香燈幾席。來安兒專一打磬。徐先生看了手,說道:「正辰時斷氣,合家都不犯凶煞。」請問月娘,三日大殮,擇二月十六日破土出殯,也有四七多日子。一面管待徐先生去了,差人各處報喪,交牌印往何千戶家去。家中破孝搭棚,俱不必細說。到三日請僧人念倒頭經,挑出紙錢去,合家大小,都披蔴帶孝。女婿陳經濟斬衰治杖,靈前還禮。月娘在暗房中出不來。李嬌兒與玉樓陪堂客。潘金蓮管理庫房收祭卓。孫雪娥率領家人媳婦在廚下打發各項人茶飯。傅夥計、吳二舅管帳,賁四管孝帳,來興管廚,吳大舅與甘夥計陪待人客。蔡老娘來洗了三次,月娘與了一套紬子衣裳,打發去了。就把孩子不一時打夥兒,傅夥計、甘夥計、吳二舅、賁四、崔本都進來看視問安。西門慶一一都分付了一遍。眾人都道:「你老人家寬心,不妨事。」見一日來問安看者,也有許多。見西門慶不好的沉重,皆嗟嘆而去。過了兩日,月娘痴心,只指望西門慶還好,誰知天數造,三十三歲而去。到於正月二十一日,五更時分,相火燒身,變出風來,聲若牛吼一般,喘息了半夜。捱到早辰巳牌時分,鳴呼哀哉,斷氣身亡!正是:

「三寸氣在千般用,一旦無常萬事休!」

古人有幾句格言說得好:

「為人多積善,不可多積財;積善成好人,積財惹禍胎。石崇當日富,難免殺身災;鄧通飢餓死,錢山何用哉!今日非古比,心地不明白。只說積財好,反笑積善呆!多少有錢者,臨了沒棺材!」

原來西門慶一倒頭,棺材尚未曾預備。慌的吳月娘叫了吳二舅與賁四到根前,開了廂子,拏出四定元寶,教他兩個看材板去。剛打發去了,不防月娘一陣就害肚里疼,急撲進去看床上倒下,就昏運不省人事。孟玉樓與潘金蓮、孫雪娥都在那邊屋里七手八腳,替西門慶戴唐巾,裝柳穿衣服。忽聽見小玉來說:「俺娘跌倒在床上!」慌的玉樓、李嬌兒就來問視。月娘手按着害肚內疼,就知道決撒了!玉樓教李嬌兒守著月娘,他便就使小廝快請蔡老娘去。李嬌兒又使玉筲,前邊教如意兒來了。比及玉樓回到里面屋里,不見李嬌兒。原來李嬌兒趕月娘昏沉,房內無人,箱子開着,暗暗拏了五定元寶,往他屋裏去了。手中拏將一搭紙,見了玉樓,只說:「尋不見草紙,我往房里取草紙去來。」那玉樓也改名叫孝哥兒。未免送些喜麵親鄰與街坊鄰舍。都說:「西門慶大官人正頭娘子,生了一個墓生兒子,就與老頭同日同時;一頭斷氣,一頭生了個兒子。世間少有蹺蹊古怪事!」不說眾人理亂這庄事。且說應伯爵聞知西門慶沒了,走來吊孝哭泣。哭了一回,吳大舅、二舅正在捲棚內看着與西門慶傳影。伯爵走來與眾人見禮,說道:「可傷,做夢不知哥沒了!」要請月娘出來拜見。吳大舅便說:「舍妹暗房出不來。如此這般,就是同日添了個娃兒!」伯爵愕然道:「有這等事!也罷,也罷!哥有了個後代,這家當有了主兒了!」落後陳經濟穿着一身重孝,走來與伯爵磕頭!伯爵道:「姐夫,姐夫煩惱,你爹沒了,你娘兒們是死水兒了!家中凡事,要你仔細。有事不可自事專,請問你二位老舅主張。不該我說,你年幼,事體上還不大十分歷練。」吳大舅道:「二哥,你沒的說。我也有公事,不得閑,見有他娘在。」伯爵道:「好大舅,雖故有嫂子,外邊事怎麼理的?還是老舅主張!自古沒舅不生,沒舅不長。一個親娘舅,比不的別人。你老人家就是個都根主兒,再有誰大如你老人家的!」因問道:「有了發引的日期?」吳大舅道:「擇在二月十六日破土,三十日出殯,也在四七之外。」不一時,徐先生來到,祭告入殮,將西門慶裝入棺材內,用長命丁釘了。安放停當,題了名旌:誥封武略將軍西門公之柩。那日何千戶來吊孝,靈前拜畢,吳大舅與伯爵陪侍吃茶,問了發引的日期。何千戶分付手下該班排軍,會答應的,一個也不許動,都在這里伺候。直過發引之後方許回衙門當差。委兩名節級管領,如有違誤,呈來重治!又對吳大舅道:「如有外邊人拖久銀兩不還者,老舅只顧說來,學生即行追治。」吊孝畢,到衙門里,一面行文開鈌,申報東京本衛去了。話分兩頭,都說來爵、春鴻同李三,一日到袞州察院投下了書禮。宋御史見西門慶書上,要討古器批文一節,說道:「你早來一步便好。昨日已都派下各府買辦去了!」尋思間,又見西門慶書中封着金葉十兩,又不好違阻了的,須得留下春鴻、來爵、李三在公廨駐劄。隨即差快子拏牌,趕回東平府批文來,封回與與春鴻書中,又與了一兩路費,方取路回清河縣,往返十日光景。走進城,就聞得路上人說:「西門大官人死了!今日三日,家中念經做齋哩!」這李三就心生奸計,路上說念來爵、春鴻:「將此批文按下,說宋老爹沒與來。咱每都投到大街張二官府那里去罷!你二人不去,我與你每人十兩銀子,到家隱住不拏出來就是了!」那來爵見財物,倒也肯了。只春鴻些不肯,口里含糊應諾。到家見門首挑着紙錢,僧人做道場,親朋吊喪者,不計其數。這李三就分路回家去了。來爵、春鴻見吳大舅、陳經濟磕了頭。問:「討的批文如何?怎的李三不來?」那來爵還不言語,這春鴻把宋御史書連批,都拏出來,遞與大舅,悉把李三路上與的十兩銀子,說的言語,如此這般,教他隱下休拏出來,同他投往張二官家去。「小的怎敢忘恩背義!敬奔家來。」吳大舅一面走到後邊,告訴月娘:「這個小的兒,就是個有恩的。叵耐李三這廝短命,見姐夫沒了幾日,就這等壞心!」因把這件事對應伯爵說:「李智、黃四借契上本利還欠六百五十兩銀子。趁着剛纔何大人分付,把這件寫紙狀子,呈到衙門里,教他替俺追追這銀子出來,發送姐夫!他同寮間,自恁要做分上。這些事兒,莫肯不依!」伯爵慌了說道:「李三卻不該行此事!老舅快休動意,等我和他說罷。」於是走到李三家,請了黃四來一處計較,說道:「你不該先把錢子遞與小廝,倒做了管手;狐狸打不成,倒惹了一屁股腰!他如今恁般恁般,要拏文書提刑所告你每哩!常言道:『官官相護』;何況又同寮之間,費恁難事?你等原抵鬬的過他?依我,不如此如此,這般這般,悄悄送上二十兩銀子與吳大舅,只當袞州府幹了事來了。我聽得說,這宗錢粮,他家已是不做了,把這批文難得掣出來,咱投張二官那里去罷。你每二人,再湊得二百兩,少了也拏不出來,再備辦一張祭卓,一者祭奠大官人,二者交這銀子與他,另立一紙欠結。你往後有了買賣,慢慢還他就是了。這個一舉而兩得,又不失了人情,有個始終!」黃四道:「你說的是!李三哥,你幹事忒慌速些了!」真個到晚夕,黃四同伯爵送了二十兩銀子到吳大舅家,如此這般:「討批文一節,累老舅張主張主!」這吳大舅已聽他妹子說,不做錢粮;何況又黑眼見了白晃晃銀子,如何不應承?於是收了銀子,到次日,李智、黃四備了一張插卓,猪首三牲,二百兩銀子,來與西門慶祭奠。吳大舅對月娘說了,拏出舊文書,從新另立了四百兩一紙久帖,饒了他五十兩。余者教他做上買賣,陸續交還。把批文交付與伯爵手內,同往張二官處合夥,上納錢粮去了,不在話下。正是:

「金逄火煉方知色,  人與財交便見心。」

有詩為證:

「造物於人莫強求,  勸君凡事把心收;

你今貪得收人業,  還有收人在後頭。」

畢竟未知後來如何,且聽下回分解:

