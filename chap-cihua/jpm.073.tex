%# -*- coding: utf-8 -*-
%!TEX encoding = UTF-8 Unicode
%!TEX TS-program = xelatex
% vim:ts=4:sw=4
%
% 以上设定默认使用 XeLaTex 编译,并指定 Unicode 编码,供 TeXShop 自动识别

%第七十三回 
\chapter{潘金蓮不憤憶吹簫\KG 郁大姐夜唱鬧五更}


「巧厭多乖拙厭閒,  善言懦弱惡嫌頑,

富遭嫉妒貧遭辱,  勤又貪圖儉又慳;

觸目不分皆笑拙,  見機而作又疑奸,

思量那件合人意,  為人難做做人難。」

話說應伯爵回家去了。西門慶正在花園藏春塢坐著,看泥水匠打地爐炕。墻外燒火,裡邊地暖如春。安放花草,庶不至於煤煙薰觸。忽見平安拿進帖來,稟說:「帥府周爺那里差人送分資來了。盒內封著五封分資:周守備、荊都監、張團練、劉薛二內相,每人五十星,粗帕二方,奉引賀敬。」西門慶令左右收入後邊,拿回帖打發來人去了。且說那日楊姑娘與吳大妗子、潘姥姥坐轎子先來了。然後薛姑子、大師父、王姑子并兩個小姑子妙趣、妙鳳并郁大姐,多買了盒兒來,與玉樓做生日。吳月娘在上房擺茶,眾姊妹都在一處陪侍。須臾吃了茶,各人都取便坐了。潘金蓮想著要與西門慶做白綾帶兒,三不知走到房裡,拿過針線匣,揀一條白綾兒,用扣針兒親手〈糸秋〉龍帶兒,用纖手向減粧磁盒兒內傾了些顫聲嬌藥末兒,裝在裡面周圍。又進房來,用倒口針兒撩縫兒,甚是細法,預備晚夕要與西門慶雲雨之歡。不想薛姑子驀地進房來,送那安胎氣的衣胞符藥。這婦人連忙收過,一連陪他坐的。這薛姑子見左右無人,悄悄遞與他,向他說:「多整理完備了。你揀了壬子日空心服,到晚夕與官人在一處,管情一度就成胎氣。你看後邊大菩薩,也是貧僧替他安的胎,今也有了半肚子了。我還說個法兒,與你縫做了錦香囊,我贖道硃砂雄黃符兒,安放在裡面,帶在身邊,管情就是男胎,好不准驗。」這婦人聽了,滿心歡喜。一面接了符樂,藏放在廂中。拿過曆日來看,二十九日是壬子日。于是就稱了三錢銀子送與他說:「這個不當什麼,拿到家買根菜兒吃。等坐胎之時,你明日稍了硃砂符兒來著,我尋疋絹與你做鍾袖。」薛姑子道:「菩薩,快休計較!我不相王和尚那樣利心重。前者因過世那位菩薩念經,他說我攙了他的主顧,好不和我兩個嚷鬧,到處拿言語喪我,我的爺,隨他墮業,我不與他爭執。我只替人家行好,救人苦難。」婦人道:「薛爺你只行的事,各人心地不同。我這里勾當,你也休和他說。」薛姑子道:「法不傳六耳。我肯和他說?去年為後邊大菩薩喜事,他還說我背地得了多少錢,擗了一半與他纔罷了。一個僧家,戒行也不知,利心又重,得了十方施主錢粮,不修功果。到明日死沒,披毛戴角還不起!」說了回話,婦人教春梅:「看茶與薛爺吃。」那姑子吃了茶,又同他到李瓶兒那邊參了參靈,方歸後邊來。約後晌時分,月娘兩個放卓兒,炕屋里請坐。諸堂客明間內,坐的齊整。錦帳圍屏,放八仙卓,鋪著火盆,擺的案酒。晚夕孟玉樓與西門慶遞酒,穿著何太監與他那五彩飛魚敞埝衣衣,白綾襖子,同月娘居上。其餘四位都兩邊列坐。不一時,堂中畫燭高燒,壺內羊羔滿泛。邵鎌、韓佐兩個優兒,銀箏象板,月面琵琶,席前彈唱:「紛紛瑞靄飄,朵朵祥雲墜。」玉樓打扮粉粧玉琢,蓮臉生春,與西門慶遞酒。花枝招颭,綉帶飄飄,磕了四個頭,然後方與月娘眾姊妹俱見了禮,安席坐下。只見陳經濟向前,大姐執壺,先遞了西門慶、月娘,後與玉樓上壽。行畢禮,傍邊坐下。廚下壽麵點心添換,一齊拿上來。只見來安拿進盒兒來說:「應寶送人情來了。」西門慶教月娘收了,教來安:「送應二娘帖兒去,請你應二爹和大舅來坐坐。我曉的他娘子兒,明日也是不來,請二哥來坐坐罷。改日回人情與他就是了。」來安拿帖兒同應寶去了。西門慶坐在上面,不覺想起去年玉樓上壽,還有李大姐。今日子母五個,只少了他,由不得心中痛,眼中落淚。不一時,李銘斟上酒,下邊吃。湯飯上來了,兩個小優兒也來了,月娘分付:「你會唱『比翼成連理』不會?」韓佐道:「小的有。」纔待拿起樂器來彈唱,被西門慶叫近前來,分忖:「你唱一套『憶吹簫』我聽罷。」兩個小優連忙改調唱集賢賓:

「憶吹簫,玉人何處也!今夜病較添些。白露冷秋蓮香,粉墻低皓月偏斜。止不過暫時間饒破釵分,倒勝似數十年信絕音絕。對西風,倚樓空自嗟。望不斷巖樹重叠,悄的是流光去馬,雁陳擺蛇。」

〔逍遙樂〕「歡娛前夜,喜根燈能,香玉帶結。剛得了和協,誰承望又早離別?常記得相靠相偎笑語碟。畫堂中那日驕奢,受用些。樽中線釵,扇底紅牙,枕上蝴蝶。」

〔醋葫蘆〕「我和他,那日相逢臉帶羞,乍交歡心尚怯。半裝醉,半裝醒,半裝呆。兩情濃,到今難棄。錦帳裡鴛鴦衾,方纔溫熱。把一枝鳳凰簪兒,做了三兩截。」

又:  「我和他,挑著燈將好句兒截,背著人惱心說。直等到,碧梧窗外影兒斜。惜花心怕將春漏,涉蒼苔腳尖輕立,露珠的常污了踏青靴。」

又:  「我為他,朋情上將說話兒丟,他與我母親個,將喬擽兒摭。我為何在家中,費盡了巧唳舌。他為我褪湘裙鵑花上血。」

原來潘金蓮見唱些詞,盡知西門慶念思李瓶兒之意。唱到此句,在席上故意把手放在臉兒上,這點兒、那點兒羞他,說道:「孩兒,那里豬八戒走到冷舖中坐著,你怎的醜的沒對兒,一個後婚老婆,又不是女兒,那里討杜鵑花上血來?好了沒羞的行貨子!」西門慶道:「怪奴才,我只知道,那里曉的什麼?」那個小優唱道:

又:  「我為他,耳輪兒常熱。他為我面皮紅羞,把扇兒遮蝴蝶兒。一個相府內懷春女,一個是君前門彈劍客,半路里忽逢者。剛幾個千金夜,忽刺八拋去也!我怎肯恁隨邪,又去把墻花亂。」

〔後庭花〕「夢了些,虛飄飄,枕上蝴蝶。聽了些,咭叮噹簷前鐵。剛合上溫郎鏡,又早攔回卓氏車。我這里痛傷嗟,鴛帳冷,香消蘭麝。困將來,剛困些望陽,臺道路賒,那愁怎打疊?這相思索害他看銀河直又斜,對孤燈又滅。」

〔青歌兒〕「呀!風亂灑階前階前,黃葉一半遮。柳梢,柳稍殘月。這離情,比前春較陡些。害也斜瘦的唓嗻。待桑田重變,海枯渴,還不了風流業。浪里來,煞這愁,剛還在眼角哲,一又來到眉上惹。恨不的倩三尸腑細鑑碣。有一日綉幃中,肌玉重廝貼。我將他指尖兒輕捏,直說到樓頭北斗柄兒斜。」

唱畢,那潘金蓮不憤他唱這套,兩個在席上只顧拌嘴起來。月娘就有些看不上,便道:「六姐你也耐煩,兩個只顧且強什麼?楊姑奶奶和他大妗子,丟的在屋里冷清清的,沒個人兒陪他。你每著兩個進去陪他坐坐兒,我就來。」當下金蓮和李嬌兒往房裡陪楊姑娘、潘姥姥、大妗子坐去了。不一時,只見來安向前說:「應二娘帖兒送到了。二爹來了,大舅便來。」西門慶道:「你對過請溫師父來坐坐。」因對月娘說:「你分付廚下拿菜出來,我前邊陪他坐去。」又叫李銘:「你往前邊唱來罷。」李銘即跟著西門慶出來,西廂房內陪伯爵坐的,又謝他人情:「明日請令正好歹來看看。」伯爵道:「他怕不得來,家下沒人。」良久,溫秀才到,作揖坐下。伯爵舉手道:「早辰多有累老先生兒。」溫秀才道:「豈敢。」吳大舅也到了。相見讓位畢,一面琴童兒秉燭來。四人圍暖爐坐定,來安拿著春盛案酒,擺在卓上。伯爵燈下看見西門慶白綾襖子上,罩著青段五彩飛魚蟒衣,張爪舞牙,頭角崢嶸,揚鬚鼓鬣,金碧掩映,蟠在身上,諕了一跳。問:「哥,這衣服是那里的?」西門慶便立起身來,笑道:「你每瞧瞧,猜是那里的?」伯爵道:「俺每如何猜得著?」西門慶道:「此是東京何太監送我的。我在他家吃酒,因害冷,他拿出這件衣服與我披。這是飛魚,朝廷另賜了他蟒龍玉帶,他不穿這件,就相送了。此是一個大分上。」伯爵方極口誇獎:「這花衣服,少說也值幾個錢兒。此是哥的先兆,到明日高轉,做到都督上,愁玉帶蟒衣?何況飛魚,穿著界兒去了!」說著,琴童安放鍾筯,湯點心酒上來了。李銘在面前彈唱。伯爵道:「也該進去與三嫂遞杯酒兒纔好,如何就吃酒?」西門慶道:「我兒,你有孝順之心,往後邊與三嫂磕個頭兒就是了,說他怎的!」伯爵道:「不打緊,等我磕頭去。著緊磕不成頭,炕沿兒上見個意思兒出來就是了。」被西門慶向他頭上儘力打了一下,罵道:「你這狗材,單管恁沒大小!」伯爵道:「孩兒們若肯了,那個好意做大?」兩個又犯了回嘴,不一時拿將壽麵來。西門慶讓吳大舅、溫秀才、伯爵吃。西門慶因在後邊吃了,遞與李銘吃了。那李銘吃了,又上來彈唱。伯爵教吳大舅分付曲兒教他唱。大舅道:「不要索落他,隨他揀熟的唱去。」西門慶道:「大舅好聽瓦盆這一套兒。」一面令琴童斟上酒,李銘于是箏排雁柱,款定冰弦,這唱了一套「教人對景無言,終日減芳容。」下邊去了。只見來安上來稟說:「廚子家去,請問爹,明日叫幾名答應?」西門慶分付:「六名廚役,二名茶酒。明日具酒筵共五卓,俱要齊備。」來安應諾去了。吳大舅便問:「姐夫,明日請甚麼人?」西門慶悉把安郎中作東,請蔡九知府說了。吳大舅道:明日大巡在姐夫這里吃酒,又好了。」西門慶道:「怎的說?」吳大舅道:「還是我修倉的事,就在大巡手里題本。望姐夫明日說說,教我青白青白。到年終他若滿陞之時,圖他保舉一二,就是姐夫情分。」西門慶道:「這不打緊,大舅明日寫個履歷揭帖來,等我會便和他說。」這大舅連忙下來打恭。伯爵道:「老舅,你老人家放心。你是個都根主子,不替你老人家說,再替誰說?管情消不得吹噓之力,一箭就上垜。」前邊吃酒到二更時分散了。西門慶打發了李銘等出門,就分付:「明日俱早來伺候。」李銘等去了,小廝收進家活。上房內擠著一屋里人,聽見前邊散了,多往那房里去了。

說金蓮只說往他屋里去,慌的往外走不迭。不想西門慶進儀門來了,他便藏在影壁邊黑影兒裡,看著西門慶進入上房,悄悄走來窗下聽覷。只見玉筲站在堂屋門首,說道:「五娘怎的不進去?爹進來屋里來,和三娘多坐著不是?」又問:「姥姥怎的不見?」金蓮道:「老行貨子,他害身上疼,往房里睡去了。」良久,只聽月娘便問:「你今日怎的叫恁兩個新小王八子?唱又不會唱,只一味會三弄梅花。」玉樓道:「只你臨了教他唱鴛鴦浦蓮開,他纔依了你唱這套。好個猾小王八子,又不知叫什麼名字?一旦在這里只是頑。」西門慶道:「他兩個一個叫韓佐,一個叫邵謙。」月娘道:「誰曉的他叫什麼謙兒、李兒!」不防金蓮慢慢躡足潛踪,掀開簾兒進去,教他煖炕兒背後,便道:「你問他,正景姐姐分付的曲兒不教他唱,平白胡枝扯葉的,教他唱什麼憶吹簫李吹簫?支使的一飄個小王八子亂騰騰的,不知依那個的是。」這玉樓扭回看見是金蓮,便道:「是這一個六丫頭,你在那里來?猛可說出句話,倒諕我一跳。單愛行鬼路兒!你從多咱路在我背後?怎的沒看見你進來腳步兒响。」小玉道:「五娘在三娘背後好小一回兒。」金蓮點著頭兒,向西門慶道:「哥兒,你濃著些兒罷了!你的小見識兒,只說人不知道。他是甚『相府中懷春女?』他和我多是一般後婚老婆,什麼『他為你褪湘裙杜鵑花上血!』三個官唱兩個喏,誰見來?孫小官兒問朱吉,別的多罷了,這個我不敢許!可是你對人說的,自從他死了,好應心的菜也沒一碟子兒。沒了王屠,連毛吃豬。空有這些老婆睜著,你日逐只〈口床〉屎哩?見有大姐在上,俺每便不是上數的,可不著你那心的了!一個大姐怎當家理紀?也扶持不過你來,可可兒只是他好來?他死你怎的不拉掣住他?當初沒他來時,你也過來。如今就是諸般兒稱不上你的心了!題起他來,就疼的你這心里格地地的,拿別人當他借汁兒下麵,也喜歡的你要不的!只他那屋里水好吃麼?」月娘道:「好六姐,常言不說的:『好人不長壽,禍害一千年。』自古『鏇的不圓砍的圓。』你我本等是瞞貨,應不上他的心,隨他說去罷了!」金蓮道:「不是咱不說他,他說出來的話,灰人的心,只說人憤不過他。」那西門慶只是笑罵道:「怪小淫婦兒,胡說了!你我在那里說道這個話來?」金蓮道:「還是請黃內官那日,你沒對著應二和溫蠻子說:『從他死了,好菜也拿沒出一碟子來。』怪不的你老婆多死絕了!就是當初有他在,也不什麼的。到明日再扶一個起來,和他做對兒麼,賊沒廉恥撒根基的貨!」說的西門慶急了,跳起來,趕著拿靴腳踢他。那婦人奪門一溜烟跑了,這西門慶趕出去不見他,只見春梅站在上房門首。就一手搭伏著春梅肩背,往前邊來。月娘見他醉了,巴不的打發他前邊去睡,要聽三個姑子晚夕宣卷。于是教小玉打個燈籠,送他前邊去。金蓮和玉筲站在穿廊下黑影中,西門慶沒看見他。玉筲向金蓮道:「我猜爹管情向娘屋里去了。」金蓮道:「他醉了快發訕,由他先睡。等我慢慢進去。」這玉筲便道:「娘你等等,我取些菓子兒稍與姥姥吃去。」于是走到牀房內,袖出兩個柑子,兩個蘋波,一包蜜餞,三個石榴與婦人。婦人接的袖了,一直走到他前邊。只見小玉送了西門慶回來,說道:「五娘端的那邊?爹好不尋五娘。」這金蓮到房門首不進去,悄悄向窗眼里望裡張覷。覷看見西門慶坐在牀上,正摟著春梅做一處頑耍。恐怕攪擾他,連忙走到那邊屋里,把秋菊將菓子交付與了他。因問:「姥姥睡沒有?」秋菊道:「睡了一大回了。」囑付他:菓子好生收在揀粧內。」原復往後邊來。只見月娘、李嬌兒、孟玉樓、西門大姐、大妗子、楊姑娘并三個姑子,帶兩個小姑子妙趣、妙鳳,坐了一屋里人。姑子便盤膝坐在月娘炕上,薛姑子在當中,放著一張炕卓兒,炷了香,眾人多圍著他,聽他說佛法。只見金蓮笑掀簾子進來。月娘道:「你惹下禍來,往他屋里尋你去了。你不打發他睡,如何又來了?他到屋里打你?」金蓮笑道:「你問他敢打我不敢?」月娘道:「他不打你嫌我見,你頭里話出來的忒緊了。常言:『漢子臉上有狗毛,老婆臉上有鳳毛。』他有酒的人,我怕一時激犯他起來,激的惱了,不打你打狗不成?俺每倒替你捏兩把汗,原來你到這等潑皮!」金蓮道:「他就惱我,也不怕他。看不上那三等兒九格的!正景姐姐分付的曲兒不教唱,且東溝黎西溝耙,支使的個小王八子亂烘烘的,不知依那個的是。就是今日孟三姐好的日子,不該唱憶吹筲這套離別之詞。人也不知死那里去了,偏有那些佯慈悲假孝順,我和刺不上!」大妗子道:「你姐兒每亂了這一回,我還不知因為什麼來?姑夫好好的進來坐著,怎的又出他去了?」月娘道:「大妗子,你還不知道。那一個因想起李大姐來,說年時孟三姐生日還有他,今年就沒他了。落了幾點眼淚,教小優兒唱了一套『憶吹筲,玉人兒何處也。』這一個就不憤他唱這詞,剛纔搶白了爹幾句。搶白的那個急了,趕著踢打,這賊就走了。」楊姑娘道:「我的姐姐,你隨官人分付教他唱罷了,又搶白他怎的?想必每常見姐姐,每多全全兒的,今日只不見了李家姐姐,漢子家的心,怎麼不慘切個兒?」玉樓道:「好奶奶,這半日你還歌唱!誰嗔他唱?俺這六姐姐,平昔曉的曲子里滋味。那個誇死了的李大姐,比古人那個不如他。又尚的怎的兩個交的情厚,又怎麼沒山盟海誓,你為我,我為你,無比賽的好!這個牢成的又不顧慣,只顧拿言語白他,和他整廝亂了這半日。」楊姑娘道:「我的姐姐,原來這等聰明!」月娘道:「他什麼曲兒不知道!但題起頭兒,就知尾兒。相我若叫唱老婆和小優兒來,俺每只曉的唱出來就罷了。偏他又說那一段兒唱的不是了,那一句兒唱的差了,又那一節兒稍了。但在他爹說出來個曲兒,就和爹熱亂,兩個白搽白的,必須搽惱了纔罷。俺每使不去管他。」孟玉樓在傍戲道:「姑奶奶,你不知我,三四胎兒,只存了這個丫頭子。這丫頭子這般精靈兒古怪的!如今他大了,成了人兒,就不依我管教了。」金蓮便向他打了一下,笑道:「你又做我的,又來打上輩我的娘起來了。」玉樓道:「你看恁慣的少條兒尖教的,又來打上輩。」楊姑娘道:「姐姐,你今後讓他官人一句兒罷。常言:『一夜夫妻百日恩。』相隨百步,也有個徘徊之意。一個熱突突人兒,指頭兒似的少了一個,如何不想不疼不題念的!」金蓮道:「想怎的不想,也有個常時兒!一般都是你的老婆,做什麼擡一個滅一個?俺每多是劉湛兒鬼兒,不出村的!大姐在後邊,他也不知道。你還沒見哩,每日他從那里吃了酒來,就先到他房里,望著他影,深深唱諾,口里恰似嚼蛆一般,供著個羹飯兒著,舉筯兒,只像活的一般兒讓他,不知什麼張致!又嗔俺每不替他戴孝,俺每便不說。他又不是婆婆,胡亂帶過斷七罷了,只顧帶幾時?又與俺每亂了幾場。」楊姑娘道:「姐姐們見一半不見一半兒罷!」楊姑娘道:「好快,斷七過了這一向,又早百日來。」姑娘問:「幾時是百日?」月娘道:「早哩,到蠟月二十六日。」王姑子道:「少不的念個經兒?」月娘道:「挨年近節,忙忙的,且念什麼經?他爹只怕過年念罷了。」說著,只見小玉拿上一道土荳泡茶來,每人一盞。須臾吃畢,月娘洗手,向爐中炷了香,聽薛姑子講說佛法。先念揭曰:

「禪家法教豈非凡,  佛祖家傳在世間;

落葉風飄著地易,  等閒復上故枝難。」

此四句詩,單說著這為僧的,戒行最難。言人生就如同鐵樹一般,落得容易,全枝復節甚難;墮業容易,成佛作祖難。卻說當初治平年間,浙江寧海軍錢塘門南山淨慈古孝剎,有兩個得道的真僧,一個喚作五戒禪師,如何謂之五戒?第一不殺生命,第二不偷財物,第三不染淫聲美色,第四不飲酒茹葷,第五不妄言綺語。如何謂之明悟?言其明心見性,覺悟我真。這五戒禪師的家年方三十一歲,身不滿三尺,形容古怪;自伊師明悟,少其一目,俗名金禪,字佛教,如法了得,他與明悟是師兄師弟。一日同來寺中,訪大行禪師。禪師觀五戒佛法曉得,留在寺中做個首座。不數年大行圓覺,眾僧選他做了長老,每日到坐。那第二個明悟,年二十九歲,生得頭圓耳大,面闊口方,身體長大兔數羅汗,俗姓王,兩個如同一母所生。但遇說法,同外法應。忽一日冬盡春初時節,天道嚴寒,作雪下了兩日,雪霽天晴。這五戒禪師早辰坐在禪椅上,耳邊連連只聞得小兒啼哭,便叫一個身邊知心腹的清一道人:『你往山門前看有甚事來?報我知道。』這道人開了山門,見松樹下雪地上一塊破蓆,放著一個小孩兒。這是什麼人家丟在此處?向前看,是五六個月的女孩兒,破衣包裹。懷內片紙,寫著他生時八字。清一道:『救人一命,勝造七級浮屠。』連忙到方丈稟知長老,長老道:『善哉!難得你善心。』即抱回房中好生喂養,救他性命,這是好事。到了周歲,長老起了個名字,喚做紅蓮。日往月來,養在寺中,無人知覺,一向長老也忘了。不覺紅蓮長成十六歲。清一道人每日出鎖入鎖,如親生女一般。女子衣服鞋襪,如沙彌打扮,且是生得清俊。無事在房做針線,只指望招尋個女婿,養老送終。一日六月熱天,這五戒禪師忽想數十年前之事,逕來千佛閣後清一道人房中來。清一道:『長老希行,來此何幹?』五戒因問:『紅蓮女子在于何處?』清一不敢隱諱,請長老進房。一見就差了念頭,邪心輒起。分付清一:『你今早送他到我房中,不可有誤,你若依我,後日擡舉你,切不可泄漏與人。』清一不敢不依,暗思今夜必壞了這女身。長老見他應得不爽利,喚入方丈,與了他十兩白金及度牒。清一只得收了銀子,至晚送紅蓮到方丈。長老遂破了他身,每日藏鎖他在牀後紙帳房內,把些飯食與他吃。卻說他師弟明悟禪師在禪牀上入定回來,已知五戒差了念頭,犯了色戒,淫垢了紅蓮女子,把多年德行一旦拋棄了。我去勸醒,再不可如此。次日寺門前荷蓮花開,明悟令行者採一朵白蓮花來,插在膽瓶內,令請五戒來賞蓮花,吟詩談笑。不一時五戒至,兩個禪師坐下。明悟道:『師兄我今日見此花甚盛,竟請吾兄賞玩,吟詩一首。』行者拿茶吃了,預備文房四寶。五戒道:『將那荷根為題。』明悟道:『便將蓮花為題。』五戒控起筆來,寫詩四句:

『一枝菡萏瓣兒張,  相伴蜀葵花正芳;

紅留似火開如錦,  不如翠蓋芰荷香。』

明悟道:『師兄有詩,小弟豈得無詩?』于是拈筆寫四句:

『春來桃杏柳舒張,  千花萬蕊鬥芬芳;

夏賞芰荷如燦錦,  紅蓮爭似白蓮香!』

寫畢呵呵大笑。五戒聽了此言,心中一悟,面有愧色。轉身辭回方丈,命行者快燒湯洗浴罷,換了一身新衣,取紙筆忙寫八句頌曰:

『吾年四十七,  萬法本歸一,

只為念頭差,  今朝去得急;

傳語悟和尚,  何勞苦相逼,

幻身如閃電,  依舊蒼天碧。』

寫畢,放在佛前,歸到禪牀上就坐化了。行者忙去報與明悟。明悟聽得大驚,走來佛前看見辭世頌,遂說:『你好卻好了,只可惜差了這一著!你如今雖得個男身去,你不信佛法三寶,必然滅佛謗僧,後世墮落苦輪,不得歸依正道,深可痛哉!你道你去得,我趕你不著。』當下歸房,令行者燒湯洗浴,坐在禪牀上:『吾今趕五戒和尚去也,汝可將兩個人神子盛了,放三日一時焚化。』說畢,亦圓寂坐化。眾僧皆驚,有如此異事?傳得四方知道,本寺連日坐化了兩僧,燒香禮拜,佈施者人山人海,擡去寺前焚化。這清一道人遂收紅蓮改嫁平人養老。不日後,五戒托生,在西川眉州,與蘇老泉居士做兒子,名喚蘇軾,字子瞻,號東坡。明悟托生與本州姓謝道法為子,為端卿,後出家為僧,取名佛印。他兩個還在一處作對,相交契厚。正是:

「自到川中數十年,  曾在芸盧頂上眠,

參透趙洲關捩子,  好姻緣做惡姻緣;

桃紅柳綠還依舊,  石邊流水響潺潺,

今影指引菩堤路,  再休錯意戀紅蓮。」

薛姑子說罷,只見玉樓房中蘭香,拿了兩方盒細巧素菜菓碟,茶食點心,收了香爐,擺在卓上,又是一壺茶,與眾人陪三個師父吃了。然後又拿葷下飯來,打開一罈麻姑酒 ,眾人圍爐吃酒。月娘便與大妗子擲色兒搶紅。金蓮便與李嬌兒猜枚。玉筲便傍邊斟酒,又替金蓮打卓底下轉了兒。須臾,把李嬌兒贏了數杯。玉樓道:「等我和你猜,你只顧贏他罷。」這玉樓要金蓮露出手來,不許他褪在袖口邊。玉筲不許他近前。當夜一連反贏了金蓮幾鍾酒,又教郁大姐彈唱。月娘道:「你唱了鬧五更俺每聽。」郁大姐便調絃高聲唱玉交枝道:

「彤雲密布剪,鵝雪花辭舞,朔風凜冽穿窗戶。你心毒,奴更受苦。爹娘罵得奴心忒狠毒,你說來的話全不顧。把更兒從頭細數。」

〔金字經〕「夜迢迢,孤另另,冷清清,更靜初,不寄平安一紙書。腮邊流淚珠,不把佳期顧,一更里無限的苦。」

〔玉交枝〕「一更纔至冷清,撇奴在帳里,番來復去如何睡?二更里淚珠垂。」

又:  「二更難過,討一覺頻頻的睡著。今宵今宵,夢兒里來托。我思他,他思我,去時節海棠花兒開了半朵,到如今樹葉兒皆零落,枉教奴痴心兒等著。」

〔金字經〕「我痴心終日家等待你,何日是可,合少離多咱命薄;命薄,孤另孤另,怎生奈何!好著教難存坐,三更里睡夢兒多。」

〔玉交枝〕「三更月上好難挨,今宵夜長。燒殘蠟燭,銀臺上淚珠流三兩行。紅綾的被兒,閒了半牀。新桃的手帕兒在誰行放,瘦損了腰肢,腰肢沈郎。」

〔金字經〕「沈郎的腰肢瘦,每日家愁斷了腸。盼望情人淚兩行;兩行,對菱花懶去粧。瘦損了嬌模樣,四更里偏夜長。」

〔玉交枝〕「四更如晝枕邊想,不覺的淚流。靈神廟里曾發呪,剪青絲兩下里收。說來的話兒不應口,到如今閃的我,似章臺柳;柳,教奴痴心等守。」

〔金字經〕「我痴心終日家等待你,何日是休?望盼情人空倚樓,倚樓,想情人一筆勾,不由把眉雙皺,五更里淚珠流。」

〔玉交枝〕「五更雞唱,看看兒天色漸曉。放聲,欲待放聲,又恐怕傍人笑,一全家心內焦。燒香告禱神前筊,負心的自有天知道,枉教奴痴心等著。」

〔金字經〕「我痴心終日家等待你,何日是了?簷外叮噹鐵馬兒敲兒敲,攪的奴睡不著。一壁廂寒鴉叫,淒淒涼涼直到曉。」

〔玉交枝〕「曉來梳洗傍粧臺,懶上畫眉。房簷上喜鵲兒喳喳的,小梅香來報喜。報道是有情郎,真個歸奴,奴向入羅幃里,向前來奴家問你。」

〔後庭花〕「我問你個負心賊,你盡知一去了,半年來怎生無個信息?我道你應舉求官去,誰想你戀烟花家貪酒杯。我為你受孤恓,在那里偎紅倚翠?我為你病懨懨減了飲食,瘦伶仃消了玉體。挨清晨怕夕晚,一更里聽天邊孤雁飛,二更里想情人魂夢里,五更里醒來時不見你。」

〔柳葉兒〕「呀!空閒了鴛鴦錦被,寂寞了蒸約蒸約鶯嘶。海神廟見放著傍州例,不由我心中氣。你盡知負心的,自有個天知道。」

〔尾聲〕「流蘇錦帳同歡會,錦被里鴛鴦成對,永遠團圓直到底。」

當下金蓮與玉樓猜枚,被玉樓贏了一二十鍾酒,坐不住,往前邊去了。到前邊叫了半日,角門纔開。只見秋菊操眼,婦人罵道:「賊奴才,你睡來?」秋菊道:「我沒睡。」婦人道:「見睡起來,你哄我。你倒自在,就不說往後來接我要兒去。」因問:「你爹睡了?」秋菊道:「爹睡了這一日了。」婦人走到炕房里,摟起裙子來,就坐在炕上烤火。婦人要茶吃,秋菊連忙傾了一盞茶來。婦人道:「賊奴才,好乾淨手兒,你倒茶我吃!我不吃這陳茶,熬的怪泛湯氣!你叫春梅來,教他另拿小銚兒頓些好甜水茶兒,多著些茶葉,頓的苦艷艷我吃。」秋菊道:「他在那邊牀屋里睡哩,等我叫他起來。」婦人道:「你休叫他,且教他睡罷。」這秋菊不依,走到那邊屋里,見春梅〈扌歪〉在西門慶腳頭,睡得正好。被他搖推醒了,道:「娘來了,要吃茶,你還不起來哩。」這春梅噦他一口,罵道:「見鬼的奴才,娘來了罷了,平白唬人刺刺的!」一面起來,慢條斯禮,撒腰拉袴,走來見婦人。只顧倚著眼兒揉眼。婦人反罵秋菊:「恁奴才!你睡的甜甜兒的,把你叫醒了。」因教他:「你頭上汗巾子跳上去了,還不往下扯扯哩。」又問:「你耳躲上墜子,怎的只帶著一隻?往那里去了?」這春梅摸了摸,果然只有一隻金玲瓏墜子。便點燈往那邊牀上尋去,尋不見。良久,不想落在牀腳踏板上,拾起來。婦人問:「在那里來?」春梅道:「都是他失驚打怪叫我起來,乞帳鈎子抓下來了,纔在踏板上拾起來。」婦人道:「我那等說著,他還只當叫起你來。」春梅道:「他說娘要吃茶來。」婦人道:「我要吃口茶兒,嫌他那手不乾淨。」這春梅連忙舀了一小銚了水,坐在火上,使他撾了些炭放在火內。須臾,就是茶湯,滌盞兒乾淨,濃濃的點上去遞與婦人。婦人問春梅:「你爹睡下多大回了?」春梅道:「我打發睡了這一日了。問娘來,我說娘在後邊還未來哩。」這婦人吃了茶,因問春梅:「我頭裡袖了幾個菓子和蜜餞,是玉簫與你姥姥吃的,交付這奴才接進來,你收了?」春梅道:「我沒見他,知道放在那里?」這婦人一面叫秋菊問他:「菓子在那里?」秋菊道:「有,我放在揀粧內哩。」走去取來。婦人數了一數,只是少了一個柑子。問他:「那里去了?」秋菊道:「娘遞與拿進來,就放在揀粧內。那個害饞癆爛了口吃他不成?」婦人道:「賊奴才,還漲漒嘴!你不偷,往那去了?我親手數了交與你的。賊奴才,你看省手拈搭的,零零落落只剩下這些兒。乾淨吃了一半,原來只孝順了你!」教春梅:「你與我把那奴才一邊臉上打與他十個嘴巴。」春梅道:「那臢臉彈子,倒沒的齷齪了我這手!」婦人道:「你與我拉他。」雙手推顙到婦人跟前。婦人用手撙著他腮頰,罵道:「賊奴才,這個柑子是你偷吃了不是?你即實實說了,我就不打你。不然取馬鞭子來,我這一旋剝,就打了不數!我難道醉了?你偷吃了,一徑裡灤混我!」因問春梅:「我醉不醉?」那春梅道:「娘清淨白淨,那討酒來?娘信他,不是他吃了。娘不信,掏他袖子,怕不的還有柑子皮兒在袖子裡不止的。」婦人于是扯過他袖子來,用手掏他袖子,用手撇著不教掏。春梅一面拉起手來,果然掏出些柑子皮兒來。被婦人儘力臉上擰了兩把,打了兩個手八,便罵道:「賊奴才,痞不長俊!奴才你諸般兒不,一相這話舌偷嘴吃偏會!剛纔掏出皮來,吃了真賍實犯拿住,你還賴那個?我如今要打你,你爹睡在這里。我茶前酒後,我且不打你。到明日清淨白省,和你算帳!」春梅道:「娘到明日,休要與他行行忽忽的。好生旋剝了,教一個人,把他實辣辣打與他幾十板子。教他忍疼,他也懼怕些。甚麼鬬猴兒似的湯那幾棍兒,他纔不放心上!」那秋菊被婦人擰的臉脹腫的,谷都著嘴,往廚下去了。婦人把那一個柑子平白兩半,又拿了個蘋婆 、石榴,遞與春梅,說道:「這個與你吃。把那個留與姥姥吃。」這春梅也不瞧,接過來似有如無掠在抽屜內。婦人把蜜蒸也要分開,春梅道:「娘不要分,我懶待吃這甜行貨子,留與姥姥吃罷。」以此婦人不分,都留下了不題。婦人走到桶子上小解了,教春梅掇進坐桶來,澡了牝。又問春梅:「這咱天有多少時分?」春梅道:「月兒大倒西,也有三更天氣。」婦人摘了頭面,走來那邊牀房里,見卓上銀燈已殘,從新剔了剔,向牀上看,西門慶正打鼾睡。于是解鬆羅帶,卸褪湘裙,坐換睡鞋,脫了褌褲,上牀鑽在被窩裡與西門慶並枕而臥。睡下不多時,向他腰間摸他那話,弄了一回,白不起。原來西門慶與春梅纔行房不久,那話綿軟,急切捏弄不起來。這婦人酒在腹中,慾情如火,蹲身在被底,把那話用口吮咂,挑弄蛙口,吞裹龜頭,只顧往來不絕。西門慶猛然醒了,見他在被窩裡,便道:「怪小淫婦兒,如何這咱纔來?」婦人道:「俺每在後邊吃酒,孟三兒又安排了兩大方盒酒菜兒。郁大姐唱著,俺每陪大妗子、楊姑娘猜枚擲骰兒,又頑了這一日,被我把李嬌兒先嬴醉了。落後孟三兒和我兩個五子三猜,俺兩個到輸了好幾鍾酒。你到是便益,睡起一覺兒來好熬我。你看我依你不依!」西門慶道:「你整治那帶子了?」婦人道:「在褥子底下不是?」一面探手取出來與西門慶看了,扎在塵柄根下,繫在腰間,拴的緊緊的。又問:「你吃了不曾?」西門慶道:「我吃了。」須臾那話,乞婦人一壁廂弄起來。只見奢稜跳腦,挺身直舒,比尋常更舒七寸有余。婦人扒在身上,龜頭昂大,兩手搧著牝戶往裡放,須臾突入牝中。婦人兩手摟定西門慶脖項,令西門慶亦扳抱其腰,在上只顧揉搓,那話漸沒至根。婦人叫西門慶:「達達,你取我的示塀主腰子,墊在你腰底下。」這西門慶便向牀頭取過他大紅綾抹胸兒,四摺叠起,墊著腰。這婦人在他身上馬伏著,那消幾揉,那話盡入。婦人道:「達達,你把手摸摸,多全放進去了,撑的裡頭滿滿兒的,你自在不自在?多揉進去。」西門慶用手摸摸,見盡沒至根,間不容髮,止剩二卵在外,心中覺翕翕然,暢美不可言。婦人道:「好急的慌,只是觸冷,咱不得拿燈兒照著幹。趕不上夏天好,這冬月間,只是冷的慌。」因問西門慶說道:「這帶子比那銀托子識好不好?強如格的陰門生疼的。這個顯得該多大又長出許多來,你不信摸摸,我小肚子七八頂到奴心。」又道:「你摟著我,等我今日一發在你身上睡一覺。」西門慶道:「我的兒,你睡達達。」摟著那婦人,把舌頭放在他口裡含著,一面朦朧星眼,欵抱香肩。睡不多時,怎禁那慾火燒身,芳心掩亂,于是兩手按著他肩膊,一舉一坐,抽徹至首,復送至根,叫:「親心肝罷了,六兒的心了。」往來抽捲,又三百回,比及精洩。婦人口中只叫:「我的親達達,把腰扱緊了。」一面把奶頭教西門慶咂,一覺一陣昏迷,淫水溢下。停不多回,婦人兩個抱摟在一處,婦人心頭小鹿實實的跳,登時四肢囷軟,香雲撩亂,于是洩出來,猶剛勁如故。婦人用帕搽之,便道:「我的達達,你不過卻怎麼的?」西門慶:「等睡起一覺來再耍罷。」婦人道:「我也挨不的身子,已軟癱熱化的。」當下雲收雨散,兩個並肩交股,枕籍于牀上寐,不覺東方之既白。正是:

「等門試把銀釭照,  一對天生連理人。」

畢竟未知後來何如,  且聽下回分解:


