%# -*- coding: utf-8 -*-
%!TEX encoding = UTF-8 Unicode
%!TEX TS-program = xelatex
% vim:ts=4:sw=4
%
% 以上设定默认使用 XeLaTex 编译,并指定 Unicode 编码,供 TeXShop 自动识别

%第二十九回 
\chapter{吳神仙貴賤相人\KG 潘金蓮蘭湯午戰}

「百年秋月與春花,  展放眉頭莫自嗟,

吃幾首詩消世慮,  酌二杯酒度韶華;

閒敲棋子心情樂,  悶撥瑤琴興趣賒,

人事與時俱不管,  且將詩酒作生涯。」

話說到次日,潘金蓮早起,打發西門慶,記掛着要做那紅鞋。拿青針線筐兒,往花園翡翠軒臺基兒上坐着,那裡描畫鞋扇,使春梅請了李瓶兒來到。李瓶兒問道:「姐姐,你抽金的是甚麼?」金蓮道:「要做一雙大紅光素段子,白綾平底鞋兒,鞋尖兒上扣繡鸚鵡摘桃。」李瓶兒道:「我有一方大紅十樣錦段子,也照依姐姐描恁一雙兒,我要做高底的罷。」於是取了針線筐,兩個同一處做。金蓮描了一隻丟下,說道:「李大姐,你替我描這一隻。等我後邊把孟三姐叫了來。他昨日對我說,他也要做鞋哩!」一直走到後邊。玉樓房中倚着護炕兒,手中也衲着一隻鞋兒哩。金蓮進門,玉樓道:「你早辦?」金蓮道:「我起的早,打發他爹往門外與賀千戶送行去了。教我約下李大姐,花園裡趕早涼做些生活。等住回日頭過熱了做不的;我纔描了一隻鞋,教李大姐替我描着,逕來約你同去,咱三個一答兒哩好做。」因問:「你手裡衲的是甚麼鞋?」玉樓道:「是昨日你看我開的,那雙玄色段子鞋。」金蓮道:「你好漢,又早衲出一隻來了!」玉樓道:「那隻昨日就衲了,這一隻衲了好些了。」金蓮接過看了一回,說:「你這個到明日使甚麼雲頭子?」玉樓道:「我不得你們小後生,花花黎黎。我老人家了,使羊皮金緝的雲頭子罷。週圍拿紗綠線鎖,出白山子兒上,白綾高底穿,好不好?」金蓮道:「也罷。你快收拾咱去來!李瓶兒那裡等着哩!」玉樓道:「你坐着,咱吃了茶去。」金蓮道:「不吃罷,咱拿了茶那裡吃去來。」玉樓分付蘭香,頓下茶送去。兩個婦人手拉着手拉手兒,袖着鞋扇,逕往外走。吳月娘到上房穿廊下坐,便問:「你們那去?」金蓮道:「李大姐使我替他叫孟三兒去,與他描鞋。」說着,一直來到花園內。三人一處坐下,拿起鞋扇,你瞧我的,我瞧你的,都瞧了一遍。先是春梅拿茶來吃了,然後李瓶兒那邊的茶到。孟玉樓房裡蘭香落後,纔拿茶至,三了吃了。玉樓便道:「六姐,你平白又做平底鞋做甚麼?不如高底鞋好着。你若嫌木底子響腳,也似我用毡底子,卻不好?走着又不響。」金蓮道:「不是穿的鞋。是睡鞋。也是他爹因我了那隻睡鞋,被小奴才兒偷了,弄油了我的,吩咐教我從新又做這雙鞋。」玉樓道:「又說鞋哩,這個也不是舌頭。李大姐在在這裡聽着,昨日因你不見了這隻鞋,來昭家孩子小鐵棍兒,怎的花園裡拾了;後來不知你怎的知道了,對他爹說,打了小鐵棍兒一頓。說把他猴子打的鼻口流血,倘在地下,死了半日;惹的一丈青,好不在後邊海罵。罵那個淫婦,王八羔子學舌,打了他小廝。說他小廝一點尿不曉孩子,曉的甚麼?便唆調打了他恁一頓。早是活了,若死了,淫婦王八羔子也不得清潔!俺再不知罵淫婦、王八羔子是誰?落後小鐵棍兒進來,他大姐姐問他:『你爹為甚麼打你?』小廝纔說;『因在花園裡耍子,拾了一隻鞋,問姑夫換圈兒來。不知甚麼人對俺爹說了,教爹打我一頓。我如今尋姑夫,問他要圈兒去也。』說畢,一直往前跑了。原來罵的王八羔子是陳姐夫。早是只李嬌兒在傍邊坐着,大姐沒在根前。若聽見時,又是一場兒!」金蓮問:「大姐姐沒說甚麼?」玉樓道:「你還說哩!大姐姐好不說你哩!說:『如今這一家子亂世為王,九條尾狐狸精出世了。把昏君禍亂的貶子休妻,想着去了的來旺兒小廝,好好的從南邊來了,東一帳,西一帳,說他老婆養着主子,又說他怎的拿刀弄杖。成日做賊哩、養汗哩、生兒禍弄的,打發他出去了。把個媳婦又逼臨的吊死了!如今為一隻鞋子,又這等驚天動地反亂。你的鞋好好穿在腳上,怎的教小廝拾了?想必吃醉了,在那花園裡和漢子不知怎的餳成一塊,纔吊了鞋!如今沒的摭羞,拿小廝頂缸,打他這一頓,又不曾為甚麼大事!』」金蓮聽了道:「沒的那扯〈毛皮〉淡!甚麼是大事?殺了人是大事了,奴才拿才刀子要殺主子!」向玉樓道:「孟三姐,早是瞞不了你。咱兩個聽見來興兒說了一聲,諕的甚麼樣兒的。你是他的大老婆,倒說這個話!你也不管,我也不管,教奴才殺了漢子纔好?老婆成日在你那後邊使喚,你縱容着他,不管教他。欺大滅小,和這個合氣,和那個合氣。各人冤有頭,債有主。你揭條我,我揭條你,吊死了你還瞞着漢子不說!早時苦了錢,好人情說下來了!不然,怎了?你這的推乾淨,說面子話兒!右右是左右,我調唆漢子也罷。若不教他把奴才老婆漢子,不條提攆的離門離戶也不筭,恒屬人挾不到我井裡頭!」玉樓見金蓮粉面通紅惱了,又勸道:「六姐,你我姊妹都一個人。我聽見的話兒,有個不對你說?說了只放在你心裡,休要使出來!」金蓮不依他,到晚等的西門慶進入他房來,一五一十告西門慶說,來昭媳婦子一丈青怎的在後邊指罵,說你打了他孩子,要邏揸兒和人攘。這西門慶不聽便罷,聽了說在心裡。到次日,要攆來昭三口子出門。多虧月娘再三攔勸下,不容他在家,打發他往獅子街房子那看守,替了平安兒來家看守大門。後次,月娘知道,甚惱金蓮,不在話下。正是:

「事不三思終有悔,  人逢得意早回頭。」

卻說西門慶在前廳打發來昭三口子,搬移獅子街看守房屋去。一日正在前廳坐,忽有看守大門的平安兒來報:「守備府周爺差人送了一位相面先生,名喚吳神仙,在門首伺候見爹。」西門慶道來人進見。」遞上守備帖兒,然後道;「有請。」須臾,那吳神仙頭戴青布道巾,身穿布袍、草履,腰繫黃絲雙穗縧,手執龜殼扇子,自外飄然進來。年約四十之上,生的神清如長江皓月,貌古似太華喬松,威儀凜凜,道貌堂堂。原來神仙有四般古怪,身如松,聲如鐘,坐如弓,走如風。但見他:

「能通風鑑,善究子平。觀乾象能識陰陽,察龍經明知風水。五星深講,三命秘談。審格局,決一世之榮枯;觀氣色,定行年之休咎。若非華岳修真客,定是成都賣卜人。」

西門慶見神仙進來,忙降階迎接,接至廳上。神仙見西門慶,長揖稽首禮就坐。須臾,茶罷。西門慶:「動問神仙,高名雅號?仙鄉何處?因何與周大人相識?」那吳神仙坐上,欠身道:「貧道姓吳名奭,道號守真,本貫浙江仙遊人。自幼從師天台山紫虛觀出家,雲遊上國。因往岱宗訪道,道經貴處。周老總兵相約,看他老夫人目疾,特送來府上觀相。」西門慶道:「老仙長會那幾家陰陽?道幾家相法?」神仙道:「貧道粗知十三家子平,善曉麻衣相法,又曉六壬神課。常施藥救人,不愛世財,隨時住世。」西門慶聽言,益加敬重,誇道:「真乃謂之神仙也!」一面令左右放卓兒,擺齊管待神仙。神仙道:「周老總兵送貧道來,未曾觀相造,豈可先要賜齋!」西門慶笑道:「仙長遠來,已定未用早齋。待用過,看命未遲。」于是陪着神仙,吃了些齋食素饌,抬過卓席,拂抹乾淨,討筆硯來。神仙道:「請先觀貴造,然後觀相尊容。」,西門慶便說與八字:「屬虎的,二十九歲了,七月二十八日子時生。」這神暗暗掐指尋紋,良久,說道:「官人貴造丙寅年,辛酉月,壬午日,丙子時,七月廿三日白露,已交八月算命。月令提剛辛酉,理傷官格。子平云:傷官傷盡復生財,財旺生官福轉來,立命申宮!是城頭土命,七歲行運辛酉,十七行壬戍,二十七癸亥,三十七甲子,四十七乙丑。官人貴造,依貧道所講,元命貴旺,八字清奇,非貴則榮之造。但戊土傷官生在七八月,身忒旺了。幸得壬午日干,丑中有癸水。水火相濟,乃成大器。丙子時,丙合辛生,後來定掌威權之職;一生盛旺,快樂安然,發福遷官,主生貴子。為人一生耿直,幹事無二。喜則和氣春風,怒則迅雷烈火。一生多得妻財,不少紗帽戴。臨死有二子送老。今歲丁未流年,丁壬相合。目下丁火來剋。若你剋我者為官鬼,必主平地登雲之喜,添官進祿之榮。大運見行癸亥,戊土得癸水滋潤,定見發生。目下透出紅鸞天喜,熊羆之兆。又命宮馹馬臨申,不過七月必見矣。」西門慶問道:「我後來運限何如?有災沒有?」神仙道:「官人休怪我說,但八字不宜陰水太多,後到甲子運中,常在陰人之上;只是多了底流星打攪,又被了壬午日破了,不出六六之年,主有嘔血流膿之災,骨瘦形衰之病。」西門慶問道:「于今如何?」神仙道:「目今流年,只多日逢破敗,五鬼在家炒鬧,些小氣惱,不足為災,都被喜氣神臨門沖散了。」西門慶道:「命中還有敗否?」神仙道:「年趕着月,月趕着日,實難矣。」西門慶聽了,滿心歡喜。便道:「先生,你相我面何如?」請尊容轉正,貧道觀之。」西門慶把座兒掇了一掇。神仙相道:「夫相者,有心無相,相逐心生。有相無心,相隨心往。吾觀官人頭圓頂短,必為享福之大;體健觔強,決是英豪之輩; 天庭高聳,一生衣祿無虧;地閣方圓,晚歲榮華定取;此幾庄兒好處。還有幾庄不足之處,貧道不敢說。」西門慶道:「仙長但說無妨。」神仙道:「請官人走兩步看。」西門慶真個走了幾步。神仙道:「你行如擺柳,必主傷妻,魚尾多紋,定終須勞碌。眼不哭而淚汪汪,心無慮而眉縮縮,若無刑剋,必損其身;妻宮剋過方可。」西門慶道:「已刑過了。」神仙道:「請出手來看一看!」西門慶舒手來與神仙看,神仙道:「智慧生於皮毛,苦樂勸乎手足;細軟豐潤,必享福逸祿之人也。兩目雌雄,必主富而多詐;眉抽二尾,一生常自足歡娛;根有三紋,中年必然多耗散;奸門紅紫,一生廣得妻財;黃氣發於高曠,旬日內必定加官; 紅色起於三陽,今歲間必生貴子;又有一件不敢說,淚堂豐厚,亦主貪花,谷道亂毛號為淫抄;且喜得鼻乃財星,驗中年之造化,承漿地閣,管末世之榮枯。」

「承槳地閣要豐隆,  準乃財星居正中;

生平造化皆由命,  相法玄機定不容。」

神仙相畢,西門慶道:「請仙長相相房下眾人。」一面令小廝:「後邊請你大娘出來。」于是李嬌兒、孟玉樓、潘金蓮、李瓶兒、孫雪蛾等眾都跟出來,在軟屏後潛聽。神仙見月娘出來,連忙道了稽首,也不敢坐,在傍邊觀相:「請娘子尊容轉正。」那吳月娘把面容朝看廳外。神仙端詳了一回說:「娘子面如滿月,家道興隆;唇若紅蓮,衣食豐足。必得貴而生子;聲響神清,必益夫而發福。請出手來。」月娘從袖口中,露出十指春蔥來。神仙道:「乾姜之手,女人必善持家;照人之鬢,坤道定須秀氣。這幾椿好處。還有些不足之處,休道貧道直說。」西門慶道:「仙長但說無妨。」神仙道:「淚堂黑痣,若無宿疾必刑夫;眼下皺紋,亦主六親若冰炭。」

「女人端正好容儀,  緩步輕如出水龜;

行不動塵言有節。  無肩定作貴人妻。」

相畢,月娘退後。西門慶道:「還有小妾輩請看看。」于是李嬌兒過來。神仙觀看良久:「此位娘子,額尖鼻小,非側室必三嫁其夫;肉重身肥,廣有衣食,而榮華安享;肩聳聲泣,不賤則孤;鼻梁若低,非貧即夭。請步幾步我看。」李嬌兒走了幾步。神仙道:

「額尖露臀并蛇行,  早年必定落風塵;

假饒不是娼門女,  也是屏風後立人。」

相畢,李嬌兒下去。吳月娘叫:「孟三姐,你也過來相一相。」神仙觀着:「這位娘子,三停平等,一生衣祿無虧。六府豐隆,晚歲榮華定取;平生少疾,皆因月孛光輝;到老無災,大抵年宮潤秀。請娘子走兩步。」玉樓走了兩步。神仙道:

「口如四字神清徹,  溫厚堪同掌上珠;

威媚兼全財命有,  終主刑夫兩有餘。」

玉樓相畢,叫潘金蓮過來。那潘金蓮只顧嬉笑,不肯過來。月娘催之再三,方纔出見。神仙抬頭觀看這個婦人,沉吟半日,方纔說道:「此位娘子,髮濃鬢重,光斜視以多淫;臉媚眉彎,身不搖而自顫;面上黑痣,必主刑夫;人中短促,終須壽夭。」

「舉止輕浮惟好淫,  眼如點添壞人倫;

月下星前長不足,  雖居大廈少安心。」

相畢金蓮,西門慶又叫李瓶兒上來,教神仙相一相。神仙觀看這個女人:「皮膚香細,乃富室之女娘;容貌端莊,乃素門之德婦。只是多了眼光如醉,主桑中之約;眉靨漸生,月下之期難定。觀臥蠶明潤而紫色,必產貴兒。體白肩圓,必受夫之寵愛。常遭疾厄,只因根上昏沉;頻過喜祥,盖謂福星明潤。此幾椿好處。還有幾椿不足處,娘子可當戒之;山根青黑,三九前後定見哭聲。法令細繵,雞犬之年焉可過。慎之,慎之!」

「花月儀容惜羽翰,  平生良友鳳和鸞;

綠門財祿堪依倚,  莫把凡禽一樣看。」

相畢,李瓶兒下去,月娘令孫雪蛾出來相一相。神仙看了,說道:「這位娘子,體矮聲高,額尖鼻小,雖然出谷遷喬,但一生冷笑無情,作事機深內重。只是吃了這四反的虧,後來必主凶亡。夫四反者,唇反無稜,耳反無輪,眼反無神,鼻反不正故也。」

「燕體蜂腰是賤人,  眼如流水不廉真;

常時斜倚門兒立,  不為婢妾必風塵。」

雪蛾下去,月娘教大姐上來相一相。神仙道:「這位女娘鼻梁仰露,破祖刑家。聲若破鑼,家私消散;面皮太急,雖溝洫長而壽亦夭;行如雀躍,處家室而衣食缺乏;不過三九,常受折麼。」

「惟夫反目性通靈,  父母衣食僅養身;

狀貌有拘難顯達,  不遭惡死也艱辛。」

大姐相畢,教春梅也上來,教神仙相相。神仙睜眼兒,目了春梅,年約不上二九,頭戴銀絲雲髻兒,白線挑衫兒,桃紅裙子,藍紗比甲兒,纏手縛腳出來,道了萬福。神仙觀看良久,相道:「此位小姐,五官端正,骨格清奇,髮細眉濃,稟性要強;神急眼圓,為人急燥。山根不斷,必得貴夫而生子。兩額朝拱,位早年必戴珠冠;行步若飛仙,聲響神清,必益夫而得祿。三九定然封贈。但乞了這左眼大,早年剋父;右眼小,周歲剋娘;左口角下只一點黑痣,主常沾啾唧之灾;右腮一點黑痣,一生受夫愛敬。」

「天庭端正五官平,  口若塗硃行步輕;

倉庫豐盈財祿厚,  一生常得貴人憐。」

神仙相畢,眾婦女皆咬指以為神相。西門慶封白銀五兩與神仙,又賞守備府來人銀五錢,拿拜帖回謝。吳神仙再三辭卻,說道:「貧道雲遊四方,風餐露宿,化救萬道,周總兵送將過來,可一時之情耳!要這財何用?決不敢受!」西門慶不得已,拿出一疋大布:「送仙長做一件大衣,何如?」神仙方纔受之。令小童接了,收在經包內,稽首拜謝。西門慶送出大門,揚長飄然而去。正是:

「柱杖兩頭挑日月,  葫蘆一個隱山川。」

西門慶送神仙出,回到後廳問月娘眾人:「所相何如?」月娘道:「相的也都好。只是三個人相不着!西門慶道:「那三個人相不著?」月娘道:「相李大姐有實疾,到明日生貴子。他見將今懷着身孕,這個也罷了。相咱家大姐明日受折磨,不知怎的折磨?相春梅後日來也生貴子,或者只怕你用了他,各人子孫也看不見。我只不信說他春梅後來戴珠冠,有夫人之分。端的咱家又沒官,那討珠冠來?就有珠冠,也輪不到他頭上!」西門慶笑道:「他相我目下有平地登雲之喜,加官進祿之榮;我那得官來?他見春梅和你每站在一處,又打扮不同,戴着銀絲雲髻兒,只當是你我親生養女兒一般,或後來匹配名門,招個貴婿,故說有些珠冠之分。自古筭的着命,筭不着好。相逐心生,相隨心滅。周大人送來,咱不好囂了他的頭,教他相相除疑罷了。」說畢,月娘房中擺下飯,打發吃了飯。西門慶手拿芭蕉扇兒,信步閑遊,來花園大捲棚內聚景堂內,週圍放下簾櫳,四下花木掩映。正值日當午時分,只聞綠陰深處,一派蟬聲;忽然風送花香,襲人撲鼻。有詩為證:

「綠樹陰濃夏日長,  樓臺倒影入池塘,

水晶簾動微風起,  一架墻薇滿院香。

別院深沉夏草青,  石榴開遍透簾明,

槐陰滿地日卓午,  時聽新蟬噪一聲。」

西門慶坐於椅上,以手扇搖涼。只見來安兒、畫童兒兩個小廝,來井上打水。西門慶道:「叫一個來拿澆冰安放盆內。」來安兒忙走向前。西門慶吩咐道:「到後邊對你春梅姐說,有梅湯提一壺來,放在這冰盤內湃着。」來安兒應諾去了。半日,只見春梅家常露着頭,戴着銀絲雲髻兒,穿着毛青布褂兒,桃紅夏布裙子,手提一壺蜜煎梅湯,笑嘻嘻走來問道:「你吃了飯了?」西門慶道:「我在後邊上房裡吃了。」春梅:「嗔道不進房裡來。把這梅湯放在冰內湃着你吃。」西門慶點頭兒。春梅湃上梅湯,走來扶着椅兒,取過西門慶手中芭蕉扇兒,替他打扇,問道:「頭裡大娘和你說甚麼話來?」西門慶道:「說吳神仙相面一節。」春梅道:「那道士平白說戴珠冠。教大娘說:『有珠冠只怕輪不到他頭上。』常言道:『凡人不可貌相,海水不可斗量。』從來旋的不圓砍的圓,各人裙帶上衣食,怎麼料得定?莫不長遠只在你做奴才罷!」西門慶笑道:「小油嘴兒,自胡亂!你若到明日有了娃兒,就替你上了頭。」于是把他摟到懷裡,手扯着手兒頑耍。問他:「你娘在後邊?在屋裡?怎的不見?」春梅道:「娘在屋裡,教秋菊熱下水要洗浴。等不的,就在床上睡了。」西門慶道:「等我吃了梅湯,等我摑混他一混去。」于是春梅向冰盆倒了一甌兒梅湯,與西門慶呷了一口,湃骨之涼,透心沁齒,如甘露洒心一般。須臾,吃畢,搭伏着春梅肩膀兒,轉過角門,來到金蓮床房中。掀開簾櫳進來,看見婦人睡在正面一張新買的螺鈿牀上。原是因李瓶兒房中,安着一張螺鈿厰廳床,婦人旋教西門慶使了六十兩銀子,也替他也買了這一張螺鈿有欄杆的床。兩邊槅扇,都是螺鈿攢造。安在床內,樓臺殿閣,花草翎毛。裡面三塊梳背,都是松竹梅,歲寒三友。掛着紫紗帳幔,錦帶銀鉤。兩邊香毬吊掛。婦人赤露玉體,止着紅綃抹胸兒,蓋着紅紗衾,枕石鴛鴦枕,在涼席之上,睡思正濃。房裡異香噴鼻,西門慶一見不覺淫心頓起,令春梅帶上門出去。悄悄脫了衣褲,上的床來,掀開紗被。見他玉體互相掩映,戲將兩股輕開,按塵柄徐徐插入牝中,比及星眸驚欠之際,已抽拽數十度矣!婦人睜開眼笑道:「怪強盜!三不知多咱進來!奴睡着了,就不知道!奴睡的甜甜兒,摑混死了我!」西門慶道:「我便罷了。若是有個漢子進來,你也推不知道!」婦人道:「我不好罵的!誰人七個頭八個膽,敢進我這房裡來?只許了你恁沒大沒小的罷了。」原來婦人因前日西門慶在翡翠軒誇獎李瓶兒身上白淨,就暗暗將茉莉花蕊兒,攪酥油定粉,把身上都搽遍了。搽的白膩光滑,異香可掬。使西門慶見了愛他,以奪其寵。西門慶於是見他身體雪白,穿着新做的兩隻大紅睡鞋。一面蹲踞在上,兩手兜其股,極力而提之,垂首觀其出入之勢。婦人道:「怪貨!只顧端詳甚麼?奴的身上黑,不似李瓶兒的身上白就是了。他懷着孩子,你便輕憐痛惜;俺每是拾兒,由着這等掇弄!」西門慶問道說:「你等着我洗澡來?婦人問道:「你怎得知道來?」西門慶把春梅告訴他話,說了一遍。婦人道:「你洗,我教春梅掇水來。」不一時,把浴盆掇到房中,注了湯,二人下床來,同浴蘭湯,共效魚水之歡。當下添湯換水,洗浴了一回。西門慶乘興把婦人仰臥在浴板之上,兩手執其雙足,跨而提之,揪騰〈扌扉〉幹,何止二三百回;其聲如泥中螃蟹一般,響之不絕。婦人恐怕香雲拖墜,一手扶着雲鬢,一手扳着盆沿,口中燕語鶯聲,百般難述,怎見這場交戰,但見:

「華池蕩漾波紋亂,翠幃高捲秋雲暗;才郎情動要爭持,稔色心忙顯手段。一個顫顫巍巍挺硬鎗,一個搖搖擺擺輪鋼劍。一個捨死忘生往裡鑽,一個尤雲殢雨將功幹。撲撲鼕鼕皮鼓催,蹕蹕礡礡鎗付劍;〈石八〉〈石八〉蹋蹋弄響聲,砰砰〈石拜〉〈石拜〉成一片。下下高高水逆流,洶洶湧湧盈清澗;滑滑搊搊怎住停,攔攔濟濟難存站。一來一往□□□,一衝一撞東西探,熱氣騰騰妖雲生,紛紛馥馥香氣散。一個逆水撑船將玉股搖,一個稍公把舵將金蓮揝;一個紫騮猖獗逞威風,一個白面妖嬈遭馬戰。喜喜歡歡美女情;雄雄糾糾男兒愿;翻翻覆覆意歡娛,鬧鬧挨挨情摸亂。你死我活更無休,千戰千贏心膽戰;口口聲聲叫殺人,氣氣昂昂情不厭。古古今今廣鬧爭,不似這番水裡戰。」

當下二人水中戰鬧了一回,西門慶精泄而止。搽抹身體乾淨,撒去浴盆。止着薄纊短襦,上床安放炕卓,菓酌飲酒,教秋菊:「取白酒 來與你爹吃。」又向床閣板上方盒中拿菓餡餅與西門慶吃,恐怕他肚中飢餓。只見秋菊半日拿上一銀注子酒來,婦人纔待斟在鍾上,摸了摸,冰涼的;就照着秋菊臉上只一潑,潑了一頭一臉,罵道:「好賊少死的奴才,我吩咐教你篩了來,如何拿冷酒與爹吃?你不知安排些甚麼心兒!」叫春梅:「與我把這奴才採到院子裡跪着去!」春梅道:「我替娘後邊捲裹腳去來,一些兒沒在根前,你就弄下硶兒了!」那秋菊把嘴谷都着,口裡喃喃吶吶說道:「每日爹娘還吃冰湃的酒兒,誰知今日又改了腔兒!」婦人聽見,罵道:「好賊奴才!你說甚麼?與我採過來!」教春梅:「每邊臉上,打與他十個嘴巴!」春梅道:「皮臉沒的。打污濁了我手,娘只教他頂着石頭跪着罷。」于是不由分說,拉到院子內,教他頂着塊大石頭跪着。不在話下。婦人從新教春梅煖了酒來,陪西門慶吃了幾鍾。掇去酒卓,放下紗帳子來,吩咐拽上房門,兩個抱頭交股體倦而寢,正是:

「若非群玉山頭覓,  多是陽臺夢裡尋。」

畢竟未知後來何如,且聽下回分解:

