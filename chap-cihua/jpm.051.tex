%# -*- coding: utf-8 -*-
%!TEX encoding = UTF-8 Unicode
%!TEX TS-program = xelatex
% vim:ts=4:sw=4
%
% 以上设定默认使用 XeLaTex 编译,并指定 Unicode 编码,供 TeXShop 自动识别

%第五十一回 
\chapter{月娘聽演金剛科\KG 桂姐躲在西門宅}

「羞看鸞鏡惜朱顏,  手托香腮懶去眠,

瘦損纖腰寬翠帶,  淚流粉面落金鈿;

薄〈亻辛〉惱人愁切切,  芳心撩亂恨綿綿,

何時借得來風便,  刮得檀郎到枕邊。」

話說潘金蓮見西門慶拏了淫器包兒在李瓶兒房里歇了,足惱了一夜沒睡,懷恨在心。到第二日,打聽西門慶往衙門裡去了,李瓶兒在屋裡梳頭,老早走到後邊,對月娘說:「李瓶兒背地好不說姐姐哩。說姐姐會那等,虔婆勢,喬作衙,別人生日喬作家管。你漢子吃醉了,進我屋裡來,我又不曾在前邊,平白對著人羞我望著我丟臉兒。交我惱了,走到前邊把他爹趕到後邊來。落後他怎的也不在後邊,還往我房裡來了?他兩個黑夜說了一夜梯已話兒。只有心腸五臟,沒曾倒與我罷了。」這月娘聽了,如何不惱!因向大妗子、孟玉樓說:「果是你昨日也在根前看,我又沒曾說他甚麼!小廝交燈籠進來,我只問了一聲:『你爹怎的不進來?』小廝倒說往六娘屋裡去了。我便說:『你二娘這裡等著,恁沒槽道,卻不進來。』論起來也不傷他,怎的說我虔婆勢,喬作衙?我是淫婦老婆?我還把他當好人看成,原來知人知面不知心,那裡看人去!乾淨是個綿裏針,肉裡刺的貨!還不知背地在漢子根前,架的甚麼舌兒哩?怪道他昨日決烈的就往前走了。俊姐姐,那怕漢子成日在你那屋裡不出門,不想我這心動一動兒。一個漢子丟與你們,隨你們去,守寡的不過!想著一娶來之時,賊強人和我門裡門外不相逢,那等怎麼過來。」大妗子在傍勸道:「姑娘罷麼,那看著孩兒的分上罷。自古宰相肚裡好行船,當家人是個惡水缸兒,好的也放在你心裡,歹的也放在心裡。」月娘道:「不拘幾時,我也要對這兩句話,等我問著他。我怎麼虔婆勢?喬作衙?」金蓮慌的沒口子說道:「姐姐寬恕他罷!常言大人不責小人過。那個小人沒罪過?他在屋裡背地調唆漢子,俺每這幾個,誰沒吃他排說過?我和他緊隔著壁兒,要與他一般見識起來,倒了不成,行動只倚逞著孩子降人!他還說的好話兒哩,說他的孩兒到明日長大了,有恩報恩,有仇報仇,俺們都是餓死的數兒,你還不知道哩!」吳大妗子道:「我的奶奶,那裡有此話說!」月娘一聲兒也沒言語。常言:路見不平,也有向燈向火。不想西門大姐平日與李瓶兒最好,常沒針線鞋面,李瓶兒不拘好綾羅段帛,就與之。好汗巾手帕兩三方,背地與大姐;銀錢是不消說。當日聽了此話,如何不告訴他?李瓶兒正在屋裡,與孩子做那端午戴的那絨線符牌兒,及各色紗小粽子兒,并解毒艾虎兒,只見大姐走來。李瓶兒讓他坐,同看做生活。李瓶兒交迎春拏茶與你大姑娘吃,一面吃了茶。,大姐道:「頭裡請你吃茶,你怎的不來?」李瓶兒道:「打發他爹出門,我趕早涼兒,與孩子做這戴的碎生活兒來。」大姐道:「有樁事兒,我也不是舌頭,敢來告你說。學說你說俺娘虔婆勢,你沒曾惱著五娘?他在後邊對著俺娘如此這般,說了你一篇是非。如今俺娘要和你對話哩!你別要說我對你說,交他怪我,你須預備些話兒,打發他。」這李瓶兒不聽便罷,聽了此言,手中拏著那針兒,通拏不起來,兩隻胳膊都軟了,半日說不出話來。對著大姐吊眼淚,說道:「大姑娘,我那裡有一字兒閒話!昨晚我在後邊,聽見小廝說他爹往我這邊來了,我就來到前邊催他往後邊去了,再誰說一句話兒來?你娘恁覷我一場,莫不我恁,不識好歹,敢說這個話!設使我就說,對著誰說來?也有個下落。」大姐道:「他聽見俺娘說,不拘幾時要對這話,他如何就慌了?要著我,你兩個當面鑼,對面鼓的對,不是?」李瓶兒道:「我對的過他那嘴頭子?自憑天罷了!他左右晝夜算計的我。只是俺娘兒兩個,到明日科裡吃他算計了一個去,也是了當!」說畢哭了。大姐坐著,勸了一回。只見小玉來請六娘,大姑娘吃飯,就後邊去了。李瓶兒丟下針指,同大姐到後邊,也不曾吃飯,回來房中,倒在床上,就睡著了。西門慶衙門中來家,見他睡,問迎春,迎春道:「俺娘一日飯也還沒吃哩!」慌了西門慶向前問道:「你怎的不吃飯?你對我說。」又見他哭的眼紅紅的,只顧問:「你心裡怎麼的?對我說。」那李瓶兒連忙起來,揉了揉眼,說道:「我害眼疼,不怎的。今日心裡懶待吃飯。」並不題出一字兒來。正是:

「滿懷心腹事,  盡在不言中。」

有詩為證:

「莫道佳人總是痴,  惺惺伶俐沒便宜;

只因會盡人間事,  惹得閒愁滿肚皮!」

大姐在後邊對月娘說:「我問他來,他說沒有此話。我對著誰說來?且是好不賭身罰咒,望著我哭哩。說娘這般看顧他,他肯說此話?」吳大妗子道:「我就不信,李大姐好個人兒,他原肯說這等謊?」月娘道:「想必兩個不知怎的有些小節不足,哄不動漢子,走來後邊戳無路兒,沒的拏我墊舌根。我這裡還多著個影兒哩!」大妗子道:「大姑娘,今後你也別要虧了人。不是我背他說,潘五姐一百個不及他為人;心地兒又好,來了咱家恁二三年,要一些歪樣兒也沒有。」正說著,只見琴童兒藍布大包袱背進來。月娘問:「是甚麼?」琴童道:「是三萬鹽引。韓夥計和崔本纔從關上掛了號來,爹說打發飯與他二人吃。如今兌銀子打包,後日二十一日好日子起身,打發他三個往楊州去。」吳大妗子道:「只怕姐夫進來,我和二位師父往他二娘房裡坐去罷。」剛說未畢,只見西門慶掀簾子進來,慌的吳妗子和薛姑子、王姑子,往李嬌兒屋裡走不迭。早被西門慶看見,問月娘:「那個薛姑子,賊胖禿淫婦,來我這裡做什麼?」月娘道:「你好恁枉口拔舌,不當家化化的,罵他怎的!他惹著你來?你怎的知道他姓薛?」西門慶道:「你還不知他弄的乾坤兒哩!他把陳參政家小姐,七月十五日,吊在地藏菴兒裡,和一個小夥阮三偷奸。不想那阮三就死在女子身上,他知情受了三兩銀子。事發拏到衙門裡,被我褪衣打了二十板,交他嫁漢子還俗。他怎的還不還俗?好不好拏到衙門裡,再與他幾拶子!」月娘道:「你有要沒緊,恁毀神謗佛的?他一個佛家弟子,想必善根還在。他平白還甚麼俗?你還不知,他好不有德行!」西門慶道:「你問他有道行,一夜接幾個漢子?」月娘道:「你就休汗邪,又討我那沒好口的罵你!」因問:「幾時打發他三個起身?」西門慶道:「我剛纔使來保會喬親家去了。他那裡出五百兩,我這裡出五百兩。二十是個好日子,打發他每起身去罷了。」月娘道:「線舖子卻交誰開?」西門慶道:「且交賁四替他開著罷。」說畢,月娘開箱子拏出銀子,一面兌了出來,交付與三人。正在捲棚內看著打包,每人兌與他五兩銀子,交他家中收拾衣裝行李,不在話下。只見應伯爵走到捲棚里,見西門慶看著打包,便問:「哥打包做甚麼?」西門慶因把二十日打發來保等往楊州支鹽去一節,告訴一遍。伯爵舉手道:「哥恭喜!此去回來,必有大利息。」西門慶一面讓他坐,喚茶來吃了。因問:「李三、黃四銀子幾時關?」應伯爵道:「也只不出這個月裡,就關出來了。他昨日對我說,如今東平府又派下二萬香來了,還要問你挪五百兩銀子,接濟他這一時之急。如今關出這批的銀子,一分也不動,都抬過這邊來。」西門慶道:「到是你看見我這裡,打發楊州去,還沒銀子,問喬親家那裡,借了五百兩在裡頭。那討銀子來?」伯爵道:「他再三央及將我對你說,一客不煩二主。你不接濟他這一步兒,交他又問那裡借去?」那西門慶道:「門外街東徐四舖少我銀子,我那裡挪五百兩銀子與他罷。」伯爵道:「可知好哩?」正說著,只見平安兒拏進帖兒來,說:「夏老爹家差了夏壽,道請爹明日坐坐。」西門慶展開柬帖云云。伯爵道:「我今敢來有樁事兒來報與哥。你知道院裡李桂兒勾當?他沒來?」西門慶道:「他從正月去了,再幾時來?我並不知道甚麼勾當!」伯爵因說起:「王招宣府裡第三的,原來是東京六黃太尉姪女兒女婿,從正月往東京拜年,老公公賞了一千兩銀子與他兩口兒過節。你還不知六黃太尉這姪女兒,生的怎麼標緻,上畫兒委的只畫半邊兒,也有恁俊俏相的!你只守著你家裡的罷了。每日被老孫、祝麻子、小張閑;三四個摽著在院裡撞,把二條菴齊家那小丫頭子齊香兒梳籠了,又在李桂兒家走。把他娘子兒的頭面都拏出來當了,氣的他娘子兒家裡上吊。不想前日這月裏老公公生日,他娘子兒到東京,只一說,老公公惱了,將這幾個人的名字送與朱太尉。朱太尉批行東平府,著落本縣拏人。昨日把老孫、祝麻子與小張閑,都從李桂兒家拏的去了。李桂兒便躲在隔壁朱毛頭家過了一夜。今日說來你這裡,央及你來了。」西門慶道:「我說正月裡都摽著他走,這裡誆人家銀子,那裡誆人家銀子,那祝麻子還對著我搗生鬼!」說畢,伯爵道:「我去罷,等住回,只怕李桂兒來,你管他不管他?他又說我來串作你。」西門慶道:「你且坐著,我還和你說哩。李三你且別要許他,等我門外討銀子出來,和你說話去。」伯爵道:「我曉的。」剛走出大門首,只見李桂姐轎子在門首,又早下轎進去了。西門慶正分付陳經濟,交他騎騾子往門外徐四家催銀子去。只見琴童兒走到捲棚內,請西門慶道:「大娘後邊請,有李桂姨來了。」這西門慶走到後邊,只見李桂姐身穿茶色衣裳,也不搽臉,用白挑線汗子搭著頭,雲鬟不整,花容淹淡,與西門慶磕著頭,哭起來說道:「爹可怎麼樣兒的?恁造化低的營生!正是關著門兒家裡坐,禍從天上來!一個王三官兒,俺每又不認的他,平日的祝麻子、孫寡嘴領了來俺家來討茶吃。俺姐姐又不在家,依著我說別要招惹他那些兒不是?俺這媽越發老的韶刀了。就是來宅裡與俺姑娘做生日的這一日,你上轎來了就是了,見祝麻子打旋磨兒跟著,從新又回去。對我說,姐姐,你不出去待他鍾茶兒,卻不難為囂了人了。他便生爺這裡來了,交我把門插了不出來。誰想從外邊撞了一夥人來,把他三個,不由分說都拏的去了。王三官兒便奪門走了,我便走在隔壁人家躲了,家裡有個人牙兒?纔使保兒來這裡接的你家去。到家把媽諕的魂兒也沒了,只要尋死。今日縣裡皂隸,又拏著票喝囉了一清早起去了。如今坐名兒,只要我往東京回話去。爹,你老人家不可怜見救救兒,卻怎麼樣兒的?娘在傍邊也替我說說兒。」西門慶笑道:「你起來。」因問:「票上還有誰的名字?」桂姐道:「還有齊香兒的名字,他梳籠了齊香兒,在他家使錢著,便該當。俺家若見了他一個錢兒,就把眼睛珠子吊了!若是沾他沾身子兒,一個毛孔兒裡生一個天疱瘡!」月娘對西門慶道:「也罷,省的他恁說誓剌剌的,你替他說說罷。」西門慶道:「如今齊香兒拏了不曾?」桂姐道:「齊香兒他在王皇親宅裡躲著里。」西門慶道:「既是恁的,你且在我這裡住兩日。倘人來尋你,我就差人往縣裡替你說去。」于是就叫書童兒:「你快寫個帖兒,往縣裡見你李老爹,就說桂姐常在我這裡答應,看怎的免提他罷。」書童應諾,穿青絹衣服去了。不一時,拏了李知縣回帖兒來。書童道:「李老爹多上覆你老爹,別的事無不領命,這個卻是東京上司行下來批文,委本縣拏人;縣裡只拘的人在。既是你老爹分上,我這裡且寬限他兩日。要免提,還往東京上司處說去。」西門慶聽了,只顧沉吟,說道:「如今來保一兩日起身,東京沒人去。」月娘道:「也罷,你打發他兩個先去,存下來保,替桂姐往東京說了這勾當,交他隨後邊趕了去,也是不遲。你看諕的他那腔兒!」那桂姐連忙與月娘和西門慶磕頭。西門慶隨使人叫將來保來,分付:「二十日你且不去罷,交他兩個先去。你明日且往東京替桂姐說這勾當來,見你翟爹,如此這般,好歹差人往衛里說說。」桂姐連忙就與來保下禮。慌的來保頂頭相退,說道:「桂姨,我就去。」西門慶一面交書童兒寫就一封書,致謝翟管家:「前日曾巡按之事,其是費心。」又封了二十兩折節禮銀子,連書交與來保。桂姐便歡喜了。拏出五兩銀子來,與來保路上做盤纏,說道:「相來俺媽還重謝保哥。」西門慶不肯,還交桂姐收了銀子。交月娘另拏五兩銀子與來保盤纏。桂姐道:「也沒這個道理!我央及爹這裡說人情,又交爹出盤纏。」西門慶道:「你笑譁我沒這五兩銀子盤纏了,要你的銀子?」那桂姐方纔收了。向來保拜了又拜,說道:「累保哥,明日好歹起身罷,只怕遲了。」來保道:「我明日早五更就走道兒了。」于是領了書信,又走到獅子街韓道國家。王六兒正在屋裡,替他縫小衣兒哩,打窗眼看見是來保,忙道:「你有甚說話?請房裡坐。他不在家,往裁縫那裡討衣裳去了,便來也。」便叫錦兒:「還不往對過徐裁家叫你爹去?你說保大爺在這裡。」來保道:「我敢來說聲,我明日且去不成,又有樁業障鑽出來。當家的留下,交我往東京替院裡李桂姐說人情去哩。他剛纔在爹根前再三磕頭禮拜,央及我。娘和爹說:『也罷,你且替他往東京走一遭,說說這勾當。且交韓夥計和崔大官兒先去。你回來再趕了去,也是不遲。』我明日早起身了,剛纔書也有了。」因問:「嫂子,你做的是甚麼?」王六兒道:「是他的小衣裳兒。」來保道:「你交他少帶衣裳。到那去處,是出紗羅段絹的窩兒裡,愁沒衣裳穿?」正說著,韓道國來了,兩個唱了喏,因把前事說了一遍。因說:「我到明日,楊州那裡尋你們?」韓道國道:「老爹分付,交俺每馬頭上投經紀王伯儒店裡下。說過世老爹,曾和他父親相交,他店內房屋寬廣,下的客商多,放財物不耽心。你只往那裡尋俺每就是了。」又說:「嫂子,我明日東京去,你沒甚鞋腳東西稍進府裡,與你大姐去?」王六兒道:「沒甚麼,只有他爹替他打的兩對簪兒,并他兩雙鞋,起動保叔稍稍進去與他。」于是用手帕包縫停當,遞與來保。一面交春香看菜兒篩酒,婦人連忙丟下生活,就放卓兒。來保道:「嫂子,你休費心,我不坐。我到家還收拾了褡褳,明日好起身。」王六兒笑嘻嘻道:「耶嚛!你怎的上門怪人家!夥計家,自恁與你餞行,也該吃鍾兒。」因說韓道國:「你好老實,卓兒不穩,你也撒撒兒讓保叔坐,只相沒事的人兒一般兒!」于是拿上菜兒來,斟酒遞來保。王六兒也陪在傍邊。三人坐定吃酒。來保吃了幾鍾,說道:「我家去罷。晚了,只怕家裡關門早。」韓道國問道:「你頭口顧不了不曾?」來保道:「明日早顧罷了。說:「舖子里鑰匙并帳簿,都交與賁四罷了,省的你又上宿去。家裡歇息歇息,好走路兒。」韓道國道:「夥計說的是。我明日就交與他。」王六兒又斟了一甌子,說道:「保叔,你只吃這一鍾,我也不敢留你了。」來保道:「嫂子,你既要我吃,再篩熱著些。」那王六兒連忙歸到壺裏,交錦兒炮熱了,傾在盞內,雙手遞與來保,說道:「沒甚好菜兒與保叔下酒。」來保道:「嫂子好說,家無常禮。」拏起酒來,與婦人對飲。一吸而同乾,方纔作辭起身。王六兒便把女兒鞋腳遞與他,說道:「累保叔好歹到府裡問聲孩子好不好,我放心些。」于是道了萬福,兩口兒齊送出門來。不說來保到家收拾行李,第二日起身東京去了,不題。單表月娘上房擺茶與桂姐吃。吳大妗子、楊姑娘,兩個姑子,都做一處坐。有吳大舅前來對西門慶說:「有東平府行下文書來,派俺本衛兩所掌印千戶管工脩理社倉,題准旨意,限六月工完,陞一級;違限,聽巡按御史查參。姐夫有銀子,借得幾兩工上使用。待關出工價來,一一奉還。」西門慶道:「大舅用多少,只顧拏去。」吳大舅道:「姐夫下顧,與二十兩罷。」一面進入後邊,見了月娘說了話,交月娘拏二十兩出來交與大舅,又吃了茶出來。因後邊有堂客,不好坐的,交西門慶留大舅大廳上吃酒。正飲酒中間,只見陳經濟走來回話,說:「門外徐四家銀子,頂上爹,再讓兩日兒。」西門慶道:「胡說,我這裡用銀子使,再讓兩日兒!照舊還去罵那狗第子孩兒?」經濟應諾。吳大舅讓姐夫坐的,陳經濟作了揖,打橫坐了。琴童兒連忙安放了鍾筯,這里前邊吃酒。且說後邊大妗子、楊姑娘、李嬌兒、孟玉樓、潘金蓮、李瓶兒、大姐,都伴桂姐在月娘房裡吃酒。先是郁大姐數了回張生遊寶塔,放下琵琶。孟玉樓在傍斟酒,哺菜兒與他吃,說道:「賊瞎賤磨的!唱了這一日,又說我不疼你。」那潘金蓮又大筯子夾腿肉,放在他鼻子上,戲弄他頑耍。桂姐因叫:「玉簫姐,你遞過那郁大姐琵琶來,我唱個曲兒與姑奶奶和大妗子聽。」月娘道:「桂姐,你心裡熱剌剌的,不唱罷。」桂姐道:「不妨事,等我唱。見爹娘替我說人情去了,我這回不焦了。」孟玉樓笑道:「李桂姐,倒還是院中人家娃娃,做臉兒快,頭里一來時,把眉頭忔縐著,焦的茶兒也吃不下去。這回說也有,笑也有。」當下桂姐輕舒玉指,頃撥冰弦,唱了一回。正唱著,只見琴童兒收進家活來。月娘便問道:「你大舅去了?」琴童兒道:「大舅去了。」吳大妗道:「只怕姐夫進來,俺每活變活變兒。」琴童道:「爹不往後邊來,往五娘房裡去了。」這潘金蓮聽見往他屋裡去了,就坐不住;趨趄著腳兒只要走,又不好走的。月娘也不等他動身,說道:「他往你屋裡去了,你去罷,省的你欠肚兒親家是的!」那潘金蓮嚷:「可可兒的走來!」口兒的硬著,那腳步兒且是去的快。來到前邊入房來,西門慶已是吃了胡僧藥,交春梅脫了衣裳,在床上帳子裡坐著哩。金蓮看見笑道:「我的兒,今日好呀!不等你娘來就上床了。俺每剛纔在後邊,陪大妗子、楊姑娘吃酒,被李桂姐唱著,灌了我幾鍾好的!獨自一個兒,黑影子裡,一步高,一步低,不知怎的就走的來了。」叫春梅:「你有茶,倒甌子我吃。」那春梅真個點了茶來。金蓮吃了,撇了個嘴與春梅,那時春梅就知其意,那邊屋早早已替他熱下水。婦人抖些檀香白礬在裏面,洗了牝。向燈下摘了頭,止撇著一根金簪子。拏過鏡子來,從新把嘴唇抹了些胭脂,口中噙著香茶 ,走過這邊來。春梅床頭上取過睡鞋來,與他換了,帶上房門出來。這婦人便將燈臺挪近床邊桌上放著,一手放下半邊紗帳子來。褪去紅褌,露見玉體。西門慶坐在枕頭上,那話帶著兩個托子,一位弄的大大的,露出來與他瞧。婦人燈下看見,諕了一跳,一手揝不過來,紫巍巍,沉甸甸,約有虎二。便眤瞅了西門慶一眼,說道:「我猜你沒別的話,已定吃了那和尚藥,弄聳的恁般大,一位要來奈何老娘。好酒好肉,王里長吃的去。你在誰人根前試了新,這回剩了些殘軍敗將,纔來我屋這裡來了?俺每是雌剩{髟巳}{髟八}{入日}的,你還說不偏心哩!嗔道那一日我不在屋裏,三不知把那行貨包子偷的往他屋裡去了。原來晚夕和他幹這個營生,他還對著人撇清搗鬼哩!你這行貨子,乾淨是個沒挽和的三寸貨。想起來,一百年不理你纔好!」西門慶笑道:「小淫婦兒!你過來。你若有本事把他咂過了,我輸一兩銀子與你。」婦人道:「汗邪了你了,你吃了甚麼行貨子,我禁的過他!」于是把身子斜躺在衽席之上,雙手執定那話,用朱唇吞裹,說道:「好大行貨子,把人的口也撐的生疼的。」說畢,出入嗚咂,或舌尖挑弄蛙口,舐其龜弦,或用口噙著,往來哺摔,或在粉臉上偎幌,百般搏弄,那話越發堅硬,〈扌造〉崛起來,裂瓜頭,凹眼圓睜,落腮鬍,挺身直豎。西門慶垂首窺見婦人香肌,掩映于紗帳之內,纖手捧定毛都魯那話,往口裡吞放。燈下一往一來動彈。不想傍邊蹲踞著一個白獅子貓兒,看見動彈,不知當做甚物件兒,撲向前用爪兒來撾,這西門慶在上,又將手中拏的洒金老鴉扇兒,只顧引鬬他耍子。被婦人奪過扇子來,把貓儘力打了一扇把子,打出帳子外去了。眤向西門慶道:「怪發訕的冤家緊著這扎扎的不得人意,又引鬬他恁上頭上臉的。一時間撾了人臉,卻怎樣的?好不好我就不幹這營生了。」西門慶道:「怪小淫婦兒!會張致死了?」婦人道:「你怎的不交李瓶兒替你咂來?我這屋裡儘著交你掇弄,不知吃了甚麼行貨子,咂了這一日,亦發咂了沒事沒事。」西門慶于是向汗巾兒上,小銀盒兒裏,用挑牙挑了些粉紅膏子藥兒,抹在馬口內。仰臥于上,交婦人騎在身上,婦人道:「等我〈扌扉〉著,你往裡放。龜頭昂大,濡研半晌,僅沒龜稜,婦人在上,將身左右捱擦,似有不勝隱忍之態。因叫道:「親達達,裏邊緊澀住了,好不難捱。」一面用手摸之,燈下窺見塵柄,已被牝戶吞進半截,撐的兩邊皆滿,無復作往來。婦人用唾津,塗抹牝戶兩邊,已而稍寬滑落,頗作往來,一舉一坐,漸沒至根。婦人因向西門慶說:「你每常使的顫聲嬌在裏頭,只是一味熱癢不可當,怎如和尚這藥使進去,從子宮冷森森,直掣到心上,這一回把渾身上下都酥麻了。我曉的今日之命,死在你手裡了,好難捱忍也!西門慶笑道:「五兒,我有個笑話兒,說與你聽。是應二哥說的,一個人死了,閻王就拏驢皮披在身上,交他變驢。落後判官查簿籍,還有他十三年陽壽,又放回來了。他老婆看見渾身都變過來了,只有陽物還是驢的,未變過來。那人道:『我往陰間換去。』他老婆慌了,說道:『我的哥哥,你這一去,只怕不放你回來怎了?由他,等我慢慢兒的挨罷。』婦人聽了,笑將扇把子打了一下子,說道:「怪不得應花子的二老婆捱慣了驢的行貨,硶說嘴的貨,我不看世界,這一下打的你!」兩個足纏了一個更次,西門慶精還不過,他在下合著眼,由著婦人蹲踞在上,極力抽提。提的龜頭刮答刮答怪響,提勾良久,又吊過身子去,朝向西門慶,西門慶雙足舉其股,沒稜露腦而提之,往來甚急。西門慶雖身接目視,而猶如無物,良久婦人情極,轉過身子來,兩手摟定西門慶脖項,合伏在身上,舒舌頭在他口裡,那話直抵牝中,只顧揉搓,沒口子叫:「親達達,罷了!五兒的死了。」須臾一陣昏迷,舌尖冰冷,泄訖一度。西門慶覺牝中一股熱氣,直透丹田,心中翕翕然美快,不可言也。已而淫津溢出,婦人以帕抹之,兩個相摟相抱,交頭疊股,嗚咂其舌,那話通不拽出來,睡時沒半個時辰,婦人淫情未定,扒上身去,兩個又幹起來。婦人一連丟了兩遭,身子亦覺稍倦。西門慶只是佯佯不採,暗想胡僧之藥通神。看看窗外雞鳴,東方漸白。婦人道:「我的心肝,你不過卻怎樣的?到晚夕你再來,等我好歹替你咂過了罷。」西門慶道:「就咂也不得過,管情只一樁事兒就過了。」婦人道:「告我說是那一樁兒?」西門慶道:「法不傳六,再得我晚夕來對你說。」早晨起來梳洗,春梅打發穿上衣裳,韓道國、崔本又早外邊伺候。西門慶出來燒了紙,打發起身,交付二人兩封書。一封到楊州馬頭上,投王伯儒店裡下;這一封就往楊州城內,抓尋苗青問他的事情下落,快來回報我。如銀子不勾,我後邊再交來保稍去。崔本道:「還有蔡老爹書沒有?」西門慶道:「你蔡老爹書還不曾寫,交來保後邊稍了去罷。」二人拜辭,上頭口去了,不在話下。西門慶冠帶了,就往衙門中來,與夏提刑相會,道及日昨多承見招之意。夏提刑道:「今日奉屈長官佳敘,再無他客。」發放已畢,各分散來家。吳月娘又早上房擺下菜蔬,請西門慶吃粥,只見一個穿青衣皂隸,騎著快馬,夾著毡包,走的滿面汗流,到大門首問平安:「此是提刑西門老爹家?」平安道:「你是那裡來的?」那人疾便下了馬作揖,便說:「我是督催皇木的安老爹先差來送禮與老爹。俺老爹與管磚廠黃老爹,如今都往東平府胡老爹那裡吃酒,順便先來拜老爹,這裡看老爹在家不在?」平安道:「有帖兒沒有?」那人向毡包內取出,連禮物都遞與平安。平安拏進去,與西門慶看見禮帖上寫著:浙紬二端,湖綿四斤,香帶一束,古鏡一圓。分付包五錢銀子,拏回帖打發來人:「就說在家拱候老爹!」那人急急去了。西門慶一面家中預備酒菜,等至日中,二位官員喝道而至。此日乘轎張蓋甚盛。先令人投拜帖,一個是「侍生安忱拜」,一個是「侍生黃葆光拜」。都是青雲白鷴補子,烏紗皂履,下轎揖讓而入。西門慶出大門迎接,至廳上敘禮。各道契闊之情,分賓主坐下。黃主事居左,安主事居右,西門慶主位相陪。先是黃主事舉手道:「久仰賢名,盛德芳譽,學生拜遲。」西門慶道:「不敢。辱承老先生先事枉駕,當容踵叩。敢問尊號?」安主事道:「黃年兄號泰宇,取『履泰定而發天光』之意。」黃主事道:「敢問尊號?」西門慶道:「學生賤號四泉,因小庄有四眼井之說。」安主事道:「昨日會見蔡年兄,說他與宋松原都在尊府打攪。」西門慶道:「因承雲峰尊命,又是敝邑公祖,敢不奉迎?小价在京,已知鳳翁榮選,未得躬賀。」又問:「幾時家中起身來?」安主事道:「自去歲尊府別後,學生到家續了親。過了年,正月就來京了。選在工部備員主事。欽差督運皇木,前往荊州。向來道經此處,敢不奉謁?」西門慶又說:「盛儀感謝不盡。」說畢,因請寬衣,令左右安放卓席。黃主事就要起身。安主事道:「實告,我與黃年兄如今還往東平胡大尹那裡赴席。因打尊府過,敢不奉謁?容日再來取擾。」西門慶道:「就是往胡公處,去路尚許遠。縱二公不餓,其如從者何?學生不敢具酌,只備一飯在此,以犒手下從者。」于是先打發轎上攢盤。廳上安放卓席,珍羞異品,極時之盛。就是湯飯點心,海鮮美味,一齊上來。西門慶將小金鍾只奉了三盃,連卓兒抬下去,管待親隨家人、吏典。少頃,兩位官人拜辭起身,向西門慶道:「生輩明日有一小柬到,奉屈賢公到我這黃年兄同僚劉老太監庄上一敘,未審肯命駕否?」西門慶道:「既蒙寵招,敢不趨命!」說畢,送出大門,上轎而去。只見夏提刑差人來邀。西門慶說道:「我就去。」一面分付備馬,走到後邊,換了衣服出來上馬。玳安、琴童跟隨,排軍喝道,打著黑扇,逕往夏提刑家來。到廳上,敘禮,說道:「適有工部督皇木安主政和磚廠黃主政來拜,留坐了半日,去了。不然,也來的早。」見畢禮數,接了衣服下來。玳安叫排軍褶了,連帶放在氈包內。見廳上面設放兩張卓席,讓西門慶居左,其次就是西賓倪秀才。座間因敘起來,問道:「老先生尊號?」倪秀才道:「學生賤名倪鵬,字時遠,號桂巖,見在府庠備數。在我這東主夏老先生門下,設館教習賢郎大先生舉業,友道之間,實有多愧。」說話間,兩個小優兒上來磕頭。吃罷湯飯,廚役上來割道。西門慶喚玳安,拏賞賜賞了廚役,分付:「取巾來戴,把冠帶衣服,送回家去,晚上來接罷。」玳安應諾,吃了點心,回馬家來不題。且說潘金蓮從打發西門慶出來,直睡到晌午纔扒起來。甫能起來,又懶待梳頭,恐怕道後邊人說他。月娘請他吃飯,也不吃,只推不好。大後晌才出房門,來到後邊。月娘因西門慶不在,要聽薛姑子講說佛法,演頌金剛科儀。正在明間內,安放一張經卓兒,焚下香。薛姑子與王姑子兩個一對坐,妙趣、妙鳳兩個徒弟,立在兩邊,接念佛號。大妗子、楊姑娘、吳月娘、李嬌兒、孟玉樓、潘金蓮、李瓶兒、孫雪娥和李桂姐,一個不少,都在根前,圍著他坐的,聽他演誦。先是薛姑子道:

「蓋聞電光易滅,石火難消。落花無還樹之期,逝水絕歸源之路。畫堂綉閣,命盡有若長空;極品高官,祿絕猶如作夢。黃金白玉,空為禍患之資;紅粉輕衣,總是塵勞之費。妻孥無百載之歡,黑暗有千重之苦。一朝枕上,命掩黃泉。空榜楊虛假之名,黃土埋不堅之骨,田園百頃,其中被兒女爭奪;綾錦千廂,死後無寸絲之分。青春未半,而白髮來侵;賀者纔聞,而吊者隨至。苦苦苦,氣化清風塵歸土!點點輪迴喚不回,改頭換面無遍數。」

「南無盡虛空遍法界,  過見未來佛法僧三寶。」

「無上甚深微妙法,  百千萬劫難遭遇,

我今見聞得受持,  願解如來真實義!」

王姑子道:「當時釋伽牟尼佛,乃諸佛之祖,釋教之主。如何出家?願聽演說。」薛姑子便唱五供養:

「釋伽佛,梵王子,捨了江山雪山去。割肉喂鷹鵲巢頂,只修的九龍吐水混金身。纔成南無大乘大覺釋伽尊。」

王姑子又道:「釋伽佛,既聽演說。當日觀音菩薩如何修行?纔有莊嚴百化身,有天道力,願聽其說。」薛姑子又道:

「大莊嚴,妙善主,辭別皇宮香山住。天人送供跏趺坐,只修的五十三參變化身,纔成南無救苦救難觀世音。」

王姑子道:「觀音菩薩,既聽其法。昔日有六祖禪師,傳燈佛,教化行西域,東歸不立文字。如何苦功,願聽其詳!」薛姑子道:

「達磨師,盧六祖,九年面壁功行苦,盧芽穿膝伏龍虎,只修的隻履折盧任往來,纔成了南無大慈大願昆盧佛。」

王姑子道:「六祖傅燈,既聞其詳。敢問昔日有個龐居士,捨家私送窮船歸海,以成正果。如何說?」薛姑子道:

「龐居士善知識,放債來生濟貧苦。驢馬夜間私相居。只修的拋妻棄子上法船,纔成了南無妙乘妙法伽藍耶。」

月娘正聽到熱鬧處,只見平安兒慌慌張張走來,說道:「巡按宋爺家,差了兩個快手,一個門子送禮來。」月娘慌了,說道:「你爹往夏家吃酒去了,誰人打發他?」正亂著,只見玳安兒放進毡包來,說道:「不打緊,等我拏帖兒,對爹說去。交姐夫且讓那門子進來,管待他些酒飯兒著。」這玳安交下氈包,拏著帖子,騎馬雲飛般走到夏提刑家,如此這般說了:「巡按宋老爺送禮來。」西門慶看了帖子,上面寫著:鮮豬一口,金酒二尊,公紙四刀,小書一部。下書「侍生宋喬年拜」。連忙分付:「到家書童快拏我的官衙雙摺手本回去。門子答賞他三兩銀子,兩方手帕,抬盒的每人與他五錢。」玳安來家,到處尋書童兒,那裡得來?急的只遊回磨轉。陳經濟又不在,交傅夥計陪著人吃酒。玳安旋打後邊樓房裡討了手帕銀子出來,又沒人封,自家在櫃上彌封停當,交傅夥計寫了大小三包。因向平安兒道:「你就不知往那去了?」平安道:「頭裡姐夫在家時,他還在家來。落後姐夫往門外討銀子去了,他也不見了!」玳安道:「別要題,已定秫秫小廝在外邊胡行亂走的,養老婆去了!」正在急躁之門,只見陳經濟與書童兩個,疊騎著騾子纔來,被玳安罵了幾句,交他寫了官御手本,打發送禮人去了。玳安道:「賊秫小廝,仰〈扌扉〉著掙了,合蓬著去。爹不在,家裡不看,跟著人養老婆去了!爹又沒使你和姐夫門外討銀子,你平白跟了去做甚麼?看我對爹說不說!」書童道:「你說不是,我怕你?你不說,就是我的兒!」玳安道:「賊狗攘的秫秫小廝,你賭幾個真個!」走向前,一個潑腳撇翻倒,兩個就磆碌成一塊子。那玳安得手,吐了他一口唾沫纔罷了。說道:「我接爹去。等我來家,和淫婦算帳!」騎馬一直去了。月娘在後邊,打發兩個姑子吃了些茶食兒,又聽他唱佛曲兒,宣念偈子兒。那潘金蓮不住在傍,先拉玉樓不動,又扯李瓶兒,又怕月娘說。月娘便道:「李大姐,他叫你,你和他去不是,省的急的他在這裡,恁有〈百刂〉劃沒是處的!」那李瓶兒方纔同他出來。被月娘瞅了一眼,說道:「拔了蘿蔔地皮寬,交他去了,省的他在這裡,跑兔子一般,原不是那聽佛法的人!」這潘金蓮拉著李瓶兒走出儀門,因說道:「大姐姐好幹這營生!你家又不死人,平白交姑子家中宣起卷來了!都在那裡圍著他怎的?咱每出來走走,就看看大姐在屋裡做甚麼哩!」于是一直走出大廳。只見廂房內點著燈,大姐和經濟正在裡面絮聒,說不見了銀子了。被金蓮向窗櫺上打了一下,說道:「後面不去聽佛曲兒,兩口子且在房裡拌的甚麼嘴兒?」陳經濟出來,看見二人。說道:「早是我沒曾罵出來!原來是五娘、六娘來了。請進來坐。」金蓮道:「你好膽子,罵不是?」進來,見大姐正在燈下納鞋,說道:「這咱晚熱剌剌的,還納鞋?」因問:「你兩口子嚷的是些甚麼?」陳經濟道:「你問他!爹使我門外討銀子去。他與了我三錢銀了,就交我替他稍銷金汗巾子來。不想到那裡,袖子里摸銀子沒了,不曾稍得來。來家他說我那裡養老婆,和我嚷罵我這一日,急的我賭身發咒。不想丫頭掃地,地下拾起來。他把銀子收了不與,還交我明日買汗巾子來。你二位老人家說,卻是誰的不是?」那大姐便罵道:「賊囚根子,別要說嘴!你不養老婆,平白帶了書童兒去做甚麼?剛纔交玳安其麼不罵出來。想必兩個打夥兒養老婆去來,去到這咱晚纔來!你討的銀子在那裡?」金蓮問道:「有了銀子不曾?」大姐道:「有了銀子。剛纔丫頭地下掃地拾起來,我拏著哩。」金蓮道:「不打緊處,我與你銀子,明日也替我帶兩方銷金汗巾子來。」李瓶兒便問:「姐夫,門外有買銷金汗巾兒,也稍幾方兒與我。」經濟道:「門外手帕巷有名王家,專一發賣各色改樣銷金點翠手帕汗巾兒,隨你問多少也有。你老人家要甚顏色?銷甚花樣?早說與我,明日一齊都替你帶來了。」李瓶兒道:「我要一方老金黃銷金,點翠穿花鳳汗巾。」經濟道:「六娘,老金黃銷上金,不顯。」李瓶兒道:「你別要管我,我還要一方銀紅綾銷江牙海水嵌八寶汗巾兒;又是一方閃色,是蔴花銷金汗巾兒。」經濟便道:「五娘,你老人家要甚花樣?」金蓮道:「我沒銀子,只要兩方兒勾了。要一方玉色綾瑣子地兒銷金汗兒。經濟道:「你又不是老人家,白剌剌的,要他做甚麼?」金蓮道:「你管他怎的?戴不的,等我往後吃孝戴!」經濟道:「那一方要甚顏色?」金蓮道:「那一方,我要嬌滴滴紫葡萄顏色四川綾汗巾兒,上銷金間點翠,十樣錦,同心結,方勝地兒,一個方勝兒裡面一對兒喜相逢,兩邊欄子兒都是纓絡出珠碎八寶兒。」經濟聽了,說道:「耶嚛!耶嚛再沒了。賣瓜子兒開廂子打〈口弟〉噴,瑣碎一大堆!」那金蓮道:「怪短命,有錢買了稱心貨,隨各人心裡所好,你管他怎的?」李瓶兒便向荷包裡拏出一塊銀子兒,遞與經濟,說:「連你五娘的,都在裡頭哩。」那金蓮搖著頭兒,說道:「等我與他罷。」李瓶兒道:「都是一答兒哩,交姐夫稍來的,又起個窖兒?」經濟道:「就是連五娘的,這銀子還多著哩。」一面取等子稱了,一兩九錢。李瓶兒道:「剩下的,就與大姑娘稍兩方來。」那大姐連忙道了萬福。金蓮道:「你六娘替大姐買了汗巾兒,把那三錢銀子拏出來,你兩口兒鬬葉兒,賭了東道兒罷。少便叫你六娘貼些出來兒,明日等你爺不在了,買燒鴨子 、白酒 咱每吃。」經濟道:「既是五娘說,拏出來。」大姐遞與金蓮,金蓮交付與李瓶兒收著。拏出紙牌來,燈下大姐與經濟鬬。金蓮又在傍替大姐指點,登時嬴了經濟三卓。忽聽前邊打門,西門慶來家,金蓮與李瓶兒纔回房去了。經濟出來迎接西門慶回來話,說:「徐四家銀子,後日先送二百五十兩來,餘者出月交還。」西門慶罵了幾句,酒帶半酣,也不到後邊,逕往金蓮房裡來。正是:

「自有內事迎郎意,  何怕明朝花不云。」

畢竟未知後來何如,且聽下回分解:
