%# -*- coding: utf-8 -*-
%!TEX encoding = UTF-8 Unicode
%!TEX TS-program = xelatex
% vim:ts=4:sw=4
%
% 以上设定默认使用 XeLaTex 编译,并指定 Unicode 编码,供 TeXShop 自动识别

%第五十九回 
\chapter{西門慶摔死雪獅子\KG 李瓶兒痛哭官哥兒}


\begin{showcontents}{}




「日落水流西復東,  春風下盡折何窮,

巫峽廟裡低含雨,  宋玉門前斜帶風;

莫將榆莢共爭翠,  深感杏花相映紅,

灞上漢南千萬樹,  幾人遊宦別離中。」

話說孟玉樓和潘金蓮在門首打發磨鏡叟去了。忽見從東一人帶著大帽眼紗,騎著騾子,走得甚急,逕到門首下來。慌的兩個婦人,往後走不迭。落後揭開眼紗,都是韓夥計來家了。平安忙問道:「貨車到了不曾?」韓道國道:「貨車進城了。稟問老爹,卸在那里?」平安道:「爹不在家,往周爺府裡吃酒去了。收拾了,交卸在對門樓上哩。你老人家請進裡邊去。」不一時,陳經濟出來,陪韓道國入後邊,見了月娘。出來廳上,拂去塵土,把行李搭連,教王經送到家去。月娘一面打發出飯來,與他吃了。不一時,貨車纔到。經濟拏鑰匙,開了那邊樓上門,就有卸車的小腳子,領籌搬運貨,一廂廂堆卸在樓上。十大車段貨,運家用酒米,直卸到掌燈時分。崔本也來幫扶照管。堆卸完畢,查數鎖門,貼上封皮,打發小腳錢出門。早有玳安往守備府,報西門慶去了。西門慶聽見家中卸貨,吃了幾鍾酒,約掌燈以後就來家。韓夥計等著見了,在廳上坐的,悉把前後往回事,說了一遍。西門慶因問:「錢老爹書下了?也見些分上不曾?」韓道國道:「全是錢老爹這封書,十車貨少使了許多稅錢。小人把段箱兩箱併一箱,三停只報了兩停,都當茶葉馬牙香櫃上稅過來了。通共十大車貨,只納了三十兩五錢鈔銀子。老爹接了報單,也沒差巡攔下來查點,就把車喝過來了。」西門慶聽言,滿心歡喜。因說:「到明日,少不的重重買一分禮謝那錢老爹。」于是分付陳經濟陪韓夥計、崔大哥坐,後邊拏菜出來,留吃了一回酒,方纔各散回家。王六兒聽見韓道國來了,王經替他駝行李搭連來家,連忙接了行李,因問:「你姐夫來了麼?」王經道:「俺姐夫看着卸行李,還等著見俺爹纔來哩。」這婦人分付丫頭春香、錦兒、伺侯下好茶好飯。等的晚上韓道國到家,拜了家堂,脫了衣裳,淨了面目,夫妻二人各訴離情一遍。韓道國悉把買賣得意一節,告訴老婆。老婆又見搭連內沉沉重重許多銀兩,因問他;替己又帶了一、二百兩貨物酒米,卸在門前店里,漫漫發賣了銀子來家。老婆滿心歡喜:「聽見王經說又尋了個甘夥計做賣手,咱每和崔大哥與他同分利錢使,這個又好了;到出月開舖子。」韓道國道:「這使着了人做賣手,南邊還少個人立庄置貨。老爹已定還裁派我去。」老婆道:「你看貨才料。自古能者多勞,你看不會做買賣,那老爹託你麼?常言:『不將辛苦意,難得世人財。』你外邊走上三年,你若懶得去,等我對老爹說了,教姓甘的和保官兒打外,你便在家賣貨就是了。」韓道國道:「外邊走熟了,也罷了。」老婆道:「可又來,你先生迷了路,在家也是閒。」說畢,擺上酒來,夫婦二人飲了幾盃闊別之酒,收拾就寢。是夜歡娛無度,不必用說。次日都是八月初一日,韓道國早到。西門慶教同崔本、甘夥計在房子內看着收卸磚瓦木石,收拾裝修土庫,不在話下。都說西門慶見卸貨物,家中無事,忽然心中想起要往鄭愛月兒家去。暗暗使玳安兒送了三兩銀子,一套紗衣服與他。鄭家鴇子聽見西門老爹來請他家姐兒,如天上落下來的一般,連忙收了禮物,沒口子向玳安:「你多頂上老爹,就說他姐兒兩個都在家里伺候老爹。請老爹早些兒下降。」玳安走來家中書房內,回了西門慶話。西門慶約午後時分,分付玳安收拾着涼轎,頭上戴著坡巾,身上穿青緯羅暗補子直身,粉底皂靴。先走在房子,看了一回裝修土庫。然後起身,坐上涼轎,放下斑竹簾來,琴童、玳安跟隨,留王經在家,止著春鴻背著直袋,逕往院中鄭月兒家來。正是:

「天仙機上整香羅,  入手先拖雪一窩;

不獨桃源能問渡,  卻來月窟伴嫦娥。」

都說鄭愛香兒頭戴着銀絲䯼髻,梅花鈿兒,周圍金纍絲簪兒。打扮的粉面油頭,花容月貌。上着藕絲裳,下着湘紋裙。見西門慶到,笑吟吟在半門里首,迎接進去。到于明間客位,道了萬福。西門慶坐下,就分付小廝琴童:「把轎回了家去,晚夕騎馬來接。」琴童跟轎家去不題。止留玳安和春鴻兩個伺候。良久,只見鴇子出來拜見,說道:「外日姐兒在宅內多有打擾。老爹家中悶的慌,來這里自恁散心走走罷了。如何多計較?又見賜將禮來。又多謝與姐兒的衣服。」西門慶道:「我那日叫他,怎的不去?只認王皇親家了。」鴇子道:「俺每如今還怪董嬌兒和李桂兒。不知是老爹生日叫唱,他每都有了禮,只俺每姐兒沒有。若早知時,已不答應王皇親家唱,先往老爹宅裡去了。老爹那里叫唱在後,咱姐兒纔待收拾起身,只見王家人來,把姐兒的衣包拏的去。落後老爹那里又差了人來,他哥子鄭奉又說:『你若不去,一時老爹動意怒了。』慌的老身背着王家人,連忙攛掇姐兒打後門起身上轎去了。」西門慶道:「先日,我在他夏老爹家酒席上,已定下他了。他若那日不去,我不消說的就惱了。怎的他那日不言不語,不做喜歡,端的是怎的說?」鴇子道:「小行貨子家,自從梳弄了,那里好生出去供唱去?到老爹宅內,見人多,不知諕的怎樣的?他從小是恁不出語,嬌養慣了。你看甚時候纔起來!老身該催促了幾遍,說:『老爹今日來,你早些起來收拾了罷。』他不依,還睡到這咱晚。」不一時,丫鬟拏茶上來,鄭愛香兒向前遞了茶吃了。鴇子道:「請老爹到後邊坐罷。」原來鄭愛香兒家,門面四間,到底五層房子。轉過軟壁,就是竹槍籬,三間大院子,兩邊四間廂房。上首一明兩暗,三間正房,就是鄭愛月兒的房。他姐姐愛香兒的房,在後邊第四層住。但見簾攏香靄,進入明間內,供養著一軸海潮觀音,兩旁掛四軸美人,按春、夏、秋、冬;惜花春起早,愛月夜眠遲,掬水月在手,弄花香滿衣。上面挂著一聯:「捲簾邀月入,諧瑟待雲來。」上首列四張東坡椅,兩邊安二條琴光漆春凳。西門慶坐下,看見上面楷書「愛月軒」三字。坐了半日,忽聽簾攏響處,鄭愛月兒出來,不戴䯼髻,頭上挽著一窩絲杭州攢,梳的黑鬖鬖光油油的。烏雲霞著四軉;雲鬢堆縱,猶若輕烟密霧。都用飛金巧貼,帶著翠梅花鈿兒,周圍金纍絲簪兒,齊插後鬢,鳳釵半卸。耳邊帶著紫瑛石墜子。上著白藕絲對衿仙裳,下穿紫綃翠紋裙。腳下露一雙紅鴛鳳嘴,胸前搖琱璫寶玉玲瓏。正面貼三顆翠面花兒,越顯那芙蓉粉面;四周圍香風縹緲,偏相襯楊柳纖腰。正是:

「若非道子觀音畫,  定然延壽美人圖。」

望上不當不正,與西門慶道了萬福,就因灑金扇兒掩著粉臉,坐在旁邊。西門慶注目停視,比初見時節兒,越發齊整。不覺心搖目蕩,不能禁止。不一時,丫鬟又拿一道茶來。這粉頭輕搖羅袖,微露春纖,取一鍾茶過來,抹去盞邊水漬,雙手遞與西門慶。然後與愛香各取一鍾相陪。吃畢,收下盞託去,請寬衣服房裡坐。西門慶叫玳安上來,把上蓋青紗衣寬了,搭在椅子上,進入粉頭房中。但見瑤窗素紗罩淡月半浸,綉幕以夜明懸伴光高燦。正面黑漆鏤金床,床上帳懸綉錦,褥隱華裀。旁設褆紅小几,博山小篆,靄沉檀樓鼻;壁上文錦囊像窑瓶,插紫笋其中。床前設兩張綉甸矮椅,旁邊放對鮫綃錦帨。雲母屏,模寫淡濃之筆;鴛鴦榻,高閣古今之書。西門慶坐下,但覺異香襲人,極其清雅,真所謂神仙洞府,人跡不可到者也。彼此攀話之間,語言調笑之際,只見丫鬟進來安放卓兒。四個小翠碟兒,都是精製銀絲細菜,割切香芹鱘絲鰉鮓鳳脯鷥羹 。然後拿上兩筯賽團圓,如明月,薄如布,白如雪,香甜可口,酥油和蜜錢麻椒鹽荷花細餅。鄭愛香兒與鄭愛月兒親手楝攢各樣菜蔬肉絲捲,就安放小泥金碟兒內,遞與西門慶吃。旁邊燒金翡翠甌兒,斟上苦艷艷桂花木樨茶。須臾,姊妹二人陪吃了餅,收下家火去。揩抹卓席,鋪茜紅氈條,床几上取了一個沉香雕漆匣,內盛象牙牌三十二扇,兩個與西門慶抹牌。當下西門慶出了個天地分,劍行十道。那愛香兒出了個地牌,花開蝶滿枝。那愛月兒出了個人牌,搭梯望月。須臾收過去,擺上酒來。但見盤堆異果,酒泛金波。卓上無非是鵝鴨雞蹄,烹龍炮鳳。珍果人間少有,隹餚天上無雙。正是:

「舞回明月墜秦樓,  歌過行雲遮楚館。」

鴛鴦杯,翡翠盞,飲玉液,泛瓊漿。姊妹二人遞上酒去,在旁箏排雁桂,款跨鮫綃,當下鄭愛香兒彈箏,愛月兒琵琶,唱了一套兜的上心來。端的詞出佳人口,有裂石遶梁之聲。唱畢,又是十二碟果仁減碟細巧品類。姊妹兩個,促席而坐,拏骰盆兒,二十個骰兒,與西門慶搶紅猜枚。飲勾多時,鄭愛香兒推更衣出去了。獨有愛月兒陪著西門慶吃酒。先是西門慶向袖中取出白綾雙欄子汗巾兒上,一頭栓著三事挑牙兒,一頭束著金穿心盒兒。鄭愛月兒只道是香茶 ,便要打開。西門慶道:「不是香茶,是我逐日吃的補藥。我的香茶不放在這面,只用布包兒包著。」于是袖中取出一包香茶桂花餅兒,遞與他。那月兒不信,還伸手往他這邊袖子裡掏。又掏出個紫縐紗汗巾兒,上栓著一副揀金挑牙兒。拏在手中觀看,甚是可愛。說道:「我見桂姐和吳銀兒都拏著這樣汗巾兒,原來是你與他的?」西門慶道:「是我揚州船上帶來的,不是我與他,誰與他的?你若愛,與了你罷。到明日再送一副與你姐姐。」說畢,西門慶就著鍾兒裡酒,把穿心盒兒內藥吃了一服。把粉頭摟在懷中,兩個一遞一口兒飲酒砸舌,無所不至。西門慶又舒手向他身上摸弄他香乳兒,緊緊就就,賽麻團滑膩。一面推開衫兒觀看,白馥馥,猶如瑩玉一般。揣摩良久,淫心輒起,腰間那話,突然而興。解開褲帶,令他纖手籠揝。粉頭見其偉是粗大,諕的吐舌害怕。雙手摟定西門慶脖心,說道:「我的親親,你今日初會,將就我,只放半截兒罷;若都放進去,我就死了。你敢吃藥養的這等大!不然如何天生恁怪刺刺的兒?紅赤赤,紫漒漒,好呵磣人子!」西門慶笑道:「我的兒,你下去替我品品。」愛月兒道:「慌怎的,往後日子多如樹葉兒。今日初會,人生面不熟。再來,等我替你品。」說畢,西門慶欲與他講歡。愛月兒道:「你不吃酒了?」西門慶道:「我不吃了,咱睡罷。」愛月兒便叫丫鬟把酒卓擡過一邊,與西門慶脫靴。他便就往後邊,更衣澡牝去了。西門慶脫靴時,還賞了丫頭一塊銀子,打發先上床睡,炷了香,放在薰籠內。良久婦人進房,問西門慶:「你吃茶不吃?」西門慶道:「我不吃。」一面掩上房門,放下綾綃來,將絹兒安在褥下,解衣上床。兩個枕上鴛鴦,被中鸂鶒。西門慶見粉頭脫了衣裳,肌膚纖細,牝淨無毛,猶如白麵蒸餅一般,柔嫩可愛。抱了抱腰肢,未盈一掬。誠為軟玉溫香,千金難買。于是把他兩隻白生生銀條股嫩腿兒,來夾在兩邊腰眼間。那話上使了託子,向花心裡頂入。龜頭昂大,濡攪半晌,方纔沒稜。那鄭月兒把眉頭縐在一處兒,兩手攀閣在枕上,隱忍難挨,朦朧著星眼,低聲說道:「今日你饒了鄭月兒罷。」西門慶于是扛起他兩隻金蓮于肩膀上,肆行抽送,不勝歡娛。正是:

「得多少春點碧桃紅綻蕊,  風欺楊柳綠翻腰。」

有詩為證:

「帶雨龍烟匝樹奇,  妖嬈身勢似難支;

水推西子無雙色,  春點河陽第一枝;

濃豔正宜吟郡子,  功夫何用寫王維,

含情故把芳心束,  留住東風不放歸。」

當下西門慶與鄭愛月兒留戀至三更,方纔回家。到次日,吳月娘打發他往衙門中去了。和玉樓、金蓮、李嬌兒都在上房坐的。只見玳安進來上房取尺頭匣兒,往夏提刑送生日禮去。四樣鮮餚,一壜酒,一疋金段。月娘因問玳安:「你爹昨日坐轎子往誰家吃酒,吃到那咱晚纔來家?想必又在韓道國家,望他那老婆去來?原來賊囚根子成日只瞞著我,背地替他幹這等繭兒!」玳安還道:「不是,他漢子來家,爹怎好的。」月娘道:「不是那里,都是誰家?」那玳安又不說,只是笑。取了段匣送禮去了。潘金蓮道:「娘,你不消問這賊囚根子,他也不肯實說。我聽見說蠻小廝昨日也跟他爹去來。你只叫了蠻小廝來問他,就是了」一面把春鴻叫到跟前。金蓮問:「你昨日跟了你爹轎子去,在誰家吃酒來?你實說便罷,不實說,如今你大娘就要打你。」那春鴻跪下便道:「娘,休打小的!待小的說就是來。小的和玳安、琴童哥三個,跟俺爹從一座大門樓進去。轉了幾條街巷到個人家,只半截門兒,都用鋸齒兒鑲了。門裡立著位娘娘,打扮的花花黎黎的。」金蓮聽見笑了說道:「囚根子,一個院裡半門子也認不的了,趕著粉頭叫娘娘起來!」金蓮問道:「那個娘娘怎麼模樣?你認的他不認的?」春鴻道:「我不認的他。生的相菩薩樣,也相娘每頭上戴著這個假壳。進入裡面,一個年老白頭的阿婆出來,望俺爹拜了一拜。落後請到大後邊,竹籬笆進去,又是一位年小娘娘出來,不戴假壳。生的銀盆臉,瓜子面,搽的嘴唇紅紅的,陪著俺爹吃酒。」金蓮道:「你每都在那里坐來?」春鴻道:「我在俺玳安、琴童哥,便在阿婆房裡,陪著俺每吃酒并肉兜子 來。」把月娘、玉樓笑的了不得。因問道:「你認的他不認的?」春鴻道:「那一個好似在咱家唱的。」玉樓笑道:「就是李桂姐了。」月娘道:「原來摸到他家去了!」李嬌兒道:「俺家沒半門子,也沒竹搶籬。」金蓮道:「只怕你不知道。你家新安的半門子是的。」問了一回,西門慶來家,往夏提刑家拜壽去了。都說潘金蓮房中養活的一隻白獅子貓兒,渾身純白,只額兒上帶龜背一道黑,名喚「雪裡送炭」又名「雪獅子」又善會口啣汗巾兒拾扇兒。西門慶不在房中,婦人晚夕常抱著他在被窩裡睡。又不撒尿屎在衣服上。婦人吃飯,常蹲在肩上喂他飯,呼之即至,揮之即去。婦人常喚他是「雪賊」。每日不吃牛肝乾魚,只吃生肉半斤,調養得十分肥壯,毛內可藏一雞彈。甚是愛惜他,終日抱在膝上摸弄,不是生好意。因李瓶兒、官哥兒哥兒平昔好貓,尋常無人處,在房裡用紅絹裹肉,令貓撲而撾食。也是合當有事,官哥兒心中不自在,連日吃劉婆子藥,略覺好些。李瓶兒與他穿上紅段衫兒,安頓在外間炕上,鋪著小褥子兒頑耍。迎春守著,奶子便在旁拏著碗吃飯。不料金蓮房中這雪獅子,正蹲在護炕上。看見官哥兒在炕上穿著紅衫兒,一動動的頑耍。只當平日哄喂他肉食一般,猛然望下一跳,撲將官哥兒,身上皆抓破了。只聽那官哥兒呱的一聲,倒咽了一口氣,就不言語了,手腳俱被風搐起來。慌的奶子丟下飯碗,摟抱在懷,只顧唾噦,與他收驚。那貓還來趕著他要撾,被迎春打出外邊去了。如意兒實承望孩子搐過一陣好了。誰想只顧常連;一陣不了,一陣搐起來。李瓶兒入在後邊。一面使迎春:「後邊請娘去,哥兒不好了,風搐著哩,叫娘快來!」那李瓶兒不聽便罷。聽了,正是:

「驚損六葉連肝肺,  諕壞三毛七孔心。」

連月娘慌的兩步做一步,走逕撲到房中。見孩子搐的兩隻眼直往上弔,通不見黑眼睛珠兒,口中白沫流出,咿咿猶如小雞叫,手足皆動。一見,心中猶如刀割相侵一般,連忙摟抱起來,臉搵著他嘴兒,大哭道:「我的哥哥,我出去好好兒,怎麼的搐起來!」迎春與奶子悉把被五娘房裡貓所諕一節說了。那李瓶兒越發哭起來,說道:「我的哥哥,你緊不可公婆意,今日你只當脫不了,打這條路兒去了!」月娘聽了,一聲兒沒言語。一面叫將金蓮來問他說:「是你屋裡的貓諕了孩子。」金蓮問:「是誰說的?」月娘指著:「是奶子和迎春說來。」金蓮道:「你著這老婆子這等張睛!俺貓在屋裡好好兒的臥著不是?你每亂道,怎的把孩子諕了,沒的賴人起來!爪兒只揀軟處捏,俺每這屋裡是好纏的!」月娘道:「他的貓,怎得來這屋里?」迎春道:「每常也來這邊屋裡走跳。」那金蓮接過來道:「早時你說,每常怎的不撾他?可可今日兒就撾起來?你這丫頭,也跟著他恁張眉瞪眼兒六說白道的!將就些兒罷了,怎的要把弓兒扯滿了,可可兒俺每自恁沒時運來!」于是使性子抽身往房里去了。看官聽說:常言道:「花枝葉下猶藏刺,人心怎保不懷毒?」這潘金蓮平日見李瓶兒從有了官哥兒,西門慶百依百隨,要一奉十,每日爭姘競寵,心中常懷嫉妒不平之氣。今日故行此陰謀之事,馴養此貓。必欲諕死其子,使李瓶兒寵衰,教西門慶復親于己。就如昔日屠岸賈養神獒害趙盾丞相一般。正是:

「湛湛青天不可欺,  未曾舉意早先知;

休道眼前無報應,  古往今來放過誰?」

月娘眾人見孩子只顧搐起來,一面熬姜湯灌他。一面使來安兒快叫劉婆去。不一時劉婆子來到,看了脈息,只顧跌腳,說道:「此遭驚諕重了,是驚風,難得過來。」急令快熬燈心薄荷湯金銀湯。取出一丸金箔丸來,向鍾兒內研化,牙關緊閉。月娘連忙拔下金簪兒來,撬開口灌下去:「過得來便罷。如過不來,告過主家奶奶,必須要灸幾蘸纔好。」月娘道:「誰敢躭?必須還等他爹來,問了他爹。不然灸了,惹他來家吆喝。」李瓶兒道:「大娘,救他命罷!若等來家,只恐遲了。惹是他爹罵,等我承當就是了。」月娘道:「孩兒是你的孩兒,隨你灸。我不敢張主。」當下劉婆子把官哥兒眉攢脖根,兩手關尺并心口,共灸了五蘸,放他睡下。那孩子昏昏沉沉,直睡到日暮時分,西門慶來家,還不醒。那劉婆見西門慶來家,月娘與了他五錢銀子藥錢,一溜烟從夾道內出去了。西門慶歸到上房,月娘把孩子風搐不好,對西門慶說了。西門慶連忙走到前邊來看視。見李瓶兒哭的眼紅紅的,問:「孩兒怎的風拍起來?」李瓶兒來滿眼落淚,只是不言語。問丫頭、奶子,都不敢說。西門慶又見官哥兒手上皮兒去了,灸的滿身火艾。心中瞧噪,又走到後邊問月娘。月娘隱瞞不住,只得把金蓮房中貓驚諕之事說了:「劉婆子剛纔看,說是急驚風。若不針灸,難過得來。若等你來,又恐怕遲了。他娘母子主張,教他灸了孩兒身上五蘸。纔放下他睡了,這半日還未醒。」西門慶不聽便罷,聽了此言,三尸暴跳,五臟氣沖;怒從心上起,惡向胆邊生。直走到潘金蓮房中,不由分說,尋著貓提溜著腳,遠向穿廊望石臺基輪起來只一捽,只聽響喨一聲,腦漿迸萬朵桃花,滿口牙零擒碎玉。正是:

「不在陽間擒鼠耗,  卻歸陰府作狸仙。」

那潘金蓮見他拏出貓去捽死了,坐在炕上風紋也不動。待西門慶出了門,口裡喃喃吶吶罵道:「賊作死的強盜,把人粧出去殺了纔是好漢!一個貓兒礙得你〈口床〉屎,亡神也似走的來捽死了。他到陰司裡,明日邊問你要命,你慌怎的!賊不逢好死變心的強盜!」這西門慶走到李瓶兒房里,因說奶子、迎春:「我教你好生看著孩兒,怎的教貓諕了他,把他手也撾了?又信劉婆子那老淫婦,平白把孩子灸的恁樣的!若好便罷;不好,把這老淫婦拏到衙門裡,與他個兩拶!」李瓶兒道:「你著孩兒緊日不得命,你又是恁樣的。孝順是兒家,他也巴不得要好哩。」當下李瓶兒只指望孩兒好來。不料被艾火把風氣反于內,變為慢風。內裡抽搐的腸肚兒皆動,尿屎皆出。大便屙出五花顏色,眼目忽睜忽閉,中朝只是昏沉不省,奶也不吃了。李瓶兒慌了,到處求神問卜打卦,皆有凶無吉。月娘瞞著西門慶有請劉婆子來家調神。又請小兒科太醫來看,都用接鼻散試之。若吹在鼻孔內打鼻涕,還看得;若無鼻涕出來,則看陰騭守他罷了。于是吹下去,茫然無知,並無一個噴涕出來。越發晝夜守著哭涕不止,連飲食都減了。看看到八月十五日將近。月娘因他不好,連自家生日都回了不做。親戚內眷就送禮來,也不請。家中止有吳大妗、楊姑娘并大師父來相伴。那薛姑子和王姑子兩個,在印經處爭分錢不平,爭又使性兒,彼此互相揭調。十四日賁四同薛姑子催討,將經卷挑將來,一千五百卷都完了。李瓶兒又與了一弔錢買布馬香燭,十五日同陳經濟早往岳廟裡進香布。把經來看著都散施盡了,走來回李瓶兒話。喬大戶家一日一遍使孔嫂兒來看。又舉薦了一個看小兒的鮑太乙來看,說道:「這個變成天弔客忤,治不得了。」白與了他五錢銀子,打發去了。灌下藥去也不受,還吐出來了。只是把眼合著,口中咬的牙格支支響。李瓶兒通衣不解帶,晝夜口接在懷中,眼淚不乾的只是哭。西門慶也不往那裡去,每日衙門中來家,就進來看孩兒。那時正值八月下旬天氣。李瓶兒守著官哥兒,睡在床上。卓上點著銀燈。丫鬟、養娘都睡熟了。覲著滿窗月色,更漏沉沉。見那孩兒只是昏昏不省人事,一向愁腸萬結,離思千端。正是:

「人逢喜事精神爽,  悶來愁腸磕睡多。」

但見:

「銀河耿耿,玉漏迢迢。穿窗皓月耿寒光,透戶涼風吹夜氣。雁聲嘹 ,孤眠才子夢魂驚;蛩韻凄涼,獨宿佳人情緒苦。譙樓禁鼓,一更未盡一更敲;別院寒砧,千搗將殘千搗起。畫簷前叮噹鐵馬,敲碎仕女情懷;銀臺上閃爍燈光,偏照佳人長嘆。一心只想孩兒好,誰料愁來在夢多。」

當下李瓶兒臥在床上,似睡不睡,夢見花子虛從前門外來,身穿白衣,恰活時一般。見了李瓶兒,厲聲罵道:「潑賊淫婦,你如何抵盜我財物與西門慶?如今我告你去也!」被李瓶兒一手扯住他衣袖,央及道:「好哥哥,你饒怒我則個!」花子虛一頓,撒手驚覺,都是南柯一夢。醒來,手裡扯著都是官哥兒的衣衫袖子。連噦了幾口,道:「怪哉,怪哉!」一聽兩更鼓時,正打三更三點。這李瓶兒諕的渾身冷汗,毛髮皆豎起來。到次日西門慶進房來,把夢中之事,告訴與西門慶。西門慶道:「知道他死到那裡去了!此是你夢想舊境。只把心來放正著,休要理他。你休害怪!如今我使小廝拏轎子接了吳銀兒,晚夕來與你做伴兒。再把老媽子叫來,伏你兩個。」玳安打院裡接了吳銀兒來。那消到日西時分,那官哥兒在奶子懷裡,只搐氣兒了。慌的奶子叫李瓶兒:「娘,你來看,哥哥這黑眼睛珠兒只往上翻。口裡氣兒,只有出來的,沒有進去的!」這李瓶兒走來,抱到懷中,一面哭起來,叫丫頭:「快請你爹去,你說孩子待斷氣也!」可好常時節又走來說話,告訴:「房子兒尋下了,門面兩間二層,大小四間,只要三十五兩銀子。」西門慶聽見後邊官哥兒重了,就打發常時節起身,說:「我不送你罷!改日我使人拏銀子和你看去。」急急走到李瓶兒房中。月娘眾人,連吳銀兒、大妗子,都在房里瞧著。那孩子在他娘懷里,把嘴一口口搐氣兒。西門慶不忍看他,走到明間椅子上坐著,只長吁短氣。那消半盞茶時,官哥兒嗚呼哀哉,斷氣身亡。時八月廿三日申時也,只活了一年零兩個月。合家大小,放聲號哭。那李瓶兒撾耳撓腮,一頭撞在地下,哭的昏過去半日,方纔甦省。摟著他大放聲哭,叫道:「我的沒救星兒,心疼殺我了!寧可我同你一答兒裡死了罷!我也不久活于世上了!我的拋閃殺人的心肝,撇的我好苦也!」那奶子如意兒和迎春,在旁哭的言不得,動不得。西門慶即令小廝收拾前廳西廂房乾淨,放下兩條寬凳,要把孩子連枕席被褥擡出去那裡挺放。那李瓶兒倘在孩兒身上,兩手摟抱著,那里肯放。口口聲聲直叫:「沒救星的冤家,嬌嬌的兒,生揭了我的心肝去了!撇的我枉費辛苦,乾生受一場,再不得見你了。我的心肝!」月娘眾人哭了一回,在旁勸他不住。西門慶走來,見他把臉抓破了,滾的寶髻鬅鬆,烏雲散亂,便道:「你看蠻子!他既然不是你我的兒女,乾養活他一場。他短命死了,哭兩聲丟開罷了。如何只顧哭了去?又哭不活他!你的身子也要緊。如今擡出去,好叫小廝請陰陽來看那是甚麼時候?」月娘道:「這個也有申時前後。」玉樓道:「我頭裡怎麼說來,他管情還等他這個時候纔去。原是申時生,還是申時死。日子又相同,都是二十三日。只是月分差些,圓圓的一年零兩個月。」李瓶兒見小廝每伺候兩旁要擡他,又哭了。說道:「慌擡他出去怎麼的?大媽媽,你伸手摸摸,他身上還熱的。」叫了一聲:「我的兒嚛,你教我怎生割捨的你去?坑得我好苦也!」一頭又撞倒在地下,放聲哭道,有山坡羊為證:

「叫一聲青天,你如何坑陷了人奴性命?叫一聲我的嬌兒呵!恨不的一聲兒就要把你叫應。也是前緣前世,那世裡少欠下你冤家債不了。輪著我今生今世,為你眼淚也拋流不盡。每日家另胆提心,費殺了我心!從來我又不曾坑人陷人,蒼天如何恁不睜眼!非是你無緣,必是我那些兒薄〈亻辛〉,撇的我面撲著地,樹倒無陰。來的竹籃打水勞而無效。叫了一聲痛腸的嬌生,奴情愿和你陰靈路上,一處兒行!」

當下李瓶兒哭了一回,把官哥兒擡出停在西廂房內。月娘向西門慶計較:「還對親家那里,并他師父廟裡說聲去。」西門慶道:「他師父廟裡,明早去罷。」一面使玳安往喬大戶家說了。一面使人請了徐陰陽來批書。又拏出十兩銀子與賁四,教他快擡了一付平頭杉板,令匠人隨即儹造了一具小棺槨兒,就要入殮。喬宅那里一聞來報,隨即喬大戶娘子就坐轎子,進門來就哭。月娘眾人都陪著大哭了一場,告訴前事一遍。不一時說了陰陽徐先生來到,看了說道:「哥兒還是正申時永逝。」月娘分付出世,教與他看看黑書。徐先生搯指,尋復又檢閱了陰陽秘書,瞧了一回:「哥兒生時八字,生于政和丙申六月廿三日申時,卒于政和丁酉八月廿三日申時。月令丁酉,日干壬子,犯天地重春。本家都要忌,忌哭聲。親人不忌。入殮之時,蛇龍鼠兔四生人,避之則吉。又黑書上云:『壬子日死者,上應寶瓶宮,下臨齊地。』他前生曾在兗州蔡家作男子。曾倚力奪人財物,吃酒落魄,不敬天地六親。橫事牽連,遭氣寒之疾。久臥床席,穢污而亡。今生為小兒,亦患風癇之疾。十日前被六畜驚去魂魄,又犯土司太歲,先亡攝去魂死。託生往鄭州王家為男子。後作千戶,壽六十八歲而終。」須臾,徐先生看了黑書:「請問老爹,明日出去,或埋或化?」西門慶道:「明日如何出得?出三日,念了經,到五日出去,墳上埋了罷。」徐先生道:「二十七日丙辰,合家本命都不犯。宣正午時掩土。」批畢書,一面就收拾入殮。已有三更天氣。李瓶兒哭著往房中尋出他幾件小道衣、道髻、鞋襪之類,替他安放在棺槨內。釘了長命釘,合家大小又哭了一場,打發陰陽去了。次日,西門慶亂著,也沒往衙門中去。夏提刑打聽得知,早晨衙門散時,就來弔問致賻慰懷。又差人對吳道官廟裡說知。到三日,請報恩寺八眾僧人在家誦經。吳道官廟裡并喬大戶家,俱備折卓三牲來祭奠。吳大舅、沈姨夫,門外韓姐夫、花大舅,都有三牲祭卓來燒布。應伯爵、謝希大、溫秀才、常時節、韓道國、甘出身、賁地傳、李智、黃四都鬬了分資,晚夕來與西門慶宿伴。打發僧人去了,叫了一起提偶的,先在哥兒靈前祭畢。然後西門慶在大廳上放卓席,管待眾人。那日院中李桂姐、吳銀兒并鄭月兒三家,都有人情來上布。李瓶兒思想官哥兒,每日黃懨懨,連茶飯兒都懶待吃。題起來,只是哭涕,把喉音都哭啞了。西門慶怕他思想孩兒,尋了拙智,白日裡分付奶子、丫鬟和吳銀兒相伴他,不離左右。晚夕西門慶一連在他房中,歇了三夜,枕上百般解勸。薛姑子夜間又替他唸楞嚴經解冤呪,勸他休要哭了:「經上不說的好:『改頭換面輪迴去,來世機緣莫想他。』當來世他不是你的兒女,都是宿世冤家債主託出來,化財化目,騙劫財物。或一歲而亡,二歲而亡,三六九歲而亡。一日一夜,萬死萬生。陀羅經上不說的好:昔日有一婦人,常持佛頂心陀羅經,日以供養不缺。乃子三年之前,曾置毒藥,殺害他命。此冤家不爭離于前後,欲求方便,致殺其母。遂以托蔭此身,向母胎中,抱母心肝,令母至生產之時,分解不得,萬死千生。及至生產下來,端正如法。不過兩歲,即便身亡。母思憶之,痛切號哭。遂即把他孩兒,拋向水中。此是三遍托蔭此身向母腹中,欲求方便,致殺其母。至第三遍,准前得生,向母胎中,百千計較,抱母心肝,令其母千生萬死,悶絕叫喚。准前得生下,特地端嚴,相見具足。不過兩歲,又以身亡,母既見之,不覺放聲大哭。是何惡業因緣?准前把孩兒直至江邊,已經數時,不忍拋棄。感得觀世音菩薩,遂化作一僧,身披百衲,直至江邊。乃謂此婦人曰:『不用啼哭。此非是你男女,是你三生前冤家,三度托生,欲殺母不得。為緣你常持誦佛頂心陀羅經,并供養不缺,所以殺汝不得。若你要見這冤家,但隨貧僧手指看之。』道罷,以神通力一指,其兒遂化作一夜叉之形,向水中而立。報言:『緣汝曾殺我來,我今故來報冤。蓋緣汝有大道心,常持佛頂心陀羅經,善神日夜擁護,所故殺汝不得。我已蒙觀世音菩薩受度了,從今永不與汝為冤。』道畢,沉水中不見。此女人兩淚交流,禮拜菩薩。歸家益修善事,後壽至九十七歲而終,轉女成男。不該我貧僧說,今你這兒子,必是宿世冤家,托來你蔭下,化目化財,要惱害你身。為緣你供養修時,那捨了此經一千五百卷,有此功行,他投害你不得,今此離身,到明日再生下來,纔是你兒女。」這李瓶兒聽了,終是愛緣不斷。但題起來,輒流涕不止。須臾,過了五日光景。到廿七日早辰,雇了八名青衣白帽小童,大紅銷金棺,與旛幢雲蓋,玉梅雪柳,圍隨前首。大紅銘旌,題著「西門冢男之柩」吳道官廟裡,又差了十二眾青衣小道童兒來,遶棺轉呪,生神玉章,動清樂送殯。眾親朋陪西門慶穿素服,走至大街東口,將及門上,纔上頭口。西門慶恐怕李瓶兒到墳上悲慟,不叫他去。只是吳月娘、李嬌兒、孟玉樓、潘金蓮、大姐家里五頂轎子,陪喬親家母大妗子和李桂姐、鄭月兒、吳舜臣媳婦鄭玉姐,往內頭去。留下孫雪娥、吳銀兒并個姑子在家,與李瓶兒做伴兒。那李瓶兒見不放他去,見棺材起身,送出到大門首,趕著棺材大放聲,一口一聲,只叫:「不來家虧心的兒嚛!」叫的連聲氣破了。不防一頭撞在門底下,把粉額磕傷,金釵墜地。慌了吳銀兒與孫雪娥,向前搊扶起來,勸歸後邊去了。到了房中,見炕上空落落的,只有他耍的那壽星博浪鼓兒,還掛在床頭上。一面想將起來,拍了卓子,由不的又哭了。山坡羊全腔為證:

「進房來,四下靜,由不的我俏嘆。想嬌兒,哭的我肝腸兒氣斷。想著生下你來,我受盡了千辛萬苦。說不的偎乾就濕,成日把你耽心兒來看。教人氣破了心腸,和我兩個結冤。實承望你與我做生兒,團圓久遠。誰知道天無眼,又把你殘生喪了。撇的我前不著村,後不著店。明知我不久也命喪在黃泉。來的咱娘兒兩個,鬼門關上一處兒眠。叫了一聲我嬌嬌的心肝,皆因是前世裡無緣,你今生壽短!」

那吳銀兒在旁,一面拉著他手,勸說道:「娘,少哭了。哥哥已是拋閃了你去了,那裡再哭得活?你須自解自嘆,休要只顧煩惱了。」雪娥道:「你又年少青春,愁到明日養不出來也怎的?這裡墻有縫,壁有眼,俺每不好說的。他使心用心,反累己身。誰不知他氣不忿你養這孩子?若果是他害了,當當來世,教他一還一報,問他要命。不知你我也被他話理了幾遭哩!只要漢子常守著他,便好。到人屋裡睡一夜兒,他就氣生氣死。早時前者你每都知道,漢子等閒不到我後邊。到了一遭兒,你看背地亂都唧喳成一塊。對著他姐兒每,說我長,道我短。那個布包兒里也看哩!俺每也不言語,每日洗著眼兒看著他。這個淫婦,到明日還不知怎麼死哩!」李瓶兒道:「罷了!我也惹了一身病在這裡,不知在今日明日死也!和他也爭執不得了。隨他罷!」正說著,只見奶子如意兒向前跪下,哭道:「小媳婦有句話,不敢對娘說。今日哥兒死了,乃是小媳婦沒造化,只怕往後爹的大娘打發小媳婦出去。小媳婦男子漢又沒了,那裡投奔?」李瓶兒見他這般說,又心中傷痛起來,說:「我有那冤家在一日,去用他一日。他豈有此話說?」便道:「怪老婆,你放孩子便沒了,我還沒死哩。總然我到明日死了,你恁在我手下一場,我也不教你出門。往後你大娘身子若是生下哥兒小姐來,你就接了奶,就是一般了。你慌亂的是此甚麼?」那如意兒方纔不言語了。這李瓶兒良久又悲慟哭起來。前腔:

「想嬌兒,想的我無顛無倒。盼嬌兒,除非是夢兒中來到。白日裡,覩物傷情,如刀剜了肺腑。到晚間,睡醒來,再不見你在我這懷兒中抱,由不的珍珠望下拋。你再不來在描金床兒上睡著頑耍,你再不來在我手掌兒上引笑,你再不來相靠著我胸膛兒來的生抱;這熱笑笑心肝割上一刀,奴為你乾生受,枉費了徒勞,稱怨了別人,撇的我無有個下稍!」

雪娥與吳銀兒兩個在旁,解勸了一回,說道:「你肚中吃了些甚麼兒?這般只顧哭了去!」一面綉春後邊拿了飯來,擺在卓上,陪他吃。那李瓶兒怎生嚥得下去?只吃了半甌兒,就丟下不吃了。西門慶在墳上,教徐先生畫了穴,把官哥兒就埋在先頭陳氏娘懷中,抱孫葬了。那日喬大戶山頭,并眾親戚,都在祭祀。就在新蓋捲棚管待飲酒一日。來家,李瓶兒與月娘、喬大戶娘子、大妗子磕著頭又哭了,向喬大娘子說道:「親家,誰似奴養的孩兒不氣長短命死了。既死了,你家姐姐做了望門無力,勞而無功。親家休要笑話。」那喬大戶娘子說道:「親家怎的這般說話?孩兒每各人壽數,誰人保得後來的事!常言:「先親後不改』,親家每又不老,往後愁沒子孫?須得慢慢來,親家也少要煩惱了。」說畢,作辭回家去了。西門慶在前廳,教徐先生灑輝,各門上都貼辟非黃符。死者煞高三丈,向東北方而去,遇日遊神沖回不出,斬之則吉。親人勿避。西門慶拏出一疋大布、二兩銀子,謝了徐先生,管待出門。晚夕入李瓶兒房中,陪他睡。夜間百般言語溫存。見官哥兒的戲耍物件都還在根前,恐怕李瓶兒看見,思想煩惱,都令迎春拏到後邊去了。正是:

「思想嬌兒晝夜啼,  寸心如割命懸絲;

世間萬般哀苦事,  除非死別共生離。」

畢竟未知後來何如,且聽下回分解:




\end{showcontents}


