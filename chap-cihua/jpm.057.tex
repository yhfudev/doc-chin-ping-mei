%# -*- coding: utf-8 -*-
%!TEX encoding = UTF-8 Unicode
%!TEX TS-program = xelatex
% vim:ts=4:sw=4
%
% 以上设定默认使用 XeLaTex 编译,并指定 Unicode 编码,供 TeXShop 自动识别

%第五十七回 
\chapter{道長老募修永福寺\KG 薛姑子勸捨陀羅經}

「本性員明道自通,  番身跳出網羅中,

修成禪那非容易,  煉就無生豈俗同,

清濁幾番隨運轉,  闢門數仞任西東,

逍遙萬億年無計,  一點神光永注空。」

話說那山東東平府地方,向來有個永福禪寺,起建自梁武帝普通二年,開山是那萬迴老祖。怎麼叫做萬迴老祖?因那老師父七八歲的時節,有個哥兒從軍邊上,音信不通,不知生死。因此上那老娘兒思想那大的孩兒,掉不下的心腸,時常在家啼哭。忽一日,那孩子問著母親說道:「娘這等清平世界,孩兒們又沒的打攪你。頓頓兒小米飯兒,咱家也儘挨的過。恁地哩你時時掉下淚來?娘你說與咱,咱也好分憂哩。」那老娘兒就說:「小孩子,你還不知道老人家的苦哩!自從你老頭兒去世,你大哥兒到邊上去做了長官,四五年地信兒也不捎一個來家。不知他死生存亡,教我老人家怎生弔的下?」說了又哭起來。那孩子說:「早是這等,有何難哉?娘,如今哥在那裏?咱做弟郎的早晚間走去,抓著哥兒,討個信來回覆你老人家,卻不是好?」那婆婆一頭哭,一頭笑起來,說道:「怪呆子!說起你哥在恁地,若是那一百二百里程途,便可去的。直在那遼東地面,去此一萬餘里,就是那好漢子,也走得要不的。直要四五個月纔到哩。笑你孩兒家怎麼去的?」那孩子就說:「嗄!若是果在遼東,也終不在個天上,我去去,尋哥兒就回也。」只見把靸鞋兒繫好了,把直裰兒整一整,望著婆兒拜個揖,一溜煙去了。那婆婆叫之不應,追之不及,愈添愁悶。也有鄰舍街坊婆兒婦女,捱肩插背,拏湯送水,說長道短,前來解勸。也有說的是的,說道:「孩兒門怎去的遠?早晚間卻回也。」因此婆婆也收著兩眶眼淚,悶悶的坐地。看看紅日西沉,東鄰西舍,一個個燒湯煮飯,一個上榻關門。那婆婆探頭探腦,那兩隻眼珠兒一直向外,恨不的趕將上去。只見遠遠的望見那黑魆魆影兒頭有一個小的兒來也。那婆婆就說:「靠天靠地,靠著日月三光,若得俺小的子兒來也,也不負了俺修齋吃素的念頭!」只見那萬迴老祖一忽地跪到跟前,說:「娘你還未睡炕哩。咱已到遼東抓著哥兒,討的平安家信來也。」婆婆笑道:「孩兒你不去的正好,免教你老人家掛心。只是不要弔著謊,哄著老娘。那里有一萬里路程朝暮往還的?」孩兒道:「娘你不信麼?」一直里卸下衣包,取出平安家信,果然是那哥兒手筆。又取出一件汗衫帶回漿洗的,也是那個婆婆親手縫紉的,毫厘不差。因此哄動了街坊,叫做「萬回」。日後捨俗出家,就叫做萬回長老。果然是道德高妙,神通廣大。曾在那後趙皇帝石虎跟前,吞下兩升鐵針兒;又在那梁武皇殿下,在頭頂上取出舍利三顆。因此勑建那永福禪寺,做那萬回老祖的香火院。正不知費了多少錢糧。正是:

「神僧出世神通大,  聖主尊隆聖澤深。」

不想那歲月如梭,時移事改。只見那萬迴老祖歸天圓寂,那些得皮得肉的上人們,一個個多化去了。只見有個憊賴的和尚,撇賴了百丈清規,養婆兒,吃燒酒 ,咱事兒不弄出來?打哄了燒苦葱,咱勾當兒不做?卻被那些潑皮賴虎,常常作酒撈錢抵當。不過一會兒,把袈裟也當了,鍾兒、磬兒多典了,殿上一椽兒賣了,沒人要的燒了,磚兒、瓦兒換酒吃了。弄得那雨淋風刮,佛像兒倒了,荒荒涼涼。燒香的也不來了。主顧門徒、做道場的、荐亡的,多是關大王賣豆腐,鬼兒也沒的上門了!一片鍾鼓道場,忽變做荒煙衰草!驀地里,三四十年,那一個扶衰起廢?原來那寺里有個道長老,原是西印度國出身。因慕中國清華,發心要到上方行腳。打從那流沙河、星宿海、漼兒水地方,走了八九個年頭,才到中華區處。迤邐來到山東地方,卓錫在這個破寺院裏面。面壁九年,不言不語。真個是:

「佛法原無文字障,  工夫好向定中尋。」

忽一日,發個念頭,說道:「呀!這寺院兒坍塌的這模樣了。你看這些蠢頭村胸的禿驢,止會吃酒口塀童飯。把這古佛道場,弄得赤白白地,豈不可惜!那一個尋得一磚半尾,重整家風?常記的古人說得好:『人傑地靈。』事到今日 咱不做主,那個做主?咱不出頭,那個出頭兒?且前日山東有個西門大官官,居錦衣之職。他家私巨萬,富比王侯。家中那一件沒有?前日餞送未西廉御史,曾在咱這裏擺設酒席。他因見咱這裏寺宇傾頹,就有個舍錢布施,鼎建重新的意思。咱那時口雖不言,心窩里已有下幾分了。今日呵,若得那個檀越為主作倡,管情早晚間把咱好事成就也!咱須辦自家去走一遭。」當時間喚起法子徒孫,打起鐘,敲起鼓,舉集大眾,上堂宣揚此意。那長老怎生打扮?只見

「身上禪衣猩血染,  雙環掛耳是黃金,

手中錫杖光如鏡,  百八胡珠耀日明;

開覺明路現金繩,  提起凡夫夢亦醒,

龐眉紺髮銅鈴眼,  道是西天老聖僧。」

那長老宣揚已畢,就教行者拏過文房四寶,磨起龍香劑,飽揝鬚筆,展開烏絲欄,寫著一篇疏文。先敘那始末根由,後勸人捨財作福。寫的行行端正,字字清新。好長老真個是古佛菩薩現身,從此辭了大眾,著上了禪鞋,戴上個斗篷笠子,一壁廂直奔到西門慶家府里來。且說西門慶辭別了應伯爵,轉到後廳,直到捲棚下卸了衣服。走到吳月娘房內,把那應伯爵荐水秀才的事體,說了一番。就說道:「咱前日東京去的時節,多虧那些親朋齊來與咱把盞。如今少不的也要整辦些兒小酒回答他。倒今日空間,沒件事體,就把這事兒完了也罷。」當下就叫了玳安拿了籃兒,到十市街坊買下些時鮮菓品,豬羊魚肉。腌臘雞鵝嗄飯之類。分付了當,就分付小廝分頭去請各位。一面拉者月娘一同走到李瓶兒房裏來看官哥。李瓶兒笑嘻嘻的接住了月娘、西門慶。西門慶道:「娘兒來看孩子哩。」李瓶兒就叫奶子抱出官哥。見眉目稀疎,就如粉塊裝成一般,笑欣欣直攢到月娘懷里來,月娘把手接著,抱起道:「我的兒,恁地乖覺。長大來定是聰明伶俐的。」又向那孩子說:「兒長大起來,恁地奉養老娘哩?」那李瓶兒就說:「娘說那里話?假饒兒子長成,討的一官半職,也先向上頭封贈起。娘那鳳冠霞帔,穩穩兒先到娘哩!好生奉養老人家。」西門慶接口便說:「兒,你長大來,還掙箇文官。不要學你家老子,做箇西班出身。雖有興頭,卻沒十分尊重。」正說著,不想那潘金蓮正在外邊聽見,不覺的怒從心上起,就罵道:「沒廉耻弄虛脾的臭娼根!偏你會養兒子哩!也不曾徑過三箇黃梅,四箇夏至;又不曾長成十五六歲,出幼過關,上學堂讀書。還是水的泡,與閻羅王合音在這裡的。怎見的就做官?就封贈那老夫人?我那怪賊囚根子,沒廉耻的貨,怎地就見的要他做箇文宦,不要像你?」正在嘮嘮叨叨,喃喃洞洞,一頭罵一頭著惱的時節,只見那玳安走將進來,叫聲五娘,說道:「爹在那裡?」潘金蓮便罵:「怪尖嘴的賊囚根子!那個曉得你什麼爹在那裡?爹怎的到我這屋裡來,他自有五花官誥的太奶奶,老封婆,八珍五鼎奉養他的在那裡?那裡問著我討?」那玳安就曉的不是路了,說:「是了。」望六娘房裡便走。走到房門前打個咳嗽,朝著西門慶道:「應二爹在廳上。」西門慶道:「應二爹纔送的他去,又做甚?」玳安道:「爹自家出去便知。」西門慶只得撇了月娘、李瓶兒,仍到那捲棚下面,穿了衣服,走到外邊迎接伯爵。正要動問間,只見那募緣來的長老已到西門慶門首了。高聲叫:「阿彌陀佛!這是西門老爹門首麼?那箇掌事的管家與吾傳報一聲?說道扶桂子,保蘭孫,求福有福,求壽有壽,東京募緣的長老求見。」原來西門慶平日原是一箇散漫好使錢的漢子。又是新得官哥,心下十分歡喜,也要幹些好事保佑孩兒。小廝也通曉得,並不嗔道作難,一壁廂進報西門慶。西門慶就說:「且教他進來看。」只見管家的三步那來兩步走,就如見子活佛的一般,慌忙請了長老,那長老進到花廳裡面,打了箇問訊,說道:「貧僧出身西印度國,行腳到東京汴梁,卓錫在永福禪寺,面壁九年,頗傳心印。止為那殿宇傾頹,琳宮倒塌。貧僧想的起來,為佛弟子,自然應的為佛出力,總不然儹到那箇身上去?因此上貧僧發了這個念頭,前日老檀越餞,行各位老爹的時,悲怜本寺廢壞,也有個良心美腹,要和本寺作主。那時諸佛菩薩,已作證盟。貧僧記的佛經上說的好:『如有世間善男子,善女人,以金錢喜捨莊麗佛像者,主得桂子蘭孫,端麗美貌,日後早登科甲,蔭子封妻之報。』故此特叩高門,不拘五百一千,要求老檀那開疏發心,成就善果。」就把錦帊展開,取出那募緣疏簿,雙手遞上。不想那一席話兒,早已把西門慶的心兒打動了。不覺的歡天喜地,接了疏簿,就叫小廝看茶。揭開疏簿,只見寫道:

「伏以白馬駝經開象教,竺騰衍法啟宗門。大地眾生,無不皈依佛祖;三千世界,盡皆蘭若裝麗。看此瓦礫傾頹,成甚名山勝境?若不慈悲喜捨,何稱佛子款人?今有永福禪寺古佛道場,焚修福地。啟建自梁武皇帝,開山是萬迴祖師。規制恢弘,彷彿那給孤園黃金鋪地;雕鏤精製,依希似祇洹舍白玉為階。高閣摩空,旃檀氣直接九霄雲表;層基亙地,大雄殿可容千眾禪僧。兩翼嵬峨,盡是琳宮紺宇;廊房潔淨,果然精勝洞天。那時鐘鼓宣揚,盡道是寰中佛國;只這緇流濟楚,卻也像塵界人天。那知歲久年深,一瞬地時移事異。莽和尚縱酒撒潑,首壞清規;獃道人懶惰貪眠,不行打掃。漸成寂寞,斷絕門徒。以致凄涼,罕稀瞻仰。兼以烏鼠穿蝕,那堪風雨漂搖?棟宇摧頹,一而二,二而三,支撐摩計,墻垣柵塌,日復日,年復年,振起無人。朱紅櫺槅,拾來煨酒煨茶;合抱梁檻,拿去換鹽換米。風吹羅漢金消盡,雨打彌陀化作塵。吁嗟乎金碧焜炫,一旦為灌莽榛荊。雖然有成有敗,終須否極泰來。幸而有道長老之虔誠,不忍見梵王宮之費敗。發大弘願,遍叩檀那。伏願咸起慈悲,盡興惻隱。梁柱椽楹,不拘大小,喜捨到高題姓字;銀錢布幣,豈論豐嬴,投櫃日疏簿標名。仰仗著佛祖威靈,福、祿、壽、永永百年千載;倚靠他伽藍明鏡,父子孫個個原祿高官。瓜瓞綿綿,森挺三槐五桂;門庭奕奕,焜煌金埒錢山。凡所營求,吉祥如意。疏文到日,各破慳心,謹疏。」

看畢,西門慶就冊葉兒收好,粧入那錦套裏頭。把插銷兒銷,錦帶兒拴著,恭恭敬敬放在卓兒上面,叉手面言,對長老說:「實不相瞞,在下雖不成個人家,也有幾萬產業,忝居武職,交遊世輩儘有。不想偌大年紀,未曾生下兒子。房下們也有五六房,只是放心不下,有意做些善果。去年第六房賤累,生下孩子。咱萬事已是足了。偶因餞選俺友,得到上方。因見廟宇傾頹,有個捨才助建的念頭。蒙老師下顧,西門慶那敢推辭?」拏著兔毫妙筆,正在躊躇之際,那應伯爵就說:「哥,你既有這片好心為姪兒發愿,何不一力獨成,也是小可的事體!」西門慶拏著筆,哈哈哩笑道:「力薄!力薄!」伯爵又道:「極少也助一千。」西門慶又哈哈地笑道:「力薄!力薄!」那長老就開口說道:「老檀越在上,不是貧僧多口,止是我們佛家的行徑,多要隨緣喜捨,終不強人所難。隨分但憑老爹發心便是。此外親友,更求檀越,吹噓吹噓。」西門慶又說道:「還是老師體亮,少也不成。」就寫上五百兩,閣了兔毫筆。那長老打個問訊謝了。西門慶又說:「我這里內官太監,府縣倉巡,一個個多與我相好的。我明日就拿疏簿去,要他們寫。寫的來,就不拘三百、二百、一百、五十,管教與老師成就這件好事。」當日留了長老素齋,相送出門。正是:

「慈悲作豪家事,  保福消災父母心。」

又有一首詞,單道那有施主的事體:

「佛法無多止在心,  種瓜種果是根因,

珠和玉珀寶和珍,  誰人拏得見閻君?

積善之人貧也好,  豪家積業枉拋銀,

若使年齡身可買,  董卓還應活到今。」

卻說西門慶送了長老,轉到廳上,與應伯爵坐地,道:「二哥,我正要差人請你,你來的正好。我前日因往西京,多虧眾親友們與咱把個盞兒。今日分付小的買辦,你家大嫂安排小酒與眾人回答,要哥在此相陪。不想遇著這個長老,鬼混了一會兒。」那伯爵就說道:「好個長老,想是果然有德性的。他說話中間,連咱也心動起來,做了施主。」西門慶說道:「二哥,你又幾曾做施主來的?疏簿又是幾時寫的?」應伯笑道:「咦!難道我出口的不是施主不成?哥,你也不曾見佛經過來?佛經上第一重的是心施,第二法施,第三才是財施。難道我從傍攛掇的,不當個心施的不成?」西門慶又笑道:「二哥,又怕你有口無心哩!」兩人拍手大笑。應伯爵就說:「小弟在此等待客來。哥有正事,自與嫂子商議去來。」只見西門慶別了伯爵,轉到內院裏頭。只見那潘金蓮嘮嘮唔唔,沒揪沒採,不覺的睡魔纏擾,打了幾個噴〈口弟〉,走到象牙床上,一忽地睡去了。那李瓶兒又為孩子啼哭,自與那奶子、丫鬟在房中坐地看官哥喜笑。只有那吳月娘與孫雪蛾兩個伴當在那里整辦嗄飯。西門慶走到面前坐地,就把那道長老募緣與那自己開疏的事,備細對月娘說了一番。又把那應伯爵耎笑打覷的說話,也說了一番。歡天喜地,大家嘻笑了一會。只見那吳月娘,畢竟是個正經的人,不慌不忙,不思不想,說下幾句話兒,到是西門慶頂門上針。正是:

「妻賢每致雞鳴警,  款與常聞藥石言。」

畢竟那說話怎麼講?月娘說道:「哥,你天大的造化,生下孩兒。你又發起善念,廣結良緣。豈不是俺一家兒的福分?只是那善念頭他怕不多,那惡念頭怕他不盡。哥,你日後那沒來回,沒正經,養婆兒,沒搭煞,貪財好色的事體,少幹幾樁兒也好。儹下些陰功與那小的子也好。」西門慶笑:「娘,你的醋話兒又來了。卻不道天地尚有陰陽,男女自然配合。今生偷情的、苟合的,都是前生分定,姻緣簿上註名,今生了還。難道是生刺刺搊搊胡扯歪斯纏做的?咱聞那佛祖西天,也止不過要黃金舖地。陰司十殿,也要些楮鏹營求。咱只消儘這家私,廣為善事,就使強姦了常娥,和姦了織女,拐了許飛瓊,盜了西王母的女兒,也不減我潑天富貴!」月娘笑道:「笑哥狗吃熱屎,原道是個香甜的,生血弔在牙兒內,怎生改得?」正在笑間,只見那王姑子同了薛姑子提一個合子,直闖進來。飛也似朝月娘道個萬福,又向西門慶拜拜了說:「老爹,你到在家里?我自前日別了,因為有些小事,不得空,不曾來看得你老人家,心子裏吊不下。今日同這薛姑子來看你!」原來這薛姑子,不是從幼出家的。少年間曾嫁丈夫,在廣成寺前居住,賣蒸餅兒生理。不料生意淺薄,那薛姑子就有些不尷不尬,專一與那些寺里的和尚行童調嘴弄舌,眉來眼去,說長說短。弄的那些和尚們的懷中,個個是硬幫幫的。乘那丈夫出去了。茶前酒後,早與那和尚們刮上了四五六個。也常有那火燒 、波波 、饅頭、栗子,拿來進奉他。又有那付應錢,與他買花。開地獄的布,送與他做裹腳。他丈夫那里曉得?以後丈夫得病死了,他因佛門情熟,這等就做了個姑子,專一在些士夫人家往來,包攬經讖。又有那些不長進要偷漢子的婦人,叫他牽引和尚進門,他就做個馬八六兒,多得錢鈔。聞的那西門慶家里豪富,見他侍妾多,又思想拐些用度,因此頻頻往來。那西門慶也不曉的,三姑六婆人家最忌出入。正是:

「當年行經是窠兒,和尚闍黎舖。中間打扮念彌陀,開口兒就說西方路。尺布裹頭顱,身穿直裰,繫個黃縧,早晚捱門傍戶。騙金銀猶是叮心窩裏,畢竟胡塗。算來不是好姑姑,幾個清名被點污。」

又有一隻歌兒道得好:

「尼姑生來頭皮光,拖子和尚夜夜忙。三個光頭,好像師父、師兄并師弟,只是鐃鈸緣何在里床?」

那薛姑子坐就把那個小合兒揭開,說道:「咱們沒有什麼孝順,拏得施主人家幾個供佛的菓子兒,權當獻新。」月娘道:「要來竟來來便了,何苦要你費心?」只見那潘金蓮睡覺,聽得外邊有人說話,又認是前番光景,便走向前來聽看。見那李瓶兒在房中弄孩子,因曉得王姑子在此,也要與他商議保佑官晉,同到月娘房中,大家道個萬福,各各坐地。西門慶因見李瓶兒不曾曉的,又把那道長老募緣,與那自家開疏捨財,替官哥求福的事情,重新又說一番。不想道惱了潘金蓮抽身竟走,喃喃噥噥,一溜煙竟自去了。只見那薛姑子站將起來,合掌著手,叫聲:「佛阿!老爹,你這等樣好心作福,怕不的壽年千歲,五男二女,七子團圓。只是我還有一件,說與你老人家,這個因果費什麼多?更自獲福無量咦!老檀越,你若幹了這件功德,就是那老瞿曇雪山修道,迦葉尊散髮鋪地,二祖可投崖飼虎,給孤老滿地黃金,也比不的你功德哩!」西門慶笑道:「姑姑且坐下,細說甚麼功果?我便依你。」那薛姑子就說:「我們佛祖留下一卷陀羅經,專一勸人法西方淨土的。佛說:那三禪天、四禪天、切利天、兜率天、大羅天、不周天,急切不能即到。唯有西方極樂世界,這是阿彌陀佛出身所在。沒有那春夏秋冬,也沒有那風寒暑熱,常常如三春時侯融合天氣。也沒有夫婦男女,其人生在七寶池中,金蓮臺上。」西門慶道:「那一朵蓮花有幾多大?生在上邊,一陣風擺,怕不骨碌碌吊在池里麼?」薛姑子道:「老爹你還不曉的,我依那經上說。佛家以五百里為一由旬。那一朵蓮花好生利害,大的緊,大的緊,大的五百由旬。寶衣隨願至,玉食自天來。又有那些好鳥和鳴,如笙簧一般。委的好個境界!因為那肉眼凡夫,不知去向,不生尊信,故此佛祖演說此經,勸人專心念佛,竟往西方見了阿彌陀佛。自此一世、二世,以至百千萬世,永永不落輪迴。那佛祖說的好:『如有人持頌此經,或將此經印刷抄寫,轉勸一人,至千萬人持誦,獲福無量。』況且此經裏面,又有獲諸童子經咒。凡有人家生育男女,必要從此發心,方得易長易養,災去福來。如今這付經板現在只沒人印刷施行。老爹你只消破些工料,印上幾千卷,裝釘完成,普施十方,那個功德,真是大的緊!」西門慶道:「也不難。只不知這一卷經,要多少布札?多少裝釘工夫?多少印刷?有個細數,纔好動彈。」薛姑子又道:「老爹你一發呆了,說那里話去細細等將起來?止消先付九兩銀子,交付那經坊裏,要他印造幾千幾萬卷,裝釘完滿,以後一攪果算還他工食布札錢兒就是了。卻怎地要細細算將出來?」正說的熱鬧,只見那陳經濟要與西門慶說話,跟尋了好一回不見。問那玳安,說在月娘房里。走到捲棚底下,剛剛湊巧遇著了那潘金蓮凭闌獨笑猛然抬起頭來,見了經濟,就是個貓兒見了魚鮮飯,一心心要啖他下去了。不覺的把一天愁悶,多改做春風和氣。兩個乘著沒有人來,執手相偎,做剝嘴咂舌頭。兩下肉麻,好生兒頑了一回兒。因恐怕西門慶出來撞見,連那算帳的事情也不吆呼,兩雙眼又像老鼠兒見了貓來,左顧右盼提防著,又沒個方便,一溜煙自出去了。且說西門慶聽罷了薛姑子的話頭,不覺心上打動了一片善念。就聽玳安取出拜匣,把汗巾上的小匙鑰兒開了,取出一封銀子,准准三十兩足色松紋,便交付薛姑子與那王姑子:「即便同去,隨分那里經坊,與我印下五千卷經。待完了,我就算帳找他。」正話間,只見那書童忙忙的來報道:「請的各位客人多到了。」少不的是吳大舅、花二舅、謝希大、常時節這一班,多各齊齊整整一齊到。西門慶忙的不迭,即便整衣出外迎接升堂。就叫小廝擺下卓兒,放下小菜兒。請吳大舅上坐了,眾人一行兒分班列次,各敘長幼,各各坐地。那些腌臘、煎熬、大魚大肉、燒雞燒鴨 、時鮮菓品,一齊兒多捧將出來。西門慶又叫道:「開那麻菇酒兒盪來。」只見酒逢知己,形跡多忘。猜枚的、打鼓的、催花的、三拳兩謊的,歌的歌,唱的唱,談風月,盡道是杜工部、賀黃門乘春賞翫;掉文袋,也曉的蘇玉局,黃魯直,赤壁清遊。投壺的定要那正雙飛、拗雙飛、八仙過海;擲色的又要那正馬軍、拗馬軍、鰍入菱窠輸酒的要喝個無滴,不怕你玉山頹倒,嬴色的又要去掛紅,誰讓你倒著接罹。頑不盡少年場光景,說不了醉鄉裏日月。正是:

「秋月春花隨處有,  賞心樂事此時同,

百年若不千場醉,  碌碌營營總是空。」

畢竟未知後來何如,且聽下回分解:

