%# -*- coding: utf-8 -*-
% !TeX encoding = UTF-8 Unicode
% !TeX spellcheck = en_US
% !TeX TS-program = xelatex
%~ \XeTeXinputencoding "UTF-8"
% vim:ts=4:sw=4
%
% 以上設定默認使用 XeLaTex 編譯,並指定 Unicode 編碼,供 TeXShop 自動識別

\chapter*{金瓶梅詞話序}
\addcontentsline{toc}{chapter}{金瓶梅詞話序 -- 欣欣子}

\begin{showcontents}{}

%\zihao{-4}\setlength\parskip{\baselineskip-\ccwd}

竊謂蘭陵笑笑生作《金瓶梅傳》,寄意於時
俗,蓋有謂也。人有七情,憂鬱為甚。上智之士,
與化俱生,霧散而冰裂,是故不必言矣。次焉者,
亦知以理自排,不使為累。惟下焉者,既不能了
扵心胸,又無詩書道腴可以撥遣,然則不致扵坐
病者幾希!吾友笑笑生為此,爰罄平日所蘊者,
著斯傳,凡一百囬。其中語句新奇,膾炙人口。
無非明人倫、戒淫奔、分淑慝、化善惡,知盛衰
消長之機,取報應輪廻之事,如在目前;始終如
脈絡貫通,如萬絲迎風而不亂也。使觀者庶幾可
以一哂而忘憂也。其中未免語涉俚俗,氣含脂
粉。余則曰:不然。《關雎》之作,樂而不淫,
哀而不傷。富與貴,人之所慕也,鮮有不至于淫
者;哀與怨,人之所惡也,鮮有不至于傷者。吾
嘗觀前代騷人,如盧景暉之《剪燈新話》、
元微之
\jiaoYL{元微之,原作元徽之}
之《鶯鶯傳》、
趙君弼之《效顰集》
、羅貫中之《水滸傳》
、丘瓊山之《鍾情麗集》
、盧梅湖之《懷春雅集》
、周靜軒之《秉燭清談》
,其後《如意傳》、《于湖記》,
其間語句文確,讀者往往不能暢懷,
不至終篇而掩棄之矣。此一傳者,雖市井之常
談,閨房之碎語,使三尺童子聞之,如飫天漿而
拔鯨牙,洞洞然易曉。雖不比古之集理趣,文墨
綽有可觀。其他關係世道風化,懲戒善惡,滌慮
洗心,不無小補。譬如房中之事,人皆好之人皆惡之人
非堯舜聖賢,鮮不為所耽。富貴善良,
是以搖動人心,蕩其素志。觀其高堂大廈,雲窗
霧閣,何深沉也;金屏綉褥,何羙麗也;鬢雲斜
軃,春酥滿胸,何嬋娟也;雄鳳雌凰迭舞,何慇
懃也;錦衣玉食,何侈費也;佳人才子嘲風咏月,
何綢繆也;鷄舌含香,唾圓流玉,何溢度也;一
雙玉腕綰復綰,兩隻金蓮顛倒顛,何猛浪也。旣
其樂矣,然楽極必悲生:如離别之機將興,憔悴
之容必見者,所不能免也;折梅逢驛使,尺素寄
魚書,所不能無也;患難迫切之中,顛沛流離之
頃,所不能脱也;陷命於刀劍,所不能逃也;陽
有王法,幽有鬼神,所不能逭也。至于淫人妻子,
妻子淫人,禍因惡積,福緣善慶,種種皆不出循
環之機。故天有春夏秋冬,人有悲歡離合,莫怪
其然也。合天時者,逺則子孫悠久,近則安享終
身;逆天時者,身名罹喪,禍不旋踵。人之處世,
雖不出乎世運代謝,然不經凶禍,不蒙耻辱者,
亦幸矣。吾故曰:笑笑生作此傳者,蓋有所謂也。

{\bigskip\mbox{}\fzqiti\large\hfill 欣欣子書于明賢里之軒 \quad }






\chapter*{金瓶梅序}
\addcontentsline{toc}{chapter}{金瓶梅序 -- 東吴弄珠客}

《金瓶梅》,穢書也。袁石公亟稱之,亦自
寄其牢騷耳,非有取扵《金瓶梅》也。然作者亦
自有意,蓋為世戒,非為世勸也。如諸婦多矣,
而獨以潘金蓮、李瓶児、春梅命名者,亦楚《檮
杌》之意也。蓋金蓮以姦死,瓶児以孽死,春梅
以淫死,較諸婦為更慘耳。借西門慶以描畫世之
大凈,應伯爵以描畫世之小丑(醜),諸淫婦以描畫世
之丑(醜)婆、凈婆,令人讀之汗下。蓋為世戒,非為
世勸也。

余嘗曰:讀《金瓶梅》而生憐憫心者,菩薩也;
生畏懼心者,君子也;
生歡喜心者,小人也;
生效法(灋)心者,乃禽獸耳。
余友人褚孝秀,偕一少年同赴歌舞之筵,
衍至「霸王夜宴」,少年垂涎曰:「男兒何可不如此!」
孝秀曰:「也只為這烏江設此一着耳。」
同座聞之,歎為有道之言。
若有人識得此意,方許他讀《金瓶梅》也。
不然,石公幾為導淫宣慾之尤矣!
奉勸世人,勿為西門之後車,可也。

{\bigskip\mbox{}\fzqiti\large\hfill 萬曆丁巳季冬東吴弄珠客漫書扵金閶道中 \quad}







\chapter*{跋}
\addcontentsline{toc}{chapter}{跋 -- 廿公}

《金瓶梅》,傳為世廟時一鉅公寓言,蓋有所刺也。
然曲盡人間醜態,其亦先師不删鄭、衛之旨乎?
中間䖏處埋伏因果,作者亦大慈悲矣。
今後流行此書,功德無量矣。不知者竟目為淫書,
不惟不知作者之旨,併亦寃却流行者之心矣!
特為白之。

{\bigskip\mbox{}\fzqiti\large\hfill 廿公書 \quad}





\chapter*{新刻金瓶梅詞話}
\addcontentsline{toc}{chapter}{新刻金瓶梅詞話}

詞曰\zhuYL{詞曰,以下四阕词,前三阕见于《词林纪事》卷二十二,元中峰禅师所撰。原文为 ...}

閬苑\zhuYL{閬苑 -- 閬风之苑。又称玄圃,传说在昆仑之巅,仙人居处。屈原《离骚》: ....}
瀛洲,\zhuYL{瀛洲 -- 传说中的海上仙山《史记 秦始皇本纪》齐人徐市等上书,言海中有三神山,名曰蓬莱、方丈、瀛洲,仙人居之。}
金谷瓊樓, 算不如茅舍清幽。
野花綉地,莫也風流。
也宜春,也宜夏,也宜秋。
酒熟堪篘,%\jiaoMINE{篘, 欽定四庫全書-禦選曆代詩-卷一百十九餘}
客至須留,更無榮無辱無憂。
退閒一步,着甚來由。
但倦時眠,渴時飲,醉時謳。

短短橫墻,矮矮疏窗,忔(忄查)児小小池塘。
高低疊嶂,綠水邊傍。也有些風,有些月,有些凉。
日用家常,竹几藤牀,據眼前水色山光。
客來無酒,清話何妨。但細烹茶,熱烘盞,淺澆湯。

水竹之居,吾愛吾盧,石磷磷裝砌階除。
軒窗隨意,小巧規模。却也清幽,也瀟灑,也寬舒。
懶散無拘,此樂何如,倚闌干臨水觀魚。
風花雪月,贏得功夫。好炷些香,説些話,讀些書。

淨掃塵埃,惜取蒼苔,任門前紅葉鋪階。
也堪圖畫,還也奇哉。有數株松,數竿竹,數枝梅。
花木栽培,取次教開,明朝事天自安排。
知他富貴幾時來。且優遊,且隨分,且開懷。



四貪詞

酒

酒損精神破喪家,語言無狀鬧喧嘩。

疏親慢友多由你,背義忘恩盡是他。

切須戒,飲流霞。若能依此實無差。

失却萬事皆因此,今後逢賓只待茶。

色

休愛綠髩美朱顔,少貪紅粉翠花鈿。

損身害命多嬌態,傾國傾城色更鮮。

莫戀此,養丹田。人能寡慾壽長年。

從今罷却閑風月,紙帳梅花獨自眠。

財

錢帛金珠籠内收,若非公道少貪求。

親朋道義因財失,父子懷情為利休。

急縮手,且抽頭。免使身心晝夜愁。

児孫自有児孫福,莫與児孫作遠憂。

氣

莫使強梁逞技能,揎拳捋袖弄精神。

一時怒發無明火,
\jiaoYL{無明火, 「火」原作「穴」,据七十五回「心头一点無明火」等语改。}
到後憂煎禍及身。

莫太過,免災迍。勸君凡事放寬情。

合撒手時須撒手,得饒人處且饒人。



\end{showcontents}

