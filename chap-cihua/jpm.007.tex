%# -*- coding: utf-8 -*-
%!TEX encoding = UTF-8 Unicode
%!TEX TS-program = xelatex
% vim:ts=4:sw=4
%
% 以上设定默认使用 XeLaTex 编译,并指定 Unicode 编码,供 TeXShop 自动识别

%第七回 
\chapter{薛嫂兒說娶孟玉樓\KG 楊姑娘氣罵張四舅}

\begin{showcontents}{}



「我做媒人實可能,  全憑兩腿走慇懃,

唇鎗慣把鰥易配,  舌劍能調烈女心;

利市花紅頭上帶,  喜筵餅錠袖中撑,

只有一件不堪處,  半是成人半敗人。」

話說西門慶家中,賞翠花兒的薛嫂兒,提着花廂兒,一地哩尋西門慶不着。因見西門慶使的小廝玳安兒,問:「大官人在那裡?」玳安道:「俺爹在舖子裡,和傅二叔筭帳。」原來西門慶家開生藥舖,主管姓傅名銘字自新,排行第二,因此呼他做傅二叔。這薛嫂一直走到舖子門首,掀開簾子,見西門慶正在裡面與主管筭帳。一面點首兒,喚他出來。這西門慶見是薛嫂兒,連忙撇了主管出來,兩人走在僻靜處說話。薛嫂道了萬福,西門慶問他:「有甚說話?」薛嫂道:「我有一件親事,來對大官人說,管情中得你老人家意,就頂死了的三娘窩兒。方纔我在大娘房裡,買我的花翠,留我吃茶,坐了這一日,我就不曾敢題起。逕來尋你老人家,和你說。這位娘子,說起來你老人家也知道,是咱這南門外販布楊家的正頭娘子。手裡有一分好錢,南京拔步床也有兩張,四季衣服,粧花袍兒,插不下手去,也有四五隻廂子。珠子箍兒,胡珠環子,金寶石頭面,金鐲銀釧不消說;手裡現銀子,他也有上千兩。好三梭布,也有三二伯筩。不幸他男子漢去販布,死在外邊。他守寡了一年多,身邊又沒子女,止有一個小叔兒還小,纔十歲,青春年少,守他甚麼?有他家一個嫡親姑娘,要主張着他嫁人。這娘子今年不上二十五六歲,生的長挑身材,一表人物。打扮起來,就是個燈人兒,風流俊俏,百伶百俐。當家立紀,針指女工,雙陸棋子,不消說。不瞞大官人說,他娘姓孟,排行三姐,就住在臭水巷。又會彈了一手好月琴。大官人若見了,管情一箭就上垛;誰似你老人家有福,好得這許多帶頭,又得了一個娘子!」西門慶只聽見婦人會彈月琴,便可在他心上。就問薛嫂兒:「幾時相會看去?」薛嫂道:「我和老人家這等計議,相看不打緊。如今他家一家子,只是姑娘大。雖是他娘舅張四,山核桃差着一槅兒哩!這婆子原嫁與北邊半邊街徐公公房子裡住的孫歪頭,歪頭死了,這婆子守寡了三四十年,男花女花都無,只靠姪男姪女養活。今日已過,明日我來會大官人,咱只倒在身上求他;求只求張良,拜只拜韓信。這婆子愛的是錢財,明知道他姪兒媳婦有東西,隨問什麼人家,他也不管,只指望要幾兩銀子。大官人多許他幾兩銀子,家裡有的是那囂段子,拏上一段,買上一擔禮物,親去見他,和他講過,一拳打倒他。隨問傍邊有人說話,這婆子一力張主,誰敢怎的?」這薛嫂兒一席話,說的西門慶歡從額角眉尖出,喜向腮邊笑臉生。看官聽說:世上這媒人們,原來只一味圖撰錢,不顧人死活。無官的說做有官,把偏房說做正房。一味瞞天大謊,全無半點兒真實。正是:

「媒妁慇懃說始終,  孟姬愛嫁富家翁;

有緣千里能相會,  無緣對面不相逢。」

西門慶當日與薛嫂相約下,明日是好日期,就買禮往北邊他姑娘家去。薛嫂說畢話,提着花廂兒去了。西門慶進來,和傅夥計筭帳,一宿晚景不題。到次日,西門慶早起,打選衣帽齊整,拏了一段尺頭,買了四盤羹果,僱了一個抬盒的,薛嫂領着,西門慶騎着頭口,小廝跟隨,逕來北邊半邊街徐公公房子裡楊姑娘家門首。薛嫂先入去,通報姑娘得知,說:「近邊一個財主,敬來門外和大娘子說親。我說一家只姑奶奶是大,先來覿面,親見過你老人家,講了話,然後纔敢領去門外相看。今日小媳婦領來,見在門首下馬伺候。」婆子聽見,便道:「阿呀,保山!你如何不先來說聲?」一面吩咐了丫鬟,打掃客位收拾乾淨,頓下好茶;一面道:「有請!」這薛嫂一力攛掇,先把盒擔抬進去擺下。打發空盒擔兒出去,就請西門慶進來入見。這西門慶頭戴纏棕大帽,一撒鈎縧粉底皂靴,進門見婆子,拜四拜。婆子柱着枴,慌忙還下禮去。西門慶那裡肯,一口一聲,只叫:「姑娘請受禮!」讓了半日,婆子受了半禮,分賓主坐下,薛嫂在傍打橫。婆子便道:「大官人貴姓?」薛嫂道:「我纔對你老人家說,就忘了!便是咱清河縣數一數二的財主,西門慶大官人。在縣前開着個大生藥舖,又放官吏債,家中錢過北斗,米爛陳倉。沒個當家立紀娘子,聞得咱家門外大娘子要嫁,特來見姑奶奶,講說親事。」因說:「你兩親家都在此,漏眼不藏絲,有話當面說,省得俺媒人們架謊。這裡是姑奶奶大人,有話不先來和姑奶奶說,再和誰說?」婆子道:「官人倘然要說俺侄兒媳婦,自恁來閑講便了,何必費煩,又買禮來,使老身卻之不恭,受之有愧!」西門慶道:「姑娘在上,沒的禮物惶恐!」那婆子一面拜了兩拜,謝了,收過禮物去。薛嫂馱盤子出門,一面走來陪坐,拏茶上來,吃畢。婆子開口說道:「老身當言不言,謂之懦;我姪兒在時,做人掙了一分錢,不幸死了。如今多落在他手裡,少說也有上千兩銀子東西。官人做大做小,我不管你,只要與我姪兒念上個好經,老身便是他親姑娘,又不隔從,就與上我一個棺材本,也不曾要了你家的。我破着老臉,和張四那老狗做臭毛鼠,替你兩個硬張主。娶過門時,生辰貴長,官人放他來走走,就認俺這門窮親戚,也不過上你窮。」西門慶笑道:「你老人家放心,適間所言的話,我小人都知道了。你老人家既開口,休說一個棺材本,就是十個棺材本,小人也來得起!」說着,向靴桶裡取出六錠三十兩雪花官銀,放在面前,說道:「這個不當甚麼,先與你老人家買盞茶吃。到明日娶過門時,還找七十兩銀子、兩疋段子,與你老人家為送終之資。其四時八節,只照頭上門行走。」看官聽說:世上錢財,乃是眾生腦髓,最能動人。這老虔婆黑眼睛珠,見了二三十兩白晃晃的官銀,滿面堆下笑來,說道:「官人在上,不當老身意小。自古先說斷,後不亂。」薛嫂在傍插口說:「你老人家忒多心,那裡這等計較!我的大老爹不是那等人,自恁還要掇着盒兒認親,你老人家不知,如今知府、知縣相公來往,好不四海,結識人寬廣。你老人家能吃他多少?」一席話,說得婆子屁滾尿流,陪的坐吃了兩道茶。西門慶便要起身,婆子挽留不住。薛嫂道:「今日既見了姑奶奶說過話,明日好往門外相看。」婆子道:「我家姪兒媳婦,不用大官人相。保山,你就說我說,不嫁這樣人家,再嫁甚樣人家?」西門慶作辭起身,婆子道:「官人,老身不知官人下降,匆忙不曾預備,空了官人,休怪。」柱拐送出,送了兩步,西門慶讓回去了。薛嫂打發西門慶上馬,便說道:「還虧我主張有理麼?寧可先在婆子身上倒,還強如別人說多。」因說道:「你老人家先回去罷,我還在這裡和他說句話,咱已是會過,明日先往門外去了。」西門慶便拏出一兩銀子來,與薛嫂做驢子錢,薛嫂接了。西門慶便上馬來家。他便還在楊姑娘家說話飲酒,到日暮時分纔歸家去。話休饒舌,到次日,打選衣帽齊整,袖着插戴,騎着大白馬,玳安、平安兩個小廝跟隨,薛嫂兒便騎驢子,出的南門外來,到豬市街,到了楊家門首。原來門面屋四間,到底五層,西門慶勒馬在門首等候。薛嫂先入去半日,西門慶下馬。坐南朝北一間門樓,粉青照壁;進去裡面儀門紫牆,竹搶籬影壁。院內擺設榴樹盆景,臺基上靛缸一溜,打布凳兩條。薛嫂推開朱紅槅扇三間,倒坐客位。正面上供養着一軸水月觀音、善財童子。四面掛名人山水,大理石屏風安着兩座投箭高壺。上下椅卓光鮮,簾櫳瀟灑。薛嫂請西門慶正面椅子上坐了,一面走入裡邊。片晌出來,向西門慶耳邊說:「大娘子梳粧未了,你老人家請坐一坐。」只見一個小廝兒,拿出一盞福仁泡茶 來,西門慶吃了,收下盞托去。這薛嫂兒倒還是媒人家,一面指手畫腳,與西門慶說:「這家中除了那頭姑娘,只這位娘子是大。誰有他小叔,還小哩,不曉的什麼。當初有過世的他老公,在舖子裡,一日不筭銀子,搭錢兩大菠羅。毛青鞋面布,俺每問他買,定要三分一尺;見一日常有二三十染的吃飯,都是這位娘子主張整理。手下使着兩個丫頭、一個小廝。長了十五歲,吊起頭去,名喚蘭香;小丫頭纔十二歲,名喚小鸞,到了明日過門時,都跟他來。我替你老人家說成這親事,指望典兩間房兒住,強如住在北邊那搭剌子哩,往宅裡去不方便。你老人家去年買春梅,許了我幾疋大布,還沒與我,到明日不管一總謝罷了。」又道:「剛纔你老人家看見門首那兩座布架子,當初楊大叔在時,街道上不知使了多少錢;這房子也值七八百兩銀子,到底五層,通後街,到明日丟與小叔罷了。」正說着,只見使了個丫頭來叫薛嫂。良久,只聞環珮叮咚,蘭麝馥郁,婦人出來。上穿翠藍麒麟補子粧花紗衫,大紅粧花寬欄。頭上珠翠堆盈,鳳釵半卸。西門慶掙眼觀看那婦人,但見:

「長挑身材,粉粧玉琢;模樣兒不肥不瘦,身段兒不短不長。面上稀稀有幾點微麻,生的天然俏麗;裙下映一對金蓮小腳,果然周正堪憐。二珠金環,耳邊低挂;雙頭鸞釵,鬢後斜插。但行動,胸前搖響玉玲瓏;坐下時,一陣麝蘭香噴鼻。恰似嫦娥離月殿,猶如神女下瑤階。」

西門慶一見,滿心歡喜。薛嫂忙去掀簾子,婦人出來,望上不端不正,道了個萬福,就在對面椅上坐下。西門慶把眼上下不轉睛看了一回,婦人把頭低了。西門慶開言說:「小人妻亡已久,欲娶娘子入門為正,管理家事。未知意下如何?」那婦人道:「官人貴庚?沒了娘子多少時了?」西門慶道:「小人虛度二十八歲,七月二十八日子時建生。不幸先妻沒了一年有餘。不敢請問娘子青春多少?」婦人道:「奴家青春是三十。」西門慶道:「原來長我二歲。」薛嫂在傍插口道:「妻大兩,黃金日日長;妻大三,黃金積如山。」說着,只見小丫鬟拏了三盞蜜餞金橙子泡茶 ,銀鑲雕漆茶鍾,銀杏葉茶匙。婦人起身,先取頭一盞,用纖手抹去盞邊水漬,遞與西門慶;忙用手接了,道了萬福。慌的薛嫂向前用手掀起婦人裙子來,裙邊露出一對剛三寸恰半扠,一對尖尖趫趫金蓮腳來,穿着大紅遍地金雲頭白綾高底鞋兒,與西門慶瞧,西門慶滿心歡喜。婦人取第二盞茶來,遞與薛嫂;他自取一盞陪坐。吃了茶,西門慶便叫玳安用方盒呈上錦帕二方、寶釵一對、金戒指六個,放在托盤內拿下去。薛嫂一面教婦人拜謝了,因問官人行禮日期,奴這裡好做預備。西門慶道:「既蒙娘子見允,今月二十四日,有些微禮過門來,六月初二日准娶。」婦人道:「既然如此,奴明日就使人來對北邊姑娘那裡說去。」薛嫂道:「大官人昨日已是到姑奶奶府上講過話了。」婦人道:「姑娘說甚來?」薛嫂道:「姑奶奶聽見大官人說此樁事,好不歡喜,纔使我領大官人來這裡相見。說道:『不嫁這等人家,再嫁那樣人家?我就做硬主媒,保這門親事。』」婦人道:「既是姑娘恁的說,又好了!」薛嫂道:「好大娘子,莫不俺做媒,敢這等搗謊!」說畢,西門慶作辭起身。薛嫂送出巷口,向西門慶說道:「看了這娘子,你老人家心下如何?」西門慶道:「薛嫂,其實累了你!」薛嫂道:「你老人家請先行一步,我和大娘子說句話就來。」西門慶騎馬進城去了。薛嫂轉來向婦人說道:「娘子,你嫁得這位老公也罷了。」因問:「西門慶房裡有人沒有人?見作何生理?」薛嫂道:「好奶奶,就有房裡人,那箇是成頭腦的!我說是謊,你過去就看出來。他老人家名目,誰是不知道的!清河縣數一數二的財主,有名賣生藥放官吏債西門大官人。知縣、知府都和他往來,近日又與東京楊提督結親,都是四門親家,誰人敢惹他?」婦人安排酒飯,與薛嫂兒正吃着,只見他姑娘家使了小廝安童,盒子裡跨着鄉裡來的四塊黃米麵棗兒糕 、兩塊糖、幾個艾窩窩 ,就來問:「曾受了那人家插定不曾?奶奶說來,這人家不嫁,待嫁甚人家?」婦人道:「多謝你奶奶掛心,今日已留下插定了。」薛嫂道:「天麼,天麼!早是俺媒人不說謊!姑奶奶家使了大官兒說將來了!」婦人收了糕,出了盒子,裝了滿滿一盒子點心臘肉,又與了安僮五六十文錢:「到家多拜上奶奶。那家日子,定下二十四日行禮,出月初二日准娶。」小廝去了。薛嫂道:「姑奶奶家送來什麼?與我些包了家去,稍與孩子吃。」婦人與了他一塊糖、十個艾窩窩,千恩萬謝出門,不在話下。且說他母舅張四,倚着他小外甥楊宗保,要圖留婦人手裡東西,一心舉保與大街坊尚推官兒子尚舉人為繼室。若小可人家,還可有話說;不想聞得是縣前開生藥舖西門慶定了,他是把持官府的人,遂動不得秤了。尋思已久,千方百計,不如破他為上計。走來對婦人說:「娘子,不該接西門慶插定。還依我嫁尚推官兒子尚舉人,他又是斯文詩禮人家,又有庄田地土,頗過得日子,強如嫁西門慶。那廝積年把持官府,刁徒潑皮。他家見有正頭娘子,乃是吳千戶家女兒。過去做大是做小?都不難為你了?況他房裡又有三四個老婆,併沒上頭的丫頭。到他家人多口多,你惹氣也!」婦人道:「自古船多不礙路。若他家有大娘子,我情愿讓他做姐姐,奴做妹子。雖然房裡人多,漢子歡喜,那時難道你阻他?漢子若不歡喜,那時難道你去扯他?不怕一百人單擢着,休說他富貴人家,那家沒四五個?着緊街上乞食的,携男抱女,也挈扯着三四個妻小。你老人家忒多慮了,奴過去自有個道理,不妨事!」張四道:「娘子,我聞得此人,單管挑販人口,慣打婦熬妻,稍不中意,就令媒人賣了,你愿受他的這氣麼?」婦人道:「四舅,你老人家差矣!男子漢雖利害,不打那勤謹省事之妻;我在他家,把得家定,裡言不出,外言不入,他敢怎的?為女婦人家,好吃懶做,嘴大舌長,招是惹非;不打他,打狗不成?」張四道:「不是,我打聽他家,還有一個十四歲未出嫁的閨女,誠恐去到他家,三窩兩塊,把人多口多,惹氣怎了?」婦人道:「四舅說那裡話!奴到他家,大是大,小是小,凡事從上流看。待得孩兒們好,不怕男子漢不歡喜,不怕女兒們不孝順。休說一個,便是十個,也不妨事!」張四道:「我見此人,有些行止欠端,在外眠花臥柳,又裡虛外實,少人家債負,只怕坑陷了你!」婦人道:「四舅,你老人家又差矣!他就外邊胡行亂走,奴婦人家只管得三層門內,管不得那許多三層門外的事,莫不成日跟着他走不成?常言道:『世上錢財倘來物,那是長貧久富家。』緊着起來,朝還爺一時沒錢使,還問太僕寺借馬價銀子支來使。休說買賣的人家,誰肯把錢放在家裡?各人裙帶上衣食,老人家,到不消這樣費心。」這張四見說不動這婦人,到吃他搶了幾句的話,好無顏色。吃了兩盞清茶,起身去了。有詩為證:

「張四無端喪楚言,  姻緣誰想是前緣;

佳人心愛西門慶,  說破咽喉總是閑。」

張四羞慚歸家,與婆子商議。單等婦人起身,指着外甥楊宗保,要攔奪婦人箱籠。話休饒舌,到二十四日,西門慶行禮;請了他吳大娘來,坐轎押擔。衣服頭面、四季袍兒、羹果茶餅、布絹細綿,約有二十餘擔,這邊請他姑娘併他姐姐,接茶陪待不必細說。到二十六日,請十二位高僧念經,做水陸燒靈,都是他姑娘一力張主。這張四臨婦人起身那當日,請了幾位街坊眾鄉鄰,來和婦人講話。那日薛嫂正引着西門慶家,顧了幾個閒漢,併守備府裡討的一二十名軍牢,正進來搬擡婦人床帳嫁裝箱籠。被張四攔住,說道:「保山,且休擡!有話講。」一面邀請了街坊鄰舍進來坐下。張四先開言說:「列位高鄰聽着:大娘子在這裡,不該我張龍說,你家男子漢楊宗錫與你這小叔楊宗保,都是我外甥,是我的姐姐養的,今日不幸他死了,掙了一場錢,有人主張着你,這是親戚,難管你家務事,這也罷了!爭奈第二個外甥楊宗保年幼,一個業障都在我身上。他是你男子漢一母同胞所生,莫不家當沒他的份兒?今日對着列位高鄰在這裡,你手裡有東西沒東西,嫁人去也難管你。只把你箱籠打開,眼同眾人看一看,你還擡去,我不留下你的,只見個明白。娘子你意下如何?」婦人聽言,一面哭起來,說道:「眾位聽着,你老人家差矣!奴不是歹意謀死了男子漢,今日添羞臉又嫁人!他手裡有錢沒錢,人所共知。就是積儹了幾兩銀子,都使在這房子上;房子我沒帶去,都留與小叔,家活等件,分毫不動。就是外邊有三百四百兩銀子欠帳,文書合同,已都交與你老人家,陸續討來,家中盤纏,再有甚麼銀兩來?」張四道:「你沒銀兩也罷。如今只對着眾位,打開箱籠,有沒有看一看,你還拏了去,我又不要你的。」婦人道:「莫不奴的鞋腳,也要瞧不成?」正亂着,只見姑娘柱拐自後而出。眾人便道:「姑娘出來!」都齊聲唱喏。姑娘還了萬福,陪眾人坐下。姑娘開口:「列位高鄰在上,我是他的親姑娘,又不隔從,莫不沒我說去?死了的也是姪兒,活着的也是姪兒,十個指頭,咬着都疼。如今休說他男子漢手裡沒錢,他就是有十萬兩銀子,你只好看他一眼罷了;他身邊又無出,少女嫩婦的,你攔着不教他嫁人,留着他做什麼?」眾街鄰高聲道:「姑娘見得有理!」婆子道:「難道他娘家陪的東西,也留下他的不成!他背地又不曾私自與我什麼,說我護他,也要公道!不瞞列位說,我這姪兒平日有仁義,老身捨不得他,好溫克性兒;不然,老身也不管着他。」那張四在傍,把婆子瞅了一眼,說道:「你好失心兒,鳳凰無寶處不落!」此這一句話,道着這婆子真病。須臾怒起,紫漒了面皮,扯定張四大罵道:「張四,你休胡言亂語!我雖不能不才,是楊家正頭香主。你這老油嘴,是楊家那瞭子{入日}的?」張四道:「我雖是異姓,兩個外甥是我姐姐養的;你這老咬蟲,女生外向,行放火又一頭放水!」姑娘道:「賤沒廉耻,老狗骨頭!他少女嫩婦的,留着他在屋裡,有何筭計?既不是圖色慾,便欲起謀心,將錢肥己!」張四道:「我不是圖錢爭,奈是我姐姐養的。有差遲,多是我;過不得日子,不是你!這老殺才,搬着大,引着小,黃貓兒黑尾!」姑娘道:「張四,你這老花根!老奴才!老粉嘴!你恁騙口張舌的好淡扯!到明日死了時,不使了繩子扛子!」張四道:「你這嚼舌頭老淫婦!掙將錢來焦尾靶,怪不的恁無兒無女!」姑娘急了,罵道:「張四賊!老蒼根!老豬狗!我無兒無女,強似你家媽媽子,穿寺院,養和尚,{入日}道士!你還在睡裡夢裡!」當下兩個差些兒不曾打起來。多虧眾鄰舍勸住,說道:「老舅,你讓姑娘一句兒罷。」薛嫂兒見他二人攘打鬧裏,領率西門慶家小廝伴當,并發來眾軍牢,趕人鬧裡,七手八腳,將婦人床帳、裝奩、箱籠,搬的搬,擡的擡,一陣風都搬去了。那張四氣的眼大大的,敢怒而不敢言。眾鄰舍見不是事,安撫了一回,各人多散了。到六月初二日,西門慶一頂大轎,四對紅紗燈籠,他這姐姐孟大姨送親,他小叔楊宗保頭上扎着髻兒,穿着青紗衣撒騎在馬上,送他嫂子成親。西門慶答賀了他一疋錦段、一柄玉縧兒。蘭香、小鸞兩個丫頭,都跟了來舖床疊被;小廝琴童方年十五歲,亦帶過來伏侍。到三日,楊姑娘家,并婦人兩個嫂子,孟大嫂、二嫂都來做生日。西門慶與他楊姑娘七十兩銀子,兩疋尺頭,自此親戚來往不絕。西門慶就把西廂房裡,收拾三間與他做房,排行第三,號玉樓。令家中大小,都隨着叫三姨。到晚,一連在他房中歇了三夜。正是:

「銷金帳裡,依然兩個新人;  紅錦被中,現出兩般舊物。」

有詩為證:

「怎睹多情風月標,  教人無福也難消;

風吹列子歸何處?  夜夜嬋娟在柳梢。」

畢竟未知後來如何,且聽下回分解:



\end{showcontents}
