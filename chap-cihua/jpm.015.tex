%# -*- coding: utf-8 -*-
%!TEX encoding = UTF-8 Unicode
%!TEX TS-program = xelatex
% vim:ts=4:sw=4
%
% 以上设定默认使用 XeLaTex 编译,并指定 Unicode 编码,供 TeXShop 自动识别

%第十五回 
\chapter{佳人笑賞翫登樓\KG 狎客幫嫖麗春院}

\begin{showcontents}{}



「日墜西山月出東,  百年光景似飄蓬,

點頭纔羨朱顏子,  轉眼翻為白髮翁;

易老韶華休浪度,  掀天富貴等雲空,

不如且討紅裙趣,  依翠偎紅院宇中。」

話說光陰迅速,又早到正月十五日。西門慶這裡先一日差小廝玳安,送了四盤羹菜、兩盤壽桃、一壜酒、一盤壽麵、一套織金重絹衣服,寫吳月娘名字:「西門吳氏歛袵拜。」送與李瓶兒做生日。李瓶兒纔起來梳粧,叫了玳安兒到臥房裡,說道:「前日打擾你大娘那裡,今日又教你大娘費心送禮來。」玳安道:「娘多上覆,我爹上覆二娘,不多些微禮,與二娘賞人。」李瓶兒一面分付迎春外邊明間內放小卓兒,擺了四盒茶食,管待玳安。臨出門與二錢銀子、八寶兒一方閃色手帕:「到家多上覆你列位娘,我這裡使老馮拿帖兒請去,好歹明日都光降走走。」玳安磕頭出門,兩個擡盒子的,與一百文錢。李瓶兒這裡隨即使老馮兒,用請書盒兒,拿着五個柬帖兒,十五日請月娘與李嬌兒、孟玉樓、潘金蓮、孫雪娥。又稍了一個帖,暗暗請西門慶,那日晚夕赴席。月娘到次日。留下孫雪娥看家,同李嬌兒、孟玉樓、潘金蓮四頂轎子出門,都穿著粧花錦綉衣服,來興、來安、玳安、畫童四個小廝跟隨着,到獅子街燈巿李瓶兒新買的房子。門面四間,到底三層;臨街是樓,儀門去兩邊廂房,三間客座、一間稍間;過道穿進去第三層,三間臥房、一間廚房;後邊落地緊靠着喬皇親花園。李瓶兒知月娘眾人來看燈,臨街樓上設放圍屏桌席,懸掛許多花燈。先迎接到客位內見畢禮數,次讓人後邊明間內待茶。房裡換衣裳擺茶,俱不必細說。到午間,李瓶兒客位內設四張桌席,叫了兩個唱的董嬌兒、韓金釧兒彈唱飲酒。凡酒過五巡,食割三道,前邊樓上酒席,又請月娘眾人登樓看燈頑耍。樓簷前掛着湘簾,懸着彩燈。吳月娘穿着大紅粧花通袖襖兒、嬌綠段裙、貂鼠皮襖;李嬌兒、孟玉樓、潘金蓮都是白綾襖兒、藍段裙;李嬌兒是沉香色遍地金比甲;孟玉樓是綠遍地金比甲,頭上珠翠堆盈,鳳釵半卸,鬢後挑着許多各色燈籠兒,搭伏定樓窗往下觀看,見那燈巿中,人烟湊集,十分熱鬧。當街搭數十座燈架,四下圍列些諸門買賣。玩燈男女,花紅柳綠,車馬轟雷,鰲山聳漢。怎見好燈巿?但見:

「山石穿雙龍戲水,雲霞映獨鶴朝天。金蓮燈,玉樓燈,見一片珠璣;荷花燈,芙蓉燈,散千圍錦綉。綉毬燈,皎皎潔潔;雪花燈,拂拂紛紛。秀才燈,揖讓進止,存孔孟之遺風;媳婦燈,容德溫柔,效孟姜之節操。和尚燈,月明與柳翠相連;通判燈,鍾馗共小妹並坐。師婆燈,揮羽扇,假降邪神;劉海燈,倒背金蟾,戲吞至寶。駱駝燈,青獅燈,馱無價之奇珍,咆咆哮哮;猿猴燈,白象燈,進連城之秘寶,頑頑耍耍。七手八腳,螃蟹燈,倒戲清波;巨口大髯,鮎魚燈,平吞綠藻。銀蛾鬬彩,雪柳爭輝。隻隻隨綉帶香毬,縷縷拂華旛翠幰。魚龍沙戲,七真五老獻丹書;吊掛流蘇,九夷八蠻來進寶。村裡社鼓,隊共喧闐;百戲貨郎,俱庄庄齋鬬巧。轉燈兒一來一往,吊燈兒或仰或垂。琉璃瓶光單美女奇花,雲母障並瀛州閬苑。往東看,雕漆牀,螺鈿牀,金碧交輝;向西瞧,羊皮燈,掠彩燈,錦綉奪眼。北一帶都是古董玩器;南壁廂,盡皆書畫瓶爐。王孫爭看,小欄下蹴踘齊雲;仕女相携,高樓上妖嬈衒色。卦肆雲集,相幙星羅;講新春造化如何,定一世榮枯有准。又有那站高坡打談的,詞曲楊恭;到看這搧响鈸遊腳僧,演說三藏。賣元宵的 ,高堆菓餡;粘梅花的,齊插枯枝。剪春娥,鬢邊斜鬧東風;禱涼釵,頭上飛金光耀日。圍屏畫石崇之錦帳,珠簾彩梅月之雙清。雖然覽不盡鰲山景,也應豐登快活年。」

吳月娘看了一回,見樓下人亂,和李嬌兒各歸席上吃酒去了哩。惟有潘金蓮、孟玉樓同兩個唱的,只顧搭伏着樓窗子,型下人觀看。那潘金蓮一徑把白綾襖袖子摟着,顯他遍地金掏袖兒,露出那十指春葱來,帶着六個金馬鐙戒指兒。探着半截身子,口中磕瓜子兒,把磕了的瓜子皮兒,都吐下來,落在人身上,和玉樓兩個嘻笑不止。一回指道:「大姐姐,你來看那家房簷底下,掛了兩盞玉綉毬燈,一來一往,滾上滾下,且是到好看!」一回又道:「二姐姐,你來看這對門架子上,挑着一盞大魚燈,下面又有許多小魚鱉蝦蟹兒跟著他,倒好耍子!」一回又叫孟玉樓:「三姐姐,你看這首裡,這個婆兒燈,那老兒燈!」正看着,忽然被一陣風來,把個婆子兒燈下半截割了一個大窟窿。婦人看見,笑下了。引惹的那樓下看燈的人,挨肩擦背,仰望上瞧,通擠匝不開,都壓〈足羅〉〈足羅〉兒。須臾,哄圍了一圈人。內中有幾箇浮浪子弟,直指著談論。一個說道:「已定是那公侯府位裡出來的宅眷。」一個又猜:「是貴戚皇孫家豔妾,來此看燈。不然,如何內家粧束?」那一個說道:「莫不是院中小娘兒,是那大人家叫來這裡看燈彈唱?」又一個走過來,便道:「自我認的,你每都猜不着。你把他當唱的,把後面那四個放到那裡?我告說,這兩個婦人也不是小可人家的。他是閻羅大王的妻,五道軍將的妾,是咱縣門前開生藥舖放官吏債西門大官人的婦女。你惹他怎的?想必跟他大娘子來這裡看燈。這個穿綠遍地金背比甲的,我不認的。那穿大紅遍地金比甲兒,上帶着個翠面花兒的,倒好似賣炊餅武大郎的娘子。大郎因為在王婆茶房內捉姦,被大官踢中了,死了。把他娶在家裡做了妾。後次他小叔武松東京回來告狀,誤打死了皂隸李外傳,被大官人墊發充軍去了。如今一二年不見出來,落的這等標致了。」正說著,只見一個多口過來,說道:「你們沒要緊,指說他怎的?咱每散開罷。」樓上吳月娘,見樓下人圍的多了,叫了金蓮、玉樓歸席坐下,聽著兩個粉頭彈唱燈詞飲酒。坐了一回,月娘要起身,說道:「酒勾了。我和他二娘先行一步,留下他姊妹兩個,再坐一回兒,以盡二娘之情。今日他爹不在家,家裡無人,光丟着些丫頭們,我不放心。」這李瓶兒那裡肯放,說道:「好大娘,奴沒敬心也是的。今日大娘來兒,沒好生揀一筯兒。大節間,燈兒也沒點,飯兒也沒上,就要家去。就是西門爹不在家中,還有他姑娘們哩,怕怎的?待月色上來的時候,奴送三位娘去。」月娘道:「二娘,不是這等說。我又不大十分用酒,留下他姊妹兩個,就同我這裡一般。」李瓶兒道:「大娘不用,二娘也不吃一鍾,也沒這個道理。想奴前日在大娘府上,那等鍾鍾不辭,眾位娘竟不肯饒我。今日來到奴這湫顧之處,雖無甚物供獻,也盡奴一點勞心。」于是拿大銀鍾遞與李嬌兒,說道:「二娘好歹吃一杯兒。大娘,奴曉的,吃不的了,不敢奉大杯,只奉小杯兒哩。」于是滿斟遞與月娘。因說李嬌兒:「二娘,你用過此杯罷!」兩個唱的,月娘每人與了他二錢銀子。待的李嬌兒吃過酒,月娘起身,囑付玉樓、金蓮:「我兩個先起身。我去便使小廝拿燈籠來接你們,也就來罷。家裡沒人。」玉樓應諾。李瓶兒送月娘、李嬌兒到門首上轎去了。歸到樓上,陪玉樓金蓮飲酒。看看天晚,玉兔東生。樓上點起燈來。兩個唱的彈唱飲酒,不在話下。都說西門慶那日同應伯爵、謝希大兩個,家中吃了飯,同往燈巿裡遊玩。到了獅子街子東口,西門慶因為月娘眾人今日都在李瓶兒家樓上吃酒,恐怕他兩個看見,就不往西街去看大燈,只到買紗燈的根前就回了。不想轉過灣來,撞遇孫寡嘴、祝日念唱喏,說道:「連日不會哥,心中渴想。」見了應伯爵、謝希大,罵道:「你兩個天殺的好人兒!你來和哥遊玩,就不說叫俺一聲兒。」西門慶道:「祝兄弟,你錯怪了他兩個。剛纔也是路上相遇。」祝日念道:「如今看了燈,往那裡去?」西門慶道:「同眾位兄弟到大酒樓上吃三杯兒。不是請眾兄弟,房下們今日都往人家吃酒去了。」祝日念道:「比是哥請俺每到酒樓上,咱何不往裡邊,望望李桂姐去。只當大節間,往他拜拜年去,混他混。前日俺兩個在他家,望着俺每好不哭哩。說他從臘裡不好到如今,大官人通影邊兒不進裡面看他看兒。俺每便回說,只怕哥事忙,替哥摭過了。哥今日倒閑,俺每情願相伴哥進去走走。」西門慶因計掛着晚夕李瓶兒,還推辭道:「今日我還有小事,不得去。明日罷。」怎禁這夥人死拖活拽,于是同進去院中。正是:

「柳底花陰壓路塵,  一回遊賞一回新;

不知買盡長安笑,  活得蒼生幾戶貧。」

西門慶同眾人到了李家,桂卿正打扮着在門首站立。一面迎接入中堂相見了,都道了萬福。祝日念高叫道:「快請二媽出來!還虧俺眾人,今日請的大官人來了。」少頃,老虔婆扶拐而出,向西門慶見畢禮數,說道:「老身又不曾怠慢了姐夫,如何一向不進來看看姐姐兒?想必別處另敍了新表子來。」祝日念走來插口道:「你老人家會猜算。俺大官近日相與了絕色的表子,每日只在那裡閑走,不想你家桂姐兒。剛纔不是俺二人在燈巿裡撞見拉他來,他還不來哩。媽不信,問孫天化就是了。」因指着應伯爵、謝希大,說道:「這兩個天殺的,和他都是一路神祇。」老虔婆聽了,呷呷笑道:「好應二哥,俺家沒惱着你,如何不在姐夫面前美言一句兒?雖故姐夫裡邊頭緒兒多,常言道:『好子弟不鬫一個粉頭,粉頭不接一個孤老。』天下錢眼兒都一樣,不是老身誇口說,我家桂姐也不醜,姐夫自有眼,今也不消人說。」孫寡嘴道:「我是老實說,哥如今新敍的這個表子,不是裡面的,是外面的表子,還把裡邊人{入日}八?」教那西門慶聽了,赶着孫寡嘴只顧打,說道:「老媽,你休聽這天災人禍老油嘴,弄殺人你!」孫寡嘴和眾人笑成一塊。西門慶向袖中掏出三兩銀子來,遞與桂卿:「大節間,我請眾朋友。」桂卿哄道:「我不肯接。」遞與老媽。老媽說道:「怎麼兒,姐夫就笑話我家大節下,拿不出酒菜兒,管待列位老爹。又教姐夫壞鈔,拿出銀子。顯的俺們院裡人家,只是愛錢了。」應伯爵走過來說道:「老媽你依我收了,只當正月裡頭二主子快倉,快安排酒來俺每吃。」那虔婆說道:「這個理上都使不得。」一壁推辭,一壁把銀子接的袖了。深深道了個萬福,說道:「謝姐夫的布施。」應伯爵道:「媽,你且住,我說個笑話兒你聽了。一個子弟在院闞小娘兒,那一日作耍,裝做貧子進去。老媽見他衣服藍縷,不理他。坐了半日,茶也不拿出來。子弟說:『媽,我肚飢,有飯尋些來我吃。』老媽道:『米囤也晒,那討飯來?』子弟又道:『既沒飯,有水拿些來我洗洗臉罷。』老媽道:『少挑水錢,連日沒送水來。』這子弟向袖中取出十兩一定銀子放在桌子上,教買米顧水去。慌的老媽沒口子道:『姐夫吃了臉洗飯?洗了飯吃臉?』」把眾人都笑了。虔婆道:「你還是這等快取笑,可可兒的來?自古有恁說,沒這事。」應伯爵道:「你拿耳朵,我對你說。大官人新近請了花二哥表子後巷兒吳銀兒了,不要你家桂姐了。今日不是我們纏了他來,他還往你家來哩!」虔婆笑道:「我不信。俺桂姐今日不是強口比吳銀兒好多着哩。我家與姐夫,是快刀兒割不斷的親戚。姐夫是何等人兒,他眼裡見的多。著緊處,金子也估出個成色來。」說畢,客位內放四把校椅,應伯爵、謝希大、祝日念、孫天化四人上坐,西門慶對席。老媽下去收拾酒菜去了。半日,李桂姐出來。家常挽着一窩絲、杭州攢金纍絲釵、翠梅花鈿兒、珠子箍兒、金籠墜子。上穿白綾對衿襖兒,粧花眉子綠遍地金掏袖;下著紅羅裙子。打扮的粉粧玉琢。望下不當下正,道了萬福,與桂卿一邊一個,打橫坐下。少頃,頂老彩漆方盤,拿七盞來,雪綻盤盞兒,銀舌葉茶匙,梅桂潑滷瓜仁泡茶 ,甚是馨香美味,桂卿、桂姐,每人遞了一盞,陪着吃畢茶,接下茶托去。保兒上來打抹春臺。纔待收拾擺放案酒,忽見簾子外探頭舒腦,有幾個穿藍縷衣者,謂之架兒,進來跪下,手裡拿三四升瓜子兒:「大節間,孝順大老爹!」西門慶只認頭一個叫于春兒,問:「你每那幾位在這裡?」于春道:「還有段綿紗、青聶鉞在外邊伺侯。」段綿紗進來,看見應伯爵在裡,說道:「應爹也在這裡。」連忙磕了頭。西門慶起來,分付收了他瓜子兒,打開銀子包兒,捏一兩一塊銀子掠在地下。于春兒接了,和眾人扒在地下,磕了個頭,說道:「謝爹賞賜。」往外飛跑。有朝天子單道這架兒行藏為證:

「這家子打和,那家子撮合,他的本分少,虛頭大。一些兒不巧人騰挪,遶院裡都踅過。席面上幫閑,把牙兒閑磕,攘一回纔散火。轉錢又不多,歪斯纏怎麼?他在虎口裡求津唾。」

西門慶打發架兒出門,安排酒上來吃酒。桂姐滿泛金杯,雙垂紅袖。餚烹異品,菓獻時新。倚翠偎紅,花濃酒豔。酒過兩巡,桂卿外與桂姐,一個彈箏,一個琵琶,兩個彈者,唱了一套霽景融和。正唱在熱鬧處,見三個穿青衣黃扳鞭者,謂之圓社。手裡捧着一個盒兒,盛着一隻燒鵝 ,提着兩瓶老酒 :「大節間來孝順大官人貴人。」向前打了半跪。西門慶平昔認的,一個喚白禿子,一個是小張閑,那一個是羅回子。因說道:「你每且外邊候候兒,待俺每吃過酒,踢三跑。」于是向桌上拾了四盤下飯、一大壺酒、一碟點心,打發眾圓社吃了,整理氣毬齊備。西門慶出來,外面院子裡,先踢了一跑。次教桂姐上來,與兩個「圓社」踢。一個揸頭,一個對障。抅踢拐打之間,無不假喝彩奉承。就有些不到處,都快取過去了。反來向西門慶面前討賞錢,說:「桂姐的行頭,比舊時越發踢熟了。撇來的丟拐,教小人每湊手腳不迭。再過一二年,這邊院中,似桂姊妹這行頭,就數一數二的,蓋了群絕倫了。強如二條巷董官女兒數十倍。」當下桂姐踢了兩跑下來,使的塵生眉畔,汗濕腮邊,氣喘吁吁,腰肢困乏。袖中取出春扇兒搖涼,與西門慶携手並觀,看桂卿與謝希大、張小間踢行頭。白禿子、羅回子在傍虛撮腳兒等漏,往來拾毛。亦有朝天子一詞,單道這踢圓的始末為證:

「在家中也閒,到處刮涎,生理全不幹,氣毯兒不離在身邊。每日街頭站,窮的又不趨,富貴他偏羨。從早晨只到晚,不得甚飽餐。轉不的大錢,他老婆常被人包佔。」

西門慶正看著眾人在院內打雙陸踢毬飲酒,只見玳安騎馬來接,悄悄附耳低言,說道:「大娘、二娘家去了。花二娘教小的請爹早些過去哩。」這西門慶聽了,暗暗叫玳安把馬吊在後邊門首等着。于是酒也不吃,拉桂姐房中,只坐了沒去一回兒,就出來推淨手,于後門上馬,一溜烟走了。應伯爵使保兒去拉扯,西門慶只說我家裡有事,那裡肯回來。教玳安拿了一兩五錢銀子,打發三個圓社。李家恐怕他又往後巷吳銀兒家,使丫鬟直跟至院門首方回。應伯爵等眾人,還吃二更鼓纔散。正是:

「唾罵由他唾罵,  歡娛我自歡娛。」

畢竟未知後來何如,且聽下回分解:





\end{showcontents}
