%# -*- coding: utf-8 -*-
%!TEX encoding = UTF-8 Unicode
%!TEX TS-program = xelatex
% vim:ts=4:sw=4
%
% 以上设定默认使用 XeLaTex 编译,并指定 Unicode 编码,供 TeXShop 自动识别

%第八十四回 
\chapter{吳月娘大鬧碧霞宮\KG 宋公明義釋清風寨}

「冬夏長青不世情,  乾坤妙化屬生成,

清標不染塵埃氣,  貞操惟持泉石盟;

凡節通靈無並品,  孤霜釀味有餘馨,

世人欲問長生術,  到底芳姿益壽齡。」

話說一日吳月娘請將吳大舅來商議,要往泰安州頂上,與娘娘進香;西門慶病重之時,許的願心。那時吳大舅保定備辦香燭紙馬祭品之物。玳安、來安兒跟隨,顧了頭口騎。月娘便坐一乘暖轎子,分付孟玉樓、潘金蓮、西門大姐:「好生看家,同奶子如意兒、眾丫頭、好生看孝哥兒。後邊儀門,無事早早關了。休要出去外邊。」又分付陳經濟:「休要那去,同傅夥計大門首看顧。我約莫到月盡就來家了。」十五日早辰,燒紙通信,晚夕辭了西門慶靈,與眾姐妹置酒作別。把房門各庫門房鑰匙,交付與小玉拿鑰:「前候仔細。」次日早五更起身,離了家門,一行人顧了頭口,眾姐妹送出大門而去。那秋深時分,天寒日短,一日行兩程,六七十里之地,未到黃昏,投客店村坊安歇。次早再行,一路上秋雲淡淡,寒雁口塀妻口塀妻,樹木凋落,景物荒涼,不勝悲愴。有詩單道月娘為夫主遠涉關山答心願為證:

「平生志節傲冰霜,  一點真心格上蒼;

為夫遠許神州願,  千里關山姓字香。」

話休饒舌。一路無詞,行了數日,到了泰安州。望見泰山,端的是天下第一名山。根盤地腳,頂接天心,居齊魯之邦,有巖巖之氣象。吳大舅見天晚,投在客店歇宿一宵。次日早起上山,望岱岳廟來。那岱岳廟就在山前,乃累朝祀典,歷代封禪,為第一廟貌也!但見:

「廟居岱岳,山鎮乾坤。為山岳之至尊,乃萬福之領袖。山頭倚檻,直望弱水蓬萊。絕頂攀松,都是濃雲薄霧。樓臺森聳,金烏展翅飛來;殿宇稜層,玉兔騰身走到。雕梁畫棟,碧瓦朱簷。鳳扉亮槅映黃紗,龜背綉簾垂錦帶。遙觀聖像,九獵舞舜目堯眉;近觀神顏,袞龍袍湯肩禹背。九天司命,芙蓉掩映絳綃衣;炳靈聖公,赭黃袍偏襯藍田帶。左侍下玉簪朱履,右侍下紫綬金章。闔殿威儀,護駕三千金甲將;兩廊勇猛,擎王十萬鐵衣兵。蒿里山下,判官分七十二司;白驛廟中,土神按二十四氣。管太池鐵面太尉,日日通靈;掌生死五道將軍,年年顯聖。御香不斷,天神飛馬報丹書;祭祀依時,老幼望風祈護福。嘉寧殿祥雲香靄,正陽門瑞氣盤旋。」

正是:

「萬民朝拜碧霞宮,  四海皈依神聖帝!」

吳大舅領月娘,到了岱岳廟,正殿上進了香,瞻拜了聖像。廟祝道士在傍,宣念了文書。然後兩廊都燒化了錢紙,吃了些齋食。然後統領月娘上頂,登四十九盤,攀藤攬葛上去。娘娘金殿,在半空中,雲烟深處,約四十五里。風雲雷雨,都望下觀看。月娘眾人,從辰牌時分岱岳廟起身,登盤上頂,至申時已後,方到娘娘金殿上。名宋江牌扁,金書「碧霞宮」三字。進入宮內,瞻禮娘娘金身。怎生模樣?但見:

「頭綰九頭飛鳳髻,身穿金縷絳綃衣。藍田玉帶曳長裙,白玉圭璋檠彩袖。臉如蓮萼,天然眉目映雲鬟,唇似金朱,自在規模瑞雪體。猶如王母宴瑤池,卻似嫦娥離月殿。正大仙容描不就,威嚴形像畫難成!」

月娘瞻拜了娘娘仙容,香案邊立著一個廟祝道士,約四十年紀。生的五短身材,三溜髭鬚,明眸皓齒。頭戴簪冠,身披絳服,足穿云履。向前替月娘宣讀了還願文疏,金爐內炷了香,焚化了紙馬金銀,令左右小童收了祭供、原來這廟祝道士,也不是個守本分的。乃是前邊岱岳廟裡金住持的大徒弟,姓石雙名伯才。極是個貪財好色之輩,趨時攬事之徒。這本地有個殷太歲,姓殷,雙名天錫,乃是本州知州高廉的妻弟。常領許多不務本的人,或張弓挾彈,牽架鷹犬,在這上下二宮,專一睃看四方燒香婦女,人不敢惹他。這道士石伯才,專一藏奸蓄詐,替他賺誘婦女到方丈,任意姦淫,取他喜歡。因見月娘生的姿容非俗,戴著孝冠兒。若非官戶娘子,定是豪家閨眷;又是一位蒼白髭鬚老子,跟隨兩個家童。不免向前稽首,收謝神福,請二位施主方丈一茶。吳大舅便道:「不勞生受,還要趕下山去。」伯才道:「就是下山,也還早哩。」不一時,說至方丈。裡面糊的雪白,正面芝麻花坐牀,柳黃錦帳,香几上供養一軸洞賓戲白牡丹圖畫。左右一聯淡濃之筆,大書:「携兩袖清風舞鶴,對一軒明月談經。」問吳大舅上姓。大舅道:「在下姓吳名鎧,這個就是舍妹吳氏。因為夫主未還香愿,不當取擾上宮。」伯才道:「既是令親,俱延上坐。」他便主位坐了,便叫徒弟守清、守禮看茶。原來他手下有個徒弟,一個叫郭守清,一個叫郭守禮,皆十六歲,生的標致,頭上戴青段道髻,用紅絨繩扎住總角,後用兩根飄帶。身穿青絹道服,腳上涼鞋淨襪,渾身香氣襲人。客至則遞茶遞水,斟酒下菜。到晚來,背地來掇箱子,拿他解纔填餡。明雖為師兄徒弟,實為師父大小老婆。更有一件不可說,脫了褲子,每人小幅裡夾著一條大手巾。看官聽說:但凡人家好兒好女,切記休要送與寺觀中出家,為僧作道;女孩兒做女冠姑子,都稱瞎男盜女娼。十個九個都着了道兒。有詩為證:

「琳宮梵剎事因何,  道即天尊釋即佛,

廣栽花草虛清意,  待客迎賓假做作;

美衣麗服裝徒弟,  浪酒開茶戲女娥,

可惜人家嬌養子,  送與師父作老婆。」

不一時,兩個徒弟守清、守禮,房中安放卓兒,就擺齋上來,都是美口甜食,蒸煠餅饊,鹹□春饌,各樣菜蔬,擺滿春臺。白定磁盞兒,銀杏葉匙,絕品雀舌甜水好茶,收下家火去。就擺上案酒,大盤大碗餚饌,都是雞鵝魚鴨葷菜上來。□□□□斟琥珀,銀鑲盞滿泛金波。吳月娘酒來,就要起身。叫玳安近前,用紅漆盤托出一疋大布,二兩白金,與石道士作致謝之禮。吳大舅便說:「不當打攪上宮。這些微禮,致謝仙長,不勞見賜酒食。天色晚來,如今還要趕下山去。」慌的石伯才致謝不已說:「小道不才,娘娘福蔭,在本山碧霞宮做個住持,仗賴四方錢粮,不管待四方財主,作何項下使用?今聊備粗齋薄饌,倒反勞見賜厚禮,使小道都之不恭,受之有愧!」辭謝再三,方令徒弟收下去。一面留月娘、吳大舅坐:「好歹坐片時,略飲三杯,盡小道一點薄情而已。」吳大舅見款留懇切,不得已和月娘坐下。不一時熱下飯上來,石道士分付徒弟:「這個酒不中吃,另開昨日徐知府老爹送的那一罎與透瓶香荷花酒來,與你吳老爹用。」不一時,徒弟弔用熱壺,篩熱酒上來。先滿斟一杯,雙手遞與月娘。月娘不肯接。吳大舅說:「舍妹他天性不用吃酒。」伯才道:「老夫人連路風霜,用些何害?好歹淺用些。」一面倒去半鍾,遞上去,與月娘接了又斟一杯遞與吳大舅,說:「吳老爹,你老人家試嘗此酒,其味何?」吳大舅飲了一口,覺香甜絕美,其味深長。說道:「此酒甚好。!」伯才道:「不瞞你老人家說,此是青州徐知府老爹,送與小道的酒。他老夫人、小姐、公子,年年來岱岳廟燒香建醮,與小道相交極厚。他小姐衙內,又寄名在娘娘位下兒。小道立心平淡,慇懃香火,一味志誠,甚是敬愛小道。常年這岱岳廟上下二宮錢粮,有一半征收入庫。近年多虧了我這恩主徐知府老爹,題奏過,也不征收,都全放常住用度,侍奉娘娘香火。餘者接待四方香友。」這裡說話,下邊玳安、平安跟從轎夫,下邊自有坐處,湯飯點心,大盤大碗酒肉,都吃飽了。看官聽說:這石伯才窩藏殷天錫,賺引月娘到方丈,要暗中取事。豈不加意奉承?飲了幾杯,吳大舅見天晚,要起身。伯才道:「日色將落,晚了,趕不下山去。倘不棄,在小道方丈,權宿一宵,明早下山從容些。」吳大舅道:「爭奈有些小行李在店內,誠恐一時小人嚕躁。」伯才笑道:「這個何須挂意?如有絲毫差遲,聽得是我這裡進香的,不拘村坊店道,聞風害怕!好不好把店家拿來本州夾打,就教他尋賊人下落。」吳大舅聽了,便坐住。伯才拿大鍾斟上酒,吳大舅見酒利害,遂往後邊閣上觀看隨喜,伯才便教徒弟守清引領,拿鑰匙開門,教大舅觀看去了,這月娘覺身子乏困,便要牀上側側兒。這石伯才一面把房門拽上,外邊坐去了。也是合當有事,月娘方纔牀上〈扌歪〉著,忽聽裏面响喨了一聲。牀背後紙門內,跳出一個人來,淡紅面貌,二柳髭鬚,約三十年紀。頭戴滲青巾,身穿紫錦袴衫。雙關抱住月娘,說道:「小生姓殷名天錫,乃高太守妻弟。久聞娘子乃官豪宅眷,天然國色。思慕已久,渴欲一見,無由得會。今既接英標,乃三生有幸,死生難忘也!」一面按著月娘在牀上求歡。月娘諕的慌做一團,高聲大叫:「清平世界,朗朗乾坤,沒事把良人妻室,強攔在此做甚?」就要奪門而走。被天錫死苦攔擋不放,便跪下說:「娘子禁聲,下顧小生,懇求憐允!」那月娘越高聲叫的聲緊了。口口大叫:「救人!」來安、玳安聽見是月娘聲音,慌慌張張,走去後邊閣上,叫大舅說:「大舅快去!我娘在方丈和人合口哩!」這吳大舅兩步做一步,奔到方丈推門,那裡推得開!只見月娘高聲:「清平世界,攔燒香婦女在此做甚麼?」這吳大舅便叫:「姐姐休慌,我來了!」一面拿石頭把門砸開,那殷天錫見有人來,撒開手,打牀背後一溜烟走了。原來這石道士牀背後,都有出路。吳大舅砸開方丈門,問月娘道:「姐姐,那廝玷污不曾?」月娘道:「不曾玷污。那廝打牀背後走了。」吳大舅道士,那石道士躲去一邊,只教徒弟來支調。被大舅大怒,喝令手下跟隨玳安、來安兒,把安兒,把道士門窗戶壁都打碎了。一面保月娘出離碧霞宮,上了轎子,便趕下山來。約黃昏時分起身,走了半夜,投天明趕到山下客店內。如此這般,告店小二說。小二叫苦連聲,說:「不合惹了殷太歲,他是本州知州相公妻弟,有名殷太歲。你便去了,把俺開店之家,他遭塌凌辱,怎肯干休?」吳大舅便多與他一兩店錢,取了行李,保定月娘轎子,急急奔走。後面殷天錫不捨,率領二三十閑漢,各執腰刀短棍,趕下山來。吳大舅一行人,兩程做一程。約四更時分,趕到一山凹裡。遠遠樹木叢中,有燈光。走到跟前,都是一座石洞,裡面有一老僧秉燭念經。吳大舅問:「老師,我等頂上燒香,被強人所趕,奔下山來。天色昏黑,迷踪失路至此,敢問老師,此處是何地名?從那條路回家?」老僧道:「此是岱岳東峯。這洞名喚雪澗洞。貧僧就叫雲洞禪師,法名普靜,在此修行二三十年。你今遇我,實乃有緣!休往前去,山下狼虫虎豹極多。明日早行,一直大道,就是你清河縣了。」吳大舅道:「只怕有人追趕。」老師把眼一觀,說:「無妨,那強人趕至半山,已回去了。」因問月娘姓氏。吳大舅道:「此乃吾妹,西門之妻。因為夫主,來此進香。得遇老師搭救,恩有重報,不敢有忘。」于是在洞內歇了一夜。次日五更,月娘拿出一疋大布謝老師。老師不受,說:「貧僧只化你親生一子,作個徒弟。你意下如何?」吳大舅道:「吳妹止生一子,指望承繼家業。若有多餘,就與老師作徒弟出家。」月娘道:「小兒還小,今纔不到一周歲兒,如何來得?」老師道:「你只許下,我如今不問你要。要過十五年纔問你要哩。」月娘口中不言。「過十五年在作理會。」遂許下老師。看官聽說:不當今日許老師一子出家,後來十五年之後,天下荒亂,月娘攜領孝哥還兒,往河南投奔雲離守就昏去。路遇老師度化在永福寺,落法為僧。此事表過不題。次日,月娘辭了老師,往前所進。走了一日,前有一山攔路。這座山名喚清風山,生的十分險惡。但見:

「八面嵯峨,四圍險峻。古怪喬松盤翠蓋,搓崆迓樹挂藤蘿。瀑布飛來,寒氣逼人毛髮冷;巔崖直下,清光射目夢魂驚。澗水時聞,推一人齊晌。峯巒倒卓,山鳥聲哀,糜鹿成群,狐狸結黨。穿荊棘往來跳躍,尋野食前後呼號。佇立草坡,一望並無商旅店,行來山徑,週迴盡是死屍坑。若非佛祖修行處,定是強人打劫場。」

原來這山,喚做清風山,山上有座清風寨,寨中有三個強寇。一名錦毛虎燕順,一名矮腳虎王英,一個白面郎君鄭天壽,手下聚五百小嘍囉,專一打家劫道,放火殺人,人不敢惹他。當下吳大舅一行人,騎頭口,簇擁著月娘轎子,進入山來。那時日色已落,天色昏黑,不見村坊店道。正在危懼之際,不防地下拋去一條絆馬索子,把吳大舅頭口絆落倒,跌坐塹坑內。原來山下小嘍囉,見月娘轎子,搶上山來,吳大舅一行人,報與三個強寇。閃出一夥小嘍囉,騎著馱垛,逕入山來。吳大舅一行人,都被拿到寨前。三個強寇在寨上,正陪山東及時雨宋江飲酒。宋江因殺了娼婦閻婆惜,逃躲至此。三人留他寨中住幾日。宋江看見月娘頭戴孝髻,身穿縞素衣服,舉止端莊,儀容秀麗,斷非常人妻子,定是富家閨眷,因問其姓氏。月娘向前道了萬福:「大王,妾身吳氏之女,千戶西門慶之妻,守節孤霜。因為夫主病重,許下泰山香願。先在山上被殷天錫所趕,走了一日一夜,要回家去。不想天晚,誤從大王山下所過。行李馱垛,都不敢要,只是乞饒性命還家,萬幸矣!」宋江因見月娘詞氣哀惋動人,便有幾分慈憐之意。乃便欠身向燕順道:「這位娘子,乃是我同僚正官之妻,有一面之識。為夫主到此進香,因被殷天錫所趕,誤到此山所過,有犯賢弟清蹕,也是個烈婦。看我宋江的薄面,放他回去,以全他名節罷!」王英便說:「哥哥爭奈小弟沒個妻室,讓與小弟做個押寨夫人罷!」遂令小嘍囉,把月娘據入他後寨去了。宋江向燕順、鄭天賜道:「我恁說一場,王英兄弟就不肯教我做個人情?」燕順道:「這兄弟諸般都好,自吃了有這些毛病!見了婦人女色,眼裏火就愛。」那宋江也不吃酒,同二人走到後寨,見王英正摟著月娘求歡。宋江走到根前,一把手將王英拉著前邊,便說道:「賢弟既做英雄,犯了『溜骨腿』三字,不為好漢!你要尋妻室,等宋江替你做媒,保一個實女子好的,行茶過水,娶來做個夫人。何必要這再醮做甚麼?」王英道:「哥哥,你且胡亂權讓兄弟這個罷!宋江道:「不好,我宋江久後決然替賢弟擇娶一個好的。不爭你今日要了這個婦人,惹江湖上好漢恥笑。殷天錫那廝,我不上梁山便罷;若上梁山,決替這個婦人報了仇!」看官聽說:後宋江到梁山做了寨主,因為殷天錫奪了柴皇城花園,使黑旋風李逵殺了殷天錫,大鬧了高唐州。此事表過不題。當日燕順見宋江說此話,也不問王英肯不肯,喝令轎夫上來,把月娘擡了去。吳月娘見放了他,向前拜謝宋江說:「蒙大王活命之恩。」宋江道:「阿呀!我不是這山寨大王,我是鄆城縣客人。你是拜這三位大王便了。」月娘拜畢,吳大舅保著離了山寨,過了清風山,往清河縣大道前來。正是:

「撞碎玉籠飛彩鳳,  頓開金鎖走蛟龍。」

有詩為證:

「世上只有人心歹,  萬物還教天養人;

但交方寸無諸惡,  狼虎叢中也立身。」

畢竟未知後來如何,且聽下回分解:

