%# -*- coding: utf-8 -*-
%!TEX encoding = UTF-8 Unicode
%!TEX TS-program = xelatex
% vim:ts=4:sw=4
%
% 以上设定默认使用 XeLaTex 编译,并指定 Unicode 编码,供 TeXShop 自动识别

%第七十六回 
\chapter{孟玉樓解膃吳月娘\KG 西門慶斥逐溫葵軒}


「動靜謀為要三思,  莫將煩惱自招之;

人生世上風波險,  一日風波十二時。」

話說西門慶見月娘半日不出去。又親自進來催促了一遍。見月娘穿衣裳,方纔請進任醫官,到上房明間內坐下。見正面酒金軟壁,兩邊安放春凳,地平上鋪着毡毯,安放火盆。少頃,月娘從房內出來,五短身材,團面皮兒,黃白淨兒。樣樣兒不肥不瘦,身體兒不短不長。兩兩春山,月鈎一雙鳳眼;纖長春笋,露甄妃之玉。朱唇點漢署之香,望上拜道了萬福。慌的任醫官躲在傍邊,屈身還禮。月娘李就在對面一椅坐下。琴童安放卓兒綿裀,月娘向袖口邊飾玉腕,露青蔥,教任醫官胗脉。良久,月娘抽身回房去了。房中小廝拿出茶來,吃畢茶,任醫官說道:「老夫人原來稟的氣血弱,尺脉來的又浮澀,雖有胎氣,有些榮衞失調,易生嗔怒,又動了肝火。如今頭目不清,中腕有些阻滯,作其煩悶。四肢之內,血少而氣多。」月娘使琴童來說:「娘如今只是有些頭疼心脹肐膊發麻,肚腹往下墜着疼,腰酸,吃飲食無味。」任醫官道:「我已知道,說得明白了。」西門慶道:「不瞞后溪說,房下如今見懷臨月身孕。因着氣惱,不能運轉,滯在胸膈間。望乞老先生留神加減一二,足見厚情。」任醫官道:「豈勞分付,學生無不用心!此去就奉過藥來,清胎、理氣、和中、養榮、蠲痛之劑。老夫人服過,要戒氣惱,就厚味也少吃。」西門慶道:「望乞老先生把他這胎氣好生安一安。」任醫官道:「已定安胎理氣,養其榮衞。不勞多喋,學生自有斟酌。」西門慶復說:「學生第三房下有些肚冷,望乞有暖宮丸藥見賜來。」任醫官道:「學生謹領,就封過來。」說畢,起身走到前廳院內,見許多教坊樂工伺候,因問:「老翁今日府上有甚事?」西門慶悉言:「巡按宋公連兩官司,請巡撫候石泉老先生,在舍擺酒。」這任醫官聽了,越發心中駭然尊敬西門慶,在門前揖讓上馬禮去,比尋日不同,倍加敬重。西門慶送他回來,隨即封了一兩銀子,兩方手帕,即使琴童拿盒兒騎馬討藥去。李嬌兒、孟玉樓眾人都在月娘屋裡裝定菓盒,搽抹銀器,便說:「大娘你頭里還要不出去,怎麼知道你心中如此這般病。」月娘道:「甚麼好成樣的老婆,由他死便死了罷!不知那淫婦他怎麼的,行動管着俺們,你是我婆婆?無故只是大小之分罷了!我還大他八個月哩!漢子疼我,你只顧好看我一般兒里!他不討了他口裡話,他怎麼和我大嚷大鬧?若不是你們攛掇我出去,我後十年也不出去。隨他死教他死去!常言道:『一雞死,一雞鳴。』新來雞兒,打鳴不好聽?我死了,把他立起來,也不亂,也不嚷,纔拔了蘿蔔地皮寬!」玉樓道:「大娘,耶嚛!耶嚛!那里有此話!俺每就待他賭個大誓,這六姐,不是我說他,要的不知好歹,行事兒有些勉強,恰似咬群出尖兒的一般,一個大有口沒心的貨子。大娘你若惱他,可是錯惱了。」月娘道:「他是比你沒心?他一團兒心哩。他怎的會悄悄聽人兒,行動拿話兒說諷着人說話?」玉樓道:「娘,你是個當家人,惡水缸兒,不恁大量些罷了,卻怎樣兒的?常言:『一個君子,待了十個小人。』你手放高些,他敢過去了。你若與他一般見識起來,他敢過不去!」月娘道:「只有了漢子與他做主兒,着把那大老婆且打靠後!」玉樓道:「哄那個哩!如今像大娘心裡恁不好,他爹敢往那屋裡去麼?」月娘道:「他怎的不去?于是他說的,他屋裡拿猪毛繩子套他不去。一個漢子的心,如同沒籠頭的馬一般,他要喜歡那一個,只喜歡那個。敢攔他?攔他,又說是浪了!」玉樓道:「罷麼,大娘,你已是說過,通把氣兒納納兒。等我教他來與娘磕頭賠個不是,趁着他大妗子在這里,你每兩個笑開了罷。你不然,教他爹兩下里不作難?就行走也不方便。但要往他屋裡去,又不怕你惱?若不去,他又不敢出來!今日前邊恁擺酒,俺每都在這定菓盒,忙的了不得,落得他在屋裡,是全躲猾兒悄靜兒,俺每也饒不過他。大妗子,我說的是不是?」大妗子道:「姑娘也罷,他三娘也說的是。不爭你兩個話差,只顧不見面,教他姑夫也難,兩下里都不好行走的。」那月娘通一聲也不言語。這孟玉樓抽身就往前走。月娘道:「孟三娘,不要叫他去,隨他來不來罷。」玉樓道:「他不敢不來?若不來,我可拿猪毛繩子套了他來。」一直走到金蓮房中,見他頭也不梳,把臉黃着,坐在炕上。玉樓道:「六姐,你怎的裝憨兒?把頭梳起來。今日前邊擺酒,後邊恁忙亂,你也進去走走兒,怎的只顧使性兒起來?剛纔如此這般,俺每對大娘說了,勸了他這一回。你去到後邊,把惡氣兒揣在懷裡,將出好氣兒來,看怎的,與他下個禮,賠了不是兒罷!你我既在簷底下,怎敢不低頭?常言:『甜言美語三冬暖,惡語傷人六月寒。』你兩個已是見過話,只顧使性兒到幾時?人受一口氣,佛受一爐香。你去與他陪過不是兒,天大事都了。不然,你不教他爹兩下里也難。待要往你這邊來,他又惱。」金蓮道:「耶嚛!耶嚛!我拿甚麼比他?可是他說的,他是真材實料,正經夫妻,你我都是趁來的露水兒?能有多大湯水兒?比他的腳指頭兒,也比不上的!」玉樓道:「你又他說不是!我昨日不說的,一棒打三四個人,那就好。嫁了你的漢子,也不是趁將來的。當初也有個三媒六婆,白恁就跟了往你家來?來砍一枝損百株。『兔死狐悲,物傷其類。』就是六姐惱了你,還有沒惱你的!有勢休要使盡,有話休要說盡。凡事看上顧下,留些兒防後纔好!不管螺虫螞蚱,一例都說着,對着他三位師父、郁大姐;人人有面,樹樹有皮,俺每臉上就沒些血兒!一切來往都罷了,你不去都怎樣兒的?少不的逐日唇不離腮,還在一處兒!你快些把頭梳了,咱兩個一答兒後邊去。」那潘金蓮見他這般說,尋思了半日,忍氣吞聲,鏡臺前拿過抿鏡,只抿了頭,戴上䯼髻,穿上衣裳,同玉樓逕到後邊上房內。玉樓掀開簾兒先進去,說道:「大娘,我怎的走了去,就牽了他來,他不敢不來。」便道:「我兒,還不過來與你娘磕頭?」在傍邊便道:「親家,孩兒年幼,不識好歹,冲撞親家。高擡貴手,將就他罷,饒過這一遭兒。到明日再無禮,犯到親家手裡,隨親家打,我老身卻不敢說了!」那潘金蓮插燭也似與月娘,磕了四個頭,跳起來趕着玉樓打,值道:「漢子邪了你這麻淫婦!你又做我娘來了!」連眾人都笑了。耶月娘忍不住也笑了。玉樓道:「賊奴才,你見你主子與了你好臉兒,就料毛兒打起老娘來了!」大妗子道:「這個你姊妹們笑開,恁歡喜歡喜卻不好?就是俺這姑娘一時間一言半語聐聒的,你每人家廝擡廝敬,儘讓一句兒就罷了。常言:『牡丹花兒雖好,還要綠葉兒扶持。』」月娘道:「他不言語,那個好說他?」金蓮道:「娘是個天,俺每是個地。娘容了俺每,俺每骨禿扠着心裡!」玉樓也打了他肩背一下,說道:「我的兒,你這回兒打你一面口袋了。」便道:「休要說嘴,俺每做了這一日活,也該你來助助忙兒。」這金蓮便洗手剔甲,在炕上與玉樓裝定菓盒,不在話下。那孫雪娥單管率領家人媳婦,竈上整理菜蔬。廚役又在前邊大廚房內,烹炮蒸煮,燒錦纏羊,割獻花猪。琴童討將藥來,西門慶看了藥帖,把丸藥送到玉樓房中,煎藥與月娘。月娘便問玉樓:「你也討藥來?」玉樓道:「還是前日分付那根兒,下首裡只是有些怪疼。我教他爹對任醫官說,稍帶兩服丸子藥來我吃。」月娘道:「你還是前日空心掉了冷氣了,那里管下寒的是?」按下後邊,都說前廳。宋御史先到了,看了卓席,西門慶陪他在捲棚內坐。宋御史又深謝其爐鼎之事:「學生還當奉價。」西門慶道:「早知我正要奉送公祖,猶恐見都,豈敢云價?」宋御史道:「這等,何以克當?」一面又作揖致謝。茶罷,因說起地方民情風俗一節,西門慶大略可否而答之。次問其有司官員,西門慶道:「卑職自知其本府胡正尹,民望素著,李知縣吏事克勤,其餘不知其詳,不敢妄說。」宋御史問道:「守禦周秀,曾與執事相交,為人都也好不好?」西門慶道:「周總兵雖歷練老成,還不如濟州荊都監,青年武舉出身,才勇兼備。公祖倒看他看。」宋御史道:「莫不是都監荊忠?執事何以相熟?」西門慶道:「他與我有一面之交,昨日遞了個手本與我,也要乞望公祖情盼一二。」宋御史道:「我也久聞他是個好將官。」又問其次者,西門慶道:「卑職還有妻兄吳鏜,見任本衙右所正千戶之職。昨日委管修義倉例,該陞擢指揮。亦望公祖提援,實卑職之沾恩惠也!」宋御史道:「既是令親,到明日類本之時,不俾他加陞本等職級,我還保舉他見任管事。」這西門慶連忙作揖謝了。因把荊都監并吳大舅履歷手本遞上。宋御史看了,即令書辦吏典收執。分付:「到明日類本之時,呈行我看。」那吏典收下去了。西門慶又令左右悄悄遞了三兩銀子與他。那書吏如同印板刻在心上,不在話下。正說話間,前廳鼓樂響。左右來報,兩個老爹都到了。慌的西門慶即出迎接,到廳上敘禮。這宋御史慢慢纔走出花園角門,眾官見畢禮數。觀其正中,擺設大插卓一張,五老定勝 ,方糖高頂一簇盤,大飲五牲菓品,甚至齊整。周圍卓席其豐勝,心中大悅。都望西門慶謝道:「生受!容當奉補。」宋御史道:「分資誠為不足。四泉看我的分上,罷了;諸公也不消補奉。」西門慶道:「豈有此禮?」一面各分次序坐下。左右拿上茶來,眾官都說:「候老先生那里,已各人差官邀去了。還在都府衙未起身哩!」兩邊俳長樂工,鼓樂笙笛蕭管方響,在二門裡伺候的鐵桶相似。看看等到午後時分,只見一疋報馬來到,說:「候爺來了。」這里兩邊鼓樂一齊響起,眾官都出大門前邊接。宋御史在二門裡相候。不一時,藍騎馬道過盡,候巡撫穿大紅孔雀,戴貂鼠暖耳,渾金帶,坐四人大轎,直至門首下轎。眾官迎接進來。宋御史亦換了大紅金云白豸員領,犀角帶,相讓而入。到於大廳上,敘畢禮數。各官廷參畢,然後與西門慶拜見。宋御史道:「此是主人西門千兵,見在此間理刑,亦是蔡老先生門下。」這候巡撫即令左右官吏拿雙紅友生候蒙卑拜帖遞與西門慶。門慶雙手接了,分付家人捧上去。一面參拜畢,寬衣上坐。眾官兩傍僉坐。宋御史居主位。捧畢茶,階下動起樂來。宋御史把盞遞酒,簪花,捧上尺頭,隨即擡下卓席來,裝在盒內,差官吏送到公廳去了。然後上坐獻湯飯,廚役上來割獻花猪,俱不必細說。先是教坊間弔上隊舞回數,都是官司新錦綉衣裝,撮弄百戲,十分齊整。然後纔是海鹽子弟上來磕頭,呈上關目揭帖,候公分付搬演裝晉公還帶記,唱了一摺下來,又割錦纏羊。端的花簇錦攢,吹彈歌舞。筲韶盈耳,金貂滿座。有詩為證:

「華堂非霧亦非烟,  歌遏行雲酒滿筵;

不但紅蛾垂玉珮,  果然綠鬢插金蟬。」

候巡撫只坐到日西時分,酒過數巡,歌唱兩摺下來,令左右拿下來五兩銀子,分賞廚役、茶酒、樂工、腳下人等,就穿衣起身。眾官俱送出大門,看着上轎而去。回來,宋御史與眾官辭謝西門慶,亦告辭而歸。西門慶送了回來,打發樂工散了。因見天色尚早,分付把卓席休動,教廚役上來攢整菜蔬肴饌,一面使小廝請吳大舅來,并溫秀才、應伯爵、傅夥計、甘夥計、賁地傅、陳經濟來坐聽唱。拿下兩卓酒饌肴品,打發海鹽子弟吃了,等的人來,教他唱四節記、冬景、韓熙夜宴。擡出梅花來放在兩邊卓上,賞梅飲酒。原來那日賁四、來興兒管廚,陳經濟管酒,傅夥計、甘夥計看管家火。聽見西門慶請,都來傍邊坐的。不一時,溫秀才過來作揖坐下。吳大舅、吳二舅、應伯爵都來了。應伯爵與西門慶聲唱:「前日空過眾位嫂子,又多謝重禮!」西門慶笑罵道:「賊天殺的狗材!你打窗戶眼兒內偷瞧的你娘們好!」伯爵道:「你休聽人胡說,豈有此理?我想來也沒人。」指王經道:「就是你這賊狗骨禿兒,乾淨來家就學舌!我到明日把你這小狗骨禿兒肉也咬了!」說畢,吃了茶。吳大舅要到後邊,西門慶陪下來,向吳大舅:「如此我今對宋大巡替大舅說了說那個,他看了揭帖,交付書辦收了。我又與了書辦三兩銀子,連荊大人的都放在一處。他親口說下,到明日類本之時,自有意思。」吳大舅聽見,滿心歡喜,連忙與西門慶唱喏:「多累姐夫費心!」西門慶道:「我就說是我妻兄。他說既是令親,我已定見過分上。」于是同到房中見了月娘。月娘與他哥哥道萬福。大舅向大妗子說道:「你往家去罷了!家沒人,如何只顧不出去了?」大妗子道:「三姑娘留下,教我過了初三日,初四日家去罷哩。」吳大舅道:「既是姑娘留你,到初四日去便了。」說畢月娘留他坐,不坐。來到前邊,安排上酒來飲酒。當下吳大舅、二舅、應伯爵、溫秀才上坐,西門慶主位,傅夥計、甘夥計、賁地傅、陳經濟兩邊打橫,共五張卓兒。下邊戲子鑼鼓響動,搬漁韓熙夜宴,郵亭住遇。正在熱鬧處,忽見玳安來說:「喬親家爹那里使了喬通在下邊,請爹說話。」這西門慶隨即下席,到東角門首見喬親家喬通。喬通道:「爹說昨日空過親家,爹使我送那援例銀子來,一封三十兩,另外又拿着五兩,與吏房使用。」西門慶道:「我明日早封過與胡大尹,他就與了劄付來。又與吏房銀子做甚麼?你還拿回去。」一面分付玳安教廚下拿了酒飯點心,在書房內管待喬通,打發去了。語休饒舌,當日唱了郵亭兩摺,約有一更時分,西門慶前邊人散了,收了家火,進入月娘房來。月娘正與大妗子在炕上坐的,大妗子見西門慶進來,連忙往那邊屋裡去了。西門慶因向月娘說:「我今日替你哥,如此這般對宋巡按說,他許下加他除加陞一級,還教他見任管事,就是指揮僉事。我剛纔已對你哥說了,他好不喜歡。只在年終,就題本旨意下來。」月娘便道:「沒的說,他一個窮衞家官兒,那里有二三百兩銀子使?」西門慶道:「誰問他要一百文錢兒?我就對宋御史說,是我妻兄。他親口既許下,無有個不做分上的。」月娘道:「隨你與他幹,我不管你。」西門慶便問玉筲:「替你娘煎了藥?拿來我瞧,打發你娘吃了罷。」月娘道:「你去,休管他。等我臨睡自家吃。」那西門慶纔待往外走,被月娘又妗回來,問道:「你往那去?是往前頭去,趁早兒不要去。他頭里與我陪了不是了,只少你與他陪不是去哩!」西門慶道:「我不往他屋裡去。」月娘道:「你不往那屋裡去,往誰屋裡去?那前頭媳婦子跟前,也省可去。惹的他昨日對着大妗子好不拿話兒咂我,說我縱容着你,要他圖你喜歡哩!你又恁沒廉恥的!」西門慶道:「你理那小淫婦兒怎的?」月娘道:「你只依我今日,偏不要往前邊去,也不要你在我這屋裡。你往下邊李嬌姐房裡睡去。隨你明日去不去,我就不管你了。」這西門慶兒恁說,無法可處,只得往李嬌兒房裡歇了一夜。到次日,臘月初一日,早往衙門中去,同何千戶發牌、陞廳、畫卯、發放公文,一早辰纔來家。又打點禮物猪酒,并三十兩銀子,差玳安往東平府送胡府尹去。胡府尹收下禮物。即時討過劄付來。西門慶在家請了陰陽徐先生,廳上擺設猪羊酒菓,燒紙還願心畢,打發徐先生去了。因見玳安到了,看了回帖,已封過劄付來,上面用着許多印信,填寫喬洪本府義官名目。一面使玳安送兩盒胙肉與喬大戶家,就請喬大戶來吃酒,與他劄付瞧。又分送與吳大舅、溫秀才、應伯爵、謝希大、傅夥計、甘夥計、韓道國、賁地傅、崔本每人都是一盒,俱不在話下。一面又發帖兒,初三日請周守禦、荊都監、張團練、劉、薛二內相、何千戶、范千戶、吳大舅、喬大戶、王三官兒共十位客,叫一起雜耍樂工,四個唱的。那日孟玉樓在月娘房內攢了帳,遞與西門慶,就交代與金蓮管理使用,銀錢他不管了。因問月娘道:「大娘,你昨日吃了藥兒,可好些?」月娘道:「怪不的人說怪浪肉!平白教人家漢子捏了捏手,今日好了,頭也不疼,心口也不發脹了。」玉樓笑道:「大娘,你原來只少他一捏兒!」連大妗子也笑了。西門慶來,又問月娘。月娘道:「該那個管,你交與那個就是了。來問我怎的?誰肯讓的誰?」這西門慶方纔兌了三十兩銀子,三十吊錢,交與金蓮管理,不在話下。良久,喬大戶到了,西門慶陪他廳上坐的,如此這般,拿胡府尹義官喬洪名字挽例上納白米三十石,以濟邊儲。滿心歡喜,連忙向西門慶打恭致謝:「多累親家費心,容當叩謝。」因說:「明日喬通好生送到家去。若親家見招,在下有此冠帶,就取來陪他也不妨。」西門慶道:「初三日,親家好歹早些下降。」一面吃畢茶,分付琴童:「西廂房書房裡放卓兒,親家請那里坐,還暖些。」到書房,地爐內籠着火。西門慶與喬大戶對面坐下。因告訴說:「昨日巡按兩司請候撫院之事,侯老甚喜。明日起身,少不的俺同僚每都送郊外方回。」纔抹卓兒收拾放菜兒,只見應伯爵到了。斂了幾分人情,叫應寶用盒兒拿來,交與西門慶說:「此列位奉賀哥的分資。」西門慶打開觀看,裡面頭一位就是吳道官,其次應伯爵、謝希大、祝日念、孫寡嘴、常時節、白來創、李智、黃四、杜三哥,共十分人情。西門慶道:「我的這邊,還有舍親吳二舅、沈姨夫,門外任醫官、花大哥并三個夥計、溫葵軒,也有二十多人,就在初四請罷。」一面令左右收進人情後邊去,使琴童兒:「拿馬請你吳大舅來陪你喬親家爹坐。」因問:「溫師父在家不在?」來安兒道:「溫師父不在家,從早辰望朋友去了。」不一時,吳大舅來到,連陳經濟五人共坐,把酒來斟。卓上擺列許多熱下飯、湯碗,無非是猪蹄羊頭,燒爛煎煿,雞魚鵝鴨,添案之類。飲酒中間,西門慶因向吳大舅說:「喬親家恭喜的事,今日已領下義官劄付來了。容日我這里備禮寫文軸,咱每從府中迎賀迎賀。」喬大戶道:「惶恐!甚大職役,敢起動列位親家費心?」忽有本縣衙差人送曆日來了,共二百五十本。西門慶拿回帖賞賜,打發來人去了。應伯爵道:「新曆日俺每不曾見哩。」西門慶把五十本拆開,與吳大舅、伯爵、溫秀才三人分了。伯爵看了,開年改了重和元年,該閏正月。不說當日席間猜枚行令。飲酒至晚,喬大戶先告家去。西門慶陪吳大舅坐到起更時分方散。分付伴當:「早伺候備馬,邀你何老爹到我這里,起身同往郊外送候爺。留下四名排軍,與來安、春鴻兩個跟轎往夏家去。」說畢,就歸金蓮房中來。那婦人未及他進房,就先摘了冠兒,亂挽烏雲,花容不整,朱粉懶施,渾衣兒〈扌歪〉在牀上。房內燈兒也不點,靜悄悄的。西門慶進來,便叫春梅,不應。只見婦人睡在牀內,叫着,只不做聲。西門慶便在牀上問道:「怪油嘴,你怎的恁個腔兒?」也不答應。被西門慶用手拉起來他,說道:「你如何悻悻的?」那婦人便做出許多喬張致來,把臉扭着,止不住紛紛的香腮上滾下淚來。那西門慶就是鐵石人,也把心來軟了,問他一聲兒。連忙一隻手摟着他脖子說:「怪油嘴,好好兒的,平白你兩個合甚麼氣?」那婦人半日方回言說道:「誰和他合氣來?他平白尋起個不是,對着人罵我是攔漢精趁漢精,趁了你來了!他是真材實料,正經夫妻!誰教你又來我這屋裡做甚麼?你守着他去就是了,省的我把攔着你。說你來家,只在我這屋裡纏!早是肉身聽着,你這幾夜只在我這屋裡睡來?白眉赤眼兒,你嚼舌根,一件皮襖,也說我不問他,擅自就問漢子討了。我是使的奴才丫頭?沒不往你屋裡與你磕頭去?為這小肉兒罵了那賊瞎淫婦,也說不管。偏有那些聲氣的!你是個男子漢,若是有張主的一拳柱定,那里有這些閒言悵語?怪不的俺每自輕自賤,常言道:『賤里買來賤里賣,容易得來容易捨。』趁將你家來,與你家做小老婆,不氣長!自古人善得人欺,馬善得人騎。便是如此。你看昨日生怕氣了他,在屋裡守着的是誰?請太醫的是誰?在跟前攛撥侍奉的是誰?苦惱俺每這陰山背後,就死在這屋裡,也沒個人兒來啾問!這個就見出那人的心來了!還教舍着那眼淚兒,走到後邊,與他賠個不是!」說着,那桃花臉上止不住又滾下珍珠兒,倒在西門慶懷裡嗚嗚咽咽,哭的捽鼻涕,彈眼淚。西門慶一面摟抱着,勸道:「罷麼,我的兒?我連日心中有事,你兩家各省這一句兒就罷了。你教我說誰的是?昨日要來看你,他說我來與你賠不是,不放我來。我往李嬌兒睡了一夜。雖然我和人睡,一片心只想着你!」婦人道:「罷麼,我也見出你那心來了。一味在我面上虛情假意,倒老還疼你那正經夫妻。他如今見替你懷着孩子,俺每一根草兒,拿甚麼比他!」被西門慶摟過脖子來,親了個嘴道:「怪油嘴,休要胡說!」只見秋菊拿進茶來,西門慶便道:「賊奴才,好乾淨兒!如何教他拿茶?」因問:「春梅怎的不見?」婦人道:「你還問春梅哩,他餓的只有一口游氣兒,那屋裡倘着不是?帶今日三四日,沒吃點湯水兒,一心只要尋死在那里。說他大娘對着人罵了他奴才,氣生氣死,整哭了三四日了。」這西門慶聽了,說道:「真?」婦人道:「莫不我哄你不成?你瞧去不是!」這西門慶慌過這邊屋裡,只見春梅容粧不整,雲髻斜歪,睡在炕上。西門慶叫道:「怪小油嘴,你怎的不起?」叫着他,只不做聲推睡。被西門慶雙關抱將起來。那春梅從酩子裡伸腰,一個鯉魚打挺,險些兒沒把西門慶埽了一交,早是抱的牢,有護炕倚住不倒。春梅道:「達達,起來了手。你又來理論俺每這奴才做甚麼?也沾辱了你這兩隻手!」西門慶道:「小油嘴兒,你大娘說了你兩句兒罷了!只顧使起性兒來了。說你這兩日沒吃飯?」春梅道:「吃飯不吃飯,你管他怎的?左右是奴才貨兒,死便隨他死了罷!我做奴才,一來也沒幹壞了甚麼事,並沒教主子罵我一句兒,攩我一下兒。做甚麼為這{入日}遍街搗遍巷的賊瞎婦,教大娘這等罵我!嗔俺娘不管我,莫不為瞎婦扯倒打我五板兒?等到明日韓道國老婆不來便罷,若來,你看我指與他,一頓好的不罵!原來送了這瞎淫婦來,就是個禍根!」西門慶道:「就是送了他來,也是好意。誰曉的為他合起氣來了?」春梅道:「他若肯放和氣些,我好意罵他?他小量人家。」西門慶道:「我來這里,你還不倒鍾茶兒我吃。那奴才手不乾淨,我不吃他倒的茶。」春梅道:「死了王屠,連毛吃猪。我如今走也走不動,在這里還教我倒甚麼茶!」西門慶道:「怪小油嘴兒,誰教你不吃些甚麼兒!」因說道:「咱每往那邊屋裡去,我也還沒吃飯哩。教秋菊後邊取菜兒、篩酒、烤菓餡餅兒、炊鮓湯咱每吃。」于是不由分訴,拉着春梅手,到婦人房內,分付秋菊:「拿盒子後邊取吃飯的菜兒去。」不一時,拿了一方盒菜蔬,一碗燒猪頭,一碗頓爛羊肉,一碗熬雞,一碗煎煿鮮魚和白米飯,四碗吃酒的菜蔬,海蜇荳芽菜,肉鮓蝦米 之類。西門慶分付春梅把肉鮓打上幾個雞豆,加上酸笋韮菜,和上一大碗香噴噴餛飩湯來,放下卓兒擺下。一面盛飯來,又烤了一盒菓餡餅兒。西門慶和金蓮並肩而坐,春梅在傍邊隨着同吃。三個你一杯,我一杯,吃了一更方散就睡。到次日,西門慶早起,約會何千戶來到,吃了頭腦酒 起身,同往郊外送侯巡撫去了。吳月娘這里先送了禮去,然後打伴坐大轎,排軍唱道,來安、春鴻跟隨,往夏指揮家來吃酒,看他娘子兒,不在話下。玳安、王經在家,只見午後時分,有縣前賣茶的王媽媽領着何九,來大門首尋問玳安:「老爹在家不在家?」玳安道:「王奶奶、何老人家,稀罕!今日那陣風兒吹你老人家來這里走走?」王婆子道:「沒勾當怎好來踅門踅戶?今日不因老九因為他兄弟的事,敢來央煩老爹,老身還不來哩。」玳安道:「老爹今日與侯爺送行去了。俺大娘也不在家。你老人家站站,等我進去對五娘說聲。」進入不多時,出來說道:「俺五娘請你老人家進去哩。」王婆道:「我敢進去?你引我兒,只怕有狗。」那玳安引他進入花園金蓮房門首,掀開簾子,王婆進去。見婦人家常戴着臥兔兒,穿着一身錦段衣裳,擦抹的如粉粧玉琢,正在房中炕上,腳登着爐臺兒,坐的磕瓜子兒。房中帳懸錦綉,牀設縷金,玩器爭輝,箱奩耀目。進去不免下禮,慌的婦人答禮,說道:「老王免了罷。」那婆子見畢禮,坐在炕邊頭。婦人便問:「怎的一向不見你?」王婆子道:「老身有心中想着娘子,只是不敢來親近。」問:「添了哥哥不曾?」婦人道:「有到好了。小產過兩遍,白不存。」又問:「你兒子有了親事?」王婆道:「還不曾與他尋,他跟客人淮上來家,這一年多,家中胡亂積賺了些小本經紀,買個驢兒,胡亂磨些麵兒,賣來度日。慢慢替他尋一個兒與他。」因問:「老爹不在家了?」婦人道:「他爹今日往門外與撫按官送行去了。他大娘也不在家。有甚話說?」王婆道:「老九有樁事,央及老身來對老爹說,他兄弟何十,乞賊攀着,見拿在提刑院老爹手裡問。攀他是窩主,本等與他無干,望乞老爹案下與他分豁分豁。等賊若指攀,只不准他就是了。何十出來,到日買禮來重謝老爹。有個話帖兒在此。」一面遞與婦人。婦人看了,說道:「你留下,等你老爹來家,我與他瞧。」婆子道:「老九在前邊伺候着哩,明日教它來討話罷。」婦人一面叫秋菊看茶來。須曳?秋菊拿了一盞茶來,與王婆吃了。那婆子坐着說道:「娘子,你這般受福勾了!」婦人道:「甚麼勾了!不惹氣便好!成日嘔氣不了在這里。」那婆子道:「我的奶奶,你飯來張口,水來溫手。這等插金帶銀呼奴使婢,又惹甚麼氣?」婦人道:「常言道說得好,三窩兩塊,大婦小妻。一個碗內兩張匙,不是湯着就抹着,如何沒些氣兒?」婆子道:「好奶奶,你比那個不聰明?趁着老爹這等好時月,你受用到那里是那里!」說道:「我明日使他來討話罷。」于是拜辭起身。婦人道:「老王,你多坐回去不是?」那婆子道:「難為老九只顧等我,不坐罷,改日再來看你。」那婦人也不留他留兒,就放出他來了。到了門首,又叮嚀玳安。玳安道:「你老人家去,我知道。等俺爹來家,我就稟。」何九道:「安哥,我明日早來討話罷。」于是和王婆一路去了。至晚,西門慶來家,玳安便把此事稟知西門慶。門慶到金蓮房看了帖子,交付與答應的收着:「明日到衙門中稟我。」一面又令陳經濟發初三日請人帖兒,瞞着春梅,又使琴童兒送了一兩銀子并一盒點心,到韓道國家,對着他說:「是與申二姐的,教他休惱。」那王六兒笑嘻嘻接了,說:「他不敢惱,多上覆爹娘,冲撞他春梅姑娘。」俱不在言表。至晚,月娘來家,穿着銀鼠皮襖,遍地金襖兒,錦藍裙,坐大轎,打着兩個燈籠,到家先拜見大妗子,眾人然後相見。西門慶正在上房吃酒,道了萬福。當下告訴:「夏大人娘子見了我去,好不喜歡。多謝重禮。今日也有許多親鄰堂客。原來夏大人有書來了,也有與你的書,明日送來與你。也只在這初六七起身,顧車搬取家小上京去也。」說了又說:「好歹教賁四送他家到京,就回來。賁四的那孩子長兒,今日與我磕頭,好不出跳了好個身段兒!嗔道他傍邊捧着,把眼只顧偷瞧我。我也忘了!他倒是夏大人娘子叫他改換了名字,叫做瑞雲:『過來與你西門奶奶磕頭。』他纔放下茶托兒,與我磕了四個頭,我與了他兩枝金花兒。如今夏大人娘子好不喜歡,擡舉他,也不把他當房裡人,只做親兒女一般看他。」西門慶道:「還是這孩子有福,若是別人家手裡,怎麼容得?不罵奴才,少椒末兒,又肯擡舉他?」被月娘瞅了一眼,說道:「硶說嘴的貨,是我罵了你心愛的小姐兒!」那西門慶笑了,說道:「他借了賁四押家小去,我線舖子教誰看?」月娘道:「關兩日也罷了。」西門慶道:「關兩日阻了買賣。近年節,紬絹絨線正快,如何關閉了舖子?到明日等再處。」說畢,月娘進裡間脫衣裳摘頭,走到那邊房內,和大妗子坐的,家中大小都來參見磕頭。是日,西門慶在後邊雪娥房中歇了一夜,早往衙門中去了。只見何九走來問玳安討信,與了玳安一兩銀子。玳安如此這般:「昨日爹來家,就替你說了。今日到衙門中,就開出你兄弟來放了。你往衙門首伺侯。」這何九聽言,滿心歡喜,一直走衙門前去了。西門慶到衙門裡坐廳,提出強盜來,每人又是一夾二十順腿。把何十開出來放了,另拿了弘化寺一名和尚頂缺,說強盜曾在他寺內宿了一夜。世上有如此不公事,正是:

「張公吃酒李公醉,  桑樹上脫枝柳樹上報。」

有詩為證:

「宋朝氣運已將終,  執掌提刑或不公;

畢竟難逃天地眼,  那堪激濁與揚清。」

那日,西門慶家中叫了四個唱的,吳銀兒、鄭愛月兒、洪四兒、齊香兒日頭向午就來了,都拿着衣裳包兒,齊到月娘房內,與月娘大妗子眾人磕了頭。月娘在上房擺茶與他們吃了。正彈着樂器,唱曲兒,與大妗子、月娘眾人聽。忽見西門慶從衙門中來家,進房來,四個唱的都放了樂器,笑嘻嘻向前,一齊與西門慶插燭也磕了頭。坐下,月娘便問:「你怎的衙門中這咱纔來?」西門慶告訴:「今日問理好幾樁事情。」因望着金蓮說:「昨日王媽媽來說何九那兄弟,今日我已開除來放了。那兩名強盜還攀扯他,教我每人打了二十,夾了一夾,拿了門外寺裡一個和尚頂缺,明日做文書送過東平府去。又是一起奸情事,丈母養女婿的。那女婿年小不上三十多歲,名喚宋得,原與這家是養老不歸宗女婿。落後親丈母死了,娶了個後丈母周氏。不上一年,把丈人死了。這周氏年小,守不得。就與他這女婿,常時言笑自若,漸漸在家嚷的人知道,住不牢。一日道他與丈母往鄉里娘家去,周氏便向宋得說:『你我本沒事,枉躭其名。今日在此山野空地,咱兩個成其夫妻罷!』這宋得就把周氏姦了。說一度以後,娘家回還,道通姦不絕。後因為責使女,被使女傳於兩鄰,纔首告官。今日取了供招,都一日送過去了。這一到東平府,姦妻之母,係緦麻之親,兩個都是絞罪!」潘金蓮道:「要着我,把學舌的奴才打的爛糟糟的。問了他死罪也不多!你穿着青衣抱黑柱,一句話就把主子弄了!」西門慶道:「也吃我把奴才拶了幾拶子好的,為你這奴才,一時小節不完,喪了兩個人性命!」月娘道:「大不正,則小不敬。母狗不掉尾,公狗不上身!大凡還是女婦人心邪,若是那正氣的,誰敢犯邊?」連四個唱的都笑道:「娘說的是。就是俺裡邊唱的,接了孤老的朋友,還使不的,休說外頭人家。」說畢,擺飯與西門慶吃了。忽聽前廳鼓樂響,荊都監老爹來了。西門慶連忙冠帶出迎,接至廳上敘禮,謝其厚賜,分賓主坐下。茶罷,如此這般告說:「宋巡按收了說帖,已向慨許。執事恭喜,必然在邇。」荊都監聽了,又轉身下坐作揖致謝:「老翁費心,提攜之力,銘刻難忘。」西門慶又說起:「周老總兵,生亦薦言一二,宋公必有主意。」談話間,忽報劉、薛二內相公公到,鼓樂迎接進來。西門慶階降禮相讓入廳,兩個敘禮。二位內相皆穿青螺絨蟒衣,寶石縧環,正中間坐下。次後周守禦到了,一處敘話。荊都監又問周守禦說:「四泉厚情,昨日宋公在尊府擺酒,與侯公送行。曾稱頌公之厚情,宋公已留神於中,高轉在即。」周守禦亦欠身致謝不盡。落後張團練、何千戶、王三官、范千戶、吳大舅、喬大戶陸續都到了。喬大戶冠帶青衣,四個伴當跟隨。進門見畢諸公,與西門慶大椅上四拜。階人問其恭喜之事,西門慶道:「舍親家在本府援例,新受恩榮義官之職。」周守禦道:「四泉令親,吾輩亦當奉賀。」喬大戶道:「蒙列位老爹盛情,豈敢動勞!」說畢,各分次序坐下。遍里遞上一道茶來,然後收拾上座。錦屏前玳筵羅列,畫堂內寶玩爭輝,階前動一派笙歌,席上堆滿盤異菓。良久,遞酒安席畢,各家僮僕上來接去衣服,歸席坐下。王三官再三不肯下來坐。西門慶道:「尋常罷了,今日在舍,權借一日,陪諸公上座。」王三官必不得已,左邊垂首坐了。須臾上罷湯飯,廚役上來割一道燒鵝,獻小割。下邊教坊回數隊舞吊畢,撮弄雜耍百戲,院本之後,四個唱的慢慢纔上來,拜見過了。個個粧扮花兒,人人珠翠仙裳。銀箏玉玩放嬌聲,倚翠偎頻笑語。正是:

「舞裙歌板逐時新,  散盡黃金只此身;

寄與富兒休暴殄,  儉如良藥可醫貧。」

不說當日劉內相坐首席,也賞了許多銀子。飲酒作歡,至一更時分方散。西門慶打發樂工賞錢出門,四個唱的都在月娘房內彈唱。月娘留下吳銀兒過夜,打發三個唱的。恁臨去,見西門慶在廳上,拜見拜見。西門慶分付鄭愛月兒:「你明日就拉了李桂姐兩個,還來唱一日。」那鄭愛月兒就知今日有王三官兒,不叫李桂姐來唱。笑道:「爹,你兵馬司倒了墻,賊走了!」又問:「明日請誰吃酒?」西門慶道:「都是親朋。」鄭月兒道:「有應二那花子,我不來。我不要見那醜冤家怪物!」西門慶道:「明日沒有他。」愛月兒道:「沒有他纔好。若有那怪攮刀子的,俺每不來。」說畢,磕了頭,揚長去了。西門慶看着收了家火,回到李瓶兒那邊,和如意兒睡了,一宿晚景題過。次日早,往衙門送問那兩起人犯過東平府去。回來家中擺酒,請吳道官、吳二舅、花大舅、沈姨夫、韓姨夫、任醫官、溫秀才、應伯爵并會中人,李智、黃四、杜三哥并家中二個夥計,十二張卓兒。席間正是李桂姐、吳銀兒、鄭愛月兒三個粉頭遞酒。李銘、吳惠、鄭奉三個小優兒彈唱。正遞酒中間,忽平安來報:「雲二叔新襲了職,來拜爹,送禮來。」西門慶聽言,連忙道:「有請。」只見雲離守穿着青紵絲補服員領,冠冕着,腰繫金帶,後邊伴當擡着禮物,先遞上揭帖與西門慶觀看,上寫:「新襲職山東清河右衞指揮同知,門下生雲離守頓首百拜。謹具土儀貂鼠十個,海魚一尾,蝦米 一包,臘鵝 四隻,臘鴨十隻,油紙簾二架,少申芹敬。」西門慶即令左右收了,連忙致謝。雲離守道:「在下昨日纔來家,今日特來拜老爹。」于是磕頭四雙八拜,說道:「蒙老爹莫大之恩,些少土宜,表意而已。」然後又與眾人敘禮拜見西。門慶見他居官,就待他不同,安他與吳二舅一卓坐了。連忙安下鍾筯,下了湯飯,腳下人俱打發攢盤酒肉。因問起發喪替職之事,這雲離守一一數言:「蒙兵部余爺憐其家兄在鎮病亡,祖職不動,還與了個本衛見任僉書。」西門慶歡喜道:「恭喜,恭喜。容日已定來賀。」當日眾人席上每位奉陪一杯,又令三個唱的奉酒。須臾把雲離守灌的醉了。那應伯爵在席上,如線兒提的一般,起來坐下,又和李桂姐和鄭月兒彼此互相戲罵不絕。這個罵他怪門神,白臉子撒根甚的貨!那個罵他是醜冤家怪物勞!朱八戒,坐在冷舖裡賊。罵道:「我把你這兩個女人,十撇鴉胡石影子布兒,朵朵雲兒了口惡心。」不說當日酒筵笑聲,花攢錦簇,〈舟光〉籌交錯,耍頑至二更時分方纔席散。打發三個唱的去了,西門慶歸上房宿歇。到次日,起來遲,正在上房擺粥吃了,穿衣要拜雲離守。只見玳安來說:「賁四在前邊請爹說話。」西門慶就知因為夏龍溪送家小之事,一面出來廳上。只見賁四向袖中取出夏指揮書來呈上,說道:「夏老爹要教小人送送家小往京裡去,不久就回。小人稟問道老爹,去不去?」西門慶看了書中言語,無非是敘其闊別,謝其早晚看顧家小,又借賁四攜送家小之事。因說道:「他既央你,你怎的不去?」因問:「幾時起身?」賁四道:「今早他大官府叫了小人去,分付初六日家小准上車起身。小人也得月半纔回來。」說畢,把獅子街舖內鑰匙,交遞與西門慶。門慶道:「你去,我教你吳二舅來替你開兩日舖子罷。」那賁四方纔拜辭出門,往家中收拾行裝去了。這西門慶就冠冕着出門,僕從跟隨,去拜雲指揮去了。那日是大妗子家去,叫下轎子門首侍候。也是合當有事,月娘裝了兩盒子茶食點心下飯,上房管待大妗子,出門首上轎。只見畫童兒小廝,躲在門傍鞍子房兒,大哭不止。那平安兒只顧扯他。那小夥子越扯越哭起來,被月娘等聽見。送出大妗子上轎去了,便問平安兒:「賊囚,你平白拉他怎的?惹的他恁怪哭!」平安道:「溫師父那邊叫他,他白不去,只是罵小的。」月娘道:「你教他好好去罷。」因問道:「小廝,你師父那邊叫,去就是了,怎的哭起來?」那畫童道:「又不管你事,我不去罷了,你扯我怎的?」月娘道:「你因何不去?」那小廝又不言語。金蓮道:「這賊小囚兒就是個肉侫賊,你大娘問你,怎的不言語?」被平安向前打了一個嘴巴,那小廝越發大哭了。月娘道:「怪肉根子,你平白打他怎的?你好好教他說,怎的不去?」正問着,只見玳安騎了馬進來,月娘問道:「你爹來了?」玳安道:「被雲叔留住吃酒哩。使我送衣裳來了,帶毡巾去。」看見畫童兒哭,便問:「小大官兒,怎的號啕痛剜墻拱?」平安道:「對過溫師父叫着,他不去,反哭罵起我來了。」玳安道:「我的哥哥,溫師父叫你,仔細他有名的溫屁股,一日沒屁股也成不的!你每常怎麼挨他的,今日如何又躲起來了?」月娘罵道:「怪囚根子,怎麼溫屁股?」玳安道:「娘自問他就是個。」那潘金蓮得不的風兒。就是雨兒。一面叫過畫童兒來,只顧問他:「小奴才,你實說,他喚你做甚麼?你不說着,我教你大娘打你。」逼問那小廝急了,說道:「他只要哄着小的,把他行貨子放在小的屁股裡,弄的脹脹的疼起來。我說你還不快拔出來,他又不肯拔,只顧來回動。且教小的拿出來,跑過來。他又來叫小的。」月娘聽了,便喝道:「怪賊小奴才兒,還不與我過一邊去!也有這六姐,只管好審他,說的硶死了!我不知道,還當好話兒,側着耳朵兒聽!他是個不上蘆葦的行貨子!人家小廝與它使,都背地幹這個營生!」那金蓮道:「大娘,那個上蘆葦的肯幹這營生?冷舖睡的花子,纔這般所為!」孟玉樓道:「這蠻子他有老婆,怎生這等沒廉耻?」金蓮道:「他來了這一向,俺每就沒見他老婆怎生這等。」平安道:「怎麼樣兒,娘們合勝看的見他。他但往那里去,每日只出鎖見住了。這半年我只見他坐轎子往娘家去了一遭,沒到晚就來家了。每常幾時出個門兒來?只好晚夕門首出來倒榪子,走走兒罷了。」金蓮道:「他那老婆,也是個不長俊的行貨子!嫁了他,怕不的也沒見個天日兒。敢每日只在屋裡坐天牢裡?」說了回,月娘同眾人回後邊去了。西門慶約莫日落時分來家,到上房坐下。月娘問道:「雲夥計留你坐來?」西門慶道:「他在家見我去,甚是無可不可,旋放卓兒留我坐,打開一罈酒陪我吃。如今衞中荊南崗陞了,他就挨着掌印。明日連我和他喬親家,就是兩分賀禮。眾同僚都說了,要與他挂軸子。少不的教溫葵軒做兩篇文章,早些買軸子寫下。」月娘道:「還纏甚麼溫葵軒,鳥葵軒哩!平白安扎恁樣行貨子,沒廉恥!傳出去教人家知道,把醜來出盡了!」西門慶聽言,諕了一跳,便問:「怎麼的?」月娘道:「你別要來問我,你問你家小廝去。」西門慶道:「是那個小廝?」金蓮道:「情知是誰,畫童賊小奴才!俺送大妗子去,他正在門首哭。如此這般,溫蠻子弄他來!」這西門慶聽了,還有些不信。便道:「你叫那小奴才來,等我問他。」一面使玳安兒前邊把畫童兒叫到上房跪下,西門慶要拿拶子拶他,便道:「賊奴才,你實說,他叫你做甚麼?」畫童兒道:「他叫小的,要灌醉了小的,要幹小營生兒。今日小的害疼,躲出來了,不敢去。他只顧使平安叫,又打小的。教娘出來看見了。他常時問爹家中各娘房裡的事,小的不敢說。昨日爹家中擺酒,他又教唆小的偷銀器兒家火與他。又某日他望他倪師父去,拿爹的書稿兒與倪師父瞧,倪師父又與夏老爹瞧。」這西門慶不聽便罷,聽了便道:「畫虎畫龍難畫骨,知人知面不知心。我把他當個人看,誰知人皮包狗骨東西,要他何用?」一面喝令畫童兒起去,分付:「再不消過那邊去了。」那畫童磕了頭起來,往前邊去了。西門慶向月娘:「怪道前日翟親家,說我『機事不密則害成。』我想來沒人,原來是他把我的事透泄與人,我怎得曉的!這樣狗背石東西,平白養在家做甚麼!」月娘道:「你和誰說,你家又沒孩子上學,平白招攬個人在家養活,看寫禮帖兒。怪不的你我,我家有這些禮帖書柬寫,饒養活着他,還教他弄乾坤兒。家裡底事往外打探。」西門慶道:「不消說了,明日教他走道兒就是了。」一面叫將平安來了,分付:「對過對他說,家老爹要房子堆貨,教溫師父轉尋房兒便了。等他來見我,你在門首只回我不在家。」那平安兒應諾去了。西門慶告月娘說:「今日賁四來辭我,初六日起身,與夏籠溪送家小往東京去。我想來線舖子沒人,倒好教他二舅來,替他開兩日兒。左右與來昭一遞三日上宿,飯倒都在一處吃,好不好?」月娘道:「好不好隨你叫他去,我不管你,省的人又說招顧了我的兄弟。」西門慶不聽,于是使棋童兒:「請你二舅來。」不一時,請吳二舅到,在前廳陪他坐的吃酒,把鑰匙交付與他,明日同來昭早往獅子街開舖子去,不在話下。都說溫秀才見畫童兒一夜不過來睡,心中省恐。到次日,平安走來說:「家老爹多上覆溫師父,早晚要這房子堆貨,教師父別尋房兒罷。」這溫秀才聽了,大驚失色,就知畫童兒有甚話說。穿了衣巾,要見西門慶說話。平安兒道:「俺爹往衙門中去了,還未來哩。」比及來,這溫秀才又衣巾過來伺候,具了一篇長柬,遞與琴童兒,琴童又不敢接,說道:「俺爹纔從衙門中來家辛苦,後邊歇去了,俺每不敢稟。」這溫秀才就知疎遠他,一面走到倪秀才家商議,還搬移家小往舊處住去了。正是:

「誰人汲得西江水,  難洗今朝一面羞。」

「靡不有初鮮克終,  交情似水淡長情;

自古人無千日好,  果然花無摘下紅。」

畢竟未知後來如何,且聽下回分解:

