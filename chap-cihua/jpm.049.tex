%# -*- coding: utf-8 -*-
%!TEX encoding = UTF-8 Unicode
%!TEX TS-program = xelatex
% vim:ts=4:sw=4
%
% 以上设定默认使用 XeLaTex 编译,并指定 Unicode 编码,供 TeXShop 自动识别

%第四十九回 
\chapter{西門慶迎請宋巡按\KG 永福寺餞行遇胡僧}


\begin{showcontents}{}



「寬性寬懷過幾年,  人死人生在眼前,

隨高隨下隨緣過,  或長或短莫埋怨,

自有自無休歎息,  家貧家富總由天,

平生衣祿隨緣度,  一日清閑一日仙。」

話說夏壽到家,回覆了話。夏提刑隨即就來拜謝西門慶,說道:「長官活命之恩。不是托賴長官餘光,這等大力量,如何了得?」西門慶笑道:「長官放心,料著你我沒曾過為,隨他說去便了。老爺那里,自有個明見。」一面在廳上放卓兒留飯,談笑至晚,方纔作辭回家。到次日依舊入衙門裡理事,不在話下。卻表巡按曾公,見本上去不行,就知道二官打點了,心中忿怒。因蔡太師所陳七事,內多乖方舛訛,皆損下益上之事。即赴京見朝覆命,上了一道表章,極言:「天下之財,貴于通流。取民膏以聚京師,恐非太平之治。民間結糶俵糴之法不可行,當十太錢不可用,鹽鈔法不可屢更。臣聞民力殫矣,誰與守邦?」蔡京大怒,奏上徽宗天子,說他大肆倡言,阻撓國事。那時將曾公付吏部考察,黜為陝西慶州知州。陝西巡按御史宋盤,就是學士蔡攸之婦兄也。太史陰令盤就劾其私事,逮其家人,煆煉成獄,將孝序除名,竄于嶺表,以報其仇,此係後事,表過不題。再說西門慶在家,一面使韓道國與喬大戶外甥崔本,拏倉鈔早往高陽關戶部韓爺那里趕著掛號。留下來保家中,定下果品,預備大卓面酒席,打聽祭御史舡到。一日,來保打聽得他與巡按宋御史舡,一同京中起身,都行至東昌府地方,使人先來家通報。這里西門慶就會夏提刑起身。知府州縣及各衛有司官員,又早預備祇應人馬,鐵桶相似。來保從東昌府舡上,就先見了蔡御史,送了下程。然後西門慶與夏提刑出郊五十里迎接。到新河口地名百家村,先到蔡御史舡上拜見了,備言邀請宋公之事。蔡御史道:「我知道,一定同他到府。」那時東平胡知府及合屬州縣,方面有司,軍衛官員,吏典生員,僧道陰陽,都具連名手本,伺候迎接。帥府周守備、荊都監、張團練,都領人馬披執跟隨,清畢傳道,雞犬皆隱跡,鼓吹進東平府察院。各處官員都見畢,呈遞了文書,安歇一夜。到次日,只見門吏來報:「巡鹽蔡爺來拜。」宋御史急令撤去公案,連忙整冠出迎。兩個敘畢禮數,分賓主坐下。少頃,獻茶已畢。宋御史便問:「年兄事期,幾時方行?」蔡御史道:「學生還待一二日。」因告說:「清河縣有一相識西門千兵,乃本處巨族。為人清慎,富而好禮。亦是蔡老先生門下,與學生有一面之交。蒙他遠接,學生正要到他府上拜他拜。」宋御史問道:「是那個西門千兵?」蔡御史道:「他如今見是本處提刑千戶,昨日已參見過年兄了。」宋御史令左右取遞的手本來,看見西門慶與夏提刑名字,說道:「此莫非與翟雲峰有親者?」蔡御史道:「就是他。如今見在外面伺候,要央學生奉陪年兄,到他家一飯。未審年兄尊意若何?」宋御史道:「學生初到此處,不好去得。」蔡御史道:「年兄怕怎的?既是雲峰分上,你我走走何害?」于是分付看轎,就一同起行;一面傳將出來。西門慶知了此消息,與來保、賁四騎快馬先奔來家,預備酒席。門首搭照山綵棚,兩院樂人奏樂,叫海鹽戲并雜耍承應。原來宋御史將各項伺候人馬,都令散了,只用幾隊藍旗清道,官吏跟隨,與蔡御史坐兩頂大轎,打著雙簷傘,同往西門慶家來。當時哄動了東平府,抬起了清河縣,都說巡按老爺也認的西門大官人,來他家吃酒來了?慌的周守備、荊都監、張團練各領本哨人馬,把住左右街口伺候。西門慶青衣冠帶,遠遠迎接,兩邊鼓樂吹打。到大門道,下了轎進去。宋御史與蔡御史都穿著大紅獬豸繡服,烏紗皂履,鶴頂紅帶,從人執著兩把大扇。只見五間廳上,湘簾高捲,錦屏羅列。正面擺兩張吃看卓席,高頂方糖,定勝簇盤 ,十分齊整。二官揖讓進廳,與西門慶敘禮。蔡御史家人具贄見之禮,兩端湖紬,一部文集,四袋芽茶,一面端溪硯。宋御史只投了個宛紅單拜帖,上書:「侍生宋喬年拜。」向西門慶道:「久聞芳譽,學生初臨此地,尚未盡情,不當取擾。若不是蔡年兄見邀,同來進拜,何以幸接尊顏!」慌的西門慶倒身下拜,說道:「僕乃一介武官,屬于按臨之下。今日幸蒙清顧,蓬蓽生光。」于是鞠恭展拜,禮容甚謙。宋御史亦答禮相還,敘了禮數。當下蔡御史讓宋御史居左,他自在右。西門慶垂首相陪。茶湯獻罷,階下蕭韶盈耳,鼓樂喧闐,動起樂來。西門慶遞酒安席已畢,下邊呈獻割道,說不盡餚列珍羞,湯陳桃浪,酒泛金波,端的歌舞聲容,食前方丈。西門慶知道手下跟從人多,階下兩位轎上跟從人,每位五十瓶酒,五百點心,一百斤熟肉,都領下去。家人吏書門子人等,另在廂房中管待,不必用說。當日西門慶這席酒,也費勾千兩金銀。那宋御史又係江西南昌人,為人浮躁。只坐了沒多大回,聽了一摺戲文,就起來?慌的西門慶再三固留。蔡御史在傍便說:「年兄無事,再消坐一時。何遽回之太速耶?」宋御史道:「年兄還坐坐,學生還欲到察院中處分些公事。」西門慶早令手下把兩張卓席,連金銀器已都裝在食盒內,共有二十擡,叫下人夫伺候。宋御史的一張大卓席,兩壜酒,兩牽羊,兩對金絲花,兩疋段紅,一副金臺盤,兩把銀執壺,十個銀酒盃,兩個銀折盃,一雙牙筯。蔡御史的也是一般的,都遞上揭帖。宋御史再三辭道:「這個我學生怎麼敢領?」因看著蔡御史。蔡御史道:「年兄貴治所臨,自然之道。我學生豈敢當之?」西門慶道:「些須微儀,不過乎侑觴而已,何為見外?」比及二官推讓之次,而卓席已擡送出門矣。宋御史不得已,方令左右收了揭帖,向西門慶致謝,說道:「今日初來荊識,既擾盛席,又承厚貺,何以克當?餘容圖報不忘也!」因向蔡御史道:「年兄還坐坐,學生告別。」于是作辭起身。西門慶還要遠送,宋御史不肯,急令請回,舉手上轎而去。西門慶回來,陪侍蔡御史,解去冠帶,請去捲棚內後坐。因分付把樂人都打發散去,只留下戲子。西門慶令左右重新安放卓席,擺設珍羞菓品上來,二人飲酒。蔡御史道:「今日陪我這宋年兄坐,便僭了。又和管待盛庫酒器,何以克當!」西門慶笑道:「微物惶恐,表意而已。」因問道:「宋公祖尊號?」蔡御史道:「號松原,松樹之松,原泉之原。」又說起:「頭里他再三不來。被我學生因稱道四泉盛德,與老先生那邊相熟,他纔來了。他也知府上與雲峰有親。」西門慶道:「想必翟親家有一言于彼。我觀宋公為人,有些蹺蹊。」蔡御史道:「他雖故是江西人,倒也沒甚蹺蹊處。只是今日初會,怎不做些模樣?」說畢,笑了。西門慶便道:「今日晚了,老先生不回舡上去罷了。蔡御史道:「我明早就要開舡長行」。西門慶道:「請不棄在舍留宿一宵,明日學生長亭送餞。」蔡御史道:「過蒙愛厚。」因分付手下人:「都回門外去罷,明日來接。」眾人都應諾去了,只留下兩個家人伺候。西門慶見手下人都去了,走下席來,來叫玳安兒,附耳低言,如此這般,分付:「即去院中,坐名叫了董嬌兒、韓金釧兒兩個,打後門裡,用轎子擡了來,休交一人知道。」那玳安一面應諾去了。

西門慶復上席陪蔡御史吃酒。海鹽子弟在傍歌唱。西門慶因問:「老先生到家多少時就來了?令堂老夫人起居康健麼?」蔡御史道:「老母倒也安。學生在家,不覺荏苒半載。回來見朝,不想被曹禾論劾,將學生敝同年一十四人之在史館著,一時皆黜授外職。學生便說在西臺,新點兩淮巡鹽。宋年兄便在貴處巡按,他也是蔡老先生門下。」西門慶問道:「如今安老先生在那里?」蔡御史道:「安鳳山他已陞了工部主事,往荊州催儹皇木去了。也待好來也。」說畢,西門慶交海鹽子弟上來遞酒。蔡御史分付:「你唱個漁家傲我聽。」子:

「別後杳無書,不疼不痛病難除。恨凄凄旅館有誰相知,魚沉不見雁傳書。三山美人知何處?眠思夢想,此情為誰,懨懨憔瘦,一似風中柳絮。知他幾時再得重相會。」

〔皂羅袍〕

「滿日黃花初綻,怪淵明怎不回還。交人盼得眼睛穿,冤家怎不行方便。從伊別後,相思病纏;昏昏如醉,汪汪淚漣。知他幾時再得重相見?」

「我愛他桃花為面,笋生成十指纖纖。我愛他春山淡淡柳拖煙,我愛他清俊一雙秋波眼烏鴉堆鬢。青絲翠綰;玳鈎月釣,丹霞襯臉。教人想得肝腸斷。」

「戍鼓鼕鼕初轉,聽樓頭畫角聲殘。搥床搗枕數千番,長吁短嘆千千遍。精神撩亂,語言倒顛;忘冷廢寢,和衣淚漣。終朝懞憧昏沉倦。」

「我為你終朝思念,在那里耍笑貪歡。忽然想起意懸懸,一番題起一番怨。恩深如海,情重似山;佳期非偶,離別最難。常言道:藕斷絲不斷。」

正唱著,只見玳安走來請西門慶下邊說話。玳安道:「叫了董嬌兒、韓金釧兒打後門來了,在娘

房裡坐著哩。」西門慶道:「你分付把轎子擡過一邊纔好。」玳安道:「擡過一邊了。」這西門慶走至上房,兩個唱的向前磕了頭。西門慶道:「今日請你兩個來,晚夕在山子下扶侍你蔡老爹。他如今見在巡按御史,你不可怠慢了他。用心扶侍他,我另酬答你兩個。」那韓金釧兒笑道:「爹不消分付,俺每知道。」西門慶因戲道:「他南人的營生,好的是南風。你每休要扭手扭腳的。」董嬌兒道:「娘在這裡聽著,爹你老人家羊角葱靠南牆,越發老辣。已是了;王府門首磕了頭,俺們不吃這井里水了。」這西門慶笑的往前邊來。走到儀門首,只見來保和陳經濟拏著揭帖走來,與西門慶看。說道:「剛纔喬親家爹說,趁著蔡老爹這回閑,爹倒把這件事,對蔡老爹說了罷。只怕明日起身忙了。」西門慶道:「交姐夫寫了俺兩個名字在此,你跟了來。」那來保跟到捲棚槅子外邊跪著。西門慶飲酒中間,因題起:「有一事在此,不敢于瀆。」蔡御史道:「四泉有甚事,只顧分付,學生無不領命。」西門慶道:「去歲因舍親那邊,在邊上納過些粮草,坐派了有些鹽引,正派在貴治楊州支鹽。只是望乞到那里,青目青目,早些支放,就是愛厚。」因把揭帖遞上去。蔡御史看了,上面寫著:「商人來保、崔本,舊派准鹽三萬引,乞到日早掣。」蔡御史看了笑道:「這個甚麼打緊?」一面把來保叫至近前跪下,分付:「與你蔡爺磕頭。」蔡御史道:「我到楊州,你等逕來察院見我。我比別的商人早掣取你鹽一個月。西門慶道:「老先生下顧,早放十日就勾了。」蔡御史把原帖就袖在袖內,一面書童傍邊斟上酒。子弟又唱下山虎:

「中秋將至,漸覺心酸。只見穿窗月,不見故人還。聽叮噹砧聲滿耳。嘹嚦嚦北雁南還,怎不交人心中慘然?料想相思,斷送少年。黃昏後,更漏殘,把銀燈剔盡方眠。」

「當初携手,月下並肩。說下山盟海誓,對天禱告。若有個負意忘恩,早歸九泉。一向如何音信遠,空教我卜金錢,廢寢忘餐,有誰見怜?黃昏後,更漏殘,把銀燈剔盡方眠。」

〔尾聲〕

「蒼天若肯行方便,早遣情人到枕邊,免使書生獨自眠。」

唱畢,當下掌燈時分,蔡御史便說:「深擾一日,酒告止了罷。」因起身出席,左右便欲掌燈。西門慶道:「且休掌燭。請老先生後邊更衣。」于是從花園里遊翫了一回,讓至翡翠軒。那里又早湘簾低簇,銀燭熒煌,設下酒席完備。海鹽戲子,西門慶已命手下管待酒飯,與了二兩賞錢,打發去了。書童把捲棚內家活收了,關上角門,只見兩個唱的,盛粧打扮,立于階下,向前花枝招颭磕頭。但見:

「綽約容顏金縷衣, 香塵不動下階墀;

時來水濺羅裙濕, 好似巫山行雨歸。」

蔡御史看見,欲進不能,欲退不可。便說道:「四泉,你如何這等愛厚,恐使不得!」西門慶笑道:「與昔日東山之遊,又何別乎?」蔡御史道:「恐我不如安石之才,而君有王右軍之高致矣。」于是月下與二妓攜手,不啻恍若劉、阮之入天台。因進入軒內,見文物依然。因索布筆,要留題。西門慶即令書童,連忙將端溪硯,研的墨濃,拂下錦箋。這蔡御史終是狀元之才,拈筆在手,文不加點,字走龍蛇,燈下一揮而就,作詩一首。詩曰:

「不到君家半載餘,  軒中文物尚依稀,

雨過書童開樂圃,  風回仙子步花臺;

飲將醉處鍾何急,  詩到成時漏更催,

此去又添新悵望,  不知何日是重來?」

寫畢,交書童粘于壁上,以為後日之遺焉。因問二妓:「你等叫甚名字?」一個道:「小的姓董,名喚嬌兒,他叫韓金釧兒。」蔡御史又道:「你二人有號沒有?」董嬌兒道:「小的無名娼妓,那討號來?」蔡御史道:「你等休要太謙。」問至再三,韓金釧兒方說:「小的號玉卿。」董嬌兒道:「小的賤號薇仙。」蔡御史一聞「薇仙」二字,心中甚喜,遂留意在懷。令書童取棋卓來,擺下棋子。蔡御史與董嬌兒兩個著棋。西門慶陪侍。韓金釧兒把金樽,在旁邊遞酒。書童拍手歌唱玉芙蓉。唱道:

「東風柳絮飄,玉砌蘭芽小,這春光艷冶,巧鬬難描。墻頭紅粉紅佳人笑,蹴罷鞦韆香汗消。尋芳興,不辭路遙。我只見酒旗搖曳杏花稍。」

唱畢,蔡御史嬴了董嬌兒一盤棋。董嬌兒吃過,回奉蔡御史。韓金釧兒這里,遞與西門慶,陪飲一盃。書童又唱道:

「風吃蕉尾翻,雨灑荷珠亂,見坐人盤鬢如蟬。湘紈半掩芙蓉面,綵袖輕飄賽小蠻。秋波臉,雨情牽好難。引的人意遲寂寞淚闌干。」

飲了酒,兩人又下。董嬌兒贏了,連忙遞酒一盃與蔡御史。西門慶在傍,又陪飲一盃。書童又唱:

「黃花遍地開,百草皆凋敗,小蛩吟唧唧空階。牛郎夜夜依然在,織女緣何不見來?懨懨害,糊突夢怎猜?我會他激滴濕,表記鳳頭鞋。」

唱畢,蔡御史道:「四泉,夜深了,不勝酒力了。」于是走出外邊來,跕立在于花下。那時正是四月半頭時分,月色纔上。西門慶道:「老先生,天色還早哩。還有韓金釧,未曾賞他一盃酒。」蔡御史道:「正是,你喚他來,我就此花下立飲一盃。」于是韓金釧兒拏大金桃盃滿斟一杯,用纖手捧遞上去,董嬌兒在傍捧菓。書童拍手又唱風四個:

「梨花散亂飛,不見遊蜂翅,小窗前鵲踏枯枝。愁聞冒雪尋梅至,忽聽銅壺更漏遲。傷心事,把離情自思。我為他寫情書,閣不住筆尖兒。」

蔡御史吃過,斟上一盃賞與韓金釧兒,因告辭道:「四泉,今日酒太多了,令盛价收過去罷。」于是與西門慶握手相語,說道:「賢公盛情盛德,此心懸懸。若非斯文骨肉,何以至此?向日所貸,學生耿耿在心,在京已與雲峰表過。倘我後日有一步寸進,斷不敢有辜盛德!」西門慶道:「老先生何出此言?倒不消介意。」那韓金釧兒見他一手拉著董嬌兒,知局就往後邊去了。到了上房里,月娘便問:「你怎的不陪他睡來了?」韓金釧笑道:「他留下董嬌兒了。我不來,只在那裡做甚麼?」良久,西門慶亦告了安置,進來。叫了來興兒,分付;「明日早五更,打發食盒酒米,點心下飯。叫了廚役跟了往門外永福寺去,那里與你蔡老爹送行。兩個小優兒答應,休要誤了。」來興兒道:「家里二娘上壽,沒人看來。」西門慶道:「留下棋童兒買東西,叫廚子後邊大灶上做罷。」不一時,書童、玳安收下家活來。又討了一壺好茶,往花園里去,與蔡老爹漱口。翡翠軒書房,床上舖陳衾枕,俱各完備。蔡御史見董嬌兒手中拏著一把湘妃竹泥金面扇兒,上面水墨畫著一種湘蘭,平溪流水。董嬌兒道:「敢煩老爹賞我一首詩在上面。蔡御史道:「無可為題,就指著你這薇仙號。」于是燈下來興,拈起筆來,寫了四句在上:

「小院閑庭寂不譁,  一池月上浸窗紗;

邂逅相逢天未晚,  紫薇郎對紫薇花。」

寫畢,那董嬌兒連忙拜謝了,兩個收拾上床就寢。書童、玳安與他家人在明間里睡,一宿晚景不題。次日早辰,蔡御史與了董嬌兒一兩銀子,用紅布大包封著。到于後邊,拏與西門慶瞧。西門慶笑說道:「文職的營生,他那里有大錢與你?這個就是上上籤了。」因交月娘每人又與了他五錢,早從後門打發他去了。書童舀洗面水,打發他梳洗穿衣。西門慶出來,在廳上陪他吃了粥。手下又早伺候轎馬來接,與西門慶作辭,謝了又謝。西門慶又道:「學生日昨所言之事,老先生到彼處,學生這里書去,千萬留神一二,足仞不淺。」蔡御史道:「休說賢公華扎下臨,只盛价有片布到,學生無不奉行。」說畢,二人同上馬。左右跟隨出城外,到于永福寺,借長老方丈,擺酒餞行。來興兒與廚役,早已安排卓席停當。李銘、吳惠兩個小優彈唱。數盃之後,坐不移時,蔡御史起身。夫馬坐轎,在于山門外伺候。臨行,西門慶說起苗青之事:「乃學生相知,因詿誤在舊大巡曾公案下,行牌往揚州案候捉他。此事情已問結了。倘見宋公,望乞借重一言,彼此感激。蔡御史道:「這個不妨。我見宋年兄說,設使就提來,放了他去就是了。」西門慶又作揖謝了。看官聽說:後來宋御史往濟南去,河道中又與蔡御史會在那舡上,公人揚州提了苗青來。蔡御史說道:「此係曾公手裡案外的,你管他怎的?」遂放回去了。倒下詳去東平府,還只把兩個舡家決不待時,安童便放了。正是:

「心事如此如此,  天理未然未然。」

有詩單表人情之有虧人處。詩曰:

「公道人情兩是非,  人情公道最難為;

若依公道人情失,  順了人情公道虧。」

胡知府已受了西門慶夏提刑囑託,無不做分上。要說此係後事。當日西門慶要送至舡上。蔡御史不肯,說道:「賢公不消遠送,只此告別。」西門慶道:「萬惟保重,容差小价問安。」說畢,蔡御史上轎而去。西門慶回到方丈坐下,長老走來遞茶,頭戴僧伽帽,身披袈裟,小沙彌拿著茶托,遞茶去,合掌道了問訊。西門慶答禮相還。見他雪眉交白,便問:「長老多大年紀?長老道:「小僧七十有五。」西門慶道:「倒還這等康健!」因問:「法號稱呼甚麼?」長老道:「小僧法名道堅。」「有幾位徒弟?」長老道:「止有兩個小徒,本寺也有三十余僧行。」西門慶道:「你這寺院,倒也寬大,只是欠修整。」長老道:「不瞞老爹說,這座寺,原是周秀老爹蓋造,常住裏沒錢粮修理,丟得壞了。」西門慶道:「原來就是你守備府周爺的香火院。我見他家庄子不遠,不打緊處。你稟了你周爺寫個緣簿,一般別處也再化著。來我那里,我也資助你些布施。」道堅連忙合掌問訊謝了。西門慶分付玳安兒,書袋內取一兩銀子,謝長老:「今日打攪長老這里。」道堅道:「小僧不知老爺來,不曾預備齋供。」西門慶道:「我要往後邊更更衣去。」道堅連忙叫小沙彌開便門。西門慶更了衣,因見方丈後面五間大禪堂,有許多雲遊和尚,在那里敲著木魚念經。西門慶不因不由,信步走入裡面觀看。見一個和尚,形骨古怪,相貌搊搜。生的豹頭凹眼,色若紫肝。戴了雞蠟箍兒,穿一領肉紅直裰。頦下髭鬚亂拃,頭上有一腦光簷。就是個形容古怪真羅漢,木除火性獨眼龍。在禪床上,旋定過去了。垂著頭,把脖子縮到腔子裏,鼻口中流下玉筋來。西門慶口中不言,心內暗道:「此僧必然是個有手段的高僧;不然,如何有此異相?等我叫醒他,問他個端的。」于是應聲叫那位僧人:「你是那里人氏?何處高僧?雲遊到此?」叫了頭一聲,不答應;第二聲,也不言語,第三聲,只見這個僧人,在禪床上把身上打了個挺,伸了伸腰,睜開一隻眼,跳將起來,向西門慶點了點頭兒,粗聲應道:「你問我怎的?貧僧行不問名,坐不改姓,乃西域天竺國密松林齊腰峰寒庭寺下來的胡僧,雲遊至此,施藥濟人。官人,你叫我有甚話說?」西門慶道:「你既是施藥濟人,我問你求些滋補的藥兒,你有也沒有?」胡僧道:「我有!我有!」又道:「我如今請你到家,你去不去?」胡僧道:「我去!我去!」西門慶道:「你說去,即此就行。」那胡僧直豎起身來,向床頭取過他的鐵柱杖來拄著,背上他的皮褡褳,褡褳內盛著兩個藥葫蘆兒,下的禪堂,就往外走。西門慶分付玳安,叫了兩個驢子,同師父先往家去,等著我就來。胡僧道:「官人不消如此。你騎馬只顧先行,貧僧也不騎頭口,管情比你先到。」西門慶道:「已定是個有手段的高僧,不然如何這等朗言?」恐怕他走了,分付玳安好歹跟著他同行。于是作辭長老上馬,僕從跟隨,逕直進城來家。那日四月十七日,不想是王六兒生日,家中又是李嬌兒上壽,有堂客吃酒。後晌時分,只見王六兒家沒人使,使了他兄弟王經來請西門慶。分付他宅門首,只尋玳安兒說話。不見玳安在門首,只顧立,立了約一個時辰。正值月娘與李嬌兒送院里李媽媽出來上轎。看見一個十五六歲扎包髻兒小廝,問:「是那里的?」那小廝三不知走到根前,與月娘磕了個頭,說道:「我是韓家,尋安哥說話。」月娘問:「那安哥?」平安在傍邊,恐怕他知道是王六兒那里來的,恐怕他說岔了話,向前把他拉過一邊,對月娘說:「他是韓家夥計家使了來尋玳安兒,問韓夥計幾時來?」以此哄過,月娘不言語,回後邊去了。不一時,玳安與胡僧先到門首,走的兩腿皆酸,渾身是汗,抱怨的要不的。那胡僧體貌從容,氣也不喘。平安把王六兒那邊使了王經來請爹尋他說話一節,對玳安兒說了:「不想大娘正送院里李奶奶出來,門首上轎,看見。他冒冒勢勢,走到根前,與大娘磕頭。大娘問他,說我是韓家的,早是我在傍邊,拉過一邊。落後大娘問我,我說是韓夥計家的,使他來問他韓夥計幾時來?大娘纔不言語了。早是沒曾禡覺出來。等住回娘若問你,也是這般說。」那玳安走的睜睜的,只顧搧扇子:「今日造化低的也,怎的平白爹交我領了這賊禿囚來,好近遠兒,從門外寺里,直走到家。路上通沒歇腳兒。走的我上氣不接著下氣兒!爹交顧驢子與他騎,他又不騎。便便走著沒事沒事的,難為我這兩條腿了!把鞋底子也磨透了,腳也踏破了,攘氣的營生!」平安道:「爹請他來家做甚麼?」玳安道:「誰知道?他說問他討甚麼藥哩!」正說著,只聞唱道之聲。西門慶到家,看見胡僧在門首,說道:「吾師乃人中神也,果然先到。」一面讓至裏面大廳上坐。西門慶叫書童接了衣裳,換了小帽,陪他坐的。那胡僧睜眼觀見廳堂高遠,院于深沉,門上掛的是龜背紋蝦鬚織抹綠珠簾,地下舖獅子滾綉毬絨毛線毯,正當中放一張蜻蜓腿螳螂肚皂色起楞的卓子,卓子上安著縧環樣須彌座大理石屏風,週圍擺的都是泥鰍頭楠木靶腫觔的校椅,兩壁掛的畫,都是紫竹桿兒綾邊瑪瑙軸頭。正是:

「鼉皮畫鼓振庭堂, 烏木春擡盛酒器。」

胡僧看畢,西門慶問道:「吾師用酒不用?」胡僧道:「貧僧酒肉齊行。」西門慶一面分付小廝:「後邊不消看素饌,拿酒飯來。」那時正是李嬌兒生日,廚下餚饌下飯都有。安放卓兒,只顧拿上來。先綽邊兒放了四碟菓子,四碟小菜,又是四碟案酒:一碟頭魚 ,一碟糟鴨 ,一碟烏皮雞 ,一碟舞鱸公。又拿了四樣下飯來:一碟羊角蔥 火川炒的核桃肉 ,一碟細切的〈食皆〉〈食禾〉樣子肉,一碟肥肥的羊貫腸 ,一碟光溜溜的滑鰍 。次又拿了一道湯飯出來,一個碗內兩個肉員子,夾著一條花觔滾子肉,名喚一龍戲二珠湯;一大盤裂破頭高裝肉包子。西門慶讓胡僧吃了,教琴童拏過團靶鉤頭雞脖壺來,打開腰州精製的紅泥頭 ,一股一股邈出滋陰摔白酒來,傾在那倒垂蓮蓬高腳鍾內,遞與胡僧。那胡僧接放口內,一吸而飲之。隨即又是兩樣添換上來:一碟寸扎的騎馬腸兒,一碟子醃臘鵝脖子 。又是兩樣艷物,與胡僧下酒:一碟子癩葡萄 ,一碟流心紅李子。落後又是一大碗鱔魚麵 與菜卷兒 ,一齊拏上來,與胡僧打散。登時把胡僧吃的楞子眼兒,便道:「貧僧酒醉飯飽,足可以勾了。」西門慶叫左右拏過酒卓去,因問他求房術的藥兒。胡僧道:「我有一枝藥,乃老君煉就,王母傳方,非人不度,非人不傳。專度有緣。既是官人厚待于我,我與你幾丸罷。」于是向褡褳內取出葫蘆兒,傾出百十丸。分付:「每次只一粒,不可多了。用燒酒 送下。」又搬向那一個葫兒捏了,取二錢一塊粉紅膏兒,分付:「每次只許用二厘,不可多用。若是脹的慌,用手捏著兩邊腿上,只顧摔打百十下,方得通。你可樽節用之,不可輕泄于人。」西門慶雙手接了,說道:「我且問你,這藥有何功效?」胡僧說:「形如雞卵,色似鵝黃。三次老君炮煉,王母親手傳方。外視輕如糞土,內覷貴乎玕琅。比金金豈換,比玉玉何償。任你腰金衣紫,任你大廈高堂。任你輕袋肥馬,任你才俊棟梁。此藥用托掌內,飄然身入洞房。洞中春不老,物外景長芳。玉山無頹敗,丹田夜有光。一戰精神爽,再戰氣血剛。不拘嬌艷寵,十二美紅妝。交接從吾好,徹夜硬如鎗。服久寬脾胃,滋腎又扶陽。百日鬚髮黑,千朝體自強。固齒能明目,陽生姤始藏。恐君如不信,拌飯與貓嚐。三日淫無度,四日熱難當,白貓變為黑,尿糞俱停亡。夏月當風臥,冬天水裏藏。若還不解泄,毛脫盡精光。每服一厘半,陽興愈健強。一夜歇十女,其精永不傷。老婦顰眉蹙,淫娼不可當。有時心倦怠,收兵罷戰場。冷水吞一口,陽回精不傷。快美終宵樂,春色滿蘭房。贈與知音客,永作保身方。」西門慶聽了,要問他求方,說道:「請醫須請良,傳藥須傳方。吾師不傳于我方兒,倘或我久後用沒了,那里尋師父去?隨師父要多少東西,我與師父。」因令玳安:「後邊快取二十兩白金來。」遞與胡僧,要問他求這一枝藥方。那胡僧笑道:「貧僧乃出家之人,雲遊四方,要這資財何用?官人趁早收回去!」一面就要起身。西門慶見他不肯傳方,便道:「師父,你不受資財。我有一疋四丈長大布,與師父做件衣服罷。」即令左右取來,雙手遞與胡僧。僧方纔打問訊謝了。臨出門,又分付:「不可多用。戒之!戒之!」言畢,背上褡褳,拴定拐杖,出門揚長而去。正是:

「柱杖挑擎雙日月,  芒鞋踏遍九軍州。」

有詩為證:

「彌勒和尚到神州,  布袋橫拖拄杖頭,

饒你化身千百化,  一身還有一身愁。」

畢竟未知後來何如,且聽下回分解:




\end{showcontents}


