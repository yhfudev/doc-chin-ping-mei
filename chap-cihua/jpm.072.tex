%# -*- coding: utf-8 -*-
%!TEX encoding = UTF-8 Unicode
%!TEX TS-program = xelatex
% vim:ts=4:sw=4
%
% 以上设定默认使用 XeLaTex 编译,并指定 Unicode 编码,供 TeXShop 自动识别

%第七十二回 
\chapter{王三官拜西門為義父\KG 應伯爵替李銘解冤}


「寒暑相推春復秋,  他鄉故國兩悠悠,

清清行李風霜苦,  蹇蹇王臣涕淚流;

風波浪裡任浮沉,  逢花遇酒且寬愁,

蝸名蠅利何時盡,  幾向青童笑白頭。」

話說西門慶與何千戶在路不題。單表吳月娘在家,因前者西門慶上東京,在金蓮房飲酒,被奶子如意兒看見。西門慶來家,反受其殃,架了月娘一篇是非,合了那氣。以此這遭西門慶不在,月娘通不招應。就是他哥嫂來看也不留,即就打發。分付平安:「無事關好大門,後邊儀門,夜夜上鎖。」姊妹每都不出了,各自在房做針指。若經濟要往後樓上尋衣裳,月娘必使春鴻或來安兒跟出跟入,常時查門戶,凡事多嚴緊了。這潘金蓮因此不得和經濟勾拾搭,只賴奶子如意兒備了舌,在月娘處,逐日只和如意兒合氣。一日月娘打點出西門慶許多衣服汗衫小衣,教如意兒做,又教他同韓嫂兒漿洗,就在李瓶兒那邊洒浪。不想金蓮這邊春梅也洗衣裳搥裙子,問他借棒槌。這如意兒正與迎春搥衣,不與他,說道:「前日你拿了把個棒搥使秋菊使著罷了,又來要。趁韓嫂在這裡,替爹搥褲子和汗衫兒哩。」那秋菊使使性子決烈的走來對春梅說:「平白教我借,他又不與。迎春倒說拿去,如意兒攔住了不肯。」春梅便道:「耶嚛,耶嚛!這怎的這等生分,大白日裡借不出個乾燈盞來。娘不肯,還要教我洗裹腳,我漿了這黃絹裙子,問人家借棒槌使使兒,還不肯與,將來替娘洗了,拿什麼槌?」教秋菊:「你往後邊問他每借來使使罷。」這潘金蓮正在房中炕上裹腳,忽然聽見,便問:「怎麼的?」這春梅便把借棒槌,如意兒不與來一節說了。只這婦人因懷著舊時仇恨,尋了不著這個由頭兒,便罵道:「賊淫婦,怎的不與?他是丫頭,你自家問他要去。不與,罵那淫婦不妨事!」這春梅還是年壯,一沖性了,不由的激犯,一陣風走來李瓶兒那邊,說道:「那個世人怎也的!要棒槌兒使使不與他。如今這屋裡,又鑽出個當家人來了!如意兒道:「耶嚛!耶嚛!這里放著棒槌,拿去使不是,誰在這里把住,就怒說起來。大娘分付,趁韓媽在這里,替爹漿出這汗衫子和綿紬褲子來等著,又拙出來要槌。秋菊來要,我說待我把你爹這衣服搥兩下兒,作拿上使去。就架上許多誑,說不與來。早是迎春姐這里聽著!」不想潘金蓮隨即就跟了來,便罵道:「你這個老婆,不要說嘴!死了你家主子,如今這屋裡就是你。你爹身上衣服,不著您恁個人兒拴束,誰應的上他那心?俺這些老婆死絕了,教你替他漿洗衣服。你死拿這個法兒降伏俺每,我好耐警耐怕兒!」如意兒道:「五娘怎的這說話!大娘不分付,俺每好意掉攬替爹整埋也怎的!」金蓮道:「賊歪刺骨雌漢的淫婦!還漒說什麼嘴!半夜替爹遞茶兒扶被兒是誰來?討披襖兒穿是誰來?你背地幹的那繭兒?你說我不知道!偷就偷出肚子來,我也不怕!」如意道:「正景有孩子還死了哩,俺每到的那些兒!」這金蓮不聽便罷,聽了心頭火起,粉面通紅,走向前一把手,把老婆頭髮扯住,只用手摳他腹。這金蓮就被韓嫂兒向前勸開了。罵道:「沒廉耻的淫婦,嘲漢的淫婦!俺每這里還閒的聲喚,你來雌漢子。合你在這屋里是什麼人兒?你就是來旺兒媳婦子從新又出世來了,我也不怕你!」那如意兒一壁哭著,一壁挽頭髮,說道:「俺每後來,也不知什麼來旺兒媳婦子,只知在爹家做奶子。」金蓮道:「你做奶子,行你那奶子的事。怎的在屋裡狐假虎威起精兒來!老娘成年拿雁,教你弄鬼兒去了!」正罵著,只見孟玉樓後後慢慢的走將來,說道:「六姐,我請你後邊下棋,你怎的不去?卻在這里亂些什麼?」一把手拉進到他房中坐下,說道:「你告訴我說,因為什麼起來?」這金蓮消了回氣,春梅遞上茶來,呵了些茶,便道:「你看,教這賊淫婦氣的我手也冷了,茶也拿不起來!」說道:「我在屋裡正描鞋,你使小鸞來請我。我說且倘倘兒去,〈扌歪〉在牀上還未睡去著,也見這小肉兒,百忙且搥裙子。我說你就帶著把我的裹腳槌搥出來。半日只聽的亂起來,教秋菊問他要椿槌使使,他不與。把棒槌匹手奪下了,說道『前日拿了個去不見了,又來要。如今緊等著與爹搥衣服。』教我心裡就惱起來,使了春梅:『你去罵那賊淫婦。從幾時就這等大膽降伏人,俺每手裡教你降伏。你是這屋裡什麼兒?押折橋竿兒娶你來?你比來旺兒媳婦子差些兒!』我就隨跟了去,他還嘴裡【石必】裡剥剌的。教我一頓捲罵,不是韓嫂兒死氣日賴在中間拉著我,我把賊沒廉耻雌漢的淫婦,口裡肉也掏出他的來!要俺每在這屋裡點韮買蔥,教這淫婦在俺每手裡弄鬼兒也沒鬼!大姐姐那些兒不是;他想著把死的來旺賊奴才淫婦,慣的有些摺兒!教我和他為冤結仇。落後一染膿帶,還垛有我身上,說是我弄出那奴才去了。如今這個老婆,又是這般慣他,慣的恁沒張倒置的!你做奶子,行奶子的事。許你在跟前花黎胡哨!俺每眼裡是放的下砂子底人!有那沒廉恥的貨,人也不知死的那里去了,還在那屋裡纏。但往那里回來,就望著他那影作個揖,口裡一似嚼蛆的,不知說的什麼!到晚夕到吃茶,淫婦就起來連忙替他送茶。又忔忽兒替他蓋被兒,兩個就弄將起來。就是了久慣的淫婦!他說丫頭遞茶,許你去撑頭獲腦去雌漢子!是什麼問他要披襖兒?沒廉恥他便連忙鋪子拿了細段來,替他裁披襖兒。你還沒見哩,斷七那日,學他爹爹就進屋裡燒布去,見丫頭老婆正在炕上坐看撾子兒,他進來收不及,反說道:『姐兒,你每耍耍,供養的匾盒和酒,也不要收到後邊去,你每吃了罷。』這等縱容,看他謝的什麼?這淫婦請說:『爹來不來,俺每不等你了!』不想我兩步三步就扠進去,說的他眼張失道,于是就不言語,行貨子什麼好老婆,一個賊活人妻淫婦,這等你餓眼見瓜皮,不管了好歹的你收攬答下。原來是一個眼裡火,爛桃行貨子!想有些什麼好正條兒。那淫婦的漢子,說死了。前日漢子抱著孩子,沒有門首打探兒?還是瞞著人搗鬼,張眼兒溜睛的!你看一向在人眼前,花哨星那樣花哨,就別摸兒改樣的!你看又是個李瓶兒出世了。那大姐姐成日在後邊,只推聾兒裝啞的,人但開口,就說不是了。」那玉樓聽了只是笑。金蓮道:「南京沈萬三,北京枯柳樹,人的名兒,樹的影兒,怎麼不饒的?雪裡消死屍,自然消他出來!」玉樓道:「原說這老婆沒漢子,如何又鑽出漢子來了?」金蓮道:「天不著風兒晴不的,人不著謊兒成不的。他不整攛瞞著,你家肯要他?想著一來時,餓答的個臉,黃皮兒寡瘦的,乞乞縮縮那等腔兒。看你賊淫婦吃了這二年飽飯,就生事兒雌起漢子來了!你如今不禁下他來,到明日又教他上頭腦上臉的。一時桶出個孩子,當誰的?」玉樓笑道:「你這六丫頭,倒且是有權術。」說畢,坐了一回,兩個往後邊下棋去了。正是:

「三光有影遺誰繫,  萬事無根只自生。」

有詩為證:

「一掬陽和動物華,  深紅淺綠總萌芽;

野梅亦足供清玩,  何必辛夷樹上花。」

話休饒舌,有日後晌時分,西門慶來到清河縣,分付賁四、王經跟行李先往家去。他便送何千戶到衙門中,看看收拾打歸公廨乾淨,他便騎馬來家。進入後廳,吳月娘接著拂去塵土。舀水淨面畢,就令丫鬟院子內放卓兒,滿爐焚香,對天地位下告訴願心。月娘便問:「你為什麼許願心?」西門慶道:「且休說,我性命來家!」往回路上之事,告訴一遍:「昨日十一月二十三日,剛過黃河,行到沂水縣八角鎮上,遭遇大風。那風那等兇惡,沙石迷目,通不放前進。天色又晚,百里不見人。眾人多慌了。況一個裝馱垛又多,誠恐鑽出個賊怎了。前行投到古寺中,和尚又窮,夜晚連燈火沒個兒。各人隨身帶著些乾糧麵食,借了燈火來,熬了些豆粥,每人各吃一頓。砍了些柴薪草根,喂了馬,我便與何千戶在一個禪炕上抵足一宿。次日風住了,方纔起身。這場苦,比前日還更苦十分!前日雖是熱天,還好些。這遭又是寒冷天氣,又耽許多懼怕,幸得平地還罷了,若在黃河,遭此風浪怎了!我頭行路上許了些願心,到臘月初一日,宰豬羊祭賽天地。」月娘又問:「你頭裡怎不在家,卻往衙門裡做甚麼?」西門慶道:「夏龍溪已陞做指揮直駕,不得來了。新陞將作監何太監姪兒何千戶,名永壽,貼刑,不上二十歲,捏出水兒來的一個小後生,任事兒不知道。他太監再三央及我,凡事看顧教道他。我不送到衙門裡,安頓他個住處,他知道什麼?他如今一千二百兩銀子,也是我作成他,要了夏龍溪那房子。如今且教他在衙門裡住著,待夏大人搬取了家小,他的家眷纔搬來。昨日夏大人甚是願意,在京不知什麼人走了風,投到俺每去京中,他又早使了錢,不多少錢,不多少銀子,尋了當朝林真人分上,對堂上朱大尉說,情願以指揮職銜,再要提刑三年。朱大尉來對老爺說,把老爺難的要不的。若不是翟親家在中間竭力維持,把我撑在空地裡去了。去時親家好不怪我,說我幹事不謹密。不知他什麼人對他說來?」月娘道:「不是我說,你做事有些三慌子,火燎腿樣,有不的些事兒,詐不實的告這個說一湯,那個說一湯,恰似逞強賣富的!正是:有心算無心,不備怎防備?頭見你幹,人家曉的不耐煩了。人家悄悄幹的事兒停停脫脫,你還不知道哩!」西門慶又說:「夏大人臨來,再三央我早晚看顧看顧他家裡。容日你買份禮兒走走去。」月娘道:「他娘子出月初二日生日,就一事兒喜歡說。你今後把這狂樣來改了。常言道:『逢人且說三分話,未可全拋一片心。』老婆還有個裡外心兒,休說世人!」正說,只見玳安說:「賁四問爹,要往夏大人家,說著去不去?」西門慶道:「你教他吃了飯去。」玳安道:「他說不吃罷。」李嬌兒、孟玉樓、潘金蓮、孫雪娥、大姐多來參見,道萬福問話兒陪坐的。西門慶又想起前番往東京回家,還有李瓶兒在,今日卻沒他了。一面走到他前邊房內,與他靈床作揖,因落了幾點眼淚。如意兒、迎春、綉春多來向前磕頭。月娘隨即使小玉請在後邊擺飯吃了。一面分付討出四兩銀子,賞跟隨小馬兒上的人,拿帖兒回謝周守備了去。又教來興兒宰了半口豬,半腔羊,四十斤白麵,一包白米,一壜酒,兩腿火燻,兩隻鵝,十隻雞,柴炭兒又并許多油鹽醋之類,與何千戶送下程。又叫了一名廚役,在那里答應。正在廳上打點,差玳安送去。忽琴童兒進來說道:「溫師父和應二爹來望。」西門慶連忙道:「有請。」溫秀才穿著綠段道袍,伯爵是紫絨襖子,從前進來參見西門慶,連連作揖,道其風霜辛苦,西門慶亦道:「蒙二公早晚看家。」伯爵道:「我又看家哩!我早起來時,忽聽房上喜鵲喳喳的叫。俺房下就先說:『只怕大官人來家了,你還不走的瞧瞧去?』我便說:『哥從十二日起身,到今還得上半月期,怎的來得快?我三日一遍在那里問,還沒見來的信息。』房下說:『來不來,你看看去。』教我穿衣裳到宅裡。不想說哥來家了。走到對過會溫老先兒,不想溫老師也纔穿衣裳,說我就同老翁一答兒過去罷。」因問了今東京路上的人,又見許多下飯酒米裝在廳檯上,出來擺放,便問道:「誰家的?」西門慶道:「新同僚何大人,如此同來,家小還未到,且在衙門中權住,送分下程與他。又發柬明日請他來家坐了吃接風酒,再沒人。請二位與大哥奉陪。」伯爵道:「又一件,吳大舅與哥是官,溫老先戴著方巾,我一個小帽兒,怎陪得他坐?不知把我當甚麼人兒看;我惹他不笑話?」西門慶笑道:「這等把我買的段子忠靖巾,借與你戴著。等他問你,只說道是我的大兒子,好不好。」說畢,眾人笑了。伯爵道:「說正經話,我頭八寸三,又戴不的你的。」溫秀才道:「學生也是八寸二分。倒將學生方巾與老翁戴戴何如?」西門慶道:「老先生不要借與他。他到明日說慣了,往禮部當官身去,又來纏你。」溫秀才笑道:「好說!老先生兒好說,連我扯下水去了。」家拿上茶來吃了。溫秀才問:「夏公已是京任,不來了?」西門慶道:「他已做了堂尊了!直掌囪簿大鳴,穿麟服,使藤棍。如此華任,又來做什麼?」須臾,看寫了帖子兒,擡下程出門,教玳安送去了。西門慶拉溫秀才、伯爵廂房內暖炕上籠了火,那里坐。又使琴童先往院里叫吳惠、鄭春、邵奉、左順四名小優兒,明日早來伺候。不一時,放卓兒,陪二人吃酒。來安兒拿上案來擺下。西門慶分付:「再取隻鍾筯兒,請你姐夫來坐坐。」良久,陳經濟走來作揖,打橫坐下。四人圍爐共坐,把酒來斟。因說回東京一路上的話。伯爵道:「哥你的心好,一福能壓百禍。就有小人,一時自然多消散了。」溫秀才道:「善人為邦百年,亦可以勝殘去殺。休道老先生為王事驅馳,上天也不肯有傷善類。西門慶因問:「家中沒甚事?」經濟道:「家中爹去後,倒也無事。只是工部安老爹那里,差人來問了兩遭。昨日還來問,我回說還沒有來家裡。」正說著,只見來安兒拿了大盤子黃芽韮豬肉盒兒上來。西門慶陪著纔吃了一個兒,忽有平安走來報:「衙門里各房令史和眾節級來稟事。」西門慶即到廳上站立,今他進見。二人跪下:「請問老爹幾時上任?官司公用銀兩,動支多少?」西門慶道:「你們只照舊時整理就是了。」令史道:「去年只老爹一位到任。如今老爹轉正,何老爹新到任,兩事並舉,比尋常不同。」西門慶道:「既是如此,添十兩銀子,三十兩買辦就是了。」二人應喏下去。西門慶又叫回來,分付:「上任的日期,你還問何老爹擇幾時?」二人道:「何老爹纔定准在二十八日上任。」西門慶道:「既如此,你每伺候就是了。」二人到衙門領了銀子出來,定卓席買辦去了。落後喬大戶又來拜望道喜。西門慶留坐,不坐,吃茶起身去了。當下西門慶陪二人至掌燈時方散。西門慶往月娘房裡歇了,一宿題過。到次日,家中置酒與何千戶接風。文嫂又早打聽得西門慶來家,對王三官說了,具個柬帖兒來看請。西門慶這里買了二付豕蹄,兩尾鮮魚,兩隻燒鴨 ,一壜南酒 ,差玳安送去,與太太補生日之禮。他那里賞了玳安三錢銀子,這不在話下。正廳上設下酒,錦屏耀目,卓椅鮮明,地鋪錦毡,壁挂名人山水。吳大舅、應伯爵、溫秀才多來的早。西門慶陪坐吃茶。使人邀請何千戶,不一時小優兒上來磕頭。應伯爵便問:「哥,今日怎的不叫李銘?」西門慶道:「他不來我家來,我沒的請他去。」這伯爵便道:「你惱他每?」不言語了。正說話中間,只見平安慌忙拿帖兒稟說:「帥府周爺來拜,下馬了。」吳大舅、溫秀才、應伯爵都躲在西廂房內。西門慶冠帶柮來,迎至廳上敘禮,道及轉陞恭喜之事。西門慶又謝他人馬,于是分賓主坐著。周守備問京中見朝之事,西門慶一一說了。周守備道:「龍溪不來,已定差人來取家小上京去。」西門慶道:「就取也待出月,如今何長官且在衙門權住著哩。夏公的房子與了他住,也是我替他主張的。」守備道:「這等更妙!」因見堂中擺設卓席,問道:「今日所延甚客?」西門慶道:「聊具一酌,與何大人接風。同僚之間,不好意思。」二人吃了茶,周守備起身說道:「容日合衙列位,與二公奉賀。」西門慶道:「豈敢動勞!多承先施!」作揖出門,上馬而去。西門慶回來脫了衣服,又陪三人坐的,在書房中擺飯。何千戶到午後方來。吳大舅等各相見敘禮畢,各敘寒溫。茶湯換罷,各寬衣服。何千戶見西慶家道相稱,酒筵齊整,四個小優,銀箏象板,玉阮琵琶,遞酒上坐,堂中金爐焚獸炭,玉盞泛羊羔。放下簾了,合席春風,滿堂和氣。正是:

「得多少金樽浮綠醑,  玉燭剪春聲。」

飲酒至起更時分,何千戶方起身往衙門中去了。吳大舅、應伯爵、溫秀才各辭回去了。西門慶打發小優兒出門,分付收了家火,往前邊金蓮房中來。婦人在房內濃施朱粉,復整新粧,薰香澡牝,正盼西門慶進他房來。滿面笑容,向前替他脫衣解帶,連忙教春梅點茶與他吃。吃了打發上床歇宿。端的暖衾暖被,錦帳生春,麝香藹藹。被窩中相挨素體,枕蓆上緊貼酥胸。口吐丁香,蚪含珠。婦人雲雨之際,百媚俱生。西門慶扣拽之後,靈犀已透。睡不著,枕上把離言深講;交接後,淫情未足,定從下品鸞簫。這婦人的說無非只是要拴西門慶之心,又况拋離了半月,在家久曠幽懷,淫情似火。得到身,恨不得鑽入他腹中。那話把來品弄了一夜,再不離口。西門慶要下床溺尿,婦人還不放,說道:「我的親親,你有多少尿,溺在奴口裡,替你嚥了罷!省的冷呵呵的,熱身子,你又下去凍著,倒值了多的。」這西門慶聽了,越發歡喜無已。叫道:「乖乖兒,誰似你這般疼我!」于是真個溺在婦人口內,婦人用口接著,慢慢一口多嚥了。西門慶問道:「好吃不好吃?」金蓮道:「略有些鹹味兒,你有香茶 與我些壓壓?」西門慶道:「香茶在我白綾祅內,你自家拿。」這婦人向床頭拉過他袖子來,掏掏了幾個,放在口內纔罷。

「待臣不及相如渴,  特賜金莖露一盃。」

看官聽說:大抵妾婦之道,蠱惑其夫,無所不至。雖屈身忍辱,殆不為恥。若夫正室之妻,光明正大,豈肯為此?是夜西門慶與婦人儘力盤桓無度。次日早往衙門中,何千戶上任吃公宴酒,兩院樂工動樂承應。午後纔回家,排軍隨即抬來卓席來。王三官那里,又差人早來邀請。西門慶使玳安段舖中要了一套衣服,包在毡包內,纔收拾出來,右左來報:「工部安老爺來拜。」慌的西門慶整衣不迭,出來迎接。安郎中食經正等丞的俸,繫金廂帶,穿白鷴補子,跟著許多官吏,滿面笑容,相攜到廳敘禮。彼此道及公恭賀之,分賓主坐下。安郎中道:「學生差人來問幾次,說四泉還未回。」西門慶道:「正是,京中要等見朝引奏,纔起身回。」須臾,茶湯吃罷,安郎中方說:「學生敬來有一事,不當奉瀆。今有九江大戶蔡少塘,乃是蔡老先生第九公子,來上京朝覲。前日有書來,有早晚便到。學生與宋松泉、錢雲野、黃泰宇四人作東,借府上設席請他,未知允否?」西門慶道:「老先生尊命,豈敢有違約!定幾時?」安郎中道:「在二十七日。明日學生送分子過來,煩盛使一辦,足見厚愛矣。」說畢,又上了一道茶,作辭起身,上馬喝道而去。西門慶即出門,前往王招宣府中來赴席。到門道先投了拜帖。王三官聽的西門慶到了,連忙出來迎接,至廳上敘禮。原來五間大廳,毬門蓋造五脊五獸,重簷滴水,多是菱花槅廂。正面欽賜牌額,金字題曰:「世忠堂」兩邊門對寫著:「啟運元勳第,山河【石帶】礪家」廳內設着虎皮公座,地下鋪著裁毛絨毯。王三宮與西門慶行畢禮,尊西門慶上座,他便傍設一椅相陪。須臾紅漆丹盤,拿上茶來。交手遞了茶,左右收了去。彼此扳了些說話,然後安排酒筵遞酒。原來王三官叫了兩名小優兒彈唱。西門慶道:「請出老太太拜見拜見。」慌的王三官令左右後邊說。少頃出來說道:「請老爹後邊見罷。」王三官讓西門慶進內。西門慶道:「賢契你先導引。」于是逕入中堂。林氏又早戴著滿頭珠翠,身穿大紅通袖袍兒,腰繫金鑲碧玉帶,下著玄錦百花裙,搽抹的如銀人也一般。梳著縱鬢,點著朱唇,耳帶一雙胡珠環子,裙拖垂兩挂玉佩叮〈口東〉。西門慶一面將身施禮,請太太轉上。林氏道:「大人是客,請轉上了。」半日,兩個人平磕頭。林氏道:「小兒不識好歹,前日沖瀆大人。蒙大人寬宥,又處斷了那些人,知感不盡!今日備了一杯水酒,請大人過來,老身磕個頭兒謝謝。如何又蒙大人見賜將禮來,使我老身卻之不恭,受之有愧!」西門慶道:「豈敢!學生因為公事,往東京去了,誤了與老太太拜壽,些須薄禮,胡亂送與老太太賞人便了。」因見文嫂兒在傍,便道:「老文,你取付臺兒來,等我與太太送杯杯壽酒。」連忙呼玳安上來。原來西門慶毡包內預備著一套遍地金時樣衣服,紫丁香色,通袖段襖,翠藍拖泥裙,放在盤內獻上。林氏一見,金彩奪目,先是有五七分歡喜。文嫂隨即捧上金盞銀臺。王三官便叫兩個小優拿樂器進來彈唱,林氏道:「你看叫進來做什麼?在外答應罷了!」一面攆出來。當下西門慶把盞畢,林氏也回奉了一盞,與西門慶謝了。然後王三官與西門慶遞酒。西門慶纔待送下禮去,林氏便道:大人請起,受他一禮兒。」西門慶道:「不敢,豈有此禮!」林氏道:「好大人,怎生這般說!你恁大職級,做不起他個父親?小兒自幼失學,不曾跟著那好人;若不是大人垂愛,凡事也指教為個好人,今日我跟前,教他拜大人做了義父,但看不是處,一任大人教訓,老身並不護短。」西門慶道:「老太太難故說得是,但令郎賢契,賦性也聰明,如今年少,為小試行道之端。往後自然心地開闊,改過遷善,老太太倒不必介意。」當下教西門慶轉上,王三官把盞,遞了三鍾酒,受其四拜之禮。遞畢,西門慶亦轉下與林氏作揖謝禮,林氏笑吟吟,深深還了萬福。自以此後,王三官見著西門慶以父稱之,有這等事。正是:

「常將壓善欺良意,  權作尤雲殢雨心?」」

詩人看到此,必甚不平,故作詩以嘆之。詩曰:

「從來男女不通酬,  賣俏營奸真可羞;

三官不解其中意,  饒貼親娘還磕頭。」」

又詩:

「大家閨閣要嚴防,  牝雞司晨最不良;

不但孛得家聲喪,  有愧當時節義堂。」」

遞畢酒,林氏分付王三官:「請大人前邊坐,寬衣服。」玳安拿忠靖巾來換了。不一時,安席坐下,小優彈唱起來,廚役上來割道,玳安拿賞賜伺候。當時席前唱了一套新水令:

「翠簾深,小房櫳,滴玉鈎抵控馳茸,斗蜆龜背錦屏風。春意溶溶,梅稍上暗香動。」

〔喬牌兒〕「瑣窗橫,倒挂綠毛鳳。梨雲一片羅浮夢,夜深沉。」

〔甜水令〕「瓊樹生花,玉龍晚凍,瑞雪舞迴鳳。碧落塵淡,自窺丹雲接□□,臭門珠玄。」

〔折桂令〕「錦排場賞玩,春正二八仙鬟。十六歌童花底藏門;尊前暗令,席上投隻嬌滴滴爭奸競寵,幸孜孜倚翠偎紅。走斝飛觥,換的移玄妙,清誰揖撥輕籠。」

〔水仙子〕「麝媒香靄,綉美帶葉鳳。臘光搖金蝶,象床春暖花;胡的脂粉香,珠翠叢,彩雲深,羅騾龍涎細,金爐獸,相暖溶溶,和氣春風。」

〔雁兒落得勝令〕「銀箏秋雁橫,玉管鶯弄。花明翡翠翹,酒滿玻璃寺。衫袖捧金尊,羅帕春葱。橙嫩霜剖,茶香帶雪烹。歡濃,醉後情從重。筵終,更深樂未窮。」

〔沽美酒〕「轉秋波,一笑中,透犀兩情。道燈下端祥可重種,似嫦娥出月玄,知女下巫峰。」

〔太平令〕「欹鬢彈金釵飛鳳,舞裙憁翠縷蟠龍。粉汗溫鉛華嬌容。舌尖吐丁香微送。臂釧封守,原是一對兒雛鸞嬌鳳。」

〔川撥棹〕「喜相逢,相逢可意種柳因花慵。玉暖酥融,那一回風流受用。巍巍寶髻鬆,困藤秋冰橫,曲彎彎眉黛濃。七弟兄醉烘玉窮暈微紅,龍花蝶玉歡情,縱有身在醉魂中。蕊珠玄里遊仙夢,梅花酒恰便以雲雨蹤。沒亂殺,見慣司空。禁故簾龍,馬棟鄰雞唱終。玉漏滴咽,雖龍艮倚燼落螢,沙寶到曉光籠。碧天邊日那融融。」

〔收江南〕「呀,倒聽的轆轤聲,在粉墻東。早鴉啼金并下梧桐。春嬌滿眼未惺越,將一段幽歡密寵,等閒驚覺忽忽。

當下食割五道,歌吟二套,秉獨上來。西門慶起身更衣告辭。王三官再三款留,又邀到他那邊書院中;獨獨的一所書院,三間小軒,裡面花木掩映,文物消洒,金粉箋扁曰:「三泉詩舫。」四壁挂四軸古畫:軒轅問道,伏生墳典,丙吉問牛,宋京觀史。西門慶便問:「三泉是何人?」王三官只顧隱避,不敢回答,半日纔說:「是兒子的賤號。」西門慶便一聲兒沒言語。抬過高壺來,只顧投壺飲酒,四個小優兒在傍彈唱。林氏後邊和丫鬟養娘,只顧打發添換菜蔬果碟兒上來飲酒。吃到二更時分,西門慶已帶半酣,作辭起身。賞小優兒三錢銀子。親送到大門,看他上轎。兩個排軍打著燈火,西門慶頭戴暖耳,身披貂裘,作辭回家。到家想著金蓮白日裡話,逕往他房中。原來婦人還沒睡哩,纔摘去冠兒,挽著雲髻,淡粧濃抹,正在房內倚靠著梳檯腳,登著爐臺兒,口中磕瓜子兒等待。火邊茶烹玉蕊,卓上香裊金猊。見西門慶進來,慌的輕移蓮步,欵蹙湘裙,向前接衣裳安放。西門慶坐在床上,春梅拿淨甌兒,婦人從新用纖手抹盞邊水漬,點了一盞濃濃艷艷芝蔴鹽笋栗系瓜仁核桃仁夾春不老海青拿天鵝木樨玫瑰潑滷六安雀舌芽茶。西門慶剛呷了一口,美味香甜,滿心欣喜。然後令春梅脫靴解帶,打發在床。婦人在燈下摘去首飾,換了睡鞋,兩個被翻紅浪,枕欹彩鴛,並頭交股而寢。春梅向卓上罩罩合銀荷,雙掩鳳槅,歸那邊房中去了。西門慶將一隻肐膊支婦人枕著,精赤條摟在懷中,猶如軟玉溫香一般。兩個酥胸相貼,玉股交【木匝】,臉兒廝搵,嗚咂其舌。婦人一把扣了瓜子穰兒,用碟兒盛著,安在枕頭邊,將口兒噙著,舌支密喃送下口中。不一時,甜唾融心,靈犀春透。婦人不住手下邊捏弄他那話,打開淫器包兒,把銀托子。西門慶因問道:「我的兒,我不在家,你想我不曾?」婦人道:「你去了這半個月來,奴那刻兒放下心來。晚間夜又長,獨自一個又睡不著。隨問怎的暖床暖鋪,只是害冷。伸著腿兒觸冷伸不開。手中了的酸了,數著日子兒白盼不到。枕邊眼淚不知流勾多少!落後春梅小肉兒,他見我短嘆長吁,晚間鬬著我下棋。坐到起更時分,俺娘兒兩個一炕兒通廝腳兒睡。我的哥哥,奴心便是如此;不知你的心兒如何?」西門慶道:「怪油嘴,這一家雖是有他們,誰不知我在你身上偏多。」婦人道:「罷麼,你還哄我哩!你那吃著碗裏,看著鍋裡的心兒,你說我不知道!想著你和來旺兒媳婦子密調油也似的,把我來就不理了。落後李瓶兒生了孩子,見我如同烏眼雞一般。今日多往那去了?止的奴老實的還在。你就是那風裡揚花,滾上滾下。如今又興起那如意兒賊歪刺骨來了!他隨問怎的,只是奶子。見放著他漢子,是個活人妻。不爭你要了他,到明日又教漢子好在門首放羊兒好刺。你為官為宦,傳出去什麼好聽?你看這賊淫婦,前日你去了,同春梅兩個為一個棒槌,和我兩個大嚷大鬧,通不讓我一句兒哩!」西門慶道:「罷麼,我的兒,他隨問甚怎的,只是個手下人。他那里有七個頭八個膽,敢頂撞你?你高高手兒他過去了,低低手兒他過不去。」婦人道:「嚛!說高高手兒他過不去了的話!沒了李瓶兒,他就頂了窩兒。學你對他說:『你若伏侍的好,我把娘這分家當就與你罷。』你真個有這個話來?」西門慶道:「你休胡猜疑我,那里有些話?你寬恕他,我教他明日與你磕頭陪不是罷。」婦人道:「我也不要他陪不是,我也不許你到那屋裡睡。」西門慶道:「我在那邊睡,也非為別的。因越了不過李大姐情,一兩夜不在那邊歇了。他守靈兒,誰和他有私鹽私醋?」婦人道:「我不信你這摭溜子,人也死了一百日來,還守什麼靈?在那屋裡也不是守靈。屬米倉的,上半夜搖鈴,下半夜丫頭似的,聽好柳聲!」幾句說的西門慶急了,摟個脖子來,親了個嘴,說道:「怪小淫婦兒,有這些張致的!」于是令他吊過身子去,隔山抅火,那話自後插入牝中,把手在被窩內,接抱其股,竭力搧磞的連聲響喨。一面令婦呼叫大東大西,問道:「你怕我不怕?再敢管著?」婦人道:「怪奴才,不管著你,待好上天也!我曉的也丟不開這淫婦,到明日問了我,方許你那邊去。他若問你要東西,對我說,也不許你悄悄偷與他。若不依,我打聽出來,看我嚷的塵鄧鄧的!不讓我,就擯洗了這淫婦,也不差什麼兒!又相李瓶兒來頭,教你哄了,險些不把打到贅字號去了!你這波答子爛桃行貨子,豆芽菜,有甚正條綑兒也怎的!老娘如今也賊了些兒子!」西門慶笑道:「你這小淫婦兒,原來就是六禮約!」當下兩個殢雨龍雲,纏到三更方歇。正是:

「有窗有鳥賣有機,  啣得春來枝上說。」」

有詩可證:

「帶雨籠煙世所稀,  妖嬈身勢似難支;

終宵故把芳心訴,  留在東風不放歸。」」

兩個並頭交股,睡到天明。婦人淫情未足,便不住只往西門慶手里捏弄那話,登時把塵柄捏弄起來,叫道:「親達達,我一心要你身上睡睡。」一面扒伏在西門慶身上倒澆燭,接著他脖子只顧揉搓。教西門慶兩手扳住他腰,扳的緊緊的。他便在上極力抽拔,一面扒伏在他身上揉一回。那話漸沒至根,餘者被托子所阻不能入。婦人便道:「我的達達,等我白日里替你縫一條白綾帶子,你把和尚與你那末子藥,裝些在裡面。我再墜上兩根長帶兒,等睡睡時,你扎他在根子上,卻拿這兩根帶扎拴後邊,腰里拴的兒緊的,又溫火又得全放進,強如這根托子,榰澆著格的人疼,又不得盡美。」西門慶道:「我的兒,你做下,藥在桌上磁盒兒內,你自家裝上就是了。」婦人道:「你黑夜好歹來,咱晚夕拿與他試試看,好不好?」于是兩個頑耍一番。只見玳安拿帖兒進來,問春梅:「爹起身不曾?安老爹差人送分資來了,又擡了兩壜金華酒 ,四盆花樹進來。」春梅道:「爹還沒起身,教他等等兒。」玳安道:「他好小近路兒,還要趕新河口閘上回說話哩。」不想西門慶在房中聽見,隔窗叫玳安問了話,拿帖兒進,折開看著,上寫道:

「奉去分資四封,共八兩。惟少塘卓席,除者散酌而已。仰冀從者留神,足見厚愛之至!外具蒔花二盆,清玩;浙酒 二樽,少助待客之需。希莞納,幸甚!」

西門慶看了,一面起身,且不流頭,戴着毡巾,穿着絨氅衣,走出到廳上,令安老爹人進見,遞上分資。西門慶見西盆花草,一盆紅梅,一盆白梅,一盆茉莉,一盆辛夷,兩壜南酒 ,滿心歡喜。連忙收了,發了回帖,賞了來人五錢銀子。因問:「老爹們明日多咱時分來?用戲子不用?」來人道:「多得早來。戲子用海鹽的,不要這裡的。」一面打發了。西門慶分付左右,把花草擡放藏春塢書房中擺放。旋叫泥水匠隔山拘火,打了兩座暖坑。恐怕煤煙薰觸,寄委春鴻來安澆灌茶水不得有誤。西門慶使玳安叫戲子去。一面兌銀子與來安兒買辦。那日又是孟玉樓上壽,院中叫小優兒,晚夕彈唱。按下一頭。卻說應伯爵在家,拿了五個箋帖,教應寶揣著盒兒,往西門慶對過房子內,央溫秀才寫請書,要請西門慶五位夫人,二十八日家中滿月。剛出門,轉了街口,只見後邊一人高叫道:「二爺請回來。」伯爵扭頭回看,是李銘,立住了腳。李銘走到眼前問道:「二爺往那里去?」伯爵道:「我到溫師父那里有些事兒去。」李銘道:「到家中,小的還有句話兒說。」只見後邊一個閒漢掇著盒兒。這伯爵不免又到家堂屋內。李銘連忙磕了個頭,起來把盒兒掇進來放下。揭開,卻是燒鴨二隻 ,老酒 二瓶,說道:「小人沒甚,這些微物兒,孝順二爹賞人。小的有句話,逕來央及二爹。」一面跪在地下不起來。伯爵一把手拉起,說道:「傻孩兒,你有話只管和我說,怎的買禮來與我?」李銘道:「小的從小兒在爹宅內答應這幾年,如今爹到看顧別人,不用小的了。就是桂姐那邊的事,各門各戶,小的一家兒是不知道。不爭爹因著那邊怪我,難為小的了。這負屈啣冤,沒處聲訴,逕來告二爹。二爹倘到宅內見了爹,替小的加句美語兒說說。就是桂姐有些一差半錯,不干小的事。爹動意惱小的不打緊,同行中人越發欺負小的了。」伯爵道:「你原來這些時也沒往宅內答應去?」李銘道:「小的沒曾去。」伯爵道:「嗔道昨日你爹從東京來,在家擺酒與何老爹接風,請了我、何大舅、溫師父同坐,叫了吳惠、鄭春、邵春、左順在那里答應,我說怎的不見你?我問你爹,你爹說:『他沒來,我沒的請他去。』傻孩兒,你還不走跳著些兒還好,你與誰賭鱉氣哩?」李銘道:「爹宅內不呼喚,小的怎的好去?前日他每四個在那里答應,今日三娘上壽,安官兒早辰在裡邊又叫了兩名小的兒去了。明日老爹擺酒,又是他每四個,倒沒小的。小的心裡怎麼有個不急的?只望二爹替小的一說,明白小的還來與二爹磕頭。」伯爵道:「我沒有個不替你說的。我從前已往,不知替人完美了多少勾當。你央及我這些事兒,我不替你說?你依著我把這禮兒你還拿回去。你自那里錢兒,我受你的!你如今親跟了我去,等我慢慢和你爹說。」李銘道:「二爹不收此禮,小的也敢去了。雖然二爹不稀罕,也盡小的一點窮心罷了。」千恩萬謝,再三央告,伯爵把禮收了。討出三十文錢,打發拿盒人回去。說道:「盒子且放在二爹這里,等小的到宅內回來取罷。」于是與伯爵同出門,轉灣抹角,來到西門慶對門房子裡。到書院門首,搖的門環兒響,說道:「葵軒老先生在家麼?」這溫秀才正在書窗下寫帖兒,忙應道:「請里面坐。」畫童開門。伯爵在明間內坐的,正面列四張東坡椅兒,挂著一軸莊子惜寸陰圖。兩邊貼著墨刻,左右一聯書著:「瓶梅香筆研,窗雪冷琴書。」一間挂著布門簾。溫秀才聽見他來,一面即出來相見,敘禮讓坐。說道:「老翁起來的早,往那里去來?」伯爵道:「敢來煩瀆大筆,寫幾個請書兒。如此這般,二十八日小兒滿月,請宅內他娘們坐坐。」溫秀才道:「帖在那里?將來學生寫。」伯爵即令應寶取出五個帖兒遞過去。這溫秀才拿到房內,研起墨來,纔來寫得兩個。只見棋童慌慌張張走來說道:「溫師父再寫兩個帖兒,大娘的名字,如今請東頭喬親家娘和大妗子去。頭里琴童來取了門外韓大姨和孟二妗子那兩個帖兒,打發去了不曾?」溫秀才道:「你姐夫看著。打發去這半日了。」棋童道:「溫師父寫了這兩個,還再寫上三個,請黃四嬸、傅大娘、韓大嬏和甘夥計娘子的,我使來安兒來取。」不一時打發去了,只見來安來取這四個帖兒。伯爵問:「你爹在家里?衙門中去了?」來安道:「爹今日沒往衙門里去,在廳上看著收禮。喬親家那送禮來了。二爹請過那邊坐的。」伯爵道:「我寫了這帖兒就去。」溫秀才道:「老先生昨日王宅赴席來晚了。」伯爵問起那王宅,溫秀才道:「是招宣府中。」伯爵就知其故。良久,來安等了帖兒去,方纔與伯爵寫得完備。李銘過這邊來,西門慶鬔著頭,只在廳上收禮,打發回帖。傍邊排擺卓面。見伯爵來,唱喏畢,讓坐。廳上生著一盆炭火。伯爵謝前日厚情。因問:「哥定這卓席做什麼?」西門慶把安郎中來央浼作東,請蔡九府之事告與他說一遍。伯爵問道:「明日是戲子?小優?」西門慶道:「叫了一起海鹽子弟,我這里又預備下四名小優兒答應。」伯爵道:「哥,那四個?」西門慶道:「吳惠、邵奉、鄭春、左順。」伯爵道:「哥怎的不用李銘?」西門慶道:「他已有了高枝兒,又稀罕我這里做什麼?」伯爵道:「哥怎的說這個話?你喚他,他纔來。也不知道你一向惱他。但是各人勾當,不干他事。三嬸那邊幹事,他怎得曉得?你到休要屈了他。他今早到我那里,哭哭啼啼告訴我:『休說小的姐姐在爹宅內,只小的答應該幾年,今日有了別人,倒沒小的。」他再三賭神發呪,並不知他三嬏在那邊在那邊一字兒。你若惱他,卻不難為他了。他小人有什麼大湯水兒。你若動動意兒,他怎的禁得?」便教李銘:「你過來,親自告訴你爹。你只顧躲著怎的?自古醜媳婦怕見公婆。」那李銘便過來,站在槅子邊,低頭歛足,足見僻廳鬼兒一般,看著二人說話,再不敢言語。聽得伯爵叫他,一面走進去,直著腿兒跪著地下,只顧磕頭,說道:「爹再訪,那邊事小的但有一字知道,小的車碾馬踏,遭官刑楪死!爹從前已往,天高地厚之恩,小的一家粉身碎身也報不過來。不爭今日惱小的,惹的同行人恥笑,他也欺負小的,小的再向那里是個主兒!」說畢,號啕痛哭,跪在地下,只顧不起身。伯爵在傍道:「罷罷!哥,是看他一場。大人不見小人之過。休說沒他不是,就是他不是處,他既如此,你也將就可恕他罷。你過來,自古穿黑衣抱黑拄,你爹既說開,就不惱你了。」李銘道:「二爹說的是,知過必改,往後知道了。」伯爵道:「打麵面口袋,你這回纔到過醮來了。」西門慶沉吟半晌,便道:「既你二爹再三說,我不惱你了,起來答應罷。」伯爵道:「你還不快磕頭哩!」那李銘連忙磕個頭,立在傍邊。伯爵方纔令應寶取出五個請帖兒來,遞與西門慶說道:「二十八日小兒彌月,請列位嫂子過舍光降光降。」西門慶展開觀看,上面寫著:

「二十八日小兒彌月之辰,寒舍薄具豆觴,奉酬厚腆。千希魚軒賁臨,不勝幸荷!

下書

應門杜氏歛袵拜。」

西門慶看畢,教來安兒:「連盒兒送與大娘瞧去。管情後日去不成。寔和你說,明日是你三娘生日,家中又是安郎中擺酒。二十八日他又要往看夏大人娘子去。如何的成?」伯爵道:「哥殺人,嫂子不去,滿園中菓子兒再靠著誰哩?我就親自進屋裡請去。」少須,只見來安拿出空盒子來了:「大娘說,多上覆,知道了。」伯爵把盒兒遞與應寶接了,笑了道:「哥,剛纔你就哄我起來。若是嫂子不去,我就把頭磕爛了,也好歹請嫂子走走去。」于是西門慶教伯爵:「你且休去,在書房中坐坐。等我梳了頭兒,咱每吃飯。」說畢,入後邊去了。這伯爵便向李銘道:「如何?剛纔不是我這般說著,他甚是惱你。他有錢的性兒,隨他說幾句罷了。常言嗔拳不笑面。如今時年,尚個奉承的,拿著大本錢做買賣,還放三分和氣,你若撑硬船兒,誰理你?休說你每隨機應變,全要似水兒活,纔得轉出錢來。你若撞東墻,別人吃飯飽了,你還忍餓。你答應他幾年,還不知他性兒?明日交你桂姐趕熱腳兒來,兩當一兒,就與三娘做生日,就與他陪了禮來兒,一天事多了了。」李銘道:「二爹說得是,小的到家,過去就對三媽說。」說著,只見來安兒放卓兒,說道:「應二爹請坐,爹就出來。」不一時,西門慶梳洗出來,陪伯爵坐的,問他:「你連日不見老祝、孫天化?」伯爵道:「我不令他來,他知道哥惱他。我便說,還是哥十分情分,看上顧下。那且蝹蟲螞蚱,一例撲了去,你敢怎樣的?他每發下誓,再不和玉家小廝走。說哥昨日在他家吃酒菜,他每也不知道。」西門慶道:「昨日他如此這般,置了一席大酒請了我,拜認我做乾老子。吃到二更來了。他每怎樣的再不和來往?只不干礙著我的事,隨他去,我管他怎的?我不真個是他老子,我管他不成?」伯爵道:「哥這話說絕了,他兩個一二日也要來與你服個禮兒,解釋解釋。」西門慶道:「你教他只顧來,平白服甚禮?一面來安兒拿上飯來,無非是炮烹美口餚饌。西門慶吃粥,伯爵用飯。吃畢,西門慶問:「那兩個小優兒來了不曾?」來安道:「來了這一日了。」西門慶叫他了和李銘一答兒吃飯。一個韓佐,一個邵鎌,向前來磕了頭,下邊吃飯去了。良久,伯爵起身說道:「我去罷,家里不知怎樣等著我哩。小人家兒幹事最苦。先從爐臺底下,直買起到堂屋門首,那些兒不要買?」西門慶道:「你去幹了事,晚間來坐坐。與你三娘上壽,磕個頭兒,也是你的孝順。」伯爵道:「這個已定來,還教房下送人情來。」說畢,一直去了。正是:」

「得意友來情不厭,  知心人至話相投。」」

有詩為證:

「順情說好話,  幹直惹人嫌;

世事淡方好,  人情耐久看。」

畢竟未知後來何如,且聽下回分解:


