%# -*- coding: utf-8 -*-
%!TEX encoding = UTF-8 Unicode
%!TEX TS-program = xelatex
% vim:ts=4:sw=4
%
% 以上设定默认使用 XeLaTex 编译,并指定 Unicode 编码,供 TeXShop 自动识别

%第六回 
\chapter{西門慶買囑何九\KG 王婆打酒遇大雨}

\begin{showcontents}{}



「可怪狂夫戀野花,  因貪淫色受波喳,

亡身喪命皆因此,  破業傾家總為他;

半晌風流有何益,  一般滋味不須誇,

一朝禍起蕭墻內,  虧殺王婆先做牙。」

卻說西門慶便對何九說去了。且說王婆拿銀子來買棺材冥器,又買些香燭紙錢之類,歸來與婦人商議,就于武大靈前點起一盞隨身燈。鄰舍街坊都來看望,那婦人虛掩着粉臉假哭。眾街坊問道:「大郎得何病患便死了?」那婆娘答道:「拙夫因害心疼得慌,不想一日一日越重了,看來不能夠好,不幸昨夜三更鼓死了,好是苦也!」又哽哽咽咽假哭起來。眾鄰舍明知道此人死的不明,不敢只顧問他。眾人盡勸道:「死是死了,活的自要安穩過;娘子省煩惱,天氣暄熱。」那婦人只得假意兒謝了,眾人各自散去。王婆抬了棺材來,又去請仵作團頭何九,但是入殮用的都買了;并家裡一應物件,也都買了。就于報恩寺叫了兩個禪和子,晚夕伴靈拜懺。不多時,何九先撥了幾個火家整頓。且說何九到巳牌時分,慢慢的走來,到紫石街巷口,迎見西門慶,叫道:「老九何往?」何九答道:「小人只去前面,殮這賣炊餅的武大郎屍首。」西門慶道:「且借一步說話。」何九跟着西門慶,來到轉角頭一個小酒店裡坐下,在閣兒內,西門慶道:「老九請上坐。」何九道:「小人是何等之人,敢對大官人一處坐的!」西門慶道:「老九何故見外?且請坐!」二人讓了一回坐下。西門慶吩咐酒保:「取瓶好酒來!」酒保一面舖下菜蔬菓品案酒之類,一面盪上酒來。何九心中疑忌,想道:「西門慶自來不曾和我吃酒,今日這盃酒,必有蹺蹊。」兩個飲勾多時,只見西門慶自袖子裡摸出一錠雪花銀子,放在面前,說道:「老九,休嫌輕微,明日另有酬謝。」何九叉手道:「小人無半點用功效力之處,如何敢受大官人見賜銀兩?若是大官有使令,小人也不敢辭!」西門慶道:「老九,休要見外,請收過了。」何九道:「大官人便說不妨。」西門慶道:「別無甚事。少刻他家自有些辛苦錢,只是如今殮武大的屍身,凡百事周全,一床錦被遮蓋則個,余不多言。」何九道:「我道何事,這些小事,有甚打緊?如何敢受大官人銀兩!」西門慶道:「老九!你若不受時,便是推卻。」何九自來懼西門慶是個刁徒,把持官府的人,只得收了銀子,又吃了幾盃酒。西門慶呼酒保來,記了帳目,明日來我舖子內支錢。兩個下樓,一面出了店門。臨行,西門慶道:「老九,是必記心!不可泄漏,改日另有補報!」吩咐罷,一直去了。何九心中疑忌:「我殮武大身屍,他何故與我這十兩銀子?此事必蹺蹊。」一面來到武大門首,只見那幾個火家,正在門首伺候,王婆也等的久哩。火家在那裡,何九便問火家:「這武大是甚病死了?」火家道:「他家說害心疼病死了。」何九入門,揭起簾子進來。王婆接着道:「久等多時了,陰陽也來了半日,老九如何這咱纔來?」何九道:「便是有些小事絆住了腳,來遲了一步。」只見那婦人穿着一件素淡衣裳,白布䯼髻,從裏面假哭出來。何九道:「娘子省煩惱,大郎已是歸天去了。」那婦人虛掩着淚眼道:「說不得的苦!拙夫心疼症候,幾個日子便把命丟了。撇得奴好苦!」這何九一面上上下下看了婆娘的模樣,心裡自忖的道:「我從來只聽得人說武大娘子,不曾認得他。原來武大郎討得這個老婆在屋裡!西門慶這十兩銀子使着了!」一面走向靈前,看武大屍首。陰陽宣念經畢,揭起千秋旛,扯開白絹,用五輪八寶翫着那兩點神水。定睛看時,見武大指甲青,唇口紫,面皮黃,眼皆突出,就知是中惡。傍邊那兩個火家說道:「怎的臉也紫了,口唇上有牙痕,口中出血?」何九道:「休得胡說!兩日天氣十分炎熱,如何不走動些?」一面七手八腳,葫蘆提殮了,裝入棺材內,兩下用長命釘釘了。王婆一力攛掇,拏出一吊錢來,與何九打發眾火家去了。就問:「幾時出去?」王婆道:「大娘子說,只三日便出殯,城外燒化。」眾火家各分散了。那婦人當夜擺着酒請人。第二日,請四個僧念經。第三日早五更,眾火家都來扛抬棺材,也有幾個鄰舍街坊,吊孝相送。那婦人帶上孝,坐了一乘轎子,一路上口內假哭養家人,來到城外化人場上,便教舉火,燒化棺材,并武大屍首,燒得乾乾淨淨,把骨殖撤在池子裏。原來那日齋堂管待,一應都是西門慶出錢整頓。那婦人歸到家中樓上去了,設個靈牌,上寫:「亡夫武大郎之靈」。靈床子前,點一盞琉璃燈,裏面貼些經旛錢布,金銀錠之類。那日卻和西門慶做一處,打發王婆家去,二人在樓上任意縱橫取樂,不比先前在王婆茶坊裡,只是偷雞盜狗之歡;如今武大已死,家中無人,兩個恣情肆意停眠整宿。初時西門慶恐鄰舍瞧破,先到王婆那邊坐一回,今武大死後,帶着跟隨小廝,逕從婦人家後門而入。自此和婦人情沾肺腑,意密如膠,常時三五夜不曾歸去,把家中大小,丟的七顛八倒,都不喜歡。原來這女色坑陷得幾時,必有敗!有鷓鴣天為證:

「色膽如天不自由,  情深意密兩綢膠,

貪歡不管生和死,  溺愛誰將身體修;

只為恩深情欝欝,  多因愛闊恨悠悠,

要將吳越冤仇解,  地老天荒難歇休。」

光陰迅速,日月如梭。西門慶刮剌那婦人,將兩個月餘。一日將近端陽佳節,但見:

「綠楊裊裊垂絲碧,海榴點點胭脂赤。微微風動幔,颯颯涼侵扇;處處遇端陽,家家共舉觴。」

西門慶自岳廟上回來,到王婆茶坊裡坐下。那婆子連忙點一盞茶來,便問:「大官人往那裡去來?怎的不過去看看大娘子?」西門慶道:「今日往廟上走走,大節間,記掛着,來看看大姐。」婆子道:「今日他娘潘媽媽在這裡,怕還未去哩。等我過去看看,回大官人。」這婆子一面走過婦人後門看時,婦人正陪潘媽媽在房裡吃酒,見婆子來,連忙讓坐。婦人撮下笑來道:「乾娘來得正好!請陪俺娘,且吃個進門盞兒,到明日養個好娃娃!」婆子笑道:「老身又沒有老伴兒,那裡得養出來?你年小少壯,正好養哩!」婦人道:「常言:小花不結老花兒結。」婆子便看着潘媽媽:「你看,你女兒這等傷我,說我是老花子!到明日還用着我老花子!」說罷,潘媽道:「他從小兒是這等快嘴,乾娘休要和他一般見識!」原來這婆子撮合得西門慶和這婦人刮刺上了,早晚替他通事慇懃兒,提壺打酒,靠些油水養口。一面對他娘潘媽說:「你家這姐姐,端的百伶百俐,不枉了好個婦女!到明日不知什麼有福的人受用他?」潘媽媽道:「乾娘既是撮合山,全靠乾娘作成則個。」一面安下鍾筯,婦人斟酒在他面前,婆子一連陪了幾盃酒;吃得臉紅紅的,又怕西門慶在那邊等候,連忙丟了個眼色與婦人,告辭歸去。婦人就知西門慶來了,于是一力攛掇他娘起身去了,將房中收拾乾淨,燒些異香,從新把娘的殘饌撤去,另安排一席齊整酒肴,預備陪侍。西門慶從月臺上過來,婦人從梯凳接着到房中,道個萬福坐下。原來婦人自從武大死後,怎肯帶孝?樓上把武大靈牌丟在一邊,用一張白布蒙着,羹飯也不揪採,每日只是濃粧豔抹,穿顏色衣服,打扮嬌樣,陪伴西門慶做一處作歡頑耍。因見西門慶兩日不來,就罵:「負心的賊,如何撇閃了奴,又往那家另續上心甜的兒了。把奴冷丟,不來揪採!」西門慶道:「便是家中小妾,昨日沒了,殯送忙了兩日。今日往廟上去,替你置了些首飾珠翠衣服之類。」那婦人滿心歡喜。西門慶一面喚過小廝玳安來,氈包內取出,一件件把與婦人,婦人方纔拜謝收了。小女迎兒,尋常被婦人打怕了,以此不瞞他,令他拏茶與西門慶吃。一面婦人安放桌兒,陪西門慶吃茶。西門慶道:「你不消費心,我已與了乾娘銀子,買酒肉嗄飯果品去了。大節間,正要和你坐一坐。」婦人道:「此是待俺娘的,奴存下這桌整菜兒。等到乾娘買來,且有一回耽閣,咱且吃着。」婦人陪西門慶臉兒相貼,腿兒相壓,並肩一處飲酒。且說婆子提着個籃子,拏着一條十八兩秤,走到街上,打酒買肉。那時正值五月初旬天氣,大雨時行。只見紅日當天,忽一塊濕雲處,大雨傾盆相似。但見:

「烏雲生四野,黑霧鎖長空;刷剌剌漫空障日飛來,一點點擊得芭蕉聲碎。狂風相助,侵天老檜掀翻;霹靂交加,泰、華、嵩、喬震動。洗炎驅暑,潤澤田苗;洗炎驅暑,佳人貪其賞玩;潤澤田苗,行人忘其泥濘。正是:江淮河濟添新水,翠竹紅榴洗濯清。」

那婆子正打了一瓶酒,買了一籃魚肉雞鵝菜蔬菓品之類,在街上遇見這大雨,慌忙躲在人家房簷下,用手巾裹着頭,把衣服都淋濕了。等了一歇,那兩腳慢了些,大步雲飛來家。進入門來,把酒肉放在廚房下,走進房來,看見婦人和西門慶飲酒,笑嘻嘻道:「大官人和大娘子好飲酒,你看把婆子身上衣服都淋濕了,到明日就叫大官人賠我!」西門慶道:「你看老婆子,就是個賴精!」婆子道:「我不是賴精,大官人少不得賠我一疋大海青。」婦人道:「乾娘,你且飲過盪熱酒盞兒。」那婆子陪着飲了三盃,說道:「老身往廚下烘乾衣裳去。」一面走到廚下,把衣服烘乾。那雞鵝嗄飯,割切安排停當,用盤碟盛了。菓品之類,都擺在房中。盪上酒來,西門慶與婦人重斟美酒,共設佳肴,交盃疊股而飲。西門慶飲酒中間,看見婦人壁上掛着一面琵琶,便道:「久聞你善彈,今日好歹彈個曲兒,我下酒。」婦人笑道:「奴自幼初學一兩句,不十分好,官人休要笑耻。」西門慶一面取下琵琶來,摟婦人在懷,看他放在膝兒上,輕舒玉笋,款弄冰絃,慢慢彈着,唱了一個兩頭南調兒:

「冠兒不戴懶梳粧,髻挽青絲雲鬢光;金釵斜插在烏雲上。喚梅香,開籠廂,穿一套素縞衣裳,打扮的是西施模樣。出綉房,梅香,你與我捲起簾兒,燒一柱兒夜香。」

西門慶聽了,喜歡的沒入腳處,一手摟過婦人粉項來,就親了個嘴,稱誇道:「誰知姐姐你有這段兒聰明!就是小人在构欄三街兩巷相交唱的,也沒你這手好彈唱!」婦人笑道:「蒙官人抬舉,奴今日與你百依百隨。是必過後,休忘了奴家。」西門慶一面捧着他香腮,說道:「我怎肯忘了姐姐!」兩個殢雨尤雲,調笑頑耍。少頃,西門慶又脫下他一隻綉花鞋兒,擎在手內,放一小盃酒在內,吃鞋盃耍子,婦人道:「奴家好小腳兒,官人休要笑話!」不一時,二人吃得酒濃,淹閉了房門,解衣上床頑耍。王婆把大門頂着,和迎兒在廚房中,動啖用着。二人在房內,顛鸞倒鳳,似水如魚,取樂歡娛,那婦人枕邊風月,比娼妓尤甚,百般奉承,西門慶亦施逞鎗法打動,兩個女貌郎才俱在妙齡之際。有詩單道其態。詩曰:

「寂靜蘭房簞枕涼,  才子佳人至妙頑,

纔去倒澆紅臘燭,  忽然又掉夜行船;

偷香粉蝶餐花萼,  戲水蜻蜓上下旋,

樂極情濃無限趣,  靈龜口內吐清泉。」

當日西門慶在婦人家盤桓至晚,欲回家,留下幾兩散碎銀子,與婦人做盤纏。婦人再三挽留不住,西門慶帶上眼罩,由門去了。婦人下了簾子,關上大門,又和王婆吃了一回酒,各散去了。正是:

「倚門相送劉郎去,  烟水桃花去路迷。」

畢竟未知後來如何,且聽下回分解:



\end{showcontents}
