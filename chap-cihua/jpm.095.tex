%# -*- coding: utf-8 -*-
%!TEX encoding = UTF-8 Unicode
%!TEX TS-program = xelatex
% vim:ts=4:sw=4
%
% 以上设定默认使用 XeLaTex 编译,并指定 Unicode 编码,供 TeXShop 自动识别

%第九十五回 
\chapter{平安偷盜假當物\KG 薛嫂喬計說人情}

「有福莫享盡,  福盡身貧窮,

有勢莫倚盡,  勢盡冤相逢;

福宜常自惜,  勢宜常自恭,

人間勢與福,  有始多無終。」

話說孫雪娥,賣在酒家為娼不題。話分兩頭,卻說吳月娘自從大姐死了,告了陳經濟一狀到官,大家人來昭也死了。他妻一丈青帶著小鐵棍兒,也嫁人去了。來興兒看守門戶。房中繡春與了王姑子做了徒弟,出家去了。那來興兒自從他媳婦惠秀死了,一向沒有妻室。奶子如意兒要便引著孝哥兒,在他屋裡玩耍吃東西。來興兒又打酒和奶子吃,兩箇嘲戲勾來去,就刮剌上了;非止一日,但來前邊,歸入後邊,就臉紅。月娘察知其事,罵了一頓。家醜不可外揚,與了他一套衣裳,四根簪子,一件銀壽字兒,一件梳背兒,揀了箇好日子,就與了來興兒完房,做了媳婦子。白日上灶,看哥兒,後邊扶侍。到夜間,往前邊他屋裡睡去。一日,八月十五日,月娘生日。有吳大妗、二妗子,并三箇姑子,都來與月娘做生日,在後邊堂屋裡吃酒。晚夕都在孟玉樓住的廂房內,吳大妗、二妗子,三箇姑姑子,同在一處睡。聽宣卷到二更時分,中秋兒便在後邊灶上看茶,由著月娘叫,都不應。月娘親自走到上房裡,只見玳安兒正按著小玉,在炕上幹得好。看見月娘推開門進來,慌的湊手腳不迭。月娘便一聲兒也沒言語,只說得一聲:「賊臭肉!不在後邊看茶去,那屋裡師父宣了這一日卷,要茶吃,且在這裡做甚麼哩!」那小玉道:「中秋兒灶上我教他頓茶哩。」低著頭往後邊去。玳安便走出儀門,往前邊來。過了兩日,大妗子、二妗子、三箇女僧,都家去了。這月娘把來興兒房騰出,收拾了與玳安住。卻叫來興兒搬到來昭屋裡,看守大門去了。替玳安做了兩床鋪蓋,做了一身裝新衣服,盔了一頂新網新帽,做了雙新靴襪。又替小玉張了一頂䯼髻,與了他幾件金銀首飾,四根金頭銀腳簪,環墜戒指之類,兩套段絹顏色衣服,擇日完房,就配與玳安兒做了媳婦。白日裡還進來,在房中答應月娘,只晚夕臨關儀門時,便出去和玳安歇去。這丫頭揀好東好西,甚麼不拏出來和玳安吃。這月娘當看見,只推不看見。常言道:「溺愛者不明,貪得者無厭。羊酒不均,駟馬奔鎮;處家不正,奴婢抱怨。」卻說平安兒見月娘把小玉配與玳安,做了媳婦兒。與了他一間房住,衣服穿戴,勝似別人。他比玳安倒大兩歲,今年二十二歲,倒不與他妻室,一間房住。一日,在假當舖,看見傅夥計當了人家一副金頭面,一柄鍍金的鈎子,當了三十兩銀子。那家只把銀子使了一箇月,加了利錢,就來贖討。傅夥計同玳安尋出來,放在舖子大櫥櫃內的。不提防,這平安兒見財起心,就連匣兒偷了。走去南瓦子裡開坊子的武長腳家,有兩個私窠子,一箇叫薛存兒,一箇叫伴兒,在那裡歇了兩夜。王八見他使錢兒猛大,匣子蹙著金頭面,撅著銀挺子打酒,與鴇兒買東西。戳於土番,就把截在屋裡,打了兩箇耳刮子,就拏了。也是合當有事,不想吳典恩新陞巡檢,騎著馬,頭裡打著一對板子,從街上過來。看見問:「拴的甚麼人?」土番跪下稟說:「如此這般,拐帶出來瓦子裡宿娼,拏金銀頭面行使。小的可疑,拿了。」吳典恩分付:「與我帶來審問。」一面拿到巡檢廳兒內。吳典恩坐下,兩邊弓皂排列。土番拴平安兒到根前,認的是吳典恩,當初是他家夥計:「已定見了我就放的。」開口就說:「小的是西門慶家平安兒。」吳典恩道:「你既是他家人,拿這金東西,在這坊子裡做甚麼?」平安道:「小的大娘借與親戚家頭面戴,使小的取去。來晚了,城門閉了,小的投在坊子權借宿一夜。不料被土番拿了。」吳典恩罵道:「你這奴才胡說!你家只是這般頭面多,金銀廣,教你這奴才把頭面拿出來老婆家歇宿行使!想必是你偷盜出來頭面,趁早說來,免我動刑!」平安道:「委的親戚家借去頭面,家中大娘使我討去來,並不敢說謊。」吳典恩大怒,罵道:「此奴才真賊!不打如何肯認?」喝令左右:「與我拿夾棍夾這奴才!」一面套上夾棍起來,夾的小廝猶如殺豬叫,叫道:「爺,休夾小的,放小的實說了罷!」吳典恩道:「你只實說,我就不夾你。」平安兒道:「小的偷的假當舖當的人家一副金頭面,一柄鍍金鈎子。」吳典恩問道:「你因甚麼偷出來?」平安道:「小的今年二十二歲,大娘許了替小的娶媳婦兒,不替小的娶。家中使的玳安兒小廝,纔二十歲,倒把房裡丫頭配與他完了房。小的因此不憤,纔偷出假當舖這頭面走了!」吳典恩道:「想必是這玳安兒廝,與吳氏有奸,纔先把丫頭與他配了妻室。你只實說,沒你的事,我便饒了你。」平安兒道:「小的不知道。」吳典恩道:「你不實說,與我拶起來。」左右套上拶子。慌的平安兒沒口子說道:「爺休拶小的,等小的說就是了。」吳典恩道:「可又來!你只說了,須沒你的事!」一面放了拶子。那平安說:「委的俺大娘與玳安兒有奸,先要了小玉丫頭。俺大娘看見了,就沒言語,倒與了他許多衣服首飾東西,配與他完房。」這吳典恩一面令吏典上來抄了他口詞,取了供狀,把平安監在巡檢司,等著出牌提吳氏、玳安、小玉來審問這件事。那日卻說解當舖櫥櫃裡不見了頭面,把傅夥計諕慌了。問玳安,玳安說:「我在生藥舖子裡看,你在這邊吃飯,我不知道。」傅夥計道:「我把頭面匣子放在櫥裡,如何不見了?」一地裡尋平安兒尋不著,急的傅夥計插香賭誓。那家子討頭面,傅夥計只推還沒尋出來哩。那人走了幾遍,見沒有頭面,只顧在門前嚷道,說:「我當了兩箇月,本利不少你的,你如何不與我?頭面、鈎子,值七八十兩銀子!」傅夥計見平安兒一夜沒來家,就知是他偷出去了。四下使人找尋不著。那討頭面主兒,又在門首嚷亂。對月娘說,賠他五十兩銀子,那人還不肯,說:「我頭面值六十兩。鈎子連寶石珠子鑲嵌,共值十兩。該賠七十兩銀子。」傅夥計又添了他十兩,還不肯,定要與傅夥計合口。正鬧時,有人來報說:「你家平安兒偷了頭面,在南瓦子養老婆,被吳巡檢拏在監裡。還不教人快認贓去?」這吳月娘聽見吳典恩做巡檢,是咱家舊夥計,一面請吳大舅來商議。連忙寫了領狀,第二日教傅夥計領贓去:「有了原物在,省得兩家賴。教人家人在門前放屁!」傅夥計拿狀子到巡檢司,實承望吳典恩看舊時分上,領得頭面出來。不想反被吳典恩老狗老奴才儘力罵了一頓,叫皂隸拉倒要打。褪去衣裳,把屁股脫了半日,饒放起來。說道:「你家小廝在這裡供出吳氏與玳安許多奸情來。我這裡申過府縣,還要行牌提取吳氏來對證。你這老狗骨頭,還敢來領贓!」倒吃他千奴才萬老狗,罵將出來,諕的往家中走不迭。來家不敢隱諱,如此這般,對月娘說了。月娘不聽便罷,聽了,正是:

「分開八塊頂梁骨,  傾下半桶冰雪來。」

慌的手腳麻木!又見那個討頭面人在門前大嚷大鬧,說道:「你家不見了我頭面,又不與我原物,又不賠我銀子,只哄著我兩頭回來走!今日哄我去領贓,明日等領頭面。端的領的在那裡?這等不合理!」那傅夥計陪下,拖將好言央及安撫他:「略從容兩日,就有頭面出來了。若無原物,加倍賠你!」那人說:「等我回聲當家的去。」說畢去了。這吳月娘憂上加憂,眉頭不展,使小廝請吳大舅來商議,教他尋人情對吳典恩說,掩下這樁事罷。吳大舅說:「只怕他不受人情,要些賄賂打點他。」月娘道:「他當初這官,還是咱家照顧他的。還借咱家一百兩銀子,文書俺爹也沒收他的。今日反恩將仇報起來!」吳大舅說:「姐姐說不的那話了!從來忘恩背義,纔一箇兒也怎的?」吳月娘道:「累及哥哥,上緊尋箇路兒。寧可送他幾十兩銀子罷,領出頭面來,還了人家,省得合費舌!」打發吳大舅吃了飯去了。月娘送哥哥到大門首。也是合當事情湊巧,只見薛嫂兒提著花箱兒,領著一箇小丫鬟過來。月娘叫住便問:「老薛,你往那裡去?怎的一向不來俺這裡走走?」薛嫂道:「你老人家倒且說的好,這兩日好不忙哩!偏有許多頭緒兒!咱家小奶奶那裡使牢子、大官兒,叫了好幾遍,還不得空兒去哩!」月娘道:「你看媽子撒風!你又做起俺小奶奶來了!」薛嫂道:「如今不做小奶奶,倒做了大奶奶了!」月娘道:「他怎的做大奶奶?」薛嫂道:「你老人家還不知道,他好小造化兒!自從生了哥兒,大奶奶死了,守備老爺就把他扶了正房,做了封贈娘子!正景二奶奶孫氏,不如他。手下買了兩箇奶子,四箇丫頭扶侍。又是兩箇房裡得寵學唱的姐兒,都是老爺收用過的。要打時就打他倘棍兒!老爺敢做的主兒?自恁還恐怕氣了他!那日不知因甚麼,把雪娥娘子打了一頓,把頭髮都撏了。半夜叫我去領出來,賣了八兩銀子。如今孫二娘房裡,使著箇荷花丫鬟。他手裡倒使著四五箇,又是兩箇奶子,還言人少!二娘又不敢言語,成日奶奶長奶奶短,只哄著他。前日對我說:『老薛,你替我尋箇小丫頭來我使。』嫌那小丫頭不會做生活,不會上灶。他屋裡事情冗雜。今日我還睡哩,大清早辰,又早使牢子叫了我兩遍,教我快往宅裡去。問我要兩副大翠重雲子鈿兒,又要一付九鳳鈿銀根兒,一箇鳳口裡啣一串珠兒,下邊墜著青紅寶石金牌兒。先與了我五兩銀子。銀子不知使的那裡去了,還沒送與他生活去哩!這一見了我,還不知怎生罵我哩!我如今就送這丫頭去。」月娘道:「你到後邊,等我瞧瞧怎樣翠鈿兒?」一面讓薛嫂到後邊明間內坐下。薛嫂打開花箱,取出與吳月娘看。果然做的好樣範!約四指寬,通掩過䯼髻來,金翠掩映,翡翠重疊,背面貼金。那九級鈿,每箇鳳口內啣著一掛寶珠牌兒,十分奇巧。薛嫂道:「自這付鈿兒做著本錢三兩五錢根子。那付重雲子的,只一兩五錢銀子,還沒尋他的錢。」正說著,只見玳安兒走來,對月娘說:「討頭面的又來這邊嚷哩。等不的領贓:『領到幾時?』若明日沒頭面,要和傅二叔打了,到箇去處理會哩!傅二叔心裡不好,往家去了。那人嚷了回去了。」薛嫂問:「是甚麼勾當?」月娘便長吁了一口氣,如此這般告訴薛嫂說:「平安兒奴才偷去印子舖人家當的一付金頭面,一箇鍍金的鈎子,走在城外坊子裡養老婆。被吳巡檢拏住,監在監裡。人家來討頭面,沒有,在門前嚷鬧。吳巡檢又勒掯刁難,不容俺家領贓。打夥計,將來要錢。白尋不出箇頭腦來!如何是好?死了漢子,敗落一齊來,就這等被人欺負!好苦也!」說著,那眼中淚紛紛落將下來。薛嫂道:「好奶奶,放著路兒不會尋!咱家小奶奶,你這裡寫箇帖兒,等我對他說聲,教老爺差人分付巡檢司;莫說一副頭面,就十副頭面,也討去了!」月娘道:「周守備他是武職官,他管的著那巡檢司?」薛嫂道:「奶奶你還不知道。如今周爺,朝廷新與他的勑書,好不管的事情寬廣!地方河道,軍馬錢糧,都在他裡打卯遞手本。又河東水西,捉拏強盜賊情,正在他手裡。」月娘聽了,便道:「既然管著,老薛就累你多上覆龐大姐聲。一客不煩二主,教他在周爺面前,美言一句兒,問巡檢司討出頭面來,我破五兩銀子謝你。」薛嫂道:「好奶奶,錢恁中使!我見你老人家剛纔慘惶,我倒下意不去。你教人寫了帖兒,不吃茶罷。等我到府裡和小奶奶說成了,隨你老人家。不成,我還來回你老人家話。」這吳月娘一面叫小玉擺茶與薛嫂吃。薛嫂兒道:「這咱晚了,不吃罷。你只教大官兒寫了帖兒,我拏了去罷。你不知我一身的事在我身上哩!」月娘道:「我曉的你也出來這半日了,吃了點心兒去。」小玉即便放卓兒,擺上茶食來。月娘陪他吃茶。薛嫂兒遞與丫頭兩箇點心吃。月娘問:「丫頭幾歲了?薛嫂道:「今年十二歲了。」不一時,玳安兒前邊寫了說帖兒。薛嫂兒吃了茶,放在袖內,作辭月娘,提著花箱出門,轉灣抹角,逕到守備府中。春梅還在緩床炕上睡,還沒起來哩。只見大丫鬟月桂進來說:「老薛來了。」春梅便叫小丫頭翠花,把裡面窗撩開了。日色照的紗窗,十分明亮。薛嫂進去說道:「奶奶這裡還未起來?」放下花箱,便磕下頭去。春梅道:「不當家化化的,磕甚麼頭!」說道:「我心裡不自在,今日起來的遲些。」問道:「你做的我翠雲子,和九鳳鈿兒,拏了來不曾?」薛嫂道:「奶奶這兩副鈿兒,好不費手!昨日晚夕,我纔打翠花舖子裡討將來。今日要送來,不想奶奶又使了牢子去。」一面取出來,與春梅過目。春梅還嫌翠雲子做的不十分現撇,還安放在紙匣兒內,交與月桂收了,看茶與薛嫂兒吃。薛嫂便叫小丫鬟進來,與奶奶磕頭。春梅問:「是那裡的?」薛嫂兒道:「二奶奶和我說了好幾遍,說荷花只做的飯,教我替他尋箇小孩子,學做些針指。我替他領了這箇孩子來了。到是鄉裡人家女孩兒,今年纔十二歲,正是養材兒。只好狗漱着學做生活。」春梅道:「你亦發替他尋箇城裡孩子,還伶便些。這鄉裡孩子,曉的甚麼?也是前日一箇張媽子,領了兩箇鄉裡丫頭子來。一箇十一歲,那一箇十二歲了。一箇叫生金,一個叫活寶。兩箇且是不善,都要五兩銀子,孃老子就在外頭等著要銀子。我說且留他住一日兒,試試手兒,會答應不會,教他明日來領銀子罷。死活留下他一夜。丫頭們不知好歹,與了他些肉湯子泡飯吃了。到第二日天明,只見丫頭們嚷亂起來。我便罵賊奴才,亂的是甚麼?原來那生金,撒了被窩尿;那活寶溺的褲子提溜不動!把我又是那笑,又是那砢硶。等的張媽子來,還教他領的去了。」因問:「這丫頭要多少銀子?」薛嫂兒道:「要不多。只四兩銀子,他老子要投軍使。」春梅教海棠:「你領到二娘房裡去,明日兌銀子與他罷。」又叫月桂:「拏大壺內有金華酒 ,篩來與薛嫂兒吃盪寒。再有甚點心,拏上一盒子與他吃。」又說:「大清早辰,拏寡酒灌他。」薛嫂道:「桂姐,且不要篩上來,等我和奶奶說了話著。剛纔在那裡,也吃了些甚麼來了。」春梅道:「你對我說,在誰家吃甚來?」薛嫂道:「剛纔大娘那頭,留我吃了些甚麼來了。如此這般,望著我好不哭哩!說平安兒小廝,偷了印子鋪內人家當的金頭面,還有一把鍍金的鈎子,在外面養老婆。吃番子拏在巡檢司拶打。這裡人家要頭面嚷亂,使傅夥計領贓。那吳巡檢舊日是咱那裡夥計,有爹在日,照顧他的官。今日一旦反面無恩,夾打小廝攀扯人。又不容這裡領贓,要錢纔准,把夥計打罵將來,諕的夥計不好了,躲的往家去了。央我來,多多上覆你老人,不知咱家老爺管的著這巡檢司。可憐見舉眼兒無親的,教你替他對老爺說聲,領出頭面來,交付與人家去了,大娘親來拜謝你老人家。」春梅問道:「有箇帖兒沒有?不打緊,有你爺出巡去了,怕不的今晚來家,等我對你爺說。」薛嫂兒道:「他有說帖兒有此。」向袖中取出。這春梅看了,順手就放在窗戶檯上。不一時,托盤內拿上四樣嗄飯菜蔬。月桂拏大銀鍾,滿滿斟了一鍾,流沿兒遞與薛嫂道:「我的奶奶,我原捱內了這大行貨子!」春梅笑道:「比你家老頭子那大貨差些兒。那箇你倒捱了,這箇你倒捱不的?好歹與我捱了。要不吃,月桂你與我捏著鼻子灌他。」薛嫂道:「你且拏了點心,與我打了底兒著。」春梅道:「這老媽子單管說謊!你纔說在那裡吃了來,這回又說沒打底兒?」薛嫂道:「吃了他兩箇茶食,這咱還有哩?」月桂道:「薛媽媽,你且吃了這大鍾酒,我拏點心與你吃。俺奶奶又怪我沒用,要打我哩!」這薛嫂沒奈何,只得吃了。被他灌了一鍾,覺心頭小鹿兒劈劈跳起來。那春梅努努箇嘴兒,又叫海棠斟滿一鍾教他吃。薛嫂推過一邊,說:「我的好孃人家,我卻一點兒也吃不的了!」海棠道:「你老人家捱了月桂姐一下子,不捱我一下子,奶奶要打我!」那薛嫂兒慌的直撅兒跪在地下。春梅道:「也罷,你拏過那餅與他吃了,教他好吃酒。」月桂道:「薛媽媽,誰似我恁疼你?留下恁好玫瑰果餡餅兒 與你吃!」就拿過一大盤子頂皮酥玫瑰餅兒來。那薛嫂兒只吃了一箇,別的春梅都教他袖在袖子裡:「到家稍與你家老王八吃!」薛嫂兒吃酒,蓋著臉兒,把一盤子火薰肉,醃臘鵝 ,都用草紙包,布子裹,塞在袖內。海棠使氣白賴,又灌了半鍾酒。見他嘔吐上來,纔收過家伙去,不要他吃了。春梅分付:「明日來討話說,兌丫頭銀子與你。」又使海棠問孫二娘去。回來說:「丫頭留下罷,教大娘娘與他銀子。」臨出門拜辭,春梅分付:「媽媽,休推聾裝啞。那翠雲子做的不好,明日另帶兩副好的我瞧。」薛嫂道:「我知道。奶奶叫箇大姐送我送,看狗咬了我腿。」春梅笑道:「俺家狗都有眼,只咬到骨禿根前就住了。」一面使蘭花送出角門來。話休饒舌。周守備至日落時分,牌兒馬藍旗作隊,叉槊後隨,出巡來家。進入後廳,左右丫鬟接了冠服,進房見了春梅小衙內,心中歡喜,坐下。月桂、海棠拿茶吃了。將出巡回之事,告訴一遍。不一時,放卓兒擺飯。飯罷,掌上燭,安排盃酌飲酒。因問:「前邊沒甚事?」一面取過薛嫂拿的帖兒來與守備看,說:「吳月娘那邊如此這般,小廝平安兒偷了頭面,被吳巡檢拏住監禁,不容領贓,只拷打小廝,攀扯誣賴吳氏奸情,索要銀兩,呈詳府縣等事。」守備看了說:「此事正是我衙門裡事,如何呈詳府縣?吳巡檢那廝這等可惡!我明出牌,連他都提來發落!」又說:「我聞得這吳巡檢是他門下夥計,只因往東京與蔡太師進禮,帶挈他做了這箇官。如何倒要誣害他家?」春梅道:「見是這等說。你替他明日處處罷。」一宿晚景題過。次日旋教吳月娘家補了一紙狀,當廳出了箇大花欄批文,用一箇封套裝了。上面批:「山東守禦府為失盜事,仰巡檢司官,連人解繳。右差虞侯張勝、李安准此。」當下二人領出公文來,先到吳月娘家。月娘管待了酒飯,每人與了一兩銀子鞋腳錢。傅夥計家中睡倒了。吳二舅跟隨到巡檢司。吳巡檢見平安監了兩日,不見西門慶家中人來打點。正教吏典做文書,申呈府縣。只見守禦府中兩箇公人到了,拏出批文來與他。見封套上朱紅筆標著:「仰巡檢司官連人解繳。」拆開見裡面吳氏狀子,諕慌了。反賠下,拖與李安、張勝每人二兩銀子。隨即做文書,解人上去。到于守備府前,伺候半日。待約守備升廳,兩邊軍牢排下,然後帶進人去。這吳巡檢把文書呈遞上去。守備看了一遍,說:「此正是我這衙內裡事,如何不申解前來我這裡發送?只顧延捱監滯,顯有情弊!」那吳巡檢稟道:「小官纔待做文書申呈老爺案下,不料老爺鈞批到了。」守備唱道:「你這狗官可惡!多大官職,這等欺玩法度,抗違上司!我欽奉朝廷勅命,保障地方,巡捕盜賊,提督軍門,兼管河道,職掌開載已明。你如何拏了起件,不行申解?妄用刑杖拷打犯人,誣攀無辜,顯有情弊!」那吳巡檢聽了,摘去冠帽在階前只顧磕頭。守備道:「本當參治你這狗官,且饒你這遭。下次再若有犯,定行參究!」一面把平安提到廳上說道:「你這奴才,偷盜了財物,還肆言謗主人家!都是你恁如此,也不敢使奴才了!」喝令左右:「與我打三十大棍,放;將贓物封貯,教本家人來領去。」一面喚進吳二舅來,遞了領狀,守備這裡還差張勝拏帖兒同送到西門慶家,見了分上。吳月娘打發張勝酒飯,又與了一兩銀子。走來府裡,回了守備、春梅話。那吳巡檢乾拏了平安兒一場,倒折了好幾兩銀子。月娘還了那人家頭面、鈎子兒。是他原物,一聲兒沒言語去了。傅夥計到家,傷寒病睡倒了。只七日光景,調治不好,嗚呼哀哉死了!月娘見這等合氣,把印子舖只是收本錢贖討,再不假當出銀子去了。止是教吳二舅同玳安在門首生藥舖子,日逐轉得來家中盤纏。此事表過不題。一日,吳月娘叫將薛嫂兒來,與了三兩銀子。薛嫂道:「不要罷,傳的府裡小奶奶怪我。」月娘道:「天不使空人,多有累你!我見他不題出來就是了。」于是買了四盤下飯,宰了一口鮮豬,一罈南酒 ,一疋紵絲尺頭,薛嫂押著,來守備府中致謝春梅。玳安穿著青絹褶兒,用描金匣兒盛著禮帖兒,逕到裡邊見春梅。薛嫂領著到後堂,春梅出來,戴了金梁冠兒,金釵梳,鳳鈿,上穿繡襖,下著錦裙,左右丫鬟養娘侍奉。玳安兒扒倒地下磕頭。春梅分付放卓兒擺茶食,與玳安吃。說道:「沒上事,你奶奶免了罷。如何又費心送這許多禮來?你周爺已定不肯受。」玳安道:「家奶奶說,前日平安兒這場事,多有累周爺、周奶奶費心。沒甚麼,些小微禮兒,與爺、奶奶賞人便了。」春梅道:「如何好受的?」薛嫂道:「你老人家若不受,惹那頭又怪我!」春梅一面又請進守備來計較了,止受了豬酒下飯,把尺頭回將來了。與了玳安一方手帕,三錢銀子。抬盒人二錢。春梅因問:「你奶奶、哥兒好麼?」玳安說:「哥兒好不耍子兒哩!」又問玳安兒:「你幾時籠起頭去包了網巾?幾時和小玉完房來?」玳安道:「是八月內來。」春梅道:「到家多頂上你奶奶,多謝了重禮!待要請你奶奶來坐坐,你周爺早晚又出巡去。我到過年正月裡哥兒生日,我往家裡走走。」玳安道:「你老人家若去,小的到家就對俺奶奶說,到那日來接奶奶。」說畢,打發玳安出門。薛嫂便向玳安兒說:「大官兒,你先去罷,奶奶還要與我說話哩!」那玳安兒押盒擔來家,見了月娘說:「如此這般,守備只受了豬酒下飯,把尺頭回將來了。春梅姐讓到後邊,管待茶食吃。問了回哥兒好,家中長短,與了我一方手帕,三錢銀子。抬盒人二錢銀子。多頂上奶奶,多謝重禮!都不受來,被薛嫂兒和我再三說了,纔受了下飯豬酒,抬回尺頭。要不是,請奶奶過去坐坐。一兩日周爺出巡去。他只到過年正月孝哥生日,來家裡走走。」告說:「他住著五間正房,穿著錦裙繡襖,戴著金梁冠兒,出落的越發胖大了!手下好少丫頭、奶子侍奉!」月娘問:「他其實說明年往咱家來?」玳安兒道:「委的對我說來。」月娘道:「到那日咱這邊使人接他去。」因問:「薛嫂怎的還不來?」玳安道:「我出門,他還坐著說話,教我先來了。」自此兩家交往不絕。正是:

「世情看冷暖,  人面逐高低!」

有詩為證:

「得失榮枯命裡該,  皆因年月日時栽;

胸中有志應須至,  囊裡無財莫論才。」

畢竟未知後來如何,且聽下回分解:

