%# -*- coding: utf-8 -*-
%!TEX encoding = UTF-8 Unicode
%!TEX TS-program = xelatex
% vim:ts=4:sw=4
%
% 以上设定默认使用 XeLaTex 编译,并指定 Unicode 编码,供 TeXShop 自动识别

%第七十七回 
\chapter{西門慶踏雪訪鄭月\KG 賁四嫂倚牖盼佳期}


「飛彈參差拂早梅,  強欺寒色尚低回,

風憐落娼留香與,  月令深情借艷開;

梁殿得非肖帝瑞,  齊宮應是玉兒媒,

不知謝客離腸醒,  臨水應添萬恨來。」

話說溫秀才求見西門慶不得,自知慚愧,隨携家小搬移原舊家去了。西門慶收拾書院,做了客座,不在話下。一日尚舉人來拜辭,起身上京會試,問西門慶借皮箱毡衫。西門慶陪他坐的待茶,又送贐禮與他。因說起:「喬大戶、雲離守兩位舍親,一授義官,一襲祖職,見任管事。欲求兩篇軸文奉賀,不知老翁可有相知否?借重一言,學生具幣禮拜求。」尚舉人笑道:「老翁何用禮為?學生敝同窗聶兩湖,見在武庫肄業,與小兒為師在舍,本領雜作極富。學生就與他說,老翁差盛使持軸,送到學生那邊。」西門慶連忙致謝,茶畢起身。西門慶這裡隨即封了兩方手帕、五錢白金,差琴童送軸子并毡杉皮箱到尚舉人處收下。那消兩日光景,寫成軸文,差人送來。西門慶挂在壁上,但見青段錦軸,金字輝煌,文不加點,心中大喜。只見應伯爵來問:「喬大戶與雲二哥的事,幾時舉行?軸文做了不曾?溫老先兒怎的連日不見?」西門慶道:「又題甚麼溫老先生兒?通是個狗類之人!」如此這般,告訴伯爵一遍。伯爵道:「哥,我說此人言過其實,虛浮之甚!早時你有後眼,,不然教調壞了咱家小兒們了!」又問:「他二公賀軸,何人寫了?」西門慶道:「昨日尚小塘來拜我,說他朋友聶兩湖善於詞藻,央求聶兩湖作了文章,已寫了來,你瞧。」于是引伯爵到廳上,觀看一遍,喝采不已。說道:「人情都全了。哥,你早送與人家預備。」西門慶道:「明日好日期,備羊酒花紅菓盒,早差人送去。」正說着,忽報:「夏老爹兒子來拜辭,明日初八日早起身去也。小的答應爹不在家,他說教對何老爹那里,明早差人那邊看守去。」西門慶觀見六摺帖兒上寫着:「寅家晚生夏承恩頓首拜,謝辭。」西門慶道:「連尚舉人搭他家,就是兩分香絹贐儀。」分付琴童:「連忙買了,教你姐夫封了,寫帖子送去。」正在書房中留伯爵吃飯,忽見平安兒慌慌張張,拿進三個帖兒來報:「參議汪老爹,兵備雷老爹,郎中安老爹來拜。」西門慶看帖兒,「江伯彥、雷啟元、安忱拜。」連忙穿衣裳繫帶。伯爵道:「哥,你有事,我吃了飯去罷。」西門慶道:「我明日會你哩。」一面整衣出迎,三員官皆相讓而入,一個白鵬,一個雲鷺、一個穿豸補子,手下跟從許多官吏。進入大廳敘禮,道及向日厚擾之事。少頃,茶罷,坐話間,安郎中便道:「雷東谷、汪少華并學生又來干凟,有浙江本府趙大尹,新陞大理寺正,學生三人借尊府奉請。已發柬,定初九日赴會。主家共五席,戲子學生那里叫來。未知肯允諾否?」西門慶道:「老先生分付,學生埽門拱候。」安郎中令吏取分資三兩遞上。西門慶令左右收了,相送出門。雷東谷向西門慶道:「前日錢龍野書到,說那孫文相乃是舍夥計,學生已并除他開了。曾來相告不曾?」西門慶道:「正是。多承老先生費心,容當叩拜。」雷兵備道:「你我相愛間,何為多較!」言畢,相揖上轎而去。原來潘金蓮自從當家管理銀錢,另頂了一把新等子,每日小廝買進菜蔬來,教拏至跟前,與他瞧過,方數錢與他;他又不數,只教春梅數錢提等子。小廝被春梅罵的狗血噴了頭背,出生入死,行動就說落,教西門慶打。以此眾小廝皆互相抱怨,都說:「在三娘手裡使錢好,五娘行動沒打不說話。」卻說次日,西門慶早往衙門中散了,對何千戶說:「夏龍溪家小已起身去了,長官沒曾委人那里看守鎖門戶去?」何千戶道:「正是,昨日那邊着人來說,學生原差小价去了。」西門慶道:「今日同長官到那里看看去。」于是出衙門,並馬兩個到了夏家宅內。家小已是去盡了,伴當在門首伺候。兩位官府下馬,進到廳上。西門慶引着何千戶前後觀看了。又到他前邊花亭,見一片空地無甚花草。西門慶道:「長官來到,明日還收拾了耍子所在,裁些花翠,把這座亭子修理修理。」何千戶道:「這個已定。學生開春從新修整修整,添些磚瓦木石,蓋三間捲棚,早晚請長官來消閑散悶。」西門慶因問:「府上寶眷有多少來住?」何千戶道:「學生這房頭不上數口,還有幾房家人并伴當,不過十數人而已。」西門慶道:「似此還住不了,這宅子前後五十余間房。」看了一回,分付家人收拾打埽,關閉門戶,不日寫書往東京回老公公話,趕年裡搬取家眷。當日西門慶作別回家,何千戶看了一回,還歸衙門裡去了。次日纔搬行李來住,不在言表。西門慶剛到家下馬兒,見何九買了一疋尺頭,四樣下飯,雞鵝,一罈酒,來謝西門慶。又是劉內相差官送了一食盒,大小純紅挂黃蠟燭,二十張桌圍,八十股官香,一盒沉速料香,一罈自造內酒 ,一口鮮豬。西門慶進門,劉公公家人就磕頭說道:「家公公多上覆,這些微禮,與老爹賞人。」西門慶道:「前日空過老公公,送這厚禮來?」便令左右:「快收了,請管家等等兒。」少頃,畫童兒拿出一鍾茶來,打發吃了。西門慶封了五錢銀子賞錢,拿回帖打發去了。一面請何九進去。見西門慶在廳上站立,換了冠帽,戴着白毡忠靖冠,見何九,一把手扯在廳上來。何九連忙倒身磕下頭:「向蒙老爹天心,超生小人兄弟,感恩不淺!」請西門慶受禮。西門慶不肯受,磕頭,拉起還說:「老九,你我舊人,快休如此!」說道:「老爹今非昔比,小人微末之人,豈敢僭坐?」只站立在傍邊。西門慶上陪着吃了一盞茶,說道:「老九,你如何又費心送禮來?我斷然不受。若有甚麼人欺負你,只顧來說,我親替你出氣。倘縣中派你甚差事,我拿帖兒與你李老爹說。」何九道:「家老爹恩典,小人知道。小人如今也老了,差事已告與小兒何欽頂替着哩。」西門慶道:「也罷,也罷!你清閑些了。」說道:「既你不肯,我把這酒禮收了。那尺頭你還拿去,我也不留你坐了。」那何九千恩萬謝,拜辭去。西門慶坐廳上,看着打點禮物;菓盒、花紅、羊酒、軸文等,各人分資,先差玳安送往喬大戶家去。後叫王經送雲離守家去。玳安回來,喬家與了五錢銀子。王經到雲離守家,管待了茶食,與了一疋真青大布,一雙琴鞋,回門下辱愛生雙帖兒:「多上覆老爹,改日奉請。」西門慶滿心歡喜,到後邊月娘房中擺飯吃,因向月娘說:「賁四去了 吳二舅在獅子街賣貨,我今日倒閒,往那裡看去。」月娘道:「你去不是,若是要酒菜兒,早使小廝來家說。」西門慶道:「我知道。」一面分付備馬,就戴着毡忠靖巾,貂鼠暖耳,綠絨補子〈衤旋〉褶,粉底皂靴,琴童、玳安跟隨,逕往獅子街來。到房子內,吳二舅與來昭正挂着花拷拷兒,發賣紬絹絨線絲綿,擠一舖子人做買賣,打發不開。西門慶下馬,看了看,走到後邊暖房內坐下。吳二舅走來作揖,回說:「一日也攢銀錢二十兩。」西門慶又分付來昭妻一丈青:「二舅茶飯,每日這里依舊打發,休要誤了!」來昭妻道:「逐日頓美酒飯,都是我自整理。」西門慶見天陰晦上來,但見彤雲密布,冷氣侵人,作雪的模樣。忽然想起要往院中鄭月兒家去。即令琴童:「騎馬家中取我的皮襖來,問你大娘有酒菜兒,稍一盒與你二舅吃。」琴童應諾到家,不一時,取了西門慶長身貂鼠皮襖,後面排軍拿了一盒酒菜,裏面四碟醃雞下飯,煎炒鵓鴿,四碟海味案酒,一盤韮盒兒,一錫瓶酒。西門慶陪二舅在房中吃了三杯,分付二舅:「你晚夕在此上宿,自用,我家去罷。」于是帶上眼紗,騎馬,玳安、琴童跟隨,逕進构攔,往鄭愛月兒家來。轉過東街口,只見天上紛紛揚揚,飄下一天瑞雪來。正是:

「拳頭大塊空中舞,  路上行人只叫苦。」

但見:

漠漠嚴寒匝地,這雪兒下得正好;扯絮撏綿,裁織片片,大如拷栳。見林門竹笋茅茨,爭些被他壓倒!富豪俠,卻言消灾障,猶嫌小,圍向那紅爐獸炭,穿的是貂裘綉襖。手撚梅花,唱道是國家祥瑞,不念貧民些小。高臥有幽人,吟詠多詩章。

西門慶隨路踏着那亂瓊碎玉,貂襖沾濡粉蝶,馬蹄蕩滿銀花。進入构攔,到於鄭愛月兒家門首下馬。只見丫鬟看見,飛報進來說:「老爹來了。」鄭媽媽出來迎接,到中堂見禮。說道:「前月多謝老爹重禮,姐兒又在宅內打攪;又教他大娘、三娘賞他花翠汗巾。」西門慶道:「那日空了他來。」一面坐下。西門慶令玳安把馬牽進來,自有院落安放。老馮道:「請爹後邊明間坐罷。月姐纔起來梳頭,只說老爹昨日來,到伺候了一日。今日他中有不快,起來的遲些。」西門慶一面進入他後邊往房明間內,但見綠窗半啟,毡幙低張。地平上黃銅大盆,生着炭火。西門慶坐在正面椅上。先是鄭愛香兒出來相見了,遞了茶,然後愛月兒纔出來。頭挽一窩絲,杭州儹,翠梅花鈕兒,金鈒釵梳海獺臥兔兒。打扮的霧靄雲鬟,粉粧粉,香花琢。上穿白綾襖兒,綠遍地錦比甲,下着大幅湘紋裙子。高高顯一對小小金蓮,猶如新月,狀若蛾眉;好似羅浮仙子臨凡境,神女巫山降世間。粉頭出來笑嘻嘻的向西門慶道了萬福,說道:「爹,我那一日來晚了。緊自前邊人散的遲;到後邊大娘又只顧不放俺每,留着吃飯,來家有三更天了。」西門慶笑道:「小油嘴兒,你倒和李桂姐兩個,把應花子打的好響爪兒。」鄭愛月兒道:「誰教他怪物勞,在酒席上屎口兒傷俺每來。那一日,祝麻子也醉了,哄我要送俺每來。我便說沒爹這裡燈籠送俺每,蔣胖子弔在陰溝裡,缺臭了你了!」西門慶道:「我昨日聽見洪四兒說,祝麻子又會着王三官兒,大街上請了榮嬌兒。」鄭月兒道:「只在榮嬌兒家歇了一夜,燒了一炷香,不去了。如今還在秦玉芝兒走着哩。」說了一回話,道:「爹,只怕你冷,往房裡坐的。」這西門慶到了房中,脫去貂裘,和粉頭圍爐共坐。房中香氣襲人。只見丫鬟來放卓兒,四碟細巧菜蔬,安下三個薑碟兒。須臾,拿了三甌兒黃芽韮菜肉包,一寸大的水角兒來。姊妹二人陪西門慶,每人吃了一甌兒。愛月兒又撥了上半甌兒,添與西門慶。門慶道:「我勾了,纔在那邊房子線舖,陪你吳二舅吃了兩個點心來了。心裡要來你這里走走,不想天氣落雪,家中使小廝取了皮襖,穿上就來了。」愛月兒道:「爹前日不會下我?教昨日等了一日,不見爹。不想爹今日來了!」西門慶道:「昨日家中有兩位士夫來望,亂着,就不曾來得。」愛月兒道:「我要問爹,有貂鼠買個兒與我,我要做了圍脖兒戴。」西門慶道:「不打緊。昨日舍夥計打遼東來,送了我十個好貂鼠。你娘們都沒圍脖兒,到明日一總做了,送一個來與你。」愛香兒道:「爹只認的月姐,就不送與我一個兒?」西門慶道:「你姊妹兩個,一家一個。」于是愛香、愛月兒連忙起身道了萬福。西門慶分付:「休見桂姐、銀姐說。」鄭月兒道:「我知道。」因說:「到明日李桂姐見吳銀兒在那裡過夜,問我他幾時來了?我沒瞞他,教我說昨日請周爺,俺每四個都在這里唱了一日。爹說有王三官兒在這里,不敢請你的。今日是親朋會中人吃酒,纔請你來唱。他一聲兒也沒言語。」西門慶道:「你這個回的他好。前日李銘我也不要他唱來,再三央及你應二爹來說;落後,你三娘生日,桂姐買了一分禮來,再三與我陪不是,你娘們說着,我不理他。昨日我竟留下銀姐,使他知道。」愛月兒道:「不知三娘生日,我失誤了人情。」西門慶道:「等明日你雲老爹擺酒,我前日你和銀姐那里唱一日。」愛月兒道:「爹分付,我去。」不一時,丫鬟收拾飯卓去。粉頭取出個鸂鶒木匣兒,傾出三十二扇象牙牌來,和西門慶在炕毡條上抹牌頑耍。愛香兒也坐在傍邊看牌。院內雪飛風舞梨花,紛紛只顧下。但見:

「恍惚漸迷鴛甃,頃刻拂滿蜂鬚。似玉龍鱗甲遶空飛,白鶴羽毛搖地落。好若數蟹行沙上,猶賽亂瓊堆砌間。」

正是:

「盡道豐年瑞,豐年瑞若何?長安有貧者,宜瑞不宜多。」

當下三人抹了回牌勝負,須臾,擺上酒來飲酒。卓上盤堆異菓,肴列珍羞。茶煮龍團,酒斟琥珀。詞歌金縷,笑容朱唇。愛香與愛月兒一邊一個捧酒,不免箏排雁柱,款跨鮫綃,姊妹兩個彈着,唱了一套青衲襖:

「想多嬌,情性兒標;想多嬌,恩意兒好。想起携手同行共歡笑,吟風咏月將詩句兒嘲。女溫柔,男俊俏,正青春年紀小。誰人望將比目魚分開,瓶墜簪折,今日早魚沉雁杳。」

〔罵玉郎〕 「多嬌一去無消耗,想着俺情似漆,意如膠。常記的共枕同歡樂,想着他花樣嬌、柳樣柔,傾國傾城貌。」

〔大迓鼓〕 「千般丰韻嬌,風流俊俏,體態妖嬈,所為諸般妙。搊箏撥阮,歌舞吹簫,總有丹青難畫描。」

〔感皇恩〕 「呀,好教我無緒無聊!意攘心勞,懶將這杜詩溫,韓文敘,柳文學。我這里愁懷越焦,這些時容貌添憔。不能勾同歡樂,成配偶,到有分受煎熬。」

〔東歐令〕 「潘郎貌,沈郎腰,可惜相逢無下稍!心腸懊惱傷懷抱,烈火燒佛廟,滔滔綠水渰藍橋,想思病怎生逃!」

〔採茶歌〕 「相思病怎生逃,離愁人擺的堅牢,鐵石人見了也魂消!愁似南山堆積積,悶如東海水滔滔!」

〔賺〕 「誰想今朝,自古書生多命薄;傷懷抱,痴心惹的傍人笑,對難陳告?」

〔烏夜啼〕 「想當初偎紅倚翠,踏青鬬草。相逢對景同歡樂。到春來,語呢喃,燕子尋巢;到夏來,荷蓮香,開滿池沼;到秋來,菊滿荒郊;到冬來,瑞雪飄飄。想當初畫堂歌舞,列着佳肴。今日個孤枕旅館無着落,鬼病侵難醫療。好教我情牽意惹,心痒難撓。」

〔節節高〕 「悶懨懨睡不着,想多嬌,知音解呂明宮調。諸般好閉月羞花貌,言語嬌媚心聰俏,恰似仙子行來到,金蓮款步鳳頭翹,朱唇皓齒微微笑。」

〔鷯鶉兒〕 你看他體態輕盈,更那堪衣穿素縞,脂粉施蛾眉淡埽。看了他萬總妖嬈,難畫描。酒泛羊羔,寶鴨香飄,銀燭高燒,成就了美滿夫妻,穩取同心到老。」

〔尾聲〕 「青雲有路終須到,生前無分也難消,把佳期叮嚀,休忘了。」

唱一套,姐兒兩個拿上骰盆兒來,和西門慶搶紅頑笑。杯來盞去,各添春色。西門慶忽把眼看見鄭愛月兒房中牀傍側首錦屏風上,挂着一軸愛月美人圖,題詩一首:

「有美人兮逈出群,  輕風斜拂石榴裙,

花開金谷春三月,  月轉花陰夜十分;

玉雪精神聯仲琰,  瓊林才貌過文君,

少年情思應須慕,  莫使無心托白雲。」

下書「三泉主人醉筆。」

西門慶看了,便問:「三泉主人是王三官兒的號?」慌的鄭愛月兒連忙摭說道:「這還是他舊時寫下的。他如今不號三泉了,號小軒了。他告人說,學爹說:『我號四泉,他怎的號三泉?』他恐怕爹惱,因此改了號小軒。」一面走向前,取筆過來,把那「三」字就塗抹了。西門慶滿心歡喜,說道:「我並不知他改號一節。」粉頭道:「我聽見他對一個人說來,我纔曉的。他去世的父親號逸軒,他故此改號小軒。」說畢,鄭愛香兒往下邊去了,獨有愛月兒陪西門慶在房內,兩個並肩叠股,搶紅飲酒。因說起林太太來,怎的大量,好風月:「我在他家吃酒,那日王三官請我到後邊拜見。還是他主意,教三官拜認為我義父,教我受他禮,委託我指教他成日。」粉頭拍手大笑道:「還虧我指與這條路兒,到明日連三官兒娘子,不怕屬了爹!」西門慶道:「我到明日,我先燒與他一炷香;到正月里,請他和三官娘子往我家看燈吃酒。看他去不去?」粉頭道:「爹,你還不知三官娘子生的怎樣標致,就是個燈人兒,沒他那一段兒風流妖艷!今年十九歲兒,只在家中守寡。王三官兒通不着家。爹,你看用個工夫兒,愁不是你的人。」兩個說話之間,相挨相湊。只見丫鬟拿上幾樣細菓碟兒來,都是減碟菓仁,風菱鮮柑,螳郎雪梨 ,蘋婆 ,蚫螺,冰糖橙丁之類。粉頭親手奉與西門慶下酒。又用舌尖噙鳳香餅蜜送入他口中,又用纖手掀起西門慶藕合段〈衤旋〉子,看見他白綾褲子。西門慶一面解開褲帶,露出那話來教他弄。粉頭見根下束着銀托子,那話猙獰跳腦,紫漒光鮮。西門慶令他品之,這粉頭真個低垂粉頸,輕啟朱唇,半吞半吐,或進或出,嗚咂有聲,品弄了一回。靈犀已透,淫心似火,欲求講歡。粉頭便往後邊去了。西門慶出房更衣,見雪越下得甚緊。回到房中,丫鬟向前挂起錦慢,款設鴛枕,展放鮫綃,薰熱香球,牀上舖的被褥甚厚,打發脫靴解帶,先上牙牀。粉頭澡牝回來,俺上雙扉,共入危帳。正是

:「得多少春色嬌還媚, 惹蝶芳心軟欲濃。」

有詩為証:

「聚散無憑在夢中,  起來殘燭映紗紅;

鍾情自古多神念,  誰道陽臺路不通。」

兩個雲雨歡娛,到一更時分起來,丫鬟掌燈進房,整衣理鬢,後釃美酒,重整佳肴,又飲勾幾杯。問玳安:「有燈籠傘沒有?」玳安道:「琴童家去取燈籠傘來了。」這西門慶方纔作別了。鴇子、粉頭,相送出門,看着上馬。鄭月兒揚聲叫道:「爹若叫我,早些來說。」西門慶道:「我知道。」一面上馬,打着傘出院門。一路踏雪到家中,對着吳月娘只說在獅子街和吳二舅飲酒,不在話下。一宿晚景題過,到次日卻是初八日,打聽何千戶行李都搬過夏家房子內去了。西門慶這邊送了四盒細茶食,五錢折帕慶房賀儀過去。只見應伯爵驀地走來,西門慶見雪晴,天有風色甚冷,留他前邊書房中向火,叫小廝放卓兒,拿菜留他吃粥。因說起:「昨日喬親家、雲二哥禮并折帕都送過去了。你的人情,我這邊已是替你每家封了二錢出上了,你那里不消與他罷。只等發柬請吃酒。」那應伯爵舉手謝了。西門慶道:「何大人已搬過去了。今日我送茶并慶房人情,你不送些茶兒與他?」伯爵道:「他請人?」又問:「昨日安大人三位來做甚麼?那兩位是何人?」西門慶道:「那兩位一個雷兵備,一個汪參議,都是浙江人。因在我這里擺酒,明日要請杭州趙霆知府,新陞京堂大理寺丞,是他每本府父母官,如何不敬代一張卓面,餘者散席。戲子他那里叫來,俺這里少不的叫兩個小優兒答應便了。通身只三兩分資。」伯爵道:「大凡文職好細,三兩銀子勾做甚麼?哥少不得賠些兒。」西門慶道:「這雷兵備就是問黃四小舅子孫文相的。昨日沒曾對我題起,開除他罪名來了。」伯爵道:「你說他不仔細?如今還記着,折准擺這席酒纔罷了。」說話之間,伯爵叫應寶:「你叫那個人來見你大爹。」西門慶便問:「是何人?」伯爵道:「我那邊左近住一個小後生,倒也是舊人家出身,父母都沒了,自幼在王皇親家宅內答應好幾年了,也有了媳婦兒了。因在庄子上和一般家人不和,出來了。如今閒着,做不的甚麼買賣兒。他與應寶是朋友,央及應寶,要投尋個人家做房家人。今早應寶對我說:『爹倒好舉薦與大爹宅內答應,又伯大爹少人使。』我便說:『不知你爹用不用。』」因問應寶:「叫他甚麼名字?你叫他進來。」應寶道:「他姓來,叫來友兒。」只見那來友兒穿着青布四塊瓦布襪,靸鞋,扒在地上磕了個頭,起來簾外站立。伯爵道:「若論這狗拘的。膂力儘有,掇輕服重,都去的。」因問:「你多少年紀了?」那人道:「小的二十歲了。」又問:「你媳婦沒子女?」那人道:「只光兩口兒。」應寶道:「不瞞爹說,他媳婦纔十九歲兒。廚竈針線,大小衣裳,都會做。」西門慶見那人低頭並足,為人朴實,便道:「既是你應二爹來說,用心在我這里答應。」分付:「揀個好日期,寫紙文書,兩口兒搬進來罷。」那個磕了個頭,西門慶教琴童兒領着後邊見月娘眾人,磕頭去了。對月娘說:「就把來旺兒原住的那一間房,與他居住。」伯爵坐了回,家去了。應寶同他寫了一紙投身文書,交與西門慶收了,改名來爵,不在話下。卻說賁四娘子,自從他家長兒與了夏家,每日買東買西,只央及平安兒和來安、畫童兒,或是隔壁韓嫂兒的兒子小雨兒。西門慶家中這些大官兒,常在他屋裡坐的,打平和兒吃酒。賁四娘子兒和氣,就定出菜兒來。或要茶水,應手而至。就是賁四一時舖中歸來撞見 亦不見怪。以此今日他不在家,使着,那個不替他動?且玳安兒與平安兒,常帶他屋裡坐的多。初九日,西門慶與安郎中、汪參議、雷兵備擺酒,請趙知府。那日早辰,來爵兒兩口兒就搬進來。他媳婦兒後邊見月娘眾人磕頭。月娘見他穿着紫紬襖,青布披襖,綠布裙子。生的五短身材,瓜子面皮兒,搽胭抹粉,施朱唇,纏的兩隻腳趫趫的。問起來,諸般計指都會做。起了他個名字,叫做惠元,與惠秀、惠祥,一遞三日上竈不題。玳安與平安常在他屋裡坐的多。一日,門外楊姑娘沒了,安童兒來報喪。西門慶這邊整治了一張插卓,三牲湯飯,又封了五兩香儀。吳月娘、李嬌兒、孟玉樓、潘金蓮四頂轎子起身,都往北邊與他燒紙弔孝。琴童兒、棋童兒、來爵兒、來安兒四個,都跟轎子,不在家。西門慶在對過段舖子書房內,看着毛襖匠與月娘做貂鼠圍脖,先攢出一個圍脖兒,使玳安送與院中鄭月兒去。封了十兩銀子與他過節。鄭家管待玳安酒饌,與了他三錢銀子買瓜子兒磕,走來回西門慶話,說:「月姨多上覆,多謝了。前日空過了爹來。與了小的三錢銀子。」西門慶道:「你收了罷。」因問他:「賁四不在家,你頭里從他屋裡出萊,做甚麼來?」玳安道:「賁四娘子,從他女孩兒嫁了,沒人使。常央及小的每替他買買甚麼兒。」西門慶道:「他既沒人使,你每替他勤勤兒也罷。」又悄悄向玳安道:「你慢慢和他說,如此這般:『爹要來你這屋裡來看你看兒,你心如何?』看他怎麼的說。他若肯了,你問他討個汗巾兒來與我。」玳安道:「小的知道了。」領了西門慶言語,應諾下去。西門慶使陳經濟看着裁貂鼠,就走到家中來。只見王經向顧銀舖內,取了金赤虎,又是四對金頭銀簪兒,交與西門慶。門慶留下兩對在書房內,餘者袖進李瓶兒房內。坐下,與了如意兒那赤虎,又與他一對簪兒。把那一對簪兒,就與了迎春。二人接了,連忙插燭也似磕了頭。西門慶令迎春取飯去。須臾,拿了飯來。吃了飯,出來,在書房內坐下。只見玳安慢走到眼前,見王經在傍,不言語。西門慶使王經後邊取茶去。那玳安方說:「小的將爹言語對他說了,他笑了。約會晚上些,伺候等爹過去坐坐。叫小的拿了這汗巾兒來。」西門慶見紅綿紙兒包着一方紅綾織錦迥紋汗巾兒,聞了聞,噴鼻香,滿心歡喜,連忙袖了。只見王經拿茶來,吃了,又走過對門,看着匠人做生活去。忽報花大舅來了。西門慶道:「請過來這邊坐。」花子油走到書房暖閣兒裡,作揖坐下,致謝外日多有相擾。叙話間,畫童兒對門拿過茶來吃了。花子油悉把:「門外客人有五百包無錫米 ,凍了河,緊等要買賣了回家去。我想着姐夫倒好買下等價錢。」西門慶道:「我平白要他做甚麼?凍河還沒人要,到開河般來了,越發價錢跌了。如今家中也沒銀子。」即分付玳安:「收拾放卓兒,家中說看菜兒來。」一面使畫童兒:「請你應二爹來陪你花爹。」坐了一時,伯爵來到。三人共坐在一處,圍爐飲酒,卓上擺設四盤四碟,都是煎炒雞魚,燒爛下飯。又叫孫雪娥烙了兩炷餅,又是四碗肚肺乳線湯。良久,只見吳道官徒弟應春,送節禮疏誥來。西門慶請來同坐吃酒,攬李瓶兒百日經,與他銀子去。吃到日落時分,二人先起身去了。次後甘夥計收了舖子,又請來坐,與伯爵擲骰猜枚談話。不覺到掌燈已後,吳月娘眾人轎子到了,來安走來回話。伯爵道:「嫂子們今日都往那里去了?」西門慶道:「北邊他楊姑娘沒了。今日三日念經,我這里備了張插卓祭祀,又封了香儀兒,都去弔問弔兒。」伯爵道:「他老人家也高壽了。」西門慶道:「敢也有七十五六兒,男花女花都沒有,只靠他門外侄兒那里養活。材兒也是我這里替他備下的,這幾年了。」伯爵道:「好好兒,老人家有了黃金入櫃,就是一場事了,哥的大陰騭。」說畢,酒過數巡,伯爵與甘夥作辭去了。西門慶道:「十一日該姐夫這里上宿。」玳安道:「那邊舖子里,傅二叔也家去了,只小的一個在舖子里睡。」西門慶就起身走過來,分付後生王顯:「仔細火燭。」王顯道:「小的知道。」看著把門關上了。這西門慶見沒人,兩三步就走入賁四家來。只見賁四娘子兒,在門首獨自站立已久。見對門關的門響,西門慶從黑影中走至跟前。這婦人連忙把封門一開,西門慶鑽入裡面。婦人還扯上封門,說道:「爹請裡邊紙門內坐罷。」原來裡間槅扇廂着後半間,紙門內又有個小炕兒,籠着旺旺的火,卓上點着燈,兩邊護炕,從新糊的雪白,挂着四扇弔屏兒。那婦人頭上勒着翠藍銷金箍兒䯼髻,插着四根金簪兒,耳朵上兩個丁香兒。上穿紫紬襖,青綃絲披襖,玉色綃裙子。向前與西門慶道了萬福,連忙遞了一盞茶兒與西門慶吃。因悄悄說:「只怕隔壁韓嫂兒知道。」西門慶道:「不訪事,黑影子,他那里曉的。」于是不由分說,把婦人摟到懷中,就親嘴。拉近枕頭來,解衣按在炕沿子上,扛起腿來,就聳那話,上已束着托子。剛插入牝中,就拽了幾拽。婦人下邊淫水直流,把一條藍布褲子都濕了。西門慶拽出那話來,向順袋內取出包兒顫聲嬌來,蘸了些在龜頭上,攮進去,方纔澀住淫津,肆行抽拽。婦人雙手板着西門慶肩膊,兩相迎湊,在下颺聲顫語,呻吟不絕。這西門慶乘着酒興,架其兩腿在胳膊上,只顧沒稜露腦,銳進長驅,肆行搧磞,何止二三百度?須兒弄的婦人雲髻鬅鬆,舌尖冰冷,口不能言。西門慶則氣喘吁吁,靈龜暢美,一泄如注。良久拽出那話來,淫水隨出,用帕搽之。兩個整衣繫帶,復理殘粧。西門慶向袖中掏出五六兩一包碎銀子,又是兩對金頭簪兒,遞與婦人,節間買花翠帶。婦人拜謝了,悄悄打發出來。那邊玳安在舖子里,惠心只聽這邊門環兒響,便開大門,放西門慶進來,自知更無一人曉的。後次朝來暮往,也入港一、二次。正是:

「若要人不知,  除非己莫為。」

不想被韓嫂兒冷眼睃見,傳的後邊金蓮知道了。這金蓮亦不識破他。一日,臘月十五日,喬大戶家請吃酒。西門慶這里會同應伯爵、吳大舅一齊起身。那日有許多親朋做戲飲酒,至二更方散。第二日每家一張卓面,俱不必細說。單表崔本治了二千兩湖州紬絹貨物,臘月初旬起身,顧船裝載,趕至臨清馬頭,教後生榮海看守貨,便顧頭口來家取車稅銀兩。到門首下頭口,琴童道:「崔大哥來了,請廳上坐。爹在對門房子里,等我請去。」一面走到對門,不見西門慶。因問平安兒。平安兒道:「爹敢進後邊去了?」這琴童兒走到上房問月娘。月娘道:「賊見鬼的囚!你爹從早辰出去,再幾時進來!」又到各房裡并花園書房都瞧遍了,沒有。琴童在大門首揚聲道:「省恐殺人,不和爹往那里去了?白尋不着。大白日里把爹來不見了,崔大哥來了這一日,只顧教他坐他着。」那玳安分明知道,不做聲語,不想西門慶從前邊進來,把眾小廝乞了一驚。原來西門慶在賁四屋裡,入港纔出來。那平安打發西門慶進去了,望着琴童兒吐舌頭兒,都替他捏兩把汗,都道:「管情崔大哥去了,有幾下子打。」不想西門慶走到廳上,崔本見了,磕頭畢,交了書帳說:「船到馬頭,少車稅銀兩。我從臘月初一日起身,在楊州與他兩個分別,他每往杭州去了,俺每都到苗親家住了兩日。」因說:「苗青替老爹使了十兩個銀子,擡了楊州衛一個千戶家女,十六歲了,名喚楚雲。說不盡的花如臉,玉如肌,星如眼,月如眉,腰如柳,襪如鈎,兩隻腳兒兒恰剛三寸,端的有沉魚落雁之容,閉月羞花之貌。腹中有三千小曲、八百大曲、端的風流如水晶、盤內走明珠。態度似紅杏枝頭推曉日。苗青如今還養在家,替他打廂奩,治衣服,待開春,韓夥計、保官兒船上帶來,伏侍老爹,消愁解悶。」西門慶聽了,滿心歡喜。說道:「你船上稍了來也罷,又費煩他治甚衣服,打甚粧奩,愁我家沒有?」于是恨不的騰雲展翅,飛上楊州搬取嬌姿,賞心樂事。正是:

「鹿分鄭相應難辨,  蝶化莊周未可知。」

有詩為証:

「問道楊州一楚雲,  偶憑出鳥語來真;

不知好物都離隔,  試把梅子問夫人。」

西門慶陪崔本吃了飯,兌了五十兩銀子做車稅錢。又寫書與錢主事,令煩青目。言訖,當下作辭,往喬大戶家回話去了。平安見西門慶不尋琴童兒。都說:「我兒,你不知有多少造化。爹進來,若不是,綁着鬼有幾下打。」琴童笑道:「只你知爹性兒。」比及起了貨來,獅子街卸下,就是下旬時分。西門慶正在家打發送節禮,忽見荊都監差人拿帖來問:「宋大巡題本已上京數日,未知旨意下來不曾?伏惟老翁差人,察院衙門一打聽為妙。」這西門慶即差答應節級,拿着五錢銀子,往巡按公衙書辦打聽。果然昨日東京邸報下來,寫抄得一紙全報來,與西門慶觀看。上面道寫甚的:

「山東巡按監察御史宋喬年一本,循例舉劾地方文武官員,以勵人心,以隆聖治事:竊惟吏以撫民,武以禦亂;所以保障地方,以司民命者也。苟非其人,則處置乖方,民受其害,國何賴焉?此國家莫急於文武兩途,而激勸之典不容不亟舉也。臣奉命按臨山東等處,親歷省察風俗。至於吏政民瘼,監司守禦,無不留心咨訪。復令安撫大臣,詳加鑒別各官賢否,頗得其實。茲當差滿之期,敢不一一陳之。山東左布政陳四箴,操履忠貞,撫民有方;廉使趙訥,綱紀肅清,士民服習;提學副使陳正彙,操砥礪之行,嚴督率之條。又訪得兵備副使雷啟元,軍民咸服其恩威,僚慕悉推其練達;濟南府知府張徙叔夜,經濟可望,才堪司牧;東平府知府胡師文,居任清慎,視民如傷;徐州府知府韓邦奇,志務清修,才堪廊廟;蔡州府知府葉照,屏海寇而道不拾遺,惠民疇而懇田不涸。此數臣者,皆當薦獎而優擢者也。又訪得左參議馮廷鵠,傴僂之形,桑榆之景、形若木偶,尚肆貪婪;東昌府知府徐松,縱妾父而通賄,所致騰謗於公堂;慕羨餘而誅求,詈言聲輙遍於閭閻。此二臣者,所當亟賜罷斥者也。再訪得左軍院僉書守禦周秀,器宇恢弘,操持老練,得將帥之體,軍心允服,賊盜潛消,濟州兵馬都監荊忠,年力精強,才猶練達,冠武科而稱為儒將,勝算可以臨戎;號令而極其嚴明,長策雜能禦侮;袞州兵馬都監溫璽,夙閑韜略,熟習弓馬,休養騎卒以備不虞,供力設險以防不測。此三臣者,所當亟賜遷擢者也。清河縣千戶吳有德,以練達之才,得衛守之法。驅兵以擣中堅,靡攻不克;儲食以資糧餉,無人不飽。推心置腹,人思效命。實一方之保障,為國家之屏藩。宜特加超擢,鼓舞臣寮。階下誠以臣言可採,舉而行之,庶幾官爵不濫,而人心思奮,守牧得人而聖治有賴矣!等因。奉欽依該部知道。續該吏兵二部題前事,看得御史宋喬年所奏,內劾舉地方文武官員,無非體國之忠,出于公論。詢訪得實,以裨聖治之事。伏乞聖明俯賜施行,天下幸甚,生民幸甚。奉欽依依擬行。」

西門慶一見,滿心歡喜,拏着邸報走到後邊對月娘說:「宋道長本下來了,已是保舉你哥陞指揮僉事,見任管屯。周守禦與荊大人都有獎勵,轉副參統制之任。如今快使小廝請他來,對他說聲。」月娘道:「你使人請去,我交丫鬟看下酒菜兒。我愁他這一上任,也要銀子使。」西門慶道:「不打緊,我借與他幾兩銀子也罷了。」不一時,請得吳大舅到了。西門慶送那題奏旨意與他瞧。吳大舅連忙拜謝西門慶與月娘說道:「多累姐夫、姐姐扶持,恩當重報,不敢有忘。」西門慶道:「大舅,你若上任擺酒沒銀子使,我這里兌一千兩銀子,你那里使者。」那吳大舅又作揖謝了。于是就在月娘房中,安排上酒來吃酒。月娘也在旁邊陪坐。西門慶即令陳經濟把全抄寫了一本,與大舅拏着。即羞玳安拏帖,送邸報往荊都監、周守禦兩家報喜去。正是:

「勸君不費鐫研石,  路上行人口是碑。」

畢竟未知後來如何,且聽下回分解:

