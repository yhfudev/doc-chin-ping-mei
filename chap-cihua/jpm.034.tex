%# -*- coding: utf-8 -*-
%!TEX encoding = UTF-8 Unicode
%!TEX TS-program = xelatex
% vim:ts=4:sw=4
%
% 以上设定默认使用 XeLaTex 编译,并指定 Unicode 编码,供 TeXShop 自动识别

%第三十四回 
\chapter{書童兒因寵攬事\KG 平安兒含憤戳舌}

「自恃官豪放意為,  休將喜怒作公私,

貪財不顧綱常壞,  好色全忘義理虧;

狎客盜名求勢利,  狂奴乘飲弄奸欺,

欲佔後世興衰理,  今日施為可類知。」

話說韓道國走到門首打聽,見渾家和他兄弟韓二拴在舖中去了。急急走來獅子街舖子內,和來保計議。來保:「你還早央應二叔來,對當家的說了,拏個帖兒對縣中李老爹一說,不論多大事情都了了。」這韓道國竟到應伯爵家。他娘子兒使丫頭出來,回:「沒人在家,不知往那裡去了,只怕在西門大老爹家。」韓道國道:「沒在宅裡。」問應寶,也跟出去了。韓道國慌了,往抅攔院里抓尋。原來伯爵被湖州何蠻子的兄弟何二蠻子,號叫何兩峯,請在四條巷內,何金蟾兒家吃酒。被韓道國抓著了,請出來。伯爵吃的臉紅紅的,帽簷上插著剔牙杖兒。韓道國唱了喏,拉到僻靜處,如此這般告他說。伯爵道:「既有此事,我少不得陪你去。」于是作辭了何兩峯,與道國先生同到的,問了端的。道國央及道:「只望二叔往大官府宅裡說說,討個帖兒。只怕明早解縣上去,轉與李老爹案下,求青目一二,只不教你姪婦見官。事畢重謝二叔,磕頭就是了。」說著,跪在地下。伯爵用手拉起來,說道:「賢契,這些事兒,我不替你處?你取張布兒,寫了個說帖兒,我如今同你到大官府裡對他說,把一切閑話多丟開,你只說我常不在家,被街坊這夥光棍時常打磚掠瓦,欺負娘子。眾人稱:你兄弟韓二氣忿不過,和他嚷亂,反被這夥人群住,揪採在地亂行踢打,同拴在舖裏。望大官府討個帖兒,對李老爹說,只不教你令正出官,管情見個分上就是了。」那韓道國取筆硯,連忙寫了說帖,安放袖中。伯爵領他逕到西門慶門首,問守門的平安兒:「爹在家?」平安道:「爹在花園房裡,二爹和韓大叔請進去。」那應伯爵狗也不咬,走熟了的,同韓道國進入儀門,轉過大廳,由鹿頂鑽山進去,就是花園角門。抹過木香棚,兩邊松牆,松牆裏面三間小捲棚,名喚翡翠軒,乃西門慶夏月納涼之所。前後簾櫳掩映,四面花竹陰森,周圍擺設珍禽異獸,瑤草琪花,各極其盛。裏面一明兩暗書房,有畫童兒小廝在那里掃地,說:「應二爹和韓大叔來了!」二人掀開簾子,進入明間內。只見書童在書房裡,看見應二爹和韓大叔,便道:「請坐,俺爹剛纔進後邊去了。」一面使畫童兒請去。伯爵見上下放著六把雲南瑪瑙、漆減金釘籐絲甸矮矮東坡椅兒,兩邊掛四軸天青衢花綾裱白綾邊名人的山水,一邊一張螳螂蜻蜒腳,一封書大理石心璧畫的幫桌兒,桌兒上安放古銅爐流金仙鶴。正面懸著「翡翠軒」三字。左右粉箋弔屏上寫著一聯:「風靜槐陰清院宇,日長香篆散簾櫳。」伯爵于是正面椅上坐了,韓道國拉過一張椅子打橫。畫童後邊請西門慶去了。良久,伯爵走到裏邊書房內。裏面地平上安著一張大理石黑漆縷金涼床,掛著青紗帳幔。兩邊綵漆描金書厨,盛的都是送禮的書帕、尺頭,几席文具,書籍堆滿。綠紗窗下,安放一隻黑漆琴桌,獨獨放著一張螺蜔交椅。書篋內都是往來書柬拜帖,并送中秋禮物帳簿。應伯爵取過一本,揭開觀開,上面寫著:蔡老爺、蔡大爺、朱太尉、童太尉;中書蔡四老爹、都尉蔡五老爹,并本縣知縣、知府四宅;第二本是周守備、夏提刑、荊都監、張團練,并劉、薛二內相。都是金段尺頭、豬酒金餅、鰣魚海鮮 、雞鵝大禮,各有輕重不同。這裡二人等侯不題。且說畫童兒走到後邊金蓮房內,問:「春梅姐,爹在這裡?」春梅罵道:「賊見鬼,小奴才兒!爹在間壁六娘房裡不是?巴巴的跑來這里問。」畫童便走過這邊。只見綉春在石臺基上坐的,悄悄問:「爹在房裡?應二爹和韓大叔來了,在書房裡,請爹說話。」綉春道:「爹在房裡,看著娘與哥裁衣服哩!」原來西門慶拏出兩疋尺頭來,一疋大紅紵絲,一疋鸚哥綠潞紬,教李瓶兒替官哥裁毛衫兒、披襖、背心兒、護頂之類。在洒金炕上正舖著大紅氈條。奶子抱著哥兒在旁邊,迎春執著熨斗。只見綉春進來,悄悄拉迎春一把。迎春道:「你拉我怎麼的?拉撇了這火落在氈條上。」李瓶兒便問:「你平白拉他怎的?綉春道:「畫童說:應二爹來了,請爹說話。」李瓶兒道:「小奴才兒!應二爹來,你進來說就是了,巴巴的扯他!」西門慶分付畫童:「請二爹坐坐,我就來。」于是看裁完了衣服,便衣出來書房內見伯爵,二人作揖坐下。韓道國打橫。西門慶喚畫童取茶來。不一時,銀匙雕漆茶鍾,蜜餞金澄泡茶吃了,收了盞托去。伯爵就開言說道:「韓大哥,你有甚話?對你大官府說。」西門慶道:「你有甚話說來?」韓道國纔待說街坊有夥不知姓名棍徒,被應伯爵攔住,便道:「賢侄,你不是這等說了。噙著骨禿露著肉,也不是事。對著你家大官府在這里,越發打開後門說了罷。韓大哥常在舖子里上宿,家下沒人,止是他娘子兒一人,還有箇孩兒。左右街坊有幾個不三不四的人,見無人在家,時常打磚涼瓦鬼混,欺負的急了,他令弟韓二哥看不過,來家聲罵了幾句,被這起光棍不由分說,群住打了個臭死。如今都拴在舖裏,明早解往本縣正宅,往李大人那里去。見他哭哭啼啼,敬央煩我來對哥說討個帖兒,差人對李大人說說,青目一二,有了他令弟也是一般,只不要他令正出官就是了。」因說:「你把那說帖兒拏出來與你大官人瞧,好差人替你去。」韓道國便向袖中取去,連忙雙膝跪下,說道:「小人忝在老爹門下,萬乞老爹看應二叔分上,俯就一二,舉家沒齒難忘!」慌的西門慶一把手拉起,說道:「你請起來。」于是觀看帖兒,上面寫著:「犯婦王氏,乞青目免提。」西門慶道:「這帖子不是這等寫了,只有你令弟韓二一人就是了。」向伯爵道:「比時我拏帖對縣裏說,只分付地方改了報單,明日帶來我衙門里發落就是了。」伯爵教韓大哥:「你還與大老爹下個禮兒,這等亦發好了。」那韓道國又倒身磕頭下去。西門慶教玳安:「你外邊快叫個答應的班頭來。」不一時,叫了個穿青衣的節級來,在旁邊伺侯。西門慶叫近前分付:「你去牛皮街韓夥計住處,問是那牌那舖地方,對那保甲說,就稱是我的鈞語,分付把王氏即時與我放了,查出那幾個光棍名字來,改了報帖,明日早解提刑院我衙門里聽審。」那節級應諾,領了言語出門。伯爵道:「韓大哥,你即一同跟了他幹你的事去罷,我還和大官人說句話。」那韓道國千恩萬謝出門,與節級同往牛皮街分付去了。西門慶陪伯爵在翡翠軒坐下,因令玳安:「放桌兒。後邊對你大娘說,昨日磚廠劉公公送的木樨荷花酒 ,打開篩了來;我和應二叔吃,就把糟鰣魚 蒸了來。」伯爵舉手道:「我還沒謝的哥,昨日蒙哥送了那兩尾好鰣魚與我,送了一尾與家兄去;剩下一尾,對房下說拏刀兒劈開,送了一段與小女;餘者打成窄窄的塊兒,拏他原舊紅糟兒培著,再攪些香油 ,安放在一個磁罐內,留著我一早一晚吃飯兒。或遇有個人客兒來,蒸恁一碟兒上去,也不枉辜負了哥的盛情。」西門慶告訴:「劉太監的兄弟劉百戶,因在河下管蘆葦場,撰了幾兩銀子,新買了一所庄子,在五里店。拏皇木蓋房。近日被我衙門裡辦事,依著夏龍溪,饒了他一百兩銀子,還要動本參送,申行省院。劉太監慌了,親自拏著一百兩銀子到我這里,再三央及,只要事了。不瞞說,咱家做著些薄生意了,料著也過了日,那裡希罕他這樣錢!況劉太監平與我相交,時常受他些禮。今日因這些事情,就又薄了面皮,教我絲毫沒受他的!只教他相房屋邊連夜拆了。到衙門裡,只打了他家人劉三二十,就發落開了。事畢,劉太監感不過我這些情,宰了一口豬,送我一罈自造荷花酒,兩包糟鰣魚,重四十斤,又兩疋粧花織金段子,親自來謝,彼此有光,見箇情分,錢恁自中使。」伯爵道:「哥,你是希罕這個錢的!夏大人他出身行伍,起根立地上沒有,他不撾些兒,拏甚過日?哥,你自從到任以來,也和他問了幾樁事兒?」西門慶道:「大小也問了幾件公事,別的倒也罷了,只吃了他貪濫蹹婪的,有事不問青水皂白,得了錢在手裡就放了,成什麼道理!我便再三扭著不肯。你我雖是個武職官兒,掌著這刑條,還放些體面纔好。」說未了,酒菜齊至。先放了四碟菜菓,然後又放了四碟案酒,鮮紅鄧鄧的泰州鴨蛋 ,曲灣灣王瓜拌遼東金蝦 ,香噴噴油煠的燒骨禿 ,肥肥乾蒸的劈酒雞。第二道又是四碗嗄飯:一甌兒濾蒸的燒鴨 、一甌兒水晶膀蹄 、一甌兒白煠豬肉 、一甌兒炮炒的腰子 。落後纔是裏外青花白地磁盤,盛著一盤紅馥馥柳蒸的糟鰣魚 ,馨香美味,入口而化,骨刺皆香。西門慶將小金菊花盃斟荷花酒,陪伯爵吃。不說兩個說話兒,坐更餘方散。且說那夥人見青衣節級下地方,把婦人王氏放回家去,又拘總甲查了各人名字,明早解提刑院問理,都各人面面相覷,就知韓道國是西門慶家夥計,尋的本家攊子,只落下韓二一人在舖裏,都說這事弄的不好了。這韓道國又送了節級五錢銀子,登時問保甲查寫了那幾個名字,送到西門慶宅內,單等次日早解。過一日,西門慶與夏提刑兩位官到衙門裡坐廳。該地方保甲帶上人去,頭一起就是韓二跪在頭裡。夏提刑先看報單:牛皮街一牌四舖,總早蕭成,為地方喧鬧事。第一個就叫韓二,第二個車淡,第三個世寬,第四個游守,第五個郝賢,都叫過花名去。然後問韓二:「為什麼起來?」那韓二先告道:「小的哥是買賣人,常不在家去的。小男幼女,被街坊這幾個光棍,要便彈打胡博詞扠兒,坐在門首,胡歌野調,夜晚打磚,百般欺負。小的在外另住,來哥家看視。含忍不過,罵了幾句,被這夥群虎棍徙,不由分說,揪倒在地,亂行踢打,獲在老爺案下,望老爺查情。」夏提刑便問:「你怎麼說?」那夥人一齊告道:「老爺休信他巧對,他是耍錢的搗鬼,他哥不在家,和他嫂子王氏有姦。王氏平日倚逞刁潑,毀罵街坊,昨日被小的每捉住,見有底衣為證。」夏提刑因問保甲蕭成:「那王氏怎的不見?」蕭成怎的好回節級放了,只說:「王氏腳小,路上走不動,便來。」那韓二在下邊,兩隻眼只看著西門慶。良久,西門慶欠身望夏提刑道:「長官也不消要這王氏,想必王氏有些姿色,這光棍因調戲他不遂,捏成這個圈套。」因叫那為首的車淡上去,問道:「你在那里捉住那韓二來?」眾人道:「昨日在他屋裏捉來。」又問韓二:「王氏是你什麼人?」保甲道:「是他嫂子兒。」又問保甲:「這夥人打那里進他屋裡?」保甲道:「越牆進去。」西門慶大怒,罵道:「我把你這起光棍!他既是小叔,王氏也是有服之親,莫不不許上門行走?相你這起光棍,你是他什麼人?如何敢越牆進去?況他家男子不在,又有幼女在房中,非姦即盜了。」喝令左右:「拏夾棍來!」每人一夾,二十大棍。打的皮開肉綻,鮮血迸流。況四五個都是少年子弟,出娘胞胎未經刑杖,一個個打的號哭動天,呻吟滿地。這西門慶也不等夏提刑開口,分付:「韓二出去聽侯。把四個都與我收監,不日取供送問。」四人到監中,都互相抱怨,個個都懷鬼胎。監中人都嚇諕他:「你四個若送問,都是徒罪。到了外府州縣,皆是死數。」這些人慌了,等的家下人來送飯,稍信出去,教各人父兄使錢,上下尋人情。內中有拏人情央及夏提刑,說:「這王氏的丈夫,是你西門老爹門下的夥計。他在中間扭著要送問,同僚上,我又不好處得。你須還尋人情和他去,纔好出來。」也有央吳大舊出來的說。人都知西門慶家有錢,不敢來打點。四家父兄都慌了,會在一處。內中一箇說道:「也不消再央吳千戶,他也不依。我聞得人說,東街上住的開紬絹舖應大哥兄弟應二,和他契厚。咱不如每人拏出幾兩銀子,湊了幾十兩銀子,封與應二,教他過去替咱每說說,管情極好。」于是車淡的父兄,開酒店的車老兒為首,每人拏十兩銀子來,共湊了四十兩銀子,齊到應伯爵家,央他對西門慶說。伯爵收下,打發眾人去了。他娘子兒便說:「你既替韓夥計出力,擺布這些起人,如何又攬下這銀子,反替他說方便,不惹韓夥計怪?」伯爵道:「我可知不好說的。我如今如此這般,拏十五兩銀子去,悄悄進與他管書房的書童兒,教他取巧說這樁事。你不知他爹大小事兒,甚是托他,專信他說話,管情一箭就上垛。」于是把銀子兌了十五兩包放袖中,早到西門慶家,西門慶還未回來。伯爵進入廳上,只見書童正從書房內出來,頭帶瓦楞帽兒,札著玄色段子總角兒,撇著金頭蓮瓣簪子,身上穿著蘇州絹直裰,玉色紗〈衤旋〉兒,涼鞋淨襪,說道:「二爹請客位內坐。」交畫童兒後邊拏茶去,說道:「小廝,我使你拏茶與應二爹,你不動,且耍子兒。等爹來家,看我說不說!」那小廝就拏茶去了。伯爵便問:「你爹衙門里還沒來家?」書童道:「剛纔答應的來說,爹衙門散了,和夏老爹門外拜客去了。二爹有甚說話?」伯爵道:「沒甚話。」書童道:「二爹前日說的韓夥計那事,爹昨日到衙門裡,把那夥人都打了收監。明日做文書,還要送問他。」伯爵拉他到僻靜處,和他說:「如今又一件,那夥人家屬,如此這般,聽見要送問,多害怕了。

昨日晚夕到我家,哭哭啼啼,再二跪著央及我,教對你爹說。我想已是替韓夥計說在先,怎又好管他的,惹的韓夥計不怪?沒奈何,教他四家處了這十五兩銀子,看你巧取對你爹說,看怎麼,將就饒他放了罷。」因向袖中取出銀子來,遞與書童。書童打開看了,大小四錠零四塊,說道:「既是應二爹分上,交他再拏五兩來,待小的替他說,還不知爹肯不肯?昨日吳大舅親自來和爹說了,爹不依。小的虼【虫喿】臉兒,好大面皮兒!實對二爹說,小的這銀子,不獨自一個使,還破些鉛兒,轉達知俺生哥的六娘,遶個灣兒替他說,纔了他此事。」伯爵道:「既如此,等我和他說,你好歹替他上心些,他後晌些來討回話。」書童道:「爹不知多早來家,你教他明日早來罷。」說畢,伯爵去了。這書童把銀子拏到舖子,【釒劉】下一兩五錢來,教買了一罈金華酒 、兩隻燒鴨 、兩隻雞、一錢銀子鮮魚、一肘蹄子、二錢頂皮酥菓餡餅兒、一錢銀子的搽穰捲兒 。把下飯送到來興兒屋裏,央及他媳婦惠秀替他整理,安排端正。那一日,不想潘金蓮不在家,從早間坐轎子往門外潘姥姥家做生日了。書童使畫童兒用方盒把下飯先拏在李瓶兒房中,然後又提了一罈金華酒 進去。李瓶兒便問:「是那里的?」畫童道:「是書童哥送來孝順娘兒。」李瓶兒笑道:「賊囚!他怎的孝順我?」良久,書童兒進來,見李瓶在描金炕床上,舒著雪藕般玉腕兒,帶著鍍金鐲釧子,引著玳瑁貓兒和哥兒耍子。因說道:「賊囚!你送了這些東西來,與誰吃?」那書童只是笑。李瓶兒道:「你不言語,笑是怎的說?」書童道:「小的不孝順娘,再孝順誰?」李瓶兒道:「賊囚!你平日好好的,怎麼孝順我?是的,你不說明白,我也不吃。常言說的好:『君子不吃無名之食』。」那書童把酒打開,菜蔬都擺在小桌上,教迎春取了把銀素,篩了來,傾酒在鍾內,雙手遞上去,跪下說道:「娘吃過,等小的對娘說。」李瓶兒道:「你有甚事?說了我纔吃你的;不說,你就跪一百年,我也是不吃。」又道:「你起來說。」那書童于是把應伯爵所央四人之事,從頭訴說一遍:「他先替韓夥計說了,不好來說得。央及小的先來稟過娘。等爹問,休說是小的說,只假做花大舅那頭使人來說。小的寫下個帖兒,在前邊書房內,只說是娘遞與小的,教與爹看。娘屋裡再加一美言。況昨日衙門裡,爹已是打過他罪兒,爹胡亂做個處斷,放了他罷,也是老大的陰騭!」李瓶兒笑道:「原來也是這個事。不打緊,等你爹來家,我和他說就是了,你平白整治這些東西來做什麼?」又道:「賊囚!你想必問他起發些東西了?」書童道:「不瞞娘說,他送了小的五兩銀子。」李瓶兒道:「賊囚!你倒且是會排舖撰錢。」于是不吃小鍾,旋教迎春取了付大銀衢花盃來,先吃了兩鍾,然後也回斟一盃與書童吃。書童道:「小的不敢吃,吃了快臉紅,只怕爹來看見。」李瓶兒道:「我賞你吃,怕怎的?」于是磕了頭起來,一吸而飲之。李瓶兒把各樣嗄飯,揀在一個碟兒里,教他吃。那小廝一連陪他吃了兩大盃,怕臉紅,就不敢吃,就出來了。到了前邊舖子里,還剩了一半點心、嗄飯,擺在櫃上。又打了兩提罈酒,請了傅夥計、賁四、陳經濟、來興兒、玳安兒眾人,都一陣風捲殘雲,吃了箇淨光,就忘了教平安兒吃。那平安兒坐在大門首,把嘴谷都著。不想西門慶約後晌,從門外拜了客來家,平安看見也不說。那書童聽見喝道之聲,慌的收拾不迭。兩三步扠到廳上,與西門慶接衣服。西門慶便問:「今日沒人來?」書童道:「沒人。」西門慶脫了衣服,摘去冠帽,帶上巾幘,走到書房內坐下。書童兒取了一盞茶來遞上。西門慶呷了一口放下,因見他面帶紅色,便問:「你那里吃酒來?」這書童就向桌上硯臺下,取著一紙柬帖與西門慶瞧。說道:「此是後邊六娘叫小的到房裡,與小的這個柬帖,是花大舅那里送來說車淡等。那六娘教小的收著與爹瞧,因賞了小的一盞酒吃,不想臉就紅了。」西門慶把帖觀看,上寫道:「犯人車淡四名,乞青目。」看了遞與書童,分付:「放下我書篋內,教答應的明日衙門裡稟我。」書童一面接了,放在書篋內,又走在旁邊侍立。西門慶見他吃了酒,臉上透出紅白來,紅馥馥唇兒,露著一口糯更牙兒,如何不愛?于是淫心輒起,摟在懷裡,兩個親嘴砸舌頭。那小郎口噙香茶桂花餅 ,身上薰的噴鼻香,西門慶用手撩起他衣服,褪了花袴兒,摸弄他屁股,因囑付他:「少要吃酒,只怕糟了臉。」書童道:「爹分付,小的知道。」兩個在屋裡,正做一處。且說一個青衣人,騎了一匹馬,走到大門首,跳下馬來,問守門的平安,作揖問道:「這里是問刑的西門老爹家?」那平安兒因書童兒不請他吃東道,把嘴頭子撅著,正沒好氣,半日不答應。那人只顧立著,說道:「我是帥府周老爺差來,送轉帖與西門老爹看,明日與新平寨坐營須老爹送行。明日在永福寺擺酒,也有荊都監老爹、掌刑夏老爹、營里張老爹,每位分資一兩,剛纔多到了,逕來報知。累門上哥稟稟進去,小人還等回話。」那平安方拏了他的轉帖入後邊,打聽西門慶在花園童書房內。走到裡面,剛轉過松牆,只見畫童兒在窗外基臺上坐的,見了平安擺手兒。那平安就知西門慶與書童幹那不急的事。悄悄走在窗下聽覷半日,聽見裏邊氣呼呼,跐的地平一片聲響。西門慶叫道:「我的兒,把身子弔正著,休要動。」就半日沒聽見動靜。只見書童出來,與西門慶舀水洗手。看見平安兒、畫童兒在窗子下站立,把臉飛紅了,往後邊拏去了。平安拏轉帖進去。西門慶看了,取筆畫了知,分付:「後邊問你二娘討一兩銀子,教你姐夫封了付與他去。」平安兒應諾去了。書童拏了水來,西門慶洗畢手,回到李瓶兒房中,李瓶兒便問:「你吃酒?教丫頭篩酒你吃。」西門慶看見桌子底下,放著一罈金華酒 ,便問:「是那里的?李瓶兒不好說是書童兒買進來的,只說:「我一時要想些酒兒吃,旋使小廝街上買了這罈酒來,打開只吃了兩鍾兒,就懶待吃了。」西門慶道:「阿呀!前頭放著酒,你又拏銀子買!因前日買酒,我賒了丁蠻子的四十罈河清酒 ,丟在西廂房內,你要吃時,教小廝拏鑰匙取去。」說畢,李瓶兒還有頭里吃酒的,一碟燒鴨子、一碟雞肉、一碟鮮魚沒動,教迎春安排了四碟小菜,切了一碟火薰肉,放下桌兒在房中,陪西門慶吃酒。西門慶更不問這嗄飯是那里?可是平日家中受用,管待人家,這樣東西無日不吃。西門慶飲酒中間,想起問李瓶兒:「頭里書童拏的那帖兒,是你與他的?」李瓶兒道:「是門外花大舅那里來說,教你饒了那夥人罷。」西門慶道:「前日吳大舅來說,我沒依。若不是,我定要送問這起光棍。既是他那里分上,我明日到衙門里,每人打他一頓,放了罷。」李瓶兒道:「又打他怎的?打的那雌牙露嘴,什麼模樣!」西門慶道:「衙門是這等衙門,我管他雌牙不雌牙,還有比他嬌貴的。昨日衙門中,問了一起事,咱這縣中過世陳參政家,陳參政死了,母張氏守寡,有一小姐因正月十六日在門首看燈,有對門住的一箇小夥子兒名喚阮三,放花兒看,見那小姐生得標致,就生心調胡博詞,琵琶唱曲兒調戲他。那小姐聽了,邪心動。使梅香暗暗把這阮三叫到門裏,兩個只親了個嘴,後次竟不得會面。不期阮三在家,思想成病,病了五個月不起。父母那裡不使錢請醫看治?看看至死,不久身亡。有一朋友周二訂計說:『陳宅母子每年中元節令,在地藏寺薛姑子那里做伽藍會燒香。你許薛姑子十兩銀子,藏他在僧房內,與小姐相會,管病就要好了。』那阮三喜歡,果用其計。薛姑子受了十兩銀子,在方丈內,不期小姐午寢,遂與阮三苟合。那阮三剛病起來,久思色慾。一旦得了,遂死在女子身上。慌的他母親,忙領女子回家。這阮三父母怎肯干罷!一狀告到衙門裡,把薛姑子、陳家母子都拏了。依著夏龍溪,知陳家有錢,就要問在那女子身上。便是我不肯,說:『女子與阮三雖是私通,阮三久思不遂,況又病體不痊,一旦苟合,豈不傷命?』那薛姑子不合假以作佛事,窩藏男女通姦,因而致死人命,況又受贓,論了個知情,褪衣打二十板,責令還俗。其母張氏,不合引女入寺燒香,有壞風俗。同女每人一拶,二十敲,取了個供招,都釋放了。若不然,送到東平府,女子穩定償命。」李瓶兒道:「也是你老大個陰騭。你做這刑名官,早晚公門中與人行些方便兒。別的不打緊,只積你這點孩兒罷!」西門慶道:「可說什麼哩?」李瓶兒道:「別的罷了,只是難為那女孩兒。虧那小嫩指頭兒上,怎的禁受來?他不害疼?」西門慶道:「疼的兩個手,拶的順著指頭兒流血。」李瓶兒道:「你到明日,也要少拶打人,得將就些兒,那里不是積福處!」西門慶道:「公事可惜不的情兒。」這裡兩個正飲酒中間,只見春梅掀簾子進來,見西門慶正和李瓶兒腿壓著腿兒吃酒。說道:「你每自在吃的好酒兒,這咱晚就不想使個小廝,接待娘去?只有來安兒一個跟著轎子,隔門隔戶,只怕來晚了,你倒放心!」西門慶見他花冠不整,雲鬢篷鬆,便滿臉堆笑道:「小油嘴兒!我猜你睡來?」李瓶兒道:「你頭上挑線汗巾兒跳上去了,還不往下拉拉。」因讓他:「好甜金華酒 ,你吃鍾兒。」西門慶道:「你吃,我使小廝接你娘去。」那春梅一手挾着桌頭,且兜叉,因說道:「我纔睡起來,心裡惡拉拉,懶待吃。」西門慶道:「你看出來,小油嘴吃好少酒兒。」李瓶兒道:「左右今日你娘不在,你吃上一鍾兒,怕怎的?」春梅道:「六娘,你老人家自飲,我心里本不待吃。有俺娘在家不在家便恁的?就是娘在家,遇着我心不耐煩,他讓我,我也不吃。」西門慶道:「你不吃,呵口茶兒罷。我使迎春前頭叫個小廝,接你娘去。」因把手中吃的那盞木樨芝蔴薰笋泡茶遞與他。那春梅似有如無,接在手里,只呷了一口,就放下了。說道:「你教迎春叫去?我已叫了平安兒在這裡,他還大些,教他接去。西門慶隔窗就叫平安兒,那小廝應道:「小的在這里伺候。」西門慶道:「你去了,誰看大門?」平安道:「小的委付棋童兒在門上。」西門慶道:「既如此,你快拏個燈籠接去罷。」于是逕拏了燈籠,來迎接潘金蓮。迎到半路,只見來安兒跟着轎子從南來了。原來兩個是熟抬轎的,一個叫張川兒,一個叫魏聰兒。走向前,一把手接住轎扛子,說道:「小的來接娘來了。」金蓮就叫平安兒問道:「你爹在家?是你爹使你來接我?誰使你來?」平安道:「是爹使我來倒少倒少,是姐使了小的接娘來了。」金蓮道:「你爹想必衙門裡沒來家?」平安道:「沒來家?門外拜了人,從後晌就來家了,在六娘房裡吃的好酒兒。若不是姐旋叫了小的進去,催逼着拏燈籠來接娘,還早哩!小的見來安一個跟着轎子,又小,只怕來晚了,路上不方便,須得個大的兒來接纔好。又沒人看守大門,小的委付棋童兒在門首,小的纔來了。」金蓮又問:「你來時,你爹在那里?」平安道:「小的來時,爹還在六娘房里吃酒哩。姐稟問了爹,纔打發了小的來了。」金蓮聽了,在轎子內半日沒言語。冷笑罵道:「賊強人!把我只當亡故了的一般,一發在那淫婦屋裏睡了長覺也罷了!到明日,只交長遠倚逞那尿胞種,只休要晌午錯了!張川兒這里聽着,也沒別人。你腳千家門、萬家戶,那裡一箇纔尿出來多少時兒的孩子,拏整綾段尺頭裁衣裳與他穿。你家就是王十萬,使的使不的?」張川兒接過來道:「你老人家不說,小的也不敢說。這個可是使不的!不說可惜,倒只恐折了他。花麻痘疹還沒見,好容易就能養治的大?去年東門外一個高貴大庄屯人家,老兒六十歲,見居着祖父的前程,手里無碑記的銀子,可是說的牛馬成群,米糧無數,丫鬟侍妾,只成群立紀;穿袍兒的,身邊也有十七八個,要個兒子花看樣兒也沒有。東廟里打齋,西寺裡修供,捨經施像,那里沒求到?不想他第七個房里生了個兒子,喜歡的了不得,也像咱當家的一般,成日同掌兒上看擎,錦繡綾羅窩兒裡抱大,糊了五間雪洞兒的房,買了四五個養娘扶侍,成日見了風也怎的!那消三歲因出痘疹丟了。休怪小的說,倒是潑丟潑養的還好。」金蓮道:「潑丟潑養,恨不得成日金子兒裹着他哩!」平安道:「小的還有庄事對娘說。小的若不說,到明日娘打聽出來,又說小的不是了。便是韓夥計說的那夥人,爹衙門裡都夾打了,收在監裡,要送問他。今早應二爹來和書童兒說話,想必受了幾兩銀子,大包子 拏到舖子裡,就硬鑿了二三兩使了。買了許多東西嗄飯,在來興屋裡,教他媳婦子整治了,掇到六娘屋裡。又買了兩罈金華酒 ,先和六娘吃了。又走到前舖子裡,和傅二叔、賁四、姐夫、玳安、來興眾人打夥兒,直吃到爹來家時分,纔散了哩!」金蓮道:「他就不讓你吃些?」平安道:「他讓小的?好不大膽的蠻奴才!把娘每還不放到心上。不該小的說,還是爹慣了他。爹先不先和他在書房裡幹的齷齪營生。況他在縣里當門子,什麼事兒不知道!爹若不早把那蠻奴才打發了,到明日,咱這一家子乞他弄的壞了!」金蓮問道:「在李瓶兒屋里吃酒,吃的多大回?」平安兒道:「吃了好一日兒,小的看見他吃的臉通紅纔出來。」金蓮道:「你爹來家,就不說一句兒?」平安道:「爹也打牙粘住了,說什麼?」金蓮罵道:「恁賊沒廉耻的昏君強盜!賣了兒子招女婿,彼此騰倒着做!你便圖毛乍他那屎屁股門子,奴才左右{入日}你家愛娘子。」囑付平安:「等他再和那蠻奴才在那里幹這齷齪營生,你就來告我說。」平安道:「娘分付,小的知道。老川在這里聽着,也沒走了裡話;他在咱家也答應了這幾年,也是舊人。小的穿青衣,抱黑住,娘就是小的主兒。小的有話兒,怎不告娘說?娘只放在心裡,休要題出小的一字兒來。」于是跟着轎子,直說到家門首。潘金蓮下了轎,上穿着丁香色南京雲紬〈扌寨〉的五彩納紗,喜相逢天圓地方補子對衿衫兒,下着白碾光絹一尺寬攀枝耍娃娃桃線拖泥裙子;胸前〈扌寨〉帶金玲瓏〈扌寨〉領兒,下邊羊皮金荷包。先進到後邊月娘房裡,拜見月娘。月娘道:「你住一夜,慌的就來了?」金蓮道:「俺娘要留我住,他又招了俺姨那里一個十二歲的女孩兒在家養活,,都擠在一個炕上,誰住他?又恐怕隔門隔戶的,教我就來了。俺娘多多上覆姐姐,多謝重禮。」于是拜畢月娘,又到李嬌兒、孟玉樓眾人房裡,多拜了。回到前邊,打聽西門慶在李瓶兒屋裡吃酒,逕來拜李瓶兒。李瓶兒見他進來,連忙起身笑着抑接,兩個齊拜,說道:「姐姐來家早,請坐吃鍾酒兒。」教迎春:「快拏座兒與與你五娘坐。」金蓮道:「今日我偏了盃,重復吃了隻席兒,不坐了。」說着,揚長抽身就去了。西門慶道:「好奴才,恁大膽,來家就不拜我拜兒。那金蓮接過來道:「我拜你?還沒修福來哩!奴才不大膽,什麼人大膽?」看官聽說:潘金蓮這幾句話,分明譏諷李瓶兒,說他先和書童兒吃酒,然後又陪西門慶,豈不是雙席兒?那西門慶怎曉的就里?正是:

「情知這是針和線,  就地引起是非來。」

畢竟未知後來何如,且聽下回分解:


