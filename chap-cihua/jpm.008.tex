%# -*- coding: utf-8 -*-
%!TEX encoding = UTF-8 Unicode
%!TEX TS-program = xelatex
% vim:ts=4:sw=4
%
% 以上设定默认使用 XeLaTex 编译,并指定 Unicode 编码,供 TeXShop 自动识别

%第八回 
\chapter{潘金蓮永夜盼西門慶\KG 燒夫靈和尚聽淫聲}

\begin{showcontents}{}



「靜悄房櫳獨自猜,  鴛鴦失伴信音乖,

臂上粉香猶未泯,  床頭楸面暗塵埋;

芳容消瘦虛鸞鏡,  雲鬢蓬鬆墜玉釵,

駿驥不來勞望眼,  空餘鴛枕淚盈腮。」

話說西門慶自從娶了玉樓在家,燕爾新婚,如膠似漆。又遇着陳宅那邊使了文嫂兒來通信,六月十二日就要娶大姐過門。西門慶促忙促急,儹造不出床來,就把孟玉樓陪來的一張南京描金彩漆拔步床陪了大姐。三朝九日,足亂了約一個月多,不曾往潘金蓮家去。把那婦人每日門兒倚遍,眼兒望穿,使王婆往他門首去了兩遍。門首小廝常見王婆,知道是潘金蓮使來的,多不理他。只說:「大官人不得閑哩。」婦人盼他急的緊,只見婆子回了婦人,婦人又打罵小女兒街上去尋覓。那小妮子怎敢入他那深宅大院裡去,只在門首踅探了一兩遍,不見西門慶,就回來了。來家又被婦人噦罵在臉上,打在臉上,怪他沒用,便要教他跪着;餓到晌午,又不與他飯吃。那時正值三伏天道,十分炎熱。婦人在房中害熱,吩咐迎兒熱下水,伺候澡盆,要洗澡。又做了一籠夸餡肉魚兒,等西門慶來吃。身上只着薄纊短衫,坐在小杌上。盼不見西門慶來到,嘴谷都的罵了幾句負心賊,無情無緒,悶悶不語。用纖手向腳上脫下兩隻紅綉兒來,試打一個相思卦,看西門慶來不來。正是:

「逢人不敢高聲語,  暗卜金錢問遠人。」

有山坡羊為證:

「凌波羅襪,天然生下,紅雲染就相思卦;似耦生芽,如蓮御花,怎生纏得些娘大!柳條兒比來剛半扠。他不念咱,咱想念他!想著門兒,私下簾兒,悄呀,空教奴被兒哩,叫著他那名兒罵。你怎戀烟花,不來我家!奴眉兒淡淡教誰晝?何處緣楊拴繫馬?他辜負咱,咱念戀他。」

當下婦人打了一回相思卦,見西門慶不來了,不覺困倦來,就歪在床上盹睡着了。約一個時辰醒來,心中正沒好氣。迎兒問:「熱了水,娘洗澡也不洗?」婦人便問:「角兒 蒸熱了?拏來我看。」迎兒連忙拏到房中。婦人用纖手一數,原做下一扇籠,三十個角兒,翻來覆去,只數了二十九個,少了一個角兒。便問:「往那裡去了?」迎兒道:「我並沒有看見,只怕娘錯數了。」婦人道:「我親數了兩遍,三十個角兒,要等你爹來吃,你如何偷吃了一個?好嬌態淫婦奴才!你害饞癆饞痞,心裡要想這個角兒吃!你大碗小碗〈口床〉搗不下飯去,我做下的孝順你來!」于是不由分說,把這小妮子跣剝去了身上衣服,拏馬鞭子下手打了二三十下,打的妮子殺豬也似叫。問着他:「你不承認?我定打下百數。」打的妮子急了,說道:「娘休打,是我害餓的慌,偷吃了一個。」婦人道:「你偷了,如何賴我錯數了?眼看着就是個牢頭禍根淫婦!有那亡八在時,輕學重告;今日往那裡去了,還在我跟前弄神弄鬼!我只把你這牢頭淫婦,打下你下截來!」打了一回,穿上小衣,放起他來,吩咐在旁打扇。打了一回扇,口中說道:「賊淫婦,你舒過臉來,等我搯你這皮臉兩下子。」那迎兒真個舒着臉,被婦人尖指甲搯了兩道血口子,纔饒了他。良久,走到鏡臺前,從新粧點,出來門簾下站立。也是天假其便,只見西門慶家小廝玳安,夾着毡包,騎着馬,打婦人門首過的。婦人叫住他:「往何處去來?」那小廝平日說話乖覺,常跟西門慶在婦人家行走,婦人嘗與他浸潤,他有甚不是,在西門慶面前,替他說方便,以此婦人往來就滑。一面下馬來,說道:「俺爹使我送人情,往守備府裡去來。」婦人叫進來問他:「你爹家中有甚事?如何一向不來傍個影兒看我一看?想必另續上了一個心甜的姐妹,把我做個網巾圈兒,打靠後了。」玳安道:「俺爹再沒續上姐妹,只是這幾日家中事忙,不得脫身來看得六姨。」婦人道:「就是家中有事,那裡丟我恁個半月,音信不送一個兒!只是不放在心兒上。」因問玳安:「有甚麼事?你對我說。」那小廝嘻嘻只是笑,不肯說。「有樁事兒罷了,六姨只顧吹毛求問怎的?」婦人道:「好小油嘴兒!你不對我說,我就惱你一生!」小廝道:「我對六姨說,六姨休對爹說是我說的。」婦人道:「我不對他說便了。」玳安如此這般,把家中娶孟玉樓之事,從頭至尾,告訴了一遍。這婦人不聽便罷,聽了由不的那裡眼中淚珠兒,順着香腮流將下來。玳安慌了,便道:「六姨,你原來這等量窄,我故便不對你說;對你說,便就如此!」婦人倚定門兒,長歎了一口氣說道:「玳安,你不知道,我與他從前已往那樣恩情,今日如何一旦拋閃了!」止不住紛紛落下淚來。玳安道:「六姨,你何苦如此?家中俺娘也不管着他。」婦人便道:「玳安,你聽告訴。」另有前腔為證:

「喬才心邪,不來一月,奴綉鴛衾曠了三十夜;他俏心兒別,俺痴心兒呆,不合將人十分熱。常言道:容易得來,容易捨。與過也!緣分也!」

說畢,又哭了。玳安道:「六姨,你休哭,俺爹怕不的也只在這兩日頭,他生日待來也。你寫幾個字兒,等我替你稍去,與俺爹瞧看了,必然就來。」婦人道:「是必累你請的他來,到明日我做雙好鞋與你穿;我這裡也要等他來,與他上壽哩!他若不來,都在你小油嘴身上。他若是問起你來這裡做什麼,你怎生回答他?」玳安道:「爹若問小的,只說在街上飲馬,六姨使王奶奶叫了我去,稍了這個柬帖兒,多上覆爹,好歹請爹過去哩。」婦人笑道:「你這小油嘴!到是再來的紅娘,倒會成合事兒哩!」說畢,令迎兒把桌上蒸下的角兒 裝了一碟兒,打發玳安兒吃茶。一面走入房中,取過一幅花箋,又輕拈玉管,款弄羊毛,須臾,寫了一首寄生草,詞曰:

「將奴這知心話,付花箋,寄與他;想當初結下青絲髮,門兒倚遍簾兒下,受了些沒打弄的,躭驚怕;你今果是負了奴心,不來,還我香羅帕!」

寫就,疊成一個方勝兒,封停當,付與玳安兒收了:「好歹多上覆他,待他生日,千萬走走,奴這裡來專望。」那玳安吃了點心,婦人又與數十文錢。臨出門上馬,婦人道:「你到家見你爹,就說六姨好不罵你,他若不來,你就說六姨到明日,坐轎子親自來哩。」玳安道:「六姨,自吃你賣糞團的,撞見了敲板兒蠻子,叫冤屈麻飯肐膽的帳!騎着木驢兒,磕瓜子兒,瑣碎昏昏。」說畢,騎上馬去了。那婦人每日長等短等,如石沉大海一般,那裡得個西門慶影兒來。看看七月將盡,到了他生辰,這婦人挨一日似三秋,盼一夜如半夏,等了一日,杳無音信;盼了多時,寂無形影。不覺銀牙暗咬,星眼流波。至晚,旋叫王婆來,安排酒肉,與他吃了。向頭上拔下一根金頭銀簪子與他,央往西門慶家走走,去請他來。王婆道:「咱晚來茶前酒後,他定也不來。待老身明日侵早,往大官宅上,請他去罷。」婦人道:「乾娘是必記心,休要忘了。」婆子道:「老身管着那一門兒來,肯誤了勾當!」當下這婆子非錢而不行,得了這根簪子,吃得臉紅紅,歸家去了。原來婦人在房中,香薰鴛被,款剔銀燈,睡不着,短歎長吁,翻來覆去。正是:

「得多少琵琶夜久殷勤弄,  寂寞空房不忍彈。」

于是獨自彈着琵琶,唱一個綿搭絮為證:

「當初奴愛你風流,共你剪髮燃香,雨態雲踪兩意投,背親夫和你情偷。怕甚麼傍人講論,覆水難收;你若負了奴真情,正是緣木求魚空自守!」

又

「誰想你另有了裙釵,氣的奴似醉如痴,斜傍定幃屏,故意兒猜。不明白,怎生丟開!傳書寄柬,你又不來。你若負了奴的恩情,人不為仇天降災!」

又

「奴家又不曾愛你錢財,只愛你可意的冤家,知重知輕性兒乖。奴本是朵好花兒園內初開,蝴蝶餐破,再也不來。我和你那樣的恩情,前世裡前緣今世裡該!」

又

「心中猶豫,展轉成憂。常言婦女痴心,惟有情人意不周。是我迎頭和你把情偷,鮮花付與,怎肯干休?你如今另有知心,海神廟裡和你把狀投!」

原來婦人一夜翻來覆去,不曾睡着。到天明,使迎兒:「過間壁瞧那王奶奶,請你爹去了不曾?」迎兒去了不多時,說:「王奶奶老早就出去了。」且說那婆子,早晨梳洗出門來,到西門慶門首,問門上:「大官人在家?」都說不知道。在對門牆腳下,等不勾多時,只見傅夥計來開舖子,婆子走向前來,道了萬福。「動問一聲,大官人在家麼?」傅夥計道:「你老人家尋他怎的?這早來問着我,第二個人也不知他。」說:「大官人昨日壽日,在家請客吃酒,吃了一日酒,到晚拉眾朋友往院裡去了,一夜通沒來家。你往那裡尋他去?」這婆子拜辭出縣前,來到東街口,正往构欄那條巷去。只見西門慶騎馬遠遠從東來,兩個小廝跟隨,吃的醉眼摩娑,前合後仰。被婆子高聲叫道:「大官人,少吃些兒怎的。」向前一把手,把馬嚼環扯住。西門慶醉中問道:「你是王乾娘?你來有甚話說?」那婆子向他耳畔低言。道不數句,西門慶道:「小廝來家對我說來,我知道六姐惱我哩,我如今就去。」那西門慶一面跟着他,兩個一遞一句,整說了一路話。比及時到婦人門首,婆子先入去報道:「大娘子!且喜還虧老身去了,沒半個時辰,把大官人請得來了!」婦人聽見他來,連忙叫迎兒收拾房中乾淨,一面出房來迎接。西門慶搖着扇兒進來,帶酒半酣;進入房來,與婦人唱喏。婦人還了萬福,說道:「大官人,貴人稀見面,怎的把奴來丟了,一向不來傍個影子?家中新娘子陪伴,如膠似漆,那裡想起奴家來!還說大官人不變心哩。」西門慶道:「你休聽人胡說,那討甚麼新娘子來?只因小女出嫁,忙了幾日,不曾得閑工夫來看你,就是這般話。」婦人道:「你還哄我哩!你若不是憐新棄舊,再不外邊另有別人,你指着旺跳身子說個誓,我方信你。」那西門慶道:「我若負了你情意,生碗來大疔瘡,害三五年黃病,扁擔大蛆虫冓口袋!」婦人道:「賊負心的!扁擔大蛆虫冓口袋,管你甚事!」一手向他頭上把帽兒撮下來,望地下只一丟。慌的王婆地下拾起來,見一頂新纓子瓦楞帽兒,替他放在桌上。說道:「大娘子,只怪老身不去請大官人來,就是這般的!還不與帶上着,試了風。」婦人道:「那怕負心強人陰寒死了,奴也不疼他!」一面向他頭上拔下一根簪兒,拏在手裡觀看,都是一點油金簪兒,上面鈒着兩溜子字兒:「金勒馬嘶芳草地,玉樓人醉杏花天。」卻是孟玉樓帶來的。婦人猜做那個唱的與他的,奪了放在袖子裡不與他,說道:「你還不變心哩!奴與你的簪兒那裡去了?都帶着那個的這根簪子?」西門慶道:「你那根簪子,前日因吃酒醉了,跌下馬來,把帽子落了,頭髮散開;尋時就不見了。」婦人道:「你哄三歲小孩兒也不信;哥哥兒,你醉的眼花恁樣了,簪子落地下,就看不見?」王婆在傍插口道:「大娘子,你休怪大官人,他離城四十里,見蜜蜂兒拉屎,出門交獺象拌了一交,原來覷遠不覷近。」西門慶道:「緊自他麻犯人,你又自作耍!」婦人因見手中擎着一根紅骨細洒金金釘鉸川扇兒,取過來迎亮處只一照,原來婦人久慣知風月中事,見扇兒多是牙咬的碎眼兒,就是那個妙人與他的扇子。不由分說,兩把折了。西門慶救時,已是扯的爛了,說道:「這扇子是我一個朋友卜志道送我的,今日纔拏了三日,被你扯爛了。」那婦人奚落了他一回,只見迎兒拿茶來,叫迎兒放下茶托,與西門慶磕頭。王婆道:「你兩口子聐聒了這半日,也勾了,休要誤了勾當,老身廚下收拾去也。」婦人一面吩咐迎兒房中放桌兒,預先安排下與西門慶上壽的酒肴,無非是燒雞熟鵝鮮魚肉酢菓品之類。須臾,安排停當,拏到房中,擺在桌上。婦人向箱中取出與西門慶做下上壽的物事,用盤托盛着,擺在面前,與西門慶觀看。一隻玄色段子鞋,一雙挑線密約深盟隨君膝下,香草邊闌松竹梅花,歲寒三友,醬色段子護膝,一條紗綠潞紬,永祥雲嵌八寶,水光絹裡兒,紫線帶兒,裡面裝着排草梅桂花兜肚。一根並頭蓮辨簪兒,簪兒上鈒着五言四句詩一首云:「奴有並頭蓮,贈與君關髻;凡事同頭上,切勿輕相棄。」西門慶一見,滿心歡喜,把婦人一手摟過,親了個嘴,說道:「那知你有如此一段聰慧,少有!」婦人教迎兒執壺,斟一盃與西門慶,花枝招颺,插燭也似磕了四個頭。那西門慶連忙拖起來,兩個並肩而坐,交杯換盞飲酒。那王婆陪着吃了幾杯酒,吃的臉紅紅的,告辭回家了。二人自在取樂頑耍,迎兒打發王婆出去,關上大門,廚下坐的。婦人陪伴西門慶飲酒多時,看看天色晚來,但見:

「密雲迷晚岫,暗霧鎖長空;群星與皓月爭輝,綠水共青天映碧。僧投古寺,深林中嚷嚷鴉飛;客奔荒村,閭巷內汪汪犬吠。枝上子規啼夜月,園中粉蝶戲花來。」

當下西門慶吩咐小廝回馬家去,就在婦人家歇了。到晚夕二人如顛狂鷂子相似,儘力盤桓,淫慾無度。常言道:「樂極生悲,泰極否來。」光陰迅速,單表武松自從領了知縣書禮,離了清河縣,送禮物馱擔,到東京朱太尉處,下了書禮,交割了箱馱,街上各處閑行了幾日,討了回書,領一行人,取路回山東大路而來。去時三四月天氣,回來都淡暑新秋,路上水雨連綿,遲了日限,前後往回,也有三個月光景。在路上雨水所阻,只覺得神思不安,身心恍惚,趕回要看哥哥,不免差了一個士兵,預先報與知縣相公。又私自寄了一封家書,與他哥哥武大,說他也不久,只在八月內回還。那士兵先下了知縣相公稟帖,然後逕奔來抓尋武大家。可可天假其便,王婆正在門首。那士兵見武大家關着,纔要叫門,婆子便問:「你是尋誰的?」士兵道:「我是武都頭差來,下書與他哥哥。」婆子道:「武大郎不在家,都上墳去了。你有書信,交與我就是了,等他歸來,我遞與他也是一般。」那士兵向前唱了一個喏,便向身邊取出家書來,交與王婆,忙忙促促騎上頭口,飛的一般去了。這王婆拏着那封書,從後門走過婦人家來。迎兒開了門,婆子入來,原來婦人和西門慶狂了半夜,約睡至飯時,還不起來。王婆叫道:「大官人娘子起來,匆匆有句話和你們說。如今如此如此,這般這般,武二差士兵寄了書來,他與哥哥說他不久就到,我接下幾句話兒,打發他去了。你們不可遲滯,早處長便。」那西門慶不聽萬事皆休,聽了此言,正是:

「分門八塊頂梁骨,  傾下半桶冰雪來。」

一面與婦人多起來,穿上衣服,請王婆到房內坐了,取出書來與西門慶看了。武松書中寫着,不過中秋回家,二人都慌了手腳,說道:「如此怎了?乾娘遮藏我每則個,恩有重報,不敢有忘!我如今與大姐情深意海,不能相捨;武二那廝回來,便要分散,如何是好?」婆子道:「大官人,有什麼難處之事!我前日已說過了,幼嫁由爹娘,後嫁由自己,古來叔嫂不通門戶;如今已自大郎百日來到,大娘子請上幾位眾僧,來把這靈牌燒了,趁武二未到家來,大官人一頂轎子,娶了家去。等武二那廝回來,我自有話說,他敢怎的?自此你二人自在一生,無些鳥事。」西門慶便道:「乾娘說的是。」正是:

「人無剛骨,  安身不牢。」

當日西門慶和婦人用畢早飯,約定八月初六日,是武大郎百日,請僧念佛燒靈;初八日晚,擡娶婦人家去,三人計議已定。不一時,玳安拏馬來接回家,不在話下。光陰似箭,日月如梭,又早到八月初六日。西門慶拏了數兩散碎銀錢、二斗白米、齋襯,來婦人家。教王婆報恩寺請了六個僧,在家做水陸超度武大,并天晚夕除靈。道人頭五更就挑了經擔來,舖陳道場,懸掛佛像。王婆伴廚子在灶上安排整理齋供。西門慶那日就在婦人家歇了。不一時,和尚來到,搖響靈杵,打動鼓鈸,宣揚諷誦,咒演法華經,禮拜梁王懺,早辰發牃,請降三寶,證盟功德,請佛獻供午刻召亡施食,不必細說。且說潘金蓮怎肯齋戒,陪伴西門慶睡到日頭半天,還不起來。和尚請齋主拈香僉字,證盟禮佛,婦人方纔起梳洗,喬素打扮,來到佛前參拜。那眾和尚見了武大這個老婆,一個個都昏迷了佛性禪心,一個個多關不住心猿意馬,都七顛八倒,酥成一塊。但見:

「班首輕狂,念佛號不知顛倒,維摩昏亂,誦經言豈顧高低。燒香行者,推倒花瓶,秉燭頭陀,錯拏香盒。宣盟表白,大宋國稱做大唐;懺罪闍黎,武大郎念為大武。長老心忙,打鼓錯拏徒弟手;沙彌心蕩,磬搥打破老僧頭。從前苦行一時休,萬個金剛降不住。」

那婦人佛前燒了香,僉了字,拜禮佛畢,回房去了。依舊陪伴西門慶做一處,擺上酒席葷腥來,自去取樂。西門慶吩咐王婆:「有事你自答應便了,休教他來聒噪六姐。」婆子哈哈笑道:「大官人你到放心,由着老娘和那禿廝纏。你兩口兒,是會受用!」看官聽說:世上有德行的高僧,坐懷不亂的少。古人有云:「一個字便是『僧』,二個字便是『和尚』,三個字是個『鬼樂官』,四個字是『色中餓鬼』。」蘇東坡又云:「不禿不毒,不毒不禿;轉毒轉禿,轉禿轉毒。」此一篇議論,專說這為僧戒行,住着這高堂大廈,佛殿僧房,吃着那十方檀越錢糧,又不耕種,一日三餐。又無甚事縈心,只專在這色慾上留心。譬如在家俗人,或士農工商,富貴長者,小相俱全,每被利名所絆;或人事往來,雖有美妻少妾在旁,忽想起一件事來關心,或探探甕中無米,囤內少柴,早把興來沒了,都輸與這和尚每許多。有詩為證:

「色中餓鬼獸中狨,  壞教貪淫玷祖風;

此物只宜林下看,  不堪引入畫堂中。」

當時這眾和尚見了武大這個老婆喬模喬樣,多記在心裡。到午齋往寺中歇晌回來,婦人正和西門慶在房裡飲酒作歡。原來婦人臥房,正在佛堂一處,止隔一道板壁;有一個僧人先到,走在婦人窗下水盆裡洗手,忽然聽見婦人在房裡,顫聲柔氣,呻呻吟吟,哼哼唧唧,恰似有人在房裡交姤一般。于是推洗手,立住了腳,聽勾良久。只聽婦人口裡嗽聲呼叫西門慶:「達達,你休只顧〈扌扉〉打到幾時,只怕和尚來聽見,饒了奴,快些丟了罷!」西門慶道:「你且休慌!我還要在蓋子上燒一下兒哩!」不想都被這禿廝聽了個不亦樂乎。落後眾和尚都到齊了,吹打起法事來,一個傳一個,都知道婦人有漢子在屋裡,不覺都手之舞之,足之蹈之。臨佛事完滿,晚夕送靈化財出去,婦人又早除了孝髻,換了一身豔衣服,在簾裡與西門慶兩個並肩而立,看着和尚化燒靈座。王婆舀將水,點一把火來,登時把靈牌并佛燒了。那賊禿冷眼瞧見簾子裡,一個漢子和婆娘影影綽綽,並肩站立,想起白日裡,聽見那些勾當,只個亂打鼓〈扌扉〉鈸不住。被風把長老的僧伽帽刮在地上,露見青旋旋光頭,不去拾,只顧〈扌扉〉鈸打鼓,笑成一塊。王婆便叫道:「師父布馬也燒過了,還只個〈扌扉〉打怎的?」和尚答道:「還有紙爐蓋子上沒燒過。」西門慶聽見,一面令王婆快打發襯錢與他。長老道:「請齋主娘子,謝謝!」婦人道:「王婆說免了罷!」眾和尚道:「不如饒了罷。」一齊笑的去了。正是:

「遺踪堪入時人眼,  不買胭脂畫牡丹。」

有詩為證:

「淫婦燒靈志不平,  和尚竊壁聽淫聲;

果然佛道能消罪,  亡者聞之亦慘魂。」

畢竟未知後來如何,且聽下回分解:



\end{showcontents}
