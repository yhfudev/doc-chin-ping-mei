%# -*- coding: utf-8 -*-
%!TEX encoding = UTF-8 Unicode
%!TEX TS-program = xelatex
% vim:ts=4:sw=4
%
% 以上设定默认使用 XeLaTex 编译,并指定 Unicode 编码,供 TeXShop 自动识别

%第九十一回 
\chapter{孟玉樓愛嫁李衙內\KG 李衙內怒打玉簪兒}


\begin{showcontents}{}



「百歲光陰疾似飛,  其間花景不多時,

秋凝白露寒蛩泣,  春老黃昏杜宇啼;

富貴繁華身上孽,  功明事跡目中魑,

一場春夢由人做,  自有青天報不欺。」

話說一日,陳經濟聽見薛嫂兒說,西門慶家孫雪娥,被來旺因姦抵盜財物,拐出在外,事發,本縣官賣,被守備府里買了,朝夕受春梅打罵。這陳經濟乘着這個因由,使薛嫂兒往西門慶家對月娘說,只是經濟風里言風里話,在外聲言發語,說不要大姐,寫了狀子,巡撫,巡按處,要告月娘;說西門慶在日,收着他父親寄放許多金銀箱籠細軟之物。這月娘一來因孫雪娥被來旺兒盜財拐去,二者又是來安兒小廝走了,三者家人來興媳婦惠秀家又死了,剛打發出去,家中正七事八事。聽見薛嫂兒來說此話,諕的慌了手腳。連忙顧轎子,打發大姐家去。但見大姐床奩箱廚陪嫁之物,交玳安顧人都擡送到陳經濟家。經濟說:「這是他隨身嫁,我的床帳粧奩,還有我家寄放的細軟金銀箱籠,須索還我。」薛嫂道:「你大丈母說來,當初丈人在時,止收下這個床奩嫁粧,並沒見你的別的箱籠。」經濟又要使女元宵兒,薛嫂兒和玳安兒來對月娘說,月娘道不肯把元宵兒與他,說:「這丫頭是李嬌兒房中使的,如今沒人看哥兒,留着早晚看哥兒哩。」把中秋兒打發將來,說:「原是買了扶侍大姐的。」這經濟又不要中秋兒,兩頭回來,只交薛嫂兒走。他娘張氏,便向玳安說:「哥哥,你到家頂上你大娘,你家姐兒們多,豈可希罕這個使女看守,既是與了大姐里好一向,你姐夫已是收用過他了,你大娘只顧留怎的?」玳安一回到家,把此話對月娘說了。月娘無言可對,只得把元宵兒打發將來。經濟這里收下,滿心歡喜,說道:「可怎的也打我這條道兒來!」正是:

「饒你奸似鬼,  也吃我洗腳水!」

按下一頭,都來一處。單說李知縣兒子李衙內,自從清明郊外那日,在杏花庄酒樓,看見月娘、孟玉樓,兩口一般打扮,生的俱有姿色,使小張閒打聽,回報俱是西門慶妻小。衙內有心愛孟玉樓,見生的長挑身材,瓜子回皮,面上稀稀有幾點白麻子兒,模樣兒風流俏麗。原來衙內喪偶,鰥居已久,一向着媒婦,各處求親,多不遂意。及有玉樓,終有懷心,無門可入,未知嫁與不嫁,從違如何。不期雪娥緣事在官,已知是西門慶家出來的。周旋委曲,在伊父案前,將各犯用刑研審,追出贓物數目,稽其來領。月娘害怕,又不使人見官,衙內失望,因此纔將贓物入官,雪娥官賣。至是衙內謀之于廊吏何不違,徑使官媒婆陶媽媽,來西門慶家訪求親事。許說成此門親事,免縣中打卯,還賞銀五兩。這陶媽媽聽了,喜歡的疾走如飛。一日到于西門慶門首,來昭正在門首立,只見陶媽媽向前道了萬福,說道:「動問管家哥一聲,此是西門老爹家?」那來昭道:「你是那里來的?這是西門老爹家。老爹下世了,來有甚話說?」陶媽媽道:「累及管家進去稟聲,我是本縣官媒人,名喚陶媽媽。奉衙內小老爹鈞語分付,說咱宅內有位奶奶要嫁人,敬來說頭親事。」那來昭喝道:「你這婆子,好不近理!我家老爹沒了一年有餘,止有兩位奶奶守寡,並不嫁人。常言:『疾風暴雨,不入寡婦之門。』你這媒婆有要沒緊,走來誓撞甚親事?還不走快着,惹得後邊奶奶知道,一頓好打!」那陶媽媽笑說:「管家哥,常言:『官差吏差,來人不差。』小老爹不使我,我敢來做甚麼?嫁不嫁,起動進去稟聲,我好回話去。」這來昭道:「也罷!與人方便,自己方便。你少待片時,等我進去。兩位奶奶,一位奶奶有哥兒,一位奶奶無哥兒,不知是那一位奶奶要嫁人?」陶媽媽道:「衙內小老爹說,是清明那日郊外曾看見來,是面上有幾點白麻子兒的那位奶奶。」這來昭聽了,走到後邊,如此這般,告月娘說:「縣中使了個官媒人在外面。」倒把月娘吃了一驚,說:「我家裡並沒半個字兒送出外邊,人怎得曉的?」來昭道:「曾在郊外清明那日見來,說臉上有幾個白麻子兒的那位奶奶。」月娘便道:「莫不孟三姐也臘月裡蘿蔔動個心,忽剌八要往前進嫁人?」正是:

「世間海水知深淺,  惟有人心難忖量。」

一面走到玉樓房中坐下,便問:「孟三姐,奴有件事兒來問你。外邊有個保山媒人,說是縣中小衙內,清明那日曾見你一面,說你要往前進。端的有此話麼?」看官聽說:當時沒巧不成話,自古姻緣着緣牽。那日郊外孟玉樓看見衙內生的一表人物,風流博浪,兩家年甲多相彷彿,又會走馬拈弓弄箭,彼此兩情四目都有意,已在不言之表,但未知有妻子無妻子,口中不言,心內暗度:「況男子漢已死,奴身邊又無所出,雖大娘有孩兒,到明日長大了,各肉兒各疼,歸他娘去了,閃的我樹倒無陰,竹籃兒打水!」又見月娘自有了孝哥兒,心腸兒都改變,不似往時,「我不如往前進一步,尋上個葉落歸根之處,還只顧傻傻的守些甚麼?到沒的躭閣了奴的青春,辜負了奴的年少!」正在思慕之間,不想月娘進來說此話,正是清明郊外看見的那個人,心中又是歡喜,又是羞愧,口裡雖說:「大娘,休聽人胡說,奴並沒此話。」不覺把臉來飛紅了。正是:

「含羞對眾慵開口,  理鬢無言只搵頭。」

月娘說:「既是客人心裡事,奴也管不的許多!」一面叫來昭:「你請那保山來。」來昭來門首,喚陶媽媽進到後邊。月娘在上房明間內。正面供養着西門慶靈床。那陶媽媽旋畢禮數,坐下。小丫鬟秀春倒茶吃了,月娘便問:「保山來有甚事?」那陶媽媽便道:「小媳婦無事不登三寶殿,奉本縣正宅衙內分付,敬來說咱宅上有一位奶奶要嫁人,講說親事。」月娘道:「是俺家這位娘子嫁人,又沒曾傳出去,你家衙內怎得知道?」陶媽媽道:「俺家衙內說來,清明那日,在郊外親見這位娘子,生的長挑身材,瓜子面皮,臉上有稀稀幾個白麻子兒的,便是這位奶奶。」月娘聽了,不消說,就是孟三姐了。于是領陶媽媽到玉樓房中,明間內坐下。等勾多時,玉樓梳洗打扮出來。那陶媽媽道了萬福,說道:「就是此位奶奶,果然語不虛傳!人材出眾,蓋世無雙!堪可與俺衙內老爹,做得個正頭娘子。你看從頭看到底,風流實無比;從頭看到腳,風流往下跑!」玉樓笑道:「媽媽休得亂說!且說你衙內今年多大年紀?原娶過妻小來沒有?房中有人也無?姓甚名誰?鄉貫何處?地里何方?有官身無官身?從實說來,休要揭謊。」陶媽媽道:「天麼,天麼!小媳婦你是本縣官人,不比外邊媒人快說謊。我有一句說一句,並無虛假。俺知縣老爹年五十多歲,止生了衙內一人,今年屬馬的,三十一歲,正月二十三日辰時建生。見做國子監上舍,不久就是舉人進士;有滿腹文章,弓馬熟閒,諸子百家無不通曉。沒有大娘子二年光景,房內只有一個從嫁使女答應,又不出才兒。要尋個娘子當家,一地里又尋不着門當戶對婦。敬來宅上說此親事,若成,免小媳婦縣中打卯,還重賞在外。若是咱宅上,做這門親事,老爹說來;門面差徭,坟塋地土錢粮,一例盡行蠲免。有人欺負,指名說來,拏到縣裡任意拶打。」玉樓道:「你衙內有兒女沒有?原籍那里人氏?誠恐一時任滿,千山萬水帶去,奴親都在此處,莫不也要同他去?」陶媽媽道:「俺衙內老爹身邊男花女花沒有,好不單徑!原籍是咱北京真定府棗強縣人氏,過了黃河,不上六七百里。他家中田連阡陌,騾馬成群,人丁無數。走馬牌樓,都是撫按明文,聖旨在上,好不赫耀驚人!如今娶娘子到家,做了正房,扶正房入門為正。過後他得了官,娘子便是五花官誥,坐七香車,為命婦夫人,有何不好?」這孟玉樓被陶媽媽一席話,說得千肯萬肯,一面喚蘭春放卓兒看茶食點心,與保山吃。因說:「保山,你休怪我叮嚀盤問,你這媒人們說謊的極多,初時說的天花亂墜,地湧金蓮,及到其間,並無一物,奴也吃人哄怕了。」陶媽媽道:「好奶奶,只要一個比一個,清自清,渾自渾!歹的帶累了好的!小媳婦並不搗謊,只依本分說媒,成就人家好事。奶奶肯了,討個婚帖兒與我,好回小老爹話去。」玉樓取了一條大紅段子,使玳安交鋪子里傅夥計寫了生時八字。吳月娘便說:「你當初原是薛嫂兒說的媒,如今還使小廝叫將薛嫂兒來,兩個同拏了帖兒去,說此親事,纔有理。」不多時,使玳安兒叫薛嫂兒見陶媽媽,道了萬福。當行見當行,拏着帖兒,出離西門慶家門,往縣中回衙內話去。一個是這里冰人,一個是那頭保山。兩張口,四十八個牙,這一去,管取說得月裡嫦娥尋配偶,巫山神女嫁襄王。陶媽媽在路上問薛嫂兒:「你就是這位娘子的原媒?」薛嫂道:「然者,便是。」陶媽媽問他:「原先嫁這裡根兒,是何人家的女兒,嫁這里是再婚兒?」這薛嫂兒便一五一十,把西門慶當初從楊家娶來的話,告訴一遍。因見婚帖兒上寫女命三十七歲,十一月二十七日子時生,說:「只怕衙內嫌娘子年紀大些,怎了?他今年纔三十一歲,倒大六歲。」薛嫂道:「咱拏了這婚帖兒,交個路過的先生,算看年命妨礙不妨礙,若是不對,咱瞞他幾歲兒,不算發了眼。」正走中間,也不見路過晌板先生。只見路南遠遠的一個卦肆,青布帳幔,掛着兩行大字:「子平推貴賤,鐵筆判榮枯;有人來算命,直言不容情。」帳子底下,安放一張卓席。裡面坐着個能寫快算靈先生。這兩個媒人,向前道了萬福,先生便讓坐下。薛嫂道:「有個女命人,累先生算一算。」向袖中拏出三分命金來,說:「不當輕視,先生權且收了。路過不曾多帶錢來。」先生道:「此是合婚的意思。說八字。」陶媽媽遞與他婚帖,看上面有八字生日年紀,先生道:「此是合婚。」一面掐指尋紋,把算子搖一搖,開言說道:「這位女命,今年三十七歲了,十一月二十七日子時生,甲子月,辛卯日,庚子時,理取印綬之格。女命逆行,見在丙申運中。丙合辛生,往後幸有威權,執掌正堂夫人之命。四權中天星多,雖然財命,益夫發福,受夫寵愛。不久定見妨尅,果然見過了不曾?」薛嫂道:「已尅過兩位夫主了。」先生道:「若見過,後來得了屬馬的。」薛嫂兒道:「他往後有子沒有?」先生道:「子早哩,命中直到四十一歲,纔有一子送老,一生好造化,富貴榮華真無比。」取筆批下命詞八句:

「花盛果收奇異時,  欣遇良君立鳳池,

嬌姿不失江梅態,  三揭紅羅兩畫眉;

携手相邀登玉殿,  含羞獨步捧金巵,

會看馬首昇騰日,  脫卻寅皮任意移。」

薛嫂問道:「先生如何是『會看馬首昇騰日,脫郊寅皮任意移?』這兩句,俺每不懂,起動先生講說講說。」先生道:「馬首者,這位娘子如今嫁個屬馬的夫主,方是貴星,享受榮華。寅皮是尅過的夫主,是屬虎的。雖故受寵愛,只是偏房,往後一路功名,直到六十八歲,有一子送終,夫妻偕老。」兩個媒人,收了命狀歲罷,問先生:「與屬馬的也合的着?」先生道:「丁火庚金,火逢金鍊,定成大器,正好。」當下改做三十四歲。兩個拜辭了先生,出離卦肆,逕到縣中。衙內正坐,門子報入,良久,喚進陶嫂。二嫂旋下磕頭。衙內便問那個婦人:「是那里的?」陶媽媽道:「是項媒人。」因把親事說成,且訴一遍,說:「娘子人材無比的好,只爭年紀大些。小媳婦不敢壇便,隨衙內老爹尊意,討了個婚帖在此。」于是遞上去。李衙內看了,上寫着三十四歲,十一月二十七日子時生,說道:「就大兩三歲也罷。」薛嫂兒插口道:「老爹見的多,自古:妻大兩,黃金長;妻大三,黃金山。這位娘子,人才出眾,性格溫柔,諸子百家,當家理紀,自不必說。」衙內道:「既然好,已是見過,不必再相。命陰陽擇吉日良時,行茶禮過去就是了。」兩個媒人稟說:「小媳婦幾時來侍候?」衙內道:「事不可稽遲,你兩個明日來討話,往他家說。」分付左右:「每人且賞與他一兩銀子,做腳步錢。」兩個媒人歡喜出門,不在話下。這李衙內見親事已成,喜不自勝,即喚吏何不違來,兩個商議。對父親李知縣說了,令陰陽生擇定四月初八日行禮,十五吉日良時,准娶婦人過門。就兌出銀子來,委托何不違、小張閒買辨茶紅酒禮,不必細說。兩個媒人,次日討了日期,往西門慶家,回月娘、孟玉樓話。正是:

「姻緣本是前生定,  曾向藍田種玉來。」

四月初八日,縣中備辨十六盤羹果茶餅,一付金絲冠兒、一副金頭面、一條瑪瑙帶、一付玎璫七事、金鐲銀釧之類、兩件大紅宮錦袍兒、四套粧花衣服、三十兩禮錢,其餘布絹棉花,共約二十餘擡。兩個媒人跟隨,廊吏何不違押担,到西門慶家下了茶。十五日,縣中撥了許多快手閒漢,來搬擡孟玉樓床帳嫁妝箱籠。月娘看着,但是他房中之物,盡數都交他帶去。原舊西門慶在日,把他一張八步彩漆床,陪了大姐。月娘就把潘金蓮房那張螺鈿床,陪了他。玉樓交蘭香跟他過去。留下小鸞與月娘看哥兒,月娘不肯,說:「你房中丫頭,我怎好留下你的?左右哥兒有中秋兒、綉春和奶子也勾了。」玉樓止留下一對銀回回壺與哥兒耍子,做一念兒,其餘都帶過去了。到晚夕,一頂四人大轎,四對紅紗鐵落燈籠,八個皂隸跟隨,來娶孟玉樓。玉樓戴着金梁冠兒,插着滿頭珠翠、胡珠子,身穿大紅通袖袍兒,繫金鑲瑪瑙帶、玎璫七事;下着柳黃百花裙,先辭拜西門慶靈位,然後拜月娘。月娘說道:「孟三姐,你好狠也!你去了,撇的奴孤另另獨自一個,和誰做伴兒?」兩個携手哭了一回。然後家中大小,都送出大門。媒人替他上紅羅銷金蓋袱,抱着金寶瓶。月娘守寡,出不的門,請大姨送親。穿大紅粧花袍兒、翠藍裙,滿頭珠翠,坐大轎,送到知縣衙裡來。滿街上人看見說:「此是西門大官人第三娘子,嫁了知縣相公兒子衙內,今日吉日良時,娶過門。」也有說好的;也有說歹的。說好者:「當初西門大官人,怎的為人做人,今日死了,止是他大娘子守寡,正大有兒子,房中攪不過這許多人來,都交各人前進來,甚有張主。」有說歹的,街談巷議,指戮說道:「此是西門慶家第三個小老婆,如今嫁人,當初這廝在日,專一違天害理貪財好色,姦騙人家妻子。今日死了,老婆帶的東西,嫁人的嫁人,拐帶的拐帶,養漢的養漢,做賊的做賊。都野雞毛兒零撏了!常言:『三十年遠報。』而今眼下就報!」旁人都如此發這等暢快言語。孟大姨送親到縣衙內,舖陳床帳停當,留坐酒席來家。李衙內將薛嫂兒、陶媽媽叫到根前,每人五兩銀子,一段花紅利市,打發出門。至晚,兩個成親,極盡魚水之歡,曲盡于飛之樂。到次日,吳月娘這邊,送茶完飯。楊姑娘已死,孟大妗子、二妗子、孟大姨,都送茶到縣中。衙內這邊下回書,話眾親戚女眷做三日,扎彩山、吃筵席,都是三院樂人妓女,動鼓樂扮演戲文。吳月娘那日亦滿頭珠翠,身穿大紅通袖袍兒,百花裙、繫蒙金帶,坐大轎,來衙中做三日赴席,在後廳吃酒。知縣奶奶出來陪待。月娘回家,因見席上花攢錦簇,歸到家中,進入後邊院落,見靜悄悄,無個人接應。想起當初有西門慶在日,姊妹們那樣熱鬧;往人家赴席來家,都來相見說話,一條板凳,姊妹們都坐不了。如今並無一個兒了!一面撲着西門慶靈床兒,不覺一陣傷心,放聲大哭。哭了一回,被丫鬟小玉勸止,住了眼淚。正是:

「平生心事無人識,  只有穿窗皓月知。」

這里月娘憂悶不題。都說李衙內和玉樓兩個,女貌郎才,如魚似水。正合着油瓶蓋上,每日燕爾新婚。在房中廝守,一步不離。端詳玉樓容貌,觀之不足,看之有餘,越看越愛。又見帶了兩個從嫁丫鬟,一個蘭香,年十八歲,會彈唱;一個小鸞,年十五歲,俱有顏色。心中歡喜沒人腳處。有詩為證:

「堪誇女貌與郎才,  天合姻緣禮所該;

十二巫山雲雨會,  兩情願保百年偕。」

原來衙內房中先頭娘子丟了一個大丫頭,約三十年紀,名喚玉簪兒,專一搽胭抹粉,作怪成精。頭上打着盤頭揸髻,用手帕苫蓋。周圍勒銷金箍兒,假充作䯼髻。又插着些銅釵蠟片、敗葉殘花。耳朵上帶雙甜瓜墜子,身上穿一套前露殿月後露〈衤戲〉怪綠喬紅的裙襖。在人前好似披荷葉老鼠。腳上穿着雙裡外油劉海笑撥舡樣四個眼的剪絨鞋,約尺二長。臉上搽着一面鉛粉,東一塊白,西一塊紅,好似青冬瓜一般。在人跟前輕聲浪顙,做勢拏班。衙內未娶玉樓來時,他便逐日頓羹頓飯,殷勤扶侍;不說強說,不笑強笑,何等精神。自從娶過玉樓來,見衙內日逐和他床上睡,如膠似漆般打熱,把他不去揪採。這丫頭就有些使性兒起來。一日,衙內在書房中看書,這玉簪兒在廚下頓熱了一盞好果仁炮茶,雙手用盤兒托來。到書房裡面,笑嘻嘻掀開簾兒,送與衙內。不想衙內看了一回書,搭伏定書卓,就睡着了。這玉簪兒叫道:「爹,誰似奴疼你,頓了這盞好茶兒與你吃。你家那新娶的娘子,還在被窩裡睡得好覺兒!怎不交他那小大姐送盞茶來與你吃?」因見衙內打盹,在根前只顧叫不應。說道:「老花子,你黑夜做夜作,使乏了也怎的,大白日打睡磕睡!起來吃茶!」叫衙內醒了,看見是他,喝道:「怪硶奴才!把茶放下,與我過一邊里去。」這玉簪兒便臉羞紅了,使性子把茶丟在卓上。出來說道:「好不識人敬重!奴好意用心,大清早辰送盞茶兒來你吃,倒喓喝罷我!常言:『醜是家中寶,可喜惹煩惱!』我醜,你當初瞎了眼,誰教你要我來家的?值我的那大精〈毛皮〉!」簏被衙內聽見,趕上儘力踢了兩靴腳。這玉簪兒登時把那付奴臉,膀的有房梁高。也不搽臉了,也不頓茶造飯了,趕着玉樓也不叫娘,只你也我也的。無人處,一個屁股就同在玉樓床上坐,玉樓亦不去理他。他背地又壓伏蘭香、小鸞,說:「你休趕着我叫姐,只叫姨娘。我與你娘,係大小五分。」又說:「你只背地叫罷,休對着你爹叫,你每日跟逐我行,用心做活,你若不聽堵歌,老娘拏煤鍬子請你!」後來幾次,見衙內不理他,他就撒懶起來。睡到日頭半天,還不起來,飯兒也不做,地兒也不掃。玉樓分付蘭香、小鸞:「你休靠玉簪兒了。你二人自去廚下做飯,打發你爹吃罷。」他又氣不憤,使性謗氣,牽家打活,在廚房內打小鸞、罵蘭香:「賊小奴才,小淫婦兒,碓磨也有個先來後到!先有你娘來?先有我來?都你娘兒們占了罷,不獻這個勤兒也罷了!當原先俺死了那個娘,也沒曾失口叫我聲玉簪兒。你進門幾日,就題名道姓叫我,我是你手里使的人也怎的!你未來時,我和俺爹同床共枕,那一日不睡到齋時纔起來?和我兩個如糖拌蜜,如蜜攪酥油一般打熱。房中事,那些兒不打我手裡過?自從你來了,把我蜜罐兒也打碎了,把我姻緣也拆散開了!一攆攆到我明間,冷清清支板凳打官舖。再不得嘗着俺爹那件東西兒甚麼滋味兒!正也沒聲處訴!你當初在西門慶家,也曾做第三個小老婆來,你小名兒叫玉樓,敢說老娘不知道?你來在俺家,你識我見,大家膿着些罷了!會那等大廝不道喬張致,呼張喚李,誰是你買到的,屬你管轄?」不識那玉樓在房中聽見,氣得發昏,連套手戰,只是不敢聲言對衙內說。一日熱天,也是合當有事。晚夕,衙內分付他廚下熱水,拏浴盆來房中,要和玉樓洗澡。玉樓便說:「你交蘭香熱水罷,休要使他。」衙內不從,說道:「我偏使他,休要慣了這奴才。」玉簪兒見衙內要水和婦人洗澡,共浴蘭湯,效魚水之歡,借于飛之樂,心中正沒好氣。拏浴盆進房,往地下只一墩,用大鍋燒上一鍋滾水,口內喃喃吶吶說道:「也沒見這浪淫婦,刁鑽古怪,禁害老娘!無過也只是個浪精〈毛皮〉,沒三日不拏水洗!像我與俺主子睡,成月也不見點水兒,也不見展污了甚麼佛眼兒!偏這淫婦,會兩番三次,刁蹬老娘!」直罵出房門來。玉樓聽見,也不言語。衙內聽了此言,心中大怒,澡也洗不成,精脊梁,靸着鞋,向床頭取拐子,就要走出來。婦人攔阻住,說道:「隨他罵罷,你好惹氣?只怕熱身子出去,風試着你,倒值了多的!」衙內那裡按納得住,說道:「你休管他,這奴才無禮!」向前一把手,採住他頭髮,拖踏在地下,輪起拐子,雨點打將下來。饒玉樓在旁勸着,也打有二三十下在身。打的這丫頭急了,跪在地下,告說:「爹,你休打我,我有句話兒和你說。」衙內罵:「賊奴才,你說!」有山坡羊為證:

「告爹行,停嗔息怒,你細細兒聽奴分訴。當初你將八兩銀子財禮錢,娶我當家理紀,管着些油鹽醬醋。你吃了飯吃茶,只在我手裡抹布。沒了俺娘,你也把我陞為個署府。咱兩個同舖同床,何等的頑耍?奴桉家伏業,纔把這活來做。誰承望你哄我,說不娶了。今日又起這個毛心兒里來呵,把往日恩情,弄得半星兒也無!叫了聲爹,你忒心毒!我如今不在你家了,情願嫁上個姐夫!」

衙內聽了,亦發惱怒起來,又狠了幾下。玉樓勸道:「他既要出去,你不消打,倒沒得氣了你。」衙內隨令伴當即時叫將媒人陶媽媽來,把玉簪兒領出去,變賣銀子來交,不在話下。正是:

「蚊蟲遭扇打,  只為嘴傷人。」

有詩為證:

「百禽啼後人皆喜,  惟有鴉鳴事若何;

見者多嫌聞者唾,  只為人前口嘴多。」

畢竟未知後來何如,且聽下回分解:





\end{showcontents}


