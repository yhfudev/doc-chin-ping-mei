%# -*- coding: utf-8 -*-
%!TEX encoding = UTF-8 Unicode
%!TEX TS-program = xelatex
% vim:ts=4:sw=4
%
% 以上设定默认使用 XeLaTex 编译,并指定 Unicode 编码,供 TeXShop 自动识别

%第五十二回 
\chapter{應伯爵山洞戲春嬌\KG 潘金蓮花園看蘑菇}


「海棠深院雨初收,  苔徑無風蝶自由,

百結丁香誇美麗,  三眠楊柳弄輕柔:

小桃酒膩紅尤淺,  芳草寒餘綠漸稠,

寂寂珠簾歸燕子,  子規啼處一春愁。」

話說那日西門慶在夏提刑家吃酒,宋巡按送禮與他,心中十分歡喜,夏提刑亦敬重不同往日,攔門勸酒;吃至二更天氣,纔放回家。潘金蓮又早向燈下除去冠兒,露著粉面油頭,交春梅床上設放衾枕,搽抹涼蓆乾淨。薰香澡牝,等候西門慶。進門接著,見他酒帶半酣,連忙替他脫了衣裳,春梅點茶來吃了,打發上床歇息。見婦人脫得光赤條身子,坐著床沿,低垂著頭,將那白生生腿兒橫抱膝上纏腳換剛三寸,恰半窄,大紅平底睡鞋兒。西門慶一見,淫心輙起,塵柄挺然而興,因問婦人要淫器包兒。婦人連忙向褥子底下,摸出來,遞與他。西門慶把兩個托子都帶上,一手摟過婦人在懷裡,因說:「你達今日要和你幹個後庭花兒,你肯不肯?」那婦人瞅了一眼,說道:「好個沒廉恥冤家!你成日和書童兒小廝幹的不值了,又纏起我來了。你和那奴才幹去不是!」西門慶笑道:「怪小油嘴兒,罷麼!你若依了我,又稀罕小廝做甚麼?你不知你達心裏,好的是這樁兒。管情放到里頭去,我就過了。」婦人被他再三纏不過,說道:「奴只怕挨不的你這大行貨,你把頭子上圈去了一個,我和你要一遭試試。」西門慶真個除去硫黃圈,根下只柬著銀托子,令婦人馬爬在床上,屁股高蹶,將唾津塗抹在龜頭上,往來濡研頂入。龜頭昂健,半響僅沒其稜,婦人在下,蹙眉隱忍,口中咬汗子難捱,叫道:「達達慢著些,這個比不的前頭,撐得裏頭熱炙火燎,疼起來。」這西門慶叫道:「好心肝,你叫著達達不妨事,到明日買一套好顏色粧花紗衣服與你穿。」婦人道:「那衣服倒也有在,我昨日見李桂姐穿的,那五色線搯羊皮金挑的,油鵝黃銀條紗裙子倒好看,說是裡邊買的。他每都有,只我沒這條裙子。倒不知多少銀子,你倒買一條我穿罷了!」西門慶道:「不打緊,我到明日替你買。」一壁說著,在上頗作抽拽,只顧沒稜露腦,淺抽深送不已。婦人回首流眸叫道:「好達達,這裏緊著人,疼的要不的。如何只顧這般動作起來了,我央及你,好歹快些丟了罷!」這西門慶不聽,且扶其股,翫其出入之勢,一面口中呼道:「潘五兒,小淫婦兒,你好生浪浪的叫著達達,哄出你達達〈尸從〉兒來罷!」那婦人真個在下,星眼朦朧,鶯聲款掉,柳腰款擺,香肌半就,口中艷聲柔語,百般難述。良久,西門慶覺精來,兩手扳其股,極力而〈扌扉〉之,扣股之聲,響之不絕,那婦人在下邊,呻吟成一塊,不能禁止。臨過之時,西門慶把婦人屁股只一扳,塵柄盡沒至根,直低于深異處,其美不可當,于是怡然感之,一泄如注。婦人承受其精,二體偎貼良久,拽出塵抦,但見猩紅染莖,蛙口流涎,婦人以帕抹之,方纔就寢,一宿晚景題過。次日,西門慶早辰到衙門中回來,有安主事、黃主事那裏差人來下請書:二十二日,在磚廠劉太監庄上設席,請早去。西門慶打發人去了,從上房吃了粥,正出廳來。只見篦頭的小周兒,扒倒地下磕頭,在傍伺候。西門慶道:「你來得正好,我正要尋你篦篦頭哩。」于是走到花園翡翠軒小捲棚內,西門慶坐在一張京椅兒上,除了巾幘,打開頭髮。小周兒在後面卓上,鈾下梳篦家活,與他篦頭榔髮。觀其泥垢,辨其風雪,跪下討賞錢,說:「老爹今歲必有大遷轉,髮上氣色甚旺!」西門慶大喜,篦了頭,又交他取耳,搯捏身上。他有滾身上一弄兒家活,到處都與西門慶滾捏過,又行導引之法,把西門慶弄的渾身通泰。賞了他五錢銀子,交他吃了飯伺候,與哥兒剃頭。西門慶就在書房內,倒在大理石床上,就睡著了。那日楊姑娘起身,王姑子與薛姑子要家去。吳月娘將他原來的盒子,都裝了些蒸酥茶食,打發起身。兩個姑子,每人又是五錢銀子;兩個小姑子,與了他兩疋小布兒,管待出門。薛姑子又囑付月娘:「到壬子日,把那藥吃了,管情就有喜事。」月娘道:「薛爺,你這一去,八月裏到我生日,好歹走走,我這里盼你哩!」薛姑子合掌問訊道:「打攪菩薩這里!我到那日已定來。」于是作辭月娘,眾人都送到大門首。月娘與大妗子回後邊去了,只有孟玉樓、潘金蓮、李瓶兒、西門大姐、李桂姐,穿著白銀條紗對衿衫兒,鵝黃縷金挑線紗裙子,戴著銀絲䯼髻,翠水祥雲鈿兒,金累絲簪子,紫夾石墜子,大紅鞋兒,抱著官哥兒,來花園裡遊翫。李瓶兒道:「桂姐,你遞過來,等我抱罷。」桂姐道:「六娘,不妨事,我心裏要抱抱哥子。」孟玉樓道:「桂姐,你還沒到你爹新收拾書房兒瞧瞧來!」到花園內,金蓮見紫薇花開得爛熳,摘了兩朵與桂姐戴。于是順著松墻兒,到翡翠軒,見裡邊擺設的床帳屏几,書畫琴棋,極其消灑。床上綃帳銀鈎,冰簟珊枕,西門慶正倒在床上,睡思正濃。傍邊流金小篆,焚著一縷龍涎,綠窗半掩,窗外芭蕉低映。那潘金蓮且在卓上,掀弄他的香盒兒,玉樓和李瓶兒都坐在椅兒上。西門慶忽翻過身來,看見眾婦人都在屋裡。便道:「你每來做甚麼?」金蓮道:「桂姐要看看你的書房裡,俺每引他來瞧瞧。」那西門慶見他抱著官哥兒,又引鬬了一回。忽見畫童來說:「應二爹來了。」眾婦人都亂走不迭,往李瓶兒那邊去了。應伯爵走到松墻邊,看見桂姐抱著官哥兒,便道:「好呀!李桂姐在這裏!」故意問道:「你幾時來?」那桂姐走了,說道:「罷麼。怪花子,又不關你事,問怎的?」伯爵道:「好小淫婦兒,不關我事?也罷,你且與我個嘴罷。」于是摟過來,就要親嘴。被桂姐用手只一推,罵道:「賊不得人意怪攘刀子!若不是怕諕了哥子,我這一扇把子打的你!」西門慶走出來,看見伯爵拉著桂姐,說道:「怪狗材,看諕了孩兒!」因交書童:「你抱哥兒送與你六娘去。」那書童連忙接過來。奶子如意兒正在松墻拐角邊等候,接的去了。伯爵和桂姐兩個,站著說話,問:「你的事怎樣的?」桂姐道:「多虧爹這里可怜見,差保哥替我往東京說去了。」伯爵道:「好好,也罷了,如此你放心些。」說畢,桂姐就往後邊去了。伯爵道:「怪小淫婦兒,你過來,我還和你說話。」桂姐道:「我走走就來。」于是也往李瓶兒這邊來了。伯爵與西門慶纔唱喏,兩個在扜內坐的。西門慶道:「昨日我在夏龍溪家吃酒,大巡宋道長那里差人送禮,送了一口鮮豬。我恐怕放不的,今早旋叫了廚子來卸開,用椒料 連豬頭燒了。你休去了!如今請了謝子純來,咱每打雙陸同享了罷。」一面使琴童兒:「快請你謝爹去,你說應二爹在這裡。」琴童兒應諾,一直去了。伯爵因問:「徐家銀子,討了來了?」西門慶道:「賊沒行止的狗骨禿!明日纔有。先與二百五十兩,你交他兩個後日來;少,我家裏湊與他罷。」伯爵道:「這等又好了。怕不的他今日買些鮮物兒來孝順你!」西門慶道:「倒不消交他費心。」說了一回。西門慶問道:「老孫、祝麻子兩個,都起身去了不曾?」伯爵道:「這咱哩,從李桂兒家拏出來,在縣里監了一夜。第二日,三個一條鐵索,都解上東京去了。到那裏,沒個清潔來家的。你只說成日圖飲酒食肉前架蟲,好容易吃的果子兒!似這等苦兒也是他受。路上這等大熱天,著鐵索扛著,又沒盤纏,有甚麼要緊!」西門慶笑道:「怪狗材,充軍擺站的不過。誰交他成日跟著王家小廝,只胡撞來?李六他尋的苦兒他受!」伯爵道:「哥,你說的有理。蒼蠅不鑽沒縫的雞彈!他怎的不尋我和謝子純?清的只是清,渾的只是渾!」正說著,謝希大到了。唱畢喏坐下,只顧搧扇子。西門慶問道:「你怎的走恁一臉汗?」希大道:「哥別題,大官兒去遲了一步兒,我不在家了。看剛出大門,可可他就到了。今日平白惹了一肚子氣!」伯爵問道:「你惹的又是甚麼氣?」希大道:「大清早辰,老孫媽媽子走到我那裏說我弄了他去!因主何故?恁不合理的老淫婦!你家漢子成日摽著人在院裏頑,酒快肉吃,大把家撾了銀子錢家去。你過陰去來,誰不知道?你討保頭錢,分與那個一分兒使,也怎的!交我扛了兩句,走出來,不想哥這裏呼喚。」伯爵道:「我剛纔這裏和哥不說,新酒放在兩下哩,清自清,渾自渾,出不的。咱每怎麼說來?我說跟著王家小廝,到明日有一欠;今日如何,撞到這網里,怨暢不的人!」西門慶道:「王家那小廝,看甚大氣概,幾年兒了,腦子還未變全。養老婆,還不勾俺每那咱撒下的,羞死鬼罷了!」伯爵道:「他曾見過甚麼大頭面?且比哥那咱的勾當?題起來,把他諕殺了罷了!」說畢,小廝拿茶上來吃了。西門慶道:「你兩個打雙陸,後邊做著個水麵 ,等我叫小廝拿麵來咱每吃。」不一時,琴童來放卓兒,畫童兒用方盒拿上四個靠山小碟兒,盛著四樣小菜兒,一碟十香瓜茄 ,一碟五方荳鼓,一碟醬油浸的鮮花椒,一碟糖蒜 ,三碟兒蒜汁,一大碗豬肉滷,一張銀湯匙,三雙牙筯,擺放停當。西門慶走來坐下,然後拿上三碗麵來,各人自取澆滷,傾上蒜醋。那應伯爵與謝希大,拏起筋來,只三扒兩嚥,就是一碗;兩人登時,狠了七碗。西門慶兩碗還吃不了。說道:「我的兒,你兩個吃這些!」伯爵道:「哥,今日這麵是那位姐兒下的?又爽口,又好吃。」謝希大道:「本等滷打的停當。我只是剛纔家裏吃了飯來了,不然,我還禁一碗。」兩個吃的熱上來,把衣服脫了,搭在椅子上。見琴童兒收家活,便道:「大官兒,到後邊取些水來,俺每漱漱口。」謝希大道:「溫茶兒又好,熱的盪的死蒜臭。」少頃,畫童兒拿茶至。三人吃了茶,出來外邊松墻外,各花臺邊走了一遭。只見黃四家送了四盒子禮來,平安兒掇進了,與西門慶瞧,一盒鮮烏菱,一盒鮮荸薺 ,四尾冰湃的大鰣魚,一盒枇杷果。伯爵看見,說道:「好東西兒!他不知那裏剜的送來?我且嚐個兒著。」一手撾了好幾個,遞了兩個與謝希大,說道:「還有活到老死,還不知此物甚麼東西哩!」西門慶道:「怪狗材,還沒供養佛,就先撾了吃。」伯爵道:「甚麼沒供佛,我且入口無賍著。」西門慶分付:「交到後邊收了。問你三娘討三錢銀子賞他。」伯爵問:「是李錦送來?是黃寧兒?」平安道:「是黃寧兒。」伯爵道:「今日造化了這□骨禿了,又賞他這三錢銀子。」這裏西門慶看著他兩個打雙陸不題。且說桂姐和他姑娘李嬌兒、孟玉樓、潘金蓮、李瓶兒、大姐,都在後邊上房明間內吃了飯,在穿廊下坐的。只見小周兒在影壁前,探頭舒腦的。李瓶兒道:「小周兒,你來的好,且進來與小大官兒剃剃頭,把頭髮都長長了。」小周兒連忙向前,都磕了頭說:「剛纔老爹分付,交小的進來,與哥兒剃頭。」月娘道:「六姐,妳拏曆頭看看,好日子歹日子?就與孩子剃頭!」這金蓮便交小玉取了曆頭來,揭開看了一回,說道:「今日是四月廿一日,是個庚戍日,定婁金金狗當直,宜祭祀、官帶出行、裁衣沐浴、剃頭、修造動土,宜用午時。好日期!」月娘道:「既是好日子,交丫頭熱水,你替孩兒洗頭。交小周兒慢慢哄著他剃。小玉在傍,替他用汗巾兒接著頭髮兒。那裏纔剃得幾刀兒下來,這官哥兒呱的聲怪哭起來。那小周連忙趕著他哭,只顧剃。不想把孩子哭的那口氣敝下去,不言語了,臉便脹的紅了。李瓶兒也諕慌手腳,連忙說:「不剃罷,不剃罷!」那小周兒諕的收不迭家活,往外沒腳子跑。月娘道:「我說這孩子,有些不長俊,護頭,自家替他剪剪罷。平白交進來剃,剃的好麼?」天假其變,那孩子彆了半日氣,放出聲來了。李瓶兒一塊石頭方纔落地,只顧抱在懷裏,拍哄著他,說道:「好小周兒,恁大膽,平白進來,把哥哥頭來剃了去了!剃的恁半落不合接,欺負我的哥哥!還不拏回來,等我打與哥哥出氣!」於是抱到月娘根前。月娘道:「不長俊的小花子兒,剃頭耍了你,便益了這等哭,剩下這些,到明日做剪毛賊!」引鬬了一回,李瓶兒交與奶子。月娘分付:「且休與他奶吃,等他睡一回兒與他吃。」奶子抱的他前邊去了。只見來安兒進來,取小周兒的家活,說:「門首諕的小周兒臉焦黃的。」月娘問道:「他吃了飯不曾?」來安道:「他吃了飯,爹賞他五錢銀子。」月娘交來安:「你拏一甌子酒出去與他。諕著人家,好容易討這幾個錢!」小玉連忙篩了一盞,拏了一碟臘肉,交來安與他吃了,往家去了。吳月娘因交金蓮:「你看看曆頭,幾時是壬子日?」金蓮看了,說道:「二十三是壬子日,交芒種五月節。」便道:「姐姐,你問他怎的?」月娘道:「我不怎的,問一聲兒。」李桂姐接過曆頭來看了,說道:「這二十四日苦惱,是俺娘的生日,我不得在家。」月娘道:「前月初十日,是你姐姐生日過了。這二十四日,可可兒又是你媽的生日了。原來你院中人家,一日害這樣病,做三個生日,日里害思錢病,黑夜思漢子的病;早辰是媽的生日,晌午是姐姐生日,晚夕是自家生日。怎的都擠在一塊兒?趁著姐夫有錢,竄掇著都生日了罷。」桂姐只是笑,不做聲。只見西門慶使了畫童兒來請,桂姐方向月娘房中粧點,勻了臉,往花園中來。捲棚內,又早放下八仙卓兒,前後放下簾櫳來。卓上擺設許多肴饌,兩大盤燒豬肉,兩盤燒鴨子 ,兩盤新煎鰣魚,四碟玫瑰點心,兩碟白燒荀雞 ,兩碟燉爛鴿子雛兒 。然後又是四碟臟子、血皮、豬肚 、釀腸 之類。眾人吃了一回,桂姐在傍拏鍾兒遞酒。伯爵道:「你爹聽著說,不是我索落你,事情兒已是停當了。你爹又替你縣中說了,不尋你了;虧了誰?還虧了我,再三央及你爹,他纔肯了,平白他肯替你說人情去了?隨你心處的甚麼曲兒,你唱個兒我聽下酒,也是拏勤勞准折。」桂姐笑罵道:「怪硶花子,你虼虫塀喿兒,好大面皮兒!爹他肯信你說話?」伯爵道:「你這賊小淫婦兒,你經還沒唸,就先打和尚起來!要吃飯,休要惡了火頭。你敢笑和尚沒丈母?我就單丁擺佈不起你這小淫婦兒?你休笑譁,我半邊俏還動的!」被桂姐拏手中扇把子,儘力向他身上打了兩下。西門慶笑罵道:「你這狗材,到明日論個男盜女娼,還虧了原問處。」笑了一回,桂姐慢慢纔拏起琵琶,橫擔膝上,啟朱唇,露皓齒,唱了個伊州三台令:

「思量你好辜恩,便忘了誓盟,遇花朝月夕良辰,好交我虛度了青春。悶懨懨,把欄杆凭倚,疑望他怎生全無個音信?幾回自將,多應是我薄緣輕。」

〔黃鶯兒〕

「誰想有這一種,(伯爵道:「陽溝里翻了船,後十年也不知道。」)減香肌,憔瘦損;(伯爵道:「愛好貪他,閃在人水里。」)鏡鸞塵鎖,無心整。脂粉輕勻,花枝又懶簪;空教黛眉蹙破春山恨。(伯爵道:「你記的說,接客千個,情在一人,無言對鏡長吁氣,半是思君半恨君。你兩個當初好,如今就為他耽些驚怕兒也罷,不抱怨了!」桂姐道:「汗邪了你,怎的胡說!」)最難禁,(伯爵道:「你難禁,別人卻怎樣禁的?」)樵樓上畫角,吹徹了斷腸聲!(伯爵道:「腸子倒沒斷,這一回來,提你的斷了線,你兩個休提了。」被桂姐儘力打了一下,罵道:「賊們攘的,今日汗歪了你,只鬼混人的!」)

〔集賢賓〕

「幽窗靜悄,月又明,恨獨倚幃屏。驀聽的孤鴻,只在樓外鳴,把萬愁又還題醒。更長漏永,早不覺燈昏香盡,眠未成,他那裏睡得安穩?」(伯爵道:「傻小淫婦兒,他怎的睡不安穩?又沒拏了他去,落合的在家裏睡覺兒裡。你便在人家躲著,逐日懷著羊皮兒,直等東京人來,一塊石頭方落地。」桂姐被他說急了,便道:「爹,你看應花子來!不知怎的,只發訕纏我!」伯爵道:「你這回纔認得爹了?」桂姐不理他,彈著琵琶又唱。)

〔雙聲叠韻〕

「思量起,思量起,怎不上心。(伯爵道:「揉著你那癢癢處,不由你不上心。」)無人處,無人處,淚珠兒暗傾。」(伯爵道:「一個人慣溺床,那一日,他娘死了,守孝,打鋪在靈前睡,晚了,不想又溺下了。人進來看見褥子濕,問:『怎的來?』那人沒的回答,只說:『你不知,我夜間眼淚打肚裏流出來了。』就和你一般,為他聲說不的,只好背地哭罷了。」桂姐道:「沒羞的孩兒,你看見來?汗邪了你哩!」)「我怨他,我怨他,說他不盡;(伯爵道:「我又一件說,你怎的不怨天,赤道得了他多少錢?見今日躲在人家,把買賣都誤了!說他不盡,是左門神白臉子,極古來子。不知道甚麼兒的,好哄他。」)誰知道這裏先走滾。(伯爵道:「可知拏著到手中,還飛了哩!」)自恨我當初,不合地認真!」(伯爵道:「傻小淫婦兒,如今年程在這裏,小歲小孩兒出來,也哄不過,何況風月中子弟,你和他認真?你且住了,等我唱個南枝兒你聽:『風月事,我說與你聽,如今年程,論不的假真,個個人古怪精靈,個個人久慣老誠。倒將計活埋,他瞎缸暗頂。老虔婆只要圖財,小淫婦兒少不的拽著脖子往前掙!苦似投河,愁如覓井。幾時得把業罐子填完,就變驢變馬也不幹這個營生!』」當下把桂姐說的哭起來了。被西門慶向伯爵頭打了一扇子,笑罵道:「你這斷了腸子的狗材,生生兒吃你把人就歐殺了!」因叫桂姐:「你唱,不要理他。」謝希大道:「應二哥,你好沒趣,今日左來右去,只欺負我這乾女兒!你再言語,口上生個大疔瘡!」那桂姐半日拏起琵琶又唱。)

〔簇御林〕

「人都道他志誠,(伯爵纔待言語,被希大把口按了,說道:「桂姐,你唱,休理他!」李桂姐又唱道。)卻原來廝勾引,眼睜睜,心口不相應。(希大放了手。伯爵又說:「相應倒好了,弄不出此事來了。心口裏不相應,如今虎口裏倒相應;不多,也只兩三炷兒。」桂姐道:「白眉赤眼,你看見來?」伯爵道:「我沒看見,在樂星堂裏不是?」連西門慶眾人都笑起來了。)山誓海盟,說假道真,險些兒不為他錯害了相思病!(伯爵道:「好保蟲兒,只有錯買了的,沒有錯賣了的。你院中人,肯把病兒錯害了?」)負人心,看伊家做作,如何交我有前程?」(伯爵道:「前程也不敢指望他,到明日,少不了他個招宣襲了罷!」)

〔琥珀貓兒〕

「日疏日遠,再相逢,枉了奴痴心寧耐等。(伯爵道:「等到幾日?到明日東京了畢事,再回爐也是不遲。」)想巫山雲雨夢難成,薄情,猛拚今生,和你鳳拆鸞,鳳拆鸞!」

〔尾聲〕

「冤家下得忒薄〈亻辛〉,割捨的將人孤另,那世里恩情,番成做話柄!」

唱畢,謝希大道:「罷罷!叫畫童兒接過琵琶去,等我酬勞桂姐一杯酒兒!」伯爵道:「等我哺菜兒。我本領兒不濟事,拏勤勞准折罷了。」桂姐道:「花子過去,誰理你!你大拳打了人,這回拏手來摸挲!」當下希大一連遞了桂姐三杯酒。拉伯爵道:「咱每還有那兩盤雙陸了罷。」于是二人又打雙陸。西門慶遞了個眼色與桂姐,就往外走。伯爵道:「哥你往後邊去,捎些香茶 兒出來。頭里吃了些蒜,這回子倒反帳兒,惡泛泛起來了。」西門慶道:「我那里得得香茶兒來?」伯爵道:「哥,你還哄我哩。杭州劉學官送了你好少兒著,你獨吃也不好。」西門慶笑的後邊去了。那桂姐□□出來,在太湖石畔推搯花兒戴,也不見了。伯爵與希大一連打了三盤雙陸,等西門慶白不見出來,問畫童兒:「你爹在後邊做甚麼哩?」畫童兒道:「爹在後邊,就出來了。」伯爵道:「就出來,卻往那去了?」因交謝希大:「你這裏坐著,等我尋他尋去。」那謝希大且和書童兒兩個在書卓上下象棋。原來西門慶只走到李瓶兒房裏,就出來了。在木香棚下,看見李桂姐,就拉到藏春塢雪洞兒里,把門兒掩著,兩個坐在矮床兒上說話。原來西門慶走到李瓶兒房裏,吃了藥出來。把桂姐摟在懷中,坐于腿上,一徑露出那話來,與他瞧。把桂姐諕了一跳,便問:「怎的就這般大?」西門慶悉把吃胡僧藥,告訴了一遍。先交他低垂粉頸,款啟惺唇,品咂了一回。然後輕輕搊起他剛半扠,恰三寸,好錐靶,賽藕芽,步香塵,舞翠盤,千人愛,萬人貪,兩隻小小金蓮來,跨在兩邊肐膊,穿著大紅素段白綾高底鞋兒,粧花金欄膝褲,腿兒用紗綠線帶紮著,抱到一張椅兒上,兩個就幹起來。不想應伯爵到各亭兒上,尋了一遭,尋不著,打滴翠巖小洞兒里穿過去。到了木香棚,抹轉葡萄架,到松竹深處藏春塢邊,隱隱聽見有人笑聲,又不知在何處。這伯爵慢慢躡足潛蹤,掀開簾兒,見兩扇洞門兒虛掩,在外面只顧聽覷。聽見桂姐顫著聲兒,將身子只顧迎播著西門慶叫:「達達,快些了事罷,只怕有人來。」被伯爵猛然大叫一聲,推開門進來。看見西門慶把桂姐扛著腿子,在椅兒上正幹得好,說道:「快取水來,潑潑兩個攘心的,摟到一答里了。」李桂姐道:「怪攘刀子,猛的進來,諕了我一跳!」伯爵道:「快些兒了事,好容易?也得值那些數兒是的!怕有人來看見,我就來了。且過來,等我抽個頭兒著。」西門慶便道:「怪狗材!快出去罷了,休鬼混!我只怕小廝來看見。」那應伯爵道:「小淫婦兒,你央及我央及兒;不然,我就要喝起來,連後邊嫂子們都嚷的知道。你既認做乾女兒了,好意交你躲住兩日兒,你又偷漢子!交你了不成?」桂姐道:「去罷,應怪花子。」伯爵道:「我去罷!我且親個嘴著。」于是按著桂姐,親訖一嘴,纔走出來。西門慶:「怪狗材!還不帶上門哩!」伯爵一面走來,把門帶上,說道:「我兒,兩個儘著搗,儘著搗。搗吊底子,不關我事。」纔走到那個松樹兒底下,又回來說道:「你頭里許我的香茶,在那裏?」西門慶道:「怪狗材,等住會,我與你就是了,又來纏人!」那伯爵方纔一直笑的去了。桂姐道:「好個不得人意的攘刀子的!」這西門慶和桂姐兩個在雪洞內,足幹勾約一個時辰,吃了一枚紅棗兒,纔得了事,雨散雲收。有詩為證:

「海棠枝上鶯梭急,  綠竹陰中燕語頻;

閒來付與丹青手,  一段春嬌畫不成。」

少頃,二人整衣出來。桂姐向他袖子內,掏出好些香茶 來袖了。西門慶則使的滿身香汗,氣喘吁吁,走來馬纓花下溺尿。李桂姐腰裏模出鏡子來,在月窗上擱著,整雲理鬢,往後邊去了。西門慶走到李瓶兒房裏,洗洗手出來。伯爵問他要香茶,西門慶道:「怪花子,你害了痞!如何只鬼混人!」每人搯了一撮與他。伯爵道:「只與我這兩個兒!由他由他,等我問李家小淫婦兒要。」正說著,只見李銘走來磕頭。伯爵道:「李日新在那裏來?你沒曾打聽得他每的事怎麼樣兒了?」李銘道:「俺桂姐虧了爹這裏。這兩日縣里也沒人來催,只等京中示下哩。」伯爵道:「齊家那小老婆子出來了?」李銘道:「齊香兒還在王皇親宅內躲著哩。桂姐在爹這里好,誰人敢來尋?」伯爵道:「要不然也費手,虧我和你謝爹,再三央勸你爹:『你不替他處處兒,交他那里尋頭腦去?』」李銘道:「爹這裏不管,就了不成;俺三嬸老人家,風風勢勢的,幹出甚麼事?」伯爵道:「我記的這幾時是他生日。俺每會了你爹,與他做做生日。」李銘道:「爹們不消了。到明日事情畢了,三嬸和桂姐愁不請爹們坐坐。」伯爵道:「到其間,俺每補生日就是了。」因叫他近前:「你且替我吃了這鍾酒著,我吃了這一日了,吃不的了。」那李銘接過銀把鍾來,跪著一飲而盡。謝希大交琴童,又斟了一鐘與他。伯爵道:「你敢沒吃飯?卓上還剩了一盤點心。」謝希大又拏兩盤燒豬頭肉 和鴨子,遞與他。李銘雙手接的,下邊吃去了。伯爵用筯子又撥了半段鰣魚與他,說道:「我見你今年還沒食這個哩,且嚐新著。」西門慶道:「怪狗材,都拏與他吃罷了,又留下做甚麼?」伯爵道:「等住回,吃的酒闌上來餓了,我不會吃飯兒?你每那里江南此魚,一年只過一遭兒!吃到牙縫兒里,剔出來,都是香的,好容易!公道說,就是朝廷還沒吃哩!不是哥這里,誰家有?」正說著,只見畫童兒拿出四碟鮮物兒來:一碟烏菱,一碟荸薺 ,一碟雪藕,一碟枇杷。西門慶還沒曾放到口裏,被應伯爵連碟子都撾過去,倒的袖了。謝希大道:「你也留兩個兒我吃。」也得手撾一碟子烏菱來,只落下藕在卓子上。西門慶搯了一塊,放在口內,別的與了李銘吃了。分付畫童,後邊再取兩個枇杷來賞李銘。李銘接的袖了,到家和與三媽吃。李銘吃了點心,上來拏箏過來,纔彈唱了。伯爵道:「你唱個花藥欄,俺每聽罷!」李銘調定箏絃,拏腔唱道:

「新綠池邊,猛拍欄杆,心事向誰論?花也無言,蝶也無言,離恨滿懷縈牽。恨東君不解留去客,嘆舞紅飄絮,蝶粉輕沾。景依然,事依然,悄然不見郎面。」

「俺想別時正逢春,海棠花初綻,蕊微分開現。不覺的榴花噴,紅蓮放沉水,果避暑搖紈扇。霎時間菊花黃,金風動,敗葉桐梧變。」

「逡巡見臘梅開水花墜,暖閣內把香醪旋。四季景偏多,思想心中怨。不知俺那俏冤家冷清清獨自個,悶懨懨,何處耽寂怨?」

「金殿喜重重嗟怨,自古風流誤少年。那嗟暮春天!生怕到黃昏,愁怕到黃昏,獨自個悶不成歡。換寶香薰被,誰共宿?嘆夜長枕冷衾寒。你孤眠,我孤眠只是夢裏相見。」

〔貨郎兒〕

「有一日稱了俺平生心愿,成合了夫妻謝天;今生天對兒好姻緣,冷清清耽寂寞,愁沉沉受熬煎。」

〔醉太平煞尾〕

「只為俺多情的業冤,今日恨惹情牽。想當初說山盟言誓在星前,擔閣了風流少年。有一日朝雲暮雨成姻眷,畫堂歌舞排歡宴,羅幃錦帳永團圓。花燭洞房成連理,休忘了受過熬煎有萬千。」

當日三個吃至掌燈時候,還等著後邊拿出綠荳白米飯來吃了,纔去。伯爵道:「哥,明日不得閒?」西門慶道:「我明日往磚廠劉監庄子上,安主事、黃主事兩個昨來請我吃酒,早去了。」伯爵道:「李三、黃四那事,我後日會他來罷!」西門慶點頭兒,分付:「交他那日後晌來,休來早了。」三人也不等送就去了。西門慶交書童看著收家活,就歸後邊孟玉樓房中歇去了,一宿無話。到次日西門慶早起,也沒往衙門中去。吃了粥,冠帶著,騎馬拏著金扇,僕從跟隨,出城南三十里,逕往劉太監庄上來赴席。那日書童與玳安兩個,都跟去了,不在話下。潘金蓮趕西門慶不在家,與李瓶兒計較,將陳經濟輸的那三錢銀子,又交李瓶兒添出七錢來,交來興兒買了一隻燒鴨 ,兩隻雞,一錢銀子下飯,一罈金華酒 ,一瓶白酒 ,一錢銀子裹餡涼糕,交來興兒媳婦整理端正。金蓮對著月娘說:「大姐姐,那日鬬牌嬴了陳姐夫三錢銀子。李大姐又添七錢,今治了東道兒,請姐姐在花園裏吃。」吳月娘就同孟玉樓、李嬌兒、孫雪娥、大姐、桂姐,先在捲棚內吃了一回。然後拿了酒菜兒,往山子上,一個最高的臥雲亭兒上,那裏下棋投壺耍子。孟玉樓便與李嬌兒、大姐、孫雪娥都往翫花樓上去,凭欄杆,望下著那山子前面,牡丹畦、芍藥圃、海棠軒、薔微架、木香棚、玫瑰樹,端的有四時不謝之花,八節長春之景。觀了一回,下來。小玉、迎春卻在臥雲亭上,侍奉月娘斟酒下菜。月娘猛然想起:「今日倒不請陳姐夫來坐?」大姐道:「爹又使他今日往門外徐家催銀子去了,也待好來也。」不一時陳經濟來到,穿著玄色練絨紗衣,腳下涼鞋淨襪,頭上纓子瓦楞帽兒,金簪子,向月娘眾人作了揖,就拉過大姐,一處坐下。向月娘說:「徐家銀子討了來了。共五封,二百五十兩,送到房裡,玉筲收了。」于是穿杯換盞,酒過數巡,各添春色。月娘與李嬌兒、桂姐三個下棋;玉樓、李瓶兒、孫雪娥、大姐、經濟便向各處遊翫觀花草。惟有金蓮在山子後,那芭蕉叢深處,將手中白紗團扇兒,且去撲蝴蝶為戲。不防經濟驀地走在背後,猛然叫道:「五娘,你不會撲蝴蝶,我等與你撲。這蝴蝶就和你老人家一般,有些毬子心腸,滾上滾下的走滾大。」那金蓮扭回粉頸,科睨秋波,對著陳經濟笑罵道:「你這少死的賊短命!誰要你撲?將人來聽見,敢待死也!我曉得你也不怕死了,搗了幾鍾酒兒,在這裡來鬼混!」因問:「你買的汗巾兒,怎了?」那經濟笑嬉嬉,向袖子中取出,一手遞與他。說道:「六娘的都在這裡了。」又道:「汗巾兒稍了來,你把甚來謝我?」于是把臉子挨向他身邊。被金蓮只一推。不想李瓶兒抱著官哥兒,并奶子如意兒跟著,從松墻那邊走來。見金蓮和經濟兩個,在那裡嬉戲撲蝴蝶,李瓶兒這裡,趕眼不見,兩三步就鑽進去山子裡邊。猛叫道:「你兩個撲個蝴蝶兒,與官哥兒耍子!」慌的那潘金蓮恐怕李瓶兒瞧見,故意問道:「陳姐夫與了汗巾子不曾?」李瓶兒道:「他還沒與我哩!」金蓮道:「他剛纔袖著,對著大姐姐,不好與咱的,悄悄遞與我了。」于是兩個坐在花臺石上打開。兩個分了。金蓮見官哥兒脖子裡,圍著條白挑線汗巾子,手裡把著個李子,往口裏吮。問道:「是你的汗巾子?」李瓶兒道:「是剛才他大媽媽,見他口裏吮李子,流下水,替他圍上這汗巾子。」兩個只顧坐在芭蕉叢下。李瓶兒說道:「這答兒裡,到且是蔭涼,咱在這裡,坐一一回兒罷!」因使如意兒:「你去叫迎春屋裡取孩子的小枕頭兒,帶涼席兒,放他在這里,俏俏兒就取骨牌來,我和五娘在這裡抹回牌兒,你就在屋裡看罷。」如意兒去了。不一時,迎春取了枕席并骨牌來。李瓶兒鋪下席,把官哥兒放在小枕頭兒上俏著,交他頑耍,他便和李金蓮抹牌。抹了一回,交迎春往屋裡,燉一壺好茶來。不想孟玉樓在臥雲亭欄杆一看見,點手兒叫李瓶兒,說:「大姐姐叫你說句兒?就來。」那李瓶兒撇下孩子,交金蓮看著:「我就來。」那金蓮記掛經濟在洞兒裡,那裡又去顧那孩子。趕空兒兩三步,走入洞門首,交經濟說:「沒人,你出來罷!」經濟就叫婦人住去瞧蘑菇:「裏面長出這些大頭蘑菇來了。」哄的婦人入到洞裡,就折鐵腿跪著,要和婦人雲雨。兩個正接著親嘴,也是天假其便,李瓶兒走到亭子上,吳月娘說:「孟三姐和桂姐投壺輸了,你來替他投兩壺兒。」李瓶兒道:「底下沒人看孩子哩!」玉樓道:「左右有六姐在那裡,怕怎的?」月娘道:「孟三姐,你去替他看看罷!」李瓶兒道:「三娘累你。亦發抱了他來罷。」交小玉:「你去,就抱他的席和小枕頭兒來。」那小玉和玉樓走到芭蕉叢取,孩子便躺在席上,登手登腳的怪哭,並不知金蓮在那裡。只見傍邊大黑貓,見人來,一滾煙跑了。玉樓道:「他五娘那裡去了?耶嚛!耶嚛!把孩子丟在這裡,吃貓諕了他了!」那金蓮便從傍邊雪洞兒裡鑽出來,說道:「我在這裡淨了淨手,誰往那裡去來?那裡有貓來諕了他,白眉赤眼兒的!」那玉樓也更不往洞裏看,只顧抱了官哥兒拍哄著他,往臥雲亭兒上去了。小玉拏著枕席的去了。金蓮恐怕他學舌,隨屁股也跟了來。月娘問:「孩子怎的哭?」玉樓道:「我去時,不知是那裡一個大黑貓,蹲在孩子頭根前。月娘說:「乾淨諕著孩兒!」李瓶兒道:「他五娘看著他哩。」玉樓道:「六姐往洞兒裡淨手去來。」金蓮走上來說玉樓:「你怎的恁白眉赤眼兒的,我在那裡討個貓來?他想必餓了,要奶吃哭,就賴起人了!」李瓶兒見迎春拏上茶來。就使他叫奶子來喂哥兒奶。那陳經濟見無人,從洞兒鑽出來,順著松墻兒,抹轉過捲棚,一直行前邊角門往外去了。正是:

「雙手劈開生死路,  一身跳出是非門。」

月娘見孩子不吃奶,只是哭,分付李瓶兒:「你抱他到屋裡,好好打發他睡罷。」于是也不吃酒,眾人都散了。原來陳經濟也不曾與潘金蓮得手,做為燕侶鶯儔,只得做了個蜂頭花嘴兒,事情不巧。歸到前邊廂房中,有些咄咄不樂。正是:

「無可奈何花落去,  似曾相識燕歸來。」

有折桂令為證:

「我見他戴花枝,笑撚花枝。朱唇上,不抹胭脂,似抹胭脂;逐日相逢,似有情兒,未見情兒。欲見許,何曾見許?似推辭,未是推辭!約在何時?會在何時?不相逢,他又相思;既相逢,我反相思。」

畢竟未知後來何如?且聽下回分解:
