%# -*- coding: utf-8 -*-
%!TEX encoding = UTF-8 Unicode
%!TEX TS-program = xelatex
% vim:ts=4:sw=4
%
% 以上设定默认使用 XeLaTex 编译,并指定 Unicode 编码,供 TeXShop 自动识别

%第八十七回 
\chapter{王婆子貪財受報\KG 武都頭殺嫂祭兄}


\begin{showcontents}{}



「平生作善天加福,  若是剛強定禍殃,

舌為柔和終不損,  齒因堅硬必遭傷;

杏桃秋到多零落,  松栢冬深愈翠蒼,

善惡到頭終有報,  高飛遠走也難藏。」

話說陳經濟僱頭口起身,叫了張團練一個伴當跟隨,早上東京去不題。都表吳月娘打發潘金蓮出門,次日使春鴻叫薛嫂兒來,要賣秋菊。這春鴻正走到大街,撞見應伯爵,叫住問春鴻:「你往那裡去?」春鴻道:「家中大娘使小的叫媒人薛嫂兒來。」伯爵問:「叫媒人做甚麼?」春鴻道:「賣五娘房裡秋菊丫頭。」伯爵又問:「你五娘為甚麼打發出來,在王婆子家住着,說要尋人家嫁人,端的有此話麼?」這春鴻便如此這般:「因和俺姐夫有些說話,大娘知道了,先打發了春梅小大姐,然後打了俺姐夫一頓,趕出往家去了。昨日纔打發出俺五娘來。」伯爵聽了,點了點頭兒,說道:「原來你五娘和你姐夫有楂兒,看不出人來!」又向春鴻說:「孩兒,你爹已是死了,你只顧還在他家做甚麼?終是沒出產。你心裡還要歸你南邊去?這裡尋個人家跟罷,心下如何?」春鴻道:「便是這般說老爹已是沒了,家中大娘好不嚴緊。各處買賣都收了,房子也賣了,琴童兒、畫童兒多走了,也攬不過這許多人口來。小的待回南邊去,又沒順便人帶去。這城內尋個人家跟,又沒個門路。」伯爵道:「傻孩兒!人無遠見,安身不牢。千山萬水,又往南邊去做甚?誰人帶去?你肚裏會幾句唱,愁這城內尋不出主兒來答應?我如今舉保個門路與你。如今大街坊張二老爹家,有萬萬貫家財,百間房屋,見頂補了你爹在提刑院做掌刑千戶。如今你二娘,又在他家做了二房。我把你送到他宅中答應他。他見你會唱南曲,管情一箭就上垛,留下你做親隨大官兒。又不比你在這家裡,他性兒又好,年紀小小,又倜儻,又好愛好,你就是個有造化的。」這春鴻扒到地下,就磕了個頭:「有累二爹!小的若見了張老爹,得一步之地,買禮與二爹磕頭。」伯爵一把手拉着春鴻說:「傻孩兒,你起來,我無有個不作成人的,肯要你謝?你那得錢兒來?」春鴻道:「小的去了,只怕家中大娘找尋小的怎了?」伯爵道:「這個不打緊,我問你張二老爹討個帖兒,封一兩銀子與他家。他家銀子不敢受,不怕把你不雙手兒送了去。」說畢,春鴻往薛嫂兒家,叫了薛嫂兒,見月娘,領秋菊出來,只賣了五兩銀子,交與月娘,不在話下。都說應伯爵領春鴻到張二官宅裡見了。張二官見他生的清秀,又會唱南曲,就留下他答應。使拏拜帖兒,封了一兩銀子,往西門慶家討他箱子。那日吳月娘家中,正陪雲離守娘子范氏吃酒。先是雲離守襲過哥雲將參將指揮,補在清河左衛做同知。見西門慶死了,吳月娘守寡,手裡有東西,就安心有垂涎圖謀之意。此日正買了八盤羹果禮物,來看月娘。見月娘生了孝哥,范氏房內亦有一女,方兩月兒,要與月娘結親。那日吃酒,遂兩家割衫襟,做了兒女親家,留下一雙金環為定禮。聽見玳安兒拏進張二官府帖兒,并一兩銀子,說:「春鴻投在他家答應去了。使人來討他箱子衣服。」月娘見他現做提刑官,不好不與他,銀子也不曾收,只得把箱子與將出來。初時應伯爵對張二官說:「西門慶第五娘子潘金蓮,生的標致,會一手琵琶,百家詞曲,雙陸象棋,無不通曉,又會寫字。因為年小守不的,又和他大娘子合氣,今打發出來,在王婆家聘嫁人。」這張二官一替兩替,使家人拏銀子往王婆家相看。王婆只推他大娘子分付,不倒口要一百兩銀子。那人來回講了幾遍,還到八十兩上,王婆還不吐口兒。落後春鴻到他宅內,張二官聽見春鴻說,婦人在家養着女婿,因為如此,打發出來。這張二官就不要了。對着伯爵說:「我家現放着十五歲未出幼兒子,上學攻書,要這樣婦人來家做甚?」又續見李嬌兒說,金蓮當初用毒藥擺佈死了漢子,被西門慶占將來家,又偷小廝,把第六個娘子生了兒子,娘兒兩人,生生吃他害殺了。以此張二官就不要了。話分兩頭,都說春梅賣到守備府中,守備見他生的標致伶俐,舉止動人,心中大喜。與了他三間房住,手下使一個小丫鬟,就一連在他房中歇了三夜三日,替他裁了兩套衣裳。薛嫂兒去,賞了薛嫂五錢銀子。又買了個使女扶侍他,立他做二房。大娘子一目失明,吃長齋念佛,不管閑事。還有生姐兒的孫二娘,在東廂房住。春梅在西廂房,各處鑰匙,都教他掌管,甚是寵愛。他一日,聽薛嫂兒說,潘金蓮出來了,在王婆家聘嫁。這春梅晚夕,啼啼哭哭,對守備說:「俺娘兒兩個在一處廝守這幾年,他大氣兒不曾呵着我,把我當親女兒一般看承。自知拆散開了,不想今日他也出來了!你若肯娶他將來,俺娘兒們還在一處過好日子。」又說:「他怎的好模樣兒,諸家詞曲都會,又會彈琵琶。聰明俊俏,百伶百俐!屬龍的,今纔三十二歲兒。他若來,奴情願做第三的也罷。」於是把守備念轉了,使手下親隨張勝、李安,封了兩方手帕、二錢銀子,往王婆家相看。果然生的好個出色的婦人。王婆開口指稱:「他家大娘子,要一百兩銀子。」張勝、李安講了半日,還了八十兩,那王婆還不肯。走來回守備,又添了五兩,復使二人拏着銀子和王婆子說。王婆子只是假推:「他大娘子不肯,不轉口兒要一百兩,媒人錢要不要罷,天也不使空人!」這張勝、李安,只得又拏回銀子來稟守備。丟了兩日,怎禁這春梅晚夕哭哭啼啼:「好歹再添幾兩銀子娶了來和奴做伴兒,死也甘心!」守備見春梅只是哭泣,只得又差了大管家周忠,同張勝、李安氈包內拏着銀子,打開與婆子看,又添到九十兩上。婆子越發張致起來,說:「若九十兩到不的,如今提刑張二老爹家抬的去了。」這周忠就惱了,分付李安把銀子包了,說道:「三隻蟾沒處尋,兩腳老婆愁那裡尋不出來?這老淫婦連人也不識,你說那張二官府怎的?俺府裡老爺管不着你?不是新娶的小夫人再三在老爺跟前說念,要娶這婦人,平白出這些銀子要他何用?」李安道:「勒掯俺兩番三次來回走,賊老淫婦,越發鸚哥兒了!」拉周忠說:「管家哥,咱去來。到家回了老爺,好不好教牢子拏去,拶與他一頓拶子。這婆子終是貪着陳經濟那口食,由他罵,只是不言語。二人到府中回稟守備說:「已添到九十兩,還不肯。」守備說:「明日兌與他一百兩,拏轎子抬了來罷。」周忠說:「爺就添了一百兩,王婆子還要五兩媒人錢。且丟他兩日。他若張致,拏到府中,且拶與他一頓拶子,他纔怕!」看官聽說:大段潘金蓮生有地兒,死有處。不爭被周忠說這兩句,有分交,這婦人從前做過事,今朝沒興一起來!有詩為證:

「人生雖未有前知,  禍福因由更問誰;

善惡到頭終有報,  只爭來早與來遲。」

按下一頭,都說一人。單表武松自從西門慶墊發孟州牢城充軍之後,多虧小管營施恩看顧。次後施恩與蔣門神爭奪快活林酒店,被蔣門神打傷,央武松出力,反打了蔣門神一頓。不想蔣門神妹子玉蘭,嫁與張都監為妾,賺武松去,假捏賊情,將武松拷打,轉又發安平寨充軍。這武松走到飛雲浦,又殺了兩個公人,復回身殺了張都監、蔣門神全家老小,逃躲在施恩家。施恩寫了一封書,皮箱內封了一百兩銀子,教武松到安平寨與知寨劉高,教看顧他。不想路上聽見太子立東官,放郊天大赦,武松就遇赦回家。到清河縣下了文書,依舊在縣當差,還做都頭。來到狐中,尋見上鄰姚二郎,交付迎兒。那時迎兒已長大十九歲了,收攬來家,一處居住,打聽西門慶已死,「你嫂子出來了,如今還在王婆狐家,早晚嫁人。」這漢子聽了,舊仇在心。正是:

「踏破鐵鞋無處覓,  得來全不費工夫!」

次日,裏幘穿衣,逕出門來到王婆門首。金蓮正在簾下站着,見武松來,連忙閃入裡間去。武松掀開簾子,來問:「王媽媽在家?」那婆子正在磨上掃麵,連忙出來應道:「是誰叫老身?」見是武松,道了萬福。武松深深唱諾。婆子道:「武二哥,且喜!幾時回家了?」武松道:「遇赦回家,昨日纔到。一向多累媽媽看家,改日相謝。」婆子笑嘻嘻道:「武二哥比舊時保養,鬍子渣兒也有了,且是好身量,在外邊又學得這般知禮!」一面讓坐,點茶吃了。武松道:「我有一庄事和媽媽說。」婆子道:「有甚事?武二哥只顧說。」武松道:「我聞的人說,西門慶已是死了,我嫂子出來,在你老人家這裡居住。敢煩媽媽對嫂子說,他若不嫁人便罷,若是嫁人,如今迎兒大了,娶得嫂子家去,看管迎兒,早晚招個女婿,一家一計過日子,庶不教人笑話。」婆子初時還不吐口兒,便道:「他是在我這裡,倒不知嫁人不嫁人?」次後聽見武松重謝他,便道:「等我慢慢和他說。」那婦人便簾內聽見武松言語,要娶他看管迎兒;又見武松在外,出落得長大,身材胖了,比昔時又會說話兒。舊心不改,心下暗道:「這段姻緣,還落在他家手裡!」就等不得王婆叫他,自己出來,向武松道了萬福,說道:「既是叔叔還要奴家去看管迎兒,招女婿成家,可知好哩!」王婆道:「又一件,如今他家大娘子,要一百兩雪花銀子纔嫁人。」武松道:「如何要這許多?」王婆道:「西門大官人當初為了他使了許多,就打恁個銀人兒也勾了!」武松道:「不打緊,我既要請嫂嫂家去,就使一百兩也罷。另外破五兩銀子,謝你老人家。」這婆子聽見,喜歡的屁滾尿流,沒口說:「還是武二哥知禮,這幾年江湖上見的事多,真是好漢!」婦人聽了此言,走到屋裡,又濃點了一盞瓜仁泡茶 ,雙手遞與武松吃了。婆子問道:「如今他家要發脫的緊,又有三四處官戶人家爭着娶,都回阻了,價錢不兌。你這銀子,作速些便好。常言:『先下米,先吃飯。』千里姻緣着線牽。休要落在別人手內。」婦人道:「既要娶奴家,叔叔上緊些。」武松便道:「明日就來兌銀,晚夕請嫂嫂過去。」那王婆還不信武松有這些銀子,胡亂答應去了。到次日,武松打開皮箱,拏出小管營施恩與知寨劉高那一百兩銀子來,又另外包了五兩碎銀子,走到王婆家,拏天平兌起來。那婆子看見白晃晃擺了一卓銀子,口中不言,心內暗道:「雖是陳經濟許下一百兩,上東京去取,不知幾時到來?仰着合着,我見鐘不打,都打籌鐘?」又見五兩謝他,連忙收了。拜了又拜,說道:「還是武二哥曉禮,知人甘苦!」武松道:「媽媽收了銀子,今日就請嫂嫂過門。」婆子道:「武二哥且是好急性,門背後放花兒,你等不到晚了!也待我往他大娘子那裡交了銀子,纔打發他過去。」又道:「你今日帽兒光光,晚夕做個新郎。」那武松緊着心中不自在,那婆子不知好歹,又徯落他。打發武松出門,自己尋思:「他家大娘子自交我發脫,又沒和我則定價錢。我今胡亂與他一、二十兩銀子,滿纂的就是了。綁着鬼,也落他多一半養家。」一面把銀鑿下二十兩銀子,往月娘家裡交割明白。月娘問:「甚麼人家娶了去了?」王婆道:「兔兒沿山跑,還來歸舊窩!嫁了他小叔,還吃舊鍋裡粥去了!」月娘聽了,暗中跌腳;常言:仇人見仇人,分外眼睛明。與孟玉樓說:「往後死在他小叔子手裡罷了!那漢子殺人不斬眼,豈肯干休?」不說月娘家中嘆息,都表王婆交了銀子到家,下午時,教王潮兒先把婦人箱籠卓兒送過去。這武松在家,又早收拾停當。打下酒肉,安排下菜蔬。晚上婆子領婦人進門,換了孝,戴着新䯼髻,身穿紅衣服,搭着蓋頭。進門來,見明間內明亮亮點着燈燭,武大靈牌供養在上面,先自有些疑忌。由不的髮似人揪,肉如鈎搭。進入門來,到房中。武松分付迎兒把前門上了拴,後門也頂了。王婆見了,說道:「武二哥,我去罷,家裡沒人。」武松道:「媽媽請進房裡吃盞酒。」武松教迎兒拿菜蔬擺在卓上,須臾盪上酒來,請婦人和王婆吃酒。那武松也不讓,把酒斟上,一連吃了四、五碗酒。婆子見他吃得惡,便道:「武二哥,老身酒勾了,放我去,你兩口兒自在吃盞兒罷!」武松道:「媽媽且休得胡說,我武二有句話問你。」只聞颼的一聲響,向衣底掣出一把二尺長薄背厚刃扎刀子來,一隻手籠着刀靶,一隻手按住俺心,便睜圓怪眼,倒豎剛鬚,便道:「婆子休得吃驚!自古冤有頭,債有主,休推睡裡夢裡,我哥哥性命都在你身上。」婆子道:「武二哥,夜晚了,酒醉拏刀弄杖,不是耍處!」武松道:「婆子休胡說!我武二就死也不怕!等我問了這淫婦,慢慢來問你這老猪狗。若動一動步兒,身上先吃我王七刀子。」一面回過臉來,看着婦人罵道:「你這淫婦聽着,我的哥哥怎生謀害了?從實說來,我便饒你。」那婦人道:「叔叔如何冷鍋中豆兒爆,好沒道理!你哥哥自害心疼病死了,干我甚事?」說猶未了,武松把刀子忔楂的插在卓子上,用左手揪住婦人雲髻,右手匹胸提住,把卓子一腳踢番,碟兒盞兒都落地打得粉碎。那婦人能有多大氣脉?被這漢子隔卓子輕輕提將過來,拖出外間靈卓子前。那婆子見頭勢不好,便去奔前門走;前門又上了拴。被武松大扠步趕上,揪番在地,用腰間纏帶解下來,四手四腳綑住,如猿猴獻果一般,便脫身不得,口中只叫:「都頭不消動怒,大娘子自做出來,不干我事!」武松道:「老猪狗!我都知了,你賴那個?你教西門慶那廝墊發我充軍去,今日我怎生又回家了?西門慶那廝都在那裡?你不說時,先剮了這個淫婦,後殺愀這老猪狗!」提起刀來,便望那婦人臉上撇兩撇。婦人慌忙叫道:「叔叔且饒,放我起來,等我說便了。」武松一提,提起那婆娘,旋剝淨了,跪在靈卓子前。武松喝道:「淫婦快說!」那婦人諕的魂不附體,只得從實招說。將那時收簾子打了西門慶起,并做衣裳入馬通姦,後秉的踢傷了武大心坎,用何下藥,王婆怎地教唆下毒,撥置燒化,又怎的娶到家去,一五一十,從頭至尾說了一遍。王婆聽見,只是暗地叫苦,說:「傻才料,你老實說了,都教老身怎的支吾?」這武松一面就靈前一手揪着婦人,一手澆奠了酒,把紙錢點着,說道:「哥哥,你陰魂不遠,今日武二與你報仇雪恨!」那婦人見頭勢不好,纔待大叫,被武松向爐內撾了一把香灰塞在他口,就叫不出來了。然後劈腦揪番在地,那婦人掙扎,把䯼髻簪環都滾落了。武松恐怕他掙扎,先用油靴只顧踢他肋肢,後用兩隻腳踏他兩隻胳膊,便道:「淫婦自說你伶俐,不知你心怎麼生着?我試看一看!」一面用手去攤開他胸脯。說時遲,那時快,把刀子去婦人白馥馥心窩內,只一剜,剜了個血窟礲,那鮮血就邈出來。那婦人就星眸半閃,兩隻腳只顧登踏。武松口噙着刀子,雙手去斡開他胸脯,撲扢的一聲,把心肝五臟生扯下來,血瀝瀝供養在靈前。後方一刀割下頭來,血流滿地。迎兒小女在旁看見,諕的只掩了瞼。武松這漢子,端的好狠也!可憐這婦,正是:

「三寸氣在千般用,  一日無常萬事休!」

亡年三十二歲。但見:

「手到處青春喪命,刀落時紅粉亡身。七魄悠悠,已赴森羅殿上;三魂渺渺,應歸無間城中。星眸緊閉,直挺挺屍橫光地下,銀牙半咬,血淋淋頭在一邊離。好似初春大雪壓折金線柳,臘月狂風吹折玉梅花。這婦人綿媚不知歸何處,芳魂今夜落誰家?」

古人有詩一首,單悼金蓮死的好苦也:

「堪悼金蓮誠可憐,  衣服脫去跪靈前,

誰知武二持刀殺,  只道西門綁腿頑;

往事堪嗟一場夢,  今身不值半文錢,

世間一命還一命,  報應分明在眼前。」

當下武松殺了婦人,那婆子看見,大叫:「殺人了!」武松聽見他叫,向前一刀,也割下頭來,拖過屍首,一邊將婦人心肝五臟,用刀插在樓後房簷下。那時也有初更時分,倒扣迎兒在屋裏。迎兒道:「叔叔,我也害怕。」武松道:「孩兒,我顧不得你了!」武松跳過王婆家來,還要殺他兒子王潮兒。不想王潮合當不該死,聽見他娘這邊叫,就知武松行兇。推前門不開,叫後門也不應。慌的走去街上叫保甲,那兩鄰明知武松兇惡,誰敢向前?武松跳過墻來,到王婆房內,只見點着燈,房內一人也沒有。一面打開王婆箱籠,就把他衣服撒了一地。那一百兩銀子,止交與吳大娘二十兩,還剩了八十五兩,并些釵環首飾,武松一股皆休,都包裹了。提了朴刀,越後墻,趕五更挨出城門,投十字坡張青夫婦那裡躲住,做了頭佗,上梁山為盜去了。正是:

「平生不作縐眉事,  世上應無切齒人。」

畢竟未知後來如何,且聽下回分解:





\end{showcontents}


