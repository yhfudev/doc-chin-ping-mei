%# -*- coding: utf-8 -*-
%!TEX encoding = UTF-8 Unicode
%!TEX TS-program = xelatex
% vim:ts=4:sw=4
%
% 以上设定默认使用 XeLaTex 编译,并指定 Unicode 编码,供 TeXShop 自动识别

%第九十二回 
\chapter{陳經濟被陷嚴州府\KG 吳月娘大鬧授官廳}

「暑往寒來春復秋,  夕陽西下水東流,

雖然富貴皆由命,  運去貧窮亦自由;

事遇機關須進步,  人逢得意早回頭,

將軍戰馬今何在,  野草閑花滿地愁。」

話說當日李衙內打了玉簪兒一頓,即時叫了陶媽媽來領出賣了八兩銀子,買了箇十八歲使女,名喚滿堂兒上竈,不在話下。卻表陳經濟自從西門大姐來家,交還了許多床帳粧奩箱籠家火。三日一場嚷,五日一場鬧,問他孃張氏要本錢做買賣。他母舅張團練來問他母親借了五十兩銀子,復謀管事。被他吃醉了,往在張舅門上罵嚷。他張舅受氣不過,另問別處借了銀子,幹成管事,還把銀子交還將來。他母親張氏著了一場重氣,染病在身,日逐臥床不起,終日服藥,請醫調治。吃他逆毆不過,兌出兩百兩銀子交他。陳定在家門首,打開兩間房子,開布舖做買賣。逐日結交朋友陸三郎、楊大郎,狐朋狗黨,在舖中彈琵琶、抹骨牌、打雙陸、吃半夜酒,看看把本錢弄下去了。陳定對張氏說:「他每日飲酒花費。」張氏聽信陳定言語,不托他。經濟反說陳定染布去,剋落了錢,把陳定兩口兒攆出來外邊居住,卻搭了楊大郎做夥計。這楊大郎名喚楊先彥,綽號為鐵指甲,專一糶風賣雨,架謊鑿空,撾著人家本錢就使。他祖貫係沒州脫空縣拐帶村無底鄉人氏。他父親叫做楊不來,母親白氏。他兄弟叫楊二風。他師父是崆峒山拖不洞火龍庵精光道人,那里學的謊。他渾家是沒驚著小姐,生生吃謊諕死了。他許人話如捉影撲風,騙人財似探囊取物。這經濟問孃又要出二百兩銀子來添上,共湊了五百兩銀子,信著他往臨清販布去。這楊大郎到家收拾行李,沒底兒褡褳,裝著些軟斯金榆錢兒,拏一張黑心雕弓,騎一匹白眼龍馬,跟著經濟從家中起身,前往臨清馬頭上尋缺貨去。三里抹過沒州縣,五里來到脫空村,有日到于臨清。這臨清閘上,是箇熱鬧繁華大馬頭去處。商賈往來,船隻聚會之所,車輛輻輳之地。有三十二條花柳巷,七十二座管絃樓。這經濟終是年小後生,被這鐵指甲楊大郎領著遊娼樓,串酒店,每日睡睡,終宵蕩蕩,貨物到販得不多。因走在一娼樓館上,見了一箇粉頭,名喚馮金寶,生的風流俏麗,色藝雙全。問:「青春多少?」鴇子說:「姐兒是老身親生之女,止是他一人掙錢養活。今年青春纔交二九一十八歲。」經濟一見,心目蕩然,與了鴇子五兩銀子房金,一連和他歇了幾夜。楊大郎見他愛這粉頭,留連不捨,在旁花言說念,就要娶他家去。鴇子開口要銀一百五十兩,講到一百兩上,兌了銀子,娶到來家,一路上抬著。楊大郎和經濟押著貨物車走,一路上揚鞭走馬,那樣歡喜!正是:

「多情燕子樓,  馬道空回首;

載得武陸春,  陪作鸞鳳友。」

他孃張氏見經濟貨到,販得不多,把本錢到娶了一箇唱的來家,又著了口重氣,嗚呼哀哉,斷氣身亡。這經濟不免買棺裝殮,念經做七。停放了一七光景,發送出門,祖塋合葬。他母舅張團練看他孃面上,亦不和他一般見識。這經濟坟上覆墓回來,把他孃正房三間,中間供樣靈位,那兩間收拾與馮金寶住,大姐到住著耳房。又替馮金寶買了丫頭重喜兒伏侍。門前前楊大郎開著舖子,家里大酒大肉,買與唱的吃。每日只和唱的睡,把大姐丟著不去瞅睬。一日,打聽孟玉樓嫁了李知縣兒子李衙內,帶過許多東西去。三年任滿,李知縣陞在浙江嚴州府做了通判,領憑起身,打水路赴任去了。這陳經濟因想起昔日在花園中,拾了孟玉樓那根簪子,吃醉又被金蓮所得,落後還與了他收到如今。就把這根簪子做箇見證,把物趕上嚴州去,只說玉樓先與他有了姦,與了他這根簪子,不合又帶了許多東西,嫁了李衙內,都是昔日楊戩寄放金銀箱籠,應沒官之物。那李通判一箇文官,多大湯水?聽見這箇利害口聲,不怕不教他兒子雙手把婆奉與我。我那時取將來家,與馮金寶又做一對兒,落得好受用!正是:

「計就月中擒玉兔,  謀成日裡捉金鳥。」

經濟不來到好,此這一來,正是:

「失曉人家逢五道,  溟泠餓鬼撞鍾馗。」

有詩為證:

「趕到嚴州訪玉人,  人心難忖是石沈;

侯門一旦深如海,  從此蕭郎落陷坑。」

卻說一日陳經濟打點他孃箱中,尋出一千兩金銀。留下一百兩與馮金寶家中盤纏。把陳定復叫進來看家,并門前舖子發賣零碎布疋。與他楊大郎,又帶了家人陳安,押著九百兩銀子,從八月中秋起身,前往湖州販了半船絲綿紬絹,來到清江浦江口馬頭上,灣泊住了船隻。投在箇店主人陳二店內,夜間點上燈光,交陳二郎殺雞取酒,與楊大郎共飲。飲酒中間,和楊大郎說:「夥計,你暫且看守船上貨物,在二郎店內略住數日。等我和陳安拏些人事禮物,往浙江嚴州府,看家姐嫁在府中,多不上五日,少只三日期程就來。」楊大郎道:「哥去只顧去,兄弟情愿店中等候哥到日,一同起身。」這陳經濟千不合萬不合和陳安身邊帶了些銀兩,人事禮物。有日取路逕到嚴州府,進入城內,投在寺中安下。打聽李通判到任一箇月,家小船隻,纔到三日光景。這陳經濟不敢怠慢,買了四盤禮物、兩疋紵絲尺頭,兩罈酒,陳安押著;他便揀選衣帽齊整,眉目光鮮,逕到府衙內前與門吏作揖道:「報一聲,說我是通判李老爹衙內,新娶娘子的親孟二舅來探望。」這門吏聽了,不敢怠慢,隨即稟報進去。衙內正在書房中看書,聽見是婦人兄弟,令左右先把禮物抬進來,一面忙整衣冠,道:「有請。」把陳經濟請入府衙廳上敘禮,分賓主坐下,說道:「前日做親之時,怎的不會二舅?」經濟道:「在下因在川廣販貨,一年方回。不知家姐嫁與府上,有失親近。今日敬備薄禮,來看看家姐。」李衙內道:「一向不知,失禮,恕罪恕罪!」須臾,茶湯已罷。衙內令左右把禮帖并禮物取進去:「對你娘說,二舅來了。」孟玉樓正在房中坐的,只聽小門子進來報說:「孟二舅來了。」玉樓道:「一二年不曾回家,再有那箇孟舅?莫不是我二哥孟銳來家了,千山萬水來看我?」只見伴當拏進禮物和帖兒來,上面寫著眷生孟銳。就知是他兄弟,一面道:「有請。」令蘭香收拾後堂乾淨。玉樓裝點打扮,伺候出見。只見衙內讓進來。玉樓在簾內觀看,可霎作怪,不是他兄弟,卻是陳姐夫:「他來做甚麼?等我出去,見他怎的說話?常言:『親不親,故鄉人;美不美,鄉中水。』雖然不是我兄弟,也是我女婿人家。」一面整裝出來拜見。那經濟說道:「一向不知姐姐嫁在這裡,沒曾看得。」還說得這句,不想門子來請衙內,外邊有客來了。這衙內分付玉樓管待二舅,就出去待客去了。玉樓見經濟磕下頭,連忙還禮,說道:「姐夫免禮,那陣風兒刮你到此處?」敘畢禮數,讓坐,叫蘭香看茶出來。吃了茶,彼此敘了些家常話兒。玉樓因問:「大姐好麼?」經濟就把從前西門慶家中出來,并討箱籠的一節話,告訴玉樓。玉樓又把清明節上坟,在永福寺遇見春梅在金蓮坟上燒布的話,告訴他。又說:「我那時在家中,也常勸你大娘,疼女兒,就疼女婿;親姐夫,不曾養活了外人。他聽小人言語,把姐夫打發出來。落後姐夫討箱子,我就不知道。」經濟道:「不瞞你老人家說,我與六姐相交,誰人不知?生生吃他信奴才言語,把他打發出去,纔乞武松殺了!他若在家,那武松有七箇頭八箇膽,敢往你家來殺他?我這仇恨,結的有海來深!六姐死在陰司里,也不饒他!」玉樓道:「姐夫也罷,丟開了手的事!自古冤仇只可解,不可結!」說話中間,丫鬟放下卓兒,擺上酒來,盃盤餚品,堆滿春抬。玉樓斟上一盃酒,雙手遞與經濟說:「姐夫遠路風塵,無事破費,且請一盃兒水酒。」這經濟用手接了,唱了喏,亦斟一盃回奉婦人,敘禮坐下。因見婦人姐夫長姐夫短叫他,口中不言,心內暗道:『這淫婦怎的不認犯?只叫我姐夫?等我慢慢的探他。』當下酒過三巡,餚添五道,彼此言來語去,說得入港。這經濟酒蓋著臉兒,常言:酒情深似海,色膽大如天。見無人在跟前,先丟的幾句邪言說入去,說道:「我兄弟思想姐姐,如渴思漿,如熱思涼!想當初在丈人家,怎的在一處下棋抹牌,同坐雙雙,似背蓋一般!誰承望今日各自分散,你東我西!」玉樓笑道:「姐夫好說。自古清者清,而渾者渾,久而自見。」這經濟笑嘻喜向袖中,取出一包雙人兒的香茶 ,遞與婦人,說:「姐姐,你若有情,可憐見兄弟,吃我這個香茶兒。」說著,就連忙跪下。那婦人登時一點紅從耳畔起,把臉飛紅了!一手把香茶包兒,掠在地下,說道:「好不識人敬重!奴好意遞酒與你吃,到戲弄我起來!」就撇了酒席,往房裡去了。經濟見他不就,一面拾起香茶來,發話道:「我好意來看你,你到變了卦兒!你敢說你嫁了通判兒子好漢子,不采我了!你當初在西門慶家做第三箇小老婆,沒曾和我兩箇有首尾?」因向袖中取出舊時那根金頭銀簪子,拏在手內說:「這箇物是誰人的?你既不和我有姦,這根簪兒怎落在我手裡?上面還刻著玉樓名字!你和大老婆串同了,把我家寄放的八箱子金銀細軟,玉帶寶石東西,都是當朝楊戩寄放應沒官之物,都帶來嫁了漢子。我教你不要謊,到八字八金塀夏兒上和你答話!」玉樓見他發話,拏的簪子,委的他頭上戴的金頭蓮瓣簪兒,昔日在花園中不見,怎的落在這短命手裡?恐怕嚷的家下人知道!須臾變作笑吟吟臉兒,走將出來,一把手拉住經濟說道:「好姐夫,奴鬬你耍子,如何就惱起來?」因觀看左右無人,悄悄說:「你既有心,奴亦有意。」兩箇不由分說,摟著就親嘴。這陳經濟把舌頭似蛇吃燕子一般,就舒到他口裡,交他咂。說道:「你叫我聲親親的姐夫,纔算你有我之心。」婦人道:「且禁聲,只怕有人聽見。」經濟悄悄向他說:「我如今治了半船貨,在清江浦等候。你若肯下顧時,如此這般,到晚夕假扮門子私走出來,跟我上船家去,成其夫婦,有何不可?他一箇文職官,怕是非,莫敢來抓尋你不成?」婦人道:「既然如此,也罷!」約會下:「你今晚在府牆後等著,如有一個金銀細軟,打牆上繫過去,與你接了。然後奴纔扮做門子,打門裡出來,跟你上船去罷。」看官聽說:正是:

「佳人有意,  那怕粉墻高萬丈;

紅粉無情,  總然共坐隔千山!」

當時孟玉樓若嫁得箇痴蠢之人,不如經濟,經濟便下得這箇鍬鐝著。如今嫁箇李衙內,有前程,又是人物風流,青春年少,恩情美滿,他又抅你做甚?休說平日又無連手。這箇郎君,也早合當倒運,就吐實話,泄機與他,到吃婆娘哄賺了。正是:

「花枝葉下猶藏刺,  人心難保不懷毒!」

當下二人會下話,這經濟吃了幾盃酒,少頃,告辭回去。李衙內連忙送出府門,陳安跟隨而去。衙內便問婦人:「你兄弟住那里下處?我明日回拜他去,送些嗄程與他。」婦人便說:「那里是我兄弟,他是西門慶家女婿。如此這般,來抅搭,要拐我出去。奴已約下他,今晚夜至三更,在後牆相等。咱不好將計就計,把他當賊拏下,除其後患如何?」衙內道:「叵耐這廝無端!自古無毒不丈夫,不是我去尋他,他自來送死!」一面走出外邊,叫過左右伴當心腹快手,如此這般,預備去了。這陳經濟不知機變,至半夜三更,果然帶領家人陳安,來府衙後牆下,以咳嗽為號。只聽牆內玉樓聲音,打牆上掠過十條索子去。那邊繫過一大包銀子來。原來是庫內拏的二百兩贓罰銀子。這經濟纔待教陳安拏著走。忽聽一聲梆子響,黑影裡閃出四五條漢,叫聲:「有賊了!」登時把經濟連陳安都綁了,稟知李通判,分付:「都且押送牢裡去,明日問理。」原來嚴州府正堂知府姓徐,名喚徐崶,係陝西臨洮府人氏,庚戍進士,極是箇清廉剛正之人。次日早升堂,左右排兩行官吏。這李通判上去,畫了公座,庫子呈稟賊情事,帶經濟上去,說:「昨夜至三更時分,有先不知名,今知名賊人二名陳經濟、陳安,鍬開庫門鎖鑰,偷出贓銀二百兩,越牆而過,致被捉獲,來見老爺。」徐知府喝令:「帶上來!」把陳經濟并陳安揪簇採擁,驅至當廳跪下。知府見年小清俊,便問:「這廝是那里人氏?因何來我這府衙公廨夜晚做賊,偷盜官庫贓銀數多,有何理說?」那陳經濟只顧磕頭聲冤。徐知府道:「你做賊如何聲冤?」李通判在旁欠身便道:「老先生不必問他,眼見得贓證明白,何不加起刑來?」徐知府即令左右拏下去打二十板。李通判道:「人是苦蟲,不打不成。不然,這賊便要展轉!」當下兩邊皂隸,把經濟、陳安拖番,大板打將下來。這陳經濟口內只罵:「誰知淫婦孟三兒陷我至此,冤哉苦哉!」這徐知府終是黃堂出身官人,聽見這一聲,必有緣故,纔打到十板上,喝令:「住了!且收下監去,明日再問。」李通判道:「老先生不該發落他,常言:『人心似鐵,官法如爐。』從容他一夜不緊,就翻異口詞。」徐知府道:「無妨,吾自有主意。」當下獄卒把經濟、陳安押送監中去訖。這徐知府心中有些疑忌,即喚左右心腹近前,如此這般:「下監中探聽經濟所犯來歷,即便回報。」這幹事人假扮做犯人,和經濟晚間在一〈扌匣〉上睡,問其所以:「我看哥哥青春年少,不是做賊的。今日落在此刑憲,打屈官司!」經濟便說:「一言難盡!小人本是清河縣西門慶女婿。這李通判兒子新娶的婦人孟氏,是俺丈人的小,舊與我姦的。今帶過我家老爺楊戩,寄放十箱金銀寶玩之物來他家,我來此間問他索討,反被他如此這般欺負,把我當賊拏了,苦打成招,不得見其天日,是好苦也!」這人聽了,走來退廳,告報徐知府。知府道:「如何?我說這人聲冤叫孟氏,必有緣故。」到次日升堂,官吏兩旁侍立。這徐知府把陳經濟、陳安提上來,摘了口詞,取了張無事的供狀,喝令釋放。李通判在旁邊不知,還再三說:「老先生,這廝賊情既的,不可放他!」反被徐知府對佐貳官儘力數說了李通判一頓,說:「我居本府正官,與朝廷幹事。不該與你家官報私仇,誣陷平人作賊!你家兒子娶了他丈人西門慶妾孟氏,帶了許多東西,應沒官贓物,金銀箱籠來。他是西門慶女婿,逕來索討前物。你如何假捏賊情,拏他入罪,教我替你家出力?做官養兒養女,也要長大!若然如此,公道何堪!」當廳把李通判數說的滿面羞,垂首喪氣而不敢言。

陳經濟與陳安便釋放出去了。良久。徐知府退廳。這李通判回到本宅,心中十分焦燥。夫人便問:「相公每常退衙,歡天喜地;今日這般心中不快,何說?」那李通判大喝一聲:「你女婦人家,曉得甚麼?養的好不肖子!今日吃徐知府當堂對眾同僚官吏,儘力上數落了我一頓,可不氣殺我也!」夫人慌了,便問:「甚麼事?」李通判即把兒子叫到跟前,喝令左右:「拏大板,氣殺我也!」說道:「你當初為娶這箇婦人來家,今是他家女婿因這婦人帶了許多裝奩金銀箱籠,口口聲聲稱是當朝逆犯楊戩,奇放應沒官之物,來問你要。說你假盜出庫中官銀,當賊情拏他。我道一字不知,反被正宅徐知府,對眾數說了我這一頓!此是我頭一日官未做,你照顧我的!我要你這不肖子何用?」即令左右,雨點般大板打將下來。可憐打得這李衙內皮開肉綻,鮮血迸流。夫人見打得不像模樣,在旁哭泣勸解。孟玉樓又在後廳角門首,掩淚潛聽。當下打了三十大板。李通判分付左右:「押著衙內,即時與我把婦人打發出門,令他任意改嫁,免惹是非,全我名節。」那李衙內心中怎生捨得離異?只顧在父母跟前哭啼哀告:「寧把兒子打死在爹爹跟前,並捨不得婦人!」李通判把衙內用鐵索墩鎖在後堂,不放出去,只要囚禁死他。夫人哭道:「相公,你做官一場,年紀五十餘歲,也只落得這點骨血!不爭為這婦人,你囚死他。往後你年老休官,倚靠何人?」李通判道:「不然,他在這里,須帶累我受人氣!」夫人道:「你不容他在此,打發他兩口兒上原籍真定府家去便了。」通判依聽夫人之言,放了衙內,限三日就起身,打點車輛,同婦人歸棗強縣家裡攻書去了。卻表陳經濟與陳安,出離嚴州府,到寺中取了行李,逕往清江浦陳二店中來尋楊大郎。說:「三日前往府前尋你去,說你監在牢中,他收拾了貨船,起身往家中去。」這經濟未信,向河下不見船隻,撲了空,說道:「這天殺的!如何不等我來,就起身去了?」況新打監中出來,身邊盤纏已無。和陳安不免搭在人船上,把衣衫解當,討吃歸家。忙忙似喪家之犬,急急如漏網之魚。隨路找尋楊大郎,並無蹤跡。那時正值秋暮天氣,樹木凋零,金風搖落,甚是淒涼。有詩八句,單道這秋天行人最苦:

「栖栖芰荷枝,  葉葉梧桐墜,

蛩鳴腐草中,  雁落平沙地;

細雨濕青林,  霜重寒天氣,

不是路行人,  怎曉秋滋味。」

有日經濟到家,陳定正在門首。看見經濟來家,衣衫襤褸,面貌黧黑,諕了一跳。接到家中,問:「貨船到於何處?」經濟氣得半日不言,把嚴州府遭官司一節說了:「多虧正宅徐知府放了我。不然性命難保!今被楊大郎這天殺的,把我貨物不知拐得往那里去了?」先使陳定往他家探聽。他家說:「還不曾來家。」陳經濟又親去問了一遭,並沒下落,心中著慌,走入房來。那馮金寶又和西門大姐,扭南面北。自從經濟出門,兩個合氣,直到如今。大姐便說馮金寶:「拏著銀子錢,轉與他鴇子去了。他家保兒成日來,瞞藏背掖,打酒買肉,在屋裡吃。家中要的沒有,睡到晌午,諸事兒不買,只熬俺們!」馮金寶又說大姐:「成日橫草不拈,豎草不動,偷米換燒餅吃。又把煮的醃肉,偷在舴裡和丫頭元宵兒同吃。」這陳經濟就信了,反罵大姐:「賊不是才料淫婦!你害饞癆饞痞了?偷米出去換燒餅吃!又和丫頭打夥兒偷肉吃!」把元宵兒打了一頓,把大姐踢了幾腳。這大姐急了,趕著馮金寶兒撞頭罵道:「好養漢的淫婦!你偷盜的東西,與鴇子不值了!到學舌與漢子,說我偷米偷肉!犯夜的到拏住巡更的了!教漢子踢我,我和你這淫婦換兌了罷,要這命做甚麼?」這經濟道:「好淫婦,你換兌他?你還不值他箇腳指頭兒里!」也是合當有事,禍便是這般起。於是一把手採過大姐頭髮來,用拳撞腳踢拐子打,打得大姐鼻口流血,半日甦醒過來。這經濟便歸娼的房裡睡去了,由著大姐在下邊房裡,嗚嗚咽咽,只顧哭泣。元宵兒便在外間睡著了。可憐大姐到半夜,用一條索子,懸梁自縊身死。亡年二十四歲。到次日早辰,元宵起來,推裡間不開。上房經濟和馮金寶還在被窩裡。使他丫頭重喜兒來叫大姐了,取木盆洗坐腳,只顧推不開。經濟還罵:「賊淫婦,如何還睡,這咱晚不起來?我這一跺開門進去,把淫婦鬢毛都拔淨了!」重喜兒打窗眼內望裡張看,說道:「他起來了,且在房裡打鞦韆耍子兒哩!」又說:「他提偶戲耍子兒。」只見元宵瞧了半日,叫道:「爹,不好了!俺娘吊在床頂上吊死了!」這小郎纔慌了,和娼的齊起來,跺開房門,向前解卸下來,灌救了半日,那得口氣兒來?原來不知多咱時分,嗚呼哀哉死了!正是:

「不知真性歸何處,  疑在行雲秋水中!」

陳定聽見大姐死了,恐怕連累,先走去西門慶家中,報知月娘。月娘見狀大姐吊死了,經濟娶娼的在家,正是:

「冰厚三尺,  不是一日之寒!」

率領家人、小廝、丫鬟、媳婦七八口,往他家來。見了大姐屍首吊的直挺挺的,哭喊起來。將經濟拏住揪採亂打,渾身錐子眼兒,也不計數。娼的馮金寶躲在床底下,採出來也打了箇臭死。把門窗戶壁都打得七零八落,房中床帳裝奩,都還搬的去了。歸家請將吳大舅、二舅來商議。大舅說:「姐姐,你趁此時咱家死了人不到官,到明日他過不的日子,還來纏要箱籠。人無遠慮,必有近憂。不如到官處斷開了,庶杜絕後患。」月娘道:「哥見得是。」一面寫了狀子。次日,月娘親自出官,來到本縣,投官廳下遞上狀去。原來新任知縣姓霍,名大立,湖廣黃崗縣人氏,舉人出身,為人鯁直,聽見係人命重事,即升廳受狀。見狀上寫著:

「告狀人吳氏,年三十四歲,係已故千戶西門慶妻。狀告為惡婿欺凌孤孀,聽信娼婦,熬打逼死女命,乞憐究治,以存殘喘事:比有女婿陳經濟,遭官事投來氏家潛住數年。平日吃酒行兇,不守本分,打出吊入;是氏懼法,逐離出門。豈期經濟懷恨在家,將氏女西門氏時常熬打,一向含忍。不料伊又娶臨清娼婦馮金寶來家,奪氏女正房居住。聽信唆調,將女百般痛辱熬打,又採去頭髮,渾身踢傷。受忍不過,比及將死。于本年八月廿三日三更時分,方纔將女上吊縊死。若不具告,切思經濟恃逞兇頑,欺氏孤寡,聲言還要持刀殺害等語,情理難容乞賜行拘到案,嚴究女死根因,盡法如律!庶兇頑知警,良善得以安生,而死者不為含冤矣!為此具狀上告

本縣青天老爺    施行。」

這霍知縣在公座上看了狀子,又見吳月娘身穿縞素,腰繫孝裙,係五品職官之妻。生的容貌端莊,儀容閑雅。欠身起來說道:「那吳氏起來,我據看,你也是箇命官娘子。這狀上情理,我都知了。你請回去,不必在這裡。今後只令一家人在此伺候就是了。我就出牌去拏他。」那吳月娘連忙拜謝了知縣出來,坐轎子回家,委付來昭廳下伺候。須臾,批了呈狀,委的兩箇公人,一面白牌,行拘陳經濟、娼婦馮金寶,并兩鄰保甲正身,赴官聽審。這經濟正在家裡亂喪事。聽見月娘告下狀來,縣中差公人發牌來拏他,諕的魂飛天外,魄喪九霄!那馮金寶已被打的渾身疼痛,睡在床上。聽見人拏他,諕的勢不知有無!陳經濟沒高低使錢打發公人吃了酒飯,一條繩子,連娼的都拴到縣裡。左鄰范綱,右鄰孫紀,保甲王寬兒。霍知縣聽見拏了人來,即時升廳。來昭跪在上首,陳經濟、馮金寶一行人跪在階下。知縣看了狀子,便叫經濟上去說:「你是陳經濟?」又問:「那是馮金寶?」那馮金寶道:「小的是馮金寶。」知縣因問經濟:「你這廝可惡!因何聽信娼婦打死西門氏,方今上吊,有何理說?」經濟磕頭告道:「望乞青天老爺察情,小的怎敢打死他?因為搭夥計在外,被人坑陷了資本,著了氣來家,問他要飯吃,他不曾做下飯,委被小的踢了兩腳。他到半夜,自縊身死了。」知縣喝道:「你既娶下娼婦,如何又問他要飯吃?尤說不通!吳氏狀上說,你打死他女兒,方纔上吊,你還不招認?」經濟道:「吳氏與小的有仇,故此誣賴小的,望老爺察情!」知縣大怒說:「他女兒見死了,還推賴那箇?」喝令左右:「拏下去,打二十大板。」提馮金寶上來,拶了一拶,敲一百敲,令公人帶下收監。次日,委典史臧不息帶領吏書保甲鄰人等,前至經濟家出擡出屍首,當場檢驗。身上都有青傷,脖項間亦有繩痕,生前委因經濟踢打傷重,受忍不過,自縊身死。取供具結,填圖解檄,回報縣中。知縣大怒,褪衣又打了經濟、金寶十板。問陳經濟夫毆妻至死者絞罪,馮金寶遞決一百,發回本司院當差。這陳經濟慌了,監中寫出帖子,對陳定說:「把布舖中本錢,連大姐頭面,共湊了一百兩銀子,暗暗送與知縣。」知縣一夜把招卷改了,止問了箇逼令身死,係雜犯,准徒五年,運灰贖罪。吳月娘再三跪門哀告,知縣把月娘叫上去,說道:「娘子,你女兒項上見有繩痕,如何問他毆殺條律?人情莫非忒偏問麼?你怕他後邊纏擾你,我這裡替你取了他杜絕文書,令他再不許上你門就是了。」一面把經濟提到跟前分付道:「我今日饒你一死,務要改過自新,不許再去吳氏家纏擾!再犯到我案下,決然不饒。即便把西門氏買棺裝殮,發送葬埋來回話。我這裡好申文書,往上司去。」這經濟得了箇饒,交納了贖罪銀子,歸到家中抬屍入棺,停放一七,念經送葬,埋城外。前後坐了半箇月監,使了許多銀子,唱的馮金寶也去了,家中所有的都乾淨了,房兒也典了,刪刮刺出箇命兒來,再也不敢聲言丈母了!正是:

「禍福無門人自招,  須知樂極有悲來。」

有詩為證:

「風波平地起蕭墻,  義重因深不可忘;

水溢藍橋應有會,  三星權且作參商。」

畢竟未知後來如何,且聽下回分解:

