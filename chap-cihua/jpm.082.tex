%# -*- coding: utf-8 -*-
%!TEX encoding = UTF-8 Unicode
%!TEX TS-program = xelatex
% vim:ts=4:sw=4
%
% 以上设定默认使用 XeLaTex 编译,并指定 Unicode 编码,供 TeXShop 自动识别

%第八十二回 
\chapter{潘金蓮月夜偷期\KG 陳經濟畫樓雙美}

「記得書齋作會時,  雲蹤雨跡少人知,

晚來鸞鳳栖雙枕,  剔盡銀燈半吐輝;

思往事,夢魂迷,  今宵喜得效于飛,

顛鸞倒鳳無窮樂,  從此雙雙永不離。」

話說潘金蓮與陳經濟自從西門慶孝堂在廂房裡得手之後,兩箇人嚐着甜頭兒,日逐白日偷寒,黃昏送煖。或倚着肩嘲笑,或並坐調情。搯打揪撏,通無忌憚。或有人跟前,不得說話,將心事寫成,搓在紙條兒,內丟在地下。你有話傳與我,我有話傳與你。一日四月天氣,潘金蓮將自己袖的一方銀絲汗巾兒,裹着一箇玉色紗挑線香袋兒,裡面裝安息排草,玫瑰花瓣兒,并一縷頭髮,又着些松栢兒,一面挑着「松栢長青」,一面是「人如花面」八字,封的停當,要與經濟。不想經濟不在廂房內,遂打窗眼內投進去。後經濟開門,進入房中,看見彌封甚厚,打開都是汗巾香袋兒。布上寫一詞,名寄生草:

「將奴這銀絲帕,并香曩寄與他。當中結下青絲髮,松栢兒要你常牽掛,淚珠兒滴寫相思話。夜深燈照的奴影兒孤,休負了夜深潛等茶{艹縻}架。」

這經濟見詞上許他在荼{艹縻}架下,等候私會佳期。隨即封了一柄金湘妃竹扇兒,亦寫一詞在上面答他,袖入花園內。不想月娘正在金蓮房中坐着,這經濟三不知恰進角門,就叫:「可意人在家不在?」這金蓮聽見是他語音,恐怕月娘聽見決撒了,連忙走出來掀起簾子,看見是他,佯做擺手兒,說:「我道是誰來?原來是陳姐夫來尋大姐。大姐剛纔在這裡,和他們往花園亭子摘花兒去了。」這經濟見有月娘在房裡,就把物事暗暗遞與婦人袖了,他就出去了。月娘便問:「陳姐夫來做甚麼?」金蓮道:「他來尋大姐,我回他往花園中去了。」以此瞞過月娘。不久月娘起身回後邊去了。金蓮向袖中取出物事,拆開,都是湘妃竹白紗扇兒一把。上畫一種青蒲,半溪流水。有水仙子一首為證:

「紫竹白紗甚逍遙,綠□青蒲巧製成,金鉸銀錢十分妙。妙人兒堪用著,遮炎天少把風招。有人處常常袖著,無人處慢慢輕搖。休教那俗人見偷了!」

婦人一見其詞,到于晚夕月上時,早把春梅、秋菊兩個丫頭,打發些酒與他吃,關在那邊炕屋睡。然後他便在房中,綠窗半啟,絳燭高燒,收拾床鋪衾枕,薰香澡牝,獨立木香棚下,專等經濟今晚來赴佳期。都說西門大姐那日被月娘請去後邊,聽王姑子宣卷去了。止有元宵兒在屋裡,經濟梯已與了他一方手帕,安付他着守房中:「我往你五娘那邊,請我下棋去。等大姑娘進來,你快叫我去。」那元宵兒應諾了。這經濟得手,走來花園中。那花篩月影,參差掩映。走在荼{艹縻}架下,遠遠望着;見婦人摘去冠兒,半挽烏雲,上着藕絲衫,下着翠紋裙,腳襯凌波羅襪,從木香棚下來。這經濟猛然從荼{艹縻}架下突出,雙手把婦人抱住。把婦人諕了一跳,說:「呸!小短命!猛可鑽出來,諕了我一跳!早是我,你摟便將就罷了!若是別人,你也恁大膽摟起來?」經濟吃的半酣兒笑道:「早知摟了你,就錯摟了紅娘,也是沒奈何!」兩個于是相摟相抱,携手進入房中。房中熒煌煌掌着燈燭,卓上設着酒餚。一面頂了角門,並肩而坐飲酒。婦人便問:「你來,大姐知不知?」經濟道:「大姐後邊聽宣卷去了。我安付下元宵兒,有事來這裡叫我。只說在這裡下棋哩。」說畢,兩個歡笑做一處。飲酒多時,常言:風流茶說合,酒是色媒人。不覺竹葉穿心,桃花上臉,一個嘴兒相親,一個腮兒廝搵。罩了燈上床交接。婦人摟抱經濟,經濟亦揣換着婦人。婦人唱六娘子:

「入門來將奴摟抱在懷,奴把錦被兒伸開。俏冤家頑的十分怪;嗏,將奴腳兒擡!腳兒擡!操亂了烏雲䯼髻兒歪。」

經濟亦占回前詞一首:

「兩意相投情掛牽,休要閃的人孤眠。山盟海誓說千遍,殘情上放著天,放著天。你又青春咱少年!」

兩人雲雨纔畢,只聽得元宵叫門,說:「大姑娘進房中來了。」這經濟慌的穿衣出門去了。正是:

「狂蜂浪蝶有時見,  飛入梨花無處尋。」

原來潘金蓮那邊三間樓上,中間供養佛像,兩邊稍間堆放生藥香料。兩個自此以後,情沾肺腑,意密如膠,無日不相會做一處。一日,也是合當有事。潘金蓮早辰梳粧打扮,走來樓上觀音菩薩前燒香。不想經濟正拏鑰匙上樓開庫房間,拏藥材香料,撞遇在一處。這婦人且不燒香,見樓上無人,兩個摟抱着親嘴砸舌。一個叫親親五娘,一個呼心肝性命,說:「趁無人,咱在這裡幹了罷!」一面解退衣褲,就在一張春凳上,雙鳧飛肩,靈根半入,不勝綢繆。有生藥名水仙子為證:

「當歸半夏紅石,可意檳榔招做女婿。浪蕩根插入蓽麻內,母丁香左右偎,大麻花一陣昏迷。白水銀撲簇簇下,紅娘子心內喜,快活殺兩片陳皮。」

當初沒巧不成話。兩個正幹得好,不防春梅正上樓來,拿盒子取茶葉看見。兩個湊手腳不迭,都吃了一驚。春梅恐怕羞了他,連忙倒退回身子,走下胡梯。慌的經濟兜小衣不迭,婦人正穿裙子,婦人便叫春梅:「我的好姐姐,你上來,我和你說話。」那春梅於是走上樓來。金蓮道:「我的好姐姐,你姐夫不是別人,我今教你知道了罷。俺兩個情孚意合,拆散不開!你千萬休對人說,只放在心裡!」春梅便說:「好娘,說那裡話!奴伏侍娘這幾年,豈不知娘心腹,肯對人說!」婦人道:「你若肯遮蓋俺們,趁你姐夫在這裡,你也過來和你姐夫睡一睡,我方信你。你若不肯,只是不可憐見俺每了!」那春梅把臉羞的一紅一白,只得依他;卸下湘裙,解開褌帶,仰在凳上,儘着這小夥兒受用。有這等事!正是:

「明珠兩顆皆無價,  可奈檀郎盡得鑽!」

有紅繡鞋為證:

「假認做女婿親厚,往來和丈母歪偷!人情裡包藏鬼胡油!明講做兒女禮,暗結下燕鶯儔,他兩個見今有!」

當下經濟耍了春梅,拏茶葉出去了。潘金蓮便與春梅打成一家,與這小夥兒暗約偷期,非止一日,只背着秋菊。婦人偏聽春梅說話,衣服首飾,揀心愛者與之,託為心腹。六月初一日,金蓮娘潘姥姥老病沒了,有人來說。吳月娘買一張插卓、三牲、冥布,教金蓮坐轎子,往門外探喪祭祀。去了一遭回來。到次日,都是六月初三日,金蓮起來的早,在月娘房裡坐着說了半日話出來。走在大廳院子裡墻根下,急了溺尿。正撩起裙子,蹲踞溺尿。原來西門慶死了,沒人客來往,等閒大廳儀門,只是閒閉不開。經濟在東廂房住,纔起來。忽聽見有人在墻根石榴花樹下,溺的尿刷刷的響。悄悄向窗眼裡張看,都不想是他。便道:「是那個撒野,在這裡溺尿?撩起衣服,看濺濕了裙子了!」這婦人連忙繫上裙子,走到窗下問道:「原來你在屋裡,這咱纔起來?好自在!大姐沒在房裡麼?」經濟道:「在後邊幾時出來!昨夜三更纔睡。大娘後邊拉住我聽宣紅羅寶卷,與他聽坐到那咱晚,險些兒沒把腰累〈疒羅〉瘑了!今日白扒不起來。」金蓮道:「賊牢成的,就牢成的,就休搗謊哄我!昨日我不在家,你幾時在上房內聽宣卷來?丫鬟說你昨日在孟三兒屋裡吃飯來!」經濟道:「早是大姐看着,俺們都在上房內,幾時在他屋裡去來?」說着,這小夥兒站在炕上,把那話弄的硬硬的,直豎的一條棍,隔窗眼裡舒過來。婦人一見,笑的要不的,罵道:「怪賊牢拉的短命!猛可舒出你老子頭來,諕了我一跳!你趁早好好抽進去,我好不好拿針剌與你一下子,教你忍痛哩!」經濟笑道:「你老人家這回兒又不待見他起來,你好歹打發他個好處,也是你一點陰騭!」婦人罵道:「好個怪牢成久慣的囚根子!」一面向腰裡摸出面青銅小鏡兒來,放在窗欞上,假做勾臉照鏡。一面用朱唇吞裹吮咂他那話,吮咂的這小郎君,一點靈犀灌頂,滿腔春意融心。正是:

「自有內事迎郎意,  慇懃愛把紫簫吹。」

原來婦人做作如此,若有人看見,只說他照鏡勾臉,麼不顯其事。其淫蠱顯然通無廉耻!正砸在熱鬧處,忽聽的有人走的腳步兒響。這婦人連忙摘下鏡子,走過一邊。經濟便把那話抽回去。都不想是來安兒小廝走來說:傅大郎前邊,請姐夫吃飯哩。」經濟道:「教你傅大郎且吃着,我梳頭哩,就來。」來安兒回去了。婦人便悄悄向經濟說:「晚夕你休往那裡去了,在屋裡。我使春梅叫你,好歹等我,有話和你說。」經濟道:「謹依來命!」婦人說畢,回房去了。經濟梳洗畢,往舖中自做買賣不題。不一時,天色晚來,那日月黑星密,天氣十分炎熱。婦人令春梅燒湯熱水,要在房中洗澡。修剪足甲,床上收拾衾枕,趕了蚊子,放下紗帳子。小篆內炷了香。春梅便叫:「娘,不知今日是頭伏?你不要些鳳仙花染指甲?我替你尋些來。」婦人道:「你尋去。」春梅道:「我直往那邊大院子裡纔有,我去拔幾根來。娘教秋菊尋下杵臼,搗下蒜。」婦人附耳低言,悄悄分付春梅:「你就廂房中請你姐夫晚夕來,我和他說話。」這春梅去了。這婦人在房中,比及洗了香肌,修了足甲,也有好一回。只見春梅拔了幾棵鳳仙花來,整叫秋菊搗了半夜。婦人又與了他幾鍾酒吃,打發他廚下先睡了。婦人燈光下染了十指春葱,令春梅拿凳子放在天井內,鋪着涼簟衾枕納涼。約有更闌時分,但見朱戶無聲,玉繩低轉,牽牛織女二星,隔在天河兩岸。又忽聞一陣花香,幾點螢火。婦人手拈紈扇,正伏枕而待。春梅把角門虛掩。正是:

「待月西廂下,  迎風戶半開;

隔墻花影動,  疑是玉人來。」

原來經濟約定搖木槿花樹為號,就知他來了。婦人見花枝搖影,知是他來,便在院內咳嗽接應。他推開門進來,兩個並肩而坐。婦人便問:「你來,房中有誰?」經濟道:「大姐今日沒出來。我已安付元宵兒在房裡,有事先來叫我。」因問:「秋菊睡了?」婦人道:「已睡熟了。」說畢,相摟相抱,二人就在院凳內上,赤身露體,席枕交歡,不勝繾綣但見:

「情興兩和諧,摟定香肩臉搵腮。手捻香乳綿似軟,實奇哉!掀起腳兒脫繡鞋,玉體著郎懷,舌送丁香口便開。倒鳳顛鸞雲雨罷,囑多才,明朝千萬早些來!」

兩個雲雨畢,婦人拏出五兩碎銀子來,遞與經濟說:「門外你潘姥姥死了,棺材已是你爹在日與了他。三日入殮時,你大娘教我去探喪燒布來了。明日出殯,你大娘不放我去說你爹熱孝在身,只見出門。這五兩銀子交與你,明日央你蚤去門外,發送發送你潘姥姥,打發擡錢,看着下入土內,你來家,就同我去一般。」這經濟一手接了銀子,說:「這個不打緊,你分付,我幹事;受人之託,必當終人之事!我明日絕早出門,幹畢事來,回你老人家。」說畢,恐大姐進房,老早歸廂房中去了。」一宿晚景休題。到飯時就來家。金蓮纔起來,在房中梳頭。經濟走來回話,就門外昭化寺裡,拿了兩枝茉莉花兒來婦人戴。婦人問:「棺材下了葬了?」經濟道:「我管何事?不打發他老人家黃金入了櫃,我敢來回話?還剩了二兩六七錢銀子,交付與你妹子收了,盤纏度日。千恩萬謝,多多上覆你。」婦人聽見他娘入土,落下淚來。便叫春梅:「把花兒浸在盞內,看茶來與你姐夫吃。」不一時,兩盒兒蒸酥四碟小菜,打發經濟吃了茶,往前邊去了。由是越發與這小夥日親日近。一日七月天氣,婦人早辰約下他;「你今日休往那裡去,在房中等着。我往你房裡,和你耍耍。」這經濟答應了。不料那日被崔本邀了他,和幾個朋友,往門外耍子。去了一日,吃的大醉來家,倒在床上,就睡着了,不知天高地下。黃昏時分,金蓮驀地到他房中。見他挺在床上,行李兒也顧不的,推他推不醒,就知他在那裡吃了酒來。可霎作怪,不想婦人摸他袖子裡,弔去一根金頭蓮瓣簪兒來。上面鈒着兩溜字兒:「金勒馬嘶芳草地,玉樓人醉杏花天。」迎亮一看,就知是孟玉樓簪子。怎生落在他袖中?想必他也和玉樓有些首尾,不然他的簪子,如何他袖着?怪道:「這短命,幾次在我面上無情無緒!我若不留幾個字兒與他,只說我沒來。等我寫四句詩在壁上,使他知道。待我見了,慢慢追問他下落。」于是取筆在壁上寫了四句,詩曰:

「獨步書齋睡未醒,  空勞神女下巫雲;

襄王自是無情緒,  辜負朝朝暮暮情。」

寫畢,婦人回房中去了。都說經濟睡起一覺,酒醒過來,房中掌上燈,因想起今日婦人來相會,我都醉了。回頭見壁上寫了四句詩在上,墨跡猶新。念了一遍,就知他來到,空回去了。打了送上門的風月兒,白丟了!心中懊悔不已:「這咱的起更時分,大姐、元宵兒都在後邊未出來。我若往他那邊去,角門又關了!」走來槿花下搖花枝為號,不聽見裡面動靜。不免踩有太湖石,扒過粉墻去。那婦人見他有酒,醉了挺覺,大恨歸房,悶悶在心,就渾衣上床〈扌歪〉睡。不料半夜,他扒過墻來。見院內無人,想丫鬟都睡了,悄悄躡足潛蹤,走到房門首。見門虛掩,就挨身進來。窗閒月色,照見床上婦人,獨自朝裡歪着。低聲叫可意人數聲,不應。說道:「你休怪我,今日崔大哥眾朋友邀了我往門外五星原庄上,射箭耍子了一日,來家就醉了。不知你到,有負你之約,恕罪!恕罪!」那婦人也不理他。這經濟見他不理,慌了。一面跪在地下,說了一遍,又重複一遍。被婦人家反手望臉上撾了一下,罵道:「賊牢拉負心短命!還不悄悄的,丫頭聽見!我知道你有個人,把我不放到心!你今日端的那去來?」經濟道:「我本被崔大哥拉了門外射箭去,灌醉了來,就睡着了!失誤你約,你休惱我!我看見你留詩在壁上,就知惱了你!」婦人道:「怪搗鬼牢拉的,別要說嘴,與我禁聲!你搗的鬼,如泥彈兒圓,我手內放不過你!今日便是崔本叫了你吃酒,醉了來家。你袖子裡這根簪子,都是那裡的?」經濟道:「本是那日花園中拾的來,今纔兩三日了。」婦人道:「你還{入日}神搗鬼,是那花園裡拾的?你再拾一根來我纔算!這簪子是孟三兒那麻淫婦的頭上簪子,我認千真萬真!上面還鈒着他名字,你還哄我?嗔道前日我不在,他叫你房裡吃飯。原來你和他七個八個,我問着,你還不成認!你不和他兩個有首尾,他的簪子緣何到你手裡?原來把我的事,都透露出與他!怪道前日他見了我笑,原來有你的話在裡頭!自今以後,你是你,我是我,綠豆皮兒,請退了!」于是急的經濟賭神發呪,繼之以哭道:「我經濟若與他有一字絲麻皂線,靈的是東岳城隍,活不到三十歲,生來碗大疔瘡,害三五年黃病,要湯不見,要水不見!」那婦人終是不信,說道:「你這賊才料,說來的牙疼誓!虧你口內不害硶!」兩個絮聒一回,見夜深了,不免解卸衣衫,挨身上床倘下。那婦人把身子扭過,倒背着他,使個性兒不理他,由着他姐姐長,姐姐短,只是反手望臉上撾過去,諕的經濟氣也不敢出一聲兒來,乾霍亂了一夜,就不誤{入日}成〈毛皮〉頭。天明,恐怕丫頭起身,依舊越墻而過,往前道廂房中去了。有醉扶歸詞為證:

「我嘴揾著他油䯼髻,他背靠著胸肚皮。早難送香腮左右偎,只在頂窩兒裡長吁氣!一夜何曾見面皮。只覷著牙梳背!」

看官聽說:往後金蓮還把這根簪子,與了經濟。後來孟玉樓嫁了李衙內,往嚴州府去。經濟還拿着這根簪子做證見,認玉樓是姐,要暗中成事。不想玉樓哄逃,反陷經濟牢獄之災。此事表過不題。正是

「三光有影遣誰繫,  萬事無根共自生。」

畢竟後來如何,且聽下回分解:

