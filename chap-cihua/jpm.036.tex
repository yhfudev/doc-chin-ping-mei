%# -*- coding: utf-8 -*-
%!TEX encoding = UTF-8 Unicode
%!TEX TS-program = xelatex
% vim:ts=4:sw=4
%
% 以上设定默认使用 XeLaTex 编译,并指定 Unicode 编码,供 TeXShop 自动识别

%第三十六回 
\chapter{翟謙寄書尋女子\KG 西門慶結交蔡狀元}



「富川遙望劍江西,  一片孤雲對夕暉,

有淚應投煙樹斷,  無書堪寄雁麟稀;

問安已負三千里,  流落空懷十二時,

海闊天高都是念,  憑誰為我說歸期。」

話說次日西門慶早與夏提刑出郊外接了新巡按,又到庄上犒勞做活的匠人,至晚來家。有平安進門就稟:「今日有東昌府下文書快手往京裡,順便稍了一封書帕來,說是太師爺府里翟大爹寄來的書與爹。小的接了,交進大娘房裡去了。那人明日午後來討回書。」西門慶聽了,走到上房,取書拆開觀看。上面寫著什麼言詞:

「京都侍生翟謙頓首書拜即擢大錦堂西門大人門下:久仰山斗,未接豐標;屢辱厚情,感媿何盡!前蒙馳論,生銘刻在心,凡百于老爹左右,無不盡力扶持。所有理事,敢托盛价煩瀆,想已為我處之矣。今因便鴻,薄具帖金十兩奉賀,兼候起居。伏望俯賜回音,生不勝感激之至。外新狀元蔡一泉,乃老爺之假子。奉勅回藉省視,道經貴處,仍望留之一飯;彼亦不敢有忘也。至祝,至祝。秋後一日信」

西門慶看畢,只顧咨嗟不已,說道:「快教小廝叫媒人去!我什麼營生就忘死了,再想不起來。」吳月娘便問:「什麼勾當?你對我說。」西門慶道:「東京太師老爺府裡翟管家,前日有書來說無子,來央及我這里替他尋個女子。不拘貧富,不限財禮,只要好的。他要圖生長。粧奩財禮該使多少,教我開了寫去,他一封封過銀子來。往後他在老爺面前,一力好扶持我做官。我一向亂著,上任七事八事,就把這事忘死了,想不起來。來保他又日逐往舖子里去了,又不題我。今日他老遠的又教人稍書來,問尋的親事怎樣的了。又寄了十兩折禮銀子賀我。明日原差人來討回書,你教我怎樣回答他?教他就怪死了!叫了媒人,你分付他好歹上緊替他尋著。不拘大小人家,只要好女兒;或十五六,十七八的也罷!該多少財禮,我這里與他。再不,把李大姐房裡綉春倒好模樣兒,與他去罷。」月娘道:「我說你是個火燎腿行貨子!這兩三個月,你早做什麼來?人家央你一場,替他看個真正女子去,他也好謝你。那丫頭你又收過他,怎好打發去的?你替他當個事幹,他到明日也替你用的力。如今施捏佛施燒香,急水裡怎麼下得漿?比不的買什麼兒,拏了銀子到市上就買的來了。一個人家閨門女子,好歹不同,也等教媒人慢慢踏看將來。你到說的好容易自在話兒!」西門慶道:「明日他來要回書,怎麼回答他?」月娘道:「虧你還斷事!這些勾當兒便不會打發人?等那人明日來,你多與他些盤纏,寫在書上回覆了他去。只說女子尋下了,只是衣服粧奩未辦,還待幾時完畢,這里差人送去。打發去了,你這里教人替他尋也不遲。此一舉兩得其便,纔幹出好事來,也是人家托你一場。」西門慶笑道:「說的有理。」一面叫將陳經濟來,隔夜修了回書。次日,下書人來到。西門慶親自出來,問了備細。又問:「蔡狀元幾時船到?好預備接他。」那人道:「小人來時,蔡老爹纔辭朝,京中起身。翟爹說,只怕蔡老爹回鄉,一時缺少盤纏,煩老爹這里多少只顧借與他。寫書去翟爹那里,如數補還。」西門慶道:「你多少覆翟爹,隨他要多少,我這里無不奉命。」說畢,命陳經濟讓去廂房內管待酒飯。臨去,交割回書,又與了他五兩路費。那人拜謝,歡喜出門,長行去了。正是:

「意急欲搖飛虎,  心忙抨碎紫花鞭。」

看官聽說;當初安忱取中頭甲,被言官論他先朝宰相安惇之弟,係黨人子孫,不可以魁多士。徽宗御遷,早不得已,把蔡蘊擢為第一,做了狀元。投在蔡京門下,做了假子,陞秘書省正事,給假省親。且說月娘家中,使小廝叫了老馮、薛嫂兒并別的媒人來,分付:「各處打聽人家有好女子,拿帖兒來說。」不在話下。一日西門慶使來保往新河口打聽蔡狀元船隻,原來和同榜進士安忱同船。這安進士亦因家貧未續親,東也不成,西也不就,辭朝還家續親,因此二人同船。來到新河口,來保拏著西門慶拜帖來到船上見,就送了一分嗄程,酒麵雞鵝嗄飯鹽醬之類。況且蔡狀元在東京,翟謙已是預先和他說了:「清河縣有老爺門下一個西門千戶,乃是大巨家,富而好禮。亦是老爺抬舉,見做理刑官。你到那里,他必然厚侍。」這蔡狀元牢記在心。見西門慶差人遠來迎接,又餽送如此大禮,心中甚喜。次日到了,就同安進士進城拜西門慶。西門慶已是叫廚子家裡預備下酒席。因在李知縣衙內吃酒,看見有一起蘇州戲子唱的好,問書童兒。說:「在南門外磨子營兒那里住。」旋叫了四個來答應。蔡狀元那日封了一端絹帕,一部書,一雙雲履;安進士亦是書帕二事,四袋芽茶,四柄杭扇,各具官袍烏紗,先投拜帖進去。西門慶冠冕迎接至廳上,敘禮交拜。家童獻畢贄儀,然後分賓主而坐。先是蔡狀元舉手欠身說道:「京師翟雲峰甚是稱道賢公閥閱名家,清河巨族。久仰德望,未能識荊。今得晉拜堂下,為幸多矣。」西門慶答道:

「不敢。昨日雲峰書來,具道二位老先生華輈下臨,理當迎接。奈公事所羈,幸為寬恕。」因問:「二位老先生仙鄉?尊號?」蔡狀元道:「學生蔡蘊,本貫滁州之匡廬人也,賤號一泉。僥倖狀元,官拜秘書正字。給假省親,得蒙皇上俞允。不想雲峰先生,稱道盛德,拜遲!」安進士道:「學生乃浙江錢塘縣人氏,賤號鳳山。見除工部觀政,亦給假還鄉續親。敢問賢公尊號?」西門慶道:「在下卑官武職,何得號稱!」詢之再三,方言:「賤號四泉。累蒙蔡老爺抬舉,雲峰扶持,襲錦衣千戶之職。見任理刑,實為不稱。」蔡狀元道:「賢公抱負不凡,雅望素著,休得自謙。」敘畢禮話,請去花園捲棚內寬衣。蔡狀元辭道:「學生歸心匆匆,行舟在岸,就要回去;既見尊顏,又不遽舍。奈何,奈何!」西門慶道:「蒙二公不棄蝸居,伏乞暫駐文旆,少留一飯,以盡芹獻之情。」蔡狀元道:「既是雅情,學生領命。」一面脫去衣服,二人坐下。左右又換了一道茶上來。蔡狀元以目瞻顧西門慶家園池花館,花木深秀,一望無際。心中大喜,極口稱羡,誇道:「誠乃勝蓬瀛也!」于是抬過棋卓來下棋。西門慶道:「今日有兩個戲子,在此伺候,以供燕賞。」安進士道:「在那里?何不令來一見?」不一時,四個戲子跪下磕頭。蔡狀元問道:「那兩個是生旦?叫甚名字?」于是走向前說道:「小的是裝生的,叫苟子孝;那一個裝旦的,叫周順;一個貼旦,叫袁琰;那一個裝小生的,叫胡慥。」安進士問:「你每是那里子弟?」苟子孝道:「小的都是蘇州人。」安進士道:「你等先粧扮了來,唱個我每聽。」四個戲子下邊粧扮去了。西門慶令後邊 取女衣釵流與他。教書童也粧扮起來。共三個旦兩個生,在席上先唱香囊記。大廳正面設兩席,蔡狀元、安進士居上,西門慶下邊主位相陪。飲酒中間,唱了一摺下來。安進士看見書童兒裝小旦,便道:「這個戲子是那里的?」西門慶道:「此是小价書童。」安進士叫上去,賞他酒去,說道:「此子絕妙,而無以加矣!」蔡狀元又叫別的生旦過來,亦賞酒與他吃。因分付:「你唱個朝元歌『花邊柳邊』。」苟子孝答應,在旁拍手唱道:

「花邊柳邊,簷外晴絲捲;山前水前,馬上東風軟。自歎行踪,有如逢轉;盼望家鄉留戀。雁杳魚沉,離愁滿懷誰與傳?日短北堂萱,空勞魂夢牽。(合)洛陽遙遠,幾時得上九重金殿!」

唱了一箇,吃畢酒,又唱第二個:

「十載青燈黃卷,螢窗苦勉旃,雪案費精研。指望榮親,姓揚名顯。試向文場鏖戰,禮樂三千,英雄五百爭後先。快著祖生鞭,行瞻尺五天。(合前)」

安進士令苟子孝:「你每可記的玉環記『恩德浩無邊』?」苟子孝答道:「此是畫眉序,小的記得。」

「恩德浩無邊,父母重逢感非淺。幸終身托與,又與姻緣。風雲際會,異日飛騰;鸞鳳配,今諧繾綣。(合)料應夫婦非今世,前生玉種藍田。」

書童兒把酒斟,拍手唱道:

「弱質始笄年,父母恩深浩如天;報無由,媿赧此心縈牽。鴛鴦配,深沐親恩;箕帚婦,願夫榮顯。(合前)」

原來安進士杭州人,喜尚南風。見書童兒唱的好,拉著他手兒,兩個一遞一口吃酒。良久,酒闌上來。西門慶陪他復遊花園,向捲棚內下棋。今小廝拏兩卓盒,三十樣都是細巧菓菜鮮物下酒。蔡狀元道:「學生每初會,不當深擾潭府。天色晚了,告辭罷。」西門慶道:「豈有此理?」因問:「二公此回去,還到船上?」蔡狀元道:「暫借門外永福佛寺寄居。」西門慶道:「如今就門外去,也晚了。不如先生把手下從者留下一二人答應,餘者都分付回去,明日來接,庶可兩盡其情。」蔡狀元道:「賢公雖是愛客之意,其如過擾何?」當下二人一面分付手下:「都回門外寺里歇去。明日早拏馬來接。」眾人應諾去了,不在話下。二人在捲棚內下了兩盤棋,子弟唱了兩摺。恐天晚,西門慶與了賞錢,打發去了。止是書童一人,席前遞酒伏侍。看看吃至掌燈,二人出來更衣。蔡狀元接西門慶說話:「此去學生回鄉省親,路費缺少。」西門慶道:「不勞老先生分付,雲峰尊命,一定謹領。」良久,讓二人到花園:「還有一處小亭,請看。」把二人一引,轉過粉牆,來到藏春塢,乃一邊僻靜所雪洞內,裡面曉騰騰掌著燈燭,小琴卓兒早已陳設綺席菓酌之類。床榻依然,琴書瀟灑。從新復飲。書童在旁歌唱。蔡狀元問道:「大官,你會唱『紅入仙桃』?」書童道:「此是錦堂月,小的記的。」蔡狀元道:「既是記的,大官你唱。」于是把酒都斟。那書童拏住南腔,拍手唱道:

「紅入仙桃,青歸御柳,鶯啼上林春早。簾捲東風,羅襟曉寒尤峭。喜仙姑,書付青鸞;念慈母,恩同烏鳥。(合)風光好,但願人景長景,醉遊蓬島。」

安進士聽了,喜之不勝。向西門慶稱道:「此子可敬!」將盃中之酒一吸而飲之。那書童席前穿著翠袖紅裙,勒著銷金箍兒,高擎玉斝,捧上酒去,又唱道:

「難報母氏劬勞,親恩罔極,只願壽比松喬。定省晨昏,連枝上有兄嫂。喜春風,棠棣聯芳;娛晚景,松柏同操。(合前)」

當日飲至夜分,方纔歇息。西門慶藏春塢、翡翠軒兩處,俱設床帳,鋪陳綾錦被褥,就要派書童、玳安兩個小廝答應。西門慶道了安置,回後邊去了。到次日,蔡狀元、安進士跟從人夫,轎馬來接。西門慶廳上擺飯伺候。撰盤酒飯,與腳下人吃。教兩個小廝,方盒捧出禮物,蔡狀元是金段一端,領絹二端,合香五百,白金一百兩;安進士是色段一端,領絹一端,合香三百,白金三十兩。蔡狀元固辭再三,說道:「但假十數金足矣,何勞如此太多!又蒙厚腆!」安進士道:「蔡年兄領受,學生不當。」西門慶笑道:「些須微贐,表情而已。老先生榮歸續親,在下此意,少助一茶之需。」于是二人俱席上出來謝道:「此情此德,何日忘之!」一面令家人各收下去,入氈包內。與西門慶相別,說道:「生輩此去,天各一方,暫違台教。不日旋京,倘得寸進,自當圖報。」安進士道:「今日相別,何年再得奉接尊顏!」西門慶道:「學生蝸居屈尊,多有褒慢。幸惟情恕!本當遠送,奈官守在身,先此告過。」送二人到門首,看著上馬而去。正是:

「博得錦衣歸故里,  功名方信是男兒。」

畢竟未知後來何如,且聽下回分解:

