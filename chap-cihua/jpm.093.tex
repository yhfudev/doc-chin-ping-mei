%# -*- coding: utf-8 -*-
%!TEX encoding = UTF-8 Unicode
%!TEX TS-program = xelatex
% vim:ts=4:sw=4
%
% 以上设定默认使用 XeLaTex 编译,并指定 Unicode 编码,供 TeXShop 自动识别

%第九十三回 
\chapter{王杏菴仗義賙貧\KG 任道士因財惹禍}


「誰道人生運不通,  吉凶禍福並肩行,

只因風月將身陷,  未許人心直似針;

自課官途無枉屈,  豈知天道不昭明,

早知成敗皆由命,  信步而行暗黑中。」

話說陳經濟自從西門大姐死了,被吳月娘告了一狀,打了一場官司出來。唱的馮金寶又歸院中去了。剛刮剌出箇命兒來,房兒也賣了,本錢兒也沒了,頭面也使了,家火也沒了。又說陳定在外邊打發人尅落了錢,把陳定也攆去了。家中日逐盤費不週,坐吃山空,不免往楊大郎家中,問他這半船貨的下落,一日來到楊大郎門首,叫聲:「楊大郎在家不在?」不想楊光彥拐了他半船貨物,一向在外賣了銀兩,四散躲閃。及打聽得他家中吊死了老婆,他丈母縣中告他,坐了半箇月監房。這楊大郎驀地來家住着不出來。聽見經濟上門叫他,問貨船下落,一經使兄弟楊二風出來,反問經濟要人:「你把我哥哥叫的外邊做買賣,這幾箇月通無音訊。不知拋在江中,推在河內,害了性命。你倒還來我家尋貨船下落!人命要緊?你那貨船要緊?」這楊二風平昔是箇刁徒潑皮,耍子揭子。胳膊上紫肉橫生,胸前上黃毛亂長,是條直率之光棍。走出來一把手扯住經濟,就問他要人。那經濟慌忙掙開手,跑回家來。這楊二風故意拾了塊三尖瓦楔將頭顱礸破,血流滿面,趕將經濟來罵道:「我{入日}你娘眼!我見你家甚麼銀子來?你來我屋裡放屁!吃我一頓好拳頭!」那陳經濟金命水命,走投無命,奔到家,把大門關閉,如鐵桶相似,就是樊噲也撞不開。由著楊二風牽爺孃罵父母,拏大磚砸門,只是鼻口內不聽見氣兒。又況纔打了官司出來,夢條繩蛇也害怕!只得含忍過了。正是:

「嫩草怕霜霜怕日,  惡人自有惡人磨!」

不消幾時,把大房賣了,找了七十兩銀子,典了一所小房,在僻巷內居住。落後兩箇丫頭,賣了一個重喜兒,只留著元宵兒和他同舖歇。又過了不上半月,把小房倒騰了,卻去賃房居住。陳安也走了,家中沒營運;元宵兒也死了,止是單身獨自。家火卓椅都變賣了,只落得一貧如洗。未幾房錢不給,鑽入冷舖內存身。花子見他是個富家勤兒,生的清俊,叫他在熱坑上睡,與他燒餅兒吃。有當夜的過來,教他頂火夫,打梆子搖鈴。那時正值臘月殘冬時分,天降大雪,吊起風來,十分嚴寒。這陳經濟打了回梆子,打發當夜的兵牌過去,不免手提鈴串了幾條街巷。又是風雪,地下又踏著那寒冰,凍得聳肩縮背,戰戰兢兢。臨五更雞叫,只見箇病花子,倘在墻底下。恐怕死了,總甲分付他看守著他,尋箇把草教他烤。這經濟支更一夜,沒曾睡,就〈扌歪〉下睡著了。不想做了一夢,夢見那時在西門慶家,怎生受榮華富貴,和潘金蓮抅搭頑耍戲謔,從睡夢中就哭醒了。眾花子說:「你哭怎的?」這經濟便道:「你眾位哥哥,聽我訴說一遍。」有粉蝶為證:

「九臘深冬雪漫天,涼然冰凍,更搖天撼地狂風!凍得我體僵麻,心膽戰,實難扎掙!挨不過肚中饑,又難禁身上冷,住著這半邊天,端的是冷!挨不過淒涼,要尋死路,百忙裡捨不的頹命!」

〔耍孩兒一煞〕「不覺撞昏鍾,昏鍾人初定。是誰人叫我?原來是總甲張成!他那里急急呼,我這里連連應。趁今宵誰肯與我支更?也是我一時僥倖,他先遞與我幾箇燒餅。」

〔二煞〕「名承總甲憐咱冷,教我敲梆子守守更,由著他調用。但得這濟心饑錢米,那里管人貧下賤,一任教喝號提鈴!」

〔三煞〕「坐一回腳手麻,立一回肚裡疼。冷燒餅乾嚥無茶送。剛然未到三更後,下夜的兵牌叫點燈。歪踢弄,與了他四十文,方纔得買一箇姑容。」

〔四煞〕「到五更雞打鳴,大街上人漸行,眾人各去都不等。只見病花子倘在墻根下,教我煨著他,不暫停。得他口煖氣兒心纔定。剛合眼一場幽夢,猛驚回哭到天明。」

〔五煞〕「花子說氣哭怎的?我從頭兒訴始終。我家積祖根基兒重,說聲賣松槁「陳家」誰不怕名姓?多居住窰中,我祖耶耶曾把誰鹽種,我父親專結交勢耀,生下我吃酒行兇!」

〔六煞〕「先亡了打我的爹,後亡了我父親。我孃疼,專隨縱,吃酒耍錢般般會,酒肆巢窩處處通。所事兒都相稱,娶了親就遭官事,丈人家躲重投輕。」

〔七煞〕「我也曾在西門家做女婿,調風月,把丈母淫。錢場裡信著人鎖狗洞,也曾黃金美玉當場賭,也曾馱米擔柴往院裡供。歐打妻兒病死了,死了時,他家告狀,使了許多錢,方得頭輕。」

〔八煞〕「賣大房,買小房,贖小房;又倒騰。示思久遠含餘剩。饑寒苦惱妾成病,死在房簷不許停。所有都乾淨。嘴頭纔不離酒肉,沒攪汁拆賣坟塋!」

〔九煞〕「掇不的輕,負不的重;做不得傭,務不得農;未曾幹事兒先愁動。閑中無事思量嘴,睡起須教日頭紅;狗性子生鐵般硬,惡盡了十親九眷,凍餓死有那箇憐憫!」

〔十煞〕「討房錢不住催,他料我也住不成,沙鍋破碗全無用。幾推趕出門兒外,凍骨淋皮無處存,不免冷舖將身奔。但得箇時通運轉,我那其間忘不了恩人。」

「頻年困苦痛妻亡,  身上無衣口絕糧,

馬死奴逃房又賣,  隻身獨自走他鄉;

朝依肆店求遺饌,  暮宿庄團倚敗墻,

只有一條身後路,  冷舖之中去打梆。」

卻說陳經濟晚夕在冷舖存身。白日間街頭乞食。清河縣城內,有一老者,姓王名宣,字廷用,年六十餘歲。家道殷實,為人心慈。好仗義疎財,廣結交,樂施捨,專乙濟貧拔苦,好善敬神。所生二子,皆當家成立,長子王軋,襲祖職為牧馬所掌印正千戶;次子王震,充為府學庠生。老者門首搭了箇主管,開著箇解當舖兒。每日豐衣足食,閑散無拘,在梵宇聽經,琳宮講道。無事在家門首施藥救人,拈素珠念佛。因後園中有兩株杏樹,道號為杏庵居士。一日,杏庵頭戴重簷幅巾,身穿水合道服,在門首站立。只見陳經濟打他們首過,向前扒在地下磕了箇頭。慌的杏菴還不迭,說道:「我的哥,你是誰?老拙眼昏,不認得你。」這經濟戰戰兢兢,站立在旁邊,說道:「不瞞你老人家,小人是賣松橋陳洪兒子。」老者想了半日,說:「你莫不是陳大寬的令郎麼?」因見他的衣服襤褸,形容憔悴,說道:「我賢姪,你怎的弄得這等模樣?」便問:「你父親、母親可安麼?」經濟道:「我爹死在東京,我母親也死了!」杏菴道:「我聞得你在丈人家往來?」經濟道:「家外父死了,外母把我攆出來。他女兒死了,告我到官,打了一場官司,把房兒也賣了。有些本錢兒,都吃人坑了。一向閑著,沒有營運。」杏菴道:「賢姪,你如今在那里居住?」經濟半日不言不語,說:「不瞞你老人家說,如此如此。」杏菴道:「可憐,賢姪,你原來討吃哩!想著當初你府上那樣根基人家!我與你父親相交,賢姪你那咱還小哩,纔扎著總角上學哩!一向流落到此地位,可傷,可傷!你還有甚親家,也不看顧你看顧兒?」經濟道:「正是。俺張舅那里,一向也久不上門,不好去的。」問了一回話,老者把他讓到裡面客位裡,令小廝放卓兒,擺出點心嗄飯來,教他儘力吃了一頓。見他身上單寒,拏出一件青布綿道袍兒,一頂毡帽,又一雙毡襪綿鞋,又秤一兩銀子,五百銅錢,遞與他,分付說:「賢姪,這衣服鞋襪,與你身上穿;那銅錢與你盤纏,賃半間房兒住。這一兩銀子,你拏着做上些小買賣兒,也好糊口過日子。強如在冷舖中,學不出好人來!每月該多少房錢,來這里老拙與你。」這陳經濟扒在地下磕頭謝了,說道:「小姪知會。」拏着銀錢,出離了杏菴門首,也不尋房子,也不做買賣,把那五百文錢,每日只在酒店麵店,以了其事。那一兩銀子,搗了些白銅頓罐,在街上行使。吃巡邏的當土賊拏到該坊節級處,一頓拶打,使的罄盡,還落了一屁股瘡。不消兩日。把身上綿衣也輸了,襪兒也換來嘴吃了,依舊原在街上討吃。一日,又打王杏菴門首所過。杏菴正在門首,只見經濟走來磕頭,身上衣襪都沒了,止戴著那毡帽,精腳靸鞋,凍的乞乞縮縮。老者便問:「陳大官做得買賣如何?房錢到了,來取房錢來了?」那陳經濟半日無言可對,問之再三,方說:「如此這般,都沒了!」老者便道:「阿呀!賢姪,你這等就不是過日子的道理!你又拈不的輕,負不的重,但做了些小活路兒,還強如乞食,免教人恥笑,有玷你父祖之名!你如何不依我說?」一面又讓到裡面,教安童拿飯來與他吃飽了。又與了他一條袷褲,一領白布衫,一雙裹腳,一吊銅錢,一斗米。「你拏去,務要做上了小買賣,賣些柴炭豆兒,瓜子兒,也過了日子。強似這等討吃!」這經濟口雖答應,拏錢米在手,出離了老者門,那消數日,熟食肉麵,都在冷舖內,和花子打夥兒都吃了。要錢,又把白布衫袷褲都輸了。大正月裡,又抱著肩兒,在街上走。不好來見老者,走在他們首房,山墻底下,向日陽站立。老者冷眼看見他,不叫他。他挨挨搶搶,又到根前,扒在地下磕頭。老者見他還依舊如此,說道:「賢姪,這不是常策!咽喉深似海,日月快如梭!無底坑如何填得起?你進來,我與你說。有一箇去處,又清閒,又安得你身,只怕你不去。」經濟跪下哭道:「若得老伯見怜,不拘那里,但安下身,小的情愿就去!」杏菴道:「此去離城不遠,臨清馬頭上,有座晏公廟;那裡魚米之鄉,舟船輻輳之地,錢糧極廣,清幽消灑。廟主任道士,與老拙相交極厚。他手下也有兩三箇徒弟徒孫。我備分禮物,把你送與他做箇徒弟出家,學些經典吹打,與人家應福,也是好處。」經濟道:「老伯看顧,可知好哩!」杏菴道:「既然如此,你去。明日是箇好日子,你早來,我送你去。」經濟去了,這王老連忙叫了裁縫來,就替經濟做了兩件道衣,一頂道髻,鞋襪俱全。次日經濟果然來到。王老教他空屋裡洗了澡,梳了頭,戴上道髻,裡外換了新襖新褲。上蓋青絹道衣,下穿雲履毡襪。備了四盤羹果,一罈酒,一疋尺頭,封了五兩銀子,他便乘馬,顧了一匹驢兒,與經濟騎著。安童、喜童跟隨,兩箇人抬了盒担,出城門,逕往臨清馬頭晏公廟來,止七十里,一日路程。比及到晏公廟,天色已晚,但見:

「日影將沈,繁陰已轉。斷霞映水散紅光,落日轉山生碧霧。綠楊影裡,時聞鳥雀歸林;紅杏村中,每見牛羊入圈。」

正是:

「溪邊漁父投林去,  野外牧童跨犢歸。」

王老到于馬頭上,過了廣濟閘大橋,見無數舟船,停泊在河下。來到晏公廟前下馬,進入廟來。只見青松欝欝,翠柏森森。兩邊八字紅牆,正面三間朱戶。端的好座廟宇!但見:

「山門高聳,殿閣崚層。高懸勑額金書,彩畫出朝入相。五間大殿塑龍王一十二尊,兩下長廊刻水族百千萬眾。旗竿凌漢,帥字招風。四通八達,春秋社禮享依時;雨順風調,河道民間皆祭賽。萬年香火威靈在,四境官民仰賴安。」

山門下,早有小童看見,報入方丈。任道士忙整衣出迎。王杏菴令經濟和禮物,且在外邊伺候。不一時,任道士把杏菴讓入方丈松鶴軒敘禮說:「王老居士怎生一向不到敝廟隨喜?今日何幸,得蒙下顧!」杏菴道:「只因家中俗冗所羈,久失拜望。」敘禮畢,分賓主而坐,小童獻茶。茶罷,任道士道:「老居士今日天色已晚,你老人家不去罷了?」分付把馬牽入後槽喂息。杏菴道:「沒事不登三寶殿,老拙敬來有一事干瀆,未知尊意肯容納否?」任道士道:「老居士有何見教?只顧分付。小道無不領命。」杏菴道:「今有故人之子,姓陳,名經濟,年方二十四歲。生的資格清秀,倒也伶俐。只是父母去世太早;自幼失學。若說他祖父根基,也不是無名少姓人家子孫,有一分家當。只因不幸遭官事沒了家,無處棲身。老拙念他乃尊舊日相交之情,欲送他來貴宮作一徒弟。未知尊意如何?」任道士便道:「老居士分付,小道怎敢違阻?奈因小道命蹇,手下雖有兩三箇徒弟,都不省事,沒一箇成立的!小道常時惹氣。未知此人誠實不誠實?」杏菴道:「這箇小的,不瞞尊師說,只顧放心,一味老實本分!膽兒又小,所事兒伶範,堪可作一徒弟。」任道士問:「幾時送來?」杏菴道:「見在山門外伺候。還有些薄禮,伏乞笑納。」慌的任道士道:「老居士何不早說?」一面道:「有請!」于是抬盒人抬進禮物,任道士見帖兒上寫著:「謹具粗段一端,魯酒 一墫,豚蹄一副 ,燒鴨二隻 ,樹果二盒,白金五兩,知生王宣頓首拜。」連忙稽首謝道:「老居士何以遠勞,見賜許多重禮!使小道卻之不恭,受之有愧!」只見陳經濟頭戴著金梁道髻,身穿青絹道衣,腳下雲履淨襪,腰繫絲縧,生的眉清目秀,齒白唇紅,面如傳粉,走進來向任道士倒身下拜,拜了四雙八拜。任道士因問:「多少青春?」經濟道:「屬馬,交新春二十四歲了。」任道士見他果然伶俐,取了他箇法名,叫做陳宗美。原來任道士手下有兩箇徒弟,大徒弟姓金名宗明,二徒弟姓徐名宗順,他便叫陳宗美。王杏菴都請出來,見了禮數。一面收了禮物,小童掌上燈來,放卓兒,先罷飯,後吃酒。餚品盃盤,堆滿卓上,無非是雞蹄、鵝鴨、魚蝦之類。王老吃不多酒,師徒輪番勸彀幾巡,王老不勝酒力告辭,房中自有床舖安歇一宿。到次日清辰,小童舀水淨面,梳洗灌漱畢。任道士又早來遞茶。不一時擺飯,又吃了兩盃酒,喂飽頭口,與了抬盒人力錢。王老臨起身,叫過經濟來分付:「在此好生用心,習學經典,聽師父指教。我常來看你,按季送衣服鞋腳來與你。」又向任道士說:「他若不聽教訓,一任責治,老拙並不護短。」一面背地又囑付經濟:「我去後,你要洗心改正,習本等事業。你若再不安分,我不管你了!」那經濟應諾道:「兒子理會了。」王老當下作辭任道士出山門上馬,離晏公廟回家去了。經濟是此就在晏公廟做了道士。因見任道士年老赤鼻,身體魁偉,聲音洪亮,一部髭髯,能談善飲,只專迎賓送客,凡一應大小事,都在大徒弟金宗明手裡。那時朝廷運河初開,臨清設二閘,以節水利。不拘官民船到閘上,都來廟裡,或求神福,或來祭愿,或討卦與苕,或做好事。也有布施錢米的,也有餽送香油布燭的,也有留松篙蘆蓆的。這任道士將常署裡多餘錢糧,都令吾下徒弟,在馬頭上開設錢米舖,賣將銀子來,積儹私囊。他這大徒弟金宗明,也不是箇守本分的,年約三十餘歲。常在娼樓包占樂婦,是箇酒色之徒。手下也有兩箇清紫年小徒弟,同舖歇臥,日久絮繁。因見經濟生的齒白唇紅,面如傳粉;清俊乖覺,眼裡說話,就纏他同房居住。晚夕和他吃半夜酒,把他灌醉了,在一舖歇臥。初時兩頭睡,便嫌經濟腳臭,叫過一箇枕頭上睡。睡不多回,又說他口氣噴著,令他吊轉身子,屁股貼著肚子。那經濟推睡著,不理他。他把那話弄得硬硬的,直豎一條棍,抹了些唾津在頭上,往他糞門裡只一頂。原來經濟在冷舖中被花子飛天鬼候林兒弄過的,眼子大了,那話不覺就進去了。這經濟口中不言,心內暗道:「這廝合敗!他討得十分便益多了,把我不知當做甚麼人兒?也來報伏!與他箇甜頭兒,且教他在我手內納些敗缺!」一面故意聲叫起來。這金宗明恐怕老道士聽見,連忙掩住他口,說:「好兄弟,禁聲!隨你要的,我都依你。」經濟道:「你既要抅搭我,我不言語,須依我三件事。」宗明道:「好兄弟,休說三件,就是十件事,我也依你。」經濟道:「第一件,你既要我,不許你再和那兩箇徒弟睡。第二件,大小房門上鑰匙,我要執掌。第三件,隨我往那裡去,你休嗔我。你都依了我,我方依你此事。」金宗明道:「這個不打緊,我都依你。」當夜兩箇顛來倒去,整狂了半夜。這陳經濟自幼風月中撞,甚麼事不知道!當下被底山盟,枕邊海誓,淫聲艷語,摳吮舔品,把這金宗明哄得歡喜無盡。到第二日,果然把各處鑰匙都交與他手內,就不和那兩箇徒弟在一處,每日只同他一舖歇臥。一日兩,兩日三,忽一日任道士師徒三箇,都往人家應福做好事去。任道士留下他看家,徑智賺他,王老居士只說他老實,看老實不老實。臨出門分付:「你在家好看著那後邊養的一群雞。」說道:「是鳳凰。我不久功成行滿,騎他上昇,朝參玉帝。那房內做的幾缸,都是毒藥汁。若是徒弟壞了事,我也不打他,只與他這毒藥汁吃了,直教他立化。你須用心看守,我午齋回來,帶點心與你吃。」說畢,師徒去了。這經濟關上門,笑道:「豈可我這些事兒不知道?那房內幾缸黃米酒,哄我是甚毒藥汁!那後邊養的幾隻雞,說是鳳凰,要騎他上昇!」于是揀肥的宰了一隻,退的淨淨,煮在鍋裡。把缸內酒 ,用鏇子舀出來,火上篩熱了,手撕雞肉,蘸著蒜醋,吃了箇不亦樂乎!還說了四句:「黃銅鏇,舀清酒,煙籠皓月;白污雞,蘸爛蒜,風捲殘雲。」正吃著,只聽師父任道士外邊叫門。這經濟連忙收拾了家伙,走出來開門。任道士見他臉紅,問他怎的來?這經濟徑低頭不言語。師父問:「你怎的不言語?」經濟道:「告稟師父得知。師父去後,後邊那鳳凰不知怎的飛了去一隻。教我慌了,上房尋了半日,沒有。怕師父來家打,待要拏刀子抹,恐怕疼;待要上吊,死怕斷了繩子跌著;待要投井,又怕井眼小掛脖子。算計的沒處去了,把師父缸內的毒藥汁,舀了兩碗來吃了!」師父便問:「你吃下去覺怎樣的?」經濟道:「吃下去半日,不死不活的,倒像醉了的一般。」任道士聽言,師徒門都笑了,說:「還是他老實!」又替他使錢討了一張度牒,以此往後,凡事並不防範。正是:

「三日賣不得一担真,  一日賣了三担假。」

這陳經濟因此常拏著銀錢,往馬頭上遊玩。看見院中架兒陳三兒,說:「馮金寶兒他鴇子死了。他又賣在鄭家,叫鄭金寶兒。如今又在大酒樓上趕趁哩,你不看他看去?」這小夥兒舊情不改,拏著銀錢跟定陳三兒,逕往馬頭大酒樓上來。此不來倒好,若來,正是:

「五百載冤家來聚會,  數年前姻眷又相逢。」

有詩為證:

「人生莫惜金縷衣,  人生莫負少年時,

見花欲折須當折,  莫待無花空折枝!」

原來這座酒樓,乃是臨清第一座酒樓,名喚謝家酒樓。裡面有百十座閣兒,周圍都是綠欄杆。就緊靠著山岡,前臨官河,極是人烟熱鬧去處,舟船往來之所。怎見得這座酒樓齊整?

「雕簷映日,畫棟飛雲。綠欄杆低接軒窗,翠簾櫳高懸戶牖。吹笙品笛,盡都是公子王孫;執盞擎盃,擺列著歌姬舞女。消磨醉眼,倚青天萬疊雲山;勾喏吟魂,翻瑞雪一河烟水。白蘋渡口,時聞漁父鳴榔;紅蓼灘頭,每見釣翁擊楫。樓畔綠楊啼野鳥,門前翠柳繫花驄。」

這陳三兒弔經濟上樓,到一箇閣兒裡坐下,烏木春檯,紅漆凳子。便叫店小二連忙打抹了春檯,拏一付鍾筯,安排一分上品酒果下飯來擺著,使他下邊叫粉頭去了。須臾,只聽樓梯響,馮金寶上來,手中拏著箇廝鑼兒,見了經濟,深深道了萬福。常言:「情人見情人,不覺簇地兩行淚下。」正是:

「數聲嬌語如鶯囀,  一串珍珠落線頭!」

經濟一見,便拉他一處坐,問道:「姐姐,你一向在那裡來,不見你?」這馮金寶收淚道:「自從縣中打斷出來,我媽不久著了驚諕,得病死了。把我賣在鄭五媽兒家做粉頭。這兩日子弟稀少,不免又來在臨清馬頭上趕趁酒客。昨日聽見陳三兒說,你在這裡開錢舖,要見你一見。不期你今日在此樓上吃酒,會見一面,可不想殺我也!」說畢,又哭了。經濟便取袖中帕兒,替他抹了眼淚,說道:「我的姐姐,你休煩惱,我如今又好了。自從打出官司來,家業都沒了。投在這晏公廟,一向出家做了道士。師父甚是重托我。往後我常來看你。」因問:「你如今在那裡安下?」金寶便說:「奴就在這橋西酒家店劉二那裡,有百十間房子,四外行院窠子妓女,都在那裡安下。白日裡便來這各酒樓趕趁。」說著,兩箇挨身做一處飲酒。陳三兒盪酒上樓,拏過琵琶來。金寶彈唱了箇曲兒,與經濟下酒。名普天樂:

「淚雙垂,垂雙淚,三盃別酒,別酒三盃。鸞鳳對拆開,拆開鸞鳳對。嶺外斜暉看看墜,看看墜嶺外暉,天昏地暗,徘徊不捨,不捨徘徊!」

兩人吃得酒濃時,未免解衣雲雨,下箇房兒。這陳經濟一向不曾近婦女,久渴的人。合得遇金寶,儘力盤桓。尤雲殢雨,未肯即休。但見:

「一箇玉臂忙搖,一箇柳腰款擺。雙睛噴火,星眼郎當。一箇汗浹胸膛,發狠要贏三五陣;一箇香消粉黛,呻吟叫彀數千聲。戰良久,靈龜深入性偏剛,鬬彀多時,一般清泉往裡邈。幾番鏖戰烟蘭妓,不似今番這一遭。」

須臾事畢,各整衣衫。經濟見天色晚來,與金寶作別,與了金寶一兩銀子,與了陳三兒三百文銅錢。囑付:「姐姐,我常來看你,咱在這搭兒裡相會。你若想我,使陳三兒叫我去。」下樓來,又打發了店主人謝三郎三錢銀子酒錢。經濟回廟中去了。這馮金寶送至橋邊方回。正是:

「盼穿秋水因錢鈔,  哭損花容為鄧通!」

畢竟未知如何,且聽下回分解:

