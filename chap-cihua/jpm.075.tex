%# -*- coding: utf-8 -*-
%!TEX encoding = UTF-8 Unicode
%!TEX TS-program = xelatex
% vim:ts=4:sw=4
%
% 以上设定默认使用 XeLaTex 编译,并指定 Unicode 编码,供 TeXShop 自动识别

%第七十五回 
\chapter{春梅毀罵申二姐\KG 玉簫愬言潘金蓮}


「萬里新墳盡十年,  修行莫待鬢毛斑,

死生事大宜須覺,  地徹時常非等閒;

道業未成何所賴,  人身一失幾時還,

前途暗黑路途險,  十二時中自著研。」

此八句單道這善有善報,惡有惡報;如影隨形,如谷應聲。你道打坐參禪,皆成正果。相這愚夫愚婦,在家修行的豈無成道?禮佛者,取佛之德;念佛者,感佛之恩;看經者,明佛之理;坐禪者,踏佛之境;得悟者,正佛之道。非同容易。有多少先作後修,先修後作,有如吳月娘者,雖有此報,平日好善看經,禮佛布施,不應今此身懷六甲,而聽此經法。人生貧富、壽夭、賢愚,雖蒙父母受氣成胎中來,還要懷妊之時,有所應召。古人妊娘懷孕,不倒坐,不偃臥,不聽淫聲,不視邪色,常玩弄詩書金玉異物,常令瞽者誦古詞。後日生子女,必端正俊美,長大聰慧。此文王胎教之法也。今吳月娘懷孕,不宜令僧尼宣卷,聽其生死輪迴之說。後來感到一尊古佛出世,投胎奪舍,日後被其顯化而去,不得承受家緣,蓋可惜哉!正是:

「前程暗黑路途險,  十二時中自著研。」

此係後事表過不題。當下後邊聽宣畢黃氏寶卷,各房宿歇。單表潘金蓮在腳門處久站立,忽見西門慶過來,相携到房中。見西門慶只顧坐在牀上,便問:「你怎的不脫衣裳?」那西門慶摟定婦人,笑嘻嘻說道:「我特來對你說聲,我要過那邊歇一夜兒去,你拿那淫器包兒來與我。」婦人罵道:「賊牢!你在老婦手裡使巧兒,拿些面子話兒來哄我。我剛纔不在角門首站著,你過去的不耐煩了!又肯來問我?這個是你早辰和那歪刺骨兩個商定了腔兒!好在和他個{入日}窩去,一徑拿我扎篾子。嗔道頭里不使丫頭,使他來送皮襖兒,又與我磕了頭兒來。小賊歪刺骨,把我當甚麼人兒?在我手內弄判子!我還是李瓶兒時,教你活埋我?雀兒不在那窩兒裡,我不醋了。」西門慶笑道:「那里有此勾當?他不來與你磕個頭兒,你又說他的那不是!」婦人沉吟良久,說道:「我放你去便去,不許你拿了這包子去和那歪刺骨弄答的齷齷齪齪的,到明日還要來和我睡,好乾淨兒!」西門慶道:「你不與我,使慣了都怎樣的?」纏了半日,婦人把銀托子掠與他,說道:「你要,拿了這個行貨子去。」西門慶道:「與我這個也罷。」一面接的袖子,趔趄著腳兒就往外走。婦人道:「你過來,我問你,莫非你與他停眼整宿,在一舖兒長遠睡,惹的那兩個丫頭也羞耻?無故只是睡那一回兒,還教他另睡去。」西門慶道:「誰和他長遠睡?」說畢就走。婦人又叫回來,說道:「你過來,我分付你,慌走怎的?」西門慶道:「又說甚麼?」婦人道:「我許你和他睡便睡,不許你和他說甚閒話,教他在俺每跟前欺心大膽的。我到明自打聽出來,你就休要進我這屋裡來,我就把你下截咬下來!」西門慶道:「怪小淫婦兒,瑣碎死了!」一直走過那邊去了。春梅便向婦人道:「由他去,你管他怎的?婆婆口絮,媳婦耳頑,倒沒的教人與你為仇結仇,誤了咱娘兒兩個下棋。」一面叫秋菊關上角門,放卓兒擺下棋子。婦人問:「你姥姥睡了?」春梅道:「這咱哩,後邊散了,來到屋裡就睡了。」這裡房中春梅與婦人下棋不題。且說西門慶走過李瓶兒房內,掀開一簾子,如意兒正與迎春、綉春炕上吃飯。見了西門慶慌的跳起身來,西門慶道:「你每吃飯吃飯。」于是走出明間李瓶兒影跟前一張交椅下坐下不一時,只見如意兒笑嘻嘻走出來,說道:「爹,這裡冷,你往屋裡坐去罷。」這西門慶一把手摸到懷裡,摟過來就親了個嘴,一面走到房中牀正面坐了。火爐上頓著茶,迎春連忙點茶來吃了。如意兒在炕邊烤著火兒站立,問道:「爹你今日沒酒?外邊散的早?」西門慶道:「我明日還要早船上拜拜蔡知府去,不是也還坐一回。」如意兒道:「爹,你還吃酒?斟酒與爹吃。還有頭里後邊送來與娘供養的一卓菜兒、一素兒金華酒 。湯飯俺每吃了,酒菜還沒敢動,留有預備,只把爹用。」西門慶道:「你每吃了罷了。」分付:「下飯不要別的,好細巧拿幾碟兒來,我不吃金華酒 。」一面教綉春:「你打了燈籠,往花園藏春軒書房內,還有一罈葡萄酒 ,你問王經要了來,斟那個酒我吃。」那綉春應喏,打著燈籠去了。迎春連忙放卓兒,拿菜兒。如意兒道:「姐,你揭開盒子,等我揀兩樣兒與爹下酒。」于是燈下揀了一碟鴨子肉,一碟鴿子鶵兒,一碟銀絲鮓,一碟搯的銀苗豆芽菜,一碟黃芽韮和海蜇 ,一碟燒臟肉釀腸兒 ,一碟黃炒的銀魚,一碟春不老炒冬笋,兩眼春槅,不一時擺在卓上,抹得鍾筯乾淨,放在西門慶面前。良久,綉春前邊取了酒來,打開篩熱了,如意兒斟在鍾內,遞與西門慶嚐了嚐,無比美酒,紅紅的顏色。當下如意兒就挨近在卓上邊立,待奉斟酒,又親剝炒栗子兒與他下酒。那迎春知局,往後邊廚房內與綉春坐去了。這西門慶見無人在跟前,教老婆坐在他膝蓋兒上摟著,與他一遞一口兒吃酒。老婆剝菓仁兒,放在他口裏。西門慶一面解開他穿的玉色袖子對衿襖兒扭扣兒并抹胸兒,露出他白馥馥酥胸,用手揣摸著他奶頭,誇道:「我的兒,你達達不愛你別的,只愛你道好白淨皮肉兒,與你娘的一般樣兒!我摟著你,就如同摟著他一般!」如意兒笑道:「爹沒的說,還是娘的身上白,我見五娘雖好模樣兒,也中中兒的,紅白肉色兒,不如後邊大娘、三娘倒白淨肉色兒,三娘只是多幾個麻兒。倒是他雪姑娘生的清秀又白淨,五短身子兒。」又道:「我有句說話兒對爹說,迎春姐有件正面戴的仙子兒,要與我。他要問爹討娘家常戴的金赤虎,正月里戴。爹與他了罷!」西門慶道:「你沒正面戴的,等我叫銀匠拿金子另打一件與你,你娘的頭面廂兒,你大娘都拿的後邊去了,怎好問他要的?」老婆道:「也罷,你還另打一件赤虎與我罷!」一面走下來就磕頭謝了。兩個吃了半日酒,如意兒道:「爹,你叫姐來,與他一杯酒吃,惹的他不惱麼?」這西門慶便叫迎春,不應。老婆親走到廚房內,說道:「姐,爹叫你哩。」迎春一面到跟前。西門慶令如意兒斟了一甌酒兒與他,又揀了兩筯菜兒放在酒托兒上。那迎春站在傍邊,一面吃了。老婆道:「你叫綉春姐來吃些兒。」那迎春去了,回來說道:「他不吃哩。」走去良久,迎春向炕上抱他舖蓋後邊睡去。迎春道:「我不往後邊,在明間板凳上賣良姜?我與綉春廚房炕上睡去。茶在火上等爹吃,你自家倒倒罷!」如意兒道:「姐,你去帶上後邊門,等我插去。」那迎春抱了被褥,一直後邊去了。這老婆陪西門慶吃了一回酒,收拾家火,點茶與西門慶吃了。插上後門,原來另預備著一牀兒舖蓋,與西門慶睡,都是綾絹被褥,扣花枕頭,在枕上薰的煖烘烘的。老婆便問:「爹,你在炕上睡?牀上睡?」西門慶道:「我在牀上睡罷。」如意兒便把舖蓋抱在牀上舖下,打發西門慶上牀解衣,替他脫了靴襪。他便打了水,拿出明間內澡洗了牝,掩上房門,將燈臺拿在牀邊一張小卓兒上擱放。然後,他方脫了衣褲上牀鑽入被窩裡,與西門慶相摟相抱,並枕而臥。婦人用手捏弄他那說兒,上邊束著托子,猙獰跳腦,又喜又怕,兩個口吐丁香,交摟在一處。西門慶見他仰臥在被窩內。脫的精赤條條,恐怕凍著他,取過他的抹胸兒替他蓋著胸膛上,兩手執其兩足,極力抽提。老婆氣喘吁吁,被他{入日}得面如火熱,又道:「這袵腰子,還是娘在時與我的。」西門慶道:「我的心肝,不打緊處。到明日舖子裡,拿半個紅段子,與你做小衣兒穿,再做雙紅段子睡鞋兒穿在腳上,好伏侍我。」老婆道:「可知好哩!爹與了我,等我閒著做。」西門慶道:「我只要忘了,你今年多少年紀?你姓甚麼?排行幾姐?我只記你男子漢姓熊。」老婆道:「他便性熊,叫熊旺兒。我娘家姓章,排行第四,今年三十二歲。」西門慶道:「我原來還大你一歲。」一壁幹著,一面口中呼他:「章四兒,我的兒,你用心伏侍我,等明日你大娘生了孩兒,你好生看奶著,你若有造化,也生長一男半女,我就扶你起來,與我做一房小,就頂你娘的窩兒;你心下如何?」老婆道:「奴男子漢已是沒了,娘家又沒人,奴情愿一心只伏侍爹,再有甚麼二心,就死了不出爹這門!若爹可憐見,可知好哩!」這西門慶見他言語兒投著機會,心中越發喜歡,揝著他雪白的兩隻腿兒,穿著一雙綠羅扣花鞋兒,只顧沒稜露腦,兩個搧幹抽提。抽提的老婆在下無般不叫出來,嬌聲怯怯,星眼濛濛。良久,卻令他馬伏在下,且舒雙足,西門慶披著紅綾被,騎在他身上,投那話入牝中。燈光下兩手按著他雪白的屁股,只顧搧打,口中叫:「章四兒,你好去叫著親達達,休要住了,我丟與你罷!」那婦人在下舉股相就,真個口中顫聲柔語,呼叫不絕。足頑了一個時辰,西門慶方纔精泄。良久拽出塵柄來,老婆取帕兒替他搽拭,摟著睡到五更雞叫時分散。老婆又替吮咂。西門慶告他說:「你五娘怎的替我咂半夜,怕我害冷,連尿也不教我下來溺,都替我〈口厭〉了。」老婆道:「不打緊,等我也替爹吃了就是了。」這西門慶真個把胞膈尿都溺在老婆口內,當下兩個婍妮溫存,萬千囉躁,{入日}搗了一夜。次日,老婆先起來開了門,預備盆中,打發西門慶穿衣梳洗出門。到前邊分付玳安:「早教兩名排軍,把捲棚正面放的流金八仙鼎,寫帖兒擡送到宋御史老爹察院內交付明白,討回帖來。」又教陳經濟封了一疋金段,一疋色段,教琴童毡包內拿著,預備下馬,要早往清河口拜蔡知府去。正在月娘房內吃粥,月娘問他:「應二哥那裡,俺每莫不都去?也留一個兒在家裡看家,留下他姐在家陪大妗子做伴兒罷。」西門慶道:「我已預備下五分人情,你的是一方兜肚,一個金墜兒,五錢銀子。他四個每人都是二錢銀子,一方手帕,都去走走罷。左右有大姐在家陪大妗子,就是一般。我已許下應二,都往他家去來。」月娘聽了,一聲兒沒言語。李桂姐便拜辭說道:「娘,我今日家去罷。」月娘道:「慌去怎的?再住一日兒不是?」桂姐道:「不瞞娘說,俺媽心裡不自在,俺姐不在,家中沒人,改日正月間來住兩日兒罷。」拜辭了西門慶。月娘裝了兩個茶食盒子,與桂姐一兩銀子,吃了茶,打發出門。西門慶纔穿上衣服,往前邊去,忽有平安兒來報:「荊都監老爹來拜。」西門慶即出迎接,至廳上敍禮。荊都監穿著補服員領,戴著暖耳,腰繫金帶,叩拜堂上道:「久違欠恭,高轉失賀之意。」西門慶道:「多承厚貺,尚未奉賀。」敍畢契闊之情,分賓主坐下。左右獻上茶湯,荊都監便道:「良騎俟候何往?」西慶道:「京中太師老爺第九公子九江蔡知府,昨日巡按宋公祖與工部安鳳山、錢雲野、黃秦宇都借學生這裡作東,請他一飯。蒙他昨日具拜帖與我,我豈可不回拜他拜去?誠恐他一時起身去了。」荊都監道:「正是小弟一事來奉凟兒,巡按宋公過年正月間差滿,只怕年終舉劾地方官員,望乞四泉借重,與他一說,聞知昨日在宅上吃酒,故此斗膽恃愛。倘得寸進,不敢有忘。」西門慶道:「此是好事,你我相厚,敢不領命。你寫個說帖來,幸得他後日還有一席酒在我這裡,等我抵回和他說,又好些。」這荊都監連忙下坐位來,又與西門慶打一躬:「多承盛情,啣結難忘!」便道:「小弟已具了履歷手本在此。」一面喚椽房寫字的取出,荊都監親手遞上與西門慶觀看。上面寫著:「山東等處兵馬都監清河左衛指揮僉事荊忠,年三十二歲,係山後檀州人。由祖後軍功累陞本衛左所正千戶。從某年由武舉中式,歷陞今職,管理濟州兵馬歷年餘。」文一一開載明白。西門慶看畢,荊都監又向袖中取出禮物來遞上,說到:「薄儀望乞笑留。」西門慶見上面寫著:白米二百石,說道:「豈有此理!這個學生斷不敢領。以此視人,相交何在?」荊都監道:「不然,總然四泉不受,轉送宋公,也是一般;何見拒之深耶?倘不納,小弟赤不敢奉凟。」推阻再三,西門慶只得收了,說道:「學生暫且收下。」一面接了,說道:「學生明日與他說了,就差人回報。」茶湯兩碗,荊都監拜謝起身去了。西門慶分付平安:「我不在,有甚人來拜望,帖兒接下,休往那去了,派下四名排軍把門。」說畢就上馬,琴童跟隨,拜蔡知府去了。卻說玉簫早辰打發西門慶出門,走到金蓮房中,說:「五娘,昨日怎的不往後邊去坐?晚夕眾人聽薛姑子宣黃氏女卷,坐到那咱晚。落後二娘管茶,三娘房裡又拿將酒菜來,都聽桂姐、申二姐賽唱曲兒。到有三更時分俺每纔睡。俺娘好不說五娘哩,五娘聽見爹前邊散了,往屋裡走不送。昨日三娘生日,就不放往他屋裡走兒,把攔的爹恁緊。三娘道:『沒的羞人子刺刺的,誰耐煩爭他?左右是這幾房兒隨他串去!』」金蓮道:「我待說,就沒好口,{入日}瞎了他眼來!昨日你道他在我屋裡睡來麼?」玉筲道:「前邊老大這娘屋裡,六娘又死了,爹卻往誰屋裡去?」金蓮道:「雞兒不撒尿,各自有去處。死了一個,還有一個頂窩兒的。」這玉筲又說:「俺娘怎的惱五娘,問爹討皮襖不對他說。落後爹送鑰匙到房裡,娘說了爹幾句好的:『李大姐死了,嗔俺分散他的丫頭,多少時兒,相你把他心愛的皮襖拿了與人穿,就沒話兒說了。』爹說:『他見沒皮襖穿。』娘說:『他怎的沒皮襖?放著皮襖他不穿,坐名兒只要他這件皮襖。早是死了,便指望他的;他不死,你敢指望他的!』金蓮道:「沒的那扯〈毛皮〉淡!有了一個漢子做主兒罷了,你是我婆婆,你管著我?我把攔他,我拿繩子拴著他腿兒不成!把攔他一面兒罷了,偏有那些〈毛皮〉聲浪氣的!」玉筲道:「我來對娘說,娘只放在心裡,休要說出我來。今日桂姐也家去,俺娘收拾戴頭面哩。今日要留下雪娥在家裡與大妗子做伴兒,俺爹不肯,都封下人情,五個人都教去哩。娘也快些收拾了罷!」說畢,玉筲後邊了。這金蓮向鏡臺前搽胭抹粉,插花戴翠,又使春梅後邊問玉樓:「今日穿甚顏色衣裳?」玉樓道:「你爹嗔換孝,都教穿淺淡色衣服。」這五個婦人會定了,都是白䯼髻珠子箍兒,用翠藍綃金綾汗巾兒搭著,頭上珠翠堆滿。銀紅織金段子對衿襖兒,藍段子裙兒。惟吳月娘戴著白縐紗金梁冠兒,海獺臥免兒珠子箍兒,胡珠環子,上穿著沉香色遍地粧花補子襖兒,紗綠遍地金裙。一頂大轎,四頂小轎,排軍喝路,轎內安放銅火踏。王經、棋童、來安三個跟隨,拜辭了吳大妗子、三位師父、潘姥姥,逕往應伯爵家吃滿月滿去了不題。卻說前邊如意兒和迎春,有西門慶晚夕吃酒的那一卓菜,安排停當,還有一壺金華酒 ,向罈內又打出一壺葡萄酒 來,午間請了潘姥姥、春梅、郁大姐彈唱著,在房內四五個做一處吃。到中間,也是合當有事,春梅道:「只說申二姐會唱的好挂真兒」,沒個人往後邊去,便叫他來到,好歹教他唱個挂真兒咱每聽。」迎春纔待使綉春叫去,只見春鴻走來向著火,春梅道:「賊小蠻囚兒,你原來今日沒跟了轎子去?」春鴻道:「爹派下教王經去了,留我在家裡看家。」春梅道:「賊小蠻囚兒,你不是凍的,還不尋到這屋裡來烘火?」因叫迎春:「你釃半甌子酒與他吃。」分付:「你吃了,替我後邊叫將申二姐來,你就說我要他唱個兒與姥姥聽。」那春鴻連忙把酒吃了,一直走到後邊。不想申二姐伴著大妗子、大姐、三個姑子、玉筲都在上房裡坐的,正吃芫荽芝蔴麻茶哩 。忽見春鴻掀簾子進來,叫道:「申二姐你來,俺大姑娘前邊叫你唱個兒與他聽去哩。」這申二姐道:「你大姑在這裡,又有個大姑娘出來了?」春鴻道:「是俺前邊春梅姑娘這裡叫你。」申二姐道:「你春梅姑娘他稀罕怎的,也來叫的我?有郁大姐在那里也是一般。這裡唱與大妗、奶奶聽哩。」大妗子道:「也罷,申二姐你去走走再來。」那申二姐坐住了不動身。春鴻一直走到前邊,對春梅說:「我叫他,他不來哩。都在上房坐著哩。」春梅道:「你說我叫他,他就來了。」春鴻道:「我說你叫他來:『前邊大姑娘叫你。』他意思不動,說道:『大姑娘在這裡,那裡又鑽出個大姑娘來了?』我說是春梅姑娘。他說:『你春梅姑娘他從幾時來,也來叫我?我不得閒,在這裡唱與大妗、奶奶聽哩。』大妗、奶奶到說:『你去走走再來。』他不肯來哩。」這春梅不聽便罷,聽了三尸神暴跳,五臟氣沖天,一點兒紅從耳畔起,須臾,紫遍了雙腮。眾人攔阻不住,一陣風走了上房裡,指著申二姐一頓大罵道:「你怎麼對著小廝說我那里又鑽出個大姑娘來了?稀罕他,也敢來叫我!你是甚麼總兵官娘子?不敢叫你!俺每在那毛裡夾著來,是你擡舉起來?如今從新鑽出來了,你無非只是個走千家門、萬家戶賊狗攮的瞎淫婦!你來俺家,纔走了多少時兒,就敢恁量視人家?你會曉的甚麼好成樣的套數唱?左右是那幾句,東溝籬,西溝壩,油嘴狗舌,不上紙筆的,那胡歌錦詞,就拏斑做勢起來!真個就來了俺家本司三院唱的老婆,不知見過多少,稀罕你這個兒,韓道國那淫婦家興你,俺這里不興你。你就學那淫婦,我也不怕。你好不好趁早兒去!賈媽媽與我離門離戶!」那大妗子攔阻說道:「快休要舒口。」把這申二姐罵的睜睜的,敢怒而不敢言,說道:「爹嚛嚛!這位大姐,怎的恁般粗魯性兒?就是剛纔對著大官兒,我也沒曾說甚歹。這般潑口言語瀉出來,此處不留人,也有留人處。」春梅越發惱了,罵道:「賊{入日}遍街搗遍巷的瞎淫婦!你家有恁好大姐,比是你有恁性氣,不該出來往人家求衣食,唱與人家聽。趁早兒與我走,再也不要來了!」申二姐道:「我沒的賴在你家?」春梅道:「賴在我家,教小廝把鬢毛都撏光了你的!」大妗子道:「你這孩兒,今日怎的甚樣兒的?還不往前邊去罷。」那春梅只顧不動身。這申二姐一面哭哭啼啼下炕來,拜辭了大妗子,收拾衣裳包子,也等不的轎子來,央及大妗子使平安對過叫將畫童兒來,領他往韓道國家去了。春梅罵了一頓,往前邊去了。大妗子看著大姐和玉筲說道:「他敢前邊吃了酒進來?不然如何恁沖言沖語的,罵的我也不好看的了。你教他慢慢收拾了去就是了,立逼著攆他去了,又不叫小廝領他,十分水深人不過,卻怎樣兒的,卻不急了人!」王筲道:「他們敢在前頭吃酒來?」卻說春梅到前邊,還氣狠狠的,向眾人說道:「乞我把賊瞎淫婦一頓罵,立攆了去了。若不是大妗子勸著我,臉上與這賊瞎淫婦兩個耳刮子纔好!他還不知道我是誰哩,叫著他張兒致兒,拿斑做勢兒的!」迎春道:「你砍一枝損百株,忌口些!郁大姐在這里,你卻罵瞎淫婦人。」春梅道:「不是這等說。像郁大姐在俺家這幾年,先前他還不知怎麼的,大大小小,他惡訕了那個人兒來?教他唱個兒他就唱,那里像這賊瞎淫婦大膽?不道的會那等腔兒!他再記的甚麼成樣的套數,還不知怎的拿斑兒!左來右去,只是那幾句山坡羊、瑣南枝,油里滑言語,上個甚麼擡盤兒也怎的!我纔乍聽這個曲兒也怎的!我見他心裡就要把郁大姐掙下來一般!」郁大姐道:「可不怎的!昨日晚夕大娘多教我唱小曲兒,他就連忙把琵琶奪過去他要唱。大娘說:『郁大姐,你教他先唱,你後唱罷!』」郁大姐道:「大姑娘,你休怪他,他原不知道咱家深淺。他還不知把你當進人看成好容易!」春梅道:「我剛纔不罵的你?你覆韓道國老婆那賊淫婦,你就學與他,我也不怕他!」潘姥姥道:「我的姐姐,你沒要緊,氣的恁樣兒的!」如意兒道:「等我傾杯兒酒,與大姐姐消消惱。」迎春道:「我這女兒,有惱就是氣。」便道:「郁大姐,你揀套好曲兒唱個伏侍他。」這郁大姐拿過琵琶來,說道:「等我唱個『鶯鶯鬧臥房』山坡羊兒,與姥姥和大姑娘聽罷。」如意兒道:「你用心唱,等我斟上酒。」那迎春拿起杯兒酒來,望著春梅道:「罷罷,我的姐姐,你著氣就是惱了,胡亂且吃你媽媽這鍾酒兒罷。」那春梅忍不住笑罵迎春,說道:「怪小淫婦兒,你又做起我媽來了!」說道:「郁大姐,休唱山坡羊,你唱個兒江兒水俺每聽罷!」這郁大姐在傍彈著琵琶唱:

「花家月艷,減盡了花容月艷,重門常是俺。正東風料峭,細雨連纖,落紅千萬點。香串懶重添,針兒怕待拈。瘦損喦喦,鬼病懨懨。俺將這舊恩情重檢點。愁壓損,兩眉翠尖,空惹的張郎憎厭。這些時,對鶯花不捲簾。」

「槐陰庭院,靜悄悄槐陰庭院,芭蕉新乍展。見鶯黃對對。蝶粉翩翩,情人天樣遠。高柳噪新蟬,清波戲彩鴛。行過闌前,坐近他邊,則听得是誰家唱採蓮。急攘攘,愁懷萬千。拈起柄香羅紈扇,上寫阮郎歸詞半篇。」

「炎蒸天氣,挨過了炎蒸天氣,祈涼人綉幃。怪燈花相照,月色相隨,影伶仃訴與誰。征雁向南飛,雁歸人未歸。想像腰圍,做就寒衣,又不知他在那里貪戀著?並無個真實信息。倩一行人稍寄,只恐怕路迢遙衣到遲。」

「梅花相問,幾遍把梅花相問,新來瘦幾個。笑香消容貌,玉減精神,比花枝先瘦損。翠被懶重溫,爐香夜夜薰。著意溫存,斷夢勞魂,這些時睡不安眠不穩。枕兒冷,燈兒又昏。獨自個向誰評論?百般的放不下心上的人。」

這里彈唱吃酒不題。西門慶從新河口拜了蔡九知府回來下馬。平安就稟:「今日有衙門里何老爹差答應的來,請爹明日早進衙門中拿了一起賊情審問。又本府胡老爹送了一百本新曆日,荊都監老爹差了家人送了一口鮮豬,一罈豆酒,又是四封銀子。姐夫收下了,沒敢與他回帖兒,等爹來打發。晚上他家人還來見爹說話哩。只胡老爹家與了回帖,賞了來一錢銀子。又是喬親家爹送帖兒,明日請爹吃酒。」玳安兒又拏宋御史回帖兒來回話:「小的送到察院內,宋老爹說明日還奉價過來。賞了小的并擡盒人五錢銀子,一百本曆日。」西門慶叫了陳經濟來,問了四包銀子,已久交到後邊去了。西門慶走到廳上,春鴻連忙報與春梅眾人,說道:「爹來家了,還吃酒哩!」春梅道:「怪小蠻囚兒,爹來家,隨他來去,管俺每腿事!沒娘在家,他也不往俺這邊來。」眾人打夥兒吃酒頑笑,只顧不動身。西門慶到上房,大妗子、三個姑子都往這邊屋里坐的。玉筲向前與他接了衣裳坐下,放卓兒打發他吃飯。教來興兒定卓席,三十日與宋巡按擺酒,與巡撫侯爹送行。初一日宰豬羊,家中祭祀,還願心的。初三日請劉、薛二內相,帥府周爺眾位吃慶官酒。分付已了,玉筲在傍,請問:「爹,你吃酒放卓兒,釃甚麼酒你吃?」西門慶道:「有菜兒擺上來,有剛纔荊都監送來的那豆酒取來,打開我嚐嚐看好不好吃。」只見來安兒來家回話。玉筲連忙便提酒來,打破泥頭,傾在鍾內,遞與西門慶呷了一呷,碧靛般清,其味深長。西門慶令:「斟來我吃。」須臾,擺上菜來,西門慶在房中。卻說來安同排軍拿了兩個燈籠,晚夕接了月娘來家。月娘便穿著銀鼠皮披藕金段襖兒,翠藍裙兒。李嬌兒等,都是貂鼠皮襖,白綾襖兒,紫丁香色織金裙子。原來月娘見金蓮穿著李瓶兒皮襖,把金蓮舊皮襖與了孫雪娥穿了,都到上房拜了西門慶。惟雪娥與西門慶磕頭起來,又與月娘磕頭。都過那邊屋裡去了,拜大妗子、三個姑子。月娘便坐著與西門慶說話,說:「應二嫂見俺每都去,好不喜歡!酒席上有隔壁馬家娘子和應大嫂、杜二娘,也有十來位堂客,叫了兩個女兒彈唱。養了好個平頭大臉的小廝兒,原來他房裡春花兒比舊時黑瘦了好些,只剩下個大驢臉一般的,也不自在哩!那時節亂的他家裡大小不安,本等沒人手。臨來時,應二哥與俺每磕頭,謝了又謝。多多上復你:多謝重禮。」西門慶道:「春花兒那成精奴才,也打扮出來見人?」月娘道:「他比那個沒鼻子,沒眼兒?是鬼兒,出來見不的!」西門慶道:「那奴才撒把黑豆,只好教豬拱罷!」月娘道:「我就聽不上你恁說嘴。自你家的好,拿掇的出見的人!」那王經在傍,他立著說道:「俺應二爹見娘們去,先頭上不敢出來見,躲在下邊房裡,打窗戶眼兒望前瞧。被小的看見了,說道:『你老人家沒廉恥,平白瞧甚麼?』他趕著小的打。」西門慶笑的沒眼縫兒,說道:「你看這賊花子!等明日他來着,老實抹他一臉粉!」王經笑道:「小的知道了!」月娘喝著:「這小廝便要胡說!他幾時瞧來?平白枉口拔舌的!一日誰見他個影兒,只臨來時,纔與俺每磕頭。」王經站了一回出來了。月娘起身過這邊屋裡,拜大妗子并三個師父。西門大姐與玉筲眾丫頭媳婦都來磕頭。月娘便問:「怎的不見申二姐?」眾人都不做聲。玉筲說:「申二姐家去了。」月娘道:「他怎的不等我來,先就家去?」大妗子隱瞞不住,把春梅罵他之事說了一遍。月娘就有幾分惱,說道:「他不唱便罷了,這丫頭慣的沒張倒置的,平白罵他怎麼的?怪不的俺家主子也沒那正主子,奴才也沒個規矩,成甚麼道理!」望著金蓮道:「你也管他管兒,慣的通沒些摺兒!」金蓮在傍笑著說道:「也沒見這個瞎曳麼的,風不搖,樹不動;你走千家門、萬家戶,在人家無非只是唱。人叫你唱個兒,也不失了和氣,誰教他拏斑兒做勢的?他不罵的他嫌腥!」月娘道:「你倒且是會說話兒的!合理都像這等,好人歹人,都乞他罵了去,也休要管他一管兒了?」金蓮道:「莫不為瞎淫婦,打他幾棍兒?」月娘聽了他這句話,氣的把臉通紅了,說道:「慣著他明日把六鄰親戚,都教他罵遍了罷!」于是起身,走過西門慶這邊來。西門慶便問:「怎麼的?」月娘道:「情知是誰!你家使的好規矩的大姐,如此這般把申二姐罵的去了!」對西門慶說。西門慶笑道:「誰教他不唱與他聽來?也不打緊處,到明日使小廝送一兩銀子補伏他,也是一般。」玉筲道:「申二姐盒子還在這里,沒拿去哩!」月娘見西門慶笑,說道:「不說叫將他來,嗔喝他兩句。虧你還雌著嘴兒,不知笑的是甚麼!」玉樓、李嬌兒見月娘惱起來,都先歸去房裡。西門慶只顧吃酒。良久,月娘進裡間內脫衣裳、摘頭,便問玉筲:「這廂上四包銀子,是那里的?」西門慶說:「是荊都監送來幹事的二百兩銀子。明日要央宋巡按圖幹陞轉。」玉筲道:「頭里姐夫送進來,我放在箱子上,就忘了對娘說。」月娘道:「人家的,還不收進櫃裡去哩。」玉筲一面安放在廚櫃中不題。金蓮在那邊屋里,只顧坐的,等著西門慶一答兒往前邊去,今日晚夕要吃薛姑子符藥與他交姤,圖任子日好生子。見西門慶不動身,走來掀著簾兒叫他,說:「你不往前邊去?我等不的你,我先去也!」西門慶道:「我兒,你先走一步兒,我吃了這些酒就來。」那金蓮一直往前邊去了。月娘道:「我偏不要你去,我還和你說話哩!你兩人合穿著一條褲也怎的?是強汗世界,巴巴走來我這屋裡,硬來叫他!沒廉耻的貨!自你是他老婆,別人不是他的老婆?」因說西門慶:「你這賊皮搭行貨子,怪不的人說你。一視同仁都是你的老婆,休要顯出來便好,就吃他在前邊把攔住了!從東京來,通影邊兒不進後邊歇一夜兒,教人怎麼不惱你?冷竈著一把兒,熱竈著一把兒纔好。通教他把攔住了!我便罷了,不和你一般見識;別人他肯讓的過?口兒內雖故不言語,好殺他心兒裡有幾分惱!今日孟三姐在應二嫂那里,通一日恁甚麼兒沒吃。不知掉了口冷氣,只害心淒惡心!來家,應二嫂遞了兩鍾酒,都吐了。你還不往他屋里瞧他瞧去?」這西門慶聽了,說道:「真個他心裡不自在?」分付:「收了家火罷,我不吃酒了。」于是走到玉樓房中,只見婦人已脫了衣裳,摘去首飾,渾衣兒歪在炕上,正倒著身子嘔吐。蘭香便熱煤炭在地。西門慶見他呻吟不止,慌問道:「我的兒,你心裡怎麼的來?對我說,明日請人來看。」婦人一聲不言,只顧嘔吐。被西門慶一面扶起他來,與他坐的。見他兩隻手只揉胸前,便問:「我的心肝,你心裡怎麼?你告訴我。」婦人道:「我害心凄的慌,你問他怎的?你幹你那營生去!」西門慶道:「我不知道,剛纔上房對我說,我纔曉的。」婦人道:「可知你曉的,俺每不是你老婆,你疼心愛的去了。」西門慶于是摟過粉項來,就親個嘴,說道:「怪油嘴,就徯落我起來!」便叫蘭香:「快頓好苦艷茶兒來與你娘吃。」蘭香道:「有茶伺候著哩。」一面捧茶上來。西門慶親手拿在他口兒邊吃。婦人道:「拏來等我自家吃。會那等喬劬勞,旋蒸勢賣兒的,誰這里爭你哩!今日日頭打西出來,稀罕往俺這屋裡來走一走兒?也有這大娘,平白你說他,爭出來糊包氣。」西門慶道:「你不知我這兩日,七事八事,心不得個閒。」婦人道:「可知你心不得閒,可不了一了心愛的扯落著你哩!把俺每這僻時的貨兒,都打到揣了號聽題去了。後十年挂在你那心裡!」見西門慶嘴搵著他香腮,便道:「吃的那爛酒氣,還不與我過一邊去!人一日黃湯辣水兒,誰嚐嚐著來?那里有甚麼神思且和你兩個纏!」西門慶道:「你沒吃甚麼兒?叫丫頭拿飯來咱每吃,我也還沒吃飯哩。」婦人道:「你沒的說。人這里淒疼的了不得,且吃飯?你要吃,你自家吃去。」西門慶道:「你不吃,我敢不吃了。咱兩個收拾睡去罷,明日早使小廝請任醫官來看你。」婦人道:「由他去,請甚麼任醫官、李醫官,教劉婆子來,吃他服藥也好了。」西門慶道:「你睡下,等我替你心口內撲撒撲撒,管情就好了。你不知道,我專一會揣骨捏病,手到病除。」婦人道:「我不好罵出來,你會揣甚麼病?」西門慶忽然想起昨日劉學官送了十圓廣東牛黃清心蠟丸,那藥酒兒吃下極好。即使蘭香:「問你大娘要,在上房磁罐兒內盛著,就拿素兒帶些酒來。」玉樓道:「休要酒,俺這屋裡有酒。」不一時,蘭香到上房要了兩丸來。西門慶看見篩熱了酒,剝去蠟,裏面露出金丸來,看著玉樓吃下去。西門慶因令蘭香:「趁著酒,你篩一鍾兒來,我也吃了藥罷。」被玉樓瞅了一眼,說道:「就休那汗邪,你要吃藥,往別人房裡去吃。你這里且做甚麼哩!卻這等胡作做,你見我不死來,攛掇上路兒來了,緊教人疼的鬼兒也沒了,還要那等掇弄人!虧你也下般的,誰耐煩和你兩個只顧涎纏!」西門慶笑道:「罷罷,我的兒,我不吃藥了,咱兩個睡罷。」那婦人一面吃畢藥,與西門慶兩個解衣上牀同寢。西門慶在被窩內,替他手撲撒著酥胸,揣摸香乳,一手摟其粉項,問道:「我的親親,你心口這回吃下藥覺好些?」婦人道:「疼便止了,還有些嘈雜。」西門慶道:「不打緊,消一回也好了。」囚說道:「你不在家,我今日兌了五十兩銀子與來興兒,後日宋御史擺酒,初一燒紙還願心,到初三再破兩日工夫,把人都請了罷。受了人家多少人情禮物,只願挨著,也又不是事。」婦人道:「你請也不在我,不請也不在我。明日三十日,我叫小廝來攢帳交與你,隨你交付與六姐,教他管去。也該教他管管兒。卻是他昨日說的,甚麼打緊處,雕佛眼兒便難,等我管。」西門慶道:「你聽那小淫婦兒,他勉強著,緊處他就慌了。亦發擺過這幾席酒兒,你交與他就是了。」玉樓道:「我的哥哥,誰養的你恁乖?還說你不護他,這些事兒就見出你那心裡來了。擺過酒兒交與他,俺每是合死的?像這清早辰,得梳了頭,小廝你來我去,秤銀子換錢,把氣也掏乾了!饒費了心,那個道個是怎的?」西門慶接著道:「我的兒,常這道:『當家三年狗也嫌!』」說著,一面慢慢搊起這一雙腿兒,跨在胳膊上,摟抱在懷裡,揝著他白生生的小腿兒,穿著大紅綾子的綉鞋兒,說道:「我的兒,你達不愛你別,只愛你這兩隻白腿兒。就是普天下婦人,選遍了也沒你這兩隻腿兒柔嫩可愛。」婦人道:「我個說嘴的貨!誰信那綿花嘴兒,可可兒的,就是普天下婦人選遍了沒有來,愁好的沒有,也要千取萬不說俺每皮肉兒粗糙,你拿左話兒來右說著哩!」西門慶道:「我的心肝,我有句謊就死了我!」,婦人道:「怪行貨子,沒要緊賭什麼誓!」這西門慶說著,把那話帶上銀托子,插放入他牝中。婦人道:「我說你行行就下道兒來了。」便道:「且住,賊小肉兒!不知替我拿下了不曾沒有?」遂伸手,向牀褥子底下,摸出絹子來,預備著抹搽,因摸見銀托子,說道:「從多咱三不知就帶上這行貨子了,還不趁早除下來哩。」那西門慶那里肯依,抱定他一隻腿在懷裡,只顧沒稜露腦,淺抽深送,須臾淫水浸出,往來有聲,如狗嗏鏹子一般。婦人不面用絹子抹之,隨抹隨出,口裏內不住的作柔顫聲,叫他:「達達,你省可往裏去,奴這兩日好不腰酸,下邊流白漿子出來!」西門慶道:「我到明日,問任醫官討服暖藥來,你吃就好了。」不說兩個在牀上歡娛頑耍。早表吳月娘在上房陪著大妗子、三位師父,晚夕坐的說話,因說起春梅怎的罵申二姐,罵的哭涕,又不容他坐在轎子去。旋央及大妗子對叫過畫童兒,送到他往韓道國家去。大妗子道:「本等春梅出來的言語粗魯,饒我那等說著,還鎗截的言語罵出來,他怎的不急了?他平昔不曉的恁口潑罵人。我只說他吃了酒!」小玉道:「他每五個在前頭吃酒兒進來。」月娘道:「恁不合理的行貨子,生生把個丫頭慣的恁沒大沒小,上頭上臉的!還嗔人說哩!到明日,不管好歹,人都乞他罵了去罷!要俺每在屋裡做甚麼?一個女兒,他走千家門、萬家戶,教他傳出去好聽!敢說西門慶家那大老婆,也不知怎麼的出來的?亂世不知那個是主子,那個是奴才?不說你們這等慣的沒些規矩,恰似俺每不長俊一般,成個甚麼道理!」大妗子道:「隨他去罷。他姑夫不言語,好惹氣?」當夜無語,歸到房中。次日西門慶早起往衙門中去了。這潘金蓮見月娘攔了西門慶不放了,又誤了壬子日期,心中甚是不悅。次日老早使來安叫了頂轎子,把潘姥姥打發往家去了。吳月娘早辰起來,三個姑子要辭家去。月娘每個一盒茶食,與了五錢銀子。又許下薛姑子正月裏庵裡打齋,先與他一兩銀子請香燭紙馬。到臘月還送香油 白麵細米素食,與他齋僧供佛。因擺下茶,在上房內管待,間大妗子一巡吃。先請了李嬌兒、孟玉樓、大姐都坐下,問玉樓:「你吃了那蠟丸,心口內不疼了?」玉樓道:「今早吐了兩口酸水纔好了。」叫小玉:「往前邊請潘姥姥和五娘來吃點心。」玉筲道:「小玉在後邊蒸點心哩,我去請罷。」于是一直走到前邊金蓮房中,便問:「姥姥怎的不見?後邊請姥姥和五娘吃茶哩。」金蓮道:「他今日早辰我打發他家去了。」玉筲道:「怎的不說聲,三不知就去了?」金蓮道:「住人心淡,只顧住著怎的?也住了這幾日子。他家中丟著孩子,也沒人看。我教他家去了。」玉筲道:「我拿了塊臘肉兒,四個甜醬瓜茄子 ,與他老人家,誰知他就去了?五娘,你替他老人家收著罷。」于是遞與秋菊,放在抽屜內。這玉筲便向金蓮說道:「昨日晚夕五娘來了,俺娘如此這般了,對著爹,好不說五娘強汗世界,與爹兩個拿穿著一條褲子,沒廉耻,怎的把攔著爹在前邊,不放後邊來。落後把爹打發三娘房裡歇了一夜。又對著大妗子、三位師父,怎的說五娘慣著春梅沒規矩,毀罵申二姐。爹到明日,還要送一兩銀子與申姐姐遮羞。」一五一十,說了一遍。這金蓮聽說在心。玉筲先來回月娘說:「姥姥起早往家去了,五娘便來也。」月娘便望著大妗子說道:「你看昨日說了他兩句,今日使性子也不進來說聲兒,老早說打發他娘去了。我猜姐姐管情又不知心裡安排著,要起甚麼水頭哩!」當下月娘自在屋裡說話,不防金蓮暗走到明間簾下聽覷多時了。猛可開言說道:「大娘說的,我打發了他家去,我好把攔漢子!」月娘道:「是我說來你如今怎麼的?我本等一個漢子,從東京來了,成日只把攔在你那前頭,道不來後邊傍個影兒!原來只你是他的老婆,別人不是他的老婆?行動題起來,別人不知道,我知道。就是昨日李桂姐家去了,大妗子問了聲:『李桂姐住了一日兒,如何就家去了?他姑夫因為甚麼惱他?』教我還說:『誰知為甚麼惱他?』你便就攩著頭兒說:『別人不知道,自我曉的。』你成日守著他,怎麼不曉的?」金蓮道:「他不來往我那屋裡去,我成日莫不拿豬毛繩子套他去不成?那個浪的慌了也怎的!」月娘道:「你不浪的慌?你昨日怎的他在屋裡坐好好兒的,你恰似強汗世界一般,掀著簾子,硬著來人叫他前邊去,是怎麼說?漢子頂天立地,吃辛受苦,犯了甚麼罪來,你拿豬毛繩子套他?賤不識高低的貨!俺每倒不言語,只顧趕人不得趕上,一個皮襖兒,你悄悄就問漢子討了穿在身上,挂口兒也不來後邊題一聲兒!都是這等起來,俺每在這屋裡放小鴨兒?就是孤老院裡,也有個甲頭!一個使的丫頭,和他貓鼠同眠,慣的有些摺兒!不管好歹,就罵人。倒說著你嘴頭子不伏個燒埋!」金蓮道:「是我的丫頭也怎的?你每打不是?我也在這里,還多著個影兒哩!皮襖是我問他要來,莫不只為我要皮襖開門來?也拿了幾件衣裳與人,那個你怎的就不說來?丫頭便是我慣了他,我也浪了圖漢子喜歡;像這等的,卻是誰浪?」吳月娘乞他這兩句觸在心上。便紫漒了雙腮,說道:「這個是我浪了!隨你怎的說,我當初是女兒填房嫁他,不是趁來的老婆!那沒廉耻趁漢精便浪,俺每真材實料不浪。」被吳大妗在跟前攔說:「三姑娘,你怎的?快休舒口。」饒勸著,那月娘口裏話紛紛發出來,說道:「你害殺了一個,只少我了!」孟玉樓道:「耶嚛,耶嚛!大娘,你今日怎的這等惱的大發?連累著俺每,一棒打著好幾個人也!沒見這六姐,你讓大姐一句兒也罷了。只顧打起嘴來了!」大妗子道:「常言道:『要打沒好手,廝罵沒好口。』不爭你姊妹們攘開,俺每親戚在這里住著也羞。姑娘你不依我去呀,嗔我這里?叫轎子來,我家去罷!」李嬌兒一面拉住大妗子。那潘金蓮見月娘罵他這等言語,坐在地下,就打滾打臉上自家打幾個嘴巴,頭上䯼髻都撞落一邊。放聲大哭叫起來,說道:「我死了罷,要這命做什麼!你家漢子說條念款說將來,我趁將你家來了?彼時恁的,也不難的勾當。等他來家,與了我休書,我去就是了!你趕人不得趕上!」月娘道:「你看,就是了,潑腳子貨!別人一句兒還沒說出來,你看他嘴頭子就相淮洪一般,他還打滾兒賴人!莫不等的漢子來家,好老婆把我別變了就是了!你放恁個刁兒,那個怕你麼?」那金蓮道:「你是真材實料的,誰敢辨別你!」月娘越發大怒,說道:「好不真材實料,我敢在這屋裡養下漢來?」金蓮道:「你不養下漢,誰養下漢來?你就拿主兒來與我!」玉樓見兩個拌的越發不好起來,一面拉起金蓮,往前邊去罷,卻說道:「你恁的怪刺刺的,大家都省口些罷了,只顧亂起來!左右是兩句話,教他三位師父笑話!你起來,我送你前邊去罷!」那金蓮只顧不肯起來,被玉樓和玉筲一齊扯起來,送他前邊去了。大妗子便勸住月娘,只說道:「娘娘,你身上又不方便,好惹氣?分明沒要緊,你姊妹們歡歡喜喜,俺每在這里住著有光。似這等合氣起來,又不依個勸,卻怎樣兒的?」那三個姑好見嚷鬧起來,打發小姑兒吃了點心,包了盒子,告辭月娘眾人,起來道問訊。月娘道:「三位師父,休要笑話。」薛姑子道:「我的佛菩薩,沒的說,誰家竈內無烟?心頭一點無明火,些兒觸著便生烟。大家儘讓些就罷了!佛法上不說的好:『冷心不動一孤舟,淨埽靈臺正好修。若還繩慢鎖頭鬆,就是萬個金剛也降不住。』為人只把這心猿意馬牢拴住了,成佛作祖,都打這上頭起。貧僧去也,多有打擾菩薩。好好兒的,我回去也。」一面打了兩個問訊。月娘連忙還萬福,說道:「空過師父,多多有慢。另日著人送齋襯去。」即叫大姐:「你和那二娘送送三位師父出來,看狗。」于是打發三個姑子出門。月娘陪大妗子眾人坐著,說道:「你看這回氣的我兩隻胳膊都軟了,手冰冷的。從早辰吃了口清茶,還汪在心裡!」大妗子道:「姑娘,我這等勸你,少攬氣,你不依我。你又是臨月的身子,有甚麼緊!」月娘道:「嫂子,早是你在這里住看著,又是我和他合氣?如今犯夜倒拿住巡更的;我到容了人,人到不肯容我。一個漢子你就通身把攔住了,和那丫頭通同作弊,在前頭幹的那無所不為的事。人幹不出來的,你幹出來!女婦人家,通把個廉耻也不顧!他燈臺不明,自己還張著嘴兒說人浪。想著有那一個在,成日和那一個合氣。對著俺每,千也說那一個的不是。他就是清淨姑姑兒了!單管兩頭和番,曲心矯肚,人面獸心,行說的話兒,就不承認了。賭的那誓諕人子。我洗著眼兒看著他,到明日還不知怎麼樣兒死哩!早時剛纔你每看著,擺著茶兒,還好意等他娘來吃。誰知他三不知的,就打發的去了。就安排著要嚷的心兒,悄悄兒走來這里聽,聽怎的,那個怕你不成?待等那漢子來,輕學重告,把我休了就是了!」小玉道:「俺每都在屋裡守著爐臺站著,不知五娘幾時走來,在明間內坐著,也不聽見他腳步兒响。」孫雪娥道:「他單為行鬼路兒,腳上只穿毡底鞋,你可知聽不見他腳步兒响。想著起頭兒一來時,該和我今日多少氣,背地打夥兒嚼說我,教爹打我那兩頓。娘還說我和他便生好鬬的!」月娘道:「他活埋慣了人,今日還要活埋我哩!你剛纔不見他那等撞頭打滾撒潑兒,一徑使你爹來家知道,管就把我翻倒底下!」李嬌兒笑道:「大娘沒的說,反了世界!」月娘道:「你不知道,他是那九條尾的狐狸精!把好的乞他弄死了,且稀罕我能有多少骨肉兒!你在俺家這幾年,雖是個院中人,不像他久慣牢頭。你看他昨日那等氣勢,硬來我屋裡叫漢子:『你不往前邊去,我等不你,先去。』恰似只他一個人的漢子一般,就占住了。不是我心中不惱,他從東京來了,就不放一夜兒進後邊來。一個人的生日,也不往他屋裡走走兒去。十個指頭,都放在你口內也卻罷了!」大妗子道:「姑娘你耐煩,你又常病兒痛兒的,不貪此事,隨他去罷!不爭你為眾好,與人為怨忌仇。」勸了一回,玉筲安排上飯來,也不吃。說道:「我這回好頭疼,心口內有些惡沒沒的上來。」教玉筲:「那邊炕上放下枕頭,我且倘倘去。」分付李嬌兒:「你每陪大妗子吃飯。」那日郁大姐也要家去,月娘分付裝一盒子點心,與他五錢銀子,打發去了。卻說西門慶衙門中審問賊情,到個午牌時分纔來家,正值荊都監家人討回帖。西門慶道:「多謝你老爹重禮,如何這等計較?你還把那禮扛將回去,等我明日說成了,取家來。」家人道:「家老爹沒分付,教小的怎敢將回去?放在老爹這里,也是一般。」西門慶道:「既恁說,你多上覆,我知道了。」拏回帖,又賞家人一兩銀子。因進上房見月娘睡在炕上,叫了半日,白不答應。問丫鬟都不敢說。

走到前邊金蓮房裡,見婦人蓬頭凳腦,拿着個枕頭睡,問着又不言語,更不知怎的。一面封銀子,打發荊都監家人去了。走到孟玉樓房中問,玉樓隱瞞不住,只得把月娘和金蓮早辰嚷鬧合氣之事,且說一遍。這西門慶慌了,走到上房,一把手把月娘拉起來,說道:「你甚緊?自身上不方便,理那小淫婦兒做什麼?平白和他合甚麼氣?」月娘道:「你看說話哩!我和他合氣?是我便生好鬬尋趁他來?他來尋趁將我來!你問眾人不是?早辰好意擺下茶兒,請他娘來吃。他便使性子把他娘打發去了。走來後邊撑着頭兒,和我兩個嚷。自家打滾撞頭,䯼髻跺遍了,皇帝上位的叫。自是沒打在我臉上罷了!若不是眾人拉勸着,是也打成一塊!他平白欺負慣了人,他心裡也要把我降伏下來!行動就說,你家漢人說條念款,念將我來了,打發了我罷,我不在你家了!一句話兒出來,他就是十句頂不下來。嘴一似淮洪一般,我拿甚麼骨禿肉兒拌的他?一回那潑皮賴肉的,氣的我身子軟癱兒熱化!什麼孩子、李子,就是太子也成不的!如今倒弄的不死不活,心口內只是發帳,肚子往下鱉墜著疼,頭又疼,兩隻胳膊都麻了!剛纔桶子坐了這一回,又不下來。若下來了,乾淨了我這身子!省的死了做帶累肚子鬼!到半夜尋一條繩子,等我吊死了,隨你和他過去!往後沒的又像李瓶兒,乞他害死了罷!我曉的你三年不死老婆,也大悔氣。」這西門慶不聽便罷,越聽了越慌了。一面把月娘摟抱在懷裡,說道:「我的好姐姐,你別要和那小淫婦兒一般見識。他識什麼高低香臭?沒的氣了你,到值了多的!我往前邊罵這賊小淫婦兒去!」月娘道:「你還不敢罵他,還要拿猪毛繩子套你哩!」西門慶道:「你教他說惱了我,乞我一頓好腳!」因問月娘:「你如今心內怎麼的?吃了些什麼兒沒有?」月娘道:「誰嚐著些甚麼兒?大清早辰,纔拏起茶,等著他娘來吃,他就走來和我嚷起來。如今心內只發脹,肚子往下鱉墜著疼,腦袋又疼,兩隻胳膊都麻了。你不信摸我這手,恁半日還沒握過來!」西門慶聽了,只顧跌腳,說道:「可怎樣兒的!快著小廝去請了那請任醫官來,看了討藥去。天晚了,他趕不進門來了。」月娘道:「手不答請什麼任醫官?隨他去,有命活,沒命教他死,纔趁了人的心!什麼好的老婆?是牆上泥壞,去了一層又層。我就死了,把他扶了正就是了!恁個聰明的人兒,當不的家?」西門慶道:「你也耐煩?把那小淫婦兒只當臭屎一般丟著他哩,他怎的?你如今不請任后溪來看你看?一時氣裹住了這胎氣,弄的上不上下不下,怎麼了?」月娘道:「這等,叫劉婆子來瞧瞧,吃他服藥;再不,頭上剁兩針,由他自好了。」西門慶道:「你沒的說那劉婆子老淫婦,他會看甚麼胎產?叫小廝騎馬快請任醫官來看。」月娘道:「你敢去請?你就請了來,我也不出去。」那西門慶不依他,走到前邊,即叫琴童:「快騎馬往門外請那任老爹,緊等著,一答兒就來。」琴童應諾,騎上馬,雲飛一般去了。西門慶只在屋裡廝守著月娘,禁張丫頭,連忙熬粥兒拿上來,勸他吃粥兒,又不吃。等到後晌時分,琴童空回來了,說:「任老爹在府裡上班未回來。他家知道咱這里請,明日也不消咱這里人去,任老爹早就來了。」月娘見喬大戶一替兩替來請,便道:「太醫已是明日來了。你往喬親家那里去罷。這日晚了你不去,惹的喬親家怪。」西門慶道:「我去了,誰看你?」月娘笑道:「你看諕的那腔兒,你去我不妨事。等我消一回兒,慢慢〈門爭〉〈門坐〉著起來,與大妗子坐的吃飯。你慌的是些甚麼?」西門慶令玉筲:「快請你大妗子來和你娘坐的。」又問:「郁大姐在那里?教他唱與娘聽。」玉筲道:「郁大姐往家去,不耐煩了這咱里!」西門慶道:「誰教他去來?留他再住兩日兒也罷了。」赶著玉筲踢了兩腳。月娘道:「他見你家反宅亂要去,你管他腿事?」玉筲道:「正經罵申二姐的倒不踢!」那西門慶只做不聽見,一面穿了衣裳,往喬大戶家吃酒去了。未到起更時分就來家,到了上房,月娘正和大妗子、玉樓、李嬌兒四人坐的。大妗子見西門慶進來,忙走後邊去了。西門慶便問月娘道:「你這咱好些了麼?」月娘道:「大妗子陪了我吃了兩口粥兒,心口內不大十分脹了,還只有些頭疼腰酸。」西門慶道:「不打緊,明日任后溪來看,吃他兩服藥,解散散氣,安安胎,就好了。」月娘道:「我那等樣教你休叫他,你又叫他!白眉赤眼,教人家漢子來做什麼?你明日看我就出去不出去!」因問:「喬親家請你做什麼?」西門慶道:「他說我從東京來了,要與我坐。今日他也費心,整治許多菜蔬,叫兩個唱的,請我那里說甚麼話。落後邀過朱臺官來陪我。我熱著你心裡不自在,吃了幾鍾酒,老早就來了。」月娘道:「好個說嘴的貨,我聽不上你這巧語花言,可可兒就是熱著我來?我是那活佛出現,也不放在你那心左;相死了,終值了個破沙鍋片子!」又問:「喬親家再沒和你說什麼話?」西門慶方告說:「喬親家如今要趁著新例,上三十兩銀子納了儀官。銀子也封下了,教我對胡府尹說。我說不打緊,胡府尹昨日送了我二百本曆日,我還不曾回他禮。等我送禮時,稍了帖了與他,問他討一張儀官劄付來與你就是了。他不肯,他說納些銀子是正理。如今央這里分上討討兒,免上下使用,也省十來兩銀子。」月娘道:「既是他央及你,替他討討兒罷。你沒拿他銀子來?」西門慶道:「他銀子明日送過來,還要賣分禮來,我止住他了。到明日咱備一口豬,一罈酒,送胡府尹就是了。」說畢,西門慶晚夕就在上房睡了一夜。到次日,宋巡按擺酒,後廳筵席治酒,裝定菓品,大清早辰,本府出票撥了兩院三十名官身樂人、兩員伶官,四名排長領著,來西門慶宅中答應。西門慶分付前廳儀門裡,東廂房那里聽候,中廳西廂房與海鹽子弟做戲房。只見任醫官從早辰就騎馬來了。西門慶忙迎到廳上陪坐,道連日闊懷之事。任醫官道:「昨日盛使到,學生該斑,至晚纔來家,見尊票,今日不俟駕而來。敢問何人欠安?』西門慶道:「大賤內偶然有些失調,請后溪一診。」須臾,茶至。吃了茶,任醫官道:「昨日聞得明川說老先生恭喜,容當奉賀。」西門慶道:「菲才備員而已,何賀之有?」吃畢茶,琴童收下盞托去。西門慶分付:「後邊對你大娘說,任老爹來了,明間內收拾。」這琴童應諾,到後邊。大妗子、李嬌兒、孟玉樓都在房內,見琴童來說:任醫官進來,爹分付教收拾明間裡坐。」月娘坐著不動身,說道:「我說不要請他,平白教將人家漢子睜著活眼,把手捏腕的,不知做甚麼?教劉媽媽子來,吃兩服藥由他好了。好這等的搖鈴打鼓散著哩!好與人家漢子喂眼!」玉樓道:「大娘,這已是請人來了,你不出去,卻怎樣的?莫不回了人去不成?」大妗子又在傍邊勸著說:「姑娘,你教他看看你這脉息,還知道你這病源,不知你為甚起氣惱?傷犯了那一經?吃了他藥,替你分理上氣血,安安胎氣。你不教他看,依著你就請了劉婆子來,他曉的甚麼病源脉理?一時躭擱怎了!」月娘方動身梳頭兒,戴上冠兒。玉筲拏了鏡子,孟玉樓跳上炕去,替他拏抿子掠後鬢。李嬌兒替他勒鈿兒,孫雪娥預備拏衣裳。月娘頭上止擺著六根金頭簪兒,戴上臥兔兒。也不搽臉;薄施胭粉,淡掃蛾眉。耳邊帶著兩個金丁香兒,正面關著一件金蟾蜍分心。上穿白後對衿襖兒,插黃寬攔挑綉裙子,襯著綾波羅襪,尖尖趫趫一副金蓮,裙邊紫錦香囊,黃銅鑰匙雙垂綉帶。正是:

「羅浮仙子臨凡世,  月殿嬋娟出畫堂。」

畢竟後來如何,且聽下回分解:

