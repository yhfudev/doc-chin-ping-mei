%# -*- coding: utf-8 -*-
%!TEX encoding = UTF-8 Unicode
%!TEX TS-program = xelatex
% vim:ts=4:sw=4
%
% 以上设定默认使用 XeLaTex 编译,并指定 Unicode 编码,供 TeXShop 自动识别

%第九回 
\chapter{西門慶計娶潘金蓮\KG 武都頭誤打李外傳}


「色膽如天不自由,  情深意密兩綢繆,

只思當日同歡愛,  豈想蕭墻有後憂;

只貪快樂恣悠遊,  英雄壯士報冤仇,

天公自有安排處,  勝負輸贏卒未休。」

話說西門慶與潘金蓮燒了武大靈,換了一身豔色衣服,晚夕安排了一席酒,請王婆來作辭,就把迎兒交付與王婆養活。吩咐等武二回來,只說大娘子度日不過,他娘教他前去,嫁了外京客人去了。婦人箱籠,早先一日都打發過西門慶家去。剩下些破卓壞凳,舊衣裳,都與了王婆,西門慶又將一兩銀子相謝。到次日,一頂轎子,四個燈籠,王婆送親,玳安跟轎,把婦人擡到家中來。那條街上,遠近人家,無有一人不知此事,都懼怕西門慶是個刁徒潑皮,有錢有勢,誰敢來多管。地街上編了四句口號,說得極好:

「堪笑西門不識羞,  先奸後娶醜名留;

轎內坐著浪淫婦,  後邊跟著老牽頭。」

西門慶娶婦人到家,收拾花園內樓下三間,與他做房。一個獨獨小院角門進去,設放花草盆景,白日間人跡罕到,極是一個幽僻去處;一邊是外房,一邊是臥房。西門慶旋用十六兩銀子,買了一張黑漆歡門描金床,大紅羅圈金帳幔,寶象花揀庄,卓椅錦杌,擺設齊整。大娘子吳月娘房裡,使着兩個丫頭,一名春梅,一名玉蕭。西門慶把春梅叫到金蓮房內,令他伏侍金蓮,趕着叫娘。都用五兩銀子,另買一個小丫頭,名喚小玉,服侍月娘。又替金蓮六兩銀子買了一個上灶丫頭,名喚秋菊。排行金蓮做第五房。先頭陳家娘子陪床的名喚孫雪娥,約二十年紀,生的五短身材,有姿色。西門慶與他帶了䯼髻,排行第四;以此把金蓮做個第五房。此事表過不題。這婦人一娶過門來,西門慶家中大小,多不歡喜。看官聽說:世上婦人眼裡火的極多,隨你甚賢慧婦人,男子漢娶小,說不嗔;及到其間,見漢子往他房裡同床共枕,歡樂去了,雖故性兒好殺,也有幾分臉酸心歹。正是:

「可惜團圞今夜月,  清光咫尺別人圓。」

西門慶當下就在婦人房中宿歇,如魚似水,美愛無加。到第二日,婦人梳粧打扮,穿一套豔色衣服,春梅捧茶,走來後邊大娘子吳月娘房裡,拜見大小遞見面鞋腳。月娘在坐上仔細定睛觀看,這婦人年紀不上二十五六,生的這樣標致,但見:

「眉似初春柳葉,常含著雨恨雲愁;臉如三月桃花,暗帶著風情月意。纖腰嬝娜,拘束的燕嬾鶯慵;檀口輕盈,勾引得蜂狂蝶亂。玉貌妖嬈花解語,芳容窈窕玉生香。」

吳月娘從頭看到腳,風流往下跑;從腳看到頭,風流往上流。論風流,如水晶盤內走明珠;語態度,似紅杏枝頭籠曉日。看了一回,口中不言,心內暗道:「小廝每家來,只說武大怎樣一個老婆,不曾看見;今日果然生的標致,怪不的俺那強人愛他!」金蓮先與月娘磕了頭,遞了鞋腳;

月娘受了他四禮,次後李嬌兒、孟玉樓、孫雪娥多拜見,平敍了姐妹之禮,立在傍邊。月娘教丫頭拏個坐兒教他坐。吩咐丫頭媳婦,赶着他叫五娘。這婦人坐在傍邊,不轉睛把眼兒只看吳月娘,約三九年紀;因是八月十五日生的,故小字叫做月娘。生的面若銀盆,眼如杏子,舉止溫柔,持重寡言。第二個李嬌兒,乃院中唱的,生的肌膚豐肥,身體沉重,在人前多咳嗽一聲,上床賴追陪;解數名妓者之稱,而風月多不及金蓮也。第三個就是新娶的孟玉樓,約三十年紀,生的貌若梨花,腰如楊柳;長挑身材,瓜子臉兒,稀稀多幾點微麻,自是天然俏麗。惟裙下雙灣金蓮,無大小之分。第四個孫雪娥,乃房裡出身,五短身材,輕盈體態;能造五鮮湯水,善舞翠盤之妙。這婦人一抹兒多看到在心裡。過三日之後,每日清晨起來,就來房裡,與月娘做針指、做鞋腳,凡事不拏強拏,不動強動。指着丫頭,赶着月娘,一口一聲只叫大娘,快把小意兒貼戀幾次。把月娘喜歡的沒入腳處,稱呼他做六姐。衣服首飾揀心愛的與他,吃飯吃茶,和他同卓兒一處吃。因此,李嬌兒等眾人,見月娘錯敬他,各人都不做喜歡,說:「俺們是舊人,到不理論;他來了多少時,便這等慣了他,大姐好沒分曉!」正是:

「前車倒了千千輛,  後車倒了亦如然;

分明指與平川路,  錯把忠言當惡言。」

且說西門慶娶潘金蓮來家,住着深宅大院,衣服頭面又相趁,二人女貌郎才,正在妙年之際;凡事如膠似漆,百依百隨,淫慾之事,無日無之,按下這裡不題。單表武松八月初旬到了清河縣,且去縣裡交納了回書,知縣看了大喜,已知金銀寶物交得明白,賞了武松十兩銀子,酒食管待他,不必說。武松回到下處,房裡換了衣服鞋腳,帶上一頂新頭巾,鎖了房門,一逕投紫石街來。

兩邊眾鄰舍看見武松回來,都吃了一驚,捏兩把汗,說道:「這番蕭牆禍起!這個太歲歸來,怎肯干休!必然弄出事來。」武松走到哥哥門前,揭起簾子,探身入來,看見迎兒小女在樓穿廊下攆線。說道:「我莫不眼花了?」叫聲嫂嫂也不應,叫聲哥哥也不應。道:「我莫不耳聾!如何不見我哥嫂聲音?」向前便問迎兒小女。那迎兒小女見他叔叔來,諕的不敢言語。武松道:「你爹娘往那裡去了?」迎兒只是哭,不做聲。正問着,隔壁王婆聽得是武二歸來,生怕決撒了,只得走過,幫着迎兒支吾。武二見王婆過來,唱了個喏,問道:「我哥哥往那裡去了?嫂嫂也怎的不見?」那王婆道:「二哥請坐,我告訴你。哥哥自從你去了,到了四月間,得個拙病死了。」武二道:「我哥哥四月幾時死了?得什麼病?吃誰的藥來?」王婆道:「你哥哥四月二十頭,猛可地害急心疼起來;病了八九日,求神問卜,什麼藥吃不到,醫治不好,死了。」武二道:「我哥哥從來不曾有這病,如何心疼便死了?」王婆道:「都頭,都怎的這般說?天有不測風雲,人有旦夕禍福。今早脫下鞋和襪,未審明朝穿不穿,誰人保得常沒事!」武二道:「我哥哥如今埋在那裡?」王婆道:「你哥哥一倒了頭,家中一文錢也沒有,大娘子又是沒腳蟹,那裡去尋墳地做着。虧他左邊一個財主,前與大郎有一面之交,捨助一具棺木,沒奈何放了三日,擡出一把火燒了。」武二道:「今嫂嫂往那裡去了?」婆子道:「他少女嫩婦的,又沒的養贍過日子。胡亂守了百日孝,他娘勸他,前日他嫁了外京人去了。丟下這個業障丫頭子,教我替他養活,專等你回來交付與你,也了我一場事。」武二聽言,沉吟了半晌,便撇下了王婆出門去,逕投縣前下處去。開了門,去門房裡換了一身素淨衣服,便教士兵街上打了一條麻縧,買了一雙綿鞋、一頂孝帽,帶在頭上。又買了些果品、點心、香燭、冥紙、金銀錠之類,歸到哥哥家,從新安設武大郎靈位,安排羹飯。就在卓子上點起燈燭,舖設酒肴,掛起經旛紙繒,那消兩個時辰,安排得端正。約一更已後,武二拈了香,撲番身便拜道:「哥哥陰魂不遠,你在世時為人軟弱,今日死後不見分明;你看若是負屈啣冤,被人害了,托夢與我,兄弟替你報冤雪恨!」把酒一面澆奠了,燒化冥布,武二便放聲大哭。倒還是一路上來的人,哭的那兩家鄰舍,無不恓惶。武二哭罷,將這羹飯酒肴,和士兵、迎兒吃了。討兩條蓆子,教士兵房中傍邊睡,武二把迎兒房中睡;他便把條蓆子,就武大靈卓子前睡。約莫將半夜時分,武二翻來覆去,那裡睡得着?口裡只是長吁氣。那士兵齁齁的,卻是死人一般,挺在那裡。武二扒將起來看時,那靈卓子上,琉璃燈半明半滅。武二坐在蓆子上,自言自語,口裡說道:「我哥哥生時懦弱,死後卻無分明。」說猶未了,只見那靈卓子下,捲起一陣冷風來。但見:

「無形無影,非霧非烟;盤旋似怪風侵骨冷,凜冽如殺氣透肌寒。昏昏暗暗,靈前燈火失光明;慘慘幽幽,壁上紙錢飛散亂。隱隱遮藏食毒鬼,紛紛飄逐影魂旛。」

那陣冷風,逼得武二毛髮皆豎起來。定睛看時,見一個人從靈卓底下鑽將出來,叫道:「兄弟,我死得好苦也!」武二看不仔細,卻待向前再問時,只見冷氣散了,不見了人。武二交跌番在蓆子上坐的,尋思道:「怪哉!是夢非夢?剛纔我哥哥正要報我知道,又被我的神氣冲散了他的魂,想來他這一死,必然不明!」聽那更鼓,正打三更三點;回頭看那士兵,正睡得好。于是咄咄不樂,等到天明卻再理會。胡亂眺了一回,看看五更雞叫,東方將明,士兵起來燒湯。武二洗嗽了,喚起迎兒看家,帶領士兵出了在街上,訪問街坊鄰舍:「我哥哥怎的死了?嫂嫂嫁得何人去了?」那街坊鄰舍,明知此事,都懼怕西門慶,誰肯來管?只說:「都頭不消訪問,王婆在緊隔壁住,只問王婆就知了。」有那多口的說:「賣梨的鄆哥兒與仵作何九二人,最知詳細。」這武二竟走來街坊前去尋鄆哥,不見。那小猴子手裡拏着個柳籠菠羅兒,正糴米回來。武二便叫:「鄆哥,兄弟唱喏。」那小廝見是武二叫他,便道:「武都頭,你來遲了一步兒,須動不得手!只是一件,我的爹六十歲,沒人養贍,我卻難保你們打官司耍子。」武二道:「好兄弟,跟我來。」引他到一個飯店樓上,武二叫過貨買:「造兩分飯來。」武二對鄆哥道:「兄弟,你雖年幼,倒有養家孝順之心;我沒什麼。」向身邊摸出五兩碎銀子,遞與鄆哥道:「你且拏去,與老爹做盤費,我自有用你處;待事務畢了,我再與你十來兩銀子做本錢。你可備細說與我,哥哥和甚人合氣?被甚人謀害了?家中嫂嫂被那一個娶去?你一一說來,休要隱匿!」這鄆哥一手接過銀子,自心裡想道:「這五兩銀子,老爹也勾盤費得三五個月,便陪他打官司也不妨!」一面說道:「武二哥,你聽我說。只怕說與你,休氣苦!」于是把賣梨兒尋西門慶,後被王婆怎地打,不放進去,又怎的幫扶武大捉姦,西門慶怎的踢中了武大,心疼了幾日,不知怎的死了,從頭至尾,訴說了一遍。武二聽了,便道:「你這話說是實麼?」又問道:「我的嫂子嫁與甚麼人去了?」鄆哥道:「你嫂子乞西門慶抬到家,待搗吊底子兒。自還問他實也是虛。」武二道:「你休說謊!」鄆哥道:「我便官府面前,也只是這般說!」武二道:「兄弟,既是如此,討飯來吃。」須臾,大盤大碗吃了飯,武二還了飯錢,兩個下樓來。分付鄆哥:「你回家把盤費交與你老爹,明日早來縣府前與我證一證。」又問:「何九在那裡居住?」鄆哥道:「你這時候尋何九?你未曾來時,三日前走的不知往那裡去了?」這武二放了鄆哥家去。到次日,武二早起,先在陳先生家寫了狀子,走到縣門前,只見鄆哥在此伺候,一直帶到廳上跪下,聲冤起來。知縣看見,認的是武松,便問:「你告什麼?因何聲冤?」武二告道:「小人哥哥武大,被豪惡西門慶與嫂潘氏通奸,踢中心窩,王婆主謀,陷害性命。何九朦朧入殮,燒毀屍傷,見今西門慶霸占嫂在家為妾;見有這個小廝鄆哥是證見,望相公做主則個!」因遞上狀子,知縣接着,便問:「何九怎的不見?」武二道:「何九知情在逃,不知去向。」知縣于是摘問了鄆哥口詞,當下退廳,與佐貳官吏通同商議。原來知縣、縣丞主簿、吏典上下,多是與西門慶有首尾的;因此官吏通同計較這件事,難以問理。知縣出來,便叫武松道:「你也是個本院中都頭,不省得法度?自古捉奸見雙,捉賊見賍,殺人見傷。你哥哥屍首又沒了,又不曾捉得他奸;他今只憑這小廝口內言語,便問他殺人公事,莫非公道忒偏向麼?你不可造次,須要自已尋思!當行即行,當止即止。」武二道:「告稟相公道,這多是實情,不是小人捏造出來的。」知縣道:「你且起來,待我從長計較,可行時,便與你拏人。」武二方纔起來,走出外邊,把鄆哥留在裡面,不放回家。早有人把這件事報與西門慶得知,說武二回來,帶領鄆哥告狀一節。西門慶慌了,卻使心腹家人來保、來旺,身邊袖着銀兩,打點官吏,都買囑了。到次日早辰,武二在廳上,已告稟知縣,催逼拏人。誰想這官人貪圖賄賂,發下狀子來,說道:「武二,你休聽外人挑撥,和西門慶做對頭;這件事欠明白,難以問理。聖人云:『經目之事,猶恐未真;背後之言,豈能全信?』你不可一時造次!」當該吏典在旁,便道:「都頭,你在衙門裡,也曉得法律;但凡人命之事,須要屍傷病物踪五件事俱完,方可推問;你那哥哥屍首又沒了,怎生問理?」武二道:「既然相公不准所告,且卻有理。」收了狀子下廳來。來到下處,放了鄆哥歸家,不覺仰天長歎一聲,咬牙切齒,口中罵淫婦不絕。這漢子怎消洋這一口氣?一直奔到西門慶生藥店前,要尋西門慶廝打。正見他開舖子的傅夥計在木櫃裡面,見武二狠狠的走來聲喏,問道:「大官人在宅上麼?」傅夥計認的是武二,便道:「不在家了,都頭有甚話說?」武二道:「且請借一步說話。」傅夥計不敢不出來,被武二引到僻靜巷口說話。武二番過臉來,用手撮住他衣領,睜圓怪眼,說道:「你要死,卻是要活?」傅夥計道:「都頭在上!小人又不曾觸犯了都頭,都頭何故發怒?」武二道:「你若要死,便不要說;若要活時,你對我實說。西門慶那廝,如今在那裡?我個嫂子被他娶了多少日子?一一說來,我便罷休!」那傅夥計是個膽小之人,見武二發作,慌了手腳,說道:「都頭息怒,小人在他家,每月二兩銀子,顧着小人只開舖子,並不知他閑帳。大官人本不在家,剛纔和一相知,往獅子街大酒樓上吃酒去了,小人並不敢說謊。」武二聽了此言,方纔放了手,大扠步雲飛奔到獅子街來,諕的傅夥計半日移腳不動。那武二逕奔到獅子街橋下酒樓前。且說西門慶正和縣中一個皂棣李外傳;專一在縣在府,綽攬些公事,往來聽氣兒撰錢使。若有兩家告狀的,他便賣串兒;或是官吏打點,他便兩下裡打背。又因此縣中起了他個渾名,叫做李外傳。那日見知縣回出武松狀子,討得這個消息,說來回報西門慶知道,武二告狀不行。一面西門慶讓他在酒樓上飲酒,把五兩銀子送他。正吃酒在熱鬧處,忽然把眼向樓窗下,看武松兇人,從橋下直奔酒樓前來,已知此人來意不善;推更衣,從樓後窗只一跳,順着房山跳下人家後院內去了。那武二奔到酒樓前,便問酒保:「西門慶在此麼?」那酒保道:「西門大官和一相識,在樓上吃酒哩!」武二撥步撩衣,飛搶上樓去。只見一個人坐在正面,兩個唱的粉頭,坐在兩邊;認的是本縣皂隸李外傳,知就來報信的。心中甚怒,向前便問:「西門慶那裡去了?」那李外傳見是武二,諕得謊了,半日說不出來。被武二一腳把卓子踢倒了,碟兒盞兒都打的粉碎;兩個唱的,也諕得走不動。武二匹面向李外傳打一拳來,李外傳叫聲沒呀時,便跳起來立在凳子上,樓後窗尋出路。被武二雙提住,隔着樓前窗,倒撞落在當街心裡來,跌得個發昏。下邊酒保見武二行惡,都驚得呆了,誰敢向前?街上兩邊人多住了腳睜眼。武二又氣不捨,奔下樓;見那人已趺得半死,直挺挺在地,只把眼動。于是兜襠又是兩腳,嗚呼哀哉斷氣身亡!眾人道:「都頭,此人不是西門慶,錯打了他。」武二道:「我問他,如何不說,我所以打他。原來不經打,就死了。」那地方保甲,見人死了,又不敢向前捉武二,只得慢慢挨近上來收籠他,那裡肯放鬆。連酒保王鸞,并兩個粉頭包氏、牛氏都拴了。竟投縣衙裡來見知縣。此時哄動了獅子街,鬧了清河縣;街上看的人不計其數。多說西門慶不當死,不知走的那裡去了,卻拏這個人來頂缸。正是:

「張公吃酒李公醉,  桑樹上吃刀柳樹上暴。」

誰人受用,誰人吃官司,有這等事!有詩為證:

「英雄雪恨被刑纏,  天公何事黑漫漫;

九泉乾死食毒客,  深閨笑殺一金蓮。」

畢竟未知後來如何,且聽下回分解:
