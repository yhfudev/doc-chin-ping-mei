%# -*- coding: utf-8 -*-
%!TEX encoding = UTF-8 Unicode
%!TEX TS-program = xelatex
% vim:ts=4:sw=4
%
% 以上设定默认使用 XeLaTex 编译,并指定 Unicode 编码,供 TeXShop 自动识别

%第六十九回 
\chapter{文嫂通情林太太\KG 王三官中詐求奸}


\begin{showcontents}{}




「信手烹魚覓素音,  神仙有路足登臨,

埽階偶得任卿葉,  彈月輕移司馬琴;

桑下肯期秋有意,  懷中可犯柳無心,

黃昏誤入銷金帳,  且犯羔羊獨自斟。」

話說文嫂兒到家,平安說:「爹在對門房子裡。進去稟報。」西門慶正在書房中,和溫秀才坐的,見玳安,隨即出來小客位內坐下。玳安悉把尋文嫂兒,小的叫了,來在外邊伺候著。西門慶即令叫他進來。那文嫂悄悄掀開暖簾,進入裡面,向西門慶磕頭。西門慶道:「文嫂兒,許久不見你?」文嫂道:「小媳婦有。」西門慶道:「你如今搬在那裡住了?」文嫂道:「小媳婦因不幸,為了場官司,把舊時那房兒棄了;如今搬在大南首王家巷住哩。」西門慶分付道:「起來說話。」那文嫂一面站立在傍邊,西門慶令左右多出去。那平安和畫童都躲在角門外伺候。只玳安兒影在簾兒外邊聽說話兒。西門慶因問:「你常在那幾家大人家走跳?」文嫂道:「就是大街皇親家、守備府周爺家、喬皇親、張二老爹、夏老爹家,多相熟。」西門慶道:「你認的王招宣府裡不認的?」文嫂道:「小媳婦定門主顧,太太和三娘常照顧小的花翠。」西門慶道:「你既相熟,我有庄事兒央煩你,休要阻了我。」向袖中取出五兩一定銀子與他,悄悄和他說:「如此這般,你都怎的尋個路兒,把他太太吊在你那裡,我會他會兒?我還謝你。」那文嫂聽了,哈哈笑道:「是誰對爹說?你老人家怎的曉得來?」西門慶道:「常言:『人的名兒,樹的影兒。』我怎不得知道!」文嫂道:「若說起我這太太來,今年屬豬,三十五歲,端的上等婦人,百伶百俐,只好三十歲的。他雖是幹這營生,好不幹的最密!就是往那裡去,主大轉伴當跟著,喝有路走,逕路兒來,逕路兒去。三老爹在外為人做人,他原在人家落腳?這個人說的訛了。到只是他家裡深宅大院,一時三老爹不在,藏掖個兒去,人不知鬼不覺,倒還許說。若是小媳婦那裡,窄門窄戶,敢招惹這個,事就在頭上。就是爹賞的這銀子,小媳婦也不敢領去。寧可領了爹言語,對太太說就是了。」西門慶道:「你不收,還自推托,我就惱了;事成,我還另外賞幾個紬段你穿。你不收,阻了我。」文嫂道:「愁你老人家沒也怎的!上人着眼覷,就是福星臨!」磕了個頭,把銀子接了,說道:「待小媳婦悄悄對太太話來,回你老人家。」西門慶道:「你當件事幹,我這裡等著。你來時只在這裡來就是了,我不使小廝去了。」文嫂道:「我知道。不在明日,只在後日,隨早隨晚,討了示下,就來了。」一面走出來,玳安道:「文嫂,隨你罷了。我只要一兩銀子,也是我叫你一場,你休要獨吃。」文嫂道:「猴孫兒,隔牆掠篩箕,還不知仰著合著哩!」于是出門,騎上驢子,他兒子籠著,一直去了。西門慶和溫秀才坐了一回。良久,夏提刑來,就到家待了茶,冠冕著,同往府裡羅同知名喚羅萬象那裡吃酒去了。直到掌燈已後纔來家。且說文嫂兒拿著西門慶與他五兩銀子,到家歡喜無盡。打發會茶人散了。至後晌時分,走到王宣府宅裡,見了林太太,道了萬福。林氏便道:「你怎的這兩日不來走走看看我?」文嫂便把家中倚報會茶,趕臘月要往頂上進香一節,告訴林氏。林氏道:「你兒子去,你不去罷了。」文嫂兒道:「我如何得去?只教文〈糸堂〉兒帶進香去便了。」林氏道:「等臨期我送些盤纏與你。」文嫂便道:「多謝太太布施。」說畢,林氏叫他近前烤火,丫鬟拿茶來吃了。這文嫂一面吃了茶,問道:「三爹不在家了?」林氏道:「他有兩夜沒回家,只在裡邊歇哩。逐日搭著這夥喬人,只眠花臥柳,把花枝般媳婦兒丟在房裡通不顧,如何如何!」又問:「三娘怎的不見?」林氏道:「他還在房裡未出來哩。」這文嫂見無人,便說道:「不打緊太太寬心。小媳婦有個門路兒,管就打散了這干人,三爹收心,也再不進院去了。太太容,小媳婦便敢說;不容,定不敢說。」林氏道:「你說的話兒。那遭兒我不依你來?你有話,只顧說不妨。」這文嫂方說道:「縣門前西門大老爹,如今見在提刑院做掌刑千戶,家中放官吏債,開四五處舖面,段子舖、生藥舖、紬絹舖、絨線舖,外邊江湖又走標船,楊州興販鹽引,東平府上納香蠟;夥計主管約有數十。東京蔡太師是他乾爺,朱太尉是他舊主,翟管家是他親家。巡撫、巡按多與他相交,知府、知縣是不消說。家中田連阡陌,米爛成倉;赤的是金,白的是銀,圓的是珠,光的是寶。身邊除了大娘子,乃是清河左衛吳千戶之女,填房與他為繼室;只成房頭,穿袍兒的,也有五六個。以下歌兒舞女、得寵侍妾,不下數十。端的朝朝寒食,夜夜元宵。今老爹不上三十四、五年紀,正是當年漢子,大身材,一表人物,也曾吃藥養龜,慣調風情,雙陸象棋,無所不通;蹴踘打毬,無所不曉;諸子百家,折白道字,眼見就會。端的擊玉敲金,百伶百俐。聞知咱家乃世代簪纓人家,根基非淺;又三爹在武學肄業,也要來相交,只是不曾會過,不好來的。昨日聞知太太貴旦在邇,又四海納賢,也一心要來與太太拜壽。小媳婦使道,初會怎好驟然請見的?待小的達知老太太,討個示下,來請老爹相見。今老爹不但結識他,來往相交,只央餐他把這干人斷開了,使那行人打攪,道須玷辱不了咱家門戶。」看官聽說:水性下流,最是女婦人。當日林氏被文嫂這篇話說的心中迷留摸亂,情竇已開。便向文嫂兒較計道:「人生面不熟悉怎生好遽然相見的?」文嫂道:「不打緊,等我對老爹說,只說太太先央餐老爹,要在提刑院遞狀,告那起引誘三爹這起人。預先私請老爹來,私下先會一會。此計有何不可?」說得林氏心中大喜,約定後日晚夕等候。這文嫂討了婦人示下歸家,到次日飯時前後,走來西門慶宅內。那日西門慶從衙門回來,家中無事,正在對門房子裡書院內坐的。忽有玳安來報:「文嫂來了。」西門慶聽了,即出小客位內坐,令左右放下簾兒。良久,文嫂進入裡面,磕了頭。玳安知局,就走出來了,教二人自在說話。這文嫂便把怎的說念林氏,誇獎老爹人品家道,怎樣行特結識官府,又怎的仗義疎財,風流博浪。說得他千肯萬肯,約定明日晚間三爹不在家,家中設席等候。假以說人情為由,暗中相會。西門慶聽了,滿心歡喜,又令玳安拿了兩疋紬段賞他。文嫂道:「爹明日要去,休要早了。直到掌燈已後,街上人靜了時,打他後門首扁食巷中;他後門傍有個住房的段媽媽,我在他家等著爹。只使大官兒彈門,我就出來引爹入港,休令左近人知道。」西門慶道:「我知道,你明日先去,不可離寸地,我也依期而至。」說畢,文嫂拜辭而去。又回林氏話去了。西門慶那日歸李嬌兒房中宿歇,一宿無話。巴不到次日,培養著精神。午間,戴著白忠靖巾,便同應伯爵騎馬往謝希大家吃生日酒。席布兩個唱的。西門慶吃了幾杯酒,約掌燈上來,就逃席走出來了。騎上馬,玳安、琴童兩個小廝跟隨。那時約十九日,月色朦朧,帶著眼紗,由大街抹過,逕穿到扁食巷王招宣府後門來。那時纔點燈以後,街上人初靜之後,西門慶離他後門半舍遠把馬勒住,令玳安先彈段媽媽家門。

原來這媽媽就住著王招宣府家後房,也是文嫂舉荐,早晚看守後門,開門閉戶。但有入港,在他家落腳做眼。文嫂在他屋裡,聽見外邊彈門,連忙開了門。見西門慶來了,一面在後門裡等的西門慶下了馬,帶著眼紗兒,引進來,分付琴童牽了馬,往對門人家西首房簷下那裡等候;玳安便在段媽屋裡存身。這文嫂一面請西門慶入來,便把後門關上,上了栓。由夾道內進內,轉過一層群房,就是太太住的五間正房,傍邊一座便門閉著。這文嫂輕輕敲了門環兒,原來有個聽頭兒。少頃,見一丫鬟出來開了雙扉,文嫂導引西門慶到後堂,掀開簾籠而入。只見裡面燈燭熒煌,正面供養著他祖爺太原節度邠陽邵王王景崇的影身圖,穿著大紅團就蟒衣玉帶,虎皮校椅,坐著觀看兵書,有若關王之像,只是髯鬚短些;傍邊列著鎗刀弓矢。迎門硃紅匾上:「節義堂」三字。兩壁書畫丹青,琴書消灑。左右泥金隸書一聯:「傳家節操同松竹,報國勳功並斗山。」西門慶正觀看之間,只聽得門簾上鈴兒響,文嫂從裡拿出一盞茶來與西門慶吃。西門慶便道:「請老太太出來拜見。」文嫂道:「請老爹且吃過茶著;剛才稟過,太太知道了。」不想林氏悄悄從房門簾裡望外觀看西門慶,身材凜凜,語話非俗,一表人物,軒昂出眾。頭戴白段忠靖冠貂鼠暖耳,身穿紫羊絨鶴氅,腳下粉底皂靴,上面綠剪絨獅坐馬,一溜五道金鈕子,就是個:富而多詐奸邪輩,壓善欺良酒色徒。一見滿心歡喜,因悄悄叫過文嫂來,問:「他戴的孝是誰的?」文嫂道:「是他第六個娘子的孝。新近九月間沒了,不多些時。饒少殺,家中如今還有一巴掌殺兒;他老人家你看不出來,出籠兒的鷯鶉,也是個快鬬的。」這婆娘聽了,越發歡喜無盡。文嫂催逼他出去見他一見兒。婦人道:「我羞答答,怎好出去?請他進來見罷。」文嫂一面走出來,向西門慶說:「太太請老爹房內拜見哩。」于是忙掀門簾,西門慶進入房中。但見簾幙垂紅,地屏上毡毹

匝地,麝蘭香靄,氣暖如春。綉榻則斗帳雲橫,錦屏則軒轅月映。婦人頭上戴著金絲翠葉冠兒,身穿白綾寬紬襖兒,沉香色遍地金粧花段子鶴氅,大紅官錦寬襴裙子,老鴉白綾高底扣花鞋兒。就是個:

「綺閣中好色的嬌娘,  深閨內{入日}〈毛皮〉的菩薩。」

有詩為證:

「面膩雲濃眉又彎,  蓮步輕移實匪凡;

醉後情深歸帳內,  始知太太不尋常。」

這西門慶一見,躬身施禮,說道:「請太太轉上,學生拜見。」林氏道:「大人免禮罷。」西門慶不肯,就側身磕下頭去,拜兩拜。婦人亦敘禮相還。拜畢,西門慶正面椅子上坐了,林氏就在下邊梳背炕沿斜僉相陪坐的。文嫂又早把前邊儀門閉上了,再無一個僕人在後邊。三公子那邊角門也關了。一個小丫鬟名喚芙蓉,紅漆丹盤,拿茶上來。林氏陪西門慶吃了茶,丫鬟接下盞托去。文嫂就在傍開言,說道:「太太久聞老爹在衙門中執掌刑名,敢使小媳婦請老爹來,央煩庄事兒,未知老爹可依允不依?」西門慶道:「不知老太太有甚事分付?」林氏道:「不瞞大人說,寒家雖世代做了這招宣,夫主去世年久,家中無甚積蓄。小兒年幼,優養未曾考襲。如今雖入武學肄業,年幼失學,家中有幾個奸詐不級的人,日逐引誘他在外飄灑,把家事都失了。幾次欲待要往公門訴狀,爭奈妾身未曾出閨門,誠恐拋頭露面,有失先夫名節。今日敢請大人至寒家,訴其衷曲,就如同遞狀一般;望乞大人千萬留情,把這干人怎生處斷開了,使小兒改過自新,專習功名,以承先業。寔出大人再造之恩,妾身感激不淺,自當重謝。」西門慶道:「老太太怎生這般說,言『謝』之一字?尊家乃世代簪纓,先朝將相,何等人家!令郎兩入武學,正當努力功名,承其祖武。不意聽信遊手所哄,留連花酒,寔出少年所為。太太既分付,學生到衙門裡即時把這干人處分。懲治令郎分毫,亦可戒諭令郎,再不可蹈此故轍,庶可杜絕將來。」這婦人聽了,連忙起身向西門慶道了萬福,說道:「容日妾身致謝大人。」西門慶道:「你我一家,何出此言?」說話之間,彼此言來語去,眉目顧盼留情。不一時,文嫂放卓兒,擺上酒來。西門慶故意辭道:「學生初來進謁,倒不曾具禮來,如何反承老太太盛情留坐?」林氏道:「不知大人下降,沒作準備。寒天聊具一杯水酒,表意而已。」丫鬟篩上酒來,端的金壺斟美釀,玉盞泛羊羔。林氏起身捧酒,西門慶亦下席,說道:「我當先奉老太太一杯。」文嫂兒在傍插口說道:「老爹你且不消遞太太酒,這十一月十五日是太太生日,那日送禮來與太太祝壽就是了。」西門慶道:「阿呀,早時你說!今日初九日,差六日,我在下已定來與太太登堂拜壽。」林氏笑道:「豈敢動勞大人厚意!」須臾,大盤大碗,就是十六碗熱騰騰美味佳餚;熬爛下飯煎烤雞魚,烹炮鵝鴨,細巧菜蔬,新奇菓品。傍邊絳燭高燒,下邊金爐添火。交杯換盞,行令猜枚。笑雨嘲雲,酒為色胆。看看飲至蓮漏已沉,窗月倒影之際,一雙竹葉穿心兩個芳情已動。文嫂已過一邊,連次呼酒不至。西門慶見左右無人,漸漸促席而坐,言頗涉邪,把手捏腕之際,挨肩擦膀之間;初時戲摟粉項,婦人則笑而不言;次後款啟朱唇,西門慶則舌吐其口,鳴咂有聲,笑語密切。婦人于是自掩房門,解衣鬆珮,微開錦帳,綉衾鴛枕橫牀,鳳香薰被,相挨玉體,抱摟酥胸。原來西門慶知婦人好風月,家中帶了淫器包在身邊,又服了胡僧藥。婦人摸見他陽物甚大,西門慶亦摸其牝戶,彼此歡欣,情興如火。婦人在牀傍伺候鮫綃軟帕,西門慶被底預備塵柄猙獰。當下展猿臂,不覺蝶浪蜂狂;蹺玉腿,那個羞雲怯雨。正是:

「縱橫慣使風流陣,  那管牀頭墜玉釵。」

有詩為證:

「蘭房幾曲深悄悄,香勝寶鴨睛烟梟;夢回夜月淡溶溶,展轉牙牀春色少。無心今遇少年郎,但知敲打須富商;殢情欲共嬌無力,須教宋玉赴高唐,打開重門無鎖鑰,露浸一枝紅芍。」

這西門慶當下竭平生本事,將婦人儘力盤桓了一場。纏至更半天氣,方纔精泄。婦人則髮亂釵橫,花憔柳困,鶯聲嚥喘,依稀耳中。比及個並頭交股,摟抱片時,起來穿衣之際,婦人下牀,款剔銀燈,開了房門,照鏡整容。呼丫鬟捧水淨手。復飲香醪,再勸美酌。三杯之後,西門慶告辭起身,婦人挽留不已,叮嚀頻囑。西門慶躬身領諾,謝擾不盡。相別出門,婦人送到角門首回去了。文嫂先開後門,呼喚玳安琴童牽馬過來,騎上回家。街上已喝號提鈴,更深夜靜,但見一天霜氣,萬籟無聲。西門慶回家,一宿無話。到次日,西門慶到衙門中發放已畢,在後廳叫過該地方節級輯捕,分付如此如此,這般這般:「王招宣府裡三公子,看有甚麼人勾引他?院中在何人家行走?便與我查訪出名字來,報我知道。」因向夏提刑說:「王三公子甚不學好,昨日他母親再三央人來對我說,倒不關他這兒子事,只被這干光棍勾引他。今若不痛加懲治,將來引誘壞了人家子弟。」夏提刑道:「長官所見不錯,必須該取他。」節級輯捕領了西門慶鈞語,到當日果然查訪出各人名姓來,打了事件。到後晌時分,來西門慶宅內呈遞揭帖。西門慶見上面有孫寡嘴、祝日念、張小閒、聶鉞兒、何三、于寬、白回子,樂婦是李桂姐、秦玉芝兒。西門慶取過筆來,把李桂姐、秦玉芝兒并老孫、祝日念名字多抹了;分付只動這小閒張等五個光棍。即與我拿了,明日早帶到衙門裡來。眾公人應諾下去。至晚,打聽王三官眾人都在李桂姐家吃酒踢行頭。多埋伏在後門首;深更時分,剛散出來,眾公人把小張閒、聶鉞、于寬、白回子、何三五人都拿了。孫寡嘴與祝日念,扒李桂姐後房去了。王三官兒藏在李桂姐床身下,不敢出來。桂姐一家諕的捏兩把汗,更不知是那裡動人,白央人打聽寔信。王三官躲了一夜,不敢出來。李家鴇子又恐怕東京做公的下來拿人,到五更時分,攛掇李銘換了衣服,送王三官來家,節輯捕把小張閒等拿在聽事房,吊了一夜。到次日早辰,西門慶進衙門與夏提刑陞廳,兩邊刑杖羅列,帶人上去,每人一夾二十大棍,打得皮開肉綻,鮮血迸流,響聲震天,哀號慟地。西門慶囑付道:「我把你這起光棍,專一引誘人家子弟在院飄風,不守本分。本當重處,今始從輕責你這幾下兒。再若犯在我手裡,家然枷號在院門首示眾。」唱令左右:「扠下去!」眾人望外,金命水命,走投無命。兩位官府發放事畢,正在退廳吃茶。夏提刑因說起:「昨日京中舍親崔中書那裡書來,衙中投考察本上去了,還未下來哩;今日會了長官,咱倒好差人往懷慶府同僚林蒼峰,他那裡臨風近,打聽打聽消息去。」西門慶道:「長官至見甚明。」即喚走差答應的上來跪下,分付:「與你五錢銀子盤纏,即去南河拿俺兩個拜帖懷慶府提刑林千戶老爹那裡,打聽京中考察本示下,看經歷司行下照會來不曾?務要打聽的寔來回報。」那人領了銀子拜帖,又到司房戴土范陽毡笠,結束行裝,討了疋馬,長行去了。兩位官府起身回家。都說小張閒等從提刑院打出來,走在路上,各人省恐,更不量今日受這場虧,那裡藥線?互相埋怨。小張閒道:「莫不還是東京六黃太尉那裡下來的消息?」白回子道:「不是,若是那裡消息,怎肯輕饒素放?」常言說得好:乖不過唱的,賊不過銀匠,能不過架兒;聶鉞兒一口就說道:「你每多不知道,只我猜得著。此已定西門官府和三官兒上氣,嗔請他表子,故拿俺每煞氣。正是:龍鬬虎傷,苦了小張!」小張閑道:「列位到罷了,只是苦了我在下了。孫寡嘴、祝麻子都跟著,只把俺每頂缸了。」于寬道:「你怎的說渾話?他兩個是他的朋友,若拿來跪在地下,他在上面坐著,怎生相處?」小張閒道:「怎的不拿老婆?」聶鉞道:「兩個老婆都是他心上人。李家桂姐是他表子,他肯拿來?也休怪人,是俺每的晦氣,偏撞在這網裡!纔夏老爹怎生不言語,只是他說話?這個就見出情弊顯然來了。如今往李桂姐兒家尋王三官去,白為他打了這一屁股瘡來的!腿爛爛的,便罷了?」問他要幾兩銀子盤纏,也不吃家中老婆笑話。」于是來來去去,轉彎抹角,逕入抅攔李桂姐家。見門關的鐵桶相似,就是樊噲也撞不開。叫了半日,丫頭隔門問:「是誰?」小張閒道:「是俺每,尋三官兒說話。」丫頭回說:「他從那日半夜就往家去了,不在這裡。無人在家中,不敢開門。」這眾人只得回來,到王招宣府宅內,逕入他客位裡坐下。王三官聽見眾人來尋他,諕得躲在房裡,不敢出來。半日使出小廝永定來,說:「俺爹不在家了。」眾人道:「好自在性兒!不在家了,往那裡去了?叫不將來?」于寬道:「寔和你說了罷休推睡裡夢裡,剛纔提刑院打了俺每,押將出來,如今還要他正身兒見官去哩。」摟起腿來與永定瞧,教他進裡面去說此事,為你打的俺每有甚要緊,一個都倘在板凳上,聲疼叫喊。那王三官兒越發不敢出來,只叫:「娘怎麼樣兒,都如何救我則可?」林氏道:「我女婦人家,如何尋人情去救得?」求了半日,見外邊眾人等的急了,要請老太太說話。那林氏又不出去,只隔著屏風說道:「你每略等他等,委的在庄上,不在家了。我這裡使小廝叫他去。」小張閒道:「老太太,快使人請他來。」不然,這個癤子,也要出膿。只顧膿著不是事;俺每為他連累打了這一頓。剛纔老爹分付,押出俺每來要他。他若不出來,大家都不得清淨,就弄的不好了。」林氏聽言,連忙使小廝拿出茶來,與眾人吃。王三官諕的鬼也似,逼他娘尋人情。到至急之處,林氏方纔說道:「文嫂他只認的提刑西門官府家,昔年曾與他女兒說媒來。在他宅中走的熟。」王三官道:「就認的提刑也罷,快使小廝請他來。」林氏道:「他自從你前番說了他,使性兒一向不來走動,怎好又請他?肯來?」王三官道:「好娘,如今事在至急,請他來,等我與他陪個禮兒便了。」林氏便使永定兒悄悄打後門出去,請了文嫂來。王三官再三央及他,一口聲只叫:「文媽,你認的提刑西門大官府,好歹說個人情救我。」這文嫂故意做出許多喬張致來,說道:「舊時雖故與他宅內大姑娘說媒,這幾年誰往他門上走?大人家深宅大院,不去纏他。」王三官連忙跪下,說道:「文媽,你救我,自有重報,不敢有忘!那幾個人在前邊,只要出官,我怎去得?」那文嫂只把眼看他娘。他娘道:「也罷,你替他說說罷了。」文嫂道:「我獨自個去不得。三叔,你衣巾著,等我領你親自到西門老爹宅上,你自拜見他,央餐他等等我在傍再說,管情一天事就了了。」王三官道:「見今他眾人在前邊催逼甚急,只怕一時被他看見,怎了?」文嫂道:「有甚難處勾當?等我出去安撫他,再安排些酒肉點心茶水,哄他吃著。我悄悄領你從後門出去幹事回來,他令放也不知道。」這文嫂一再走出前廳,向眾人拜了兩拜,說道:「太太教我出來,多上覆列位哥們,本等三叔往庄上去了,不在家。使人請去了,便來也;你每略坐坐兒。吃打受罵,連累了列位。誰人不吃鹽米?等三叔來,教他知遇你們。你們千差萬差,來人不差。恒屬大家只要圖了事,上司差派,不由自已。有了三叔出來,一天大事都了了。」當時眾人一齊道:「還是文媽見的多,你老人家早出來就說句話,恁有南北的話兒,俺每也不恁急的要不的。執殺法兒,只回不在家,莫不為俺每自做出來的事也罷。你倒帶累俺每吃官棒,上司要你,假推不在家。吃酒吃肉,教人替你不成?文媽,你自曉道理的,你出來,俺每還透個路兒與你,破些東西兒,尋個分上兒說說,大家了事;你不出來見俺每,這事情也要銷徼。一個緝捕問刑衙門,平不答的就罷了?」文嫂兒道:「哥每說的是。你每略坐坐兒,我對太太說,安排些酒飯兒管待你每。你每來了這半日,也餓了。」眾都道:「還是我的文媽知人甘苦,不瞞文媽說,俺每從衙門裡打出來,黃湯兒也還沒曾嚐著哩!」這文嫂走到後邊,一力攛掇打了二錢銀子酒,買了一錢銀子點心,豬羊牛肉,各切幾大盤,拿將出去。一壁哄他眾人在前廳,大酒大肉吃著。這王三官儒巾青衣,寫了揭帖,文嫂領著,帶上眼紗,悄悄從後門出來,步行逕往西門慶家來。到了大門首,平安兒認的文嫂,說道:「爹纔在廳上,進去了。文媽有甚麼說話?」文嫂遞與他拜帖,說道:「哥哥,累你替他稟稟去。」連忙問王三官要了二錢銀子遞與他那平安兒方進去,替他稟知西門慶。西門慶見了手本拜帖上寫著:眷晚生王寀頓首百拜。一面先叫進文嫂,問了回話。然後纔開大廳槅子門,使小廝請王三官進去大廳上。左右忙掀暖簾,見西門慶頭戴忠靖冠,便衣出來迎接。見王三衣巾進來,故意說道:「文嫂怎不早說?我褻衣在此!」便令左右:「取我衣服來。」慌的王三官向前攔住:「呀!尊伯尊便,小姪敢來拜凟,豈敢動勞!」至廳內,王三官務請西門慶轉上行禮。西門慶笑道:「此是舍下。」再三不肯。西門慶居先拜下去,王三官說道:「小姪有罪在身,久仰欠拜。」西門慶道:「彼此少禮。」王三官因請西門慶受禮,說道:「小姪人家,老伯當得受禮,以恕拜遲之罪。」務讓起來,讓了兩禮;然後挪座兒斜僉坐的。少頃,吃了茶,王三官見西門慶廳上,錦屏羅列,四壁挂四軸金碧山水,座上銷著綠錦段廂嵌貂鼠椅座,地下氍毹匝地。正中間黃銅四方水磨的耀目爭輝,上面牌扁下書:「承恩」二字,係米元章妙筆,觀覽之餘,似有邵清而寧之貌。向西門慶說道:「小姪前有一事,不敢奉凟尊嚴。」因向袖中取出揭帖遞上,隨即離席跪下。被西門慶一手拉住,說道:「賢契有甚話,但說何害!」這王三官就:「小姪不才,誠為得罪。望乞老伯念先父武弁一殿之臣,寬恕小姪無知之罪,完其廉耻,免令出官。則小姪垂死之日,實有再生之幸也!啣結圖報,惶恐惶恐!」西門慶展開揭帖上面有小張閒等五人名字,說道:「這起光棍,我今日衙門裡已各重責發落,饒恕了他。怎的又央你去?」王三官道:「還是要小姪如此這般,他說老伯衙門中責罰,押出他來,還要小姪見官。在家百般稱罵喧嚷,索要銀兩,不得安生。無處控訴,前來老伯這裡請罪。」又把禮帖遞上。西門慶一見,便道:「豈有是理!」因說道:「這起光棍可惡!我倒饒了他,如何倒往那裡去攪擾!」把禮帖與王三官收了:「賢契請回,我也且不留你坐。如今即時就差人拿這起光棍去,容日奉招。」王三官道:「豈敢!蒙老伯不棄,小姪容當踵門叩謝。」千恩萬謝出門。西門慶送至二門首,說:「我褻服不好送的。」那王三官自出門,還帶上眼紗,小廝跟隨去了。文嫂還討了西門慶話。西門慶分付:「休要驚動他,我這裡差人拿去。」這文嫂同王三官暗暗到家,不想西門慶隨即差了一名節級,四個排軍,走到王招宣宅內,那起人正在那裡飲酒喧鬧,被公人進去,不由分說,都拿了,帶上鐲子。諕得眾人面如土色,說道:「王三官幹得好事,把俺每穩在你家,倒把鋤頭反弄俺每來了!」那個排軍節級罵道:「你這廝還胡說,當了甚麼?各人到老爹根前哀告,討你那命正經!」小張閒道:「大爹教導的是。」不一時,都拿到西門慶宅門首,門上排軍并平安,都張著手兒要錢,纔去替他稟。眾人不免脫下褶,并拿頭上簪圈下來,打發停當,方纔說進去。半日,西門慶出來坐廳,節級帶進去,跪在廳下。西門慶罵道:「我把你這起光棍,我倒將就了,如何指稱我這衙門,往他家嚇詐去?實說,詐了多少錢?不說,令左右拿拶子與我著實拶起來!」當下只說了聲,那左右排軍,登時取了五六把新拶子來伺候。小張閒等只顧在下叩頭哀告道:「小的並沒嚇詐分文財物。只說衙門中打出,小的每來對他說聲,他家拿出些酒食來,管待小的;小的並沒需索他的。」西門慶道:「你也不該往他家去。你這起光棍,設騙良家子弟,白手要錢,深為可惡!既不肯寔供,都與我帶了衙門裡收監,明日嚴番取供,枷號示眾。」眾人一齊哀告,哭道:「天官爺,超生小的每罷!小的再不敢上他門纏擾了。休說枷號,這一送到監裡去,冬寒時月,小的每都是死數!」西門慶道:「我把你這光棍,我道饒出你去,都要洗心改過,務安生理。不許你挨坊靠院,引誘人家子弟,詐騙財物。再拿到我衙門裡來,都活打死了!」喝令:「出去罷。」眾人得了個性命,往外飛跑走。正是:

「敲碎玉籠飛彩鳳,  頓開金鎖走蛟龍。」

西門慶發了眾人去,回至後房。月娘問道:「這個是王三官兒?」西門慶道:「此是王招宣府中三公子。前日李桂兒為他那場事,就是他。今日賊小淫婦兒不改,又和他纏,每月三十兩銀子,教他包著。嗔道一向只哄著我。不想有個底腳里人兒,又告我說,教我昨日差幹事的拿了這干人到衙門裡去,都夾打了。不想這干人又到他家裡嚷賴,指望要詐他幾兩銀子的情,只恐衙門中要他。他從來沒曾見官,慌了,央文嫂兒拿五十兩禮帖來,求我說人情。我剛纔把那起人又拿了來,詐發了一頓,替他杜絕了,再不纏他去了。人家倒運,偏生出這樣不肖子弟出來。你家父祖何等根基,又做招宣,你又見入武學,放著那功名兒不幹,家中去著花枝般媳婦兒,自東京六黃太尉姪女兒不去理論,白日黑夜,只跟著這夥光棍在院裡嫖弄,把他娘子頭面都拿出來使了。今年不上二十歲年小小兒的,通不成器!」月娘道:「你不曾溺胞尿,看看自家。乳兒老鴉笑話豬兒足,原來燈臺不照自。你自道成器的,你也吃這井裡水,無所不為,清潔了些甚麼兒?還要禁的人!」幾句說的西門慶不言語了。正擺上飯來吃,小廝來安來報:「應二爹來了。」西門慶分付:「請書房裡坐,我就來。」王經連忙開了廳上書房門,伯爵進裡面暖爐炕傍椅上坐了。良久,西門慶出來。聲喏畢,就坐在炕上兩個說話。伯爵道:「哥,你前日在謝二哥那裡,怎的老早就起身?」西門慶道:「第二日我還要早起衙門中,連日有勾當,又考察在邇,差人東京打聽消息。我比你每閒人兒?」伯爵又問:「哥,連日衙門中有事沒有?」西門慶道:「事那日沒有。」又道:「王三官兒說,哥衙門中動了,把小張閒他每五個,初八日晚夕在李桂姐屋裡,都拿的去了,只走了老孫、祝麻子兩個,今早解到衙門裡,都打出來了。眾人都往招宣府纏王三官去了,怎的還瞞著我不說?」西門慶道:「傻狗材,誰對你說來?你敢錯聽了,敢不是我衙門裡,敢是周守備府裡?」伯爵道:「守備府中那裡這管閒事!」西門慶道:「只怕是躲中提人。」伯爵道:「也不是。今早李銘對我說,那日把他一家子諕的魂也沒了。李桂兒至今諕的這兩日睡倒了,還沒曾起炕兒裡坐,怕又是東京下來拿人。今早打聽,方知是提刑院動人。」西門慶道:「我連日不進衙門,並沒知道。李桂兒既賭個誓不接他,隨他拿亂去,又害怕睡倒怎的!」伯爵見西門慶迸著臉兒待笑,說道:「哥,你是個人,連我也瞞著起來?不告我說。今日他告我說,我就知道哥的情,怎的祝麻子、老孫走了,一個輯事衙門,有個走脫了人的?此是哥打著綿羊駒〈馬婁〉戰,使李桂兒家中害怕,知道哥的手段。若多拿到衙門去,彼此絕了情意,多沒趣了。事情許一不許二。如今就是老孫、祝麻子,見哥也有幾分慚愧,此是哥明修棧道,暗度陳倉的計策。休怪我說,哥這一著做的絕了。這一個叫做真人不露相,露相不是真人。若明使函了,逞了臉,就不是乖人兒了。還是哥智謀大,見的多。」幾句說的西門慶撲吃的笑了,說道:「我有甚麼大智謀?」伯爵道:「我猜已定還有底腳裡人兒對哥說;怎得知道這等端切的?有鬼神不測之機!」西門慶道:「傻狗材,若要人不知,除非己莫為。」伯爵道:「哥衙門中如今不要王三官兒罷了。」西門慶道:「誰要他做甚麼?當初幹事的打上事件,我就把王三官、祝麻子、老孫并李桂兒、秦玉芝名字多抹了。只來打拿幾個光棍。」伯爵道:「他如今怎的還纏?」西門慶道:「我寔和你說罷。他指稱嚇詐他幾兩銀子,不想剛纔親上門來拜見,與我磕了頭,陪了不是。我還差人把那幾個光棍拿了,要枷號,他眾人再三哀告,說不敢上門纏他了。王三官一口一聲稱呼我是老伯,拿了五十兩禮帖兒,我不受他的。他到明日,還要請我家中知謝我去。」伯爵我驚道:「真個他來大哥陪不是來了?」西門慶道:「我莫不哄你?」因喚王經:「拿王三官拜帖兒,與應二爹瞧!」那王經向房子裡取出拜帖,上面寫著:晚生王寀頓首百拜。伯爵見了,口中只是極口稱贊:「哥的所算神妙不測!」西門慶分付伯爵:「你若看見他每,只說我不知道。」伯爵道:「我曉得。機不可泄,我怎肯和他說。」坐了一回吃了茶,伯爵道:「哥,我去罷。只怕一時老孫和祝麻子摸將來,只說我沒到這裡。」西門慶道:「他就來,我也不出來見他,只答應不在家。」一面叫將門上人來,都分付了:「但是他二人,只答應不在。」西門慶從此不與李桂姐上門走動,家中擺酒,也不叫李銘唱曲,就疎淡了。正是:

「昨夜浣花溪上雨,  綠楊芳草為何人?」

有詩為證:

「誰道天台訪玉真,  三山不見海沉沉;

侯門一入深如海,  從此蕭郎是路人。」

畢竟未知後來如何,且聽下回分解:





\end{showcontents}


