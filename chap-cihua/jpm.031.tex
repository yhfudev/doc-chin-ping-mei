%# -*- coding: utf-8 -*-
%!TEX encoding = UTF-8 Unicode
%!TEX TS-program = xelatex
% vim:ts=4:sw=4
%
% 以上设定默认使用 XeLaTex 编译,并指定 Unicode 编码,供 TeXShop 自动识别

%第三十一回 
\chapter{琴童藏壺覷玉簫\KG 西門慶開宴吃喜酒}


「家富自然身貴,   逢人必讓居先,

貧寒敢仰上官憐,  彼此都看錢面;

婚嫁專尋勢要,   通財邀結豪英,

不知興廢在心田,  只靠眼前知見。」

話說西門慶,次日使來保提邢所,本縣下文書,一面使人做官帽。又喚趙裁率領四五個裁縫,在家來裁剪尺頭,儹造衣服。又叫了許多匠人,釘了七八條都是四尺寬玲瓏雲母犀角鶴頂紅玳瑁魚骨香帶。不說西門慶家中熱亂。且說吳典恩那日走到應伯爵家,把做驛丞之事,再三央及伯爵,要問西門慶借銀子上下使用。許伯爵:「借銀子出來,把十兩銀子買禮物謝老兄。」說着跪在地下。慌的伯爵一手拉起,說道:「此是成人之美。大官人照顧你東京走了這遭,攜帶你得此前程,也不是尋常小可。」因問:「你如今所用多少勾了?」吳典恩道:「不瞞老兄說,我家活人家,一文錢也沒有。到明日上任參官贄見之禮,連擺酒并治衣類鞍馬,少說也得七八十兩銀子,那裡區處?如今我寫了一布文書在此,也沒敢下數兒。望老兄好歹扶持小人,在旁加美言。事成恩有重報,不敢有忘。」伯爵看了文書,因令:「吳二哥,你說借出這七八十兩銀子來,也不勾使。依我取筆來寫上一百兩恆是看我面不要你利錢。你且得手使了,到明日做上官兒,慢慢陸續還他,也是不遲。常言俗語說得好,借米下得鍋,討米下不的鍋。哄了一日是兩晌。何況你又在他家曾做過買賣,他那裡把你這幾兩銀子放在心上?」那吳典恩聽了,謝了又謝。于是把文書上,填寫了一百兩之數。當下兩個吃了茶,一同起身,來到西門慶門首。伯爵問守門平安兒:「你爹起來了不曾?」平安兒道:「俺爹起來了,在捲棚看着匠人釘帶哩。待小的稟去。」于是一直走來報西門慶說:「應二爹和吳二叔來了。」西門慶道:「請進。」不一時,二人進入裡面,見有許多裁縫匠人,七手八腳做生活。西門慶帶着小帽錦衣和陳經濟在穿廊下,看着寫見官手本揭帖。見二人,作揖讓坐。伯爵問:「哥的手本劄付,下了不曾?」西門慶道:「今早使小价往提刑府下劄付去了。今有手本還未往東平府并本縣下去。」說畢,小廝畫童兒拿上茶來。吃畢茶,那應伯爵並不題吳主管之事,走下來且看匠人釘帶。西門慶見他拿起帶來看,一徑賣弄,說道:「你看我尋的這幾條帶如何?」伯爵極口稱讚誇獎說道:「虧哥那裡尋的都是一條賽一條的好帶!難得這般寬大。別的倒也罷了,自這條犀角帶并鶴頂紅,就是滿京城拿着銀子也尋不出來。不是面獎,說是東京衛主老爺玉帶金帶空有,也沒這條犀角帶。這是水犀角,不是旱犀角。旱犀不值錢,水犀角號作通天犀。你不信,取一碗水,把犀角安放在水內,分水為兩處,此為無價之寶。又夜間燃火照千里,火光通宵不滅。」因問:「哥,你使了多少銀子尋的?」西門慶道:「你每試估估價值。」伯爵道:「這個有甚行款?我每怎麼估得出來?」西門慶道:「我對你說了罷,此帶是大街上王招宣府裡的帶。昨日晚間一個人聽見我這裡要帶,巴巴來對我說。我着賁四拿了七十兩銀子,再三回了他這條帶來。他家還張致不肯,定要一百兩。」伯爵道:「且難得這等寬樣好看。哥,你到明日繫出去,甚是霍綽。就是你同僚間見了也愛。」于是誇美了一回坐下。西門慶便向吳主管問道:「你的文書下了不曾?」伯爵道:「吳二哥文書還未下哩!今日巴巴的他央我來激煩你。雖然蒙你招顧他往東京押生辰担,蒙太師與了他這個前程,就是你擡舉他一般,也是他各人造化。說不的一品至九品,都是朝廷臣子。況他如今家中無錢。他告我說,就是如今上任見官擺酒并治衣服之類,也并許多銀子使。一客不煩二主,那處活變去?沒奈何,哥看我面,有銀借與幾兩扶持他,賙濟了這些事兒。他到明日做上官,就啣環結草也不敢忘了哥大恩人。休說他舊是咱府中夥計,在哥門下出入。就是從前後外京外府官吏,哥不知拔濟了多少。不然,你教他那裡區處去?」因說道:「吳二哥,你拿出那符兒來與你大官人瞧。」這吳典恩連忙向懷中取出,遞與西門慶觀看。見上面借一百兩銀子,中人就是應伯爵,每月利行五分。西門慶取筆把利錢抹了,說道:「既是應二哥作保,你明日只還我一百兩本錢就是了。我料你上下巴得這些銀子攪纏。」于是把文書收了。纔待後邊取銀子去,忽有提刑所夏提刑拿帖兒差了一名寫字的,拿手本三班送了十二名排軍來答應。就問討上任日期,討問字號,衙門同僚具公禮來賀。西門慶教陰陽徐先生擇定七月初二日青龍金匱黃道,宜辰時到任,拿拜帖兒回夏提刑,賞了寫字的五錢銀子,俱不必細說。應伯爵和吳典恩正在捲棚內坐的,只見陳經濟拿着一百兩銀子出來,交與吳主管說:「吳二哥,你明日只還我本錢便了。」那吳典恩一面接了銀在手,叩頭謝了。西門慶道:「我不留你坐罷,你家中執你的事去了。留下應二哥,我還和你說句話兒。」那吳典恩拿着銀子,歡喜出門。看官聽說:後來西門慶死了,家中時敗勢衰,吳月娘守寡,把小玉配與玳安為妻。家中平安兒小廝,又偷盜出解當庫頭面,在南瓦子裡宿娼。被吳驛丞拿住,痛刑拶打,教他指攀月娘與玳安有奸,要羅織月娘出官,恩將仇報。此係後事,表過不題。正是:

「不結子花休要種,  無義之人不可交。」

那時賁四往東平府并本縣下了手本來回話。西門慶留他和應伯爵陪陰陽徐先生擺飯。正吃着飯,只見西門慶舅子吳大舅來拜望。徐先生就起身。良久,應伯爵也作辭出門,來到吳主管家。吳典恩又早封下十兩保頭錢,雙手遞與伯爵,磕下頭去。伯爵道:「若不是我那等取巧說着,他會勝不肯借與你。這一百兩銀與你,隨你上下還使不了這些,還落一半家中盤纏。」那吳典恩酬謝了伯爵,治辦官帶衣類,擇日見官上任不題。那時本縣正堂李知縣,會了四衙同僚,差人送羊酒賀禮來。又拿帖兒送了一名小郎來答應,年方一十八歲,本貫蘇州府常熟縣人,喚名小張松。原是縣中門子出身,生的清俊,面如傳粉,齒白唇紅。又識字會寫,善能歌唱南曲。穿着責絹直裰,京鞋淨襪。西門慶一見小郎伶俐,滿心歡喜。就拿拜帖回覆李知縣。留下他在家答應,改換了名字,叫做書童兒。與他做了一身衣裳,新靴新帽。不教他跟馬,教他專管書房,收禮帖,拿花園鑰匙。祝日念又舉保了一個十四歲小廝來答應,亦改名棋童,每日派定和琴兒兩個,背書袋,夾拜帖匣跟馬。上任日期,在衙門中擺大酒席桌面,出票拘集三院樂工牌色長承應,吹打彈唱,後堂飲酒。日暮時分散歸。每日騎着大白馬,頭戴烏紗,身穿五彩洒線揉頭獅子補子員領,四指大寬萌金茄楠香帶,粉底皂靴,排軍喝道,張打着大黑扇,前呼後擁,何止十數人跟隨,在街上搖擺。上任回來,先拜本府縣帥府都監,并清河左右衛同僚官,然後親朋鄰舍,何等榮耀施為!家中收禮接帖子,一日不斷。正是:

「白馬血纓彩色新,  不來親者強來親;

時來頑鐵皆光彩,  運去良金不發明。」

西門慶自從到任以來,每日坐提刑院衙門中陞廳畫卯,問理公事。光陰迅速,不覺李瓶兒坐褥一月將滿。吳大妗子、二妗子、楊姑娘、潘姥姥、吳大姨、喬大戶娘子,許多親鄰堂客女眷,都送禮來,與官哥兒做彌月。院中李桂姐、吳銀兒見西門慶做了提刑所千戶,家中又生了子,亦送大禮,坐轎子來慶賀。西門慶那日在前邊大廳上擺設筵席,請堂客飲酒。春梅、迎春、玉簫、蘭香都打扮起來,在席前與月娘斟酒執壺,堂客飲酒。原來西門慶每日從衙門中來,只見外邊廳上,就脫了衣服,教書童叠了,安在書房中,止戴着冠帽進後邊去。到次日起身,旋使丫鬟來書房中取,新近收拾大廳,西廂房一間做書房,內安牀几桌椅、屏幃筆硯琴書之類。書童兒晚夕只在牀腳踏板書,搭着舖睡,未曾西門慶出來,就收拾頭腦打掃書房乾淨,伺候答應。或是在那房裡歇,早辰就使出那房裡丫鬟來前邊取衣服。取來取去,不想這小郎本是門子出身,生的伶俐乖覺又清俊,二者又各房丫頭打牙犯嘴慣熟,于是暗和上房裡玉筲兩個嘲戲上了。那日也是合當有事。這小郎正起來在書房牀地平上,插着棒兒香,正在窗戶臺上擱着鏡兒梳頭,拿紅繩扎頭髮。不料上房玉筲推開門進來,看見說道:「好賊囚,你這咱還來描眉畫眼兒的,爹吃了粥便出來。」書童也不理,只顧扎包髻兒。那玉筲道:「爹的衣服叠了,在那裡放着哩?」書童道:「在牀南頭安放着哩。」玉筲道:「他今日不穿這一套。他吩咐我,教問你要那件玄色匾金補子系布圓領玉色襯衣穿。」書童道:「那衣服在廚櫃裡。我昨日纔收了,今日又要穿他。姐,你自開門取了去。」那玉筲且不拿衣服,走來跟前,看着他扎頭,戲道:「怪賊囚!也像老婆般,拿紅繩扎着頭兒,梳的鬢這虛籠籠的。」因見他白滾紗漂白布汗掛兒上,繫着一個銀紅紗香袋兒,一個綠紗香袋兒,問他要:「你與我這個銀紅的罷。」書童道:「人家個愛物兒,你就要。」玉筲道:「你小廝家帶不的這銀紅的,只好我帶。」書童道:「早是這個罷了,打要是個漢子兒,你也愛他罷?」被玉筲故意向他肩膊上擰了一把,說道:「賊囚!你夾道賣門神,看出來的好畫兒!」不由分說,把兩個香袋子等不的解,都揪斷繫兒放在袖子內。書童道:「你好不尊貴,把人的帶子也揪斷。」被玉筲發訕,一拳一把,戲打在身上,打的書童急了,說:「姐,你休鬼混我,待我扎上這頭髮着。」玉筲道:「我且問你,沒聽見爹今日往那去?」書童道:「爹今日與縣中三宅華主簿老爹送行,在皇庄薛公公那裡擺酒,來家早下午時分。我聽見會下應二叔今日兌銀子,要買對門喬大戶家房子,那裡吃酒罷了。」玉宵道:「等住回,你休往那去了。我來和你說話。」書童道:「我知道。」玉宵于是與他約會下,拿衣服一直往後邊去了。少頃,西門慶出來,就叫書僮吩咐在家,別往那去了。先寫十二個請帖兒,都用大紅紙封套,二十二日請官家吃慶官哥兒酒。教來興兒買辦東西,添廚役茶酒,預備桌面齊整。玳安和兩名排軍送帖兒,叫唱的。留下琴童兒在堂客面前管酒。吩咐畢,西門慶上馬送行去了。那吳月娘眾姊妹請堂客到齊了,先在捲棚擺茶,然後大廳上,屏開孔雀,褥隱芙蓉。上坐席間,叫了四個妓女彈唱。果然西門慶到午後時分來家。家中安排一食菓酒菜,邀了應伯爵和陳經濟,招了七百兩銀子,往對門喬大戶家成房子去了。堂客正飲酒中間,只見玉筲拿下一銀執壺酒,并四個梨,一個柑子,逕來廂房中送與書童兒吃。推開門,不想書童兒不在裡面、恐人看見,連壺放下就出來了。可霎作怪!琴童兒正在上邊看酒,冷眼睃見玉筲進書房去,半日出來。只知有書童兒在裡邊,三不知扠進去瞧。不想書童兒外邊去,不曾進來。一壺熱酒和菓子還放在牀底下。這琴童連忙把菓子藏袖裡,將那一壺酒影着身子一直提到李瓶兒房裡。迎春和婦人都在上邊,不曾下來。止有奶子如意兒和綉春在屋裡看哥兒。那琴童進門就問:「姐在那裡?」綉春道:「他在上邊與娘斟酒哩,你問他怎的?」琴童兒道:「我有個好的兒,教他替我收着。」綉春問他甚麼,他又不拿出來。只說着,迎春從上邊拿下一盤子燒鵝肉 ,一碟玉米面玫瑰菓餡蒸餅兒與妳子吃。看見便道:「賊囚,你在這裡笑甚麼?不在上邊看酒?」那琴童方纔把壺從衣裳底下拿出來,教迎春:「姐,你與我收了。」迎春道:「此是上邊篩酒的執壺,你平白拿來做甚麼?」琴童道:「姐你休管他。此是上房裡玉筲和書童兒小廝,七個八個偷了這壺酒和些柑子、梨,送到書房中與他吃。我趕眼不見,戲了他的來。你只與好生收着,隨問甚麼人來抓尋,休拿出來。我且拾了白財兒着。」因把梨和柑子掏出來,與迎春瞧。說着:「我看篩了酒,今日該我獅子街房子裡,我上宿去也。」迎春道:「等住回抓尋壺久亂,你就承當。」琴童道:「我又沒偷他的壺。各人當場者亂,隔壁心寬。管我腿事!」說畢,揚長去了。迎春把壺藏放在裡間桌上不題。至晚,酒席上人散,查收家火,少了一把壺。玉筲往書房中尋,那裡得來?再有一把也沒了。問書童,說:「我外邊有事去,不知道。」那玉筲就慌了,一口推在小玉身上。小玉道:「{入日}昏了你這淫婦!我後邊看茶,你抱着執壺,在席上與娘斟酒。這回不見了壺兒,你來賴我!」向各處都抓尋不着。良久,李瓶兒到房來,迎春如此這般告訴:「琴童兒拿了一把進來,教我替他收着。」李瓶兒道:「這囚根子!他做甚麼拿進他這把壺來?後邊為這把壺好不反亂。玉宵推小玉,小玉推玉宵,急的那大丫頭賭身發呪,只是哭。你趁早還不快替他送進去哩,遲回管情就賴在你這小淫婦兒身上。」那迎春方纔取出壺,要送入後邊來。後邊玉筲和小玉兩個正亂這把壺不見了,兩個嚷到月娘面前。月娘道:「賊臭肉,還敢嚷的是些甚麼?你每管着那一門兒?把壺不見了?」玉簫道:「我在上邊跟着娘邊酒,他守着銀器家火,不見了,如今賴我?」小玉道:「大妗子要茶,我不往後邊替他取茶去?你抱着執壺兒,怎的不見了?敢屁股大吊了心了也怎的!」月娘道:「我省恐今日席上再無閒雜人,怎的不見了東西?等住回看這把壺從那裡出來。等住回嚷的你主子來,沒這壺,管情一家一頓。」玉筲道:「爹若打了我,我把這淫婦饒了也不算!」正亂着,只見西門慶自外來,問:「因甚嚷亂?」月娘把不見壺一節說了一遍。西門慶道:「慢慢尋就是了,平白嚷的是些甚麼?」潘金蓮道:「若是吃一遭酒,不見了一把,不嚷亂,你家是王十萬,頭醋不酸到底兒薄。」看官聽說:金蓮此話譏諷李瓶兒首先生孩子滿月,不見了也是不吉利。西門慶明聽見,只不做聲。只見迎春送壺進來。玉簫便道:「這不是壺有了!」月娘問迎春:「這壺端的在那裡來?」迎春悉把:「琴童從外邊拿到俺娘屋裡收着,不知在那裡來。」月娘因問:「琴童兒那奴才,如今在那裡?」玳安道:「他今日該獅子街房差,上宿去了。」金蓮在旁,不覺鼻子裡笑了一聲。西門慶便問:「你笑怎的?」金蓮道:「琴童兒是他家人,放壺他屋裡,想必要瞞昧這把壺的意思。要我使小廝如今叫將那奴才,老實打着,問他個下落。不然,頭裡就賴他那兩個,正是走殺金剛坐殺佛!」西門慶聽了,心中大怒,睜眼看着金蓮說道:「看着恁說起來,莫不李大姐他愛這把壺?既有了,丟開手就是了,只管亂甚麼!」那金蓮把臉羞的飛紅了,便道:「誰說姐姐手裡沒錢。」說畢,走過一邊使性兒去了。西門慶就被陳經濟來請,說:「有管磚廠劉太監差人送禮來。」往前去看了。金蓮和孟玉樓站在一處,罵道:「恁不逢好死,三等九做賊強盜!這兩日作死也怎的?自從養了這種子,恰似他生了太子一般,見了俺每如同生剎神一般,越發通沒句好話兒說了。行動就睜着兩個〈毛皮〉窟礲腰喝人!誰不知姐姐有錢!明日慣的他每小廝丫頭養漢做賊,把人{入日}遍了也休要管他!說着,只見西門慶坐了一回,往前邊去了。孟玉樓道:「你還不去?他管情往你屋裡去了。」金蓮道:「可是他說的,有孩子屋裡面熱鬧。俺每沒孩子的屋裡冷清。」正說着,只見春梅從外來。玉樓道:「我說他往你屋裡去了,你還不信哩!這春梅來叫你來了。」一面叫過春梅來問他。春梅道:「我來問玉簫要汗巾子來。他今日借了我汗巾子戴來。」玉樓問道:「你爹在那裡?」春梅:「爹往六娘房裡去了。」這金蓮聽了,心上如攛上一把火相似,罵道:「賊強人!到明日永世千年,就跌折腳也別要進我那屋裡。踹踹門檻兒,教那牢拉的囚根子把懷子骨〈扌歪〉折了。」玉樓道:「六姐,你今日怎的下恁毒口呪他?」金蓮道:「不是這說,賊三寸貨強盜那鼠雞腸的心兒,只好有三寸大一般。都是你老婆,無故只是多有了這點尿胞種子罷了。難道怎麼樣兒的?做甚麼恁擡一個滅一個,把人躧到泥裡?」正是:

「大風刮倒梧桐樹,  自有旁人話短長。」

這裡金蓮使性兒不題。且說門慶走到前邊,薛太監差了家人送了一罈內酒 ,一牽羊,兩疋金段,一盤壽桃,一盤壽麵,四樣〈革肴〉餚,一者祝壽,二者來賀。西門慶厚賞來人,打發去了。到後邊有李桂姐、吳銀兒兩個拜辭要家去。西門慶道:「你每兩個再住一日兒,到二十八日我請你帥府周老爹和提刑夏老爹、都監荊老爹、管皇庄薛公公和磚廠劉公公,有院中親耍扮戲的,教你二位只專遞酒。」桂姐道:「既留下俺每,我教人項頭家去回媽聲,放心些。」于是把兩人轎子都打發去了,不在話下。次日,西門慶在大廳上錦屏羅列,綺席鋪陳,預先發柬請官客飲酒。因前日在皇庄見管磚廠劉公公,故與薛內相都送了禮來。西門慶這裡發柬請他,又邀了應伯爵、謝希大兩個相陪。從飯時,各人衣帽齊整,又早先到了。西門慶讓他捲棚內坐待茶。伯爵因問:「今日哥席間請那幾客?」西門慶道:「有劉、薛二內相、帥府周大人,都監荊南江、敝同僚夏提刑、團練張總兵、衛士范千戶、吳大哥、吳二哥,喬老便今日使人來回了不來,連二位通只數客。」說畢,適有吳大舅、二舅到,作了揖,同坐下。左右放卓兒擺飯。吃畢,應伯爵因問:「哥兒滿月,抱出來不曾?」西門慶道:「也是因眾堂客要看,房下說且休教孩兒出來,恐風試着他。他奶子說不妨事,教奶子用被裹出來,他大媽屋裡走了遭,應了個日子兒,就進屋去了。」伯爵道:「那日嫂子這裡請去,房下也要來走走。百忙他舊時那疾又舉發了,起不的炕兒,心中急的要不的。如今趁人未到,爹倒好說聲,抱哥兒出來,俺每同看一看。」西門慶一面分付後邊:「慢慢抱哥出來,休要諕着他。對你娘說,大舅、二舅在這裡和應二爹、謝爹要看一看。」月娘教奶子如意兒用紅綾小被兒裹的緊緊的,送到捲棚角門首,玳安兒接抱到捲棚內。眾人睜眼觀看,官哥兒穿着大紅段毛衫兒,生的面白紅唇,甚是富態。都喝采誇獎不已。伯爵與希大,每人袖中掏出一方錦段兜肚,上着一個小銀墜兒。惟應伯爵與一柳五色線,上穿着十數文長命錢,教與玳安兒:「好生抱回房去,休要驚諕哥兒。」說道:「相貌端正,天生的就是個戴紗帽胚胞兒!」西門慶大喜,作揖謝了他二人重禮。伯爵道:「哥沒的說,惶恐表意罷了。」說話中間,忽報劉公公、薛公公來了。慌的西門慶穿上衣,儀門迎接。二位內相坐四人轎,穿過肩蟒,纓鎗隊。喝道而至。西門慶先讓至大廳上拜見,敍禮接茶。落後周守備荊都監、夏提刑等眾武官,都是錦綉服,遁藤棍,大扇,軍牢喝道,僚椽跟隨。須臾,都到了門首,黑壓壓的許多伺候。裡面鼓樂喧天;笙簫迭奏。上坐遞酒之時,劉、薛二內相相見。廳正面設十二張卓席,都是幗拴錦帶,花插金瓶。卓上擺着簇盤定勝 ,地下鋪着錦裀綉毬。西門慶先把盞讓坐次。劉、薛二內相再三讓遜:「還有列位大人。」周守備道:「二位老太監齒德俱尊。常言三歲內宦,居於王公之上。這個自然首坐,何消泛講?」彼此讓遜了一回,薛內相道:「劉哥,既是列位不肯,難為柬家。咱坐了罷。」于是羅圈唱了個諾,打了恭。劉內相居左,薛內相居右,每人膝下放一條手巾,兩小廝在傍打扇,就坐下了。其次者纔是周守備、荊都監眾人。須臾,階下一派簫韶,動起樂來。怎的的當日好筵席?但見:食烹異品,菓獻時新。須臾,酒過五巡,湯陳三獻。廚役上來割了頭一道小割燒鵝 ,先首位劉內相賞了五錢銀子。教坊司俳官跪呈上大紅布手本,下邊簇擁一段笑樂的院本,當先是外扮節級上開:

「法正天心順,官清民自安。妻賢夫禍少,子孝父心寬。小人不是別人,乃是上廳節級是也。手下管著許多長行樂俑匠。昨日市上買了一架圍屏,上寫著滕王閣的詩。訪問人,請問人,說是唐朝身不滿三尺王勃殿試所作。自說此人下筆成章,廣有學問,乃是個才子。我如今叫傅末抓尋著,請得他來,見他一見,有何不可?傅末的在那裡?」末云:「堂上一呼,階下百諾。稟復節級,有何使令?」外云:「我昨日見那圍屏上寫的滕王閣詩甚好,聞說乃是唐朝身不滿三尺王勃殿試所作。我如今這個樣板去,恨即時就替我請去。請得來,一錢賞賜;請不得來,二十麻杖,決打不饒。」末云:「小人理會了。」轉下去:「節級糊塗。那王勃殿試,從唐時到如今,何止千百餘年,教我那裡抓尋他去?」不免來來去去,到於文廟門首,遠遠望見一位飽學秀士過來,不免動問他一聲:「先生,你是做滕王閣詩的,身不滿三尺王勃殿試麼?」淨扮秀才笑云:「王勃殿試乃唐朝人物,今時那裡有?試哄他一哄。我就是那王勃殿試,滕王閣的詩是我做的。我先念兩句你聽:『南昌故郡,洪都新府。星分翼軫,文光射斗牛之墟;人傑地靈,徐孺下陳蕃之榻。』末云:「俺節級與了我這副樣板,身只要三尺,差一指也休請去。你這等身軀,如何充得過?」淨云:「不打緊。道在人為。你見那裡又一位王勃殿試來了。」(皆粧矮子來。將樣板比。淨越縮。)末笑云:「可充得過了。」淨云:「一件,見你節級,切記好歹小板凳兒要緊。」來來去去到節級門首。末令淨外邊伺候。淨云:「小板凳兒要緊,等進去稟報節級。」外云:「你請得那王勃殿試來了?」末云:「見請在門外伺候。」外云:「你與說,我在中門相待。榛松泡茶 ,割肉水飯 。」相見科外云:「此真乃王勃殿試也!一見尊顏,三生有幸!」磕下頭。淨慌科:「小板凳在那裡?」外又云:「亙古到今,難逢難遇。聞名不曾見面。今日見面,勝若聞名。」再磕下頭去。那淨慌科:「小板凳在那裡?」末躲過一邊去了。外云:「聞公博學廣記,筆底龍蛇,真才子也!在下如渴思槳,如熱思涼,多拜兩拜。」淨急了說道:「你家爺好?你家媽好?你家姐和妹子,一家兒都好?」外云:「都好。」淨云:「狗{入日}娘的,你既一家大小都好?也教我直直腰兒著!」正是:

「百寶粧腰帶,  珍珠絡臂鞲,

笑時能近眼,  舞罷錦纏頭。」

筵前遞酒,席上眾官都笑了。薛內相大喜,叫上來賞了一兩銀子,磕頭謝了。須臾,李銘、吳惠兩個小優兒,上來彈唱了。一個擽箏,一個琵琶。周守備先舉手讓兩位內相說:「老太監,分付賞他二人唱那套詞兒?」劉太監道:「列位請先。」周守備道:「老太監,自然之理,不必計較。」劉太監道:「兩個子弟,唱個『嘆浮生有如一夢裡』周守備道:「老太監此是這歸隱嘆世之詞,今日西門大人喜事,又是華誕,唱不的。」劉太監又道:「你會唱『雖不是八位中紫綬臣,管領的六宮中金釵女?』周守備道:「此是陳琳抱粧盒雜記,今日慶賀唱不的。」薛太監道:「叫他二人上來等我分付他。你記的普天樂『想人生最苦是離別?』夏提刑大笑道:「老太監,此是離別之詞,越發使不的。」薛太監道:「俺每內官的營生,只曉的答應萬歲爺,不曉的詞曲中滋味,憑他每唱罷。」夏提刑倒還是金吾執事人員,倚仗他刑名官,一樂工上來,分付:「你套唱三十腔。今日是你西門老爹加官進祿,又是好的日子,又是弄璋之喜,宜該唱這套。」薛內相問:「這怎的弄璋之喜?」周守備道:「二位老太監,此日又是西門大人公子彌月之辰,俺每同僚都有薄禮慶賀。」薛內相道:「我等,」因向劉太監道:「劉家,咱每明日都補禮來慶賀。」西門慶謝道:「學生生一豚犬,不足為賀,到不必老太監費心。」說畢,喚玳安裡邊交出吳銀兒、李桂姐席前遞酒。兩個唱的打扮出來,花枝招颺,望上不端不正插燭也似磕了四個頭兒。起來執壺斟酒,逐一敬奉。兩個樂工又唱一套新詞,歌喉宛轉,真有遶梁之聲。當夜前歌後舞,錦簇花攢,直飲至更餘時分,方纔薛內相起身,說道:「生等一者過蒙盛情,二者又值喜慶,不覺留連暢飲,十分擾極。學生告辭。」西門慶道:「杯茗相邀,得蒙光降,頓使蓬蓽增輝。幸再寬坐片時,以畢餘興。」眾人俱出位說道:「生等深擾,酒力不勝。」各躬身施禮相謝。西門慶再三款留不住,只得同吳大舅、吳二舅等一齊送至大門。一派鼓樂喧天,兩邊燈火燦爛,前遮後擁,唱道而去。正是:

「得多少歡娛嫌日短,  故燒高燭照紅粧。」

畢竟後項未知如何,且聽下回分解:

