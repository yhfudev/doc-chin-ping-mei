%# -*- coding: utf-8 -*-
%!TEX encoding = UTF-8 Unicode
%!TEX TS-program = xelatex
% vim:ts=4:sw=4
%
% 以上设定默认使用 XeLaTex 编译,并指定 Unicode 编码,供 TeXShop 自动识别

%第六十回 
\chapter{李瓶兒因暗氣惹病\KG 西門慶立段舖開張}


「赤繩緣盡再難期,  造化焦端敢恨誰,

殘淚驚秋和葉落,  斷魂隨月到窗遲;

金風拂面思兒處,  玉燭成灰墮淚時,

任是肝腸如鐵石,  不生悲也自生悲。」

話說當日孫雪娥、吳銀兒兩個,在旁邊勸解了李瓶兒一回云云,到後邊去了。那潘金蓮見孩子沒了,李瓶兒死了生兒,每日抖擻精神,百般的稱快。指着丫頭罵道:「賊淫婦,我只說你日頭常晌午,都怎的今日也有錯了的時節?你班鳩跌了,彈也嘴答谷了;春凳折了靠背兒,沒的倚了;王婆子賣了磨,推不的了;老鴇子死了粉頭,沒指望了。都怎的也和我一般。」李瓶兒這邊屋裡分明聽見,不敢聲言,背地裡只是弔淚。着了這暗氣暗惱,又加之煩惱憂戚,漸漸心神恍亂,夢魂顛倒兒。每日茶飯,都減少了。自從墳上葬埋了官哥兒回來,第二日吳銀兒就家去了。老馮領了十三歲丫頭來賣與孫雪娥房中使喚,要了五兩銀子,改名翠兒,不在話下。這李瓶兒一者思念孩兒,二者着了重氣,把舊時病症,又發起來,照舊下邊經水淋漓不止。西門慶請任醫官來看一遍,討將藥來吃下去。如水澆石一般,越吃藥越旺。那消半月之間,漸漸容顏頓減,肌膚消瘦,而精彩丰標,無復昔時之態矣。正是

「肌骨大都無一把,  如何禁架許多愁。」

一日九月初旬,天氣凄涼,金風漸漸。李瓶兒夜間獨宿在房中。銀床枕冷,紗窗月浸。不覺思想孩兒,欷歔長歎。似睡不睡,恍恍然恰似有人彈的窗櫺响。李瓶兒呼喚丫鬟,都睡熟了不答。乃自下床來,倒靸弓鞋,翻披綉襖,開了房門,出戶視之。彷彿見花子虛抱着官哥兒叫他,新尋了房兒,同去居住。這李瓶兒還捨不的西門慶,不肯去。雙手就去抱那孩兒。被花子虛只一推,跌倒在地。撒手驚覺,都是南柯一夢。嚇了一身冷汗,嗚嗚咽咽,只哭到天明。正是:

「有情豈不等,  著相自家迷。」

有詩為證:

「纖纖新月照銀屏,  人在幽閨欲斷魂,

益悔風流多不足,  須知恩愛是愁根。」

那時來保南京貨船又到了,使了後生王顯上來取單稅銀兩。西門慶這裡寫書差榮海拏一百兩銀子,又具羊酒金段禮物謝主事。就說此船貨過稅,還望青目一二。家中收拾舖面完備,又擇九月初四日開張。就是那日卸貨,連行李共裝二十大車。那日親朋遞果盒、掛紅者,約有三十多人。喬大戶叫了十二名吹打的樂工,雜耍撮弄。西門慶這裡,李銘、吳惠、鄭春三個小優兒彈唱。甘夥計與韓夥計都在櫃上發賣。一個看銀子,一個講說價錢。崔本專管收生活,不拘經紀買主進來,讓進去,每人飲酒二杯。西門慶穿大紅冠帶着。燒罷布,各親友都遞果盒。把盞畢,後邊廳上安放十五張桌席,五果五菜,三湯五割 ,從新遞酒上坐,鼓樂喧天。那日夏提刑家,差人送禮花紅來。西門慶回了禮物,打發去了。在座者有喬大戶、吳大舅、吳二舅、花大舅、沈姨夫、韓姨夫、吳道官、倪秀才、溫葵軒、應伯爵、謝希大、常時節。原來西門慶近日與了他五十兩銀子,使了三十五兩,典了房子。十五兩銀子做本錢,在家開了個小小雜貨舖兒,過其日月不題。近隨眾出分資來,與西門慶慶賀。還有李智、黃四、傅自新等眾夥計主管,并街坊鄰舍,都坐滿了席面。三個小優兒在席前唱了一套南呂紅衲襖:「混元初,生太極」云云。須臾,酒過五巡,食割三道。下邊樂工吹打彈唱,雜耍百戲過去,席上觥籌交錯。當日應伯爵、謝希大飛起大鍾來,杯來盞去,飲至日落時分。把眾人打發散了,西門慶只留下吳大舅、沈姨夫、倪秀才、溫葵軒、應伯爵、謝希大,從新擺上卓席,留後坐。那日新開張,夥計攢帳,就賣了五百餘兩銀子。西門慶滿心歡喜。晚夕收了舖面,把甘夥計、韓夥計、傅夥計、崔本、賁四連陳經濟,都邀來到席上飲酒。吹打良久,把吹打樂工打發去了,止留下三個小優兒在席前唱。那應伯爵坐了一日,吃的已醉上來。出來前邊解手,叫過李銘,問李銘:「那個紮包髻兒的清俊小優兒,是誰家的?」李銘道:「二爹不知道?」因掩口說道:「他是鄭奉的兄弟鄭春。前日爹在裡邊他家吃酒,請了他姐姐愛月兒了。」伯爵道:「真個?悝道前日上布送殯都有他!」于是歸到酒席上,向西門慶道:「哥你又恭喜,又擡了小舅子了。」西門慶笑道:「怪狗材,休耍胡說。」一面叫過王經來:「斟與你應二爹一大杯酒。」伯爵向吳大舅說道:「老舅,你怎麼說?這鍾罰的我沒名。」西門慶道:「我罰你這狗材,一個出位妄言!」那伯爵低頭想了想兒,呵呵笑了道:「不打緊處。等我吃我吃,死不了人。」又道:「我從來吃不得啞酒。你叫鄭春上來唱個兒我聽,我纔罷了。」當下三個小優,一齊上來彈唱。伯爵令李銘、吳惠下去:「不要你兩個。我只要鄭春單彈着箏兒,只唱個小小曲兒我下酒罷。」謝希大叫道:「鄭春你過來,依着你應二爹唱。」西門慶道:「和花子講過,有個曲兒吃一鍾酒。」于是玳安旋取了兩個大銀鍾,放在應二面前。那鄭春款按銀箏,低低唱清江引道:

「一個姐兒十六七,見一對蝴蝶戲。香肩靠粉墻,春箏彈珠淚。喚梅香,趕他去別處飛。」

鄭春唱了個:「請酒!」伯爵剛纔飲訖,那玳安在旁連忙又斟上一盃酒,鄭春又唱道:

「轉過雕闌正見他,斜倚定荼{艹縻}架。佯羞整鳳釵,不說昨宵話。笑吟吟,捏將花片兒打。」

伯爵吃過,連忙推與謝希大,說道:「罷,我是成不的,成不的!這兩大鍾把我就打發的了。」謝希大道:「俊化子,你吃不的,推于我來?我是你家有〈毛皮〉的蠻子?」伯爵道:「俊花子,我明日就做了堂上官兒,少不的是你替。」西門慶道:「你這狗材,到明日只好做個韶武。」伯爵笑道:「俊孩兒,我做了韶武,把堂上讓與你就是了。」西門慶笑令玳安兒:「拏磕瓜來打這賊花子。」那謝希大悄悄向他頭上打了一個响瓜兒,說道:「你這花子,溫老先生在這裡,你口裡只恁胡說。」伯爵道:「溫老先兒他斯文人,不管這閒事。」溫秀才道:「二公與我這東君老先生,原來這等厚。酒席中間,誠然不如此,也不樂。悅在心,樂主發散在外。自不覺手之舞之,足之蹈之如此。」座上沈姨夫向西門慶說:「姨夫,不是這等。請大舅上席還行個令兒,或擲骰,或猜枚,或看牌,不拘詩詞歌賦,頂真續麻急口令,說不過來,吃酒。這個庶幾均勻,彼此不亂。」西門慶道:「姨夫說的是。」先斟了一杯,與吳大舅起令。吳大舅拏起骰盆兒來,說道:「列位,我行一令,說差了,罰酒一杯。先用一骰,後用兩骰,遇點飲酒。」

「一百萬軍中捲白旗,  二天下豪傑少人知,

三秦王斬了余元帥,  四罵得將軍無馬騎,

五諕得吾今無口應,  六袞袞街頭脫去衣,

七皂人頭上無白髮,  八分屍不得帶刀歸,

九一丸好藥無人點,  十千載終須一撇離。」

吳大舅擲畢,遇有兩點,飲過酒。該沈姨夫起令,說道:「用一骰六擲,遇點飲酒。」說道:

「天象六色地象雙,  人數推來中二紅,

三見巫山梅五出,  算來花有幾人通?」

當下只遇了個四紅,飲過一杯,過盆與溫秀才。秀才道:「我學生奉令了。遇點要一花名,名下接四書一句頂。」

「一擲一點紅,紅梅花對白梅花。  二擲並頭蓮,蓮漪戲彩鴛。  三擲三春柳,柳下不整冠。  四擲狀元紅,紅紫不以為褻服。  五擲臘梅花,花迎劍珮星初落。六擲滿天星,星辰之遠也。」

溫秀才只遇了一鍾酒,該應伯爵行令。伯爵道:「我在下一個字也不識,行個急口令兒罷!」

「一個急急腳腳的老小,左手拏著一個黃豆巴斗,右手拏著一條綿花叉口,望前只管跑走。撞著一個黃白花狗,咬著那綿花叉口。那急急腳腳的老小,放下那左手提的那黃豆巴斗,走向前去打黃白花狗。不知手鬬過那狗,狗鬬過那手?」

西門慶笑罵道:「你這賊謅斷了腸子的天殺的,誰家一個手去鬬狗來!一口不被那狗咬了?」伯爵道:「誰教他不拏個棍兒來?我如今抄化子不見了拐棒兒,受狗的氣了!」謝希大道:「大官人,你看花子自家倒了柴,說他是花子。」西門慶道:「該罰他一鍾,不成個令。謝子純,你行罷。」謝希大道:「我這令兒比他更妙。說不過來,罰一鍾。」

「墻上一片破瓦,墻上一疋騾馬。落下破瓦,打著騾馬。不知是那破瓦打傷騾馬,不知是那騾馬踏碎了破瓦?」

伯爵道:「你笑話我的令不好,你這破瓦倒好?你家娘子兒劉大姐就是個騾馬,我就是個破瓦。俺兩個破磨對腐騾。」謝希大道:「你家那杜蠻婆老淫婦,撒把黑豆,只好喂豬拱狗,也不要他!」兩個人鬬了回嘴,每人罰了一鍾。該傅自新行令。傅自新道:「小人行個江湖令,遇點飲酒,先一後二。」

「一舟二櫓,三人搖出四川河;五音六律,七人齊唱八仙歌。九十春光齊賞翫,十一十二慶元和。」

擲畢,皆不遇。吳大舅道:「總不如傅黟計這個令兒行得切實些。」伯爵道:「太平鍾,也該他吃一杯兒。」于是親下席來,斟了一杯與傅自新吃。如今該韓夥計。韓道國道:「老爹在上,小人怎敢占先?」西門慶道:「你每行過,等我行罷。」于是韓道國道:「頭一句要天上飛禽,第二句要果名,第三句要骨牌名,第四句要一官名。俱要貫串,遇點照席飲酒。」說:

「天上飛來一仙鶴,  落在園中吃鮮桃,

卻被孤紅拏住了,  將去獻與一提學。

天上飛來一鷂鶯,  落在園中吃朱櫻,

卻被二姑拏住了,  將去獻與一公卿。

天上飛來一老鸛,  落在園中吃菱芡。

卻被三綱拏住了,  將去獻與一通判。

天上飛來一班鳩,  落在園中吃石榴,

卻被四紅拏住了,  將來獻與一戶侯。

天上飛來一錦雞,  落在園中吃苦株,

卻被五岳拏住了,  將來獻與一尚書。

天上飛來一淘鵝,  落在園中吃蘋波,

卻被綠暗拏住了,  將來獻與一照磨。」

擲畢,該西門慶擲。西門慶道:「我只擲四擲,遇點飲酒。」

「六口載成一點霞,  不論春色見梅花,

摟抱紅娘親個嘴,  拋閃鶯鶯獨自嗟。」

擲到遇紅一包,果然擲出個四來。應伯爵看見,說道:「哥今年上冬,管情高轉加官,主有慶事。」于是斟了一大杯酒與西門慶。一面喚李銘等三個,上來彈唱頑耍,至更闌方散。西門慶打發小優兒出門,看着收了家火。派定韓道國、甘夥計、崔本、來保四人輪流上宿。分付仔細門戶,就過那邊去了。一宿晚景不題。都說次日,應伯爵領了李智、黃四來交銀子,說:「此遭只關了一千四百五六十兩銀子,不勾還人。只挪了這三百五十兩銀子與老爹。等下遭銀子關出來,再找完,不敢遲了。」伯爵在旁,又替他說了兩句美言。西門慶把銀子教陳經濟來拏天平兌收明白,打發去了。銀子還擺在卓上。西門慶因問伯爵道:「常二哥說,他房子尋下了,前後四間,只要三十五兩銀子就賣了。他來對我說。正值小兒病重了,我心裡正亂着哩,打發他去了。不知他對你說來不曾?」伯爵道:「他對我說來。我說你去的不是了,他迺郎不好,他自亂亂的,有甚麼心緒和你說話?你且休回那房主兒,等我來見哥替你題就是了。」西門慶聽了便道:「也罷,你吃了飯,拏一封五十兩銀子,今日是個好日子,替他把房子成了來罷。剩下的教常二哥門面開個小本舖兒,月間撰的幾錢銀子兒,勾他兩口兒盤攪過來就是了。」伯爵道:「此是哥下顧他了。」不一時,放卓兒,擺上飯來。西門慶陪他吃了飯道:「我不留你。你拏了這銀子去,替他幹幹這勾當去罷。」伯爵道:「你這裡還教個大官,和我兩個拏這銀子去。」西門慶道:「沒的扯淡,你袖了去就是了。」伯爵道:「不是這等說。今日我還有小事去。實和哥說,家表弟杜三哥生日,早辰我送了些禮兒去。他使小廝來請我後晌坐坐,我不得來回你。教個大官兒跟了去,成了房子,我教大官兒好來回你。」說罷,西門慶道:「若是恁說,教王經跟了你去罷。」一面叫了王經跟伯爵去了。到了常時節家,常時節正在家。見伯爵至,讓進裡面坐。伯爵拏出銀子來與常時節看,說:「大官人如此如此,教我同你今日成房子去。我又不得閒,杜三哥請我吃酒。我如今了畢你的事,我方纔得去。所以叫大官兒跟了我來。成了房子,我不回他爹話去,教他回回便了。」常時節連忙叫渾家快看茶來,說道:「哥的盛情!誰肯?」一面吃畢茶,叫了房中人來,同到新市街兌與賣主銀子,寫立房契。伯爵分付與王經,歸家回西門慶話。剩的銀,教與常時節收了。他便與常時節作別,往杜家吃酒去了。西門慶看了文契,還使王經:「送與你常二叔收了。」不在話下。正是:

「求人須求大丈夫,  濟人須濟急時無;

一切萬般皆下品,  誰知陰德是良圖。」

正是:

「玉光有影遺誰繫?  萬事無根只自生。」

畢竟未知後來何如,且聽下回分解:
