%# -*- coding: utf-8 -*-
%!TEX encoding = UTF-8 Unicode
%!TEX TS-program = xelatex
% vim:ts=4:sw=4
%
% 以上设定默认使用 XeLaTex 编译,并指定 Unicode 编码,供 TeXShop 自动识别

%第一00回 
\chapter{韓愛姐湖州尋父\KG 普靜師薦拔群冤}

格言

「人生切莫將英雄,  術業精粗自不同,

猛虎尚然遭惡獸,  毒蛇猶自怕蜈蚣;

七擒猛獲恃諸葛,  兩困雲長羨呂蒙,

珍重李安真智士,  高飛逃出是非門。」

話說韓道國與王六兒,歸到謝家酒店內,無女兒,道不得個坐吃山崩。使陳三兒去,又把那何官人來續上。那何官人見他地方中沒了劉二,除了一害,依舊又來王六兒家行走。和韓道國商議:「你女兒愛姐,已是在府中守孝,不出來了。等我賣盡貨物討了賒帳,你兩口跟我往湖州家去罷,省得在此做這般道路!」那韓道國說:「官人下顧,可知好哩!」一日賣盡了貨物,討上賒帳,顧了船,同王六兒跟往湖州去了。卻表愛姐在府中,與葛翠屏兩個持貞節,姊妹稱呼,甚是合當著。白日裏與春梅做伴兒在一處。那時金哥兒大了,年方六歲。孫二娘所生玉姐,年長十歲。相伴兩個孩兒,便有甚事做。誰知自從陳經濟死後,守備又出征去了。這春梅每日珍饈百味,綾錦衣衫,頭上黃的金,白的銀,圓的珠,光照的無般不有。只是晚夕難禁,獨眠孤枕慾火燒心。因見李安一條好漢。只因打殺張勝,巡風早晚十分小心。一日冬月天氣,李安正在班房內上宿。忽聽有人敲後門。忙問道:「是誰?」只聞叫道:「你開門則個。」李安連忙開了房門,卻見一個人搶入來,閃身在燈光背後。李安看時,卻認的是養娘金匱。李安道:「養娘,你這晚來有甚事?」金匱道:「不是我私來,裏邊奶奶差出我來的。」李安道:「奶奶教你來怎麼?」金匱笑道:「你好不理會得!看你睡了不曾,教我把一件物事來與你。」向背上取下一包衣服:「把與你,包內又有幾件婦女衣服,與你娘。前日多累你押解老爺行李車輛,又救得奶奶一命。不然,也吃張勝那廝殺了!」說畢,留下衣服,出門走了兩步,又回身道:「還有一件要緊的!」又取出一錠五十兩大元寶來,撇與李安自去了。當夜過了一宿。次早起來,逕拏衣服到家,與他母親。做娘的問道:「這東西是那裏的?」李安把夜來事說了一遍。做母的聽言叫苦:「當初張勝幹壞了事,一百棍打死。他今日把東西與你,卻是甚麼意思?我今六十已上年紀。自從沒了你爹爹,滿眼只看著你。若是做出事來,老身靠誰?明早便不要去了!」李安道:「我不去,他使人來叫,如何答應?」婆婆說:「我只說你感冒風寒,病了。」李安道:「終不成不去,惹老爺不見怪麼?」做娘的便說:「你且投到你叔叔山東夜叉李貴那裏住上幾個月,再來看事故何如?」這李安終是個孝順的男子,就依著娘的話,收拾行李,往青州府投他叔叔李貴去了。春梅以後見李安不來,三四五次,使小伴當來叫。婆婆初時答應家中染病。次後見人來驗看,纔說往原籍家中打盤纏去了。這春梅終是惱恨在心不題。

時光迅速,日月如梭,又早臘月盡陽日,正月初旬天氣。統制領兵一萬二千,在東昌府屯住已久,使家人周忠捎書來家,教搬取春梅孫二娘,並金哥玉姐家小上車,止留下周忠:「東庄上請你二爺看守宅子。」原來統制還有個族弟周宣,在庄上住。周忠在府中,與周宣葛翠屏,韓愛姐看守宅。周仁與眾軍牢保定車輸往東昌府來。此這一去,不為名離故土,爭知此去少回程!有詞一篇,單道這周統制,果然是一員好將材!當此之時,中原蕩掃,志欲吞胡!但見:

「四方盜起如屯蜂,狼煙烈焰薰天紅。將軍一怒天下息,腥膻掃盡夷從風!公事忘私愿已久,此身許國不知有。金戈抑日酬戰征,麒麟圖畫功為首!雁門關外秋風烈,鐵衣披張臥寒月。汗馬卒勤二十年,贏得班班鬢如雪!天子明見萬里餘,幾番勞勣來旌書。肘懸金印大如斗。無負堂堂七尺軀!」

有日,周仁押家眷車輛到於東昌。統制見了春梅,孫二娘、金哥、玉姐,眾丫鬟家小都到了,一路平安,心中大喜。就在統制府衙後廳居住。周仁悉把東庄上叫了二爺周宣來宅,同小的老子周忠,看守宅舍。周統制又問:「怎的李安不見?」春梅道:「又題甚李安那廝!我因他捉獲了張勝,好意賞了他兩件衣服,與他娘穿。他到晚夕巡風,進入後廳,把他二爺東庄上收的籽籽銀一包五十兩,放在明間桌上,偷的去了。幾番使伴當叫他,只是推病不來。落後又使叫去,他躲的上青州原藉家去了。」統制便道:「這廝我倒看他,原來這等無恩!等我慢慢差人拏他去。」這春梅不題起韓愛姐之事。過了幾日,春梅見統制日遂理論軍情,幹朝庭國務,焦心勞思,日中尚未暇食。至於房幃色慾之事,久不沾身。因見老家人周忠次子周義,年十九歲,生的眉清目秀,眉來眼去,兩個暗地私通,就抅搭了。朝朝暮暮,兩個在房中下棋飲酒,只瞞過統制一人不知。一日,不想北國大金皇帝滅了遼國,又見東京欽宗皇帝登基,集大勢番兵,分兩路寇亂中原。大元帥粘沒喝,領十萬人馬,出山西太原府井陘道,來搶東京;副元帥斡離不,由檀州來搶高陽關。邊兵抵擋不住,慌了兵部尚書李綱,大將種師道,星夜火牌羽書,分調山東、山西、河南、河北、關東、陝西,分六路統制人馬,各依要地防守截殺。那時陝西劉延慶領延綏之兵,關東王稟領汾絳之兵,河北王煥領魏博之兵,河南辛興宗領彰衛之兵,山西楊惟忠領澤潞之兵,山東周義領青兗之兵。卻說周統制見大勢番兵來搶邊界,兵部羽書大牌星火來,連忙整率人馬,全裝披掛,兼道進兵。比及哨馬到高陽關上,金國斡離不由人馬,已搶進關來,殺死人馬無數。正值五月初旬,交陣堵截,黃沙四起,大風迷目。統制提兵進趕,不防被活立兜馬反攻,沒鞦一箭,正射中咽喉,墮馬而死。眾番將就用鈎索搭去。被這邊將士向前,僅搶屍首,馬載而還。所傷軍兵無數。可憐周統制,一旦陣亡!亡年四十七歲。正是:

「於家為國忠良將,  不辨賢愚血染沙!」

古人意不盡,作詩一首以嘆之曰:

「勝敗兵家不可期,  安危端自命為之;

出師未捷身先喪,  落日江流不勝悲。」

又鷓鴣天一首:

「定國安邦美丈夫,  心存正道氣吞胡,

謨謀國事如家事,  軍用陰符佩虎符;

胡騎盛,武功弛,  兵不用命將驕痴,

可憐身死沙場內,  千載英魂恨未舒!」

巡撫張叔夜見統制折於陣上,連忙鳴金收軍,查點折傷士卒。退守東昌,星夜奏朝庭,不在話下。部下卒載屍首,還到東昌府。春梅合家大小,號哭動天。合棺木盛殮,交割了兵符印信。一日,春梅與家人周仁,發喪載靈柩,歸清河縣不題。話分兩頭,單表葛翠屏與韓愛姐,自從春梅去後,兩個在家清茶淡飯,守節持貞,過其日月。正值春盡夏初天氣,景物鮮明。日長針指困倦,姊妹二人閑中徐步到西書院花亭上。見百花盛開,鶯啼燕語,觸景傷情。葛翠屏心還坦然;這韓愛姐一心只想念男兒陳經濟大官人,凡事無情無緒。睹物傷悲,口是心苗,形吟咏者,有詩數首為證:

翠屏先道:

「花開靜院日初晴,  深鎖重門白晝清;

倒倚銀屏春睡醒,  綠槐枝上一聲鶯。」

愛姐道:

「春事闌珊首夏時,  弓鞋款款出簾遲;

晚來悶倚粧臺立,  巧畫蛾眉為阿誰!」

翠屏又道:

「紅綿掩鏡照窗紗,  畫就雙蛾八字斜;

蓮步輕移何處去,  階前笑折石榴花。」

愛姐道:

「雪為容貌玉為神,  不遣風流涴此身;

顧影自憐還自惜,  新粧好好為何人!」

翠屏道:

「莎草連綿厚似毡,  榆莢遍地亂如錢;

誰知蕩子多輕薄,  沉醉終朝花下眠。」

愛姐道:

「亂愁依舊銷翠舉,  為甚年來瞧悴容;

離別終朝魂耿耿,  碧霄無路得相逢。」

姊妹兩個吟詩已畢,不覺潸然淚下。二爺周宣走來勸道:「你姊妹兩個,少要煩惱,須索解嘆省過罷。我連日做得夢,有些不吉。夢見一張弓,掛在旗竿上;旗竿折了,不知是凶是吉!」韓愛姐道:「倒只怕老爺邊上有些說話!」正在猶疑之間,忽見家人周仁掛著一身孝,荒荒張張走來報道:「禍事!老爺如此這般,五月初七日在邊關上陣亡了!大奶奶、二奶奶家眷載著靈車,都來了。」慌了二爺周宣,收拾打掃前廳乾淨,停放靈柩,擺下祭祀,合家大小哀號起來。一面做齋累七,僧道念經。金哥、玉姐披麻帶孝,弔客往來,擇日出殯,安葬於祖塋,俱不必細說。卻說二爺周宣,引著六歲金哥兒,行文書申奏朝廷,討祭葬,襲替祖職。朝廷各降:「兵部覆題引奏,已故統制周秀,奮身報國,沒於王事。忠勇可加!遣官諭祭一壇,墓頂追封都督之職。伊子照例優養,出幼襲替祖職。」這春梅在內頤養之餘,淫情愈盛。常留周義在香閣中,鎮日不出。朝來暮往,淫慾無度,生出骨蒸癆病症。逐日吃藥,減了飲食,消了精神,體瘦如柴,而貪淫不巳,一日過了他生辰,到六月伏暑天氣,早辰晏起。不料他摟著周義在床上,一泄之後,鼻口皆出涼氣,淫津流下一窪口,就嗚呼哀哉,死在周義身上。亡年二十九歲。這周義見沒了氣兒,就慌了手腳。向箱內盜了些金銀細軟,帶在身邊,逃走在外。丫鬟、養娘不敢隱匿,報與二爺周宣得知。把老家人周忠鎖了,押著抓尋周義。可霎作怪!正走在城外他姑娘家投住。一條索子,拴將來。巳知其情,恐揚出醜去,金哥久後不好襲職。拏到前廳,不由分說,打了四十大棍,即時打死。把金哥與孫二娘看養,一面發喪於祖塋,與統制合葬畢。房中兩個養娘并海棠、月桂,都打發各尋投向嫁人去了。止有葛翠屏與韓愛姐,再三勸他,不肯前去。一日,不想大金人馬,搶了東京、汴梁。太上皇帝與靖康皇帝,都被虜上北地去了。中原無主,四下荒亂。兵戈匝地,人民逃竄。黎庶有塗炭之哭,百姓有倒懸之苦。大勢番兵,已殺到山東地界。民間夫逃妻散,鬼哭神號,父子不相顧。葛翠屏巳被他娘家領去,各逃生命。止丟下韓愛姐無處依倚,不免收拾行裝,穿著隨身慘淡衣衫,出離了清河縣前,往臨清找尋他父母。到臨清謝家店,店也關閉,主人也走了。不想撞見陳三兒,三兒說:「你父母去年時,就跟了何官人往江南湖州去了。」這韓愛姐一路上懷抱月琴,唱小詞曲,往前抓尋到湖州何官人家,尋着父母。隨路飢食渴飲,夜住曉行。忙忙如喪家之犬,急急似漏網之魚,弓鞋又小,萬苦千辛。行了數日,來到徐州地方。天色晚來,投在孤村裏面。一個婆婆,年記七旬之上,頭綰兩道雪鬢,挽一窩絲,正在灶上杵米造飯。這韓愛姐便向前道了萬福,告道:「奴家是清河縣人氏,因為荒亂,前往江南投親。不期天晚,權借婆婆這里投宿一宵,明早就行,房金不少。」那婆婆只顧觀看這女子,不是貧難人家婢女,生的舉止典雅,容貌非俗。但見:

「烏雲不整,惟思昔日家豪。眉斂遠山,為憶當年富貴。此夜月朦雲霧瑣,牡丹花被土沉埋。」

婆婆道:「既是投宿,娘子請炕上坐。等老身造飯,有幾個挑河夫子來吃。」那老婆婆炕上柴灶,登時做出一大鍋稗稻插荳子乾飯。又切了兩大盤生菜,撮上一包鹽。只見幾個漢子,都蓬頭精腿,褌褲兜襠,腳上黃泥,流進來,放下荷鍬鐝,便問道:「老娘,有飯也未?」婆婆道:「你每自去盛吃。」當下各取飯菜,四散正吃。只見內一人,約三十四五年紀,紫面黃髮,便問婆婆:「這炕上坐的是甚麼人?」婆婆道:「此位娘子是清河縣人氏,前往江南尋父母去?天晚在此投宿。」那人便問:「娘子,你姓甚麼?」愛姐道:「奴家姓韓,我父親名韓道國。」那人向前扯住問道:「姐姐,你不是我姪女韓愛姐麼?」那愛姐道:「你倒好似我叔叔韓二。」兩個抱頭相哭做一處。因問:「你爹娘在那里?你在東京,如何至此?」這韓愛姐一五一十,從頭說了一遍:「因我嫁在守備府里,丈夫沒了,我守寡到如今。我爹娘跟了何官人往湖州去了,我要找尋去。荒亂中又沒人帶去,胡亂單身唱詞,覓些衣食前去。不想在這里撞見叔叔!」那韓二道:「自從你爹娘上東京,我沒營生過日,把房兒賣了,在這里挑河做夫子,每日覓碗飯吃。既然如此,我和你往湖州尋你爹娘去。」愛姐道:「若是叔叔同去,可知好哩!」當下也盛了一碗飯,與愛姐吃。愛姐吃了一口,見粗飯不能下咽,只吃了半碗就不吃了。一宿晚景休題過。到次日天明,眾夫子都去了。韓二交納了婆婆房錢,領愛姐作辭出門,望前途前進。那韓愛姐本來嬌嫩,弓鞋又小,身邊帶著些細軟釵梳,都在路上零碎盤纏。將到淮安上船,迤里望江南湖州來。非止一日,抓尋到湖州何官人家,尋著父母,相會見了。不想何官人巳死,家中又沒妻小,止是王六兒一人,丟下六歲女兒,有幾頃水稻田地。不上一年,韓道國也死了。王六兒原與韓二舊有揸兒,就配了小叔,種田過日。那湖州有富家子弟,見韓愛姐生的聰明標致,多來求親。韓二再三教他嫁人。愛姐割髮毀目,出家為尼姑,誓不再配他人。後年至三十二歲,以疾而終。正是:

「貞骨未歸三尺土,  怨魂先徹九重天。」

後韓二與王六兒成其夫婦,情受何官人家業田地,不在話下。卻說大金人馬,搶過東昌府來,看看到清河縣地界。只見官吏逃亡,城門晝閉,人民逃竄,父子流亡。但見:煙生四野,日蔽黃沙,封豕長蛇。互相和併。龍爭虎鬬,各自爭強。皂幟紅旗,布滿郊野。男啼女哭,萬戶驚惶。番軍虜將,一似蟻聚蜂屯;短劍長鎗,好似森林密竹。一處處死屍骸,橫三豎四;一攢攢折刀斷劍,七斷八截。個個攜男抱女,家家閉戶關門。十室九空,不顯鄉村城郭;獐奔鼠竄,那存禮樂衣冠!正是:

「得多少官人紅袖泣,  王子白衣行!」

那時西門慶家中吳月娘,見番兵到了,家家都關鎖門戶,亂竄逃去。不免也打點了些金珠寶玩,帶在身邊。那時吳大舅已死,止同吳二舅、玳安兒、小玉,領著十五歲孝哥兒,把家中前後都倒鎖了,要往濟南府投奔雲離守。一來那里避兵,二者與孝哥完就其親事去。一路上只見人人荒亂,個個驚駭。可憐這吳月娘穿著隨身衣裳,和吳二舅男女五口,雜在人隊裏,挨出城門,到於郊外,往前所行出。到於空野十字路口,只見一個和尚,身披紫褐袈裟,手執九環錫杖,腳靸芒鞋,肩上背著條布袋,袋內裹著經典,大移步迎將來,與月娘打了個問訊,高聲大叫道:「吳氏娘子,你看往那里去?還與我徒弟來!」諕月娘大驚失色,說道:「師父,你問我討甚麼徒弟?」那和尚又道:「娘子,你休推睡里夢里,你曾記的十年前在岱岳東峰,被殷天錫趕到我山洞中投宿?我就是那雪洞老和尚,法名普靜。你許下我徒弟,如何不與我?」吳二舅便道:「師父出家人,如何你不近道?此是荒亂年程,亂竄逃生。他有此孩兒,久後還要接代香火。他肯捨與你出家去?」和尚道:「你真個不與我去?」吳二舅道:「師父你休閑說,誤了人去路兒!後面只怕番兵來到,朝不保暮。」和尚道:「你既不與我徒弟,如今天色已晚,也走不出路去。番人且來不到此處,你且跟我到這寺中歇一夜,明早去罷。」吳月娘問:「師父,是那寺中?」那和尚用手只一指兒,「那路旁便是。」和尚引著,不想來到永福寺。吳月娘認的是永福寺,曾走過一遍。比及來到寺中,長老僧眾,都走去大半。止有幾個禪和尚,在後邊禪堂中打坐。佛前點著一大盞琉璃海燈,燒著一爐香。此時日色啣山時分。但見:

「十字街,熒煌燈火;九曜廟,香靄鍾聲。一輪明月掛青天,幾點疏星明碧落。六軍官內,嗚嗚畫角頻吹;五鼓樓頭,點點銅壼正滴。四邊宿霧。紛紛罩舞榭歌臺;三巿沉煙,隱隱閉綠窗朱戶。兩兩佳人歸綉閣。雙雙士子掩書幃。」

當晚吳月娘與吳二舅、玳安、小玉、孝哥兒,男女五口兒,投宿在寺中方丈內。小和尚有認的,安排了些飯食,與月娘等吃了。那普靜老師,跏跌在禪堂床上,敲木魚,口中念經。月娘與孝哥兒、小玉在床上睡,吳二舅和玳安做一處;著了慌亂,辛苦了底人,都睡著了。止有小玉,不曾睡熟,起來在方丈內,打門縫內看那普靜老師父念經。看看念至三更時,只見金風凄凄,斜月朦朦,人煙寂靜,萬籟無聲。覷那佛前海燈,半明不暗。這普靜老師,見天下荒亂,人民遭劫,陣亡橫死者數極多。發慈悲心,施廣惠力,禮白佛言世尊解冤經咒,薦拔幽魂,解釋宿冤,絕去掛礙,各去超生,再無留滯。於是誦念了百十遍解冤經咒。少頃,陰風凄凄,冷氣颼颼。有數十輩焦頭爛額,蓬頭泥面者,或斷手折臂者,或有刳腹剜心者,或有無頭跛足者,或有弔頸枷鎖者,都來悟領禪師經咒,列於兩旁。禪師便道:「你等眾生,冤冤相報,不肯解脫,何日是了?汝當諦聽吾言,隨方托化去罷!」偈曰:

「勸爾莫結冤,  冤深難解結。  一日結成冤,

千日解一徹!  若將冤報冤,  如湯去潑雪。

若將冤報冤,  如狼重見蝎!  我見結冤人,

盡被冤磨折。  我見此懺晦,  各把性悟徹。

照見本來心,  冤愆自然雪。  仗此經力深,

薦拔諸惡業。  汝當各托生,  再勿將冤結!」

「改頭換面輪迴去,  來世機緣莫再攀!」

當下眾人都拜謝而去。小玉竊看,都不認的。少頃,又一大漢進來,身七尺,形容魁偉,全裝貫來,胸前關著一矢箭。自稱:「統制周秀,因與番將對敵,折於陣上。今蒙師薦拔,今往東京托生,與沈鏡為次子,名為沈守善去也。」言未已,又一人素體榮身,口稱:「是清河縣富戶西門慶,不幸溺血而死。今蒙師薦拔,今往東京城內,托生富戶沈通為次子沈鉞去也。」小玉認的是他爹,諕的不敢言語。已而又有一人提著頭,渾身皆血,自言:「是陳經濟,因被張勝所殺。蒙師經功薦拔,今往東京城內,與王家為子去也。」已而又見一婦人,也提著頭,胸前皆血,自言:「奴是武大妻,門慶之妾,潘氏是也。不幸被仇人武松所殺。蒙師薦拔,今往東京城內黎家為女,托生去也。」已而又有一人,身軀矮小,面背青色,自言:「是武植,因被王婆唆潘氏下藥,吃毒而死。蒙師薦拔,今往徐州落鄉民范家為男,托生去也。」已而又有一婦人,面皮黃瘦,血水淋漓,自言:「妾身李氏。乃花子虛之妻,西門慶之妾,因害血山崩而死。蒙師薦拔,今往東京城內袁指揮家,托生為女去也。」已而又一男,自言:「花子虛,不幸被妻氣死。蒙師薦拔,今往東京鄭千戶家托生為男。」已而又見一女人,頸纏腳帶,自言:「西門慶家人來旺妻宋氏,自縊身死。蒙師薦拔,今往東京朱家為女去也。」已而又一婦人面黃肌瘦,自稱:「周統制妻龐氏春梅,因色癆而死。蒙師薦拔,今往東京與孔家為女,托生去也。」已而又一男子,裸形披髮,渾身杖痕,自言:「是打死的張勝,蒙師父薦拔,今往東京大興衛貧人高家為男去也。」已而又有一女人,頂上纏著索子,自言:「西門慶妾孫雪娥,不幸自縊身死。蒙師薦拔,今往東京城外貧民姚家為女去也。」已而又一女人,年小,項纏腳帶,自言:「西門慶之女,陳經濟之妻,西門大姐是也。不幸自縊身死。蒙師薦拔,今往東京城外與潘役鐘貴為女,托生去也。」已而又見一小男子,自言:「周義,亦被打死。蒙師薦拔,今往東京城外高家為男,名高留住兒,托生去也。」言畢,各恍然都不見。小玉諕的戰慄不已:「原來這和尚,只是和這些鬼說話!」正欲向床前,告訴與月娘。不料月娘睡得正熟。一靈真性,同吳二舅眾男女,身帶著一百顆胡珠,一柄寶石縧環,前往濟南府投奔親家雲離守那里避兵,就與孝哥完成親事。一路饑食渴飲,夜住曉行。到於濟南府,問一老人:「雲參將住所在於何處?」老人指道:「此去二里餘地,名靈壁寨,一邊臨河,一邊是山;這靈壁寨就在城上,屯聚有一千人馬。雲參將就在那里做知寨。」月娘五口兒到寨門,通報進去。雲參將聽見月娘遠親來了,一見如故,敘畢禮數。原來新近沒了娘子,央浼鄰舍王婆婆來陪待月娘,在後堂酒飯,甚是豐盛。吳二舅、玳安,另在一處管待。因說起避兵來就親之事,因把那百顆胡珠、寶石、縧環,教與雲離守,權為茶禮。雲離守收了,並不言其就親之事。到晚又教王婆陪月娘一處歇臥,將言說念月娘,以挑探其意說:「雲離守雖是武官,乃讀書君子。從割衫襟之時,就留心娘子。不期夫人沒了,鰥居至今。今據此山城,雖是任小,上馬管軍,下馬管民,生殺在於掌握。娘子若不棄,願成伉儷之歡,一雙兩好,令郎亦得諧秦晉之配。等待太平之日,再回家去不遲。」月娘聽言,大驚失色,半晌無言。這王婆回報雲離守。次日晚夕,置酒後堂,請月娘吃酒。月娘自知他與孝哥兒完親,連忙來到席前敘坐。雲離守乃言:「嫂嫂不知,下官在此,雖是山城,管著許多人馬,有的是財帛衣服,金銀寶物。缺少一個主家娘子;下官一向思想娘子,如渴思漿,如熱思涼。不想今日娘子到我這里,與令郎完親。天賜姻緣,一雙兩好,成其夫婦,在此快活一世,有何不可?」月娘聽了,心中大怒,罵道:「雲離守,誰知你人皮包著狗骨!我過世丈夫,不曾把你輕待,如何一旦出此犬馬之言!」雲離守笑嘻嘻向前把月娘摟住,求告說:「娘子,你自家中,如何走來我這里做甚?自古上門買賣好做。不知怎的一見你,魂靈都被你攝在身上!沒奈何,好歹完成了罷!」一面拏過酒來,和月娘吃,月娘道:「你前邊叫我兄弟來,等我與他說句話。」雲離守笑道:「你兄弟和玳安兒小廝已被我殺了!」即令左右:「取那件物事與娘子看!」不一時,燈光下血瀝瀝提了吳二舅、玳安兩顆頭來。諕的月娘面如土色,一面哭倒在地。被雲離守向前抱起:「娘子不須煩惱,你兄弟已死,你就與我為妻。我一個總兵官,也不玷辱了你。」月娘自思道:「這賊漢將我兄弟家人害了命,我若不從,連我命也喪了!」乃回嗔作喜說道:「你須依我,奴方與你做夫妻。」雲離守道:「不拘甚事,我都依。」月娘道:「你先把我孩兒完了房,我卻與你成婚。」雲離守道:「不打緊!」一面叫出雲小姐來,和孝哥兒推在一處,飲合巹盃,館同心結,成其夫婦。然後拉月娘和他雲雨。這月娘卻拒阻不肯。被雲離守忿然大怒,罵道:「賤婦!你哄的我女兒與你兒子成了婚姻,敢笑我殺不得你的孩兒?」取刀向床頭砍去,隨手而落,血濺數步之遠。正是:

「三尺利刀著頂上,  滿腔鮮血濕模糊!」

月娘見砍死孝哥兒,不覺大叫一聲。不想撒手驚覺,卻是南柯一夢。諕的渾身是汗,遍體生津。連道:「怪哉!怪哉!」小玉在旁,便問:「奶奶怎的哭?」月娘道:「適間做得一夢不祥!」不免告訴了小玉一遍。小玉道:「我倒剛纔不曾睡著,悄悄打門縫見那和尚,原來和鬼說了一夜話!剛纔過世俺爹,五娘、六娘,和陳姐夫、周守備、孫雪娥,來旺兒媳婦子、大姐,都來說話,各四散去了!」月娘道:「這寺後見埋著他每,夜靜時分,屈死淹魂,如何不來?」娘兒們也不曾說話。不覺五更雞叫,吳月娘梳洗面貌,走到禪堂中禮佛燒香。只見普靜老師在禪床上高叫:「那吳氏娘子,你如今可省悟得了麼?」這月娘便跪下參拜:「上告尊師,弟子吳氏肉眼凡胎,不知師父是一尊古佛。適間一夢中,都已省悟了!」老師道:「既已省悟,也不消前去。你就去,也無過只是如此,倒沒的喪了五口兒性命!合你這兒子有分有緣,遇著我,都是你平日一點善根所種。不然定然難免骨肉分離!當初你去世夫主西門慶,造惡非善。此子轉身,托化你家,本要蕩散其財本,傾覆其產業,臨死還當身首異處!今我度脫了他去,做了徒弟。常言:「一子出家,九祖升天!」你那夫主冤愆解釋,亦得超生去了。你不信,跟我來,與你看一看。」於是扠步來到方丈內,只見孝哥兒還睡在床。老師將手中禪杖,向他頭上只一點,教月娘眾人。忽然翻過身來,卻是西門慶,項帶沉枷,腰繫鐵索。復用禪杖只一點,依舊還是孝哥兒,睡在床上。月娘不覺見了放聲大哭,原來孝哥兒即是西門慶托生!良久,孝哥兒醒了。月娘問他:「如今你跟了師父出家,在佛前與他剃頭摩頂受記。」可憐月娘扯住慟哭了一場,乾生受養了他一場。到十五歲,指望承家嗣。不想被這個老師幻化去了!吳二舅、小玉、玳安,亦悲不勝。當下這普靜老師領了孝哥兒,起了他一個法名,喚做明悟,作辭月娘而去。臨行分付月娘:「你們不消往前途去了。如今不久番兵退去,南北分為兩朝,中原已有個皇帝。多不上十日。兵戈退散,地方寧靜了,你每還回家去,安心度日。」月娘便道:「師父,你度托了孩兒去了,甚年何日,我母子再得見面?」不覺扯住,放聲大哭起來。老師便道:「娘子休哭兒的,那邊又有一位老師來了!」哄的眾人扭頸回頭,當下化陣清風不見了。正是:

「三降塵寰人不識,  倏然飛過岱東峰!」

不說普靜老師幻化孝哥兒去了。且說吳月娘與吳二舅眾人,在永福寺住了那到十日光景,果然大金國立了張邦昌在東京稱帝,置文武百官。徽宗、欽完兩君北去。康王泥馬度江,在建康即位,是為高宗皇帝。拜宗澤為大將,復取山東、河北,分為兩朝。天下太平,人民復業。後月娘歸家,開了門戶,家產器物,都不曾疏失。後就把玳安改名做西門安,承受家業,人稱呼為西門小員外,養活月娘到老,壽年七十歲,善終而亡。此皆平日好善看經之報也,有詩為證:

「閑閱遺書思惘然,  誰知天道有循環,

西門豪橫難存嗣,  經濟顛狂定被殲;

樓月善良終有壽,  瓶梅淫佚早歸泉,

可怪金蓮遭惡報,  遺臭千年作話傳!」

金瓶梅詞話卷之一百回(終)

