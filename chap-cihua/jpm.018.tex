%# -*- coding: utf-8 -*-
% !TeX encoding = UTF-8 Unicode
% !TeX spellcheck = en_US
% !TeX TS-program = xelatex
%~ \XeTeXinputencoding "UTF-8"
% vim:ts=4:sw=4
%
% 以上設定默認使用 XeLaTex 編譯,並指定 Unicode 編碼,供 TeXShop 自動識別

%第十八回
\chapter{來保上東京幹事\KG 陳經濟花園管工}

\begin{showcontents}{}



「堪嘆人生毒似蛇,  誰知天眼轉如車,

去年妄取東鄰物,  今日還歸北舍家;

無義錢財湯潑雪,  倘來田地水推沙,

若將奸狡為活計,  恰似朝雲與暮霞。」

話分兩頭。不說蔣竹山在李瓶兒家招贅。單表來保來旺二人上東京打點,朝登紫陌,暮踐紅塵,饑餐湯飲,帶月披星。有日到東京進了萬壽城門,投旅店安歇。到次日街前打聽,只聽見過路人風裡言風裡語,多交頭接耳,街談巷議。都說兵部王尚書昨日會問明白,聖旨下來,秋後處決。止有楊督名下親屬人等,未曾拿完,尚未定奪。且待今日,便有次第。這來保等二人,把禮物打在身邊,急來到蔡府門首。舊時幹事,來了兩遍,道路久熟。立在龍德街牌樓底下,探聽府中消息。少頃,只見一個青衣人,慌慌打太師府中出來,往東去了。來保認的是楊提督府裡,親隨楊幹辦,待要叫住,問他一聲事情何如,說家主不曾吩咐招惹他,以此不言語,放過了他去了。遲了半日,兩個走到府門前望著守門官深深唱了個喏:「動問一聲,太師老爺在家不在?」那守門官道:「老爺不在家了,朝中議事未回,你問怎的?」來保又問道:「管家翟爺請出來,小人見見,有事稟白。」那官吏:「管家翟叔也不在了,跟出老爺去了。」來保道:「且住!他不實說與我,已定問我要些東西。」於是袖中取出一兩銀子遞與他。那官吏接了,便問:「你要見老爺?要見學士大爺?老爺便是大管家翟謙稟,大爺的事便是小管家高安稟,各有所掌。況老爺朝中未回,止有學士大爺在家,你有甚事?我替你請出高管家來,有甚事引你稟見大爺,也是一般。」這來保就借情道:「我是提督楊爺府中,有事稟見。」官吏聽了,不敢怠慢,進入府中,良久,只見高安出來。來保慌忙施禮,遞上十兩銀子,說道:「小人是楊爺的親,同楊幹辦一路來見老爺討信。因後邊吃飯來遲了一步,不想他先來見了,所以不曾趕上。」高安接了禮物,說道:「楊幹辦只剛纔去了,老爺還未散朝。你且待待,我引你再見見大爺罷。」一面把來保領到第二層大廳傍邊,另一座儀門進去。坐北朝南,三間敞廳,綠油欄杆,朱紅牌額,召青填地,金字大書,天子御筆欽賜「學士琴堂」四字。原來蔡京兒子蔡攸也是寵臣,見為祥和殿學士,兼禮部尚書,提點太一宮使。來保在門外伺候,高安先入說了,出來然後喚來保入見,當廳跪下。廳上垂著朱簾,蔡攸深衣軟巾,坐於堂上,問道:「是那裡來的?」來保稟道:「小人是楊爺的親家陳洪的家人,同府中楊幹辦來稟見老爺討信。不想楊幹辦先來見了,小人趕來後見。」因向懷中取出揭帖,遞上。蔡攸見上面寫著白米五百石。叫來保近前說道:「蔡老爺亦因言官論列,連日迴避。閣中之事,并昨日二法司會問,都是右相李爺秉筆。稱楊老爺的事,昨日內裡消息出來, 聖上寬恩,另有處分了。其手下用事有名人犯,待查明問罪。你還往到李爺那裡說去。」來保只顧磕頭道:「小的不認的李爺府中,望爺憐憫俯就,看家楊老爺分上。」蔡攸道:「你去到天漢橋迤北高坡大門樓處,問聲當朝右相,資政殿大學士,兼禮部尚書,名諱邦彥的你李爺,誰是不知道。也罷,我這裡還差個人同你去。」即令祇候官呈過一緘,使了圖書,就差管家高安同去見李老爺,如此這般替他說。那高安承應下了,同來保出了府門,叫了來旺,帶著禮物,轉過龍德街,逕到天漢橋李邦彥門首。正值邦彥朝散纔來家,穿大紅縐紗袍,腰繫玉帶,送出一位公卿,上轎而去。回到廳上,門吏稟報說:「學士蔡大爺差管家來見。」先叫高安進去,說了回話。然後喚來保來旺進見,跪在廳臺下。高安就在傍邊遞了蔡攸封緘,并禮物揭帖。來保下邊就把禮物呈上。邦彥看了說道:「你蔡大爺分上,又是你楊老爺親,我怎麼好受此禮物?況你楊爺昨日聖心回動已沒事。但只是手下之人,科道參語甚重,已定問發幾個。」即令堂候官取過昨日科中送的那幾個名字與他瞧,上寫著:「王黼名下書辦官董昇,家人王廉,斑頭黃玉;楊戩名下。壞事書辦官盧虎,幹辦楊盛府,椽韓宗仁、趙弘道,斑頭劉成,親黨陳洪、西門慶、胡四等,皆鷹犬之徒,狐假虎威之輩。揆置本官,倚勢害人。貪殘無比,積弊如山。小民蹙額,巿肆為之騷然!乞敕下法司,將一干人犯,或投之荒裔,以禦魑魅;或寘之典刑,以正國法。不可一日使之留于世也!」來保等見了,慌的只顧磕頭,告道:「小人就是西門慶家人,望老爺開天地之心,超生性命則個!」高安又替他跪禀一次。邦彥見五百兩金銀,只買一個名字,如何不做分上?即令左右檯書案過來,取筆將文卷上西門慶名字改作賈慶。一面收上禮物去。邦彥打發來保等出來,就拿回帖回蔡學士,賞高安、來保、來旺一封五十兩銀子。來保路上作辭高管家,回到客店,收拾行李,還了店錢,星夜回到清河縣來。早到家見西門慶,把東京所幹的事,從頭說了一遍。西門慶聽了,如提在冷水盆內,對月娘說:「早時使人去打點,不然怎了!」正是:這回西門慶性命,有如落日已沈西嶺外,卻被扶桑喚出來。於是一塊石頭,方纔落地。過了兩日,門也不關了,花園照舊還蓋,漸漸出來街上走動一日,玳安騎馬打獅子街所過,看見李瓶兒門首開個大生藥舖,裡邊堆著許多生熟藥材。朱紅小櫃,油漆牌面,吊看幌子,甚是熱鬧。歸來告與西門慶說:還不知招贅竹山一節。只說:「二娘搭了個新夥計,開了個生藥舖。」西門慶聽了,半信不信。一日,七月中旬時分,金風淅淅,玉露冷冷。西門慶正騎馬街上走著,撞見應伯爵、謝希大兩人叫住,下馬唱喏。問道:「哥一向怎的不見?兄弟到府上幾遍,見大門關著,又不敢叫,整悶了這幾日。端的哥在家做甚事?嫂子取過來不曾?也不請兄弟們吃酒?」西門慶道:「不好告訴的。因舍親家陳宅那邊為些閒事,替他亂了幾日。親事另改了日期了。」伯爵道:「兄弟每不知哥吃驚。今日既撞遇哥,兄弟二人肯空放了?如今請哥同到裡邊吳銀姐那裡吃三杯,權當解悶。」不由分說,把西門慶拉進院中來。玳安、平安牽馬,後邊跟著走。正是:

「歸去只愁紅日短,  思卿猶恨馬行遲:

世財紅粉歌樓酒,  誰為三般事不迷。」

當日西門慶被他二人拉到吳銀兒家,吃了一日酒。到日暮時分,已帶半酣,纔放出來,打馬正望家走,到於東街口上,撞見馮媽媽從南來,走得甚慌。西門慶勒住馬,問道:「你往那去?」馮媽媽道:「二娘使我往門外寺裡魚籃會,替過世二爹燒箱庫去來。趕進門來。」西門慶醉中道:「你二娘在家好麼?我明日和他說話去。」馮媽媽道:「尤得大人還問甚麼好也來?把個見見成成做熟了飯的親事兒,吃人掇鍋兒去了。」西門慶聽了,失驚問道:「莫不他嫁人去了?」馮媽媽道:「二娘那等使老身送過頭面,往你家去了幾遍不見你,大門關著。對大官兒說進去,教你早動身,你不理。今教別人成了,你還說甚的?」西門慶問:「是誰?」馮媽媽悉把半夜三更,婦人被狐狸纏著,染病著,看看至死。怎的請了大街上住的蔣竹山來看,吃了他的藥,怎的好了。某日怎的倒踏門招進來,成其夫婦。見今二娘拿出三百兩銀子,與他開了生藥舖。從頭至尾說了一遍。這西門慶不聽便罷,聽了氣的在馬上只是跌腳。叫道:「苦哉!你嫁別人,我也不惱。如何嫁那矮王八!他有甚麼起解?」于是一直打馬來家。剛下馬進儀門,只見吳月娘、孟玉樓、潘金蓮并西門大姐四個在前廳天井內,月下跳馬索兒耍子。見西門慶來家,月娘、玉樓、大姐三個都往後走了。只有金蓮不去,且扶著庭柱兜鞋。被西門慶帶酒罵道:「淫婦們間的聲喚,平白跳甚麼百索兒?」趕上金蓮踢了兩腳。走到後邊,也不往月娘房中去脫衣裳,走在西廂稍間一間書房,要了舖蓋,那裡宿歇。打丫頭,罵小廝,只是沒好氣。眾婦人站在一處,都是著恐,不知是那緣故?吳月娘甚是埋怨金蓮:「你見他進門有酒了。兩三步扠在一邊便了,還只顧在眼前笑成一塊且提鞋兒,卻被他蝗蟲螞蚱一例都罵著!」王樓道:「罵我每也罷,如何連大姐也罵起淫婦來了?沒槽道的行貨子!」金蓮接過來道:「這一家子只我是好欺負的!一般三個人在這裡,只踢我一個兒。那個偏受用著甚麼也怎的?」月娘就惱了,說道:「你頭裡何不教他連我也踢不是?你沒偏受用,誰偏受用恁的?賊不識高低貨!我到不言語,你只顧嘴頭子〈石畢〉哩礡喇的!」那金蓮見月娘惱了,便轉把話兒來摭,說道:「姐姐不是這等說。他不知那裡因著甚麼由頭兒,只拿我煞氣。要便睜著眼望著我叫,千也要打個臭死,萬也要打個臭死!」月娘道:「誰教你只要嘲他來?他不打你,卻打狗不成?」玉樓道:「大姐姐,且叫了小廝來問他聲,今日在誰家吃酒來?早辰好好出去,如何來家恁個腔兒?」不一時把玳安叫到根前,問他端的。月娘罵道:「賊囚根子!你不實說,教大小廝來吊拷你和平安兒每人都是十板子。」玳安道:「娘休打,待小的實說了罷。爹今日和應二叔每都在院裡吳家吃酒。散的早了來,在東街口上,撞遇馮媽媽,說花二娘等爹不去,嫁了大街住的蔣太醫了。爹一路上惱的要不的。」月娘道:「信那沒廉恥的歪淫婦;浪著嫁了漢子,來家拿人煞氣!」玳安道:「二娘沒嫁蔣太醫,把他倒踏門招進去了,如今二娘與了他本錢,開了好不興的大藥舖。我來家告爹說,爹還不信。」孟玉樓道:「論起來,男子漢死了多少時兒,服也還未滿,就嫁人,使不得的!」月娘道:「如今年程,論的甚麼使的使不的?漢子孝服未滿,浪著嫁人的,纔一個兒?淫婦成日和漢子酒裡眠,酒裡臥底人,他原守的甚麼貞節?」看官聽說:月娘這一句話,一棒打著兩個人。孟玉樓與潘金蓮都是再醮嫁人,孝服都不曾滿。聽了此言,未免各人懷著慚愧歸房,不在話下。正是:

「不如意處常八九,  可與人言無二三。」

卻說西門慶當晚在前邊廂房睡了一夜。到次日,把女婿陳經濟安他在花園中同賁四管工記帳。換下來昭來,教他看守大門。西門大姐白日裡便在後邊和月娘眾人一處吃酒,晚夕歸前邊廂房中歇。陳經濟每日只在花園中管,非呼喚不敢進入中堂。飲食都是小廝內裡拿出來吃。所以西門慶手下這幾房婦女,都不曾見面。一日西門慶不在家,與提刑所賀千戶送行去了。月娘因陳經濟搬來居住,一向管工辛苦,不曾安排一頓飯兒酬勞他酬勞,向孟玉樓、李嬌兒說道:「待要管,又說我多攬事。我待欲不管,又看不上。人家的孩兒在你家,每日起早睡晚,辛辛苦苦,替你家打勤勞兒,那個興心知慰他一知慰兒也怎的?」玉樓道;「姐姐,你是個當家的人,你不上心誰上心?」月娘於是吩咐廚下,安排了一桌酒餚點心,午間請經濟進來吃一頓飯。這陳經濟撇了工程,教賁四看管,逕到後邊參見月娘。作畢揖,旁邊坐下。小玉拿茶來吃了,安放桌兒,拿蔬菜案酒上來。月娘道:「姐夫每日管工辛苦;要請姐夫進來坐坐,白不得個閒。今日你爹不在家,無事,治了一杯水酒,權與姐夫酬勞。」經濟道:「兒子蒙爹娘抬舉,有甚勞苦?這等費心!」月娘遞了酒,經濟傍邊坐下。須臾,饌餚齊上。月娘陪著他吃了一回酒。月娘使小玉:「請大姑娘來這裡坐。」小玉道:「大姑娘使看手,便來。」少頃,只聽房中抹的牌响。經濟便問:「誰人抹牌?」月娘道:「是大姐與玉簫丫頭弄牌。」經濟道:「你看沒分曉。娘這裡呼喚不來,且在房中抹牌。」不一時大姐掀簾子出來,與他女婿對面坐下,一同飲酒。月娘便問:「陳姐夫也會看牌也不會?」大姐道:「他也知道些香臭兒。」當時月娘自知經濟是個志誠的女婿,卻不道是小夥子兒,詩詞歌賦、雙陸象棋,折牌道字,無所不通,無所不曉。有西江月為證:

「自幼乖滑伶俐,風流博浪牢成。愛穿鴨綠出爐銀,雙陸象棋幫襯。琵琶笙竺簫管,彈丸走馬員情。只有一件不堪聞,見了佳人是命。」

月娘便道:「既是姐夫會看牌,何不進去咱同看一看?」經濟道:「娘和大姐看罷,兒子卻不當。」月娘道:「姐夫至親間,怕怎的?」一面進入房中。只見孟玉樓正在牀上鋪茜紅毡看牌。見經濟進來,抽身就要走。月娘道:「姐夫又不是別人,見個禮兒罷。」向經濟道:「這是你三娘哩。」那經濟慌忙躬身作揖,玉樓還了萬福。當下玉樓、大姐三人同抹,經濟在傍邊觀看。抹了一回,大姐輸了下來。經濟上來又抹,玉樓出了個天地分,經濟出了恨點不到頭。吳月娘出了個四紅沉,八不就,雙三不搭兩么兒,和兒不出;左來右去配不著色頭。只見潘金蓮掀開簾子走進來,銀絲䯼髻上戴著一頭鮮花兒仙掌,體可玉貌,笑嘻嘻道:「我說是誰,原來是陳姐夫在這裡。」慌的陳經濟扭頸回頭,猛然一見,不覺心蕩目搖,精魂已失。正是:

「五百年冤家今朝相遇,  三十年恩愛一日遭逢。」

月娘道:「此是五娘。姐夫也只見個長禮兒罷。」經濟忙向前深深作揖,金蓮一面還了萬福。月娘便道:「五姐你來看,小雛兒倒把老鴉子來贏了。」這金蓮近前一手扶著牀護炕兒,一隻手拈著白紗團扇兒,在傍替月娘指點說道:「大姐姐,這牌不是這等出了。把雙三搭過來,卻不是天不同和牌,還贏了陳姐夫和三姐姐。」眾人正抹牌在熱鬧處,只見玳安抱進毡包來,說:「爹來家了。」月娘連忙攛掇小玉送陳姐夫打角門出去了。西門慶下馬進門,先到前邊工上觀看了一遍,然後踅到潘金蓮房中。金蓮慌忙接著,與他脫了衣裳,說道:「你今日送行去來的早。」西門慶道:「提刑所賀千戶新陞、新平寨知寨,合衛所相知都郊外送他,來拿帖兒來會我,不好不去的。」金蓮道:「你沒酒,教丫鬟看酒來你吃。」不一時放了桌兒飲酒,菜蔬都擺在面前。飲酒中間,因說起後日花園捲棚上梁,約有許多親朋都要來遞菓盒酒掛紅,少不得叫廚子置酒管待。說了一回,天色已晚。春梅掌燈歸房,二人上牀宿歇。西門慶因起早送行,著了辛苦,吃了幾杯酒就醉了。倒下頭鼾睡如雷,齁齁不醒。那時正值七月二十頭天氣,夜裡有些餘熱,這潘金蓮怎生睡得著。忽聽碧紗帳內一派蚊雷,不免赤著身子起身來,執著燭滿帳照蚊。照一個燒一個。回首見西門仰臥枕上,睡得正濃,搖之不醒。其腰間那話,帶著托子,纍垂偉長,不覺淫心輙起。放下燭臺,用纖手捫弄弄了一回,蹲下身去,用口吮之,吮來吮去,西門慶醒了。罵道:「怪小淫婦兒!你達達睡睡,就摑混死了。」一面起來,坐在枕上,亦發叫他在下儘著吮咂;又垂首翫之,以暢其美。正是:

「怪底佳人風性重,  夜深偷弄紫鸞蕭。」

有蚊子雙關,踏莎行詞為證:

「我愛他身體輕盈,楚腰膩細,行行一泒笙歌沸。黃昏人未掩朱扉,潛身撞入紗廚內。款傍香肌,輕憐玉體,嘴到處臙脂記。取邊廂,告就百般聲,夜深不肯教人睡。」

婦人于是頑了有一頓飯時,西門慶忽然想起一件事來,叫香梅篩酒過來,在牀前執壺而立。將燭移在牀背板上,教婦人馬爬在他面前,那話隔山取火,托入牝中,令其自動,在上飲酒取其快樂。婦人罵道:「好個刁鑽的強盜!從幾時新興出來的例兒,怪剌剌教丫頭看答著甚麼張致!」西門慶道:「我對你說了罷。當初你瓶姨和我常如此幹,叫他家迎春在傍執壺斟酒,到好耍子。」婦人道:「我不好罵出來的,甚麼瓶姨鳥姨!題那淫婦則甚?奴好心不得好報。那淫婦等不的,浪著嫁漢子去了。你前日吃了酒,你來家,一般三個人在院子裡跳百索兒,只拿我煞氣,只踢我一個兒,倒惹的人和我辨了回子嘴。想起來,奴是好欺負的!」西門慶問道:「你與誰辨嘴來?」婦人道:「那日你便進來了,上房的好不和我合氣。說我在他根前頂嘴來,罵我不識高低的貨。我想起來,為甚麼養蝦蟇得水蠱兒病,如今到教人惱我?」西門慶道:「不是我也不惱,那日應二哥他們拉我到吳銀兒家吃了酒出來,路上撞見馮媽媽子,如此這般告訴我,把我氣了個立睜。若嫁了別人,我到罷了。那蔣太醫賊矮王八,那花大怎不咬下他下截來?他有甚麼起解?招他進去,與他本錢,教他在我眼面前開舖子,大剌剌做買賣?」婦人道:「虧你有臉兒還說哩!奴當初怎麼說來?先下米的先吃飯。你不聽,只顧求他問姐姐。常信人調,丟了瓢!你做差了,你抱怨那個?」西門慶被婦人這幾句話,沖得心頭一點火起,雲山半壁通紅。便道:「你由他,教那不賢良的淫婦說去,到明日休想我這裡理他。」看官聽說:自古讒言罔行,雖君臣父子夫婦昆弟之間,猶不能免,況朋友乎?饒吳月娘恁般賢淑的婦人,居于正室,西門慶聽金蓮袵席脾睨之閒言,卒致于反目。其他可不慎哉!自是以後,西門慶與月娘尚氣,彼此觀面,都不說話。月娘隨他往那房裡去,也不管他來遲去早,也不問他。或是他進房中取東取西,只教丫頭上前答應,也不理他,兩個都把心來冷淡了。正是:

「前車倒了千千輛,  後車到了亦如然;

分明指與平川路,  錯把忠言當惡言。」

且說潘金蓮自西門慶與月姐尚氣之後,見漢子偏聽已,于是以為得志。每日抖搜著精神,粧飾打扮,希寵巿愛。因為那日後邊會遇陳經濟一遍,見小夥兒生的乖猾伶俐,有心也要抅搭他。但只畏悮西門慶,不敢下手。只等的西門慶往那裡去,不在家,便使了丫鬟叫進房中,與他茶水吃,常時兩個下棋做一處。一日,西門慶新蓋捲棚上梁,親友掛紅慶賀,遞菓盒的也有許多。各作人匠,都有犒勞賞賜。大廳上管待官客,吃到晌午時分人纔散了。西門慶看著收拾了家火,歸後邊睡去了。陳經濟走來金蓮房中,討茶吃。金蓮正在牀上彈弄琵琶道:「前邊上梁吃了恁半日酒,你就不曾吃了些甚麼?還來我屋裡要茶吃。」經濟道:「兒子不瞞你老人家說,從半夜起來,亂了這一五更,誰吃甚麼來?」婦人問道:「你爹在那裡?」經濟道:「爹後邊睡去了。」婦人道:「你既沒吃甚麼,叫春梅揀粧裡,拿我吃的那蒸酥菓餡餅兒來,與你姐夫吃。」這小夥兒就在他炕桌兒擺著四碟小菜,吃著點心。因見婦人彈琵琶,戲問道:「五娘,你彈的甚曲兒?怎不唱個兒我聽。」婦人笑道:「好陳姐夫,奴又不是你影射的,如何唱曲兒你聽?我等你爹起來,看我對你爹說不說。」那經濟笑嘻嘻,慌忙跪下,央及道:「望乞五娘可憐見,兒子再不敢了。」

那婦人笑起來了。自此這小夥兒,和這婦人日近日親。或吃茶吃飯,穿房入屋,打牙犯嘴,挨肩擦膀,通不忌憚。月娘托以兒輩,放這樣不老實的女婿在家,自家的事卻看不見。正是:

「只繞採花成釀蜜,  不知辛苦為誰甜!」

「堪嘆西門慮未通,  惹將桃李笑春風,

滿牀錦被藏賊睡,  三頓珍羞養大蟲;

愛物只圖夫婦好,  貪財常把丈人坑,

還有一件堪誇事,  穿房入屋弄乾坤。」

畢竟未知後來何如,且聽下回分解:




\end{showcontents}
