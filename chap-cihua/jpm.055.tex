%# -*- coding: utf-8 -*-
%!TEX encoding = UTF-8 Unicode
%!TEX TS-program = xelatex
% vim:ts=4:sw=4
%
% 以上设定默认使用 XeLaTex 编译,并指定 Unicode 编码,供 TeXShop 自动识别

%第五十五回 
\chapter{西門慶東京慶壽旦\KG 苗員外揚州送歌童}

「千歲蟠桃帶露携,  携來黃閣祝期頤,

八仙下降稱觴日,  七鳳團花織錦時;

六合五溪輸賀軸,  四夷三島獻珍奇,

羲和莫遣兩丸速,  願壽中朝帝者師。」

卻說任醫官看了脉息,依舊到廳上坐下。西門慶便開言道:「不知這病症看得何如?沒的甚事麼?」任醫官道:「夫人這的病,原是產後不慎調理,因此得來。目下惡路不淨,面帶黃色,飲食也沒些要緊,走動便覺煩勞。依學生愚見,還該謹慎保重。大凡婦人產後,小兒痘後,最難調理。略有些差池,便種了病根。如今夫人兩手脉息,虛而不實。按之散大,卻又軟不能自固。這病症都只為火炎肝腑,土虛木旺,虛血妄行。若今番不治,他後邊一發了不的了。」說畢。西門慶道:「如今該用甚藥纔好?」任醫官道:「只是用些清火止血的藥。黃栢知毋為君,其餘只是地黃、黃岑之類,再加減些吃下看住,就好了。」西門慶聽了,就叫書童封了一兩銀子,送任一官做藥本。任一官作謝去了。不一時,送將藥來。李瓶兒屋里煎服,不在話下。且說西門慶送了任醫官去,回來與應伯爵坐地。想起東京蔡太師壽旦已近,先期曾差玳安往杭州買辦龍袍錦繡,金花寶貝上壽禮物,俱已完備,即日要自往東京拜賀,算來日期已近,自山東來到東京,也有半個月日路程,連夜收拾行李進發,剛剛正好,再遲不的了。便進房來和月娘說知,如此這般。月娘道:「這咱時不說,如今忙匆匆的,你擇定幾時起身?」西門慶道:「明日起身也纔彀到哩,還得幾個日頭。」西門慶說畢,就走出外來,分付玳安、書童、畫童,打點衣服行李,明日跟隨東京走一遭。四個小廝,各各收拾行李不說。月娘便教小玉:「去請你各房娘,都來收拾你爹行李。」當下只有李瓶兒,一來有了孩子,二來服了藥,不出房來,其餘各房孟玉樓、潘金蓮一齊都到,走來的,多動手把皮廂、涼廂,裝了蟒衣、龍袍、段匹、上壽等物,共有二十多扛,又整頓了應用冠帶衣服等件,一齊完了。晚夕,三位娘子擺設酒餚和西門慶送行,席上西門慶各人叮囑了幾句,自進月娘房裡宿歇。次日把二十扛行李,先打發出門。又發了一張通行馬牌,仰經過驛遞,起夫馬迎送。各各停當,然后進李瓶兒房裡來,看了官哥兒,與李瓶兒說了句話,教他好好調理,我不久便來家看你。」那李瓶兒閣著淚道:「路上小心保重。」直送出廳來,和月娘、玉樓、金蓮打夥兒送出了大門。西門慶乘了涼轎,四個小廝騎了頭口,望東京進發。迤〈辶里〉行來,卻走了百里路程。那時日已傍晚,西門慶分付駐劄。驛官廝見送供應,過了一宵。明日天早,西門慶催促人馬,扛箱快行,一路看了些山明水秀。午牌時,打中火又行。路上相遇的,無非各路文武官員,進京慶賀壽旦的,也有進生辰摃的,不計其數。又行了十來日,算前途路已不多,趲到剛剛湊巧。宿了一晚,又行勾兩日,早到東京,進了萬壽城門。那時天色將晚,趕到龍德街牌樓底下,就投翟家屋裏住歇。那翟管家聞知西門慶到了,忙的出來迎接,各敘寒暄,吃了茶。西門慶叫玳安專管行李,一一交盤進了翟家裏來。翟謙交府幹收了,就擺酒和西門慶洗塵。不一時,只見剔犀官卓上列著幾十樣大菜,幾十樣小菜,都是珍羞美味,燕窩 、魚刺 絕好下飯,只沒有龍肝、鳳髓 ;其餘奇巧富麗,便是蔡太師自家受用,也不過如此。當直的拿著通天犀杯,斟上蔴姑酒兒,遞與翟謙。接過滴了天,然後又斟上來,把盞與西門慶,西門慶也回敬了。兩人坐下,糖果、熱楪、按酒之物,流水也似遞將上來。酒過兩巡,西門慶便對翟謙道:「學生此來,單為老太師慶壽,聊備些微禮,孝順太師,想不見卻。只是學生向有相攀的心,欲求親家預先稟過,但拜太師門下做個乾生子,也不枉了一生一世。不知可以啟口帶攜的學生麼?」翟謙道:「這個有何難哉?我們主人雖是朝廷大臣,卻也極好奉承,今日見了這般盛禮,自然還要陞選官爵,不惟拜做乾子,定然允哩!」西門慶聽說,不勝之喜。飲殼多時,西門慶便推不吃酒罷。翟管家道:「再請一杯,怎的不吃了?」西門慶道:「明日有正經事,卻不敢多飲。」再四相勸,只得又吃了一杯。翟管家賞了隨從人酒食,分付叫把牲口牽到後糟去。當下收過了家活,就請西門慶到後邊書房裏安歇。排下好描金暖床,絞綃帳兒。把銀鈎掛起,露出一床好錦被,香噴噴的。一班小廝扶侍西門慶脫衣脫襪,上床獨宿孤眠,西門慶一生不慣,那一晚好難捱過也。巴到天明,正待起身,那翟家門戶重掩著,那裡討水來淨臉?直挨到巳牌時分,纔有個人把匙鑰一路開將出來。隨后一個小廝拿著手巾,一個捧著銀面盆,傾了香湯,進書房來。西門慶梳洗完畢,戴上忠靖冠,穿著外蓋衣服,一個在書房裡坐。只見翟管家出來,和西門慶廝見了坐下。當直的托出一個朱紅合子,裡邊有三十來樣美味。一把銀壺,斟上酒來,吃早飯。翟謙道:「請用過早飯,學生先進府去和主翁說過,然后親家搬禮物進來。」西門慶道:「多勞費心。」酒過數杯,就拏早飯來吃了,收過家活。翟管家道:「且權坐一回,學生進府去便來。」翟家去不多時,忙跑來家向西門慶說:「老爺正在書房梳洗,外邊滿朝文武官員,都各伺候拜壽,未得廝見哩。學生已對老爺說過了,如今先進去拜賀,省的泯雜,學生也隨後便到了。」西門慶不勝歡喜,便教跟隨人拉同翟家幾個伴當,先把那二十扛金銀段疋,抬到太師府前,一行人應聲去了。西門慶冠帶,乘了轎來,只見亂哄哄的挨肩擦背,都是大小官員來上壽的。西門慶遠遠望見一個官員,也乘著轎進龍德坊來。西門慶仔細一認,倒是楊州苗員外。卻不想苗員外也望見西門慶了。兩個同下轎作揖,敘來寒溫。原來這苗員外是第一個財主,他身上也現做個散官之職。向來結交在蔡太師門下,那時也來上壽,恰遇了故人。當下兩個忙匆匆路次話了幾句,分手而別。西門慶來到太師府前,但見:

「堂開綠野,彷彿雲霄;閣起凌煙,依稀星斗。門前寬綽堪旋馬,閥閱嵬峨好豎。錦綉叢中,風送到畫眉聲巧,金銀惟裏,日映出琪樹花香。旃檀香,截成梁棟;醒酒石,滿砌階除。左右玉屏風,一個個九光紅拂;滿堂羅寶玩,一件件周鼎商彝。明晃晃懸掛著明珠十二,黑夜裡何用燈油;貌堂堂招致得珠履三千,彈短鋏盡皆名士。恁地九州四海,大小官員,多來慶賀;就是六部尚書,三邊總督,無不低頭。」正是:

「除卻萬年天子貴,  只有當朝宰相尊。」

西門慶恭身進了大門,只見中門關著不開,官員都打從角門而入。西門慶便問:「為何今日大事,卻不開大門?」翟管家道:「原來中門曾經官家行幸,因此人不敢打這門出入。」西門慶和翟管家進了幾重門,門上都是武官把守,一些兒也不混亂。見了翟謙,一個個都欠身問管家:「從何處來?」翟管家答道:「舍親打山東來拜壽老爺的。」說罷,又走過幾座門,轉幾個彎,無非是畫棟雕梁,金張甲第。隱隱聽見鼓樂之聲,如在天上的一般。西門慶又問道:「這裡民居隔絕,那裡來的鼓樂喧嚷。」翟管家道:「這是老爺教的女樂,一班共二十四人,也曉得天魔舞、霓裳舞、觀音舞,凡老爺早膳、中飯、夜燕,都是奏的。如今想是早膳了。西門慶聽言未了,又鼻子裏覺得異香馥馥,樂聲一發近了。翟管家道:「這裡老爺書房將到了,腳步兒放鬆些。」轉個迴廊,只見一座大廳如寶殿仙宮,廳前仙鶴孔雀,種種珍禽,又有那瓊花、曇花、佛桑花,四時不謝,開的閃閃爍爍,應接不暇。西門慶還未敢闖進,交翟管家先進去了,然后挨挨排排,走到堂前。堂上虎皮太師交椅上,坐一個大猩紅蟒衣的,是太師了。屏風後列有四三十個美女,一個個都宮樣粧束,執巾執扇,捧擁著他。翟管家也站在一邊。西門慶朝上拜了四拜,蔡太師也起身就羢單上回了個禮,這是初相見了。落后翟管家走近蔡太師耳邊,暗暗說了幾句話下來。西門慶理會的是那話了,又朝上拜四拜。蔡太師便不答禮,這四拜是認乾爺了,因受了四拜。后來都以父子相稱。西門慶開言道:「孩兒沒恁孝順爺爺。今日華誕,家裡備的幾件菲儀,聊表千里鵝毛之意。願老爺壽比南山。」蔡太師道:「這怎的生受!」便請坐下,當直的拏了把椅子上來,西門慶朝上作了個揖道:「告坐了。」就西邊坐地吃茶。翟管家慌跑出門來叫:「抬禮物的都進來。」二十來扛禮物,揭開了涼箱蓋,呈上一個禮目:大紅蟒袍一套、官綠龍袍一套、漢錦二十疋、蜀錦二十疋、火浣布二十疋、西洋布二十疋、其余花素尺頭共四十疋,獅蠻玉帶一圍,金鑲奇南香帶一圍,玉杯犀杯各十對,赤金攢花爵杯八隻,明珠十顆;又梯已黃金二百兩,送上蔡太師做贄見的禮。蔡太師看了禮目,又瞧了抬上二十來扛,心下十分懽喜,連聲稱多謝不迭。便教翟管家:「收進庫房去罷。」一面分付擺酒款待。西門慶因見忙沖沖,推事故辭別了蔡太師。太師道:「既如此,下午早早來罷。」西門慶作個揖起身,蔡太師送了幾步,便不送了。西門慶依舊和翟管家同出府來。翟管家府內有事,也作別進去。西門慶竟回到翟家來,脫下冠帶,又整的好飯吃了一頓。回到書房,打了個磕睡,恰好蔡太師差舍人邀請赴席。西門慶謝了些扇金,著先去,隨后就來了。便重整冠帶,預先叫玳安封下許多賞封,做一拜匣盛了,跟隨著四個小廝,乘轎望太師府來,不題。且說蔡太師那日滿朝文武官員來慶賀的,各各請酒。自次日為始,分做三停,第一是皇親內相,第二日是尚書顯要、衙門官員,第三日是內外大小等職。只有西門慶一來遠客,二來送了許多禮物,蔡太師到十分歡喜他。因此就是正日,獨獨請他一個。見說請到了新乾子西門慶,忙走出軒下相迎。西門慶再四謙遜,讓爺爺先行。自家屈著背,輕輕跨入檻內。蔡太師道:「遠勞駕從,又損隆儀,今日略坐,少表微忱。」西門慶道:「孩兒戴天履地,全賴爺爺洪福。些小敬意,何足掛懷?」兩個喁喁笑語,真似父子一般。二十個美女,一齊奏樂。府幹當直的斟上酒來,蔡太師要與西門慶把盞,西門慶力辭不敢,只領的一盞,立飲而盡,隨即坐了筵席。西門慶教書童取過一隻黃金桃杯,斠上了滿滿一杯。走到蔡太師席前,雙膝跪下道:「願爺爺千歲!」蔡太師滿面歡喜道:「孩兒起來。」接過便飲個完。西門慶纔起身,依舊坐下。那時相府華筵,珍奇萬狀,都不必說。西門慶直飲到黃昏時候,拿賞封了諸執役人,纔作謝告別道:「爺爺貴冗,孩兒就此叩謝。后日不敢再來求見了。」出了府門,仍到翟家安歇。次日,要拜苗員外,著玳安跟尋了一日,卻在皇城後李太監房中住下。玳安拏著帖子通報了。苗員外來出迎道:「學生一個兒坐著,正想個知心的朋友講講,恰好來湊巧。」就留西門慶筵燕,西門慶推卻不過,只得便住了。當下山餚海錯,不記其數。又有兩個歌童,生的眉清目秀,開喉音唱幾套曲兒。西門慶指著玳安、琴童、書童、畫童,向苗員外看著:「那班蠢材,只顧吃酒飯,卻怎地比的那兩個?」苗員外笑道:「只怕伏侍不的。老先生若愛時,就送上也何難。」西門慶謙謝不敢奪人之好。飲到更深,別了苗員外,依舊來翟家歇。那幾日內相府管事的,各各請酒,留連了八九日。西門慶歸心如箭,便叫玳安收拾行李。那翟管家苦死留住,只得又吃了一夕酒,重敘姻親,極其眷戀。次日,早起辭別,望山東而行。一路水宿風餐,不在話下。且說自從西門慶往東京慶壽,姊妹每眼巴巴望西門慶回來,多有懸掛。在屋裡做些針指,通不出來間耍。只有那潘金蓮打扮的如花以玉,嬌模喬樣,在丫環夥裏,或是猜枚,或是抹牌,說也有,笑也有,狂的通沒些成色,嘻嘻哈哈,也不顧人看見,只想著與陳經濟抅搭,便心上亂亂的焦燥起來。多少長吁短嘆,托著腮兒,呆登登本待要等經濟回來,和他做些營生。又不道經濟每日在店裡沒的閒。欲要自家出來尋著他,又有許多丫頭,往來不方便。日裡便似熬盤上蟻子一般,跑進跑出,再不坐在屋裡。那一日正是風和日暖,那金蓮身邊帶著許多麝香、合香,走到捲棚後面,只望著雪洞裡。那經濟日在店裡,那得脫身進來?望了一回不見,只得來到屋裡,把筆在手,吟哦了幾聲,便寫一封書封著,叫春梅逕送與陳姐夫。經濟接著,拆開從頭一看,卻不是書,一個曲兒。經濟看罷,慌的丟了買賣跑到捲棚後面看,只見春梅回房去對潘金蓮說了。不一時也跑到捲棚下,兩個遇著,就如餓眼見瓜皮一般,禁不的一身直鑽到經濟懷裡來,捧著經濟臉,一連親了幾個嘴,咂的舌頭一片聲响,道:「你負心的短命賊囚!自從我和你在屋裡,被小玉撞破了去後,如今一向都不得相會,這幾日你爺爺上東京去了,我一個兒坐炕上,淚汪汪只想著你,你難道耳根兒也不熱的?我仔細想來,你恁地薄情,便去著也索羅休。只到了其間,又丟你不的。常言:『痴心女子負心漢』,只你也全不留些情!」正在熱鬧間,不想那玉樓冷眼瞧破,忽然抬頭看見,順手一推,險些兒經濟跌了一交。慌忙驚散不題。那日吳月娘、孟玉樓、李瓶兒同一處坐地,只見玳安慌慌的跑進門來,見月娘磕了個頭道:「爹回來了。小的一路騎頭口,拏著馬牌先行,因此先到家。爹這時節也差不上二十里遠近了。」月娘道:「你曾吃飯沒有?」玳安道:「從早上吃來,卻不曾吃中飯。」月娘便教玳安廚下吃飯去。又教整飯,待大官人回來,自和六房姊妹同夥兒到廳上迎接。正是:

「詩人老去鶯鶯在,  公子歸時燕燕忙。」

四人閑話多時,卻早西門慶到前下轎了。眾妻妾一齊相迎進去。西門慶先和月娘廝見畢,然后孟玉樓、李瓶兒、潘金蓮依次見了。西門慶和六房妻小,各敘寒溫。落后書童、畫童也來磕了六房的頭,自去廚下吃飯。西門慶把路上辛苦,并到翟家住下,明日蔡太師厚情,與內相日日吃酒事情,備細說了一遍。因問李瓶兒:「孩子這幾時好麼?你身子怎地調理?吃的任醫官藥,有些應驗麼?我雖則往東京,一心只弔不下家事哩!店裡又不知怎樣,因此急忙回來。」李瓶兒道:「孩子也沒甚事,我身子吃藥后,略覺好些。」月娘一面教眾人收好行李及蔡太師送的下程,一面做飯與西門慶吃,到晚又設酒和西門慶接風。西門慶晚就在月娘房裏歇了兩夜,是久旱逢甘雨,他鄉遇故知,懽愛之情,多不必說。次日,陳經濟和大柤來廝見了,說了些店裡的帳目,應伯爵和常時節打聽的大官人來家,都來望西門慶。出門廝見畢,兩個一齊說:「哥哥一路辛苦。」西門慶便把東京富麗的事情,及太師管待情分,備細說了一遍,兩人只顧稱羡不已。當日西門慶留二人吃了一日酒,常時節臨起身,向西門慶道:「小弟有一事相求,不知哥可照顧麼?」說著只是低了臉,半含半吐。西門慶道:「但說不妨。」常時節道:「實為住的房子不方便,待要尋間房子安身,卻沒有銀子,因此要求哥周濟些兒,日后少不的加些利錢,送還哥哥。」西門慶道:「相處中說甚利錢!我如今忙忙地,那討銀子?且待到韓夥計貨船來家,自有個處。」說罷,常時節、應伯爵作謝去了,不在話下。且說苗員外自與西門慶相會在太師府前,便請了一席酒,席上又把兩個歌童許下了。那一日西門慶歸心如箭,卻不曾作別的他,竟自歸來了。員外還道西門慶在京,伴當來翟家問著。那翟家說:「三日前西門大官家去了。」伴當回話,苗員外纔曉的。卻不道君子一言,快馬一鞭。不送去也罷,不和我合著氣,只后邊說不的話了。便叫過兩個歌童,分付道:「我前日請山東西門大官,席上把你兩個許下他。如今他離東京回家去了,我目下就要送你們過去,你們早收拾包裹,待我稍下書打發你們。」那兩個歌童,一齊陪告道:「小的每伏侍的員外多年了,卻為何今日閃的小的們不好。又不知西門大官性格怎地,今日還要員外做主。」員外道:「你們卻不曉的,西門大官家裡,豪富潑天,金銀廣布,身居著右班左職,現在蔡太師門下做個乾兒子。就內相朝官,那個不與他心腹往來?家裡開著兩個綾段舖,如今又要開個標行,近的利錢也委的無數,況兼他性格溫柔,吟風弄月,家裡養個七八十個著頭,那一個不穿綾著襖。後房裡擺著五六房娘子,那一個不插珠挂金,那些小優們戲子們,個個借他錢鈔,服他差使。平康巷、青水巷這些角伎,人人受他恩惠,這也不消說的。只是咱前日酒席之中,已把小的子許下他了。如今終不成改個口哩?」那歌童又說道:「員外這幾年上不知費盡多少心力,教的俺們彈唱哩。如今才曉得些絃索,卻不留下自家歡樂,怎地倒送與別人快話?」說罷,不覺地撲簌簌哩吊下淚來。那員外也覺慘然不樂,說道:「小的子,你也說的是!咱也何苦定要是這等?只是:『人而無信,不知其可也。』那孔聖人說的話,怎麼違得?如今也由不得你。待咱修書一封,差令伴當送你去,教他把隻眼兒好生看覷你們。你到那邊快活,也強似在我這裡一般。」就叫那門管先生寫著一封通候的八行書信,後面又寫那相送歌童,求他青目的語兒。又寫個禮單兒,把些尺頭書帕,做個通問的禮兒。差了苗秀、苗實,齎拏書信,護送兩個歌童,一霎時拴上了頭口,帶了被囊行李,直到山東西門慶家來。那兩個歌童當時忍不住腮邊淚滴,又是主命難違,只得插燭也似磕了幾個頭,謝辭了員外,番身上馬,迤邐行來,見那青山環馬首,綠水繞行鞭,酒帘深樹裡,草舍落霞前。止為那遏行雲歌聲絕代,不覺的辭恩主跋涉風煙,這兩個思鄉念主,把那些檀板風流,陽春白雪兒多忘卻。這兩個忙投急趁,止思量早完公事,披星帶月的夜忘眠。正是:

「朝為苗府清哥客,  暮作西門侑酒人。」

遠遠望見綠樹林中,挂著一個望子。那歌童道:「哥,走了這一日了,肚里有些飢了,且吃盃酒兒去。」只見四個人兒滾鞍下馬,走入店中。那招牌上面寫的好說:「神仙留玉佩,卿相解金貂。」真個是好酒店也!四人坐下,喚顧買打上兩角酒來。攘個葱兒、蒜兒、大賣肉兒、豆腐菜兒 ,舖上幾碟,正待舒懷暢飲。忽地哩回頭看時,止見粉壁上飛白字,寫著兩行說道:

「千里不為遠,  十年歸未遲;

總在乾坤內,  何須嘆別離?」

正對著兩個歌童眼兒,不覺的賣藥有病的了,動人心處,撲簌簌流下兩行淚來,說道:「哥,我們隨著員外,指望一蒂兒到底。誰想酒席中間,一言兩句,竟把我們送與別人。人離鄉賤,未知去後若何?」那苗秀、苗實把好言知慰了一番,吃了飯,上馬又走。四個生口,十六個蹄兒,端的是走的好。不多幾個日頭,就到東平洲清河縣地面。四人拴了生口,下馬訪問端的,一直地竟到紫石街西門慶家府裡投下。卻說那西門慶,自從東京到家,每日忙不迭送禮的請酒的,日日三朋四友。既要與大娘兒接風,又要與各房兒繾綣,朝朝殢雨尤雲。以此不曾到衙門裡去走,連那告駕的帖兒,也不曾消的。那日清閒無事,且到衙門裡升堂畫卯,把那些解到的人犯,也有姦情的、鬭歐的、賭賻的、竊盜的,一一重問一番。又把那些投到文書,一一押到日僉押了一會。乘了一乘涼轎,幾個牢子喝道了,簇擁來家。只見那苗秀、苗實與那兩個歌童已是候的久了,就跟著西門慶的轎子,隨到前廳,雙膝跪下稟說:「小的是楊州苗員外,有書拜候老爺。」磕個頭起在一邊。那西門慶舉個手說著起來,就把苗員外別來的行徑,寒暄的套語,問了一會。就叫書童把銀剪子剪開護封,拆了內函封袋,打開副啟,細細看時,只見那苗秀、苗實依先跪下,奉過那許多禮物說道:「這是俺員外一點孝心,求老爹俯納。」西門慶喜之不勝,連忙叫玳安收起禮物,請起苗秀、苗實,說道:「我與千里相逢,不想就蒙員外情投意合,十分相愛,就把歌童相許,那時酒中說話,咱也忘卻多時。因為那歸的忙促,不曾叩府辭別。正在想著,不意一諾千金,遠蒙員外記憶。我記得那古人交誼,止有那范張結契,千里相從,古今以為美談。如今你們那個員外,委的也是難的!」稱長道好,細細又感謝了一番。只見那兩個歌童通新走過。又磕幾個頭,說道:「員外著小的們伏侍老爺,萬求老爺青目。」西門慶見兩個兒生得清秀,真真嫋嫋媚媚。雖不是兩節穿衣的婦人,卻勝似那唇紅齒白的妮子。歡天喜地,就請四位管家前廳茶飯。一面整辦厚禮,綾羅細軟,修書答謝員外。一面收拾房間,就叫兩個歌童,在于書房伺候著。只見那應伯爵諸人,聞此事知此事,通來探望。西門慶就叫玳安裡邊討出菜蔬、嗄飯、點心、小酒,擺著八仙卓兒,就與諸人燕飲,就叫兩個歌童前來唱,只見捧著擅板,拽起歌,唱一個:

〔新水令〕  「小園昨夜放江梅,另一番動人風味。梨花迎笑臉,楊柳妒腰圍。試問荼{艹縻}開到海棠未?」

〔駐馬聽〕  「野徑疎籬,陣陣香風來燕子;小園幽砌,紛紛晴雨過林西。芳心不與蝶潛知,暗香未許蜂先覺。闌遍倚,不知多少傷心處?」

〔雁兒落帶得勝令〕  「我則見碧陰陰西施鎖翠,紅點點鶗鳺拋珠淚;無仙仙砑光帽帽簪,虛飄飄花谷樓前墜。尚兀是芳氣襲人衣,豔質易沾泥。落處魚驚,飛來蝶欲迷。尋思憑誰寄,還悲花源未可期。」

那西門慶點著頭道:「果然唱得好!」那兩個歌童打個半跪兒,跪將下告道:「小的們還學得些小詞兒,一發歌與老爹聽。」西門慶說道:「這卻更好。」便教歌詞:

「試裂齊紈,施鉛槧,爰圖春牧。草淺淺細舖平野,散騎黃犢。一卷殘書牛背穩,數聲短笛煙光綠。想按圖題詠,賦新詞,勞心曲。  文章妙,傳芸局;音調促,偕絲竹。倚清歌,追和陽春難續。一代風流誇好事,可堪膾炙人爭錄。羨先生想像,賦高唐,情詞足。」

又:

「晝出耕圖,郊原外,東阡西陌。町疃曲群山環翠,岸塍聯絡,綠遍田疇多黍稌,麥纂纂蠶盈箔。彷彿有溪小繞柴門,山如削。  扶藜杖,徑丘壑;穿林藪,聽猿鶴。子耕耘,前妻饁服勞耕作。喬木陰森流憩處,皤然捫腹舒雙腳,羨先生想像詠豳風,村田樂。」

「寫就丹青,新圖好,溪山環繞。隱隱遍沙汀水岸,綠蘋紅蓼。一派秋光連浦淑,短蓑篛笠煙波渺。看此時網得幾鮮鱗,鱸魚小。  漁唱起,飛鴻杳;江月白,歸雲少。倚蓬窗,試覓舊盟鷗鳥。借問忘機當日事,何如此際心情悄。羨先生想像詠滄浪,起塵表。」

又:

「四野雲垂,冰花醉,平舖茅屋。紅爐暖,妻煨山芋。自斟醽醁,課僕採薪外戶。呼兒引鶴翻平陸,攬此景寫入畫圖中,娛心目。  鍾貴富,天之祿;懼盛滿,吾之欲。聘姘奇。攄寫好詞盈軸。愧我倡酬才思澀,輸他文采機關熟。羨先生想像樂桑榆,顏如玉。」

果然是聲遏行雲,歌成白雪,引的那後邊娘子們吳月娘、孟玉樓、潘金蓮、李瓶兒都來聽著,十分歡喜。齊道:「唱的好。」只見潘金蓮在人叢裡,雙眼直射那兩個歌童,口裡暗暗低言道:「這兩個小夥子,不但唱的好,就他容貌也標致的緊。」心下便已有幾分喜他了。當下西門慶打發兩個歌童東廂房安下,一面叫擺飯與苗秀、苗實吃,一面整頓禮物回書,答謝苗員外。

畢竟未知何如,且聽下回分解:
