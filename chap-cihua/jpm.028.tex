%# -*- coding: utf-8 -*-
%!TEX encoding = UTF-8 Unicode
%!TEX TS-program = xelatex
% vim:ts=4:sw=4
%
% 以上设定默认使用 XeLaTex 编译,并指定 Unicode 编码,供 TeXShop 自动识别

%第二十八回 
\chapter{陳經濟因鞋戲金蓮\KG 西門慶怒打鐵棍兒}


\begin{showcontents}{}



「風波境界立身難,  處世規模要放寬,

萬事盡從忙裡錯,  此心須向靜中安;

路當平處行更穩,  人有常情耐久看,

直到始終無悔吝,  纔生枝節便多端。」

話說西門慶扶婦人到房中,脫去上下衣裳,着薄纊短襦,赤着身體。婦人止着紅紗抹胸兒,兩個並肩疊股而坐,重斟杯酌,復飲香醪。西門慶一手摟着他粉項,一遞一口和他吃酒,極盡溫存之態。睨視婦人雲鬟斜軃,酥胸半露,嬌眼乜斜,猶如沉醉楊妃一般,纖手不住只向他腰裡摸弄那話。那話因驚,銀托子還帶在上面,軟叮噹毛都魯的,纍垂偉長,西門慶戲道:「你還弄他哩!都是你頭裡諕出他風病來了。」婦人問「怎的風病?」西門慶道:「既不是風病,如何這軟癱熱化起不來了?你還不下去央及他央及兒哩!」婦人笑瞅了他一眼,一面蹲下身子去,枕着他一隻腿,取過一條褲帶兒來,把那話拴住,用手提着,說道:「你這廝頭裡那等,頭睜睜,股睜睜,把人奈何布布的,這咱你推風症裝佯死兒!」捉弄了一回,放在粉臉上,偎〈扌晃〉良久,然後將口吮之,又用舌尖挑舐其蛙口。那話登時暴怒起來,裂瓜頭,凹眼圓睜,落腮鬍挺身直豎。西門慶亦發坐在枕頭,令婦人馬爬在紗帳內,儘着吮咂,以暢其美。俄而淫思益熾,復與婦人交接,婦人哀告道:「我的達達,你饒了奴罷,又要掇弄奴也!」是夜二人淫樂,為之無度。有詩為證:

「戰酣樂極雲雨歇,嬌眼乜斜,手持玉莖猶堅硬。告才郎,將就些些,滿飲金杯頻勸,    兩情似醉如痴。」

「雪白玉體透簾幃,  口賽櫻桃手賽荑,

一脈泉通聲滴滴,  兩情脗合色迷迷;

翻來覆去魚吞藻,  慢進輕抽貓咬雞,

靈龜不吐甘泉水,  使得嫦娥敢暫離。」

一宿晚景題過。到次日,西門慶往外邊去了,婦人約飯時起來,換睡鞋。尋昨日腳上穿的那一雙紅鞋,左來右去少一隻。問春梅,春梅說:「昨日我和爹搊扶着娘進來;秋菊抱娘的鋪蓋來。」婦人叫了秋菊來問,秋菊道:「我昨日沒見娘穿着鞋進來。」婦人道:「你看胡說!我沒穿鞋進來,莫不我精着腳進來了?」秋菊道:「娘,你穿着鞋,怎的屋裡沒有?」婦人罵道:「賊奴才!還裝憨兒!無故只在這屋裡,你替我老實尋是的。」這秋菊二間屋裡,床上床下,到處尋了一遍,那裡討那雙鞋來。婦人道:「端的我屋裡有鬼,攝了我這雙鞋去了?連我腳上穿的鞋,也不見了;要你奴才在屋裡做甚麼?」秋菊道:「倒只怕娘忘記落在花園裡,沒曾穿進來。」婦人道:「敢是{入日}昏了!我鞋穿在腳上,沒穿在腳上,我不知道。」叫春梅:「你跟着這賊奴才往花園裡尋去。尋出來便罷,若尋不出我鞋來,教他院子裡頂着石頭跪着。」這春梅真個押着他,花園到處并葡萄架根前尋了一遍兒,那裡得來?再有一隻也沒了。正是

「都被六十收拾去,  蘆花明月竟難尋。」

尋了一遍兒回來,春梅罵道:「奴才!你媒人婆迷了路兒,沒的說了。王媽媽賣了磨,推不的了!」秋菊道:「好省恐人家不知,甚麼人偷了娘的這隻鞋去了。我沒曾見娘進屋裡去,敢是你昨日開花園門,放了那個,拾了娘的鞋去了?」被春梅一口稠唾沬噦了去,罵道:「賊見鬼的奴才!又攪纏起我來了!六娘叫門,我不替他開?可可兒的就放進人來了?你拖着娘的鋪蓋,就不經心瞧瞧,還敢說嘴兒!」一面押他到屋裡,回婦人說沒有鞋。婦人教採出他院子裡跪着。秋菊把臉哭喪下水來,說:「等我再往花園尋一遍,尋不着,隨娘打罷!」春梅道:「娘休信他。花園裡也掃得乾乾淨淨的,就是針也尋出來,那裡討鞋來!」秋菊道:「等我尋不出來,教娘打就是了。你在傍戳舌怎的?」婦人向春梅道:「也罷,你跟着他這奴才,看他那裡尋去?」這春梅又押他,在花園山子底下,各雪洞兒、花池邊、松墻下尋了一遍,沒有。他也慌了,被春梅兩個耳刮子,就拉回來見婦人。秋菊道:「還有那個雪洞裡沒尋哩!」春梅道:「那裡藏春塢是爹的暖房兒,娘這一向又沒到那裡。我看尋哩,尋不出來,我如你答話!」于是押着他到於藏春塢雪洞內。正面是張坐床,傍邊香几上都尋到,沒有。又向書篋內尋,春梅道:「這書篋內都是他的拜帖紙,娘的鞋怎的到這裡?沒的摭溜子捱工夫兒。翻的他恁亂騰騰的,惹他看見,又是一場兒!你這歪刺骨,可死成了!」良久,只見秋菊說道:「這不是娘的鞋!」在一個紙包內,裹着些棒兒香排草。取出來與春梅瞧:「可怎的有了娘的鞋?剛纔就調唆打我!」春梅看見,果是一隻大紅平底鞋兒,說道:「是娘的。怎麼來到這書篋內?好蹺蹊的事!」于是走來見婦人。婦人問:「有了我的鞋?端的在那裡?」春梅道:「在藏春塢爹暖房書篋內尋出來。和些拜帖子紙、排草、安息香包在一處。」婦人拿在手內,取過他的那隻鞋來一比,都是大紅四季花,嵌八寶段子,白綾平底繡花鞋兒,綠提根兒,藍口金兒,惟有鞋上鎖線兒差些。一隻是紗綠鎖線兒,一隻是翠藍鎖線,不仔細認不出來。婦人登在腳上試了試,尋出這一隻,比舊鞋略緊些。方知是來旺兒媳婦子的鞋,不知幾時與了賊強人,不敢拿到屋裡,悄悄藏放在那裡,不想又被奴才翻將出來。看了一回。說道:「這鞋不是我的鞋;奴才快與我跪着去!」吩咐春梅:「拿塊石頭與他頂着。」那秋菊哭起來,說道:「不是娘的鞋,是誰的鞋?我饒替娘尋出鞋來,還要打我;若是再尋不出來,不知違怎的打我哩!」婦人罵道:「賊奴才休說嘴!」春梅一面掇了塊大石頭,頂在頭上。那時婦人另換了雙鞋穿在腳上。嫌房裡熱,吩咐春梅:「把粧臺放在玩花樓上,那裡梳頭去。」梳了頭,要打秋菊,不在話下。卻說陳經濟早辰從舖子裡進來尋衣服,走到花園角門首,小鐵棍兒在那裡正頑着。見陳經濟手裡拿着一副銀網巾圈兒,便問:「姑夫,你拿的甚麼?與了我耍子兒罷。」經濟道:「此是人家當的網巾圈兒,來贖,我尋出來與他。」那小猴子笑嘻嘻道:「姑夫,你與了我耍子罷,我換與你件好物件兒。」經濟道:「俊孩子!此是人家當的。你要,我另尋一副兒與你耍子。你有甚麼好物件?拿來我瞧。」那猴子便向腰裡,掏出一隻紅繡花鞋兒,與經濟看。經濟便問:「是那裡的?」那猴子笑嘻嘻道:「姑夫,我對你說了罷。我昨日在花園裡耍子,看見俺爹吊着俺五娘兩隻腿在葡萄架兒底下,一陣好風搖落;後俺爹進去了,我尋俺春梅姑姑要菓子,在葡萄架底下,拾了這隻鞋。」經濟接在手裡,曲似天邊新月,紅如退瓣蓮花。把在掌中,恰剛三寸,就知是金蓮腳上之物。便道:「你與了我,明日另尋一對好圈兒與你耍子。」猴子道:「姑夫你休哄我!我明日就問你要了。」經濟道:「我不哄你。」那猴子一面笑的耍去了。這陳經濟把鞋褪在袖中,自己尋思:「我幾次戲他,他口兒且是活。及到中間,又走滾了,不想天假其便,此鞋落在我手裡。今日我着實撩逗他一番,不怕他不上帳兒!」正是:

「時人不用穿針線,  那得工夫送巧來。」

經濟袖着鞋,逕往潘金蓮房來,轉過影壁,只見秋菊跪在院內。便戲道:「小大姐,為甚麼來投充了新軍?又掇起石頭來了。」金蓮在樓上聽見,便叫春梅問道:「是誰說他掇起石頭來了?乾淨這奴才沒頂着?」春梅道:「是姐夫來了。秋菊頂着石頭哩!」婦人便叫:「陳姐夫,樓上沒人,你上來不是?」這小夥兒,方扒步撩衣,上的樓來。只見婦人在樓前面開了兩扇窗兒,掛着湘簾,那裡臨鏡梳頭。這陳經濟走到傍邊一個小杌兒坐下,看見婦人黑油般頭髮,手挽着梳,還拖着地兒,紅絲繩兒扎着。一窩絲攢上,戴着銀絲䯼髻,還墊出一絲香雲。䯼髻內安着許多玫瑰花瓣兒,露着四鬢上,打扮的就是個活觀音。須臾,看着婦人梳了頭,掇過粧臺去,向面盆內洗了手,穿上衣裳,喚春梅:「拿茶來與姐夫吃。」那經濟只是笑,不做聲。婦人因問:「姐夫笑甚麼?」經濟道:「我笑你管情不見了些甚麼兒?」婦人道:「賊短命!我不見了,關你甚事?你怎的曉得?」經濟道:「你看我好心倒做了驢肝肺,你倒訕起我來。恁說我去罷!」抽身往樓下就走。被婦人一把手拉住,說道:「怪短命,會張致的!來旺兒媳婦子死了,沒了想頭了。卻怎麼還認的老娘?」因問:「你猜着我不見了甚麼物件兒?」這經濟向袖中取出來,提搊着鞋拽靶兒,笑道:「你看這個好的兒是誰的?」婦人道:「好短命!原來是你偷拿了我的鞋去了。教我打着丫頭,遶地裡尋。」經濟道:「你怎的到得我手裡?」婦人道:「我這屋裡再有誰來?敢是你賊頭鼠腦,偷了我這隻鞋去了!」經濟道:「你老人家不害羞!我這兩日又不往你這屋裡來,我怎生偷你的?」婦人道:「好賊短命!等我對你爹說,你到偷了我鞋,還說我不害羞。」經濟道:「你只好拿爹來諕我罷了!」婦人道:「你好小膽子兒!明知道和旺兒媳婦子七個八個,你還調戲他,想那淫婦教你戲弄。既不是你偷了我的鞋,這鞋怎落在你手裡?趁早實供出來,交還與我鞋,你還便益。自古物見主不索取,但迸半個不字,教你死無葬身之地!」經濟道:「你老人家是個女番子,且是倒會的放刁!這裡無人,咱每好講。你既要鞋,拿一件物事兒,我換與你。不然,天雷也打不出去!」婦人道:「好短命!我的鞋應當還我。教換甚麼事兒與你?」經濟笑道:「五娘,你拿你袖的那方汗巾兒賞與兒子,兒子與了你的鞋罷。」婦人道:「我明日另尋一方好汗巾兒;這汗巾兒;是你爹成日眼裡見過,不好與你的。」經濟道:「我不,別的就與我一百方,也不筭;一心我只要你老人家這方汗巾兒。」婦人笑道:「好個老成久慣的短命!我也沒氣力和你兩個纏!」于是向袖中取出一方細撮穗,白綾桃線,鶯鶯燒夜香汗巾兒,上面連銀三字兒,都掠與他。這經濟連忙接在手裡,與他深的唱個喏。婦人吩咐:「你好生藏着,休教大姐看見。他不是好嘴頭子!」經濟道:「我知道。」一面把鞋遞與他,如此這般:「是小鐵棍兒昨日在花園裡拾的,今早拿着問我換網巾圈兒耍子。」一節,告訴一遍。婦人聽了粉面通紅,銀牙暗咬,說道:「你看賊小奴才油手!把我這鞋弄的恁添黑的。看我教他爹打他不打他!」經濟道:「你弄殺我。打了他不打緊,敢就賴在我身上,是我說的,千萬休要說罷。」婦人道:「我饒了小奴才,除非饒了蝎子!」可有他兩個正說在熱鬧處,忽聽小廝來安兒來尋:「爹在前廳,請姐夫寫禮帖兒哩。」婦人連忙攛掇他出去了。下的樓來,教春梅取板子來,要打秋菊。秋菊說着不肯倘,說道:「尋將娘的鞋來,娘還要打我!」婦人把剛纔陳經濟拿的鞋遞與看,罵道:「賊奴才!你把那個當我的鞋,將這個放在那裡?」秋菊看見,把眼瞪了半日,不敢認。說道:「可是怪的勾當!怎生跑出娘的三隻鞋來了!」婦人道:「好大膽奴才!你敢是拿誰的鞋搪塞我,倒如何說我是三隻腳的蟾!這個鞋從那裡出來了?」不由分說,教春梅拉倒,打了十下。打的秋菊抱股而哭,望着春梅道:「都是你開門,教人進來,收了娘的鞋,這回教娘打我!」春梅罵道:「你倒收拾娘鋪蓋,不見了娘的鞋。娘打了你這幾下兒,還敢抱怨人!早是這隻舊鞋,若是娘頭上的簪環不見了,你也推賴個人兒就是了!娘惜情兒,還打的你少;若是我,外邊叫個小廝,辣辣的打上他二三十板,看這奴才怎麼樣的!」幾句罵得秋菊忍氣吞聲,不言語了。當下西門慶叫了經濟到前廳,封尺頭禮物,送提刑所賀千戶,新陞了淮安提刑所,掌刑正千戶。本衛親識,都與他送行,在永福寺,不必細說。西門慶差了鉞安送去,廳上陪着經濟吃了飯,歸到金蓮房中。這金蓮千不合,萬不合,把小鐵棍兒拾鞋之事,告訴一遍。說道:「都是你這沒才料的貨,平白幹的勾當,教賊萬殺的小奴才,把我的鞋拾了,拿到外頭,誰是沒瞧見?被我知道,要將過來了。你不打與他兩下,到明日慣了他!」西門慶就不問誰,告你說來,一沖性子,走到前邊。那小猴子不知,正在石臺基頑耍,被西門慶揪住頂角,拳打腳踢,殺豬也似叫起來,方纔住了手。這小猴子,倘在地下,死了半日。慌得來昭兩口子走來扶救,半手甦醒,見小廝鼻口流血,抱他到房裡問,慢慢問他,方知為拾鞋之事。拾了金蓮一隻鞋,因和陳經濟換圈兒,惹起事來。這一丈青氣忿忿的,走到後邊廚下,指東罵西,一頓海罵道:「賊不逢好死的淫婦!王八羔子!我的孩子和你有甚冤仇?他纔十一十二歲,曉的甚麼?知道〈毛皮〉生在那塊兒!平白地調唆打他恁一頓,打的鼻口都流血;假若死了他,淫婦,王人兒也不好,稱不了你甚麼願!」于是廚房裡罵了,到前邊又罵,整罵了一二日還不定教。金蓮在房中陪西門慶吃酒,還不知道。晚夕上床宿歇,西門慶見婦人腳上穿着兩隻紗紬子睡鞋兒,大紅提根兒,因說道:「阿呀!如何穿這個鞋在腳上?怪怪的不好看。」婦人道:「我只一雙紅睡鞋,倒乞小奴才拾了一隻,弄油了我的那,那裡再討第二雙來?」西門慶道:「我的兒,你到明日做一雙兒穿在腳上。你不知,我達一心只喜歡穿紅鞋兒,看着心裡愛。」婦人道:「怪奴才!可可兒的來,我想起一作事來,要說又忘了。」因令春梅:「你取那隻鞋來,與他瞧。你認的這鞋是誰的鞋?」西門慶道:「我不知道是誰的鞋。」婦人道:「你看他還打張雞兒哩!瞞着我黃貓黑尾,你幹的好萌兒!一行死了。來旺兒媳婦子的一隻臭蹄,寶上珠也一般,收藏在山子底下藏春塢雪洞兒裡,拜帖匣子內,攪眼些字紙和香兒一處放着。甚麼罕稀物件,也不當家化化的!怪不的那賊淫婦死了,墮阿鼻地獄!」指着秋菊罵:「奴才當我的鞋,又翻出來,教我打了幾下。」吩咐春梅:「趁早與我掠出去!」春梅把鞋掠在地下,看着秋菊說道:「賞與你穿了罷!」那秋菊拾在手裡,說道:「娘這個鞋,只好盛我一個腳指頭兒罷了。」婦人罵道:「賊奴才!還教甚麼〈毛皮〉娘哩!他是你家主子前世的娘。不然,怎的把他的鞋收藏的嬌貴,到明日好傳代?沒廉恥的貨!」秋菊拿着鞋,就往外走。被婦人又叫回來,吩咐:「取刀來,等我把淫婦剁做幾戳子,掠到毛司裡去。叫賊淫婦陰山背後,永世不得超生。因向西門慶道:「你看着越心疼,我越發偏剁個樣兒你瞧!」西門慶笑道:「怪奴才!丟開手罷了。我那裡有這個心?」婦人道:「你沒這個心,你就睹了誓。淫婦死的不知往那去了,你還留他鞋做甚麼?早晚有省好思想他,正經俺每和你恁一場,你也沒恁個心兒,還教人和你一心一計哩!」西門慶笑道:「罷了,怪小淫婦兒!偏有這些兒的。他就在時,也沒曾在你根前行差了禮法。」于是摟過粉項來就親了個嘴,兩個雲雨做一處。正是:

「動人春色嬌還媚,  惹蝶芳心軟意濃。」

有詩為證:

「漫吐芳心說向誰,  欲於何處寄相思;

相思有盡情難盡,  一日都來十二時。」

畢竟未知後來如何,且聽下回分解:





\end{showcontents}


