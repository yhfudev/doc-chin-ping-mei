%# -*- coding: utf-8 -*-
%!TEX encoding = UTF-8 Unicode
%!TEX TS-program = xelatex
% vim:ts=4:sw=4
%
% 以上设定默认使用 XeLaTex 编译,并指定 Unicode 编码,供 TeXShop 自动识别

%第九十九回 
\chapter{劉二醉罵王六兒\KG 張勝忿殺陳經濟}

格言:

「一切諸煩惱,  皆從不忍生,

見機而耐性,  妙悟生光明;

佛語戒無倫,  儒書貴莫爭,

好個快活路,  只是少人行。」

話說陳經濟過了兩日,到第三日,卻是五月二十五日他生日。春梅後廳整置酒肴,與他上壽,合家歡樂了一日。次日早辰,經濟說:「我一向不曾往河下去,今日沒事去走一遭。一者和主管算帳,二來就避炎暑,散走走便回。」春梅分付:「你去坐一乘轎子,少要勞碌。」交兩個軍牢抬著轎子,小姜兒跟隨,逕往河下馬頭上謝家大酒樓店中來,一路無詞。午後時分,早到河下大酒樓前,下了轎子,進入裏面。兩個主管齊來參見,說:「官府貴體好些?」那經濟一心只在韓愛姐身上,便道:「生受二位夥計掛心!」坐了一回,便起身。分付主管:「查下帳目,等我來算。」就轉身到後邊。八老又早迎見,報與王六兒夫婦。韓愛姐正在樓上凭欄盼望,揮毫洒翰,作了幾首詩詞,以遣悶懷。忽報陳經濟來了,連忙輕移蓮步,款蹙湘裙,走下樓來。母子面上,堆下笑來迎接,說道:「官人,貴人難見面,那陣風兒吹你到俺這里?經濟與母子作了揖,同進入閣兒內坐定。少頃,王六兒點茶上來。吃畢茶,愛姐道:「請官人到樓上奴房內坐。」經濟上的樓來,兩個如魚得水,似漆投膠,無非說些深情密意的話兒。愛姐硯臺底下,露出一幅花箋。經濟取來觀看。愛姐便說:「此是奴家這幾日盼你不來,閑中在樓上作得幾首詞,以消遣悶懷。恐污官人貴目!」經濟念了一遍。上寫著:

「倦倚繡床愁懶動,  閒垂繡帶鬢鬟低;

玉郎一去無消息,  一日相思十二時。」

右春

「危樓高處眺晴光,  滿架薔薇靄異香;

十二欄杆閑凭遍,  南薰一味透襟涼。」

右夏

「帳冷芙蓉夢不成,  知心人去轉傷情;

枕邊淚似階前雨,  隔著窗兒滴到明。」

右秋

「羞對菱花拭淨粧,  為郎瘦損減容光;

閉門不管閒風月,  分付梅花自主張。」

右冬

經濟看了,極口羨,喝采不已。不一時,王六兒安排酒肴上樓。撥過鏡架,就擺在梳粧卓上。兩個並坐,愛姐篩酒一盃,雙手遞與經濟,深深道了萬福,說:「官人一向不來,妾心無時不念!前八老來,又多謝盤纏,舉家感之不盡!」經濟接酒在手,還了喏,說:「賤疾不安,有失期約,姐姐休怪!」酒盡,也篩一盃,敬奉愛姐吃過。兩人坐定,把酒來斟。王六兒、韓道國上來,也陪吃了幾盃,各取方便下樓去了。教他二人自在吃幾盃,敘些闊別話兒。良久,吃得酒濃時,情興如火,免不得再把舊情一敘。交歡之際,無限恩情。穿衣起來,洗手更酌,又飲數盃,醉眼朦朧,餘興未盡。這小郎君一向在家中不快,又心在愛姐,一向未與渾家行事。今日一旦見了情人,未肯一次即休。正是:

「生死冤家,  五百年前撞在一處!」

經濟魂靈,都被他引亂。少頃,情竇復起,又幹一度。自覺身體困倦,打熬不過。午飯也沒吃,倒在床上,就睡著了。也是合當禍起,不想下邊販絲綿何官人來了。王六兒陪他在樓下吃酒。韓道國出去街上,買菜蔬肴品果子來配酒。兩個在下邊行房。落後韓道國買將果菜來,三人又吃了幾盃。約日西時分,只見酒家店坐地虎劉二,吃的酩酊大醉,軃身衣衫,露著一身紫肉,提著拳頭,走來酒樓下大叫,採去何蠻子來要打。諕的兩個主管,見經濟在樓上睡,恐他聽見。慌忙走出櫃來,向前喏說道:「劉二哥,何官人並不曾來。」這劉二那里依聽,大拔步撞入後邊韓道國屋裏,一手把門簾扯上半邊。見何官人正和王六兒並肩飲酒,心中大怒,罵那何官人:「賊狗男女!我{入日}你娘!那里沒尋你,卻在這里!你在我店中占著兩個粉頭,幾遭歇錢不與。又塌下我兩個月房錢,卻來這里養老婆?」那何官人忙出來說:「老二你請回,我去也。」那劉二罵道:「你?你這狗{入日}!」不防颼的一拳來,正打何官人面間上,登時就青腫起來。那何官人起來,奪了跑了。劉二將王六兒酒卓一腳登翻,家活都打了。王六兒便罵道:「是那里少死的賊殺才,無事來老娘屋裏放屁?老娘不是耐驚耐怕兒的人!」被劉二向前一腳,跺了個仰八叉,罵道:「我{入日}你淫婦娘!你是那里來的無名少姓私窠子?不來老爺手里報過,許你在這酒店內趁熟?還與我搬去!若搬遲,須乞我一頓好拳頭!」那王六兒道:「你是那里來的光棍搗子?老娘就沒親戚兒,許你便來欺負老娘,要老娘這命做甚麼?」一頭撞倒哭起來。劉二罵道:「我把淫婦腸子也踢斷了!你還不知老爺是誰哩?」這裡喧亂,兩邊鄰舍并街上過往人,登時圍著約有許多。不知道的旁邊人說:「王六兒你新來,不知他是守備老爺府中管事張虞候的小舅子,有名坐地虎劉二,在酒家店住,專一是打粉頭的班頭,降酒客的領袖!你讓他些兒罷,休要不知利害,這地方人誰敢惹他?」王六兒道:「還有大是他的,采這殺才做甚做?」陸秉義見劉二打得兇,和謝胖子做好做歹,把他勸的去了。陳經濟正睡在床上,聽見樓下攘亂,便起來看。時天已日西時分,問:「那里攘亂?」那韓道國不知走的往那里去了。只見王六兒披髮垢面上樓,如此這般告訴說:「那里走來一個殺才搗子,諢名喚地虎劉二,在酒家店住,說是咱府里管事張虞候小舅子,因尋酒客,無事把我踢打,罵了恁一頓去了!又把家活酒器,都打得粉碎!」一面放聲大哭起來。經濟叫上兩個主管問他,兩個都面面相覷,不敢說。陸主管嘴快,說:「是府中張主管小舅子,來這里尋何官人,說少他二個月房錢,又是歇錢,來討。見他在屋裏吃酒,不由分說,把簾子扯下半邊來,打了何官人一拳,諕的何官人跑了。又和老韓娘子兩個相罵,踢了一腳,烘的滿街人看。」這經濟恐怕天晚惹起來,分付把眾人喝散。問劉二那廝,主管道:「被小人勸他回去了。」經濟聽了,記在心內。安撫王六兒母子放心:「有我哩,不妨事。你母子只情住著,我家去自有處置。」主管算了利錢銀兩,遞與他,打發起身上馬,伴當跟隨,打著馬走。剛走趕進城來,天已昏黑,心中甚惱。到家見了春梅,交了利息銀兩。歸入房中,一宿無話。到次日,心心念念,要告春梅說。展轉尋思:「且住!等我慢慢尋張勝那廝幾件破綻,亦發教我姐姐對老爺說了,斷送了他性命!叵耐這幾次在我身上欺心,敢說我是他尋得來,知我根本出身,量視我,禁不得他!」正是:

「冤仇還報當如此,  機會遭逄莫遠圖;

踏破鐵鞋無覓處,  得來全不費工夫。」

一日,經濟來到河下酒店內,見了愛姐母子,說:「外日吃驚!」又問陸主管道:「劉二那廝不曾走動?」陸主管道:「自從那日去了,再不曾來。」又問韓愛姐。那何官人也沒來行走。這經濟吃了飯,算畢帳目,不免又到愛姐樓上,兩個敘了回衷腸之話,幹訖一度出來。因閑中叫過量酒陳三兒近前,如此這般:「打聽府中張勝和劉二幾庄破綻。」這陳三兒千不合,萬不合,說出張勝包占著府中出來的雪娥在酒家店做表子。劉二又怎的各處巢窩加三討利,舉放私債,竊逞老爺們壞事。這經濟一口聽記在心,又與了愛姐二三兩盤纏。和主管算了帳目,包了利息銀兩作別,騎頭口來家。閑話休題。一向懷意在心,一者也是冤家相湊,二來合當禍這般起來。不料東京朝中徽宗天子,見大金人馬犯邊,搶至腹內地方,聲息十分緊急。天子慌了,與大臣計議,差官往北國講和,情願每年輸納歲幣金銀彩帛數百萬。一面傳位與太子登基,改宣和七年為靖康元年。宣寡號為欽宗皇帝在位,徽宗自稱太上道君皇帝,退居龍德宮。朝中陞了李綱為兵部尚書,分部諸路人馬。種師道為大將,總督內外宣務。一日降了一道勅書來濟南府守備,陞他為山東都統制,提調人馬一萬,往東昌府駐扎,會同巡撫都御史張叔夜防守地方,阻當金兵。守備正在濟南府衙正坐,忽然左右來報:「有朝廷降勅來,請老爺接旨意!」這周守備不敢怠慢,香案迎接勅旨,跪聽宣讀。使命官開讀,其略曰:

「奉天承運皇帝制曰:朕聞文能安邦,武能定國。三皇憑禮樂而有封疆,五帝用征伐而定天下。爭從順逆,人有賢愚。朕承祖宗不拔之洪基,上皇付托之重位。創造萬事,惕然悚悮。自古舜征四凶,湯伐有苗。非用兵而不能剋,非威武而莫能安。兵乃邦家爪牙,武定封疆扞禦。茲者中原陸沉,大羊犯順。遼寇擁兵西擾,金虜控騎南侵。生民塗炭,朕甚憫焉!山東濟南制置使周秀,老練之才,干城之將。屢建奇勳,忠勇茂著。用兵有略,出戰有方。今陞為山東都統制,兼四路防禦使。會同山東巡撫都御史張叔夜,提調所部人馬,前赴高陽關防守,聽大將種師道分布截殺。安幾危之社稷,驅猖獗之腥膻!嗚乎!任賢匡國,赴難勤王,乃臣子之忠誠;旌善賞功,激揚敵愾,實朝廷之大興。名殫厥忠,以副朕意。欽哉!故諭。

下書靖康元年秋九月日諭。」

周守備開讀已畢,打發使命官去了。一面叫過張勝、李安兩個虞候近前,分付先押兩車箱馱行李細軟器物家去。原來在濟南做了一年官職,也撰得巨萬金銀。都裝在行李馱箱內委托二人:「押到家中,交割明白。晝夜巡風仔細,我不日會同你巡撫張爺,調領四路兵馬,打清河縣起身。」二人當日領了鈞旨,打點車輛起身先行,一路無詞。有日到於府中,交割明白。二人晝夜內外巡風,不在話下。卻說陳經濟,見張勝押車輛來家,守備陞了山東統制,不久將到。正欲把心腹中事,要告訴春梅。等守備來家,要發露張勝之事。不想一日,因渾家葛翠屏往娘家回門住去了,他獨自個在西書房寢歇,春梅早辰驀進房看他,見無丫鬟跟隨,兩個就解衣在房內雲雨做一處。不防張勝搖著鈴巡風過來。到書院角門外,聽見書房內彷彿有婦人笑語之聲。就鈴聲按住,慢慢走來窗下竊聽。原來春梅在裏面,與經濟交姤。聽得經濟告訴春梅說:「叵耐張勝那廝,好生欺壓於我!說我當初虧他尋得來,幾次在下人前敗壞我。昨日見我在河下開酒店來,一徑使小舅子坐地虎劉二,專一倚逞他在姐夫麾下,在那里開巢窩,放私債,把去雪娥,隱占在外姦宿。只瞞了姐姐一人眼目,昨日教他小舅子劉二,打我酒店來,把酒客都打散了。我幾次含忍,不敢告姐姐說。趁姐夫來家,若不早說知,往後我定然不敢往河下做買賣去了!」春梅聽了,說道:「這廝恁般無禮!雪娥那賊人賣了,他如何又留住在外?」經濟道:「他非是欺壓我,就是欺壓姐姐一般!」春梅道:「等他爺來家,交他定結果了這廝!」常言道:「隔墻須有耳,窗外豈無人!」兩個只管在內說,卻不知張勝窗外聽了個不亦樂乎!口中不言,心內暗道:「此時教他算計我們,我先算計了他罷!」一面撇下鈴,走到前邊班房內,取了把解腕鋼刀。說時遲,那時快,在石上磨了兩磨,走入書院中來,不想天假其便,還春梅不該死於他手!忽被後邊小丫鬟蘭花兒,慌慌走來叫春梅,報說:「小衙內金哥兒,忽然風搖倒了,快請奶奶看去。」諕的春梅兩步做來一步走,奔入後房中看孩兒去了。剛進去了,那張勝提著刀子逕奔到書房內。不見春梅,只見經濟睡在被窩內。見他進來,叫道:「阿呀!你來做甚麼?」張勝怒道:「我來殺你!你如何對淫婦說倒要害我?我尋得你來不是了!反恩將仇報?常言:『黑頭蟲兒不可救,救之就要吃人肉。』休走!吃我一刀子!明年今日是你死忌!」那經濟光赤條身子,沒處躲,摟著被,乞他拉被過一邊,向他身就扎了一刀子來。扎著軟肋,鮮血就邈出來。這張勝見他掙扎,復又一刀去,攘著胸膛上,動彈不得了!一面採著頭髮,把頭割下來。正是:

「三寸氣在千般用,  一日無常萬事休!」

可憐經濟青春不上三十九,死於非命!張勝提刀,遶屋裏床背後尋春梅不見,大拔步逕望後廳走。走到儀門首,只見李安背著牌鈴,在那里巡風。一見張勝兇神也似提著刀跑進來,便問:「那里去?」張勝不答,只顧走,被李安攔住。張勝就向李安截一刀來。李安冷笑道:「我叔叔有名山東夜叉李貴,我的本不用借!」早飛起右腳,只聽忒楞的一聲,把手中刀子踢落一邊。張勝急了,兩個就揪採在一處。被李安一個潑腳,跌番在地。解下腰間纏帶,登時綁了,攘的後廳春梅知道。說:「張勝持刀入內,小的拏住了!」那春梅方救得金哥卻甦著,聽言大驚失色。走到書院內,經濟已被殺死在房中,一地鮮血橫流,不覺放聲大哭。一面使人報知渾家,葛翠屏慌奔家來。看見經濟殺死,哭倒在地,不省人事。被春梅扶救甦省過來,拖過屍首,買棺材裝殯。把張勝墩鎖在監內,單等統制來家處治這件事。那消數日期程,軍情事務緊急,兵牌來催促,周統制調完各路兵馬,張巡撫又早先往東昌府那里等候取齊。統制在家,春梅把殺死經濟一節說了。李安將兇器放在面前,跪稟前事。統制大怒,坐在廳上,提出張勝,也不問長短,喝令軍牢:「五棍一換,打一百棍!」登時打死。隨即馬上差旗牌快手,往河下捉拏坐地虎劉二,鎖解前來。孫雪娥見拏了劉二,恐怕拏他,走到房中自縊身死。旗牌拏劉二到府中,統制也分付打一百棍,當日打死。烘動了清河縣,大鬧了臨清洲。正是:

「平生作惡欺天,  今日上蒼報應!」

有詩為證:

「為人切莫用欺心,  舉頭三尺有神明;

若還作惡無報應,  天下兇徒人食人。」

當時統制打死二人,除了地方之害。分付李安:「將馬頭大酒店還本主,把本錢收算來家。」分付春梅:「在家,與經濟做齋累七,打發城外永福寺擇吉日葬埋。」留李安、周義看家。把周忠、周仁帶去軍門等應。春梅晚夕與孫二娘置酒送餞,不覺簇地兩行淚下,說:「相公此去,未知幾時回還?出戰之間,須要仔細。番兵猖獗,不可輕敵!」統制道:「你每自在家清心寡慾,好生看守孩兒,不必憂念!我既受朝廷爵祿,盡忠報國。至於吉凶存亡,付之天也!」囑付畢,過了一宿。次日,軍馬都在城外屯集,等候統制起程。果然人馬整齊!但見:

「繡旗飄號帶,畫鼓間銅鑼。三股叉,五股叉,燦燦秋霜;六花鎗,點銅鎗,紛紛瑞雪。蠻牌引路,強弓硬弩當先;火炮隨車,大斧馬刀在後。鞍上將,似南山猛虎,人人好鬬偏爭;坐下馬,如北海蛟虬,騎騎能爭敢戰。端的刀鎗流水急,果然人馬撮風行!」

當下一路無詞。有日哨馬來報說:「不可前進,馬哨東昌府下。」達統制差一面令字藍旗,把人馬屯城外:「我報進城。」巡撫張叔夜聽見周統制人馬來到,與東昌府知府達天道出衙迎接,至公廳敘禮坐下,商議軍情,打聽聲息緊慢,駐馬一夜。次日人馬早行,往關上防守去了。不在話下。卻表韓愛姐母子在謝家樓店中,聽見經濟已死,愛姐晝夜只是哭泣,茶飯都不吃。一心只要往城內統制府中,見經濟屍首一見,死了也甘心!父母旁人,百般勸解不從。韓道國無法可處,使八老往統制府中,打聽經濟靈柩,已出了殯,埋在城外永福寺內。這八老走來回了話。愛姐一心只要到他墳上燒紙,哭一場,也是和他相交一場。做父母的,只得依他。顧了一乘轎子,到永福寺中,問長老:「葬於何處?」長老令沙彌引到寺後:「新墳堆便是。」這韓愛姐下了轎子,到墳前點著紙錢,道了萬福,叫聲:「親郎!我的哥哥!奴寔指望我你同諧到老,誰想今日死了!」放聲大哭,哭的昏暈倒了,頭撞於地下,就死過去了。慌了韓道國和王六兒向前扶救:「大姐姐!」叫不應,越發慌了。只見那日是葬了三日,春梅與渾家葛翠屏,坐著兩乘轎子,伴當跟隨,抬三牲祭物來,與他煖墓燒紙。看見一個年小的婦人,穿著縞素,頭戴孝髻,哭倒在地。一個男子漢,和一中年婦人,摟抱他,扶起來又倒了,不省人事。乞了一驚!因問:「那男子漢是那里的?」這韓道國夫婦,向前施禮,把從前已往話,告訴了一遍:「這個是我的女孩兒韓愛姐。」春梅一聞愛姐之名,就想起昔日曾在西門慶家中會過,又認得王六兒。韓道國悉把東京蔡府中出來一節說了一遍:「女孩兒曾與陳官人有一面相交,不料死了,他只要來墳前見他一見燒紙錢。不想到這里又哭倒了。當下兩個救了半日,這愛姐吐了口粘痰,方纔甦省。尚哽咽哭不出聲來。痛哭了一場,起來與春梅、翠屏,插燭也似磕了四個頭,說道:「奴與他雖是露水夫妻,他與奴說山盟,言海誓,情深意厚!實指望和他同諧到老,誰知天不從人願,一旦他先死了,撇得奴四脯著地。他在日曾與奴一方吳綾帕兒,上有四句情詩。知道宅中有姐姐,奴願做小!倘不信……向袖中取出吳綾帕兒來。上面寫詩四句,春梅同葛翠屏看了,詩云:

「吳綾帕兒織迴紋,  洒翰揮毫墨跡新;

寄與多情韓五姐,  永諧鸞鳳百年情。」

愛姐道:「奴也有個小小鴛鴦錦囊,與他佩帶在身邊。兩個都扣繡著並頭蓮。每朵蓮花瓣兒一個字兒:『寄與情郎,隨君膝下。』」春梅便問翠屏:「怎的不見這個香囊?」翠屏:「在地〈衤旋〉子上拴著不是?奴替他裝殮在棺槨內了。」當下祭畢,讓他母子到寺中,擺茶飯,與他吃了些飯食。做父母的見天色將晚,催促他起身。他只顧不思動身。一面跪著春梅、葛翠屏哭說:「情愿不歸父母,同姐姐守孝寡居,也是奴和他恩情一場!活是他妻小,死傍他魂靈!」那翠屏只顧不言語。春梅便說:「我的姐姐,只怕年小青春,守不住!只怕誤了你好時光!」愛姐便道:「奶奶說那里話?奴既為他,雖刳目斷鼻,也當守節,誓不再配他人!」囑付他父母:「你老公母回去罷,我跟奶奶和姐姐府中去也!」那王六兒眼中垂淚;哭道:「我承望你養活俺兩口兒到老,纔從虎穴龍潭中奪得你來,今日倒閃賺了我!」那愛姐口裏只說:「我不去了,你就留下我到家,也尋了無常!」那韓道國因見女孩兒堅意不去,和王六兒大哭一場,酒淚而別,回上臨清店中去了。這韓愛姐同春梅、翠屏坐轎子往府里來。那王六兒一路上悲悲切切,只是捨不的他女兒。哭了一場,又一場。那韓道國又怕天色晚了,顧上兩疋頭口,望前趕路。正是:

「馬遲心急路途窮,  身似浮萍類轉蓬;

只有都門樓上月,  照人離恨各西東。」

畢竟未知後來如何,且聽下回分解:

