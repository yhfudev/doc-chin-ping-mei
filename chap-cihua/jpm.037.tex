%# -*- coding: utf-8 -*-
%!TEX encoding = UTF-8 Unicode
%!TEX TS-program = xelatex
% vim:ts=4:sw=4
%
% 以上设定默认使用 XeLaTex 编译,并指定 Unicode 编码,供 TeXShop 自动识别

%第三十七回 
\chapter{馮媽媽說嫁韓氏女\KG 西門慶包占王六兒}

「吳〈舟光〉輕舸更遲遲,  別酒重斟惜醉携,

滄海侵愁光蕩漾,  亂山那恨色高低;

君馳蕙楫情何極,  我恣蘭干日向西,

咫尺煙波幾多地,  不須懷抱重萋萋」

話說西門慶打發蔡狀元、安進士去了。一日,騎馬帶眼紗,在街上喝道而過,撞見馮媽媽。便教小廝叫住問他:「爹說問你尋的那女子怎樣的?如何不往宅裡回話去?」那婆子兩步走到跟前,說:「這幾日我雖是看了幾個女子,都是買肉的,挑擔兒的,怎好回你老人家話。不想天使其便,眼跟前一個人家女兒,就想不起來,十分人材,屬馬兒的,交新年十五歲。若不是老婆子昨日打他門首過,他娘在門首請進我吃茶,我不得看見他哩!纔弔起頭兒沒多幾日,戴着雲髻兒,好不筆管兒般直縷的身子兒。纏得兩隻腳兒一些些,搽的濃濃的臉兒,又一點小小嘴兒,鬼精靈兒是的!他娘說他是五月端午養的,小名叫做愛姐。休說俺每愛,就是你老人家見了也愛的不知怎麼樣的了!」西門慶道:「你看這風媽媽子,我平白要他做什麼?家裡放着好少兒!實對你說了罷,此是東京蔡太師老爺府裡大管家翟爹要做二房,圖生長,托我替他尋。你若與他成了,管情不虧你。」因問道:「是誰家的女子?問他討個庚帖兒來我瞧。」馮媽媽道:「誰家的?我教你老人家知道了罷。遠不一千,近只在一磚;不是別人,是你家開絨線的韓夥計的女孩兒。你老人家要相看,等我和他老子說,討了帖兒來,約會下個日子,你只顧去就是了。」西門慶分付道:「既如此這般,就和他說。他若肯了,討了帖兒來,宅內回我話。」那婆子應諾去了。兩日,西門慶正在前廳坐的,忽見馮媽媽來回話,拏了帖兒與西門慶瞧。上寫着:「韓氏女命:年十五歲,五月初五日子時生。」便道:「我把你老人家的話,對他老子說了。既是大爹可怜見,孩兒也是有造化的;姐只是家寒,沒辦備的。」西門慶道:「你對他說,不費他一絲兒東西。凡一應衣服、首飾、粧奩、廂櫃等件,都是我這里替他辦備。還與他二十兩財禮,教他家止備女孩兒的鞋腳就是了。臨期還叫他老子送他往東京去。比不的與他做房裡人。翟管家要圖他生長做娘子,難得他女兒生下一男半女,也不愁個大富貴。」馮媽媽問道:「他那里請問你老人家,幾時過去相看?好預備。」西門慶道:「既是他應允了,我明日就過去看看罷。他那里再三有書來,要的急。就對他說,休教他預備什麼,我只吃鍾清茶就起身。」馮媽媽道:「爺爺,你老人家上門兒怪人家!就是雖不稀罕他的,也略坐坐兒。夥計家,莫不空教你老人家來了。」西門慶道:「你就不是了。你不知我有事?」馮媽媽道:「既是恁的,等我和他說。」一面先到韓道國家對他渾家王六兒一五一十,說了一遍:「宅內老爹看了你家孩子的帖兒,甚喜不盡。說來,不教你這里費一絲兒東西。一應粧奩陪送,都是宅內管。還與你二十兩銀子財禮,只教你家與孩兒做些生活鞋腳兒就是了。到明日還教你官兒送到那里。難得你家姐姐一年半載有了喜事,你一家子都是造化的了,不愁個大富貴。明日他老人家衙門中散了,就過來相看。教你一些兒休預備。他也不坐,只吃一鍾茶,看了就起身。」王六兒道:「真個?媽媽子休要說謊!」馮媽媽道:「你當家不恁的說,我來哄你不成?他好少事兒,家中人來人去,通不斷頭的。」婦人聽言,安排了些酒食與婆子吃了,打發去了,明日早來伺候。到晚韓道國來家,婦人與他商議已定。早起,往高井上叫了一擔甜水,買了些好細菓仁,放在家中,還往舖子裡做買賣去了。丟下老婆在家,艷粧濃抹,打扮的喬模喬樣;洗手剔甲,揩抹盃盞乾淨,剝下菓仁,頓下好茶,等候西門慶來。馮媽媽先來攛掇。西門慶衙門中散了,到家換了便衣靖巾,騎馬帶眼紗,玳安、琴童兩個跟隨,逕來韓道國家,下馬進去。馮媽媽連忙請入裏面坐了。良久,王六兒引着女兒愛姐出來拜見。這西門慶且不看他女兒,不轉睛只看婦人。見他上穿着紫綾襖兒,玄色段紅比甲,玉色裙子,下邊顯着趫趫的兩隻腳兒,穿着老鴉段子羊皮金雲頭鞋兒。生的長跳身材,紫月塀唐色瓜子臉,描的水鬢長長的。正是:

「未知就里何如,  先看他粧色油樣。」

但見:

「淹淹潤潤,不搽脂粉,自然體態妖嬈,嬝嬝娉娉,懶染鉛華,生定精神秀麗。兩彎眉畫遠山,一對眼如秋水。檀口輕開,勾引得狂蜂蝶亂,織腰拘束,暗帶著月意風情。若非偷期崔氏女,定然聞瑟卓文君。」

西門慶見見,心搖目蕩,不能定止。口中不說,心內暗道:「原來韓道國有這一個婦人在家,怪不的前日那些人鬼混他。」又見他女孩兒生的一表人物,暗道:「他娘母兒生的這般模樣,女兒有個不好的!」婦人先拜見了,教他女兒愛姐轉過來,望上向西門慶花枝招颭,綉帶飄飄,也磕了四個頭,起來侍立在旁。老媽連忙拏茶上來,婦人取來抹去盞上水漬,令他去遞上。西門慶把眼上下觀看,這個女子,烏雲疊鬢,粉黛盈腮;意態幽花酴麗,肥膚嫩玉生香。便令玳安氈包內取出錦帕二方,金戒指四個,白銀二十兩,教老媽安放在茶盤內。她娘忙將戒指帶在女兒手上,朝上拜謝,回房去了。西門慶對婦人說:「遲兩日,接你女孩兒往宅裡去,與他裁衣服。這些銀子,你家中替他做些鞋腳兒。」婦人連忙又磕下頭去,謝道:「俺每頭頂腳踏,都是大爹的;孩子的事,又教大爹費心。俺兩口兒就殺身也難報!虧了大爹,又多謝爹的插帶厚禮!」西門慶問道:「韓夥計不在家了?」婦人道:「他早辰說了話,就往舖子裡走了。明日教他往宅裡與爹磕頭去。」西門慶見婦人說話乖覺,一口一聲,只是爹長爹短,就把心來惑動了。臨出門,上覆他,我去哩!婦人道:「再坐坐。」西門慶道:「不坐了。」于是竟出門,一直來家,把上項告吳月娘說了。月娘道:「也是千里姻緣着線穿。既是韓夥計這女孩兒好,也是俺每費心一場。」西門慶道:「明日接他來住兩日兒,好與他裁衣服。我如今先拏十兩銀,替他打半副頭面簪鐶之類。」月娘道:「及緊儹做去,正好後日教他老子送去。咱這里不着人去罷了。」西門慶道:「把舖子關兩日也罷,還着來保同去,就府內問聲,前日差去節級送蔡駙馬的禮,到也不曾?」話休饒舌。過了兩日,西門慶果然使小廝接韓家女兒。他娘王氏買了禮,親送他來,進門與月娘大小眾人磕頭拜見,道生受,說道:「蒙大爹、大娘并眾娘每抬舉孩兒,這等費心,俺兩口兒知感不盡!」先在月娘房擺茶,然後明間內管待。李嬌兒、孟玉樓、潘金蓮、李瓶兒都陪坐。西門慶與他買了兩疋紅綠潞紬,兩疋綿紬,和他做裏衣兒。又叫了趙裁來,替他做兩套織金紗段衣服,一件大紅粧花段子袍兒。他娘王六兒安撫了女兒,晚夕回家去了。西門慶又替他買了半嫁粧,描金箱籠,鑑粧鏡架,盒罐,銅錫盆,淨桶,火架等件,非止一日,都治辦完備。寫了一封書信,擇定九月初十日起身。西門慶問縣裡討了四名快手,又撥了兩名排軍,執袋弓箭隨身。來保、韓道國雇了四乘頭口,緊緊保定車輛煖轎,送上東京去了不題。丟的王六兒在家,前出後空,整哭了兩三日。一日西門慶無事,騎馬來獅子街房里觀看,馮媽媽來遞茶。西門慶與了一兩銀子,說道:「前日韓夥計孩子的事累你,這一兩銀子,你買布穿。」婆子連忙磕頭謝了。西門慶又問:「你這兩日,沒到他那邊走走?」馮媽道:「老身那一日沒到他那里做伴兒坐?他自從女兒去了,本等他家裡沒人,他娘母靠慣了,他整哭了兩三日。這兩日纔翫下些兒來了。他又說:『孩子事,多累了爹。』問我:『爹曾與了你些辛苦錢兒沒有?』我便說:『他老人事忙,我連日宅裡也沒曾去。隨他老人家多少與我些兒,我敢爭?』他也許我等他官兒回來,重重謝我哩。」西門慶道:「他老子回來,已定有些東西,少不的謝你。」說了一回話,見左右無人,悄悄在婆子耳邊,如此這般:「你閑了,到他那裡取巧兒和他說,就說我上覆他,閑中我要他那里坐半日,看他意何如?肯也不肯?我明日還來討回話。」那婆子掩口冷冷笑道:「你老人家坐家的女兒偷皮匠,逢着的就上;一鍬撅了個銀娃娃,還要尋他娘母兒哩!夜晚些,等老身慢慢皮着臉對他說。爹,你還不知,這婦人,他是咱後街宰牲口王屠的妹子,排行叫六姐,屬蛇的,二十九歲。雖是打扮的喬樣,倒沒見他輸身。你老人家明日准來,等我問他討個話來回你。」西門慶道:「是了。」說畢,騎馬來家。婆子打發西門慶出門,做飯吃了,鎖了房門,慢慢來到牛皮巷婦人家。婦人開門,便讓進裏邊房里坐,道:「我昨日下了些麵,等你來吃,就不來了。」婆子道:「我可知要來哩!到人家,便就有許多事掛住了腿子,動不得身。」婦人道:「剛纔做的熱騰騰的飯兒,炒麵觔兒 ,你吃些。」婆子道:「老身纔吃的飯來,呼些茶罷。」那婦人便濃濃點了一盞茶,遞與他;看着婦人吃了飯。婦人道:「你看我恁苦,有我那冤家,靠定了他。自從他去了,弄的這屋裡空落落的,件件的都看了我。弄的我鼻兒烏,嘴兒黑,相個人模樣!倒不如他死了,扯斷腸子罷了!似這般遠離家鄉去了,你教我這心怎麼放的下來?急切要見他,見也不能勾!」說着,眼駿駿的哭了。婆子道:「說不得。自古養兒人家熱騰騰的,養女兒家冷清清。就是長一百歲,少不得也是人家的!你如今這等抱怨,到明日你家姐姐到府裡腳硬,生下一男半女,你兩口子受用,就不說我老身了。」婦人道:「大人家的營生,三層大兩層小,知道怎樣的!等他的長俊了,我每不知在那里晒牙揸骨去!」婆子道:「怎的恁般的說。你每姐姐比那個不聰明伶俐?愁針指女工不會?各人裙帶衣食,你替他愁?」兩個一遞一口,說勾良久。看看說得入港,婆子道:「我每說個傻話兒。你家官兒不在,前後去的恁空落落的,你晚夕一個人兒不害怕麼?」婦人道:「你還說哩,都是你弄得我。肯晚夕來和我做做伴兒?」婆子道:「只怕我一時來不到。我保舉箇人兒來與你做伴兒,你肯不肯?」婦人問是誰?婆子掩口笑道:「一客不煩二主,宅里大老爹,昨日到那邊房子里,如此這般對我說。見孩子去了,丟的你冷落,他要來和你坐半日兒。你怎麼說?這里無人,你若與凹上了,愁沒吃的、穿的、使的、用的?走上了時,到明日房子也替你尋得一所,強如在這僻格剌子里。」婦人聽了,微笑說道:「他宅里神道相似的幾房娘子,他肯要俺這醜貨兒?」婆子道:「你怎的這般說?自古道:『情人眼內出西施。』一來也是你緣法湊巧,爹他好閑人兒,不留心在你時,他昨日巴巴的肯到我房子裡說?又與了一兩銀子,說前日孩子的事累我;落後沒人在根前話,就和我說,教我來對你說,你若肯時,他還等我回話去。典田賣地,你兩家願意;我莫非說謊不成?」婦人道:「既是下顧,明日請他過來,奴這里等候。」這婆子見他吐了口兒,坐了一回,千恩萬謝去了。到次日西門慶來到,一五一十,把婦人話告訴一遍。西門慶不勝欣喜,忙秤了一兩銀子,與馮媽媽拏去治辦酒菜。那婦人聽見西門慶來,收拾房中乾淨,薰香設帳,預備下好茶好水。不一時,婆子拏籃子買了許多雞魚嗄飯菜蔬菓品,來廚下替他安排端正。婦人洗手剔甲又烙了一筋麵餅,明間內揩抹卓椅光鮮。西門慶約下午時分,便衣小帽,帶着眼紗,玳安、棋童兩個小廝跟隨,逕到門首,下馬進去。分付把馬回到獅子街房子裡去,晚上來接,止留玳安一人答應。西門慶到明間內坐下。良久,婦人扮的齊齊整整,出來拜見,說道:「前日打擾,孩子又累爹費心,一言難盡!」西門慶道:「一時不到處,你兩口兒休抱怨。」婦人道:「一家兒莫大之恩,豈有抱怨之理!」磕了四個頭。馮媽媽拏上茶來,婦人遞了茶。見馬回去了,玳安把大門關了。婦人陪坐一回,讓進裏坐。房正面紙門兒,廂的炕床,掛着四扇各樣顏色綾段剪貼的張生遇鶯鶯蜂花香的弔屏兒,上卓鑑粧鏡架,盒罐錫器家活堆滿。地下插着棒兒香,上面設着一張東坡椅兒。西門慶坐下。婦人又濃濃點一盞胡桃夾鹽笋泡茶,遞上去。西門慶吃了。婦人拉了盞,在下邊炕沿上陪坐,問了回家中長短。西門慶見婦人自己拏托盤兒,說道:「你這里還要個孩子使纔好。」婦人道:「不瞞爹說,自從俺家女兒去了,凡事不方便。那時有他在家,如今少不的奴自己動手。」西門慶道:「這個不打緊。明日教老馮替你看個十三四歲的丫頭子,且胡亂替替手腳。」婦人道:「也得俺家的來,少不得東軿西湊的,央馮媽媽尋一個孩子使。」西門慶道:「也不消。該多少銀子,等我與他。」那婦人道:「怎好又費煩你老人家?自恁累你老人家還少哩!」西門慶見他會說話,心中甚喜。一面馮媽媽進來安放卓兒,西門慶就對他說尋使女一節。馮媽媽道:「爹既是許了,你拜謝拜謝兒。南首趙嫂兒家有個十三歲的孩子,我明日領來與你看。也是一個小人家的親養的孩兒來,他老子是個巡捕的軍,因倒死了馬,少樁頭銀子,怕守備那里打,把孩子賣了,只要四兩銀子,教爹替你買下罷。」婦人連忙向前,道了萬福。不一時,擺下案碟菜蔬,篩上酒來。婦人滿斟一盞,雙手遞與西門慶。纔待磕下頭去,西門慶連忙用手拉起說:「頭里已是見過,不消又下禮了。只拜拜便了。」婦人笑吟吟道了萬福,旁邊一個小杌兒上坐下。廚下老媽將嗄飯菓菜,一一送上,又是兩筯軟餅。婦人用手揀肉絲細菜兒裹捲了,用小碟兒托了,遞與西門慶吃。兩個在房中,盃來盞去,做一處飲酒。玳安在廚房裡,老馮陪他,是有坐處,打發他吃,不在話下。彼此飲勾數巡,婦人把座兒挪近西門慶根前,與他做一處說話,遞菜兒。然後西門慶與婦人一遞一口兒吃酒。見無人進來,摟過脖子來,親嘴咂舌。婦人便舒手下邊籠揝西門慶玉莖,彼此淫心蕩漾,把酒停住不吃了。掩上房門,褪去衣褲,婦人就在裏邊炕床上,伸開被褥,那時已是日色平西時分。西門慶乘着酒興,順袋內取出銀托子來使上,婦人用手打弄,見奢稜跳腦,紫強光鮮沉甸甸,甚是粗大。一壁坐在西門慶懷裡,一面在上,兩個且摟着脖子親嘴。婦人乃蹺起一足,以手導那話入牝中,兩個挺一回。西門慶摸見婦人柔膩,牝毛疏秀,意欲交接。令婦人仰臥于床背,把雙枕以手雙足置之于腰眼間,肆行抽送,怎見的這場雲雨?但見:

「威風迷翠榻,殺氣瑣死衾。珊瑚枕上施雄,翡翠帳中鬬勇。男兒忿怒,挺身連刺黑纓鎗;女帥生嗔,拍胯著搖追命劍。一來一往,祿山會合太真妃;一撞一衝,君瑞追陪崔氏女。左右迎湊,天河織女遇牛郎;上下盤旋,仙洞嬌姿逢阮肇。鎗來牌架,崔郎相共薛瓊瓊;砲打刀迎,雙漸迸連蘇小小。一個鶯聲嚦嚦,猶如武則天遇敖曹;一個燕喘吁吁,好似審在逢呂雉。初戰時,知鎗亂刺,刺劍微迎。次後來,雙砲齊攻,膀脾夾湊。男兒氣急,使鎗只去扎心窩;女帥心忙,開口要來吞胸袋。一個使雙砲的,往來攻打內襠兵;一個輪膀脾的,上下夾迎臍下將。一個金雞獨立,高蹺玉腿弄精神;一個枯樹盤根,倒入翎花來刺牝。戰良久,朦朧星眼,但動些兒麻上來;鬬多時,款擺纖腰,再戰百愁挨不去。散毛洞主倒上橋,放水去淹軍;烏甲將軍虛點鎗,側身逃命走。臍膏落馬,須臾蹂踏肉為泥;溫緊粧呆,頃刻跌翻深澗底。大披掛,七零八斷,猶如急雨打殘花;錦套頭,力盡觔輸,恰似猛風飄敗葉。硫黃元帥,盔歪甲散走無門;銀甲將軍,守住老營還要命。」正是:

「愁雲托上九重天,  一塊敗兵連地滾。」

原來婦人有一件毛病,但凡交姤只要教漢子幹他後庭花。在下邊揉着,心子纔過。不然,隨問怎的,不得丟身子。就是韓道國與他相合,倒是後邊去的多,前邊一月,走不的兩三遭兒。第二件,積年好咂{髟巳}{髟八},把{髟巳}{髟八}常遠放在口裏,一夜他也無個足處。隨問怎的出了毧,禁不得他吮舔挑弄,登時就起。自這兩樁兒,可在西門慶心坎上。當日和他纏到起更,纔回家。婦人和西門慶說:「爹到明日再來早些,白日裏,咱破工夫,脫了衣裳,好生耍耍。」西門慶大喜。到次日,到了獅子街線舖里,就兌了四兩銀子與馮媽媽,討了丫頭使喚,改名叫做錦兒。西門慶想着這個甜頭兒,過了兩日,又騎馬來婦人家行走。原是棋童、玳安兩個跟隨。到了門首,就分付棋童,把馬回到獅子街房里去。那馮媽媽專一替他提壺打酒,街上買東西整理,通小慇懃兒,圖些油菜養口。西門慶來一遭,與婦人一二兩銀子盤纏。白日裡來,直到起更時分纔家去,瞞的家中鐵桶相似。馮媽媽每日在婦人這里打勤勞兒,往宅里也去的少了。李瓶兒使小廝叫了他兩三遍,只是不得閑。要便鎖着門去了一日。一日,小廝畫童兒撞見婆子,來家。李瓶兒說道:「媽媽子,成日影兒不見,幹的什麼貓兒頭差事?叫一遍,只是不在。通不來這里走走兒,忙的你恁樣兒的?丟下好些衣裳,帶孩子被褥,等你來幫着丫頭每折洗折洗,再不見來了。」婆子道:「我的奶奶,你倒說的且是好。寫字的拏逃軍,我如今一身故事兒哩!賣鹽的做雕鑾匠,我是那鹹人兒?」李瓶兒道:「媽媽子,你做了石佛室裡長老,請着你就是不閑。成日撰的錢,不知在那里?」婆子道:「老身大風刮了頰耳去了,嘴也趕不上在這里。撰什麼錢?你惱我,可知心裡急急的要來,再轉不到這里來。我也不知成日幹的什麼事兒哩!後邊大娘從那時與了銀子,教我門外頭替他稍個拜佛的蒲甸兒來。我只要忘了。昨日甫能想,賣蒲甸的賊蠻奴才又去了。我怎的回他?」李瓶兒道:「你還敢說,沒有他甸兒,你就信信拖拖跟了和尚去了罷了!他與了你銀子這一向,還不替他買將來。你這等裝憨打呆的!」婆子道:「等我沒也對大娘說去,就交與他這銀子去。昨日騎騾子,差些兒沒弔了他的。」李瓶兒道:「等你弔了他的,你死也!」這媽媽一直來到後邊,未曾入月娘房,先走在廚下打探子兒。只見玉簫和來興兒媳婦坐在一處。見了說道:「老馮來了!貴人,你在那里來?你六娘要把你肉也嚼下來,說影邊兒就不來了。」那婆子走到跟前,拜了兩拜,說道:「我纔到他前頭來,乞他聐聒了這一回來了。」玉簫道:「娘問你替他稍的蒲甸兒怎樣的?」婆子道:「昨日拏銀子到門外,賣蒲甸的賣了家去了。直到明年三月裡纔來哩。銀子我還拏在這里。姐你收了罷。」玉簫笑道:「怪媽媽子!你爹還在屋里兌銀子,等出去了,你還親交與他罷。」又道:「你且坐的。我問你,韓夥計送他女兒去了多少時了?也待將來。這一回來,你就造化了。他還謝你謝兒。」婆子道:「謝不謝,隨他了。他連今纔去了八日,也得盡頭,纔得來家。」不一時,西門慶兌出銀子與賁四,拏了庄子上去,就出去了。婆子走在上房,見了月娘,也沒敢拏出銀子來。只說:「蠻子有幾個粗甸子,都賣沒了回家。明年稍雙料好蒲甸來。」月娘是誠實的人,說道:「也罷,銀子你還收着。到明年,我只問你要兩個就是了。」與婆子幾個茶食吃了。後來到李瓶兒房里來。瓶兒因問:「你大娘沒罵你?」婆子道:「被我如此支吾,調的他喜歡了,倒與我些茶吃,賞了我兩個大餅定 ,出來了。」李瓶兒道:「還是昨日他往喬大戶家吃滿月的餅定 。媽媽子,不虧你這片嘴頭子,六月裡蚊子也釘死了!」又道:「你今日與我洗衣服,不去罷了。」婆子道:「你收拾討下漿,我明日蚤來罷。後晌時分,還要往一個熟主顧人家幹些勾當兒。」李瓶兒道:「你這老貨,偏有這些胡枝扯葉的!得你明日不來,我與你答話。」那婆子說笑了一回,脫身走了。李瓶兒留他:「你吃了飯去!」婆子道:「還飽着哩,不吃罷。」恐怕西門慶往王六兒家去,兩步做一步。正是:

「媒人婆地里小鬼,  兩頭來回抹油嘴;

一日走勾千千步,  只是苦了兩隻腿。」

畢竟未知後來如何,且聽下回分解:

