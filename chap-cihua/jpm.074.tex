%# -*- coding: utf-8 -*-
%!TEX encoding = UTF-8 Unicode
%!TEX TS-program = xelatex
% vim:ts=4:sw=4
%
% 以上设定默认使用 XeLaTex 编译,并指定 Unicode 编码,供 TeXShop 自动识别

%第七十四回 
\chapter{宋御史索求八仙壽\KG 吳月娘聽宣黃氏卷}


「昔年南去得娛賓,  願遜階前共好春,

螘泛羽觴蠻酒膩,  鳳啣瑤句蜀箋新;

花憐遊騎紅隨後,  草戀征軍碧繞輪,

別後清清鄭南路,  不知風月屬何人。」

話說西門慶摟抱潘金蓮,一覺睡到次日天明。婦人見他那話還直竪一條棍相似,,便道:「達,你將就饒了我罷,我來不得了,待我替你咂咂罷!」西門慶道:「怪小淫婦兒,你不若咂咂的過了,是你造化!」這婦人真個蹲向他腰間,按著他一隻腿,用口替他吮弄那話的。吮勾一個時分,精還不過,這西門慶用手按著粉項往來,只顧沒稜露腦,搖撼那話在口裏吞吐不絕,,抽拽的婦人口邊白沫橫流,殘脂在莖,精欲洩之際,婦人一面問西門慶:「二十八日,應二爹送了請帖來,請俺每,去不去?」西門慶道:「怎的不去?都收拾了去。」婦人道:「我有庄事兒央你,依不依?」西門慶道:「怪小淫婦兒,你有甚事說不是?」婦人道:「把李大姐那皮襖拿出來與我穿了罷,明日吃了酒回來,他們都穿著皮襖,只奴沒件兒穿。」西門慶道:「有年時王招宣府中當的皮襖,你穿就是了。」婦人道:「當的我不穿他。你與了李嬌兒去;把李嬌兒那皮襖卻與雪娥穿,我穿李大姐這皮襖,你今日拿出來與了我,我〈扌寨〉上兩個大紅遍地金鶴袖,襯著白綾襖兒穿,也是我與你做老婆一場,沒曾與了別人。」西門慶道:「賊小淫婦兒!單管愛小便益兒!他那件皮襖,值六十兩銀子哩,油般大黑蜂毛兒,你穿在身上是會搖擺。」婦人道:「怪奴才,你是與了張三、李四的老婆穿了?左右是你的老婆,替你裝門面的,沒的有這些聲兒氣兒的!好不好,我就不依了。」西門慶道:「你又求人,又做硬兒。」婦人道:「怪硶貨!我是你房里丫頭,在你跟前服軟!」一面說著,把那話放在粉臉上,只顧偎〈扌晃〉良久,又吞在口裡挑弄蛙口一回,又用舌尖底其琴絃,攪其龜稜,然後將朱唇裹著,只顧動動的。西門慶靈犀灌頂,滿腔春意透腦,良久精來,連聲呼:「小淫婦兒,好生裹緊著,我待過也…」言未絕,其精邈了婦人一口一面,口口接著多咽了。正時:

「自我內事迎郎意,  慇勤愛把紫簫吹。」

當日卻是安郎中擺酒,西門慶起來梳頭淨面出門。婦人還睡在被裏,便說道:「你趁閒尋尋兒出來罷。等一回你又不得閒了。」這西門慶于是走到李瓶兒房中,奶子、丫頭又早起來收拾乾淨、安頓下茶水伺候。見西門慶進來坐下,問養娘如意兒,這咱供養多時了?西門慶見如意兒穿著玉色對衿襖兒,白布裙子,葱白段子紗綠高底鞋兒,薄施朱粉,長畫蛾眉,油胭脂搽的嘴唇鮮紅的。耳邊帶著兩個金丁香兒,手上帶著李瓶兒與他四個烏金戒指兒,笑嘻嘻遞了茶,在旁邊說話兒。西門慶一面使迎春往後邊討牀房裏鑰匙去。那如意兒便問:「爹討來做什麼?」西門慶道:「我要尋皮襖與你五娘穿。」如意道:「是娘的那貂鼠皮襖?」西門慶道:「就是,他要穿穿,拿與他罷。」迎春去了,把老婆就摟在懷裡,兩手就舒在胸前,摸他奶頭,說道:「我兒,你雖然生養了孩子,奶頭兒到還恁緊。」就兩個臉對臉兒親嘴,且咂舌頭做一處。如意兒道:「我見爹常在五娘身邊,沒見爹往別的房里去,他老人家別的罷了,只是心多容不的人。前日爹不在,為了棒槌,好不和我大嚷了一場,多虧韓嫂兒和三娘來勸開了。落後爹來家,也沒敢和爹說。不知什麼多嘴的人對他說,又說爹要了我。他也告爹來不曾?」西門慶道:「他也告我來,你到明日替他陪個禮兒便了。他是恁行貨子,受不的人個甜棗兒,就喜歡的!」如意兒道:「五娘嘴頭子雖利害,倒也沒什麼心!前日我和他嚷了,第二日爹到家,就和我說好話,說爹在他身邊偏的多,就是別的娘多讓我幾分。你凡事只有個不瞞我,我放著河水不洗船?好做惡人?」西門慶道:「既是如此,大家取和些。」又許下老婆:「你每晚夕等我來這房裏睡。」如意道:「爹真個來?休哄俺每著!」西門慶道:「誰哄你來?」正說著,只見迎春取鑰匙來了。西門慶教開了牀房門,又開櫥櫃,拿出那皮襖來抖了抖,還用包袱包了,教迎春拿到那邊房裡去。如意兒悄悄向西門慶說:「我沒件好皮襖兒,你趁著手兒,再尋出來與了我罷。有娘小衣裳兒,再與我一件兒。」西門慶連忙就教他開箱子,尋出一套翠藍緞子襖兒,黃綿紬裙子;又是一件藍潞紬綿褲兒;又是一雙粧花膝褲腿兒,與了他。老婆磕頭謝了。西門慶鎖上門去了,就使送皮襖與金蓮房裡來。金蓮纔起來,在牀上裹腳,只見春梅說:「如意兒送皮襖來了。」婦人便知其意,說道:「你教他進來。」問道:「爹使你來?」如意道:「是爹教我送來與娘穿。」金蓮道:「也與了你些什麼兒沒有?」如意道:「爹賞了我兩件紬絹衣裳年下穿,教我來與娘磕頭。」于是向前磕了四個頭。婦人道:「姐姐們,這般不好!你主子既愛你,常言:『船多不礙港,車多不礙路。』那個好做惡人?你只不犯著我,我管你怎的!我這裡還多著個影兒哩!」如意兒道:「俺娘已是沒了,雖是後邊大娘承攬,娘在前邊還是主兒,早晚望娘擡舉。小媳婦敢欺心,那裡是葉落歸根之處?」婦人道:「你這衣服,少不得還對你大娘說聲是的。」如意道:「小的前者也問大娘討來,大娘說,等爹閒時拿兩件與你。」婦人道:「既說知罷了。」這如意就出來,還到那邊房裡。西門慶是往前廳去了。如意便問迎春:「你頭裡取鑰匙去,大娘怎的說?」迎春說:「大娘問,你爹要鑰匙做什麼?我也沒說拿皮襖與五娘,只說我不知道。大娘沒言語。」卻說西門慶走到廳上看著設席擺列,海鹽子弟張美、徐順、苟子孝、生旦都挑戲箱到了。李銘等四名小優兒,又早來伺候,都磕頭見了。西門慶分付打發飯與眾人吃。分付李銘三個在前邊唱,左順後邊答應堂客。那日,韓道國娘子王六兒沒來,打發申二姐買了兩盒禮物,坐轎子,他家進財兒跟著,也來與玉樓做生日。王經送到後邊,打發轎子出去了。那日門前韓大娘、孟大妗子都到了,又是傅夥計、甘夥計娘子、崔本媳婦兒,段大姐并賁四娘子。西門慶正在廳上,看見夾道內,玳安領著那個五短身子,穿綠段襖兒紅裙子,勤著藍金綃箍兒,不搽胭粉,兩個密縫眼兒,一似鄭愛香模樣,便問:「是誰?」玳安道:「是賁四嫂。」西門慶就沒言語。往後見了月娘,月娘擺茶。西門慶進來吃粥,遞與月娘鑰匙。月娘道:「你開門做什麼?」西門慶道:「六姐他說明日往應二哥家吃酒沒皮襖,要李大姐那皮襖穿。」被月娘瞅了一眼,說道:「你自家把不住自家嘴頭了。他死了,嗔人分散房裡丫頭;相你這等,就沒的話兒說了。他見放皮襖不穿,巴巴兒只要這皮襖穿,早時他死了,你只望這皮襖。他不死,你只要好看一眼兒罷了!」幾句話得西門慶閉口無言。忽報李學官來還銀子,西門慶出去,陪坐在廳上說話。只見玳安拿進帖兒說:「王招宣府送禮來了。」西門慶問:「是什麼禮?」玳安道:「是賀禮。一疋尺頭,一罈南酒 ,四樣下飯。」西門慶看帖兒上,寫著:「眷晚生王寀頓首拜。」西門慶即便叫王經拿著眷生回帖兒謝了。賞了來人五錢銀子,打發出了門。只見李桂姐門首下轎,保兒挑四方盒禮物,慌的玳安替他抱毡包,說道:「桂姨打夾道內進去罷,廳上有劉學官坐在哩。」那桂姨即向夾道內進裡邊去。來安兒把盒子挑進月娘房裡去。月娘道:「爹看見來不曾?」玳安道:「爹陪著客,還不見哩。」月娘便說道:「連盒放在明間內。」一回,客去了,西門慶進來吃飯。月娘道:「李桂姐送禮在這裡。西門慶道:「我不知道。」月娘令小玉揭開盒兒,見一盒果餡壽糕,一盒玫瑰八仙糕 ,兩隻燒鴨 ,一副豕蹄。只見桂姐從房內出來,滿頭珠翠,勒著白挑線汗巾,大紅對衿襖兒,藍段裙子,望著西門慶磕了四個頭。西門慶道:「罷了,又買這禮來做什麼?」月娘道:「剛纔桂姐對我說,怕你惱他。不干他事。說起來都是他媽的不是。那日桂姐害頭疼來,只見這王三官領著一行人,往秦玉芝兒家請秦玉芝兒。打門首過,進來吃茶,就被人進來驚散了。桂姐也沒出來見他。」西門慶道:「那一遭是沒出來見他,這一遭又是沒出來見他,自家也說不過。論起來我也難管。你這麗春院拿燒餅砌著門不成?到處幾錢兒,都是一樣,我也不惱!」那桂姐跪在地下,只顧不起來說道:「爹惱的是。我若和他沾沾,身子就爛化了,一個毛孔兒裡生個天疱瘡!都是俺媽空老了一片皮,幹的營生,沒個主意,好的也招惹,歹的也招惹來家,平白教爹惹惱!」月娘道:「你既來了說開就是了,又惱怎的?」西門慶道:「你起來,我不惱你便了。」那桂姐故作嬌張致,說道:「爹笑一笑兒,我纔起來;你不笑,我就跪一年也不起來。」不妨潘金蓮在傍插口道:「桂姐,你起來,只顧跪著他,求告他黃米頭兒,教他張致!如今在這裡你便跪著他,明日到你家他卻跪著你;你那時別要理他。」把西門慶、月娘多笑了,桂姐纔起了來。只見玳安慌慌張張來報:「宋老爹和安老爹來了。」這西門慶便教拿衣服穿了,出去迎接去了。桂姐向月娘說道:「爹,嚛嚛!從今後我也不要爹了,只與娘做女兒罷。」月娘道:你虛頭愿心,說過道過罷了。前日兩遭往裡頭去,沒在那里?」桂姐道:「天麼,天麼!可是殺人!爹沒往我家里,若是到我家,見爹一面,沾沾身子兒,就促死了我,渾身生天疱瘡!娘你錯打聽了,敢不是我那裡,多往鄭月兒家走走兩遭,請了他家小粉頭子了。我道一篇是非,就是他氣不憤架的;不然爹如何惱我?」金蓮道:「各人衣飯,他平白怎麼架你是非?」桂姐道:「五娘,你不知,俺每這里邊人,一個氣不憤一個,好不生!」月娘接過來道:「你每裡邊與外邊,怎的打偏別?也是一般,一個不憤一個。那一個有些時道兒,就要躧下去。」月娘擺茶與他吃,不在話下。卻說西門慶迎接宋御史、安郎中到廳上敍禮,每人一疋段子,一部書,奉賀西門慶。見了卓席齊整,甚是稱謝不盡。一面分賓主坐下,叫上戲子來參見。分付:「等蔡老爹到,用心扮演。」不一時吃了茶,宋御史道:「學生有一事奉凟四泉,今有巡撫侯石泉老先生,新陞太常卿,學生同兩司作東,二十九日借尊府,置杯酒奉餞,初二日就起行上京去了,未審四泉允諾否?」西門慶道:「老先生分付,敢不從命。但未知多少卓席?」宋御史道:「學生有分資在此。」即喚吏上來,毡包內取出布按兩司連他共十二封分資來,每人一兩,共十二兩銀子。要一張大插卓,餘者六卓都是散卓,叫一起戲子。西門慶答應收了,宋御史又下席作揖致謝。少頃,請去捲棚聚景堂那里坐的。不一時,鈔關錢主事也到了。三員官會在一處換了茶,擺棋子下棋。安御史見西門慶堂廡寬廣,院中幽深,書畫文物,極一時之盛。又見挂著一幅陽捧日橫批古畫,正面螺鈿屏風,屏風前安者一座八仙捧壽的流金鼎,約數尺高,甚是做得奇巧,見爐內焚著沉檀香,煙從龜、鶴、鹿口中吐出,只顧近前觀看,誇獎不已。問西門慶:「這付爐鼎造得好。」因向二官說:「我學生寫書與淮安劉年兄那裡,替我梢帶這一付來送蔡老生,還不見到。四泉不知是那里得來的?」西門慶道:「也是淮上一個人送學生的。」說畢,下棋。西門慶分付下邊,看了兩個卓盒,細巧菜蔬,菓餡點心上來,一面叫生旦在上唱南曲。宋御史道:「客尚未到,主人先吃得面紅,說不通。」安郎中道:「天寒飲一杯無礙。」原來宋御史已差公人船上邀蔡知府去了。近午時分,來人回報:「邀請了,在磚廠黃老爹那里下棋,便來也。」宋御史令起去伺候。一個下棋飲酒。安郎中喚戲子:「你每唱個宜春令奉酒。」于是貼旦唱道:

「第一來為壓驚,第二來因謝誠。殺羊茶飯,來時早已安排定。斷行人,不會親僯;請先生,和俺鶯娘匹娉。我只見他,歡天喜地,道謹依來命。」

〔五供養〕「來回顧影,文魔秀士欠酸丁。下工夫將頭顱來整,遲和疾擦倒蒼蠅。光油油輝花人眼睛,酸溜溜螫得牙根冷。天生這個後生,天生這個俊英!」

〔玉降鶯〕「今宵歡慶,我鶯娘何曾慣經。你須索要欵欵輕輕,燈兒下共交鴛頸。端祥可憎,誰無志誠。恁兩人今夜親折證,謝芳卿。感紅娘錯愛,成就了這姻親。」

〔解三醒〕「玳筵開,香焚寶鼎,綉簾外,風掃閑庭。落紅滿地胭脂冷,碧玉欄杆花弄影。准備鴛鴦夜月銷金帳,孔雀春風軟玉屏。合歡令,更有那鳳簫象板錦瑟鷥笙。」生唱:「可憐我書劍飄零無厚聘,感不盡姻親事有成。新婚燕爾安排定,除非是折桂手報答前程;我如今博得個跨鳳乘鸞客,到晚來臥看牽牛織女星。非僥倖,受用的珠圍翠繞,結果了黃卷青燈。」

〔尾聲〕「老夫人專意等。」生唱:「常言道恭敬不如從命。」紅唱:「休使紅娘再來請。」唱畢,忽吏進報:「蔡老爹和黃老爹來了。」宋御史忙令收了卓席,各整衣冠,出來迎接。蔡九知府穿素服金帶,跟著許多吏書先令人投一侍生蔡修拜帖與西門慶,進廳上。安郎中道:「此是主人西門大人,見在本處作千兵,也是京中老先生門下。」那蔡知府又作揖,稱道:「久仰,久仰!」西門慶亦道:「客當奉拜。」敍禮畢,各寬衣服坐下,左右上了茶,各人扳話。良久,就上座,西門慶令小優先在傍彈唱。蔡九知府居上,主位四坐。廚役割道湯飯,戲子呈遞手本,蔡九知府揀了雙忠記,演了兩摺,酒過數巡,宋御史令生旦上來遞酒。小優兒席前這套新水令『玉驄轎馬出皇都。』蔡知府笑道:「拙原直得多少,可謂御史青驄馬三公,乃劉郎舊索髯。」安郎中道:「今日更不道江州司馬青衫濕。」言罷,眾人都笑了。西門慶又令春鴻唱了一套『金門獻罷平胡表』,把宋御史喜歡的要不的。因向西門慶道:「此子可愛!」西門慶道:「此是小价,原是揚州人。」宋御史攜著他手兒,教他的遞酒,賞了他三錢銀子,磕頭謝了。正是:

「窗外日光彈指過,  席前花影坐間移;

一杯未盡笙歌迭,  階下申牌又報時。」

不覺日色沉西,蔡九知府見天色晚了,即令左右穿衣告辭,眾位欵留不住,俱送出大門而去,隨即差了兩名吏典,把卓席羊酒尺頭擡送到新河口下處去訖不題。宋御史于是亦作辭西門慶,因說道:「今日且不謝,後日還要取擾。」各上轎而去。西門慶送了回來,打發了戲子,分付:「後日原是你們來,再唱一日,叫幾個會唱的來,宋老爹請巡撫侯爺哩。」戲了道:「小的知道了。」西門慶令攢上酒卓,使玳安:「去請溫相公來坐坐。」再教來安兒:「去請應二爹去。」不一時,次第而至,各行禮坐下。三個小優兒在傍彈唱,把酒來斟。說鄭金、左順在後邊堂客席前。西門慶又問伯爵:「你娘們明日都去,你叫唱的?是雜耍的?」伯爵道:「哥到說得好,小人家那里擡放。將就叫了兩個唱女兒唱罷了。明日早些,請眾娘嫂子下降。」這里前廳吃酒,唱了一日。孟大姨與孟二妗子先起身去了。落後楊姑娘也要去,月娘道:「姑奶奶,你再住一日兒家去不是?薛姑子使他徒弟取了卷來,咱晚夕教他宣卷咱們聽。」楊姑娘道:「老身實和姐姐說,要不是我也住。明日俺們外弟二個侄兒定親事,使孩子來請我,我要瞧瞧去。」于是作辭而去。只有傅夥計、甘夥計娘子與賁四娘子、段大姐、月娘還留在上房陪大妗子、潘姥姥、李桂姐、申二姐、郁大姐在傍,一遞一套彈唱,兩個小優兒都打發在前邊來了。又吃至掌燈已後,三位夥計娘子,都作辭去了。止段大姐沒去,在後邊雪娥房中歇了。潘姥姥往金蓮房內去了。只有大妗子、李桂姐、申二姐和三個姑子、郁大姐和李嬌兒、孟玉樓、潘金蓮,在月娘房內坐的。忽聽前邊西門慶散了,小廝收進家活來。這金蓮慌忙抽身就往前走了,到前邊黑影兒裡,悄悄立在角門首。只見西門慶扶著來安兒打著燈,趔【走息】着脚兒,就往李瓶兒那邊走,看見金蓮在門首立著,拉了手進入房來。那來安兒便往上房教鍾筯。月娘只說西門慶進來,把申二姐、李大姐、郁大姐都打發往李嬌兒房內去了。問來安道:「你爹來沒有?在前邊做什麼?」來安道:「爹在五娘房里去了的不耐煩了!」月娘聽了,心內就有些惱,因向玉樓道:「你向恁沒來頭的行貨子!我說他今日進來往你房裡去,如何三不知又摸到他那屋裡去了?這兩日又浪風發起來,只在他前邊纏!」玉樓道:「姐姐隨他纏去,恰似咱每把這件事放在頭里爭他的一般!于是大師父說笑話兒的來頭,左右這六房裡由他串到。他爹心中所欲,你我管的他?」月娘道:「乾淨他有了話?剛纔聽見前頭散了,就慌的奔命的往前走了。」因問小玉:「灶上沒人了,與我把儀門拴上了罷。後邊請三位師父來,咱每且聽他宣一回卷著。」又把李大姐、申二姐、段大姐、郁大姐都請了來。月娘問大妗子道:「我頭裡旋叫他使小沙彌請了黃氏女卷來宣,今日可可兒楊姑娘已去了。」分付玉筲頓下好茶。玉樓對李嬌兒說:「咱兩家子輪替管茶,休要只顧累了大姐姐這屋裡。」于是各往房裡分付預備茶去。不一時,放下炕卓兒,三個姑子來到,盤膝坐在炕上。眾人俱各坐了,擠了一屋裡人,聽他宣卷。月娘洗手炷了香。這薛姑子展開黃氏女卷,高聲演說道:

「盖聞法初不滅,故歸空;道本無生,每因生而不用。由法身以垂入相,由入相以顯法身。朗朗惠燈,通開世戶;明明佛鏡,照破昏衢。百年景賴剎那間,四大幻身如泡影。每日塵勞碌碌,終朝業試忙忙。豈知一性圓明,徒逞六根貪慾。功名盖世,無非大夢一場;富貴驚人,難免無常二字。風火散時無老少,溪山磨盡幾英雄。我好十方傳句偈,八部會壇場。救大宅之烝熬,發空門之龠綸。」

偈曰:

「富貴貧窮各有由,只綠分定不須求。未曾下的春時種,空手荒田望有秋。」眾菩薩每,聽我貧僧演說佛法,道四句偈子,乃是老祖留下。如何說「富貴貧窮各有由?」相如今你道眾菩薩嫁得官人,高官厚祿,在這深宅大院,呼奴使婢,插金帶銀。在綾錦窩中長大,綺羅堆裡生成,思衣而綾錦千箱,思食而珍羞百味,享榮華,受富貴,盡皆是你前世因由,根基上有你的一般大緣分,不待求而自得。就是貧僧在此宣經念佛,也是吃著這美口茶飯,受著發心布施老人緣分,非同小可。都是龍華一會上的人,皆是前生修下的功果。你不修下時,就如春天不種下場,到了秋成時候,一片荒田,那成熟結子從那里來?正是:

「淨地靈臺好下工,得意歡喜不放鬆;五濁六根爭洗淨,參透玄門見家風。」

又:

「百歲光陰瞬息回,此身必定化飛灰;誰人肯向生前悟,悟卻無生歸去來。」

又:

人命無常呼吸間,眼觀紅日墜西山。寶山歷盡空回首,一失人身萬劫難。想這富貴榮華,如湯潑雪。仔細算來,一件無多做了虛花驚夢。我今得個人身,心中煩惱悲切。死後四大化作塵土,又不知這點靈魂往何處受苦去也。懼怕生死輪迴,往前再參一步。

唱:

〔一封書〕「生和死兩下相,嘆浮生終日忙。男和女滿堂,到無常祇自當。人如春夢終須短,命苦風燈不久常。自思量,可悲傷,題起教人欲斷腸。」開卷曰:應身長救苦,并本無去亦無來。彌陀教主大願弘深,四十八願度眾生,使人人悟本性。彌陀今惟心淨主渡苦海,苦海洪波,證菩提之妙果。持念者罪滅河沙,稱揚者福增無量。書寫讀誦者,當生華藏之天。見聞受持,臨命纔時定往西方淨土。凡念佛者斷有功,無量慈愍故慈愍大,慈愍故,皈命一切佛法僧信,禮常住三寶法輪,常輪度眾生。

偈曰:

「無上甚深做妙法,百千萬劫難遭遇。我今見聞得受持,願解如來真實意。黃氏寶卷纔展開,諸佛菩薩降臨來。爐香遍滿虛空界,佛號聲名動九垓。」昔日漢王治世,雨順風調,國泰民安。感到一位善心娘子出世。家住曹州南華縣黃員外所生一女,端嚴美色,年方七歲,吃齋把素,念金剛經報答父母深恩,每日不缺。感得觀世音菩薩半空中化魂。父母見他終日念經,若切不從。一日尋媒,吉日良時,把他嫁與一婿,姓趙名方,屠宰為生。為夫婦一十二載,生下一男二女。一日黃氏告其夫曰:「我與你為夫妻一十二載,生下嬌兒嬌女,但貪戀恩愛,永墮沉淪。妾有小詞,勸喻丈夫聽取。」詞曰:「宿緣夫婦得成雙,雖有男和女,誰會抵無常?伏望我夫主,定念與雙同,共修行終年富貴也。須草草貪名與利,隨分度時光。」這趙郎見詞,不能依隨。一日作別起身,往山東買豬去。黃氏女見丈夫去了,每日淨房寢歇,沐浴身體,燒香禮誦金剛經。

今方當下山東去,四個兒女在中堂。黃氏女在西房,香湯沐浴換衣裳。卸簪珥淺淡梳粧,每日家向西方,燒香禮拜,面念顏并寶卷,持念金剛。看經文猶未了,香烟沖散。念佛音聲朗朗,貫徹穹蒼。地獄門天堂界,豪光發現。閻羅王一見了,喜悅龍顏。莫不是陽世間,生下佛祖。急宣召二鬼判,審問端詳。有鬼判告吾王:「聆音察理,曹州府南華縣,有一善良、看經文黃氏女,持齋把素。行善心功行大,驚動天堂。」唱金剛經:「閰羅王聞言心內忙,急點無常鬼一雙。一雙急趙家庄。黃氏正看經卷,忽見仙童在面前。」「善人便是童子請,惡人須遣夜叉郎。」黃氏看經忙來問:「誰家童子到奴行?」仙童答告娘子道:「善心娘子你莫慌。不是凡間親眷屬,我是陰間童子郎。今因為你看經卷,閻王請你善心娘。」黃氏見說心煩惱,小心一一告無常:「同姓同名勾一個,如何勾我見閻王。千死萬死甘心死,怎捨嬌娃女一雙,大姐嬌姑方九歲,伴嬌六歲怎拋娘。長壽嬌兒年三歲,常抱懷中心怎忘。苦放奴家魂一命,多將功德與你行。」仙童答告娘子道:「何人似你念金剛?」

善惡二童子,被黃氏女哀告,再三不肯赴幽,留戀一二個孩兒,難拋難捨。仙童催促說道:

「善心娘子,陰間取你三更死,定不容情到四更。不比你陽間好轉限,陰司取你,若違了限,我得罪,更不輕說短長。」黃氏此時心意想,便喚女使去燒湯。香湯沐浴方纔了,將身便乃入佛堂。盤膝坐定不言語,一靈真性見閻王。」唱:

〔楚江秋〕「人生夢一場,光陰不久常。臨危個個是風燈樣,看看回步見閻王。急辦行粧,鄉臺上把家鄉望,兒啼女哭好恓惶。排鉸打鼓作道場,披麻帶孝安塋葬。」白:「不說令方恓惶事,且言黃氏赴陰靈。看看來到奈何岸,一道金橋接路行。借問此橋作何用?單等看經念佛人。奈何兩邊血浪水,河中多少罪淹魂。悲聲哭泣紛紛鬧,四面毒蛇咬露筋。前到破錢山一座,黃氏向前問原因。是你陽間人化紙,殘燒未了便拋焚。因此挑翻多破碎,積聚號作破錢山。又打枉死城下過,多少孤魂未托生。黃氏見說心慈愍,舉口便誦金剛經。河裡罪人多開眼,尸山爐剔樹騫林。鑊湯火池蓮花現,無間地徹瑞雲籠。當下仙童忙不住,急忙便去奏閻君。」唱:

〔山坡羊〕「黃氏到了那森羅寶殿,有童子先奏說,請了看經人來見。閻羅王便傳召請,黃氏拜在金階下,不由的跪在面前。有閻君問你,從幾年把金剛經念起?何年月日感得觀世音出現?這黃女又手訴說前情來訶,自從七歲吃齋,供養聖賢。望上聖聽言,從嫁了兒夫,看經心不減。」白:

「閽君當下忙傳旨,善心娘子你聽因。你念金剛多少字?凡多點化接陰陰。甚字起頭甚字落?是何兩字在中間?你若念經無差錯,放你還魂回世間。黃氏當時階下立,願王聽奴念金剛。字有五千四十九,八萬四千點畫行。如字起頭行字住,荷擔兩字在中央。黃氏說經尤未了,閻王殿前放毫光。舉手龍顏真喜悅,放你還魂看世間。黃氏聞知忙便告,願王俯就聽奴言。第一不往屠家去,第二不要染衣行。只願作個善門子,看經念佛過時光。閻王取筆忙判斷,曹州張家轉為男。他家積有家財廣,缺少墳前拜孝郎。員外夫妻俱修善,姓名四海廣傳揚。吃罷迷魂湯一盞,張家娘子腹懷躭。十月滿足生一子,左肋紅字有兩行:「此是看經黃氏女,曾嫁觀水趙令方。此是看經多因果,得為男子壽延長。」張家員外親看見,愛如璽寶喜開顏。」唱:

〔皂羅袍〕「黃氏在張家托化轉男身。相湊無差,員外見了喜添花。三年就養成人。大年方七歲,聰明秀發。攻書習字,取名俊達,十八歲科舉登黃甲。」

「卻說張俊達十八歲登科應舉,陞授曹州南華縣知縣。忽然思憶是他本鄉,到縣中赴任之後,先去王糧國稅,然後理論公廳。差兩個公差,即去請趙郎令方,我和他說話。兩個公差不敢怠慢,即到趙家來請令方。」白:

「趙令方在家中,看經念佛。兩公人,忙喎喏聽說來因。即時間,忙打扮,來到縣裡。公廳上,忙施禮,且說家門。張知縣,起躬身,便令坐。敍寒溫,分賓主,捧出茶湯。你是我親夫主,令方姓趙。我是你前妻子,黃氏之身。你不信,到靜臺,脫衣親見,左肋下,硃砂記,字寫原因。我大女,嬌姑兒,嫁人去了。第二女,伴姐姐,嫁了曹真。長壽兒,我掛牽,守我墳塋。咱兩個,同騎馬,前到先塋。」

知縣同令方兒女五人,到黃氏墳前開棺,見屍容顏不動。回來做道場七日,令方看金剛經,瑞雪紛紛,男女五人,總賀祥雲昇天去了。臨江仙一首為證:

「『黃氏看經成正果,同日登極樂。五口盡昇天,道善人傳觀音菩薩未度我。』」

「寶卷已終,佛聖已知。法界有情,同生勝會。南無一乘字無量,又真空諸佛海會悉遙聞,普使河沙同淨土。伏願經聲佛號,上徹天堂,下透地府。念佛者出離苦海,作惡者永墮沉淪。得悟者諸佛引路,放光明照徹十方。東西下,〈廴回〉光返照。南北處,親到家鄉。登無生漂舟到岸,小孩兒得見親娘。入母胎三實不怕,八十部永返安康。」偈曰:

「眾等所造諸惡業,  自始無始至如今。

靈山失散迷真性,  一點靈光串四生。

一報天地盖載恩,  二報日月照臨恩。

三報皇天水土恩,  四報爹娘養育恩。

五報祖師親傳法,  六報十類孤魂早超身。」

(摩訶般若波羅密)

薛姑子宣畢卷,已有二更天氣。先是李嬌兒房內元宵兒拿了一道茶了,眾人吃了。後孟玉樓房中蘭香拿了幾樣精製果菜,一坐壺酒來,又頓了一大壺好茶,與大妗子、段大姐、桂姐眾人吃。月娘又教玉簫拿出四盒兒細茶食餅糖之類,與三位師父點茶。李桂姐道:「三位師父宣了這一回卷,也該我唱個曲兒孝順。」月娘道:「桂姐,又起動你唱。」郁大姐道:「等我先唱道。」月娘道:「也罷,郁大姐先唱。」申二姐道:「等姐姐唱了,等我也唱個兒與娘們聽。」問月娘:「要聽什麼?」月娘道:「你唱『更深夜深靜峭』。」當下桂姐送眾人酒,取過琵琶來,輕舒玉笋,欵跨鮫綃,啟朱唇露皓齒,唱道:

「更深靜峭,把被兒熏了。看看等到月上花稍,全靜悄悄,全無消耗。敲殘了更鼓,你便纔來到。見我這臉兒不瞧,來跪在奴身邊告。我做意兒瞧,他偷眼兒瞧,甫能咬定牙,其實忍不住笑。」

又:

「勤兒推磨,好似飛蛾援火。他將我做啞謎兒包籠,我手裡登時猜破。近新來把不住船兒舵,特故里搬弄心腸軟,一似酥蜜果。者麼是誰,休道是我。便做鐵打人,其實難不過。」

又:

「疎狂{弋心}煞,薄情無奈,兩三夜不見你回來。問著他便撒頑不睬,不由人轉尋思權寧耐。他笑吟吟將被兒錦開,半掩過香羅待。我推綉鞋不去睬。你若是惱的人,慌只教氣得我害。」

又:

「花街柳市,你戀著蜂蝶採。使我這里玉潔冰清,你那里瓜甜蜜柿。恰回來無酒半裝醉,只顧里打草驚蛇,到尋找些風流罪。我欲待撾了你面皮,又恐傷了就里待。要隨順了他,其實受不的你氣。」

桂姐唱畢,郁大姐就纔要接琵琶,被申二姐要過去了。挂在胳膊上,先說道:「我唱個十二月兒掛真兒與大妗子和娘每聽罷。」于是唱道:「正月十五鬧元宵,滿把焚香天地也燒。」一套。唱畢,月娘笑道:「慢慢兒的說,左右夜長儘著你說。」那時大妗子害夜深困的慌,也沒等的郁大姐唱,吃了茶多散歸各房內睡去了。桂姐便歸李嬌兒房內,段大姐便往孟玉樓房中,三位師父便往孫雪娥後邊房裡睡。郁大姐、申二姐與玉簫、小玉在那邊炕屋裡睡。月娘同大妗子在上房內睡。俱不在話下。正是:

「參橫斗轉三更後,  一鈎斜月到紗窗。」

畢竟未知後來如何,且聽下回分解:


