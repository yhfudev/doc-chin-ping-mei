%# -*- coding: utf-8 -*-
%!TEX encoding = UTF-8 Unicode
%!TEX TS-program = xelatex
% vim:ts=4:sw=4
%
% 以上设定默认使用 XeLaTex 编译,并指定 Unicode 编码,供 TeXShop 自动识别

%第十七回 
\chapter{宇給事劾倒楊提督\KG 李瓶兒招贅蔣竹山}

\begin{showcontents}{}



「記得書齋乍會時,  雲蹤雨跡少人知,

晚來鸞鳳棲雙枕,  剔盡銀燈半吐輝;

思往事,夢魂迷,  今宵幸得效于飛。」

話說五月二十日,帥府周守備生日,西門慶那日封五星分資,兩方手帕,打選衣帽齊整,騎著大白馬,四個小廝跟隨,往他家拜壽。席間也有夏提刑、張團練、荊千戶、賀千戶一般武官兒飲酒,鼓樂迎接,搬演戲文,只是四個唱的遞酒。玳安接了衣裳,回馬來家。到日西時分,又騎馬接去。走到西街口上,撞見馮媽媽。問道:「馮媽媽那裡去?」馮媽媽道:「你二娘使我來請你爹來。顧銀匠整理頭面完備,今日拿盒送來,請你爹那裡瞧去。你二娘還和你爹說話哩。」玳安道:「俺爹今日都在守備府周老爹吃酒。我如今接去,你老人家回罷,等我到那裡對爹說就是了。」馮媽媽道:「累你好歹說聲,你二娘等著哩。」這玳安打馬逕到守備府。眾官員正飲酒在熱鬧處,玳安走到西門慶席前說道:「小的回馬家來時,在街口撞遇馮媽媽,二娘使了來說,顧銀匠送丫頭面來了,請爹瞧去;還要和爹說話哩。」西門慶聽了,拿了些點心湯飯與玳安吃了,就要起身。那周守備那裡肯放,攔門拿巨杯相勸。西門慶道:「蒙大人見賜,寧可飲一杯。還有些小事,不能盡情,恕罪恕罪!」于是一飲而盡,作辭周守備上馬,逕到李瓶兒家。婦人接著,茶湯畢,西門慶分付玳安回馬家去,明日來接。玳安去了,李瓶兒叫迎春盒兒取出頭面來,與西門慶過目,黃烘烘火焰般一付好頭面,收過去,單等二十四日行禮,出月初四日准娶。婦人滿心歡喜,連忙安排酒來,和西門慶暢飲開懷。吃了一回,使丫鬟房中搽抹涼蓆乾淨,兩個在紗帳之中,香焚蘭麝,衾展鮫綃,脫去衣裳,並肩疊股,飲酒調笑。良久,春色橫眉,淫心蕩漾,西門慶先和婦人雲雨一回,然後乘著酒興坐于床上,令婦人橫躺於袵席之上,與他品蕭。但見:

「紗帳香飄蘭麝,  蛾眉輕把簫吹,

雪白玉體透簾幃,  禁不住魂飛魄颺;

一點櫻桃小口,  兩隻手賽柔荑,

才郎情動囑奴知,  不覺靈犀味美。」

西門慶于是醉中戲問婦人:「當初有你花子虛在時,也和他幹此事不幹?」婦人道:「他逐日睡生夢死,奴那裡耐煩和他幹這營生!他每日只在外邊胡撞,就來家,奴等閒也不和他沾身。況且老公公在時,和他另在一間房睡著。我還把他罵的狗血噴了頭,好不好對老公公說了,要打白棍兒,也不算人。甚麼材料兒,奴與他這般頑耍,可不砢硶殺奴罷了!誰似冤家這般可奴之意,就是醫奴的藥一般。白日黑夜,教奴只是想你。」兩個耍一回,又幹了一回。傍邊迎春伺侯下一個小方盒,都是各樣細巧果仁、肉心雞鵝腰掌 、梅桂菊花餅兒 。小金壺內,滿泛瓊漿。從黃昏掌上燈燭,且幹且飲,直耍到一更時分。只聽外邊一片聲打的大門響,使馮媽媽開門瞧去,原來是玳安來了。西門慶道:「我分付明日來接我,這咱晚又來做甚麼?」因叫進房來問他。那小廝慌慌張張走到房門首,西門慶與婦人睡著,又不敢進來,只在簾外說話,說道:「姐姐、姐夫都搬來了。許多箱籠在家中,大娘使我來請爹快去計較話哩。」這西門慶聽了,只顧猶豫:「這咱晚端的有甚緣故?須得到家瞧瞧。」連忙起來,婦人打發穿上衣服,做了一盞暖酒與他吃,打馬一直來家。只見後堂中,秉著燈燭,女兒、女婿都來了,堆著許多箱籠牀帳家活,先吃了一驚。因問:「怎的這咱來家?」女婿陳經濟磕了頭,哭說:「近日朝中俺楊老爺被科道官參論倒了。聖旨下來,拿送南牢問罪。門下親族用事人等,都問擬枷號充軍。昨日府中楊幹辦連夜奔走,透報與父親知道,父親慌了,教兒子同大姐和些家活箱籠,就且暫在爹家中寄放,躲避些時。他便起身往東京我姑娘那裡,打聽消息去了。待的事寧之日,恩有重報,不敢有忘。」西門慶問:「你爹有書沒有?」陳經濟道:「有書在此。」向袖中取出,遞與西門慶拆開觀看。上面寫道:

「眷生陳洪頓首書奉大德西門親家見字。餘情不敍。茲因北虜犯邊搶過雄州地界,兵部王尚書不發人馬,失誤軍機,連累朝中楊老爺俱被科道官參劾太重。聖旨惱怒,拿下南牢監禁,會同三法司審問。其門下親族用事人等,俱照例發邊衛充軍。生一聞消息,舉家驚惶,無處可投。先打發小兒、令愛,隨身箱籠家活,暫借親家府上寄寓。先即上京投在家姐夫張世廉處打聽示下。待事務寧帖之日回家,恩有重報,不不敢有忘。誠恐縣中有甚聲色,生令小兒另外具銀五百兩,相煩親家費心處料。容當叩報沒齒不忘。燈下草草不宜。

仲夏二十日,洪再拜。」

西門慶看了,慌了手腳。教吳月娘安排酒飯,管待女兒、女婿。就令家下人等,打婦廳前東廂房三間,與他兩口兒居住。把箱籠細軟,都收拾月娘上房來。陳經濟取出他那五百兩銀,交與西門慶打點使用。西門慶叫了吳主管來,與了他五兩銀子,教他連夜往縣中孔目房裡,抄錄一張東京行下來的文書邸報。上面端的寫的是甚言語?

聞夷狄之禍,自古有之。周之玁狁,漢之匈奴,唐之突厥,迨及五代,而契丹浸強。又我皇宋建國,大遼縱橫中國者已非一日。然未聞內無夷狄而外萌夷狄之患者。諺云:『霜降而堂鐘鳴,雨下而柱礎潤。』以類感類,必然之理。譬猶病夫至此,腹心之疾已久。元氣內消,風邪外入,四肢百骸無非受病。雖盧扁莫之能救,焉能久乎?今天下之勢,正猶病夫尪羸之極矣。君猶元首也,輔臣猶腹心也,百官猶四肢也。陛下端拱於九重之上,百官庶政各盡職于下。元氣內充,榮衛外扞,則虜患何由而至哉!今招夷虜之患者,莫如崇政殿大學士蔡京者,本以憸邪奸險之資,濟以寡廉鮮恥之行,讒諂面諛。上不能輔君當道,贊元理化;下不能宣德布政,保愛元元。徒以利祿自資,希寵固位。樹黨懷奸,蒙蔽欺君,中傷善類。忠士為之解體,四海為之寒心。聯翩朱紫,華聚一門。邇者河湟失議,主議伐遼,內割三郡;郭藥師之叛,失陷卒致;金虜背盟,憑陵中夏。此皆誤國之大者,皆由京之不職也。王黼貪庸無賴,行此俳優。蒙京汲引,薦居政府。未幾,謬掌本兵。惟事慕位苟安,終無一籌可展。迺者,張達殘於太原,為之張皇失散。今虜之犯內地,則又挈妻子南下,為自全之計。其誤國之罪,可勝誅戮?楊戩本以紈褲膏粱,叨承祖廕,憑籍寵靈,典司兵柄,濫膺閫外。大姦似忠。怯懦無比。此三臣者,皆朋黨固結,內外萌蔽,為 陛下腹心之蠱者也。數年以來,招災致異,喪本傷元,役重賦煩,生民離散。盜賊猖獗,夷虜犯順。天下之膏腴已盡,國家之紀綱廢弛。雖擢髮不足以數京等之罪也。臣等待罪該科,備員諫職。徒以目擊奸臣誤國而不為 皇上陳之,則上辜君父之恩,下負平生所學。伏乞宸斷,將京等一干黨惡人犯,或下廷尉,以示薄罰;或 極典,以彰顯戮;或照例枷號,或投之荒裔,以禦魑魅。庶天意可回,人心暢快。國法已正,虜患自消。天下幸甚!奉聖旨蔡京姑留輔政。王黼、楊戩便拿送三法司,會問明白來說。欽此欽遵!續該三法司會問過,并黨惡人犯王黼、楊戩本兵不職,縱虜深入,荼毒生民,損兵折將,失陷內地,律應處斬。手下壞事家人,書辦官掾親黨,董升、盧虎、楊盛、龐宣、韓宗仁、陳洪、黃玉、賈廉、劉盛、趙弘道等,查出有名人犯,俱問擬枷號一個月,滿日發邊衛充軍。」

西門慶不看萬事皆休,看了耳邊廂只聽颼的一聲,魂魄不知往那裡去了。就是驚損六葉連肝肺,諕壞三毛七孔心。即忙打點金銀寶玩,馱裝停當。把家人來保、來旺叫到臥房中,悄悄分付:「如此如此,這般這般,僱頭口,星夜上東京打聽消息。不消到爾陳親家家爹下處。但有不好聲色,取巧打點停當,速來回報。」已與了他二人二十兩盤纏,絕早五更,僱腳夫起程上東京去了,不在話下。西門慶通一夜不曾睡著。到次日早,分付來昭、賁四把花園工程止住,各項匠人都且回去,不做了。每日將大門緊閉。家下人無事,亦不敢往外去。隨分人叫著,不許開。西門慶只在房裡動彈,走出來又走進去。憂上加憂,悶上添悶,如熱地蚰蜒一般。把娶李瓶兒的勾當,丟在九宵雲外去了。吳月娘見他每日在房中愁眉不展,面帶憂容,便說道:「他陳親家那邊為事,各人冤有頭債有主,你平白焦愁些甚麼?」西門慶道:「你婦人知道些甚麼?陳親家是我的親家,女兒女婿兩個業障搬來咱家住著,這是一件事。平昔街坊鄰舍,惱咱的極多。常言:『機兒不快,梭兒快;打著羊,駒驢戰。』倘有小人指戳,拔樹尋根,你我身家不保。」正是:

「關著門兒家裡坐,  禍從天上來!」

這裡西門慶在家納悶不題。且說李瓶兒等了一日兩日,不見動靜,一連使馮媽媽來了兩遍,大門關得鐵桶相似,就是樊噲也撞不開。等了半日,沒一個人牙兒出來,竟不知怎的。看看到廿四日,李瓶兒又使馮媽媽送頭面來,就請西門慶過去說話。叫門不開,去在對過房簷下。少頃,只見玳安出來飲馬,看見便問:「馮媽媽你來做甚麼?」馮媽媽說:「你二娘使我送頭面來。怎的不見動靜?請你爹過去說話哩。」玳安道:「俺爹連日有些小事兒,不得閒。你老人家還拿回頭面去,等我飲馬回來,對俺爹說就是了。」馮媽媽道:「好哥哥,我在這裡等著,你拿進頭面去和你爹說去。你二娘那裡好不惱我哩。」這玳安一面把馬拴下,走到裡邊。半日出來道:「對俺爹說了,頭面爹收下了。教你上覆二娘,再待幾日兒,我爹出來往二娘那裡說話。」這馮媽媽一直走來回了婦人話。婦人又等了幾日,看看五月將盡,六月初旬時分,朝思暮盼,音信全無。夢攘魂勞,佳期間阻。正是:

「懶把蛾眉掃,  羞將粉臉均;

滿懷幽恨積,  憔悴玉精神。」

婦人盼不見西門慶來,每日茶飯頓減,精神恍惚。到晚夕孤眠枕上,展轉躊躕,忽聽外邊打門,彷彿見西門慶來到。婦人迎門笑接,攜手進房,問其爽約之情,各訴衷腸之話。綢繆繾綣,徹夜歡娛,雞鳴天曉,頓抽身回去。婦人恍然驚覺,大叫一聲,精魂已失。慌了馮媽媽進房來看視。婦人說道:「西門慶他剛纔出去,你關上門不曾?」馮媽媽道:「娘子想得心迷了,那裡得大官人來?影兒也沒有。」婦人自此夢境隨邪,夜夜有狐狸假名抵姓,來攝其精髓。漸漸形容黃瘦,飲食不進,臥牀不起。馮媽媽向婦人說,請了大街口蔣竹山來看。其人年小,不上三十,生的五短身才,人物飄逸,極是個輕浮狂詐的人。請入臥室,婦人則霧鬢雲鬟,擁衾而臥,似不勝憂愁之狀。勉強茶湯已罷,丫鬟安放褥甸。竹山就牀診視脉息畢,因見婦人生有姿色,便開言說道:「小人適診病源,娘子肝脉絃出寸口而洪大,厥陰脉出寸口久上魚際,主六慾七情所致。陰陽交爭,乍寒乍熱,似有鬱結于中而不遂之意也。似瘧非瘧,似寒非寒。白日則倦怠嗜臥,精神短少。夜晚神不守舍,夢與鬼交。若不早治,久而變為骨蒸之疾,必有屬纊之憂矣。可惜,可惜!」婦人道:「有累先生俯賜良劑,奴好了重加酬謝。」竹山道:「小人無不用心。娘子若服了我的藥,必然貴體全安。」說畢起身。這裡使藥金五星,使馮媽媽討將藥來。婦人晚間吃了他的藥下去,夜裡得睡,便不驚恐。漸漸飲食加添起來,梳頭走動。那消數日,精神復舊。一日安排了一席酒餚,備下三兩銀子,使馮媽媽請過竹山來相謝。這蔣竹山從與婦人看病之時,懷覬覦之心,已非一日。于是一聞其請,即具服而往。延之中堂,婦人盛粧出見,道了萬福。茶湯兩換,請入房中。酒饌已陳,麝蘭香藹。小丫鬟綉春在傍,描金盤內托出三兩白金。婦人高拏玉盞,向前施禮,說道:「前日奴家心中不好,蒙賜良劑,服之見效。今粗治了一杯水酒,請過先生來知謝知謝。」竹山道:「此是小人分內之事,理當措置,何必計較?」因見三兩謝禮,說道:「這個學生怎麼敢領?」婦人道:「些須微意,不成禮數,萬望先生笑納。」辭讓了半日,竹山方纔收了。婦人遞酒,安了坐次。飲過三巡,竹山席間愉眼睃視,婦人粉粧玉琢,嬌豔驚人。先用言以挑之,因說道:「小人不敢動問,娘子青春幾何?」婦人道:「奴虛度二十四歲。」竹山道:「又一件,似娘子這等妙年,生長深閨,處于富足,何事不遂?而前日有些鬱結不足之病?」婦人聽了,微笑道:「不瞞先生,奴因拙夫去世,家事蕭條,獨自一身,憂愁思慮,何得無病?」竹山道:「原來娘子夫主歿了,多少時了?」婦人道:「拙夫從去歲十一月得傷寒病死了,今已八個月來。」竹山道:「曾吃誰的藥?」婦人道:「大街上胡先生。」竹山道:「是那東街上劉太監房子住的胡鬼嘴兒?他又不是我太醫院出身,知道甚麼脉?娘子怎的請他?」婦人道:「也是因街坊上人薦舉請他來看。還是拙夫沒命,不干他事。」竹山又道:「娘子也還有子女沒有?」婦人道:「兒女俱無。」竹山道:「可惜!娘子這般青春妙齡之際,獨自孀居,又無所出,何不尋其別進之路?甘為幽鬱,豈不生病?」婦人道:「奴近日也講著親事,早晚過門。」竹山便道:「動問娘子,與何人作親?」婦人道:「是縣前開生藥舖西門大官人。」竹山聽了道:「苦哉!苦哉!娘子因何嫁他?小人常在他家看病,最知詳細。此人專在縣中抱攬說事,舉放私債,家中挑販人口。家中不算丫頭,大小五六個老婆。著緊打躺棍兒,稍不中意,就令媒人領出賣了。就是打老婆的班頭,炕婦女的領袖。娘子早時對我說,不然進入他家,如飛蛾投火一般,坑你上不上,下不下,那時悔之晚矣。況近日他親家那邊為事,干連在家,躲避不山。房子蓋的半落不合的,多丟下了。東京門下文書,坐落府縣拿人。到明日他蓋這房子,多是入官抄沒的數兒。娘子沒來由嫁他則甚?」一篇話把婦人說的閉口無言。況且許多東西,丟在他家,尋思半晌,暗中跌腳:「怪嗔道一替兩替請著他不來,原來他家中為事哩!」又見竹山語言活動,一團謙恭。「奴明日若嫁得恁樣個人也罷了,不知他有妻室沒有?」因問道:「既蒙先生指教,奴家感戴不淺。倘有甚相知人家親事,舉保來說,奴無有個不依之理。」竹山乘機請問:「不知要何等樣人家?小人打聽的實,好來這裡說。」婦人道:「人家倒也不論乎大小,只像先生這般人物的。」這蔣竹山不聽便罷,聽了此言,喜歡的勢不知有無。于是走下席來,雙膝跪在地下,告道:「不瞞娘子說,小人內為失助,中饋乏人,鰥居已久,子息全無。倘蒙娘子垂憐見愛,肯結秦晉之緣,足稱平生之願。小人雖啣環結草,不敢有忘!」婦人笑以手携之,說道:「且請起,未審先生鰥居幾時?貴庚多少?既要做親,須得要個保山來說,方成禮數。」竹山又跪下哀告道:「小人行年二十九歲,正月二十七日卯時建生。不幸去年荊妻已故,家緣貧乏,實出寒微。今既蒙金諾之言,何用冰人之講?」婦人聽言笑道:「你既無錢,我這裡有個媽媽,姓馮,拉他做個媒證,也不消你行聘。擇個吉日良辰,招你進來,入門為贅。你意下若何?」這蔣竹山連忙倒身下拜:「娘子就如同小人重生父母,再長爹娘!宿世有緣,三生大幸矣!」一面兩個在房中,各遞了一杯交歡盞,已成其親事。竹山飲至天晚回家。婦人這裡與馮媽媽商議說:「西門慶家如此這般為事,吉凶難保。況且奴家這邊沒人,不好了一場,險不喪了性命。為今之計,不如把這位先生招他進來,過其日月,有何不可?」到次日,就使馮媽媽通信過去,擇六月十八日大好日期,把蔣竹山倒踏門招進來,成其夫婦,過了三日,婦人湊了三百兩銀子,與竹山打開門面兩間開店,煥然一新的。初時往人家看病只是走,後來買了一匹驢兒騎著,在街上往來搖擺,不在話下。正是:

「一窪死水全無浪,  也有春風擺動時。」

畢竟未知後來何如,且聽下回分解:





\end{showcontents}
