%# -*- coding: utf-8 -*-
%!TEX encoding = UTF-8 Unicode
%!TEX TS-program = xelatex
% vim:ts=4:sw=4
%
% 以上设定默认使用 XeLaTex 编译,并指定 Unicode 编码,供 TeXShop 自动识别

%第二十九回 
\chapter{吳神仙冰鑒定終身 潘金蓮蘭湯邀午戰}

詞曰:

新涼睡起,蘭湯試浴郎偷戲。去曾嗔怒,來便生歡喜。
奴道無心,郎道奴如此。情如水,易開難斷,若個知生死。

話說到次日,潘金蓮早起,打發西門慶出門。記掛著要做那紅鞋,拿著針線筐兒,往翡翠軒台基兒上坐著,描畫鞋扇。使春梅請了李瓶兒來到。李瓶兒問道:「姐姐,你描金的是甚麼?」金蓮道:「要做一雙大紅鞋素緞子白綾平底鞋兒,鞋尖上扣繡鸚鵡摘桃。」李瓶兒道:「我有一方大紅十樣錦緞子,也照依姐姐描恁一雙兒。我做高低的罷。」於是取了針線筐,兩個同一處做。金蓮描了一隻丟下,說道:「李大姐,你替我描這一隻,等我後邊把孟三姐叫了來。他昨日對我說,他也要做鞋哩。」一直走到後邊。玉樓在房中倚著護炕兒,也衲著一隻鞋兒哩。看見金蓮進來,說道:「你早辦!」金蓮道:「我起來的早,打發他爹往門外與賀千戶送行去了。教我約下李大姐,花園裡趕早涼做些生活。我才描了一隻鞋,教李大姐替我描著,逕來約你同去,咱三個一搭兒里好做。」因問:「你手裡衲的是甚麼鞋?」 玉樓道:「是昨日你看我開的那雙玄色緞子鞋。」金蓮道:「你好漢!又早衲出一隻來了。」玉樓道:「那隻昨日就衲好了,這一隻又衲了好些了。」金蓮接過看了一回,說:「你這個,到明日使甚麼雲頭子?」玉樓道:「我比不得你每小後生,花花黎黎。我老人家了,使羊皮金緝的雲頭子罷,周圍拿紗綠線鎖,好不好?」金蓮道:「也罷。你快收拾,咱去來,李瓶兒那裡等著哩。」玉樓道:「你坐著吃了茶去。」金蓮道:「不吃罷,拿了茶,那裡去吃來。」玉樓吩咐蘭香頓下茶送去。兩個婦人手拉著手兒,袖著鞋扇,逕往外走。吳月娘在上房穿廊下坐,便問:「你每那去?」金蓮道:「李大姐使我替他叫孟三兒去,與他描鞋。」說著,一直來到花園內。

便罷,聽了記在心裡。到次日,要攆來昭三口子出門。多虧月娘再三攔勸下,不容他在家,打發他往獅子街房子里看守,替了平安兒來家守大門。後次月娘知道,甚惱金蓮,不在話下。

西門慶一日正在前廳坐,忽平安兒來報:「守備府周爺差人送了一位相面先生,名喚吳神仙,在門首伺候見爹。」西門慶喚來人進見,遞上守備帖兒,然後道:「有請。」須臾,那吳神仙頭戴青佈道巾,身穿布袍草履,腰系黃絲雙穗絛,手執龜殼扇子,自外飄然進來。年約四十之上,生得神清如長江皓月,貌古似太華喬松。原來神仙有四般古怪:身如松,聲如鐘,坐如弓,走如風。但見他:

能通風鑒,善究子平。觀乾象,能識陰陽;察龍經,明知風水。五星深講,三命秘談。審格局,決一世之榮枯;觀氣色,定行年之休咎。若非華岳修真客,定是成都賣卜人。

西門慶見神仙進來,忙降階迎接,接至廳上。神仙見西門慶,長揖稽首就坐。須臾茶罷。西門慶動問神仙:「高名雅號,仙鄉何處,因何與周大人相識?」那吳神仙欠身道:「貧道姓吳名奭,道號守真。本貫浙江仙遊人。自幼從師天台山紫虛觀出家。雲游上國,因往岱宗訪道,道經貴處。周老總兵相約,看他老夫人目疾,特送來府上觀相。」西門慶道:「老仙長會那幾家陰陽?道那幾家相法?」神仙道:「貧道粗知十三家子平,善曉麻衣相法,又曉六壬神課。常施藥救人,不愛世財,隨時住世。」西門慶聽言,益加敬重,誇道:「真乃謂之神仙也。」一面令左右放桌兒,擺齋管待。神仙道:「貧道未道觀相,豈可先要賜齋。」西門慶笑道:「仙長遠來,一定未用早齋。待用過,看命未遲。」於是陪著神仙吃了些齋食素饌,抬過桌席,拂抹乾凈,討筆硯來。

神仙道:「請先觀貴造,然後觀相尊容。」西門慶便說與八字:「屬虎的,二十九歲了,七月二十八日午時生。」這神仙暗暗十指尋紋,良久說道:「官人貴造:戊寅年,辛酉月,壬午日,丙午時。七月廿三日白戊,已交八月算命。月令提剛辛酉,理取傷官格。子平雲:傷官傷盡復生財,財旺生官福轉來。立命申宮,七歲行運辛酉,十七行壬戌,二十七癸亥,三十七甲子,四十七乙醜。官人貴造,依貧道所講,元命貴旺,八字清奇,非貴則榮之造。但戊土傷官,生在七八月,身忒旺了。幸得壬午日乾,醜中有癸水,水火相濟,乃成大器。丙午時,丙合辛生,後來定掌威權之職。一生盛旺,快樂安然,發福遷官,主生貴子。為人一生耿直,幹事無二,喜則合氣春風,怒則迅雷烈火。一生多得妻財,不少紗帽戴。臨死有二子送老。今歲丁未流年,丁壬相合,目下丁火來克,克我者為官為鬼,必主平地登雲之喜,添官進祿之榮。大運見行癸亥,戊土得癸水滋潤,定見發生。目下透出紅鸞天喜,定有熊羆之兆。又命宮驛馬臨申,不過七月必見矣。」西門慶問道:「我後來運限如何?」神仙道:「官人休怪我說,但八字中不宜陰水太多,後到甲子運中,將壬午衝破了,又有流星打攪,不出六六之年,主有嘔血流濃之災,骨瘦形衰之病。」西門慶問道:「目下如何?」神仙道:「目今流年,日逢破敗五鬼在家吵鬧,些小氣惱,不足為災,都被喜氣神臨門衝散了。」西門慶道:「命中還有敗否?」神仙道:「年趕著月,月趕著日,實難矣。」

西門慶聽了,滿心歡喜,便道:「先生,你相我面如何?」神仙道:「請尊容轉正。」西門慶把座兒掇了一掇。神仙相道:「夫相者,有心無相,相逐心生;有相無心,相隨心往。吾觀官人:頭圓項短,定為享福之人;體健筋強,決是英豪之輩;天庭高聳,一生衣祿無虧;地閣方圓,晚歲榮華定取。此幾椿兒好處。還有幾椿不足之處,貧道不敢說。」西門慶道:「仙長但說無妨。」神仙道:「請官人走兩步看。」西門慶真個走了幾步。神仙道:「你行如擺柳,必主傷妻;若無刑克,必損其身。妻宮克過方好。」西門慶道:「已刑過了。」神仙道:「請出手來看一看。」西門慶舒手來與神仙看。神仙道:「智慧生於皮毛,苦樂觀於手足。細軟豐潤,必享福祿之人也。兩目雌雄,必主富而多詐;眉生二尾,一生常自足歡娛;根有三紋,中歲必然多耗散;姦門紅紫,一生廣得妻財;黃氣發於高曠,旬日內必定加官;紅色起於三陽,今歲間必生貴子。又有一件不敢說,淚堂豐厚,亦主貪花;且喜得鼻乃財星,驗中年之造化;承漿地閣,管來世之榮枯。承漿地閣要豐隆,準乃財星居正中。生平造化皆由命,相法玄機定不容。」

神仙相畢,西門慶道:「請仙長相相房下眾人。」一面令小廝:「後邊請你大娘出來。」於是李嬌兒、孟玉樓、潘金蓮、李瓶兒、孫雪娥等眾人都跟出來,在軟屏後潛聽。神仙見月娘出來,連忙道了稽首,也不敢坐,就立在旁邊觀相。端詳了一回,說:「娘子面如滿月,家道興隆;唇若紅蓮,衣食豐足,必得貴而生子;聲響神清,必益夫而發福。請出手來。」月娘從袖中露出十指春蔥來。神仙道:「乾薑之手,女人必善持家,照人之鬢,坤道定須秀氣。這幾椿好處。還有些不足之處,休怪貧道直說。」西門慶道:「仙長但說無妨。」「淚堂黑痣,若無宿疾,必刑夫;眼下皴紋,亦主六親若冰炭。女人端正好容儀,緩步輕如出水龜。行不動塵言有節,無肩定作貴人妻。」

相畢,月娘退後。西門慶道:「還有小妾輩,請看看。」於是李嬌兒過來。神仙觀看良久:「此位娘子,額尖鼻小,非側室,必三嫁其夫;肉重身肥,廣有衣食而榮華安享;肩聳聲泣,不賤則孤;鼻樑若低,非貧即夭。請步幾步我看。」李嬌兒走了幾步。神仙道:

額尖露背並蛇行,早年必定落風塵。假饒不是娼門女,也是屏風後立人。

相畢,李嬌兒下去。吳月娘叫:「孟三姐,你也過來相一相。」神仙觀道:「這位娘子,三停平等,一生衣祿無虧;六府豐隆,晚歲榮華定取。平生少疾,皆因月孛光輝;到老無災,大抵年宮潤秀。請娘子走兩步。」玉樓走了兩步,神仙道:

口如四字神清澈,溫厚堪同掌上珠。威命兼全財祿有,終主刑夫兩有餘。

玉樓相畢,叫潘金蓮過來。那潘金蓮只顧嘻笑,不肯過來。月娘催之再三,方纔出見。神仙抬頭觀看這個婦人,沉吟半日,方纔說道:「此位娘子,發濃髩重,光斜視以多淫;臉媚眉彎,身不搖而自顫。面上黑痣,必主刑夫;唇中短促,終須壽夭。舉止輕浮惟好淫,眼如點漆壞人倫。月下星前長不足,雖居大廈少安心。」

相畢金蓮,西門慶又叫李瓶兒上來,教神仙相一相。神仙觀看這個女人:「皮膚香細,乃富室之女娘;容貌端莊,乃素門之德婦。只是多了眼光如醉,主桑中之約;眉眉靨生,月下之期難定。觀臥蠶明潤而紫色,必產貴兒;體白肩圓,必受夫之寵愛。常遭疾厄,只因根上昏沉;頻遇喜祥,蓋謂福星明潤。此幾椿好處。還有幾椿不足處,娘子可當戒之:山根青黑,三九前後定見哭聲;法令細繵,雞犬之年焉可過?慎之!慎之!花月儀容惜羽翰,平生良友鳳和鸞。朱門財祿堪依倚,莫把凡禽一樣看。」

相畢,李瓶兒下去。月娘令孫雪娥出來相一相。神仙看了,說道:「這位娘子,體矮聲高,額尖鼻小,雖然出谷遷喬,但一生冷笑無情,作事機深內重。只是吃了這四反的虧,後來必主凶亡。夫四反者:唇反無棱,耳反無輪,眼反無神,鼻反不正故也。燕體蜂腰是賤人,眼如流水不廉真。常時斜倚門兒立,不為婢妾必風塵。」

雪娥下去,月娘教大姐上來相一相。神仙道:「這位女娘,鼻樑低露,破祖刑家;聲若破鑼,家私消散。麵皮太急,雖溝洫長而壽亦夭;行如雀躍,處家室而衣食缺乏。不過三九,當受折磨。惟夫反目性通靈,父母衣食僅養身。狀貌有拘難顯達,不遭惡死也艱辛。」

大姐相畢,教春梅也上來教神仙相相。神仙睜眼兒見了春梅,年約不上二九,頭戴銀絲雲髻兒,白線挑衫兒,桃紅裙子,藍紗比甲兒,纏手纏腳出來,道了萬福。神仙觀看良久,相道:「此位小姐五官端正,骨格清奇。發細眉濃,稟性要強;神急眼圓,為人急燥。山根不斷,必得貴夫而生子;兩額朝拱,主早年必戴珠冠。行步若飛仙,聲響神清,必益夫而得祿,三九定然封贈。但吃了這左眼大,早年克父;右眼小,周歲克娘。左口角下這一點黑痣,主常沾啾唧之災;右腮一點黑痣,一生受夫敬愛。天庭端正五官平,口若塗砂行步輕。倉庫豐盈財祿厚,一生常得貴人憐。」

神仙相畢,眾婦女皆咬指以為神相。西門慶封白銀五兩與神仙,又賞守備府來人銀五錢,拿拜帖回謝。吳神仙再三辭卻,說道:「貧道雲游四方,風餐露宿,要這財何用?決不敢受。」西門慶不得已,拿出一匹大布:「送仙長一件大衣如何?」神仙方纔受之,令小童接了,稽首拜謝。西門慶送出大門,飄然而去。正是:

柱杖兩頭挑日月,葫蘆一個隱山川。

西門慶回到後廳,問月娘:「眾人所相何如?」月娘道:「相的也都好,只是三個人相不著。」西門慶道:「那三個相不著?」月娘道:「相李大姐有實疾,到明日生貴子,他見今懷著身孕,這個也罷了。相咱家大姐到明日受磨折,不知怎的磨折?相春梅後來也生貴子,或者你用好他,各人子孫也看不見。我只不信,說他後來戴珠冠,有夫人之分。端的咱家又沒官,那討珠冠來?就有珠冠,也輪不到他頭上。」西門慶笑道:「他相我目下有平地登雲之喜,加官進祿之榮,我那得官來?他見春梅和你俱站在一處,又打扮不同,戴著銀絲雲髻兒,只當是你我親生女兒一般,或後來匹配名門,招個貴婿,故說有珠冠之分。自古算的著命,算不著好,相逐心生,相隨心滅。周大人送來,咱不好囂了他的,教他相相除疑罷了。」說畢,月娘房中擺下飯,打發吃了飯。

西門慶手拿芭蕉扇兒,信步閑游。來花園大卷棚聚景堂內,周圍放下簾櫳,四下花木掩映。正值日午,只聞綠陰深處一派蟬聲,忽然風送花香,襲人撲鼻。有詩為證:

綠樹蔭濃夏日長,樓臺倒影入池塘。
水晶簾動微風起,一架薔薇滿院香。

西門慶坐於椅上以扇搖涼。只見來安兒、畫童兒兩個小廝來井上打水。西門慶道:「教一個來。」來安兒忙走向前,西門慶吩咐:「到後邊對你春梅姐說,有梅湯提一壺來我吃。」來安兒應諾去了。半日,只見春梅家常戴著銀絲雲髻兒,手提一壺蜜煎梅湯,笑嘻嘻走來,問道:「你吃了飯了?」西門慶道:「我在後邊吃了。」 春梅說:「嗔道不進房裡來。說你要梅湯吃,等我放在冰里湃一湃你吃。」西門慶點頭兒。春梅湃上梅湯,走來扶著椅兒,取過西門慶手中芭蕉扇兒替他打扇,問道:「頭裡大娘和你說甚麼?」西門慶道:「說吳神仙相面一節。」春梅道:「那道士平白說戴珠冠,教大娘說『有珠冠,只怕輪不到他頭上』。常言道凡人不可貌相,海水不可斗量,從來旋的不圓,砍的圓,各人裙帶上衣食,怎麼料得定?莫不長遠只在你家做奴才罷!」西門慶笑道:「小油嘴兒,你若到明日有了娃兒,就替你上了頭。」於是把他摟到懷裡,手扯著手兒頑耍,問道:「你娘在那裡?怎的不見?」春梅道:「娘在屋裡,教秋菊熱下水要洗浴。等不的,就在床上睡了。」西門慶道:「等我吃了梅湯,鬼混他一混去。」於是春梅向冰盆內倒了一甌兒梅湯,與西門慶呷了一口,湃骨之涼,透心沁齒,如甘露灑心一般。

須臾吃畢,搭伏著春梅肩膀兒,轉過角門來到金蓮房中。看見婦人睡在正面一張新買的螺鈿床上。原是因李瓶兒房中安著一張螺鈿敞廳床,婦人旋教西門慶使了六十兩銀子,替他也買了這一張螺鈿有欄乾的床。兩邊槅扇都是螺鈿攢造花草翎毛,掛著紫紗帳幔,錦帶銀鉤。婦人赤露玉體,止著紅綃抹胸兒,蓋著紅紗衾,枕著鴛鴦枕,在涼席之上,睡思正濃。房裡異香噴鼻。西門慶一見,不覺淫心頓起,令春梅帶上門出去,悄悄脫了衣褲,上的床來,掀開紗被,見他玉體相互掩映,戲將兩股輕開,按麈柄徐徐插入牝中,比及星眼驚欠之際,已抽拽數十度矣。婦人睜開眼,笑道:「怪強盜,三不知多咱進來?奴睡著了,就不知道。奴睡的甜甜的,摑混死了我!」西門慶道:「我便罷了,若是個生漢子進來,你也推不知道罷?」婦人道:「我不好罵的,誰人七個頭八個膽,敢進我這房裡來!只許你恁沒大沒小的罷了。」原來婦人因前日西門慶在翡翠軒誇獎李瓶兒身上白凈,就暗暗將茉莉花蕊兒攪酥油定粉,把身上都搽遍了,搽的白膩光滑,異香可愛,欲奪其寵。西門慶見他身體雪白,穿著新做的兩隻大紅睡鞋。一面蹲踞在上,兩手兜其股,極力而提之,垂首觀其出入之勢。婦人道:「怪貨,只顧端詳甚麼?奴的身上黑,不似李瓶兒的身上白就是了。他懷著孩子,你便輕憐痛惜,俺每是拾的,由著這等掇弄。」西門慶問道:「說你等著我洗澡來?」婦人問道:「你怎得知道來?」西門慶道:「是春梅說的。」婦人道:「你洗,我叫春梅掇水來。」不一時把浴盆掇到房中,註了湯。二人下床來,同浴蘭湯,共效魚水之歡。洗浴了一回,西門慶乘興把婦人仰臥在浴板之上,兩手執其雙足跨而提之,掀騰(扌扉) 乾,何止二三百回,其聲如泥中螃蟹一般響之不絕。婦人恐怕香雲拖墜,一手扶著雲髩,一手扳著盆沿,口中燕語鶯聲,百般難述。怎見這場交戰?但見:

華池蕩漾波紋亂,翠幃高捲秋雲暗。才郎情動逞風流,美女心歡顯手段。叭叭嗒嗒弄聲響,砰砰啪啪成一片。滑滑����怎停住,攔攔濟濟難存站。一個顫顫巍巍挺硬槍,一個搖搖擺擺弄鋼劍。一個捨死忘生生往裡,一個尤雲滯雨將功乾。撲撲通通皮鼓催,嗶嗶啵啵槍對劍。啪啪嗒嗒弄響聲,嘭嘭湃湃成一片。下下高高水逆流,洶洶涌涌盈清澗。滑滑縐縐怎生停,攔攔濟濟難存站。一來一往,一動一撞東西探,熱氣騰騰奴雲生,紛紛馥馥香氣散。一個逆水撐船,將玉股搖;一個艄公把舵,將金蓮揝。一個紫騮猖獗逞威風,一個白麵妖嬈遭馬戰。喜喜歡歡美女情,雄雄赳赳男兒願。翻翻覆復盡歡娛,鬧鬧挨挨情摸亂。拖泥帶水兩情痴,殢雨尤雲都不辯。任他錦帳鳳鸞交,不似蘭湯魚水戰。你死我活更無休,千戰萬贏心膽戰。口口聲聲叫殺人。氣氣昂昂情厭,古古今今廣鬧爭,不似這般水裡戰。

二人水中戰鬥了一回,西門慶精泄而止。拭抹身體乾凈,撤去浴盆。止著薄纊短襦上床,安放炕桌果酌飲酒。教秋菊:「取白酒來與你爹吃。」又拿果餡餅與西門慶吃,恐怕他肚中饑餓。只見秋菊半日拿上一銀註子酒來。婦人才斟了一鐘,摸了摸冰涼的,就照著秋菊臉上只一潑,潑了一頭一臉,罵道:「好賊少死的奴才!我吩咐教你燙了來,如何拿冷酒與爹吃?你不知安排些甚麼心兒?」叫春梅:「與我把這奴才採到院子里跪著去。」春梅道:「我替娘後邊捲裹腳去來,一些兒沒在跟前,你就弄下硶兒了。」那秋菊把嘴谷都著,口裡喃喃吶吶說道:「每日爹娘還吃冰湃的酒兒,誰知今日又改了腔兒。」婦人聽見罵道:「好賊奴才,你說甚麼?與我採過來!」叫春梅每邊臉上打與他十個嘴巴。春梅道:「皮臉,沒的打污濁了我手。娘只教他頂著石頭跪著罷。」於是不由分說,拉到院子里,教他頂著塊大石頭跪著,不在話下。婦人從新叫春梅暖了酒來,陪西門慶吃了幾鐘,掇去酒桌,放下紗帳子來,吩咐拽上房門,兩個抱頭交股,體倦而寢。正是:

若非群玉山頭見,多是陽臺夢裡尋。


