%# -*- coding: utf-8 -*-
% !TeX encoding = UTF-8 Unicode
% !TeX spellcheck = en_US
% !TeX TS-program = xelatex
%~ \XeTeXinputencoding "UTF-8"
% vim:ts=4:sw=4
%
% 以上設定默認使用 XeLaTex 編譯,並指定 Unicode 編碼,供 TeXShop 自動識別

%第四回
\chapter{赴巫山潘氏幽歡 鬧茶坊鄆哥義憤}

\begin{showcontents}{}


詩曰:

璇閨繡戶斜光入,千金女兒倚門立。
橫波美目雖後來,羅襪遙遙不相及。
聞道今年初避人,珊珊鏡掛長隨身。
願得侍兒為道意,後堂羅帳一相親。

話說王婆拿銀子出門,便向婦人滿面堆下笑來,說道:「老身去那街上取瓶兒來,有勞娘子相待官人坐一坐。壺裡有酒,沒便再篩兩盞兒,且和大官人吃著,老身直去縣東街,那裡有好酒買一瓶來,有好一歇兒耽擱。」婦人聽了說:「乾娘休要去,奴酒不多用了。」婆子便道:「阿呀!娘子,大官人又不是別人,沒事相陪吃一盞兒,怕怎的!」婦人口裡說「不用了」坐著卻不動身。婆子一面把門拽上,用索兒拴了,倒關他二人在屋裡。當路坐了,一頭續著鎖。

這婦人見王婆去了,倒把椅兒扯開一邊坐著,卻只偷眼睃看。西門慶坐在對面,一徑把那雙涎瞪瞪的眼睛看著他,便又問道:「卻才到忘了問娘子尊姓?」婦人便低著頭帶笑的回道:「姓武。」西門慶故做不聽得,說道:「姓堵?」那婦人卻把頭又別轉著,笑著低聲說道:「你耳朵又不聾。」西門慶笑道:「呸,忘了!正是姓武。只是俺清河縣姓武的卻少,只有縣前一個賣飲餅的三寸丁姓武,叫做武大郎,敢是娘子一族麼?」婦人聽得此言,便把臉通紅了,一面低著頭微笑道:「便是奴的丈夫。」西門慶聽了,半日不做聲,呆了臉,假意失聲道屈。婦人一面笑著,又斜瞅了他一眼,低聲說道:「你又沒冤枉事,怎的叫屈?」西門慶道:「我替娘子叫屈哩!」卻說西門慶口裡娘子長娘子短,只顧白嘈。這婦人一面低著頭弄裙子兒,又一回咬著衫袖口兒,咬得袖口兒格格駁駁的響,要便斜溜他一眼兒。只見這西門慶推害熱,脫了上面綠紗褶子道:「央煩娘子替我搭在乾娘護炕上。」這婦人只顧咬著袖兒別轉著,不接他的,低聲笑道:「自手又不折,怎的支使人!」西門慶笑著道:「娘子不與小人安放,小人偏要自己安放。」一面伸手隔桌子搭到床炕上去,卻故意把桌上一拂,拂落一隻箸來。卻也是姻緣湊著,那只箸兒剛落在金蓮裙下。西門慶一面斟酒勸那婦人,婦人笑著不理他。他卻又待拿起箸子起來,讓他吃菜兒。尋來尋去不見了一隻。這金蓮一面低著頭,把腳尖兒踢著,笑道:「這不是你的箸兒!」西門慶聽說,走過金蓮這邊來道:「原來在此。」蹲下身去,且不拾箸,便去他繡花鞋頭上只一捏。那婦人笑將起來,說道:「怎這的羅皂!我要叫了起來哩!」西門慶便雙膝跪下說道:「娘子可憐小人則個!」一面說著,一面便摸他褲子。婦人叉開手道:「你這歪廝纏人,我卻要大耳刮子打的呢!」西門慶笑道:「娘子打死了小人,也得個好處。」於是不由分說,抱到王婆床炕上,脫衣解帶,共枕同歡。卻說這婦人自從與張大戶勾搭,這老兒是軟如鼻涕膿如醬的一件東西,幾時得個爽利!就是嫁了武大,看官試想,三寸丁的物事,能有多少力量?今番遇了西門慶,風月久慣,本事高強的,如何不喜?但見:

交頸鴛鴦戲水,並頭鸞鳳穿花。喜孜孜連理枝生,美甘甘同心帶結。一個將朱唇緊貼,一個將粉臉斜偎。羅襪高挑,肩膀上露兩彎新月;金釵斜墜,枕頭邊堆一朵烏雲。誓海盟山,搏弄得千般旖妮;羞雲怯雨,揉搓的萬種妖嬈。恰恰鶯聲,不離耳畔。津津甜唾,笑吐舌尖。楊柳腰脈脈春濃,櫻桃口微微氣喘。星眼朦朧,細細汗流香玉顆;酥胸蕩漾,涓涓露滴牡丹心。直饒匹配眷姻諧,真個偷情滋味美。

當下二人雲雨才罷,正欲各整衣襟,只見王婆推開房門入來,大驚小怪,拍手打掌,低低說道:「你兩個做得好事!」西門慶和那婦人都吃了一驚。那婆子便向婦人道:「好呀,好呀!我請你來做衣裳,不曾交你偷漢子!你家武大郎知,須連累我。不若我先去對武大說去。」回身便走。那婦人慌的扯住她裙子,紅著臉低了頭,只得說聲:「乾娘饒恕!」王婆便道:「你們都要依我一件事,從今日為始,瞞著武大,每日休要失了大官人的意。早叫你早來,晚叫你晚來,我便罷休。若是一日不來,我便就對你武大說。」那婦人羞得要不的,再說不出來。王婆催逼道:「卻是怎的?快些回復我。」婦人藏轉著頭,低聲道:「來便是了。」王婆又道:「西門大官人,你自不用老身說得,這十分好事已都完了,所許之物,不可失信,你若負心,我也要對武大說。」西門慶道:「乾娘放心,並不失信。」婆子道:「你每二人出語無憑,要各人留下件表記拿著,才見真情。」西門慶便向頭上拔下一根金頭簪來,插在婦人云髻上。婦人除下來袖了,恐怕到家武大看見生疑。婦人便不肯拿甚的出來,卻被王婆扯著袖子一掏,掏出一條杭州白縐紗汗巾,掠與西門慶收了。三人又吃了幾杯酒,已是下午時分。那婦人起身道:「奴回家去罷。」便丟下王婆與西門慶,踅過後門歸來。先去下了簾子,武大恰好進門。

且說王婆看著西門慶道:「好手段麼?」西門慶道:「端的虧了乾娘,真好手段!」王婆又道:「這雌兒風月如何?」西門慶道:「色系子女不可言。」婆子道: 「她房裡彈唱姐兒出身,什麼事兒不久慣知道!還虧老娘把你兩個生扭做夫妻,強撮成配。你所許老身東西,休要忘了。」西門慶道:「我到家便取銀子送來。」王婆道:「眼望旌捷旗,耳聽好消息。不要交老身棺材出了討輓歌郎錢。」西門慶一面笑著,看街上無人,帶上眼紗去了。不在話下。

次日,又來王婆家討茶吃。王婆讓坐,連忙點茶來吃了。西門慶便向袖中取出一錠十兩銀子來,遞與王婆。但凡世上人,錢財能動人意。那婆子黑眼睛見了雪花銀子,一面歡天喜地收了,一連道了兩個萬福,說道:「多謝大官人佈施!」因向西門慶道:「這咱晚武大還未出門,待老身往她家推借瓢,看一看。」一面從後門踅過婦人家來。婦人正在房中打發武大吃飯,聽見叫門,問迎兒:「是誰?」迎兒道:「是王奶奶來借瓢。」婦人連忙迎將出來道:「乾娘,有瓢,一任拿去。且請家裡坐。」婆子道:「老身那邊無人。」因向婦人使手勢,婦人就知西門慶來了。婆子拿瓢出了門,一力攛掇武大吃了飯,挑擔出去了。先到樓上從新妝點,換了一套艷色新衣,吩咐迎兒:「好生看家,我往你王奶家坐一坐就來。若是你爹來時,就報我知道。若不聽我說,打下你個小賤人下截來。」迎兒應諾不題。

婦人一面走過王婆茶坊裡來。正是:

合歡桃杏春堪笑,心裡原來別有仁。

有詞單道這雙關二意:

這瓢是瓢,口兒小身子兒大。你幼在春風棚上恁兒高,到大來人難要。他怎肯守定顏回甘貧樂道,專一趁東風,水上漂。也曾在馬房裡喂料,也曾在茶房裡來叫,如今弄得許由也不要。赤道黑洞洞葫蘆中賣的什麼藥?

那西門慶見婦人來了,如天上落下來一般,兩個並肩疊股而坐。王婆一面點茶來吃了,因問:「昨日歸家,武大沒問什麼?」婦人道:「他問乾娘衣服做了不曾,我說道衣服做了,還與乾娘做送終鞋襪。」說畢,婆子連忙安排上酒來,擺在房內,二人交杯暢飲。這西門慶仔細端詳那婦人,比初見時越發標緻。吃了酒,粉面上透出紅白來,兩道水鬢描畫的長長的。端的平欺神仙,賽過嫦娥。

動人心紅白肉色,堪人愛可意裙釵。裙拖著翡翠紗衫,袖挽泥金帶。喜孜孜寶髻斜歪。恰便似月裡嫦娥下世來,不枉了千金也難買。

西門慶誇之不足,摟在懷中,掀起他裙來,看見他一對小腳穿著老鴉緞子鞋兒,恰剛半叉,心中甚喜。一遞一口與他吃酒,嘲問話兒。婦人因問西門慶貴庚,西門慶告他說:「二十七歲,七月二十八日子時生。」婦人問:「家中有幾位娘子?」西門慶道:「除下拙妻,還有三四個身邊人,只是沒一個中我意的。」婦人又問: 「幾位哥兒?」西門慶道:「只是一個小女,早晚出嫁,並無娃兒。」西門慶嘲問了一回,向袖中取出銀穿心金裹面盛著香茶木樨餅兒來,用舌尖遞送與婦人。兩個相摟相抱,鳴咂有聲。那婆子只管往來拿菜篩酒,那裡去管他閒事,由著二人在房內做一處取樂玩耍。少頃吃得酒濃,不覺烘動春心,西門慶色心輒起,露出腰間那話,引婦人纖手捫弄。原來西門慶自幼常在三街四巷養婆娘,根下猶帶著銀打就,藥煮成的托子。那話煞甚長大,紅赤赤黑鬚,直豎豎堅硬,好個東西:

一物從來六寸長,有時柔軟有時剛。
軟如醉漢東西倒,硬似風僧上下狂。
出牝入陰為本事,腰州臍下作家鄉。
天生二子隨身便,曾與佳人斗幾場。

少頃,婦人脫了衣裳。西門慶摸見牝戶上並無毳毛,猶如白馥馥、鼓蓬蓬髮酵的饅頭,軟濃濃、紅縐縐出籠的果餡,真個是千人愛萬人貪一件美物:

溫緊香乾口賽蓮,能柔能軟最堪憐。
喜便吐舌開顏笑,困便隨身貼股眠。
內襠縣裡為家業,薄草涯邊是故園。
若遇風流輕俊子,等閒戰鬥不開言。

話休饒舌。那婦人自當日為始,每日踅過王婆家來,和西門慶做一處,恩情似漆,心意如膠。自古道:好事不出門,惡事傳千里。不到半月之間,街坊鄰舍都曉的了,只瞞著武大一個不知。正是:

自知本分為活計,那曉防奸革弊心。

話分兩頭。且說本縣有個小的,年方十五六歲,本身姓喬,因為做軍在鄆州生養的,取名叫做鄆哥。家中只有個老爹,年紀高大。那小廝生得乖覺,自來只靠縣前這許多酒店裡賣些時新果品,時常得西門慶繼發他些盤纏。其日正尋得一籃兒雪梨,提著繞街尋西門慶。又有一等多口人說:「鄆哥你要尋他,我教你一個去處。」鄆哥道:「起動老叔,教我那去尋他的是?」那多口的道:「我說與你罷。西門慶刮剌上賣炊餅的武大老婆,每日只在紫石街王婆茶坊裡坐的。這咱晚多定只在那裡。你小孩子家,只故撞進去不妨。」那鄆哥得了這話,謝了那人,提了籃兒,一直往紫石街走來,逕奔入王婆茶坊裡去。卻正見王婆坐在小凳兒上績線,鄆哥把籃兒放下,看著王婆道:「乾娘!聲喏。」那婆子問道:「鄆哥,你來這裡做什麼?」鄆哥道:「要尋大官人,賺三五十錢養活老爹。」婆子道:「什麼大官人?」鄆哥道:「情知是那個,便只是他那個。」婆子道:「便是大官人,也有個姓名。」鄆哥道:「便是兩個字的。」婆子道:「什麼兩個字的?」鄆哥道:「乾娘只是要作耍。我要和西門大官人說句話兒!」望裡便走。那婆子一把揪住道:「這小猴子那裡去?人家屋裡,各有內外。」鄆哥道:「我去房裡便尋出來。」王婆罵道:「含烏小囚兒!我屋裡那裡討什麼西門大官?」鄆哥道:「乾娘不要獨自吃,也把些汁水與我呷一呷。我有什麼不理會得!」婆子便罵:「你那小囚攮的,理會得什麼?」鄆哥道:「你正事馬蹄刀木杓裡切菜──水洩不漏,直要我說出來,只怕賣炊餅的哥哥發作!」那婆子吃他這兩句道著他真病,心中大怒,喝道:「含烏小猢猻,也來老娘屋裡放屁!」鄆哥道:「我是小猢猻,你是馬伯六,做牽頭的老狗肉!」那婆子揪住鄆哥鑿上兩個栗暴。鄆哥叫道:「你做甚麼便打我?」婆子罵道: 「賊肏娘的小猢猻!你敢高做聲,大耳刮子打出你去。」鄆哥道:「賊老咬蟲,沒事便打我!」這婆子一頭叉,一頭大栗暴,直打出街上去,把雪梨籃兒也丟出去。那籃雪梨四分五落滾了開去。這小猴子打那虔婆不過,一頭罵,一頭哭,一頭走,一頭街上拾梨兒,指著王婆茶坊裡罵道:「老咬蟲,我交你不要慌!我不與他不做出來不信!定然遭塌了你這場門面,交你賺不成錢!」這小猴子提個籃兒,逕奔街上尋這個人。卻正是:

掀翻孤兔窩中草,驚起鴛鴦沙上眠。


\piWenlongF{
文禹门云:此刻西门庆,早已忘记武松,此刻潘金莲,但知防备武植;此刻王婆子,惟有借金莲之货,以骗西门之财,是三人者,正是利令智昏,色迷心窍,如入茫茫大海,实有不能自主者。想当时清河县中,知其事者,应有人为之摇头,应有人为之吐舌,应有人为之切齿,应有人为之握拳。大抵为之愤懑不平者居多,而羡慕之、妒嫉之,竟思效法之者盖寡。耳闻其事,目睹其形者,俱有天理良心在。奈何后之人看此书者,明明知是《水浒传》中翻案,乌有先生说谎,子虚罗土掉皮,乃不知不觉,心往于王婆屋中,颊鸾倒风,神游于王婆床上,滞雨尤云。反而细思,能不大笑!此其人,尚可与看此书乎?不看《金瓶梅》,其心已有不堪问;再看《金瓶梅》,其事将有不可言者。果《金瓶梅》之误人欤?抑人之自误于《金瓶梅》欤?

}

\end{showcontents}
