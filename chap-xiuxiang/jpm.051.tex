%# -*- coding: utf-8 -*-
%!TEX encoding = UTF-8 Unicode
%!TEX TS-program = xelatex
% vim:ts=4:sw=4
%
% 以上设定默认使用 XeLaTex 编译,并指定 Unicode 编码,供 TeXShop 自动识别

%第五十一回 
\chapter{打貓兒金蓮品玉 鬥葉子敬濟輸金}

詩曰:

羞看鸞鏡惜朱顏,手托香腮懶去眠。瘦損纖腰寬翠帶,淚流粉面落金鈿。
薄倖惱人愁切切,芳心繚亂恨綿綿。何時借得東風便,颳得檀郎到枕邊。

話說潘金蓮見西門慶拿了淫器包兒,與李瓶兒歇了,足惱了一夜沒睡,懷恨在心。到第二日,打聽西門慶往衙門裡去了,老早走到後邊對月娘說:「李瓶兒背地好不說姐姐哩!說姐姐會那等虔婆勢,喬坐衙,別人生日,又要來管。『你漢子吃醉了進我屋裡來,我又不曾在前邊,平白對著人羞我,望著我丟臉兒。交我惱了,走到前邊,把他爹趕到後邊來。落後他怎的也不在後邊,還到我房裡來了?我兩個黑夜說了一夜梯己話兒,只有心腸五臟沒曾倒與我罷了。』」這月娘聽了,如何不惱!因向大妗子、孟玉樓說:「你們昨日都在跟前看著,我又沒曾說他甚麼。小廝交燈籠進來,我只問了一聲:『你爹怎的不進來?』小廝倒說:『往六娘屋裡去了。』 我便說:『你二娘這裡等著,恁沒槽道,卻不進來!』論起來也不傷他,怎的說我虔婆勢,喬坐衙?我還把他當好人看成,原來知人知面不知心,那裡看人去?乾凈是個綿里針、肉里刺的貨,還不知背地在漢子跟前架甚麼舌兒哩!怪道他昨日決烈的就往前走了。傻姐姐,那怕漢子成日在你屋裡不出門,不想我這心動一動兒。一個漢子丟與你們,隨你們去,守寡的不過。想著一娶來之時,賊強人和我門裡門外不相逢,那等怎的過來?」大妗子在旁勸道:「姑娘罷麼,看孩兒的分上罷!自古宰相肚裡好行船。當家人是個惡水缸兒,好的也放在心裡,歹的也放在心裡。」月娘道:「不拘幾時,我也要對這兩句話。等我問他,我怎麼虔婆勢,喬做衙?」金蓮慌的沒口子說道:「姐姐寬恕他罷。常言大人不責小人過,那個小人沒罪過?他在背地挑唆漢子,俺們這幾個誰沒吃他排說過?我和他緊隔著壁兒,要與他一般見識起來,倒了不成!行動只倚著孩兒降人,他還說的好話兒哩!說他的孩兒到明日長大了,有恩報恩,有仇報仇,俺們都是餓死的數兒──你還不知道哩!」吳大妗子道:「我的奶奶,那裡有此話說?」月娘一聲兒也沒言語。

常言:路見不平,也有向燈向火。不想西門大姐平日與李瓶兒最好,常沒針線鞋面,李瓶兒不拘好綾羅緞帛就與他,好汗巾手帕兩三方背地與大姐,銀錢不消說。當日聽了此話,如何不告訴他。李瓶兒正在屋裡與孩子做端午戴的絨線符牌,及各色紗小粽子並解毒艾虎兒。只見大姐走來,李瓶兒讓他坐,又交迎春:「拿茶與你大姑娘吃。」大姐道:「頭裡請你吃茶,你怎的不來?」李瓶兒道:「打發他爹出門,我趕早涼與孩子做這戴的碎生活兒來。」大姐道:「有樁事兒,我也不是舌頭,敢來告你說:你沒曾惱著五娘?他對著俺娘,如此這般說了你一篇是非──說你說俺娘虔婆勢,喬做衙。如今俺娘要和你對話哩!你別要說我對你說,交他怪我。你須預備些話兒打發他。」這李瓶兒不聽便罷,聽了此言,手中拿著那針兒通拿不起來,兩隻胳膊都軟了,半日說不出話來,對著大姐掉眼淚,說道:「大姑娘,我那裡有一字兒?昨晚我在後邊,聽見小廝說他爹往我這邊來了,我就來到前邊,催他往後邊去了。再誰說一句話兒來?你娘恁覷我一場,莫不我恁不識好歹,敢說這個話?設使我就說,對著誰說來?也有個下落。」大姐道:「他聽見俺娘說不拘幾時要對這話,他也就慌了。要是我,你兩個當面鑼對面鼓的對不是!」李瓶兒道: 「我對的過他那嘴頭子?只憑天罷了。他左右晝夜算計的只是俺娘兒兩個,到明日終久吃他算計了一個去,才是了當。」說畢哭了。大姐坐著勸了一回,只見小玉來請六娘、大姑娘吃飯。李瓶兒丟下針指,同大姐到後邊,也不曾吃飯,回來房中,倒在床上就睡著了。

西門慶衙門中來家,見他睡,問迎春。迎春道:「俺娘一日飯也還沒吃哩。」慌的西門慶向前問道:「你怎的不吃飯?你對我說。」又見他哭的眼紅紅的,只顧問: 「你心裡怎麼的?對我說。」李瓶兒連忙起來,揉了揉眼說道:「我害眼疼,不怎的。今日心裡懶待吃飯。」並不題出一字兒來。正是:滿懷心腹事,盡在不言中。有詩為證:

莫道佳人總是痴,惺惺伶俐沒便宜。
只因會盡人間事,惹得閑愁滿肚皮。

大姐在後邊對月娘說:「才五娘說的話,我問六娘來。他好不賭身發咒,望著我哭,說娘這般看顧他,他肯說此話!」吳大妗子道:「我就不信。李大姐好個人兒,他怎肯說這等話!」月娘道:「想必兩個有些小節不足,哄不動漢子,走來後邊,沒的拿我墊舌根。我這裡還多著個影兒哩!」大妗子道:「大姑娘,今後你也別要虧了人。不是我背地說,潘五姐一百個不及他。為人心地兒又好,來了咱家恁二三年,要一些歪樣兒也沒有。」

正說著,只見琴童兒背進個藍布大包袱來。月娘問是甚麼,琴童道:「是三萬鹽引。韓伙計和崔本才從關上掛了號來,爹說打發飯與他二人吃,如今兌銀子打包。後日二十,是個好日子,起身,打發他三個往揚州去。」吳大妗子道:「只怕姐夫進來。我和二位師父往他二娘房裡坐去罷。」剛說未畢,只見西門慶掀帘子進來,慌的吳妗子和薛姑子、王姑子往李嬌兒房裡走不迭。早被西門慶看見,問月娘:「那個是薛姑子?賊胖禿淫婦,來我這裡做甚麼!」月娘道:「你好恁枉口撥舌,不當家化化的,罵他怎的?他惹著你來?你怎的知道他姓薛?」西門慶道:「你還不知他弄的乾坤兒哩!他把陳參政的小姐弔在地藏庵兒里和一個小伙偷姦,他知情,受了三兩銀子。事發,拿到衙門裡,被我褪衣打了二十板,交他嫁漢子還俗。他怎的還不還俗?好不好,拿來衙門裡再與他幾拶子。」月娘道:「你有要沒緊,恁毀僧傍佛的。他一個佛家弟子,想必善根還在,他平白還甚麼俗?你還不知他好不有道行!」西門慶道:「你問他有道行一夜接幾個漢子?」月娘道:「你就休汗邪!又討我那沒好口的罵你。」因問:「幾時打發他三個起身?」西門慶道:「我剛纔使來保會喬親家去了,他那裡出五百兩,我這裡出五百兩。二十是個好日子,打發他每起身去罷了。」月娘道:「線鋪子卻交誰開?」西門慶道:「且交賁四替他開著罷。」說畢,月娘開箱子拿銀子,一面兌了出來,交付與三人,在捲棚內看著打包。每人又兌五兩銀子,交他家中收拾衣裝行李。

只見應伯爵走到捲棚里,看見便問:「哥打包做甚麼?」西門慶因把二十日打發來保等往揚州支鹽去一節告訴一遍。伯爵舉手道:「哥,恭喜!此去回來必得大利。」西門慶一面讓坐,喚茶來吃。因問:「李三、黃四銀子幾時關?」應伯爵道:「也只在這個月里就關出來了。他昨日對我說,如今東平府又派下二萬香來了,還要問你挪五百兩銀子,接濟他這一時之急。如今關出這批銀子,一分也不動,都抬過這邊來。」西門慶道:「到是你看見,我打發揚州去還沒銀子,問喬親家借了五百兩在裡頭,那討銀子來?」伯爵道:「他再三央及我對你說,一客不煩二主,你不接濟他這一步兒,交他又問那裡借去?」西門慶道:「門外街東徐四鋪少我銀子,我那裡挪五百兩銀子與他罷。」伯爵道:「可知好哩。」正說著,只見平安兒拿進帖兒來,說:「夏老爹家差了夏壽,說請爹明日坐坐。」西門慶看了柬帖,道:「曉得了。」伯爵道:「我有樁事兒來報與哥:你知道李桂兒的勾當麼?他沒來?」西門慶道:「他從正月去了,再幾時來?我並不知道甚麼勾當。」伯爵因說道:「王招宣府里第三的,原來是東京六黃太尉侄女兒女婿。從正月往東京拜年,老公公賞了一千兩銀子,與他兩口兒過節。你還不知六黃太尉這侄女兒生的怎麼標緻,上畫兒只畫半邊兒,也沒恁俊俏相的。你只守著你家裡的罷了,每日被老孫、祝麻子、小張閑三四個摽著在院里撞,把二條巷齊家那小丫頭子齊香兒梳籠了,又在李桂兒家走。把他娘子兒的頭面都拿出來當了。氣的他娘子兒家裡上吊。不想前日老公公生日,他娘子兒到東京只一說,老公公惱了,將這幾個人的名字送與朱太尉,朱太尉批行東平府,著落本縣拿人。昨日把老孫、祝麻子與小張閑都從李桂兒家拿的去了。李桂兒便躲在隔壁朱毛頭家過了一夜。今日說來央及你來了。」西門慶道:「我說正月里都摽著他走,這裡誰人家這銀子,那裡誰人家銀子。那祝麻子還對著我搗生鬼。」說畢,伯爵道:「我去罷。等住回只怕李桂兒來,你管他不管他,他又說我來串作你。」西門慶道:「我還和你說,李三,你且別要許他,等我門外討了銀子來,再和你說話。」伯爵道:「我曉的。」剛走出大門首,只見李桂姐轎子在門首,又早下轎進去了。伯爵去了。

西門慶正吩咐陳敬濟,交他往門外徐四家催銀子去,只見琴童兒走來道:「大娘後邊請,李桂姨來了。」西門慶走到後邊,只見李桂姐身穿茶色衣裳,也不搽臉,用白挑線汗巾子搭著頭,雲鬟不整,花容淹淡,與西門慶磕著頭哭起來,說道:「爹可怎麼樣兒的,恁造化低的營生,正是關著門兒家裡坐,禍從天上來。一個王三官兒,俺每又不認的他。平白的祝麻子、孫寡嘴領了來俺家討茶吃。俺姐姐又不在家,依著我說別要招惹他,那些兒不是,俺這媽越發老的韶刀了。就是來宅里與俺姑娘做生日的這一日,你上轎來了就是了,見祝麻子打旋磨兒跟著,從新又回去,對我說:『姐姐你不出去待他鐘茶兒,卻不難為囂了人?』他便往爹這裡來了。交我把門插了不出來,誰想從外邊撞了一伙人來,把他三個不由分說都拿的去了。王三官兒便奪門走了,我便走在隔壁人家躲了。家裡有個人牙兒!才使來保兒來這裡接的他家去。到家把媽唬的魂都沒了,只要尋死。今日縣裡皂隸,又拿著票喝羅了一清早起去了。如今坐名兒只要我往東京回話去。爹,你老人家不可憐見救救兒,卻怎麼樣兒的?娘也替我說說兒。」西門慶笑道:「你起來。」因問票上還有誰的名字。桂姐道:「還有齊香兒的名字。他梳籠了齊香兒,在他家使錢,他便該當。俺家若見了他一個錢兒,就把眼睛珠子吊了;若是沾他沾身子兒,一個毛孔兒里生一個天皰瘡。」月娘對西門慶道:「也罷,省的他恁說誓剌剌的,你替他說說罷。」 西門慶道:「如今齊香兒拿了不曾?」桂姐道:「齊香兒他在王皇親宅里躲著哩。」西門慶道:「既是恁的,你且在我這裡住兩日。我就差人往縣裡替你說去。」就叫書童兒:「你快寫個帖兒,往縣裡見你李老爹,就說桂姐常在我這裡答應,看怎的免提他罷。」書童應諾,穿青絹衣服去了。不一時,拿了李知縣回貼兒來。書童道:「李老爹說:『多上覆你老爹,別的事無不領命,這個卻是東京上司行下來批文,委本縣拿人,縣裡只拘的人到。既是你老爹分上,我這裡且寬限他兩日。要免提,還往東京上司說去。』」西門慶聽了,只顧沉吟,說道:「如今來保一兩日起身,東京沒人去。」月娘道:「也罷,你打發他兩個先去,存下來保,替桂姐往東京說了這勾當,交他隨後邊趕了去罷。你看唬的他那腔兒。」那桂姐連忙與月娘、西門慶磕頭。

西門慶隨使人叫將來保來,吩咐:「二十日你且不去罷。教他兩個先去。你明日且往東京替桂姐說說這勾當來。見你翟爹,如此這般,好歹差人往衛里說說。」桂姐連忙就與來保下禮。慌的來保頂頭相還,說道:「桂姨,我就去。」西門慶一面教書童兒寫就一封書,致謝翟管家前日曾巡按之事甚是費心,又封了二十兩折節禮銀子,連書交與來保。桂姐便歡喜了,拿出五兩銀子來與來保做盤纏,說道:「回來俺媽還重謝保哥。」西門慶不肯,還了桂姐,教月娘另拿五兩銀子與來保盤纏。桂姐道:「也沒這個道理,我央及爹這裡說人情,又教爹出盤纏。」西門慶道:「你笑話我沒這五兩銀子盤纏了,要你的銀子!」那桂姐方纔收了,向來保拜了又拜,說道:「累保哥,好歹明早起身罷,只怕遲了。」來保道:「我明日早五更就走道兒了。」

於是領了書信,又走到獅子街韓道國家。王六兒正在屋裡縫小衣兒哩,打窗眼看見是來保,忙道:「你有甚說話,請房裡坐。他不在家,往裁縫那裡討衣裳去了,便來也。」便叫錦兒:「還不往對過徐裁家叫你爹去!你說保大爺在這裡。」來保道:「我來說聲,我明日還去不成,又有樁業障鑽出來,當家的留下,教我往東京替院里李桂姐說人情去哩。他剛纔在爹跟前,再三磕頭禮拜央及我。明早就起身了。且教韓伙計和崔大官兒先去,我回來就趕了來。」因問:「嫂子,你做的是甚麼?」王六兒道:「是他的小衣裳兒。」來保道:「你教他少帶衣裳。到那去處是出紗羅緞絹的窩兒里,愁沒衣裳穿!」正說著,韓道國來了。兩個唱了喏,因把前事說了一遍,因說:「我到明日,揚州那裡尋你每?」韓道國道:「老爹吩咐,教俺每馬頭上投經紀王伯儒店裡下。說過世老爹曾和他父親相交,他店內房屋寬廣,下的客商多,放財物不耽心。你只往那裡尋俺每就是了。」來保又說:「嫂子,我明日東京去,你沒甚鞋腳東西捎進府里,與你大姐去?」王六兒道道:「沒甚麼,只有他爹替他打的兩對簪兒,並他兩雙鞋,起動保叔捎捎進去與他。」於是將手帕包袱停當,遞與來保。一面教春香看菜兒篩酒。婦人連忙丟下生活就放桌兒。來保道:「嫂子,你休費心,我不坐。我到家還要收拾褡褳,明日早起身。」王六兒笑嘻嘻道:「耶嚛,你怎的上門怪人家!伙計家,自恁與你餞行,也該吃鐘兒。」因說韓道國:「你好老實!桌兒不穩,你也撒撒兒,讓保叔坐。只象沒事的人兒一般。」於是拿上菜兒來,斟酒遞與來保,王六兒也陪在旁邊,三人坐定吃酒。來保吃了幾鐘,說道:「我家去罷。晚了,只怕家裡關門早。」韓道國問道:「你頭口雇下了不曾?」來保道:「明日早雇罷了。鋪子里鑰匙並帳簿都交與賁四罷了,省的你又上宿去。家裡歇息歇息,好走路兒。」韓道國道:「伙計說的是,我明日就交與他。」王六兒又斟了一甌子,說道:「保叔,你只吃這一鐘,我也不敢留你了。」來保道:「嫂子,你既要我吃,再篩熱著些。」那王六兒連忙歸到壺裡,教錦兒炮熱了,傾在盞內,雙手遞與來保,說道:「沒甚好菜兒與保叔下酒。」來保道:「嫂子好說,家無常禮。」拿起酒來與婦人對飲,一吸同乾,方纔作辭起身。王六兒便把女兒鞋腳遞與他,說道:「累保叔,好歹到府里問聲孩子好不好,我放心些。」兩口兒齊送出門來。

不說來保到家收拾行李,第二日起身東京去了。單表這吳大舅前來對西門慶說:「有東平府行下文書來,派俺本衛兩所掌印千戶管工修理社倉,題準旨意,限六月工完,升一級。違限,聽巡按御史查參。姐夫有銀子借得幾兩,工上使用。待關出工價來,一一奉還。」西門慶道:「大舅用多少,只顧拿去。」吳大舅道:「姐夫下顧,與二十兩罷。」一面同進後邊,見月娘說了話,教月娘拿二十兩出來,交與大舅,又吃了茶。因後邊有堂客,就出來了。月娘教西門慶留大舅大廳上吃酒。正飲酒中間,只見陳敬濟走來,與吳大舅作了揖,就回說:「門外徐四家,稟上爹,還要再讓兩日兒。」西門慶道:「胡說!我這裡等銀子使,照舊還去罵那狗弟子孩兒。」敬濟應諾。吳大舅就讓他打橫坐下,陪著吃酒不題。

且說後邊大妗子、楊姑娘、李嬌兒、孟玉樓、潘金蓮、李瓶兒、大姐,都伴桂姐在月娘房裡吃酒。先是鬱大姐數了一回「張生游寶塔」,放下琵琶。孟玉樓在旁斟酒遞菜兒與他吃,說道:「賊瞎轉磨的唱了這一日,又說我不疼你。」潘金蓮又大箸子夾塊肉放在他鼻子上,戲弄他頑耍。桂姐因叫玉簫姐:「你遞過鬱大姐琵琶來,等我唱個曲兒與姑奶奶和大妗子聽。」月娘道:「桂姐,你心裡熱剌剌的,不唱罷。」桂姐道:「不妨事。見爹娘替我說人情去了,我這回不焦了。」孟玉樓笑道: 「李桂姐倒還是院中人家娃娃,做臉兒快。頭裡一來時,把眉頭忔㥮著,焦的茶兒也吃不下去。這回說也有,笑也有。」當下桂姐輕舒玉指,頓撥冰弦,唱了一回。

正唱著,只見琴童兒收進家活來。月娘便問道:「你大舅去了?」琴童兒道:「大舅去了。」吳大妗子道:「只怕姐夫進來,我每活變活變兒。」琴童道:「爹往五娘房裡去了。」這潘金蓮聽見,就坐不住,趨趄著腳兒只要走,又不好走的。月娘也不等他動身,就說道:「他往你屋裡去了,你去罷。省的你欠肚兒親家是的。」 那潘金蓮嚷:「可可兒的──」起來,口兒里硬著,那腳步兒且是去的快。

來到房裡,西門慶已是吃了胡僧藥,教春梅脫了衣裳,在床上帳子里坐著哩。金蓮看見笑道:「我的兒!今日好呀,不等你娘來就上床了。俺每在後邊吃酒,被李桂姐唱著,灌了我幾鐘好的。獨自一個兒,黑影子里,一步高一步低,不知怎的走來了。」叫春梅:「你有茶倒甌子我吃。」那春梅真個點了茶來。金蓮吃了,努了個嘴與春梅,那春梅就知其意。那邊屋裡早已替他熱下水,婦人抖些檀香白礬在裡面,洗了牝。就燈下摘了頭,止撇著一根金簪子,拿過鏡子來,從新把嘴唇抹了脂胭,口中噙著香茶,走過這邊來。春梅床頭上取過睡鞋來與他換了,帶上房門出去。這婦人便將燈臺挪近旁邊桌上放著,一手放下半邊紗帳子來,褪去紅褲,露出玉體。西門慶坐在枕頭上,那話帶著兩個托子,一霎弄的大大的與他瞧。婦人燈下看見,唬了一跳──一手攥不過來,紫巍巍,沉甸甸──便昵瞅了西門慶一眼,說道:「我猜你沒別的話,一定吃了那和尚藥,弄聳的恁般大,一味要來奈何老娘。好酒好肉,王里長吃的去。你在誰人跟前試了新,這回剩了些殘軍敗將,才來我這屋裡來了。俺每是雌剩雞巴肏的?你還說不偏心哩!嗔道那一日我不在屋裡,三不知把那行貨包子偷的往他屋裡去了。原來晚夕和他乾這個營生,他還對著人撇清搗鬼哩。你這行貨子,乾凈是個沒輓回的三寸貨。想起來,一百年不理你才好。」西門慶笑道:「小淫婦兒,你過來。你若有本事,把他咂過了,我輸一兩銀子與你。」婦人道:「汗邪了你了。你吃了甚麼行貨子,我禁的過他!」於是把身子斜軃在衽席之上,雙手執定那話,用朱唇吞裹。說道:「好大行貨子,把人的口也撐的生疼的。」說畢,出入鳴咂。或舌尖挑弄蛙口,舐其龜弦;或用口噙著,往來哺摔;或在粉臉上擂晃,百般摶弄,那話越發堅硬(扌造)掘起來。

西門慶垂首窺見婦人香肌掩映於紗帳之內,纖手捧定毛都魯那話,往口裡吞放,燈下一往一來。不想旁邊蹲著一個白獅子貓兒,看見動彈,不知當做甚物件兒,撲向前,用爪兒來撾。這西門慶在上,又將手中拿的灑金老鴉扇兒,只顧引逗他耍子。被婦人奪過扇子來,把貓儘力打了一扇靶子,打出帳子外去了。昵向西門慶道: 「怪發訕的冤家!緊著這扎扎的不得人意,又引逗他恁上頭上臉的,一時間撾了人臉卻怎的?好不好我就不乾這營生了。」西門慶道:「怪小淫婦兒,會張致死了!」婦人道:「你怎不叫李瓶兒替你咂來?我這屋裡盡著教你掇弄。不知吃了甚麼行貨子,咂了這一日,益發咂的沒些事兒。」西門慶於是向汗巾上小銀盒兒里,用挑牙挑了些粉紅膏子藥兒,抹在馬口內,仰臥於上,教婦人騎在身上。婦人道:「等我��著,你往裡放。」龜頭昂大,濡研半晌,僅沒龜棱。婦人在上,將身左右捱擦,似有不勝隱忍之態。因叫道:「親達達,裡邊緊澀住了,好不難捱。」一面用手摸之,窺見麈柄已被牝戶吞進半截,撐的兩邊皆滿。婦人用唾津塗抹牝戶兩邊,已而稍寬滑落,頗作往來,一舉一坐,漸沒至根。婦人因向西門慶說:「你每常使的顫聲嬌,在裡頭只是一味熱癢不可當,怎如和尚這藥,使進去,從子宮冷森森直掣到心上,這一回把渾身上下都酥麻了。我曉的今日死在你手裡了。好難捱忍也!」西門慶笑道:「五兒,我有個笑話兒說與你聽──是應二哥說的:一個人死了,閻王就拿驢皮披在身上,教他變驢。落後判官查簿籍,還有他十三年陽壽,又放回來了。他老婆看見渾身都變過來了,只有陽物還是驢的,未變過來,那人道:『我往陰間換去。』他老婆慌了,說道:『我的哥哥,你這一去,只怕不放你回來怎了?等我慢慢兒的挨罷。』」婦人聽了,笑將扇把子打了一下子,說道: 「怪不的應花子的老婆挨慣了驢的行貨。硶說嘴的賊,我不看世界,這一下打的你……」

兩個足纏了一個更次,西門慶精還不過。他在下麵合著眼,由著婦人蹲踞在上極力抽提,提的龜頭刮答刮答怪響。提夠良久,又掉過身子去,朝向西門慶。西門慶雙手舉其股,沒棱露腦而提之,往來甚急。西門慶雖身接目視,而猶如無物。良久,婦人情急,轉過身子來,兩手摟定西門慶脖項,合伏在身上,舒舌頭在他口裡,那話直抵牝中,只顧揉搓,沒口子叫:「親達達,罷了,五兒肏死了!」須臾,一陣昏迷,舌尖冰冷。泄訖一度,西門慶覺牝中一股熱氣直透丹田,心中翕翕然,美快不可言也。已而,淫津溢出,婦人以帕抹之。兩個相摟相抱,交頭疊股,鳴咂其舌,那話通不拽出來。睡的沒半個時辰,婦人淫情未定,爬上身去,兩個又幹起來。婦人一連丟了兩遭身子,亦覺稍倦。西門慶只是佯佯不採,暗想胡僧藥神通。看看窗外雞鳴,東方漸白,婦人道:「我的心肝,你不過卻怎樣的?到晚夕你再來,等我好歹替你咂過了罷。」西門慶道:「就咂也不得過。管情只一樁事兒就過了。」婦人道:「告我說是那一樁兒?」西門慶道:「法不傳六耳,等我晚夕來對你說。」

早晨起來梳洗,春梅打發穿上衣裳。韓道國、崔本又早外邊伺候。西門慶出來燒了紙,打發起身。交付二人兩封書:「一封到揚州馬頭上,投王伯儒店裡下;這一封就往揚州城內抓尋苗青,問他的事情下落,快來回報我。如銀子不夠,我後邊再教來保捎去。」崔本道:「還有蔡老爹書沒有?」西門慶道:「你蔡老爹書還不曾寫,教來保後邊稍了去罷。」二人拜辭,上頭口去了,不在話下。

西門慶冠帶了,就往衙門中來與夏提刑相會,道及昨承見招之意。夏提刑道:「今日奉屈長官一敘,再無他客。」發放已畢,各分散來家。只見一個穿青衣皂隸,騎著快馬,夾著氈包,走的滿面汗流。到大門首,問平安:「此是提刑西門老爹家?」平安道:「你是那裡來的?」那人即便下馬作揖,說:「我是督催皇木的安老爹差來,送禮與老爹。俺老爹與管磚廠黃老爹,如今都往東平府胡老爹那裡吃酒,順便先來拜老爹,看老爹在家不在。」平安道:「有帖兒沒有?」那人向氈包內取出,連禮物都遞與平安。平安拿進去與西門慶看,見禮帖上寫著浙綢二端,湖綿四斤,香帶一束,古鏡一圓。吩咐:「包五錢銀子,拿回帖打發來人,就說在家拱候老爹。」那人急急去了。

西門慶一面預備酒菜,等至日中,二位官員喝道而至,乘轎張蓋甚盛。先令人投拜帖,一個是「侍生安忱拜」,一個是「侍生黃葆光拜」。都是青雲白鷳補子,烏紗皂履,下轎揖讓而入。西門慶出大門迎接,至廳上敘禮,各道契闊之情,分賓主坐下:黃主事居左,安主事居右,西門慶主位相陪。先是黃主事舉手道:「久仰賢名芳譽,學生遲拜。」西門慶道:「不敢!辱承老先生先施枉駕,當容踵叩。敢問尊號?」安主事道:「黃年兄號泰宇,取『履泰定而發天光』之意。」黃主事道: 「敢問尊號?」西門慶道:「學生賤號四泉,──因小莊有四眼井之說。」安主事道:「昨日會見蔡年兄,說他與宋松原都在尊府打攪。」西門慶道:「因承雲峰尊命,又是敝邑公祖,敢不奉迎!小價在京已知鳳翁榮選,未得躬賀。」又問:「幾時起身府上來?」安主事道:「自去歲尊府別後,到家續了親,過了年,正月就來京了。選在工部,備員主事。欽差督運皇木,前往荊州,道經此處,敢不奉謁!」西門慶又說:「盛儀感謝不盡。」說畢,因請寬衣,令左右安放桌席。黃主事就要起身,安主事道:「實告:我與黃年兄,如今還往東平胡太府那裡赴席,因打尊府過,敢不奉謁。容日再來取擾。」西門慶道:「就是往胡公處,去路尚遠,縱二公不餓,其如從者何?學生敢不具酌,只備一飯在此,以犒從者。」於是先打發轎上攢盤。廳上安放桌席。珍羞異品,極時之盛,就是湯飯點心、海鮮美味,一齊上來。西門慶將小金鐘,每人只奉了三杯,連桌兒抬下去,管待親隨家人吏典。少傾,兩位官人拜辭起身,安主事因向西門慶道:「生輩明日有一小東,奉屈賢公到我這黃年兄同僚劉老太監莊上一敘,未審肯命駕否?」西門慶道:「既蒙寵招,敢不趨命!」說畢,送出大門,上轎而去。

只見夏提刑差人來邀。西門慶說道:「我就去。」一面吩咐備馬,走到後邊換了冠帶衣服,出來上馬。玳安、琴童跟隨,排軍喝道,逕往夏提刑家來。到廳上敘禮,說道:「適有工部督催皇木安主政和磚廠黃主政來拜,留坐了半日,方纔去了。不然,也來的早。」說畢,讓至大廳,上面設放兩張桌席,讓西門慶居左,其次就是西賓倪秀才。座間因敘話問道:「老先生尊號?」倪秀才道:「學生賤名倪鵬,字時遠,號桂岩,見在府庠備數,在我這東主夏老先生門下,設館教習賢郎大先生舉業。友道之間,實有多愧。」說話間,兩個小優兒上來磕頭,彈唱飲酒不題。

且說潘金蓮從打發西門慶出來,直睡到晌午才爬起來。甫能起來,又懶待梳頭。恐怕後邊人說他,月娘請他吃飯也不吃,只推不好。大後晌才出房門,來到後邊。月娘因西門慶不在,要聽薛姑子講說佛法,演頌金剛科儀。在明間內安放一張經桌兒,焚下香。薛姑子與王姑子兩個對坐,妙趣、妙鳳兩個徒弟立在兩邊,接念佛號。大妗子、楊姑娘、吳月娘、李嬌兒、孟玉樓、潘金蓮、李瓶兒、孫雪娥和李桂姐眾人,一個不少,都在跟前圍著他坐的,聽他演誦。先是,薛姑子道:

蓋聞電光易滅,石火難消。落花無返樹之期,逝水絕歸源之路。畫堂繡閣,命盡有若長空;極品高官,祿絕猶如作夢。黃金白玉,空為禍患之資;紅粉輕衣,總是塵勞之費。妻孥無百載之歡,黑暗有千重之苦。一朝枕上,命掩黃泉。青史揚虛假之名,黃土埋不堅之骨。田園百頃,其中被兒女爭奪;綾錦千箱,死後無寸絲之分。青春未半,而白髮來侵;賀者才聞,而弔者隨至。苦,苦,苦!氣化清風塵歸土。點點輪迴喚不回,改頭換面無遍數。南無盡虛空遍法界,過去未來佛法僧三寶。無上甚深微妙法,百千萬劫難遭遇。我今見聞得受持,願解如來真實義。

王姑子道:「當時釋迦牟尼佛,乃諸佛之祖,釋教之主,如何出家?願聽演說。」

薛姑子便唱《五供養》:

釋迦佛,梵王子,舍了江山雪山去,割肉喂鷹鵲巢頂。只修的九龍吐水混金身,才成南無大乘大覺釋迦尊。

王姑子又道:「釋迦佛既聽演說,當日觀音菩薩如何修行,才有莊嚴百化化身,有大道力?願聽其說──」

薛姑子正待又唱,只見平安兒慌慌張張走來說道:「巡按宋爺差了兩個快手、一個門子送禮來。」月娘慌了,說道:「你爹往夏家吃酒去了,誰人打發他?」正說著,只見玳安兒回馬來家,放進氈包來,說道:「不打緊,等我拿帖兒對爹說去。教姐夫且請那門子進來,管待他些酒飯兒著。」這玳安交下氈包,拿著帖子,騎馬雲飛般走到夏提刑家,如此這般,說巡按宋老爺送禮來。西門慶看了帖子,上寫著「鮮豬一口,金酒二尊,公紙四刀,小書一部」,下書「侍生宋喬年拜」。連忙吩咐:「到家交書童快拿我的官銜雙摺手本回去,門子答賞他三兩銀子、兩方手帕,抬盒的每人與他五錢。」玳安來家,到處尋書童兒,那裡得來?急的只牛回磨轉。陳敬濟又不在,交傅伙計陪著人吃酒,玳安旋打後邊討了手帕、銀子出來,又沒人封,自家在柜上彌封停當,叫傅伙計寫了,大小三包。因向平安兒道:「你就不知往那去了?」平安道:「頭裡姐夫在家時,他還在家來。落後姐夫往門外討銀子去了,他也不見了。」玳安道:「別要題,一定秫秫小廝在外邊胡行亂走的,養老婆去了。」正在急唣之間,只見陳敬濟與書童兩個,疊騎騾子才來,被玳安罵了幾句,教他寫了官銜手本,打發送禮人去了。玳安道:「賊秫秫小廝,仰��著掙了合蓬著去。爹不在,家裡不看,跟著人養老婆兒去了。爹又沒使你和姐夫門外討銀子,你平白跟了去做甚麼!看我對爹說不說!」書童道:「你說不是,我怕你?你不說就是我的兒。」玳安道:「賊狗攮的秫秫小廝,你賭幾個真個?」走向前,一個潑腳撇翻倒,兩個就骨碌成一塊了。那玳安得手,吐了他一口唾沫才罷了。說道:「我接爹去,等我來家和淫婦算帳。」騎馬一直去了。

月娘在後邊,打發兩個姑子吃了些茶食,又聽他唱佛曲兒,宣念偈子。那潘金蓮不住在旁先拉玉樓不動,又扯李瓶兒,又怕月娘說。月娘便道:「李大姐,他叫你,你和他去不是。省的急的他在這裡恁有擺劃沒是處的。」那李瓶兒方纔同他出來。被月娘瞅了一眼,說道:「拔了蘿蔔地皮寬。交他去了,省的他在這裡跑兔子一般。原不是聽佛法的人。」

在裡頭了。」金蓮搖著頭兒說道:「等我與他罷。」李瓶兒道:「都一答交姐夫捎了來,那又起個窖兒!」敬濟道:「就是連五娘的,這銀子還多著哩。」一面取等子稱稱,一兩九錢。李瓶兒道:「剩下的就與大姑娘捎兩方來。」大姐連忙道了萬福。金蓮道:「你六娘替大姐買了汗巾兒,把那三錢銀子拿出來,你兩口兒鬥葉兒,賭了東道罷。少,便叫你六娘貼些兒出來,明日等你爹不在,買燒鴨子、白酒咱每吃。」敬濟道:「既是五娘說,拿出來。」 大姐遞與金蓮,金蓮交付與李瓶兒收著。拿出紙牌來,燈下大姐與敬濟鬥。金蓮又在旁替大姐指點,登時贏了敬濟三掉。忽聽前邊打門,西門慶來家,金蓮與李瓶兒才回房去了。

敬濟出來迎接西門慶回了話,說徐四家銀子,後日先送二百五十兩來,餘者出月交還。西門慶罵了幾句,酒帶半酣,也不到後邊,逕往金蓮房裡來。正是:

自有內事迎郎意,何怕明朝花不開。

