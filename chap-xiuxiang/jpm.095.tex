%# -*- coding: utf-8 -*-
%!TEX encoding = UTF-8 Unicode
%!TEX TS-program = xelatex
% vim:ts=4:sw=4
%
% 以上设定默认使用 XeLaTex 编译,并指定 Unicode 编码,供 TeXShop 自动识别

%第九十五回 
\chapter{玳安兒竊玉成婚 吳典恩負心被辱}

詩曰:

寺廢僧居少,橋灘客過稀。
家貧奴負主,官懦吏相欺。
水淺魚難住,林稀鳥不棲。
人情皆若此,徒堪悲復凄。

話說孫雪娥在灑家店為娼,不題。卻說吳月娘,自從大姐死了,告了陳敬濟一狀,大家人來昭也死了,他妻子一丈青帶著小鐵棍兒,也嫁人去了。來興兒看守門戶,房中繡春,與了王姑子做徒弟,出家去了。那來興兒自從他媳婦惠秀死了,一向沒有妻室。奶子如意兒,要便引著孝哥兒在他屋裡頑耍,吃東西。來興兒又打酒和奶子吃,兩個嘲勾來去,就刮剌上了,非止一日。但來前邊,歸入後邊就臉紅。月娘察知其事,罵了一頓。家醜不可外揚,與了他一套衣裳,四根簪子,揀了個好日子,就與來興兒完房,做了媳婦了。白日上竈看哥兒,後邊扶持,到夜間往前邊他屋裡睡去。

一日,八月十五日,月娘生日。有吳大妗、二妗子,並三個姑子,都來與月娘做生日,在後邊堂屋裡吃酒。晚夕,都在孟玉樓住的廂房內聽宣捲。到二更時分,中秋兒便在後邊竈上看茶,由著月娘叫,都不應。月娘親自走到上房裡,只見玳安兒正按著小玉在炕上幹得好。看見月娘推門進來,慌的湊手腳不迭。月娘便一聲兒也沒言語,只說得一聲:「臭肉兒,不在後邊看茶去,且在這裡做甚麼哩。」那小玉道:「我叫中秋兒竈上頓茶哩。」低著頭,往後邊去了。玳安便走出儀門,往前邊來。

過了兩日,大妗子、二妗子,三個女僧都家去了。這月娘把來興兒房騰出收拾了,與玳安住。卻教來興兒搬到來昭屋裡,看守大門去了。替玳安做了兩床鋪蓋,一身裝新衣服,盔了一頂新網新帽,做了雙新靴襪;又替小玉編了一頂鬏髻,與了他幾件金銀首飾,四根金頭銀腳簪,環墜戒指之類,兩套段絹衣服,擇日就配與玳安兒做了媳婦。白日里還進來在房中答應,只晚夕臨關儀門時便出去和玳安歇去。這丫頭揀好東好西,甚麼不拿出來和玳安吃?這月娘當看見只推不看見。常言道:「溺愛者不明,貪得者無厭」,「羊酒不均,駟馬奔鎮」,「處家不正,奴婢抱怨」。

卻說平安兒見月娘把小玉配與玳安,衣服穿戴勝似別人。他比玳安倒大兩歲,今年二十二歲,倒不與他妻室。一日在假當鋪,看見傅伙計當了人家一副金頭面,一柄鍍金鉤子,當了三十兩銀子。那家只把銀子使了一個月,加了利錢就來贖討。傅伙計同玳安尋取來,放在鋪子大櫥櫃里。不提防這平安兒見財起心,就連匣兒偷了,走去南瓦子里武長腳家──有兩個私窠子,一個叫薛存兒,一個叫伴兒,在那裡歇了兩夜。忘八見他使錢兒猛大,匣子蹙著金頭面,撅著銀挺子打酒買東西。報與土番,就把他截在屋裡,打了兩個耳刮子就拿了。

也是合當有事,不想吳典恩新升巡簡,騎著馬,頭裡打著一對板子,正從街上過來,看見,問:「拴的甚麼人?」土番跪下稟說:「如此這般,拐帶出來瓦子里宿娼,拿金銀頭面行使。小的可疑,拿了。」吳典恩分付:「與我帶來審問。」一面拿到巡簡廳兒內。吳典恩坐下,兩邊弓皂排列。土番拴平安兒到根前,認的是吳典恩當初是他傢伙計:「已定見了我就放的。」開口就說:「小的是西門慶家平安兒。」吳典恩說:「你既是他家人,拿這金東西在這坊子里做甚麼?」平安道:「小的大娘借與親戚家頭面戴,使小的敢去,來晚了,城門閉了,小的投在坊子,權借宿一夜,不料被土番拿了。」吳典恩罵道:「你這奴才,胡說!你家這般頭面多,金銀廣,教你這奴才把頭面拿出來老婆家歇宿行使?想必是你偷盜出來的。趁早說來,免我動刑!」平安道:「委的親戚家借去頭面,家中大娘使我討去來,並不敢說謊。」吳典恩大怒,罵道:「此奴才真賊,不打如何肯認?」喝令左右:「與我拿夾棍夾這奴才!」一面套上夾棍,夾的小廝猶如殺豬叫,叫道:「爺休夾小的,等小的實說了罷。」吳典恩道:「你只實說,我就不夾你。」平安兒道:「小的偷的假當鋪當的人家一副金頭面,一柄鍍金銀子。」吳典恩問道:「你因甚麼偷出來?」平安道:「小的今年二十二歲,大娘許了替小的娶媳婦兒,不替小的娶。家中使的玳安兒小廝才二十歲,倒把房裡丫頭配與他,完了房。小的因此不憤,才偷出假當鋪這頭面走了。」吳典恩道:「想必是這玳安兒小廝與吳氏有姦,才先把丫頭與他配了。你只實說,沒你的事,我便饒了你。」平安兒道:「小的不知道。」 吳典恩道:「你不實說,與我拶起來。」左右套上拶子,慌的平安兒沒口子說道:「爺休拶小的,等小的說就是了。」吳典恩道:「可又來,你只說了,須沒你的事。」一面放了拶子。那平安說:「委的俺大娘與玳安兒有姦。先要了小玉丫頭,俺大娘看見了,就沒言語,倒與了他許多衣服首飾東西,配與他完房。」這吳典恩一面令吏典上來,抄了他口詞,取了供狀,把平安監在巡簡司,等著出牌,提吳氏、玳安、小玉來,審問這件事。

那日,卻說解當鋪櫥櫃里不見了頭面,把傅伙計唬慌了。問玳安,玳安說:「我在生藥鋪子里吃飯,我不知道。」傅伙計道:「我把頭面匣子放在櫥里,如何不見了?」一地裡尋平安兒尋不著,急的傅伙計插香賭誓。那家子討頭面,傅伙計只推還沒尋出來哩。那人走了幾遍,見沒有頭面,只顧在門前嚷鬧,說:「我當了一個月,本利不少你的,你如何不與我?頭面、鉤子值七八十兩銀子。」傅伙計見平安兒一夜不來家,就知是他偷出去了。四下使人找尋不著,那討頭面主兒又在門首嚷亂。對月娘說,賠他五十兩銀子,那人還不肯,說:「我頭面值六十兩,鉤子連寶石珠子鑲嵌共值十兩,該賠七十兩銀子。」傅伙計又添了他十兩,還不肯,定要與傅伙計合口。正鬧時,有人來報說:「你家平安兒偷了頭面,在南瓦子養老婆,被吳巡簡拿在監里,還不教人快認贓去!」這吳月娘聽見吳典恩做巡簡,「是咱家舊伙計。」一面請吳大舅來商議,連忙寫了領狀,第二日教傅伙計領贓去。有了原物在,省得兩家領。

傅伙計拿狀子到巡簡司,實承望吳典恩看舊時分上,領得頭面出來,不想反被吳典恩老狗奴才儘力罵了頓。叫皂隸拉倒要打,褪去衣裳,把屁脫脫了半日,饒放起來,說道:「你家小廝在這裡供出吳氏與玳安許多姦情來,我這裡申過府縣,還要行牌提取吳氏來對證。你這老狗骨頭,還敢來領贓!」倒吃他千奴才、萬老狗,罵將出來,唬的往家中走不迭。來家不敢隱諱,如此這般,對月娘說了。月娘不聽便罷了,聽了,正是「分開八塊頂梁骨,傾下半桶冰雪來」,慌的手腳麻木。又見那討頭面人,在門前大嚷大鬧,說道:「你家不見了我頭面,又不與我原物,又不賠我銀子,只反哄著我兩頭來回走。今日哄我去領贓,明日等領頭面,端的領的在那裡?這等不合理。」那傅伙計賠下情,將好言央及安撫他:「略從容兩日,就有頭面來了。若無原物,加倍賠你。」那人說:「等我回聲當家的去。」說畢去了。

這吳月娘憂上加憂,眉頭不展。使小廝請吳大舅來商議,教他尋人情對吳典恩說,掩下這樁事罷。吳大舅說:「只怕他不受人情,要些賄賂打點他。」月娘道:「他當初這官,還是咱家照顧他的,還借咱家一百兩銀子,文書俺爹也沒收他的,今日反恩將仇報起來。」吳大舅說:「姐姐,說不的那話了。從來忘恩背義,才一個兒也怎的?」吳月娘道:「累及哥哥,上緊尋個路兒,寧可送他幾十兩銀子罷。領出頭面來還了人家,省得合口費舌。」打發吳大舅吃了飯去了。

月娘送哥哥到大門首,也是合當事情湊巧,只見薛嫂兒提著花箱兒,領著一個小丫頭過來。月娘叫住,便問:「老薛,你往那裡去?怎的一向不來走走?」薛嫂道: 「你老人家到且說的好,這兩日好不忙哩。偏有許多頭緒兒,咱家小奶奶那裡,使牢子大官兒,叫了好幾遍,還不得空兒去哩。」月娘道:「你看媽媽了撒風,他又做起俺小奶奶來了。」薛嫂道:、如今不做小奶奶,倒做了大奶奶了。」月娘道:「他怎的倒大奶奶?」薛嫂道:「你老人家還不知道,他好小造化兒!自從生了哥兒,大奶奶死了,守備老爺就把他扶了正房,做了封贈娘子。正經二奶奶孫氏不如他。手下買了兩個奶子,四個丫頭扶侍。又是兩個房裡得寵學唱的姐兒,都是老爺收用過的。要打時就打,老爺敢做主兒?自恁還恐怕氣了他。那日不知因甚麼,把雪娥娘子打了一頓,把頭髮都撏了,半夜叫我去領出來,賣了八兩銀子。今日我還睡哩,又使牢子叫了我兩遍,教我快往宅里去,問我要兩副大翠重雲子鈿兒,又要一副九鳳鈿兒。先與了我五兩銀子。銀子不知使的那裡去了,還沒送與他生活去哩。這一見了我,還不知怎生罵我哩。」月娘道:「你到後邊,等我瞧瞧怎樣翠鈿兒。」一面讓薛嫂到後邊坐下。薛嫂打開花箱,取出與吳月娘看。只見做的好樣兒,金翠掩映,背面貼金。那個鈿兒,每個鳳口內銜著一掛寶珠牌兒,十分奇巧。薛嫂道:「只這副鈿兒,做著本錢三兩五錢銀子;那副重雲子的,只一兩五錢銀子,還沒尋他的錢。」

正說著,只見玳安走來,對月娘說:「討頭面的又在前邊嚷哩,說等不的領贓,領到幾時?若明日沒頭面,要和傅二叔打了,到個去處理會哩。傅二叔心裡不好,往家去了。那人嚷了回去了。」薛嫂問:「是甚麼勾當?」月娘便長吁了一口氣,如此這般,告訴薛嫂說:「平安兒奴才,偷去印子鋪人家當的一副金頭面,一副鍍金鉤子,走在城外坊子里養老婆,被吳巡簡拿住,監在監里。人家來討頭面沒有,在門前嚷鬧。吳巡簡又勒掯刁難,不容俺家領贓,又要打將伙計來要錢,白尋不出個頭腦來。死了漢子,敗落一齊來,就這等被人欺負,好苦也!」說著那眼中淚紛紛落將下來。

薛嫂道:「好奶奶,放著路兒不會尋。咱家小奶奶,你這裡寫個貼兒,等我對他說聲,教老爺差人分付巡簡司,莫說一副頭面,就十副頭面也討去了。」月娘道: 「周守備,他是武職官,怎管的著那巡簡司?」薛嫂道:「奶奶,你還不知道,如今周爺,朝廷新與他的敕書,好不管的事情寬廣。地方河道,軍馬錢糧,都在他手裡打卯遞手本。又河東水西,捉拿強盜賊情,正在他手裡。」月娘聽了,便道:「既然管著,老薛就累你,多上覆龐大姐說聲。一客不煩二主,教他在周爺面前美言一句兒,問巡簡司討出頭面來。我破五兩銀子謝你。」薛嫂道:「好奶奶,錢恁中使。我見你老人家剛纔凄惶,我到下意不去。你教人寫了帖兒,等我到府里和小奶奶說。成了,隨你老人家;不成,我還來回你老人家話。」這吳月娘一面叫小玉擺茶與薛嫂吃。薛嫂兒道:「不吃罷,你只教大官兒寫了貼兒來,你不知我一身的事哩。」月娘道:「你也出來這半日了,吃了點心兒去。」小玉即便放卓兒,擺上茶食來。月娘陪他吃茶。薛嫂兒遞與丫頭兩個點心吃。月娘問丫頭幾歲了,薛嫂道: 「今年十二歲了。」不一時,玳安前邊寫了說貼兒。薛嫂兒吃了茶,放在袖內,作辭月娘,提著花箱出門,徑到守備府中。

春梅還在暖床上睡著沒起來哩。只見大丫鬟月桂進來說:「老薛來了。」春梅便叫小丫頭翠花,把裡面窗寮開了。日色照的紗窗十分明亮。薛嫂進來說道:「奶奶,這咱還未起來?」放下花箱,便磕下頭去。春梅道:「不當家化化的,磕甚麼頭?」說道:「我心裡不自在,今日起來的遲些。」問道:「你做的翠雲子和九鳳鈿兒拿了來不曾?」薛嫂道:「奶奶,這兩副鈿兒,好不費手!昨日晚夕我才打翠花鋪里討將來,今日要送來,不想奶奶又使了牢子去。」一面取出來,與春梅過目。春梅還嫌翠雲子做的不十分現撇,還放在紙匣兒內,交與月桂收了。看茶與薛嫂兒吃。薛嫂便叫小丫鬟進來,「與奶奶磕頭。」春梅問:「是那裡的?」薛嫂兒道: 「二奶奶和我說了好幾遍,說荷花只做的飯,教我替他尋個小孩兒,學做些針指。我替他領了這個孩子來了。到是鄉裡人家女孩兒,今年才十二歲,正是養材兒。」 春梅道:「你亦發替他尋個城裡孩子,還伶便些。這鄉裡孩子,曉的甚麼?」因問:「這丫頭要多少銀子?」薛嫂兒道:「要不多,只四兩銀子,他老子要投軍使。」春梅叫海棠:「你領到二娘房裡去,明日兌銀子與他罷。」又叫月桂:「大壺內有金華酒,篩來與薛嫂兒燙寒。再有甚點心,拿一盒子與他吃。省得他又說,大清早辰拿寡酒灌他。」

薛嫂道:「桂姐,且不要篩上來,等我和奶奶說了話著,剛纔也吃了些甚麼來了。」春梅道:「你對我說,在誰家?吃甚來?」薛嫂道:「剛纔大娘那頭,留我吃了些甚麼來了。如此這般,望著我好不哭哩。說平安兒小廝,偷了印子鋪內人家當的金頭面,還有一把鍍金鉤子,在外面養老婆,吃番子拿在巡簡司拶打。這裡人家又要頭面嚷亂。那吳巡簡舊日是咱那裡伙計,有爹在日,照顧他的官。今日一旦反面無恩,夾打小廝,攀扯人,又不容這裡領贓。要錢,才把傅伙計打罵將來。唬的伙計不好了,躲的往家去了。央我來,多多上覆你老人家。可憐見,舉眼兒無親的。教你替他對老爺說聲,領出頭面來,交付與人家去了,大娘親來拜謝你老人家。」 春梅問道:「有個貼兒沒有?不打緊,你爺出巡去了,怕不的今晚來家,等我對你爺說。」薛嫂兒道:「他有說貼兒在此。」向袖中取出。春梅看了,順手就放在窗戶臺上。

不一時,托盤內拿上四樣嗄飯菜蔬,月桂拿大銀鐘,滿滿斟了一鐘,流沿兒遞與薛嫂。薛嫂道:「我的奶奶,我怎捱的這大行貨子?」春梅笑道:「比你家老頭子那大貨差些兒。那個你倒捱了,這個你倒捱不的,好歹與我捱了。要不吃,月桂,你與我捏著鼻子灌他。」薛嫂道:「你且拿了點心,與我打個底兒著。」春梅道: 「老媽子,單管說謊。你才說吃了來,這回又說沒打底兒。」薛嫂道:「吃了他兩個茶食,這咱還有哩?」月桂道:「薛媽媽,你且吃了這大鐘酒,我拿點心與你吃。俺奶奶怪我沒用,要打我哩。」這薛嫂沒奈何,只得灌了一鐘,覺心頭小鹿兒劈劈跳起來。那春梅努個嘴兒,又叫海棠斟滿一鐘教他吃。薛嫂推過一邊說:「我的那娘,我卻一點兒也吃不的了。」海棠道:「你老人家捱一月桂姐一下子,不捱我一下子,奶奶要打我。」那薛嫂兒慌的直撅兒跪在地下。春梅道:「也罷,你拿過那餅與他吃了,教他好吃酒。」月桂道:「薛媽媽,誰似我恁疼你,留下恁好玫瑰餡餅兒與你吃。」就拿過一大盤子頂皮酥玫瑰餅兒來。那薛嫂兒只吃了一個,別的春梅都教他袖在袖子里:「到家稍與你家老王八吃。」薛嫂兒吃了酒,蓋著臉兒,把一盤子火薰肉,腌臘鵝,都用草紙包裹,塞在袖內。海棠使氣白賴,又灌了半鐘酒。見他嘔吐上來,才收過傢伙,不要他吃了。春梅分付:「明日來討話說,兌丫頭銀子與你。」臨出門,春梅又分付:「媽媽,你休推聾裝啞,那翠雲子做的不好,明日另帶兩副好的我瞧。」薛嫂道:「我知道。奶奶叫個大姐送我送,看狗咬了我腿。」春梅笑道:「俺家狗都有眼,只咬到骨禿根前就住了。」一面使蘭花送出角門來。

話休饒舌。周守備至日落時分,出巡來家,進入後廳,左右丫鬟接了冠服。進房見了春梅、小衙內,心中歡喜。坐下,月桂、海棠拿茶吃了,將出巡之事告訴一遍。不一時,放桌兒擺飯。飯罷,掌上燭,安排杯酌飲酒。因問:「前邊沒甚事?」春梅一面取過薛嫂拿的貼兒來,與守備看,說吳月娘那邊,如此這般,「小廝平安兒偷了頭面,被吳巡簡拿住監禁,不容領贓。只拷打小廝,攀扯誣賴吳氏姦情,索要銀兩,呈詳府縣」等事。守備看了說:「此事正是我衙門裡事,如何呈詳府縣?吳巡簡那廝這等可惡!我明日出牌,連他都提來發落。」又說:「我聞得吳巡簡是他門下伙計,只因往東京與蔡太題進禮,帶挈他做了這個官,如何倒要誣害他家!」 春梅道:「正是這等說。你替他明日處處罷。」一宿晚景題過。

次日,旋教吳月娘家補了一紙狀,當廳出了大花欄批文,用一個封套裝了。上批:「山東守御府為失盜事,仰巡簡司官連人贓解繳。右差虞侯張勝、李安。準此。」 當下二人領出公文來,先到吳月娘家。月娘管待了酒飯,每人與了一兩銀子鞋腳錢。傅伙計家中睡倒了,吳二舅跟隨到巡簡司。吳巡簡見平安監了兩日,不見西門慶家中人來打點,正教吏典做文書,申呈府縣。只見守御府中兩個公人到了,拿出批文來與他。見封套上朱紅筆標著:「仰巡簡司官連人解繳」,拆開,見裡面吳氏狀子,唬慌了。反賠下情,與李安、張勝每人二兩銀子。隨即做文書解人上去。到於守備府前,伺候半日。待的守備升廳,兩邊軍牢排下,然後帶進入去。這吳巡簡把文書呈遞上去,守備看了一遍,說:「此是我衙門裡事,如何不申解前來?只顧延捱監滯,顯有情弊。」那吳巡簡稟道:「小官才待做文書申呈老爺案下,不料老爺鈞批到了。」守備喝道:「你這狗官可惡!多大官職?這等欺玩法度,抗違上司!我欽奉朝廷敕命,保障地方,巡捕盜賊,提督軍務,兼管河道,職掌開載已明。你如何拿了這件,不行申解,妄用刑杖拷打犯人,誣攀無辜?顯有情弊!」那吳巡簡聽了,摘去冠帽,在階前只顧磕頭。守備道:「本當參治你這狗官,且饒你這遭,下次再若有犯,定行參究。」一面把平安提到廳上,說道:「你這奴才,偷盜了財物,還肆言謗主。人家都是你恁般,也不敢使奴才了。」喝左右:「與我打三十大棍,放了。將贓物封貯,教本家人來領去。」一面喚進吳二舅來,遞了領狀。守備這裡還差張勝拿貼兒同送到西門慶家,見了分上。吳月娘打發張勝酒飯,又與了一兩銀子。走來府里,回了守備、春梅話。

那吳巡簡乾拿了平安兒一場,倒折了好幾兩銀子。月娘還了那人家頭面、鉤子兒。是他原物,一聲兒沒言語去了。傅伙計到家,傷寒病睡倒了,只七日光景,調治不好,嗚呼哀哉死了。月娘見這等合氣,把印子鋪只是收本錢贖討,再不解當出銀子去了。止是教吳二舅同玳安,在門首生藥鋪子日逐轉得來,家中盤纏。此事表過不題。

一日,吳月娘叫將薛嫂兒來,與了三兩銀子。薛嫂道:「不要罷,傳的府里奶奶怪我。」月娘道:「天不使空人,多有累你,我見他不題出來就是了。」於是買下四盤下飯,宰了一口鮮豬,一壇南酒,一匹紵絲尺頭,薛嫂押著來守備府中,致謝春梅。玳安穿著青絹褶兒,拿著禮貼兒,薛嫂領著徑到後堂。春梅出來,戴著金梁冠兒,上穿繡襖,下著錦裙,左右丫鬟養娘侍奉。玳安扒到地下磕頭。春梅分付:「放桌兒,擺茶食與玳安吃。」說道:「沒甚事,你奶奶免了罷。如何又費心送這許多禮來,你周爺已定不肯受。」玳安道:「家奶奶說,前日平安兒這場事,多有累周爺、周奶奶費心,沒甚麼,些少微禮兒,與爺、奶奶賞人罷了。」春梅道:「如何好受的?」薛嫂道:「你老人家若不受,惹那頭又怪我。」春梅一面又請進守備來計較了,止受了豬酒下飯,把尺頭帶回將來了。與了玳安一方手帕,三錢銀子,抬盒人二錢。春梅因問:「你幾時籠起頭去,包了網巾?幾時和小玉完房來?」玳安道:「是八月內來。」春梅道:「到家多頂上你奶奶,多謝了重禮。待要請你奶奶來坐坐,你周爺早晚又出巡去。我到過年正月里,哥兒生日,我往家裡來走走。」玳安道:「你老人家若去,小的到家對俺奶奶說,到那日來接奶奶。」說畢,打發玳安出門。薛嫂便向玳安說:「大官兒,你先去罷,奶奶還要與我說話哩。」那玳安兒押盒擔回家,見了月娘說:「如此這般,春梅姐讓到後邊,管待茶食吃。問了回哥兒好,家中長短。與了我一方手帕,三錢銀子,抬盒人二錢銀子。多頂上奶奶,多謝重禮,都不受來,被薛嫂兒和我再三說了,才受了下飯豬酒,抬回尺頭。要不是請奶奶過去坐坐,一兩日周爺出巡去。他只到過年正月孝哥生日,要來家裡走走。」又告說:「他住著五間正房,穿著錦裙繡襖,戴著金梁冠兒,出落的越發胖大了。手下好少丫頭、奶子侍奉!月娘問:「他其實說明年往咱家來?」玳安兒道:「委實對我說來。」月娘道:「到那日,咱這邊使人接他去。」因問:「薛嫂怎的還不來?」玳安道:「我出門,他還坐著說話,教我先來了。」自此兩家交往不絕。正是:世情看冷暖,人面逐高低。有詩為證:

得失榮枯命里該,皆因年月日時栽。
胸中有志應須至,蠹里無財莫論才。

