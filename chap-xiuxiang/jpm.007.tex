%# -*- coding: utf-8 -*-
%!TEX encoding = UTF-8 Unicode
%!TEX TS-program = xelatex
% vim:ts=4:sw=4
%
% 以上设定默认使用 XeLaTex 编译,并指定 Unicode 编码,供 TeXShop 自动识别

%第七回 
\chapter{薛媒婆說娶孟三兒 楊姑娘氣罵張四舅}

\begin{showcontents}{}


詩曰:

我做媒人實自能,全憑兩腿走慇勤。
唇槍慣把鰥男配,舌劍能調烈女心。
利市花常頭上帶,喜筵餅錠袖中撐。
只有一件不堪處,半是成人半敗人。

話說西門慶家中一個賣翠花的薛嫂兒,提著花廂兒,一地裡尋西門慶不著。因見西門慶貼身使的小廝玳安兒,便問道:「大官人在那裡?」玳安道:「俺爹在鋪子裡和傅二叔算帳。」原來西門慶家開生藥鋪,主管姓傅名銘,字自新,排行第二,因此呼他做傅二叔。這薛嫂聽了,一直走到鋪子門首,掀開簾子,見西門慶正與主管算帳,便點點頭兒,喚他出來。西門慶見是薛嫂兒,連忙撇了主管出來,兩人走在僻靜處說話。西門慶問道:「有甚話說?」薛嫂道:「我有一件親事,來對大官人說,管情中你老人家意,就頂死了的三娘的窩兒,何如?」西門慶道:「你且說這件親事是那家的?」薛嫂道:「這位娘子,說起來你老人家也知道,就是南門外販布楊家的正頭娘子。手裡有一分好錢。南京拔步床也有兩張。四季衣服,插不下手去,也有四五隻箱子。金鐲銀釧不消說,手裡現銀子也有上千兩。好三梭布也有三二百筒。不料他男子漢去販布,死在外邊。他守寡了一年多,身邊又沒子女,止有一個小叔兒,才十歲。青春年少,守他什麼!有他家一個嫡親姑娘,要主張著他嫁人。這娘子今年不上二十五六歲,生的長挑身材,一表人物,打扮起來就是個燈人兒。風流俊俏,百伶百俐,當家立紀、針指女工、雙陸棋子不消說。不瞞大官人說,他娘家姓孟,排行三姐,就住在臭水巷。又會彈一手好月琴,大官人若見了,管情一箭就上垛。」西門慶聽見婦人會彈月琴,便可在他心上,就問薛嫂兒:「既是這等,幾時相會看去?」薛嫂道:「相看到不打緊。我且和你老人家計議:如今他家一家子,只是姑娘大。雖是他娘舅張四,山核桃──差著一隔哩。這婆子原嫁與北邊半邊街徐公公房子裡住的孫歪頭。歪頭死了,這婆子守寡了三四十年,男花女花都無,只靠侄男侄女養活。大官人只倒在他身上求他。這婆子愛的是錢財,明知侄兒媳婦有東西,隨問什麼人家他也不管,只指望要幾兩銀子。大官人家裡有的是那囂段子,拿一段,買上一擔禮物,明日親去見他,再許他幾兩銀子,一拳打倒他。隨問旁邊有人說話,這婆子一力張主,誰敢怎的!」這薛嫂兒一席話,說的西門慶歡從額角眉尖出,喜向腮邊笑臉生。正是:

媒妁慇勤說始終,孟姬愛嫁富家翁。
有緣千里能相會,無緣對面不相逢。

西門慶當日與薛嫂相約下了,明日是好日期,就買禮往他姑娘家去。薛嫂說畢話,提著花廂兒去了。西門慶進來和傅夥計算帳。一宿晚景不題。

到次日,西門慶早起,打選衣帽整齊,拿了一段尺頭,買了四盤羹果,裝做一盒擔,叫人抬了。薛嫂領著,西門慶騎著頭口,小廝跟隨,逕來楊姑娘家門首。薛嫂先入去通報姑娘,說道:「近邊一個財主,要和大娘子說親。我說一家只姑奶奶是大,先來覿面,親見過你老人家,講了話,然後才敢去門外相看。今日小媳婦領來,見在門首伺候。」婆子聽見,便道:「阿呀,保山,你如何不先來說聲!」一面吩咐丫鬟頓下好茶,一面道:「有請。」這薛嫂一力攛掇,先把盒擔抬進去擺下,打發空盒擔出去,就請西門慶進來相見。這西門慶頭戴纏綜大帽,一口一聲只叫:「姑娘請受禮。」讓了半日,婆子受了半禮。分賓主坐下,薛嫂在旁邊打橫。婆子便道:「大官人貴姓?」薛嫂道:「便是咱清河縣數一數二的財主,西門大官人。在縣前開個大生藥鋪,家中錢過北斗,米爛陳倉,沒個當家立紀的娘子。聞得咱家門外大娘子要嫁,特來見姑奶奶講說親事。」婆子道:「官人儻然要說俺侄兒媳婦,自恁來閒講罷了,何必費煩又買禮來,使老身卻之不恭,受之有愧。」西門慶道: 「姑娘在上,沒的禮物,惶恐。」那婆子一面拜了兩拜謝了,收過禮物去,拿茶上來。吃畢,婆子開口道:「老身當言不言謂之懦。我侄兒在時,掙了一分錢財,不幸先死了,如今都落在他手裡,說少也有上千兩銀子東西。官人做小做大我不管你,只要與我侄兒念上個好經。老身便是他親姑娘,又不隔從,就與上我一個棺材本,也不曾要了你家的。我破著老臉,和張四那老狗做臭毛鼠,替你兩個硬張主。娶過門時,遇生辰時節,官人放他來走走,就認俺這門窮親戚,也不過上你窮。」 西門慶笑道:「你老人家放心,所說的話,我小人都知道了。只要你老人家主張得定,休說一個棺材本,就是十個,小人也來得起。」說著,便叫小廝拿過拜匣來,取出六錠三十兩雪花官銀,放在面前,說道:「這個不當什麼,先與你老人家買盞茶吃,到明日娶過門時,還你七十兩銀子、兩匹緞子,與你老人家為送終之資。其四時八節,只管上門行走。」這老虔婆黑眼珠見了二三十兩白晃晃的官銀,滿面堆下笑來,說道:「官人在上,不是老身意小,自古先斷後不亂。」薛嫂在旁插口說:「你老人家忒多心,那裡這等計較!我這大官人不是這等人,只恁還要掇著盒兒認親。你老人家不知,如今知縣知府相公也都來往,好不四海。你老人家能吃他多少?」一席話說的婆子屁滾尿流。吃了兩道茶,西門慶便要起身,婆子挽留不住。薛嫂道:「今日既見了姑奶奶,明日便好往門外相看。」婆子道:「我家侄兒媳婦不用大官人相,保山,你就說我說,不嫁這樣人家,再嫁甚樣人家!」西門慶作辭起身。婆子道:「老身不知大官人下降,匆忙不曾預備,空了官人,休怪。」拄拐送出。送了兩步,西門慶讓回去了。薛嫂打發西門慶上馬,因說道:「我主張的有理麼?你老人家先回去罷,我還在這裡和他說句話。明日須早些往門外去。」西門慶便拿出一兩銀子來,與薛嫂做驢子錢。薛嫂接了,西門慶便上馬來家。他還在楊姑娘家說話飲酒,到日暮才歸家去。

話休饒舌。到次日,西門慶打選衣帽齊整,袖著插戴,騎著匹白馬,玳安、平安兩個小廝跟隨,薛嫂兒騎著驢子,出的南門外來。不多時,到了楊家門首。卻是坐南朝北一間門樓,粉青照壁。薛嫂請西門慶下了馬,同進去。裡面儀門照牆,竹搶籬影壁,院內擺設榴樹盆景,台基上靛缸一溜,打布凳兩條。薛嫂推開朱紅隔扇,三間倒坐客位,上下椅桌光鮮,簾櫳瀟灑。薛嫂請西門慶坐了,一面走入裡邊。片晌出來,向西門慶耳邊說:「大娘子梳妝未了,你老人家請坐一坐。」只見一個小廝兒拿出一盞福仁泡茶來,西門慶吃了。這薛嫂一面指手畫腳與西門慶說:「這家中除了那頭姑娘,只這位娘子是大。雖有他小叔,還小哩,不曉得什麼。當初有過世的官人在鋪子裡,一日不算銀子,銅錢也賣兩大菠籮。毛青鞋面布,俺每問他買,定要三分一尺。一日常有二三十染的吃飯,都是這位娘子主張整理。手下使著兩個丫頭,一個小廝。大丫頭十五歲,吊起頭去了,名喚蘭香。小丫頭名喚小鸞,才十二歲。到明日過門時,都跟他來。我替你老人家說成這親事,指望典兩間房兒住哩。」西門慶道:「這不打緊。」薛嫂道:「你老人家去年買春梅,許我幾匹大布,還沒與我。到明日不管一總謝罷了。」

正說著,只見使了個丫頭來叫薛嫂。不多時,只聞環珮叮咚,蘭麝馥郁,薛嫂忙掀開簾子,婦人出來。西門慶睜眼觀那婦人,但見:

月畫煙描,粉妝玉琢。俊龐兒不肥不瘦,俏身材難減難增。素額逗幾點微麻,天然美麗;緗裙露一雙小腳,周正堪憐。行過處花香細生,坐下時淹然百媚。

西門慶一見滿心歡喜。婦人走到堂下,望上不端不正道了個萬福,就在對面椅子上坐下。西門慶眼不轉睛看了一回,婦人把頭低了。西門慶開言說:「小人妻亡已久,欲娶娘子管理家事,未知尊意如何?」那婦人偷眼看西門慶,見他人物風流,心下已十分中意,遂轉過臉來,問薛婆道:「官人貴庚?沒了娘子多少時了?」西門慶道:「小人虛度二十八歲,不幸先妻沒了一年有餘。不敢請問,娘子青春多少?」婦人道:「奴家是三十歲。」西門慶道:「原來長我二歲。」薛嫂在旁插口道:「妻大兩,黃金日日長。妻大三,黃金積如山。」說著,只見小丫鬟拿出三盞蜜餞金橙子泡茶來。婦人起身,先取頭一盞,用纖手抹去盞邊水漬,遞與西門慶,道個萬福。薛嫂見婦人立起身,就趁空兒輕輕用手掀起婦人裙子來,正露出一對剛三寸、恰半叉、尖尖趫趫金蓮腳來,穿著雙大紅遍地金雲頭白綾高低鞋兒。西門慶看了,滿心歡喜。婦人取第二盞茶來遞與薛嫂。他自取一盞陪坐。吃了茶,西門慶便叫玳安用方盒呈上錦帕二方、寶釵一對、金戒指六個,放在托盤內送過去。薛嫂一面叫婦人拜謝了。因問官人行禮日期:「奴這裡好做預備。」西門慶道:「既蒙娘子見允,今月二十四日,有些微禮過門來。六月初二准娶。」婦人道:「既然如此,奴明日就使人對姑娘說去。」薛嫂道:「大官人昨日已到姑奶奶府上講過話了。」婦人道:「姑娘說甚來?」薛嫂道:「姑奶奶聽見大官人說此椿事,好不喜歡!說道,不嫁這等人家,再嫁那樣人家!我就做硬主媒,保這門親事。」婦人道:「既是姑娘恁般說,又好了。」薛嫂道:「好大娘子,莫不俺做媒敢這等搗謊。」說畢,西門慶作辭起身。

薛嫂送出巷口,向西門慶說道:「看了這娘子,你老人家心下如何?」西門慶道:「薛嫂,其實累了你。」薛嫂道:「你老人家先行一步,我和大娘子說句話就來。」西門慶騎馬進城去了。薛嫂轉來向婦人說道:「娘子,你嫁得這位官人也罷了。」婦人道:「但不知房裡有人沒有人?見作何生理?」薛嫂道:「好奶奶,就有房裡人,那個是成頭腦的?我說是謊,你過去就看出來。他老人家名目,誰不知道,清河縣數一數二的財主,有名賣生藥放官吏債西門慶大官人。知縣知府都和他來往。近日又與東京楊提督結親,都是四門親家,誰人敢惹他!」婦人安排酒飯,與薛嫂兒正吃著,只見他姑娘家使個小廝安童,盒子裡盛著四塊黃米面棗兒糕、兩塊糖、幾十個艾窩窩,就來問:「曾受了那人家插定不曾?奶奶說來:這人家不嫁,待嫁甚人家。」婦人道:「多謝你奶奶掛心。今已留下插定了。」薛嫂道:「天麼,天麼!早是俺媒人不說謊,姑奶奶早說將來了。」婦人收了糕,取出盒子,裝了滿滿一盒子點心臘肉,又與了安童五六十文錢,說:「到家多拜上奶奶。那家日子定在二十四日行禮,出月初二日准娶。」小廝去了。薛嫂道:「姑奶奶家送來什麼?與我些,包了家去孩子吃。」婦人與了他一塊糖、十個艾窩窩,方才出門,不在話下。

且說他母舅張四,倚著他小外甥楊宗保,要圖留婦人東西,一心舉保大街坊尚推官兒子尚舉人為繼室。若小可人家,還有話說,不想聞得是西門慶定了,知他是把持官府的人,遂動不得了。尋思千方百計,不如破為上計。即走來對婦人說:「娘子不該接西門慶插定,還依我嫁尚舉人的是。他是詩禮人家,又有莊田地土,頗過得日子,強如嫁西門慶。那廝積年把持官府,刁徒潑皮。他家見有正頭娘子,乃是吳千戶家女兒,你過去做大是,做小是?況他房裡又有三四個老婆,除沒上頭的丫頭不算。你到他家,人多口多,還有的惹氣哩!」婦人聽見話頭,明知張四是破親之意,便佯說道:「自古船多不礙路。若他家有大娘子,我情願讓他做姐姐。雖然房裡人多,只要丈夫作主,若是丈夫喜歡,多亦何妨。丈夫若不喜歡,便只奴一個也難過日子。況且富貴人家,那家沒有四五個?你老人家不消多慮,奴過去自有道理,料不妨事。」張四道:「不獨這一件。他最慣打婦煞妻,又管挑販人口,稍不中意,就令媒婆賣了。你受得他這氣麼?」婦人道:「四舅,你老人家差矣。男子漢雖利害,不打那勤謹省事之妻。我到他家,把得家定,裡言不出,外言不入,他敢怎的奴?」張四道:「不是我打聽的,他家還有一個十四歲未出嫁的閨女,誠恐去到他家,三窩兩塊惹氣怎了?」婦人道:「四舅說那裡話,奴到他家,大是大,小是小,待得孩兒們好,不怕男子漢不歡喜,不怕女兒們不孝順。休說一個,便是十個也不妨事。」張四道:「還有一件最要緊的事,此人行止欠端,專一在外眠花臥柳。又裡虛外實,少人家債負。只怕坑陷了你。」婦人道:「四舅,你老人家又差矣。他少年人,就外邊做些風流勾當,也是常事。奴婦人家,那裡管得許多?惹說虛實,常言道:世上錢財儻來物,那是長貧久富家?況姻緣事皆前生分定,你老人家到不消這樣費心。」張四見說不動婦人,到吃他搶白了幾句,好無顏色,吃了兩盞清茶,起身去了。有詩為證:

張四無端散楚言,姻緣誰想是前緣。
佳人心愛西門慶,說破咽喉總是閒。

張四羞慚歸家,與婆子商議,單等婦人起身,指著外甥楊宗保,要攔奪婦人箱籠。

話休饒舌。到二十四日,西門慶行了禮。到二十六日,請十二位素僧唸經燒靈,都是他姑娘一力張主。張四到婦人將起身頭一日,請了幾位街坊眾鄰,來和婦人說話。此時薛嫂正引著西門慶家小廝伴當,並守備府裡討的一二十名軍牢,正進來搬抬婦人床帳、嫁妝箱籠。被張四攔住說道:「保山且休抬!有話講。」一面同了街坊鄰舍進來見婦人。坐下,張四先開言說:「列位高鄰聽著:大娘子在這裡,不該我張龍說,你家男子漢楊宗錫與你這小叔楊宗保,都是我甥。今日不幸大外甥死了,空掙一場錢。有人主張著你,這也罷了。爭奈第二個外甥楊宗保年幼,一個業障都在我身上。他是你男子漢一母同胞所生,莫不家當沒他的份兒?今日對著列位高鄰在這裡,只把你箱籠打開,眼同眾人看一看,有東西沒東西,大家見個明白。」婦人聽言,一面哭起來,說道:「眾位聽著,你老人家差矣!奴不是歹意謀死了男子漢,今日添羞臉又嫁人。他手裡有錢沒錢,人所共知,就是積攢了幾兩銀子,都使在這房子上。房子我沒帶去,都留與小叔。家活等件,分毫不動。就是外邊有三四百兩銀子欠帳,文書合同已都交與你老人家,陸續討來家中盤纏。再有什麼銀兩來?」張四道:「你沒銀兩也罷。如今只對著眾位打開箱籠看一看。就有,你還拿了去,我又不要你的。」婦人道:「莫不奴的鞋腳也要瞧不成?」正亂著,只見姑娘拄拐自後而出。眾人便道:「姑娘出來。」都齊聲唱喏。姑娘還了萬福,陪眾人坐下。姑娘開口道:「列位高鄰在上,我是他是親姑娘,又不隔從,莫不沒我說處?死了的也是侄兒,活著的也是侄兒,十個指頭咬著都疼。如今休說他男子漢手裡沒錢,他就有十萬兩銀子,你只好看他一眼罷了。他身邊又無出,少女嫩婦的,你攔著不教他嫁人做什麼?」眾街鄰高聲道:「姑娘見得有理!」婆子道:「難道他娘家陪的東西,也留下他的不成?他背地又不曾自與我什麼,說我護他,也要公道。不瞞列位說,我這侄兒媳婦平日有仁義,老身捨不得他,好溫克性兒。不然老身也不管著他。」那張四在旁把婆子瞅了一眼,說道:「你好公平心兒!鳳凰無寶處不落。」只這一句話道著婆子真病,登時怒起,紫漲了面皮,指定張四大罵道: 「張四,你休胡言亂語!我雖不能是楊家正頭香主,你這老油嘴,是楊家那膫子肏的?」張四道:「我雖是異姓,兩個外甥是我姐姐養的,你這老咬蟲,女生外向,怎一頭放火,又一頭放水?」姑娘道:「賤沒廉恥老狗骨頭!他少女嫩婦的,你留他在屋裡,有何算計?既不是圖色慾,便欲起謀心,將錢肥己。」張四道: 「我不是圖錢,只恐楊宗保後來大了,過不得日子。不似你這老殺才,搬著大引著小,黃貓兒黑尾。」姑娘道:「張四,你這老花根,老奴才,老粉嘴,你恁騙口張舌的好淡扯,到明日死了時,不使了繩子扛子。」張四道:「你這嚼舌頭老淫婦,掙將錢來焦尾靶,怪不得你無兒無女。」姑娘急了,罵道:「張四,賊老蒼根,老豬狗,我無兒無女,強似你家媽媽子穿寺院,養和尚,肏道士,你還在睡夢裡。」當下兩個差些兒不曾打起來,多虧眾鄰舍勸住,說道:「老舅,你讓姑娘一句兒罷。」薛嫂兒見他二人嚷做一團,領西門慶家小廝伴當,並發來眾軍牢,趕人鬧裡,七手八腳將婦人床帳、妝奩、箱籠,扛的扛,抬的抬,一陣風都搬去了。那張四氣的眼大睜著,半晌說不出話來。眾鄰舍見不是事,安撫了一回,各人都散了。

到六月初二日,西門慶一頂大轎,四對紅紗燈籠,他小叔楊宗保頭上紮著髻兒,穿著青紗衣,撒騎在馬上,送他嫂子成親。西門慶答賀了他一匹錦緞、一柄玉絛兒。蘭香、小鸞兩個丫頭,都跟了來鋪床疊被。小廝琴童方年十五歲,亦帶過來伏侍。到三日,楊姑娘家並婦人兩個嫂子孟大嫂、二嫂都來做生日。西門慶與他楊姑娘七十兩銀子、兩匹尺頭。自此親戚來往不絕。西門慶就把西廂房裡收拾三間,與他做房。排行第三,號玉樓,令家中大小都隨著叫三姨。到晚一連在他房中歇了三夜。正是:銷金帳裡,依然兩個新人;紅錦被中,現出兩般舊物。有詩為證:

怎睹多情風月標,教人無福也難消。
風吹列子歸何處,夜夜嬋娟在柳梢。




\piWenlongF{
文禹门云:批书者,总以玉楼为作者自况,不知从何处看出,而一口咬定,惟恐旁人不理会,时时点出,是可怪也。夫玉楼诚不愧为佳人,然亦有不满人意处。夫死不满两年,家资颇颇过得,宗保劝;是乃夫胞弟,纵不能守,亦何必如此其亟,且又若此之草草也。岂一见西门庆,便魂飞魄散,如潘金莲不能自主,如李瓶儿不能自由耶/妇人急色若斯,便非善良。做大做小,亦需探听明白,杨(张)四之言不足信,有名有姓有则有势之西门大官人,一访便知。纵然谋死人家亲夫,事未宜布;彼月娘尚在,为吴千户家女儿,琴童虽幼,亦可访问出来。不能做大,且不做老二,抑屈于妓女之下,岂玉楼之初心乎?然何以一见便收插定也,谓非急色得乎?
  “贞节”二字,扣定妇人女子,未免头巾气。但有财如此,有貌如此,人皆仰而望之。乃一见一个白净小伙,便以终身相许,虽非蠢妇人,亦是丑妇人,作者何取乎而以之自况也?或日;玉楼为媒人所误耳,是诚然矣。自古英雄志士,一误不能翻身,正自不一,矧一玉楼乎?玉楼不知而嫁之,为玉楼惜可也。若作者明知西门庆不是东西,既自以为玉楼,又何必定嫁西门,为终身之玷乎?岂作者亦尝为仇人门下士乎?自比妇人,自比再蘸之寡妇,自比误嫁匪类之粗愚而美艳之妇人,果有其事,不得不振笔直书,凭空结构,我操其权,何必作此无味狡狯乎?我固谓所批有然,有不然。
\footnote{按:前评应写于光绪五年(1879)五月十一日于南陵县署以约小屋中。}

}

\piWenlongF{
文禹门又云:玉楼之未过门也,心满意足;玉楼既过门也,水落石出,月娘在上,娇儿在旁,岂无目者,而能默然乎?此正作者漏洞处,亦正作者讨巧处。若写得太重,便失玉楼性情,若写得太轻,又非当时景况。故但以三日后“来往不绝”,含糊了之。阅者万勿被他瞒过,遂谓此等事,作亦无妨,而误尽苍生也。须于无文字中求之,此两日内,有大不顺心,大不快活,许多事情,包藏其中。从此家反宅乱,从此家败人亡,皆在此一关头上。吁嗟乎! 《金瓶梅》之误人,正在此而不在彼也。】按:后评则当写于光绪六年(1880)。
\footnote{按:“总以玉楼为作者自况”,系指张竹坡原评:“至其写玉楼一人,则又作者经济学问,色色自喻皆到。”}

}



\end{showcontents}

