%# -*- coding: utf-8 -*-
%!TEX encoding = UTF-8 Unicode
%!TEX TS-program = xelatex
% vim:ts=4:sw=4
%
% 以上设定默认使用 XeLaTex 编译,并指定 Unicode 编码,供 TeXShop 自动识别

%第六十一回 
\chapter{西門慶乘醉燒陰戶 李瓶兒帶病宴重陽}

詞曰:

蛩聲泣露驚秋枕,淚濕鴛鴦錦。獨臥玉肌涼,殘更與恨長。陰風翻翠幌,雨澀燈花暗。畢竟不成眠,鴉啼金井寒。

話說一日,韓道國鋪中回家,睡到半夜,他老婆王六兒與他商議道:「你我被他照顧,掙了恁些錢,也該擺席酒兒請他來坐坐。況他又丟了孩兒,只當與他釋悶,他能吃多少!彼此好看。就是後生小郎看著,到明日南邊去,也知財主和你我親厚,比別人不同。」韓道國道:「我心裡也是這等說。明日初五日是月忌,不好。到初六日,安排酒席,叫兩個唱的,具個柬帖,等我親自到宅內,請老爹散悶坐坐。我晚夕便往鋪子里睡去。」王六兒道:「平白又叫甚麼唱的?只怕他酒後要來這屋裡坐坐,不方便。隔壁樂三嫂家,常走的一個女兒申二姐,年紀小小的,且會唱,他又是瞽目的,請將他來唱唱罷。要打發他過去還容易。」韓道國道:「你說的是。」一宿晚景題過。

到次日,韓道國走到鋪子里,央及溫秀才寫了個請柬兒,親見西門慶,聲喏畢,說道:「明日,小人家裡治了一杯水酒,無事請老爹貴步下臨,散悶坐一日。」因把請柬遞上去。西門慶看了,說道:「你如何又費此心。我明日倒沒事,衙門中回家就去。」韓道國作辭出門。到次早,拿銀子叫後生胡秀買嗄飯菜蔬,一面叫廚子整理,又拿轎子接了申二姐來,王六兒同丫鬟伺候下好茶好水,單等西門慶來到。等到午後,只見琴童兒先送了一壇葡萄酒來,然後西門慶坐著涼轎,玳安、王經跟隨,到門首下轎,頭戴忠靖冠,身穿青水緯羅直身,粉頭皂靴。韓道國迎接入內,見畢禮數,說道:「又多謝老爹賜將酒來。」正面獨獨安放一張交椅,西門慶坐下。

不一時,王六兒打扮出來,與西門慶磕了四個頭,回後邊看茶去了。須臾,王經拿出茶來,韓道國先取一盞,舉的高高的奉與西門慶,然後自取一盞,旁邊相陪。吃畢,王經接了茶盞下去,韓道國便開言說道:「小人承老爹莫大之恩,一向在外,家中小媳婦承老爹看顧,王經又蒙抬舉,叫在宅中答應,感恩不淺。前日哥兒沒了,雖然小人在那裡,媳婦兒因感了些風寒,不曾往宅里弔問的,恐怕老爹惱。今日,一者請老爹解解悶,二者就恕俺兩口兒罪。」西門慶道:「無事又教你兩口兒費心。」說著,只見王六兒也在旁邊坐下。因向韓道國道:「你和老爹說了不?」道國道:「我還不曾說哩。」西門慶問道:「是甚麼?」王六兒道:「他今日要內邊請兩位姐兒來伏侍老爹,我恐怕不方便,故不去請。隔壁樂家常走的一個女兒,叫做申二姐,諸般大小時樣曲兒,連數落都會唱。我前日在宅里,見那一位鬱大姐唱的也中中的,還不如這申二姐唱的好。教我今日請了他來,唱與爹聽。未知你老人家心下何如?若好,到明日叫了宅里去,唱與他娘每聽。」西門慶道:「既是有女兒,亦發好了。你請出來我看看。」不一時,韓道國叫玳安上來:「替老爹寬去衣服。」一面安放桌席,胡秀拿果菜案酒上來。王六兒把酒打開,燙熱了,在旁執壺,道國把盞,與西門慶安席坐下,然後才叫出申二姐來。西門慶睜眼觀看,見他高髻雲鬟,插著幾枝稀稀花翠,淡淡釵梳,綠襖紅裙,顯一對金蓮趫趫;桃腮粉臉,抽兩道細細春山。望上與西門慶磕了四個頭。西門慶便道:「請起。你今青春多少?」申二姐道:「小的二十一歲了。」又問:「你記得多少唱?」申二姐道: 「大小也記百十套曲子。」西門慶令韓道國旁邊安下個坐兒與他坐。申二姐向前行畢禮,方纔坐下。先拿箏來唱了一套《秋香亭》,然後吃了湯飯,添換上來,又唱了一套《半萬賊兵》。落後酒闌上來,西門慶吩咐:「把箏拿過去,取琵琶與他,等他唱小詞兒我聽罷。」那申二姐一逕要施逞他能彈會唱。一面輕搖羅袖,款跨鮫綃,頓開喉音,把弦兒放得低低的,彈了個《四不應•山坡羊》。唱完了,韓道國教渾家滿斟一盞,遞與西門慶。王六兒因說:「申二姐,你還有好《鎖南枝》,唱兩個與老爹聽。」那申二姐就改了調兒,唱《鎖南枝》道:

初相會,可意人,年少青春,不上二旬。黑鬖鬖兩朵烏雲,紅馥馥一點朱唇,臉賽夭桃如嫩筍。若生在畫閣蘭堂,端的也有個夫人分。可惜在章台,出落做下品。但能夠改嫁從良,勝強似棄舊迎新。

初相會,可意嬌,月貌花容,風塵中最少。瘦腰肢一捻堪描,俏心腸百事難學,恨只恨和他相逢不早。常則怨席上樽前,淺斟低唱相偎抱。一覷一個真,一看一個飽。雖然是半霎歡娛,權且將悶解愁消。

西門慶聽了這兩個《鎖南枝》,正打著他初請了鄭月兒那一節事來,心中甚喜。王六兒滿滿的又斟上一盞,笑嘻嘻說道:「爹,你慢慢兒的飲,申二姐這個才是零頭兒,他還記的好些小令兒哩。到明日閑了,拿轎子接了,唱與他娘每聽,管情比鬱大姐唱的高。」西門慶因說:「申二姐,我重陽那日,使人來接你,去不去?」申二姐道:「老爹說那裡話,但呼喚,怎敢違阻!」西門慶聽見他說話伶俐,心中大喜。

不一時,交杯換盞之間,王六兒恐席間說話不方便,叫他唱了幾套,悄悄向韓道國說:「教小廝招弟兒,送過樂三嫂家歇去罷。」臨去拜辭,西門慶向袖中掏出一包兒三錢銀子,賞他買弦。申二姐連忙嗑頭謝了。西門慶約下:「我初八日使人請你去。」王六兒道:「爹只使王經來對我說,等我這裡教小廝請他去。」說畢,申二姐往隔壁去了。韓道國與老婆說知,也就往鋪子里睡去了。只落下老婆在席上,陪西門慶擲骰飲酒。吃了一回,兩個看看吃的涎將上來,西門慶推起身更衣,就走入婦人房裡,兩個頂門頑耍。王經便把燈燭拿出來,在前半間和玳安、琴童兒做一處飲酒。

那後生胡秀,在廚下偷吃了幾碗酒,打發廚子去了,走在王六兒隔壁供養佛祖先堂內,地下鋪著一領席,就睡著了。睡了一覺起來,忽聽見婦人房裡聲喚,又見板壁縫裡透過燈亮來,只道西門慶去了,韓道國在房中宿歇。暗暗用頭上簪子刺破板縫中糊的紙,往那邊張看。見那邊房中亮騰騰點著燈燭,不想西門慶和老婆在屋裡正幹得好。伶伶俐俐看見,把老婆兩隻腿,卻是用腳帶弔在床頭上,西門慶上身止著一件綾襖兒,下身赤露,就在床沿上一來一往,一動一靜,扇打的連聲響亮,老婆口裡百般言語都叫將出來。良久,只聽老婆說:「我的親達!你要燒淫婦,隨你心裡揀著那塊只顧燒,淫婦不敢攔你。左右淫婦的身子屬了你,怕那些兒了!」西門慶道:「只怕你家裡的嗔是的。」老婆道:「那忘八七個頭八個膽,他敢嗔!他靠著那裡過日子哩?」西門慶道:「你既一心在我身上,等這遭打發他和來保起身,亦發留他長遠在南邊,做個買手置貨罷。」老婆道:「等走過兩遭兒,卻教他去。省的閑著在家做甚麼?他說倒在外邊走慣了,一心只要外邊去。你若下顧他,可知好哩!等他回來,我房裡替他尋下一個,我也不要他,一心撲在你身上,隨你把我安插在那裡就是了。我若說一句假,把淫婦不值錢身子就爛化了。」西門慶道: 「我兒,你快休賭誓!」兩個一動一靜,都被胡秀聽了個不亦樂乎。

韓道國先在家中不見胡秀,只說往鋪子里睡去了。走到緞子鋪里,問王顯、榮海,說他沒來。韓道國一面又走回家,叫開門,前後尋胡秀,那裡得來,只見王經陪玳安、琴童三個在前邊吃酒。胡秀聽見他的語音來家,連忙倒在席上,又推睡了。不一時,韓道國點燈尋到佛堂地下,看見他鼻口內打鼾睡,用腳踢醒,罵道:「賊野狗死囚,還不起來!我只說先往鋪子里睡去,你原來在這裡挺得好覺兒。還不起來跟我去!」那胡秀起來,推揉了揉眼,楞楞睜睜跟道國往鋪子里去了。

西門慶弄老婆,直弄夠有一個時辰,方纔了事。燒了王六兒心口裡並毴蓋子上、尾亭骨兒上共三處香。老婆起來穿了衣服,教丫頭打發舀水凈了手,重篩暖酒,再上佳餚,情話攀盤。又吃了幾鐘,方纔起身上馬,玳安、王經、琴童三個跟著。到家中已有二更天氣,走到李瓶兒房中。李瓶兒睡在床上,見他吃的酣酣兒的進來,說道:「你今日在誰家吃酒來?」西門慶道:「韓道國家請我。見我丟了孩子,與我釋悶。他叫了個女先生申二姐來,年紀小小,好不會唱!又不說鬱大姐。等到明日重陽,使小廝拿轎子接他來家,唱兩日你每聽,就與你解解悶。你緊心裡不好,休要只顧思想他了。」說著,就要叫迎春來脫衣裳,和李瓶兒睡。李瓶兒道:「你沒的說!我下邊不住的長流,丫頭替我煎藥哩。你往別人屋裡睡去罷。你看著我成日好模樣兒罷了,只有一口游氣兒在這裡,又來纏我起來。」西門慶道:「我的心肝!我心裡舍不的你。只要和你睡,如之奈何?」李瓶兒瞟了他一眼,笑了笑兒:「誰信你那虛嘴掠舌的。我倒明日死了,你也舍不的我罷!」又道:「亦發等我好好兒,你再進來和我睡也不遲。」西門慶坐了一回,說道:「罷,罷。你不留我,等我往潘六兒那邊睡去罷。」李瓶兒道:「原來你去,省的屈著你那心腸兒。他那裡正等的你火里火發,你不去,卻忙惚兒來我這屋裡纏。」西門慶道:「你恁說,我又不去了。」李瓶兒微笑道:「我哄你哩,你去罷。」於是打發西門慶過去了。李瓶兒起來,坐在床上,迎春伺候他吃藥。拿起那藥來,止不住撲簌簌香腮邊滾下淚來,長吁了一口氣,方纔吃了那盞藥。正是:

心中無限傷心事,付與黃鸝叫幾聲。

不說李瓶兒吃藥睡了,單表西門慶到於潘金蓮房裡。金蓮才叫春梅罩了燈上床睡下。忽見西門慶推開門進來便道:「我兒,又早睡了?」金蓮道:「稀幸!那陣風兒刮你到我這屋裡來!」因問:「你今日往誰家吃酒去來?」西門慶道:「韓伙計打南邊來,見我沒了孩子,一者與我釋悶,二者照顧他外邊走了這遭,請我坐坐。」 金蓮道:「他便在外邊,你在家又照顧他老婆了。」西門慶道:「伙計家,那裡有這道理?」婦人道:「伙計家,有這個道理!齊腰拴著根線兒,只怕肏過界兒去了。你還搗鬼哄俺每哩,俺每知道的不耐煩了!你生日,賊淫婦他沒在這裡?你悄悄把李瓶兒壽字簪子,黃貓黑尾偷與他,卻叫他戴了來施展。大娘、孟三兒,這一家子那個沒看見?吃我問了一句,他把臉兒都紅了,他沒告訴你?今日又摸到那裡去,賊沒廉恥的貨,一個大摔瓜長淫婦,喬眉喬樣,描的那水髩長長的,搽的那嘴唇鮮紅的──倒象人家那血毴。甚麼好老婆,一個大紫腔色黑淫婦,我不知你喜歡他那些兒!嗔道把忘八舅子也招惹將來,一早一晚教他好往回傳話兒。」西門慶堅執不認,笑道:「怪小奴才兒,單管只胡說,那裡有此勾當?今日他男子漢陪我坐,他又沒出來。」婦人道:「你拿這個話兒來哄我?誰不知他漢子是個明忘八,又放羊,又拾柴,一徑把老婆丟與你,圖你家買賣做,要賺你的錢使。你這傻行貨子,只好四十里聽銃響罷了!」西門慶脫了衣裳,坐在床沿上,婦人探出手來,把褲子扯開,摸見那話軟叮噹的,托子還帶在上面,說道:「可又來,你臘鴨子煮到鍋里──身子兒爛了,嘴頭兒還硬。見放著不語先生在這裡,強盜和那淫婦怎麼弄聳,聳到這咱晚才來家?弄的恁個樣兒,嘴頭兒還強哩!你賭個誓,我叫春梅舀一甌子涼水,你只吃了,我就算你好膽子。論起來,鹽也是這般咸,醋也是這般酸,禿子包網中──饒這一抿子兒也罷了。若是信著你意兒,把天下老婆都耍遍了罷。賊沒羞的貨,一個大眼裡火行貨子!你早是個漢子,若是個老婆,就養遍街,肏遍巷。」幾句說的西門慶睜睜的,只是笑。

上的床來,叫春梅篩熱了燒酒,把金穿心盒兒內藥拈了一粒,放在口裡咽下去,仰臥在枕上,令婦人:「我兒,你下去替你達品,品起來是你造化。」那婦人一徑做喬張致,便道:「好乾凈兒!你在那淫婦窟窿子里鑽了來,教我替你咂,可不臢殺了我!」西門慶道:「怪小淫婦兒,單管胡說白道的,那裡有此勾當?」婦人道: 「那裡有此勾當?你指著肉身子賭個誓麼!」亂了一回,教西門慶下去使水,西門慶不肯下去,婦人旋向袖子里掏出個汗巾來,將那話抹展了一回,方纔用朱唇裹沒。嗚咂半晌,咂弄的那話奢棱跳腦,暴怒起來,乃騎在婦人身上,縱麈柄自後插入牝中,兩手兜其股,蹲踞而擺之,肆行扇打,連聲響亮。燈光之下,窺玩其出入之勢,婦人倒伏在枕畔,舉股迎湊者久之。西門慶興猶不愜,將婦人仰臥朝上,那話上使了粉紅藥兒,頂入去,執其雙足,又舉腰沒棱露腦掀騰者將二三百度。婦人禁受不的,瞑目顫聲,沒口子叫:「達達,你這遭兒只當將就我,不使上他也罷了。」西門慶口中呼叫道:「小淫婦兒,你怕我不怕?再敢無禮不敢?」婦人道: 「我的達達,罷麼,你將就我些兒,我再不敢了!達達慢慢提,看提散了我的頭髮。」兩個顛鴛倒鳳,足狂了半夜,方纔體倦而寢。

話休饒舌,又早到重陽令節。西門慶對吳月娘說:「韓伙計前日請我,一個唱的申二姐,生的人材又好,又會唱。我使小廝接他來,留他兩日,教他唱與你每聽。」又吩咐廚下收拾餚饌果酒,在花園大卷棚聚景堂內,安放大八仙桌,合家宅眷,慶賞重陽。

不一時,王經轎子接的申二姐到了。入到後邊,與月娘眾人磕了頭。月娘見他年小,生的好模樣兒。問他套數,也會不多,諸般小曲兒倒記的有好些。一面打發他吃了茶食,先教在後邊唱了兩套,然後花園擺下酒席。那日,西門慶不曾往衙門中去,在家看著栽了菊花。請了月娘、李嬌兒、孟玉樓、潘金蓮、李瓶兒、孫雪娥並大姐,都在席上坐的。春梅、玉簫、迎春、蘭香在旁斟酒伏侍。申二姐先拿琵琶在旁彈唱。那李瓶兒在房中,因身上不方便,請了半日才來。恰似風兒颳倒的一般,強打著精神陪西門慶坐,眾人讓他酒兒也不大吃。西門慶和月娘見他面帶憂容,眉頭不展,說道:「李大姐,你把心放開,教申二姐彈唱曲兒你聽。」玉樓道:「你說與他,教他唱甚麼曲兒,他好唱。」李瓶兒只顧不說。正飲酒中間,忽見王經走來說道:「應二爹、常二叔來了。」西門慶道:「請你應二爹、常二叔在小捲棚內坐,我就來。」王經道:「常二叔教人拿了兩個盒子在外頭。」西門慶向月娘道:「此是他成了房子,買禮來謝我的意思。」月娘道:「少不的安排些甚麼管待他,怎好空了他去!你陪他坐去,我這裡吩咐看菜兒。」西門慶臨出來,又叫申二姐:「你唱個好曲兒,與你六娘聽。」一直往前邊去了。金蓮道:「也沒見這李大姐,隨你心裡說個甚麼曲兒,教申二姐唱就是了,辜負他爹的心!為你叫將他來,你又不言語。」催逼的李瓶兒急了,半日才說出來:「你唱個『紫陌紅塵』罷。」那申二姐道:「這個不打緊,我有。」於是取過箏來,頓開喉音,細細唱了一套。唱畢,吳月娘道:「李大姐,好甜酒兒,你吃上一鐘兒。」李瓶兒又不敢違阻,拿起鐘兒來咽了一口兒,又放下了。坐不多時,下邊一陣熱熱的來,又往屋裡去了,不題。

且說西門慶到於小捲棚翡翠軒,只見應伯爵與常峙節在松牆下正看菊花。原來松牆兩邊,擺放二十盆,都是七尺高,各樣有名的菊花,也有大紅袍、狀元紅、紫袍金帶、白粉西、黃粉西、滿天星、醉楊妃、玉牡丹、鵝毛菊、鴛鴦花之類。西門慶出來,二人向前作揖。常峙節即喚跟來人,把盒兒掇進來。西門慶一見便問:「又是甚麼?」伯爵道:「常二哥蒙哥厚情,成了房子,無可酬答,教他娘子製造了這螃蟹鮮並兩隻爐燒鴨兒,邀我來和哥坐坐。」西門慶道:「常二哥,你又費這個心做甚麼?你令正病才好些,你又禁害他!」伯爵道:「我也是恁說。他說道別的東西兒來,恐怕哥不稀罕。」西門慶令左右打開盒兒觀看:四十個大螃蟹,都是剔剝凈了的,裡邊釀著肉,外用椒料薑蒜米兒團粉裹就,香油煠,醬油醋造過,香噴噴,酥脆好食。又是兩大隻院中爐燒熟鴨。西門慶看了,即令春鴻、王經掇進去,吩咐拿五十文錢賞拿盒人,因向常峙節謝了。

琴童在旁掀簾,請入翡翠軒坐。伯爵只顧誇獎不盡好菊花,問:「哥是那裡尋的?」西門慶道:「是管磚廠劉太監送的。這二十盆,就連盆都送與我了。」伯爵道: 「花到不打緊,這盆正是官窯雙箍鄧漿盆,都是用絹羅打,用腳跐過泥,才燒造這個物兒,與蘇州鄧漿磚一個樣兒做法。如今那裡尋去!」誇了一回。西門慶喚茶來吃了,因問:「常二哥幾時搬過去?」伯爵道:「從兌了銀子三日就搬過去了。昨見好日子,買了些雜貨兒,門首把鋪兒也開了。就是常二嫂兄弟,替他在鋪里看銀子兒。」西門慶道:「俺每幾時買些禮兒,休要人多了,再邀謝子純你三四位,我家裡整理菜兒抬了去──休費煩常二哥一些東西──叫兩個妓者,咱每替他暖暖房,耍一日。」常峙節道:「小弟有心也要請哥坐坐,算計來不敢請。地方兒窄狹,只怕褻瀆了哥。」西門慶道:「沒的扯淡,那裡又費你的事起來。如今使小廝請將謝子純來,和他說說。」即令琴童兒:「快請你謝爹去!」伯爵因問:「哥,你那日叫那兩個去?」西門慶笑道:「叫將鄭月兒和洪四兒去罷。」伯爵道:「哥,你是個人,你請他就不對我說聲,我怎的也知道了?比李掛兒風月如何?」西門慶道:「通色絲子女不可言!」伯爵道:「他怎的前日你生日時,那等不言語,扭扭的,也是個肉佞賊小淫婦兒。」西門慶道:「等我到幾時再去著,也攜帶你走走。你月娘會打的好雙陸,你和他打兩貼雙陸。」伯爵道:「等我去混那小淫婦兒,休要放了他!」西門慶道:「你這歪狗才,不要惡識他便好。」正說著,謝希大到了,聲諾畢,坐下。西門慶道:「常二哥如此這般,新有了華居,瞞著俺每,已搬過去了。咱每人隨意出些分資,休要費煩他絲毫。我這裡整治停當,教小廝抬到他府上,我還叫兩個妓者,咱耍一日何如?」謝希大道:「哥吩咐每人出多少分資,俺每都送到哥這裡來就是了。還有那幾位?」西門慶道:「再沒人,只這三四個兒,每人二星銀子就夠了。」伯爵道:「十分人多了,他那裡沒地方兒。」

正說著,只見琴童來說:「吳大舅來了。」西門慶道:「請你大舅這裡來坐。」不一時,吳大舅進入軒內,先與三人作了揖,然後與西門慶敘禮坐下。小廝拿茶上來,同吃了茶,吳大舅起身說道:「請姐夫到後邊說句話兒。」西門慶連忙讓大舅到後邊月娘房裡。月娘還在捲棚內與眾姊妹吃酒聽唱,聽見說:「大舅來了,爹陪著在後邊說話哩。」一面走到上房,見大舅道了萬福,叫小玉遞上茶來。大舅向袖中取出十兩銀子遞與月娘,說道:「昨日府里才領了三錠銀子,姐夫且收了這十兩,餘者待後次再送來。」西門慶道:「大舅,你怎的這般計較?且使著,慌怎的!」大舅道:「我恐怕遲了姐夫的。」西門慶因問:「倉廒修理的也將完了?」大舅道:「還得一個月終完。」西門慶道:「工完之時,一定撫按有些獎勵。」大舅道:「今年考選軍政在邇,還望姐夫扶持,大巡上替我說說。」西門慶道:「大舅之事,都在於我。」

說畢話,月娘道:「請大舅前邊同坐罷。」大舅道:「我去罷,只怕他三位來有甚麼話說。」西門慶道:「沒甚麼話。常二哥新近問我借了幾兩銀子,買下了兩間房子,已搬過去了,今日買了些禮兒來謝我,節間留他每坐坐。大舅來的正好。」於是讓至前邊坐了。月娘連忙叫廚下打發萊兒上去。琴童與王經先安放八仙桌席端正,西門慶旋教開庫房,拿出一壇夏提刑家送的菊花酒來。打開碧靛清,噴鼻香,未曾篩,先攙一瓶涼水,以去其蓼辣之性,然後貯於布甑內,篩出來醇厚好吃,又不說葡萄酒。叫王經用小金鐘兒斟一杯兒,先與吳大舅嘗了,然後,伯爵等每人都嘗訖,極口稱羨不已。須臾,大盤大碗擺將上來,眾人吃了一頓。然後才拿上釀螃蟹並兩盤燒鴨子來,伯爵讓大舅吃。連謝希大也不知是甚麼做的,這般有味,酥脆好吃。西門慶道:「此是常二哥家送我的。」大舅道:「我空痴長了五十二歲,並不知螃蟹這般造作,委的好吃!」伯爵又問道:「後邊嫂子都嘗了嘗兒不曾?」西門慶道:「房下每都有了。」伯爵道:「也難為我這常嫂子,真好手段兒!」常峙節笑道:「賤累還恐整理的不堪口,教列位哥笑話。」

吃畢螃蟹,左右上來斟酒,西門慶令春鴻和書童兩個,在旁一遞一個歌唱南曲。應伯爵忽聽大卷棚內彈箏歌唱之聲,便問道:「哥,今日李桂姐在這裡?不然,如何這等音樂之聲?」西門慶道:。「你再聽,看是不是?」伯爵道:「李桂姐不是,就是吳銀兒。」西門慶道:「你這花子單管只瞎謅。倒是個女先生。」伯爵道: 「不是鬱大姐?」西門慶道:「不是他,這個是申二姐。年小哩,好個人材,又會唱。」伯爵道:「真個這等好?哥怎的不牽出來俺每瞧瞧?就唱個兒俺每聽。」西門慶道:「今日你眾娘每大節間,叫他來賞重陽頑耍,偏你這狗才耳朵尖,聽的見!」伯爵道:「我便是千里眼,順風耳,隨他四十里有蜜蜂兒叫,我也聽見了。」 謝希大道:「你這花子,兩耳朵似竹簽兒也似,愁聽不見!」兩個又頑笑了一回,伯爵道:「哥,你好歹叫他出來,俺每見見兒,俺每不打緊,教他只當唱個與老舅聽也罷了。休要就古執了。」西門慶吃他逼迫不過,一面使王經領申二姐出來唱與大舅聽。不一時,申二姐來,望上磕了頭起來,旁邊安放交床兒與他坐下。伯爵問申二姐:「青春多少?」申二姐回道:「屬牛的,二十一歲了。」又問:「會多少小唱?」申二姐道:「琵琶箏上套數小唱,也會百十來套。」伯爵道:「你會許多唱也夠了。」西門慶道:「申二姐,你拿琵琶唱小詞兒罷,省的勞動了你。說你會唱『四夢八空』,你唱與大舅聽。」吩咐王經、書童兒,席間斟上酒。那申二姐款跨鮫綃,微開檀口,慢慢唱著,眾人飲酒不題。

且說李瓶兒歸到房中,坐凈桶,下邊似尿的一般,只顧流將起來,登時流的眼黑了。起來穿裙子,忽然一陣旋暈,向前一頭撞倒在地。饒是迎春在旁搊扶著,還把額角上磕傷了皮。和奶子搊到炕上,半日不省人事。慌了迎春,忙使繡春:「快對大娘說去!」繡春走到席上,報與月娘眾人。月娘撇了酒席,與眾姐妹慌忙走來看視。見迎春、奶子兩個搊扶著他坐在炕上,不省人事。便問:「他好好的進屋裡,端的怎麼來就不好了?」迎春揭開凈桶與月娘瞧,把月娘唬了一跳。說道:「他剛纔只怕吃了酒,助趕的他血旺了,流了這些。」玉樓、金蓮都說:「他幾曾大吃酒來!」一面煎燈心薑湯灌他。半晌蘇醒過來,才說出話兒來。月娘問:「李大姐,你怎的來?」李瓶兒道:「我不怎的。坐下桶子起來穿裙子,只見眼兒前黑黑的一塊子,就不覺天旋地轉起來,由不的身子就倒了。」月娘便要使來安兒:「請你爹進來──對他說,教他請任醫官來看你。」李瓶兒又嗔教請去:「休要大驚小怪,打攪了他吃酒。」月娘吩咐迎春:「打鋪教你娘睡罷。」月娘於是也就吃不成酒了,吩咐收拾了傢伙,都歸後邊去了。

西門慶陪侍吳大舅眾人,至晚歸到後邊月娘房中。月娘告訴李瓶兒跌倒之事,西門慶慌走到前邊來看視。見李瓶兒睡在炕上,面色蠟查黃了,扯著西門慶衣袖哭泣。西門慶問其所以,李瓶兒道:「我到屋裡坐榪子,不知怎的,下邊只顧似尿也一般流將起來,不覺眼前一塊黑黑的。起來穿裙子,天旋地轉,就跌倒了。」西門慶見他額上磕傷一道油皮,說道,「丫頭都在那裡,不看你,怎的跌傷了面貌?」李瓶兒道:「還虧大丫頭都在跟前,和奶子搊扶著我,不然,還不知跌的怎樣的。」西門慶道:「我明早請任醫官來看你。」當夜就在李瓶兒對面床上睡了一夜。

次日早晨,往衙門裡去,旋使琴童請任醫官去了。直到晌午才來。西門慶先在大廳上陪吃了茶,使小廝說進去。李瓶兒房裡收拾乾凈,熏下香,然後請任醫官進房中。診畢脈,走出外邊廳上,對西門慶說:「老夫人脈息,比前番甚加沉重,七情傷肝,肺火太旺,以致木旺土虛,血熱妄行,猶如山崩而不能節制。若所下的血紫者,猶可以調理;若鮮紅者,乃新血也。學生撮過藥來,若稍止,則可有望;不然,難為矣。」西門慶道:「望乞老先生留神加減,學生必當重謝!」任醫官道: 「是何言語!你我厚間,又是明用情分,學生無不盡心。」西門慶待畢茶,送出門,隨即具一匹杭絹、二兩白金,使琴童兒討將藥來,名曰「歸脾湯」,乘熱吃下去,其血越流之不止。西門慶越發慌了,又請大街口胡太醫來瞧。胡太醫說是氣沖血管,熱入血室,亦取將藥來。吃下去,如石沉大海一般。

月娘見前邊亂著請太醫,只留申二姐住了一夜,與了他五錢銀子、一件雲絹比甲兒並花翠,裝了個盒於,就打發他坐轎子去了。花子由自從那日開張吃了酒去,聽見李瓶兒不好,使了花大嫂,買了兩盒禮來看他。見他瘦的黃懨懨兒,不比往時,兩個在屋裡大哭了一回。月娘後邊擺茶請他吃了。韓道國說:「東門外住的一個看婦人科的趙太醫,指下明白,極看得好。前歲,小媳婦月經不通,是他看來。老爹請他來看看六娘,管情就好哩。」西門慶聽了,就使琴童和王經兩個疊騎著頭口,往門外請趙太醫去了。

西門慶請了應伯爵來,和他商議道:「第六個房下,甚是不好的重,如之奈何?」伯爵失驚道:「這個嫂子貴恙說好些,怎的又不好起來?」西門慶道:「自從小兒沒了,著了憂戚,把病又發了。昨日重陽,我接了申二姐,與他散悶頑耍,他又沒好生吃酒,誰知走到屋中就暈起來,一交跌倒,把臉都磕破了。請任醫官來看,說脈息比前沉重。吃了藥,倒越發血盛了。」伯爵道:「你請胡太醫來看,怎的說?」西門慶道:「胡大醫說,是氣沖了血管,吃了他的,也不見動靜。今日韓伙計說,門外一個趙太醫,名喚趙龍崗,專科看婦女,我使小廝請去了。把我焦愁的了不的。生生為這孩子不好,白日黑夜思慮起這病來了。婦女人家,又不知個迴轉,勸著他,又不依你,叫我無法可處。」

正說著,平安來報:「喬親家爹來了。」西門慶一面讓進廳上,同伯爵敘禮坐下。喬大戶道:「聞得六親家母有些不安,特來候問。」西門慶道:「便是。一向因小兒沒了,著了憂戚,身上原有些不調,又發起來了。蒙親家掛念。」喬大戶道:「也曾請人來看不曾?」西門慶道:「常吃任後溪的藥,昨日又請大街胡先生來看,吃藥越發轉盛。今日又請門外專看婦人科趙龍崗去了。」喬大戶道:「咱縣門前住的何老人,大小方脈俱精。他兒子何歧軒,見今上了個冠帶醫士。親家何不請他來看看親家母?」西門慶道:「既是好,等趙龍崗來,來過再請他來看看。」喬大戶道:「親家,依我愚見,不如先請了何老人來,再等趙龍崗來,叫他兩個細講一講,就論出病原來了。然後下藥,無有不效之理。」西門慶道:「親家說的是。」一面使玳安拿拜帖兒和喬通去請。

那消半晌,何老人到來,與西門慶、喬大戶等作了揖,讓於上面坐下。西門慶舉手道:「數年不見你老人家,不覺越發蒼髯皓首。」喬大戶又問:「令郎先生肄業盛行?」何老人道:「他逐日縣中迎送,也不得閑,倒是老拙常出來看病。」伯爵道:「你老人家高壽了,還這等健朗。」何老人道:「老拙今年痴長八十一歲。」敘畢話,看茶上來吃了,小廝說進去。須臾,請至房中,就床看李瓶兒脈息,旋搊扶起來,坐在炕上,形容瘦的十分狼狽了。但見他──

面如金紙,體似銀條。看看減褪豐標,漸漸消磨精彩。隱隱耳虛聞磐響,昏昏眼暗覺螢飛。六脈細沉,一靈縹緲,喪門弔客已臨身,扁鵲盧醫難下手。

何老人看了脈息,出到廳上,向西門慶、喬大戶說道:「這位娘子,乃是精沖了血管起,然後著了氣惱。氣與血相搏,則血如崩。不知當初起病之由是也不是?」西門慶道:「是便是,卻如何治療?」

正論間,忽報:「琴童和王經請了趙先生來了。」何老人便問:「是何人?」西門慶道:「也是伙計舉來一醫者,你老人家只推不知,待他看了脈息,你老人家和他講一講,好下藥。」不一時,趙大醫從外而入,西門慶與他敘禮畢,然後與眾人相見。何、喬二老居中,讓他在左,伯爵在右,西門慶主位相陪。吃了茶,趙太醫便問:「列位尊長貴姓?」喬大戶道:「俺二人一姓何,一姓喬。」伯爵道:「在下姓應。老先想就是趙龍崗先生了。」趙太醫答道:「龍崗是賤號。在下以醫為業,家祖見為太醫院院判,家父見充汝府良醫,祖傳三輩,習學醫術。每日攻習王叔和、東垣勿聽子《藥性賦》、《黃帝素問》、《難經》、《活人書》、《丹溪纂要》、《丹溪心法》、《潔古老脈訣》、《加減十三方》、《千金奇效良方》、《壽域神方》、《海上方》,無書不讀。藥用胸中活法,脈明指下玄機。六氣四時,辨陰陽之標格;七表八里,定關格之沉浮。風虛寒熱之癥候,一覽無餘;弦洪芤石之脈理,莫不通曉。小人拙口鈍吻,不能細陳。」何老人聽了,道:「敢問看病當以何者為先?」趙太醫道:「古人雲,望聞問切,神聖功巧。學生先問病,後看脈,還要觀其氣色。就如子平兼五星一般,才看得準,庶乎不差。」何老人道:「既是如此,請先生進去看看。」西門慶即令琴童:「後邊說去,又請了趙先生來了。 」

不一時,西門慶陪他進入李瓶兒房中。那李瓶兒方纔睡下安逸一回,又搊扶起來,靠著枕褥坐著。這趙太醫先診其左手,次診右手,便教:「老夫人抬起頭來,看看氣色。」那李瓶兒真個把頭兒揚起來。趙太醫教西門慶:「老爹,你問聲老夫人,我是誰?」西門慶便教李瓶兒:「你看這位是誰?」那李瓶兒抬頭看了一眼,便低聲說道:「他敢是太醫?」趙先生道:「老爹,不妨事,還認的人哩。」西門慶道:「趙先生,你用心看,我重謝你。」一面看視了半日,說道:「老夫人此病,休怪我說,據看其面色,又診其脈息,非傷寒,只為雜症,不是產後,定然胎前。」西門慶道:「不是此疾。先生你再仔細診一診。」趙先生又沉吟了半晌道:「如此面色這等黃,多管是脾虛泄瀉,再不然定是經水不調。」西門慶道:「實說與先生,房下如此這般,下邊月水淋漓不止,所以身上都瘦弱了。有甚急方妙藥,我重重謝你。」趙先生道:「如何?我就說是經水不調。不打緊處,小人有藥。」

西門慶一面同他來到前廳,喬大戶、何老人問他甚麼病源,趙先生道:「依小人講,只是經水淋漓。」何老人道:「當用何藥治之?」趙先生道:「我有一妙方,用著這幾味藥材,吃下去管情就好。聽我說:甘草甘遂與碙砂,黎蘆巴豆與芫花,薑汁調著生半夏,用烏頭杏仁天麻。 這幾味兒齊加,蔥蜜和丸只一撾,清晨用燒酒送下。」

何老人聽了,便道:「這等藥恐怕太狠毒,吃不得。」趙先生道:「自古毒藥苦口利於病。怎麼吃不得?」西門慶見他滿口胡說,因是韓伙計舉保來,不好囂他,稱二錢銀子,也不送,就打發他去了。因向喬大戶說:「此人原來不知甚麼。」何老人道:「老拙適纔不敢說,此人東門外有名的趙搗鬼,專一在街上賣杖搖鈴,哄過往之人,他那裡曉的甚脈息病源!」因說:「老夫人此疾,老拙到家撮兩帖藥來,遇緣,若服畢經水少減,胸口稍開,就好用藥。只怕下邊不止,就難為矣。」說畢,起身。

西門慶封白金一兩,使玳安拿盒兒討將藥來,晚夕與李瓶兒吃了,並不見分毫動靜。吳月娘道:「你也省可與他藥吃。他飲食先阻住了,肚腹中有甚麼兒,只是拿藥淘碌他。前者,那吳神仙算他三九上有血光之災,今年卻不整二十七歲了。你還使人尋這吳神仙去,叫替他打算算那祿馬數上如何。只怕犯著甚麼星辰,替他禳保禳保。」西門慶聽了,旋差人拿帖兒往周守備府里問去。那裡回說:「吳神仙雲游之人,來去不定。但來,只在城南土地廟下。今歲從四月里,往武當山去了。要打數算命,真武廟外有個黃先生打的好數,一數只要三錢銀子,不上人家門。」西門慶隨即使陳敬濟拿三錢銀子,逕到北邊真武廟門首黃先生家。門上貼著:「抄算先天易數,每命卦金三錢。」陳敬濟向前作揖,奉上卦金,說道:「有一命煩先生推算。」寫與他八字:女命,年二十七歲,正月十五日午時。這黃先生把算子一打,就說:「這個命,辛未年庚寅月辛卯日甲午時,理取印綏之格,借四歲行運。四歲己未,十四歲戊午,二十四歲丁巳,三十四歲丙辰。今年流年丁酉,比肩用事,歲傷日乾,計都星照命,又犯喪門五鬼,災殺作炒。夫計都者,陰晦之星也。其象猶如亂絲而無頭,變異無常。大運逢之,多主暗昧之事,引惹疾病,主正、二、三、七、九月病災有損,小口凶殃,小人所算,口舌是非,主失財物。或是陰人大為不利。」抄畢數,敬濟拿來家。西門慶正和應伯爵、溫秀才坐的,見抄了數來,拿到後邊,解說與月娘聽。見命中多凶少吉,不覺──

眉間搭上三黃鎖,腹內包藏一肚愁。

