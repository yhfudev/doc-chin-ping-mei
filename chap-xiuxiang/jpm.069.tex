%# -*- coding: utf-8 -*-
%!TEX encoding = UTF-8 Unicode
%!TEX TS-program = xelatex
% vim:ts=4:sw=4
%
% 以上设定默认使用 XeLaTex 编译,并指定 Unicode 编码,供 TeXShop 自动识别

%第六十九回 
\chapter{招宣府初調林太太 麗春院驚走王三官}

詞曰:

香煙裊,羅幃錦帳風光好。風光好,金釵斜軃,鳳顛鸞倒。 恍疑身在蓬萊島,邂逅相逢緣不小。緣不小,最開懷處,蛾眉淡掃。

話說玳安同文嫂兒到家,平安說:「爹在對門房子里。」進去稟報。西門慶正在書房中和溫秀才坐的,見玳安,隨即出來,小客位內坐下。玳安道:「文嫂兒叫了來,在外邊伺候。」西門慶即令:「叫他進來。」那文嫂悄悄掀開暖簾,進入裡面,向西門慶磕頭。西門慶道:「文嫂,許久不見你。」文嫂道:「小媳婦有。」西門慶道:「你如今搬在那裡住了?」文嫂道:「小媳婦因不幸為了場官司,把舊時那房兒棄了,如今搬在大南首王家巷住哩。」西門慶吩咐道:「起來說話。」那文嫂一面站立在旁邊。西門慶令左右都出去,那平安和畫童都躲在角門外伺候,只玳安兒影在簾兒外邊聽。西門慶因問:「你常在那幾家大人家走跳?」文嫂道:「就是大街皇親家,守備府周爺家,喬皇親、張二老爹、夏老爹家,都相熟。」西門慶道:「你認的王招宣府里不認的?」文嫂道:「是小媳婦定門主顧,太太和三娘常照顧我的花翠。」西門慶道:「你既相熟,我有樁事兒央及你,休要阻了我。」向袖中取出五兩一錠銀子與他,悄悄和他說:「如此這般,你怎的尋個路兒把他太太弔在你那裡,我會他會兒,我還謝你。」那文嫂聽了,哈哈笑道:「是誰對爹說來?你老人家怎的曉得來?」西門慶道:「常言:人的名兒,樹的影兒。我怎得不知道!」文嫂道:「若說起我這太太來,今年屬豬,三十五歲,端的上等婦人,百伶百俐,只好象三十歲的。他雖是乾這營生,好不乾的細密!就是往那裡去,許多伴當跟隨,徑路兒來,逕路兒去。三老爹在外為人做人,他怎在人家落腳?──這個人傳的訛了。倒是他家裡深宅大院,一時三老爹不在,藏掖個兒去,人不知鬼不覺,倒還許。若是小媳婦那裡,窄門窄戶,敢招惹這個事?就是爹賞的這銀子,小媳婦也不敢領去。寧可領了爹言語,對太太說就是了。」西門慶道:「你不收,便是推託,我就惱了。事成,我還另外賞幾個綢緞你穿。」文嫂道:「愁你老人家沒有也怎的?上人著眼覷,就是福星臨。」磕了個頭,把銀子接了,說道:「待小媳婦悄悄對太太說,來回你老人家。」西門慶道:「你當件事乾,我這裡等著。你來時,只在這裡來就是了,我不使小廝去了。」文嫂道:「我知道。不在明日,只在後日,隨早隨晚,討了示下就來了。」一面走出來。玳安道:「文嫂,隨你罷了,我只要你一兩銀子,也是我叫你一場。你休要獨吃。」文嫂道:「猢猻兒隔牆掠篩箕,還不知仰著合著哩。」於是出門騎上驢子,他兒子籠著,一直去了。西門慶和溫秀才坐了一回,良久,夏提刑來,就冠冕著同往府里羅同知──名喚羅萬象那裡吃酒去了。直到掌燈以後才來家。

且說文嫂兒拿著西門慶五兩銀子,到家歡喜無盡,打發會茶人散了。至後晌時分,走到王招宣府宅里,見了林太太,道了萬福。林氏便道:「你怎的這兩日不來看看我?」文嫂便把家中會茶,趕臘月要往頂上進香一節告訴林氏。林氏道:「你兒子去,你不去罷了。」文嫂兒道:「我如何得去?只教文堂代進香去罷了。」 林氏道:「等臨期,我送些盤纏與你。」文嫂便道:「多謝太太佈施。」說畢,林氏叫他近前烤火,丫鬟拿茶來吃了。這文嫂一面吃了茶,問道:「三爹不在家了?」林氏道:「他又有兩夜沒回家,只在裡邊歇哩。逐日搭著這夥喬人,只眠花臥柳,把花枝般媳婦兒丟在房裡,通不顧,如何是好?」文嫂又問:「三娘怎的不見?」林氏道:「他還在房裡未出來哩。」這文嫂見無人,便說道:「不打緊,太太寬心。小媳婦有個門路兒,管就打散了這夥人,三爹收心,也再不進院去了。太太容小媳婦,便敢說;不容便不敢說。」林氏道:「你說的話兒,那遭兒我不依你來?你有話只顧說不妨。」這文嫂方說道:「縣門前西門大老爹,如今見在提刑院做掌刑千戶,家中放官吏債,開四五處鋪面:緞子鋪、生藥鋪、綢絹鋪、絨線鋪,外邊江湖又走標船,揚州興販鹽引,東平府上納香蠟,伙計主管約有數十。東京蔡太師是他乾爺,朱太尉是他衛主,翟管家是他親家,巡撫巡按都與他相交,知府知縣是不消說。家中田連阡陌,米爛成倉,身邊除了大娘子──乃是清河左衛吳千戶之女,填房與他為繼室──只成房頭、穿袍兒的,也有五六個。以下歌兒舞女,得寵侍妾,不下數十。端的朝朝寒食,夜夜元宵。今老爹不上三十一二年紀,正是當年漢子,大身材,一表人物。也曾吃藥養龜,慣調風情;雙陸象棋,無所不通;蹴踘打毬,無所不曉;諸子百家,拆白道字,眼見就會。端的擊玉敲金,百憐百俐。聞知咱家乃世代簪纓人家,根基非淺,又見三爹在武學肄業,也要來相交,只是不曾會過,不好來的。昨日聞知太太貴誕在邇,又四海納賢,也一心要來與太太拜壽。小媳婦便道:『初會,怎好驟然請見的。待小的達知老太太,討個示下,來請老爹相見。』今老太太不但結識他來往相交,只央浼他把這乾人斷開了,須玷辱不了咱家門戶。」林氏被文嫂這篇話說的心中迷留摸亂,情竇已開,便向文嫂兒較計道:「人生面不熟,怎好遽然相見?」文嫂道:「不打緊,等我對老爹說。只說太太先央浼他要到提刑院遞狀,告引誘三爹這起人,預先請老爹來私下先會一會,此計有何不可?」說得林氏心中大喜,約定後日晚夕等候。

這文嫂討了婦人示下歸家,到次日飯時,走來西門慶宅內。西門慶正在對門書院內坐的,忽玳安報:「文嫂來了。」西門慶聽了,即出小客位,令左右放下簾兒。良久,文嫂進入裡面,磕了頭,玳安知局,就走出來了。文嫂便把怎的說念林氏:「誇獎老爹人品家道,怎樣結識官府,又怎的仗義疏財,風流博浪,說得他千肯萬肯,約定明日晚間,三爹不在家,家中設席等候。假以說人情為由,暗中相會。」西門慶聽了,滿心歡喜。又令玳安拿了兩匹綢緞賞他。文嫂道,「爹明日要去,休要早了。直到掌燈,街上人靜時,打他後門首扁食巷中──他後門旁有個住房的段媽媽,我在他家等著。爹只使大官兒彈門,我就出來引爹入港,休令左近人知道。」西門慶道:「我知道。你明日先去,不可離寸地,我也依期而至。」說畢,文嫂拜辭出門,又回林氏話去了。

西門慶那日,歸李嬌兒房中宿歇,一宿無話。巴不到次日,培養著精神。午間,戴著白忠靖巾,便同應伯爵騎馬往謝希大家吃生日酒。席上兩個唱的。西門慶吃了幾杯酒,約掌燈上來,就逃席走出來了。騎上馬,玳安、琴童兩個小廝跟隨。那時約十九日,月色朦朧,帶著眼紗由大街抹過,逕穿到扁食巷王招宣府後門來。那時才上燈一回,街上人初靜之後。西門慶離他後門半舍,把馬勒住,令玳安先彈段媽媽家門。原來這媽媽就住著王招宣家後房,也是文嫂舉薦,早晚看守後門,開門閉戶。但有入港,在他家落腳做窩。文嫂在他屋裡聽見彈門,連忙開門。見西門慶來了,一面在後門裡等的西門慶下了馬,除去眼紗兒,引進來,吩咐琴童牽了馬,往對門人家西首房檐下那裡等候,玳安便在段媽媽屋裡存身。這文嫂一面請西門慶入來,便把後門關了,上了栓,由夾道進內。轉過一層群房,就是太太住的五間正房,旁邊一座便門閉著。這文嫂輕敲敲門環兒,原來有個聽頭。少頃,見一丫鬟出來,開了雙扉。文嫂導引西門慶到後堂,掀開簾攏,只見裡面燈燭熒煌,正面供養著他祖爺太原節度頒陽郡王王景崇的影身圖:穿著大紅團袖,蟒衣玉帶,虎皮交椅坐著觀看兵書。有若關王之像,只是髯須短些。迎門朱紅匾上寫著「節義堂」三字,兩壁隸書一聯:「傳家節操同松竹,報國勛功並鬥山。」西門慶正觀看之間,只聽得門簾上鈴兒響,文嫂從里拿出一盞茶來與西門慶吃。西門慶便道:「請老太太出來拜見。」文嫂道:「請老爹且吃過茶著,剛纔稟過太太知道了。」不想林氏悄悄從房門簾里望外邊觀看,見西門慶身材凜凜,一表人物,頭戴白緞忠靖冠,貂鼠暖耳,身穿紫羊絨鶴氅,腳下粉底皂靴,就是個──

富而多詐姦邪輩,壓善欺良酒色徒。

林氏一見滿心歡喜,因悄悄叫過文嫂來,問他戴的孝是誰的。文嫂道:「是他第六個娘子的孝,新近九月間沒了不多些時。饒少殺,家中如今還有一巴掌人兒。他老人家,你看不出來?出籠兒的鵪鶉──也是個快鬥的。」這婆娘聽了,越發歡喜無盡。文嫂催逼他出去,婦人道:「我羞答答怎好出去?請他進來見罷。」文嫂一面走出來,向西門慶說:「太太請老爹房內拜見哩。」於是忙掀門簾,西門慶進入房中,但見簾幙垂紅,氈毺鋪地,麝蘭香靄,氣暖如春。繡榻則鬥帳雲橫,錦屏則軒轅月映。婦人頭上戴著金絲翠葉冠兒,身穿白綾寬綢襖兒,沉香色遍地金妝花緞子鶴氅,大紅宮錦寬襴裙子,老鸛白綾高底鞋兒。就是個綺閣中好色的嬌娘,深閨內施毴的菩薩。有詩為證:

雲濃脂膩黛痕長,蓮步輕移蘭麝香。
醉後情深歸繡帳,始知太太不尋常。

西門慶一見便躬身施禮,說道:「請太太轉上,學生拜見。」林氏道:「大人免禮罷。」西門慶不肯,就側身磕下頭去拜兩拜。婦人亦敘禮相還。拜畢,西門慶正面椅子上坐了,林氏就在下邊梳背炕沿斜僉相陪。文嫂又早把前邊儀門閉上了,再無一個僕人在後邊。三公子那邊角門也關了。一個小丫鬟名喚芙蓉,拿茶上來,林氏陪西門慶吃了茶,文嫂就在旁說道:「太太久聞老爹執掌刑名,敢使小媳婦請老爹來央煩樁事兒,未知老爹可依允不依?」西門慶道:「不知老太太有甚事吩咐?」 林氏道:「不瞞大人說,寒家雖世代做了這招宣,不幸夫主去世年久,家中無甚積蓄。小兒年幼優養,未曾考襲,如今雖入武學肄業,年幼失學。外邊有幾個姦詐不良的人,日逐引誘他在外飄酒,把家事都失了。幾次欲待要往公門訴狀,誠恐拋頭露面,有失先夫名節。今日敢請大人至寒家訴其衷曲,就如同遞狀一般。望乞大人千萬留情把這乾人怎生處斷開了,使小兒改過自新,專習功名,以承先業,實出大人再造之恩,妾身感激不淺,自當重謝。」西門慶道:「老太太怎生這般說。尊家乃世代簪纓,先朝將相。令郎既入武學,正當努力功名,承其祖武,不意聽信游食所哄,留連花酒,實出少年所為。太太既吩咐,學生到衙門裡,即時把這乾人處分懲治,庶可杜絕將來。」這婦人聽了,連忙起身,向西門慶道了萬福,說道:「容日妾身致謝大人。」西門慶道:「你我一家,何出此言。」

說話之間,彼此眉目顧盼留情。不一時,文嫂放桌兒擺上酒來,西門慶故意辭道:「學生初來進謁,倒不曾送禮來,如何反承老太太盛情留坐!」林氏道:「不知大人下降,沒作整備。寒天聊具一杯水酒,表意麵已。」丫鬟篩上酒來,端的金壺斟美釀,玉盞貯佳餚。林氏起身捧酒,西門慶亦下席道:「我當先奉老太太一杯。」 文嫂兒在旁插口說道:「老爹且不消遞太太酒。這十一月十五日是太太生日,那日送禮來與太太祝壽就是了。」西門慶道:「阿呀!早時你說。今日是初九,差六日。我在下一定來與太太登堂拜壽。」林氏笑道:「豈敢動勞大人!」須臾,大盤大碗,就是十六碗美味佳餚,旁邊絳燭高燒,下邊金爐添火,交杯一盞,行令猜枚,笑雨嘲雲。

酒為色膽。看看飲至蓮漏已沉、窗月倒影之際,一雙竹葉穿心,兩個芳情已動。文嫂已過一邊,連次呼酒不至。西門慶見左右無人,漸漸促席而坐,言頗涉邪,把手捏腕之際,挨肩擦膀之間。初時戲摟粉項,婦人則笑而不言;次後款啟朱唇,西門慶則舌吐其口,鳴咂有聲,笑語密切。婦人於是自掩房門,解衣松佩,微開錦帳,輕展繡衾,鴛枕橫床,鳳香薰被,相挨玉體,抱摟酥胸。原來西門慶知婦人好風月,家中帶了淫器包在身邊,又服了胡僧藥。婦人摸見他陽物甚大,西門慶亦摸其牝戶,彼此歡欣,情興如火。展猿臂,不覺蝶浪蜂狂;蹺玉腿,那個羞雲怯雨!正是:

縱橫慣使風流陣,那管床頭墮玉釵。

西門慶當下竭平生本事,將婦人儘力盤桓了一場。纏至更深天氣,方纔精泄。婦人則發亂釵橫,花憔柳困。兩個並頭交股,摟抱片時,起來穿衣。婦人款剔銀燈,開了房門,照鏡整容,呼丫鬟捧水凈手。復飲香醪,再勸美酌。三杯之後,西門慶告辭起身,婦人輓留不已,叮嚀頻囑。西門慶躬身領諾,謝擾不盡,相別出門。婦人送到角門首回去了。文嫂先開後門,呼喚玳安、琴童牽馬過來,騎上回家。街上已喝號提鈴,更深夜靜,但見一天霜氣,萬籟無聲。西門慶回家,一宿無話。

到次日,西門慶到衙門中發放已畢,在後廳叫過該地方節級緝捕,吩咐如此這般:「王招宣府里三公子,看有甚麼人勾引他,院中在何人家行走,即查訪出名字來,報我知道。」因向夏提刑說:「王三公子甚不學好,昨日他母親再三央人來對我說,倒不關他兒子事,只被這乾光棍勾引他。今若不痛加懲治,將來引誘壞了人家子弟。」夏提刑道:「長官所見不錯,必該治他。」節級緝捕領了西門慶鈞語,當日即查訪出各人名姓來,打了事件,到後晌時分來西門慶宅內呈遞揭帖。西門慶見上面有孫寡嘴、祝實念、小張閑、聶鉞兒、向三、於寬、白回子,樂婦是李桂姐、秦玉芝兒。西門慶取過筆來,把李桂姐、秦玉芝兒並老孫、祝實念名字都抹了,吩咐:「這小張閑等五個光棍,即與我拿了,明日早帶到衙門裡來。」眾公人應諾下去。至晚,打聽王三官眾人都在李桂姐家吃酒踢行頭,都埋伏在房門首。深更時分,剛散出來,眾公人把小張閑、聶鉞、於寬、白回子、向三五人都拿了。孫寡嘴與祝實念扒李桂姐後房去了,王三官藏在李桂姐床底下,不敢出來。桂姐一家唬的捏兩把汗,更不知是那裡的人,亂央人打聽實信。王三官躲了一夜不敢出來。李家鴇子又恐怕東京下來拿人,到五更時分,攛掇李銘換了衣服,送王三官來家。

節級緝捕把小張閑等拿在聽事房吊了一夜。到次日早晨,西門慶進衙門與夏提刑升廳,兩邊刑杖羅列,帶人上去。每人一夾二十大棍,打得皮開肉綻,鮮血迸流,響聲震天,哀號慟地。西門慶囑咐道:「我把你這起光棍,專一引誘人家子弟在院飄風,不守本分,本當重處,今姑從輕責你這幾下兒。再若犯在我手裡,定然枷號,在院門首示眾!」喝令左右:「叉下去!」眾人望外,金命水命,走投無命。

兩位官府發放事畢,退廳吃茶。夏提刑因說起:「昨日京中舍親崔中書那裡書來,說衙門中考察本上去了,還未下來哩。今日會了長官,咱倒好差人往懷慶府同僚林蒼峰那裡,打聽打聽消息去。他那裡臨京近。」西門慶道:「長官所見甚明。」即喚走差的上來吩咐:「與你五錢銀子盤纏,即拿俺兩個拜帖,到懷慶府提刑林千戶老爹那裡,打聽京中考察本示下,看經歷司行下照會來不曾。務要打聽的實,來回報。」那人領了銀子、拜帖,又到司房結束行裝,討了匹馬,長行去了。兩位官府才起身回家。

卻說小張閑等從提刑院打出來,走在路上各人思想,更不料今日受這場虧是那裡藥線,互相埋怨。小張閑道:「莫不還是東京那裡的消息?」白回子道:「不是。若是那裡消息,怎肯輕饒素放?」常言說得好:乖不過唱的,賊不過銀匠,能不過架兒。聶鉞兒一口就說道:「你每都不知道,只我猜得著。此一定是西門官府和三官兒上氣,嗔請他表子,故拿俺每煞氣。正是:龍鬥虎傷,苦了小獐。」小張閑道:「列位倒罷了,只是苦了我在下了。孫寡嘴、祝麻子都跟著,只把俺每頂缸。」於寬道:「你怎的說渾話?他兩個是他的朋友,若拿來跪在地下,他在上面坐著,怎生相處?」小張閑道:「怎的不拿老婆?」聶鉞道:「兩個老婆,都是他心上人。李家桂姐是他的表子,他肯拿來!也休怪人,是俺每的晦氣,偏撞在這網裡。才夏老爹怎生不言語,只是他說話?這個就見出情弊來了。如今往李桂姐家尋王三官去!白為他打了這一屁股瘡來不成?便罷了,就問他要幾兩銀子盤纏,也不吃家中老婆笑話。」於是逕入勾欄,見李桂姐家門關的鐵桶相似。叫了半日,丫頭隔門問是誰,小張閑道:「是俺每,尋三官兒說話。」丫頭回說:「他從那日半夜就回家去了,不在這裡。無人在家中,不敢開門。」這眾人只得回來,到王招宣府內,逕入他客位里坐下。王三官聽見眾人來尋他,唬得躲在房裡不敢出來。半日,使出小廝永定兒來說:「俺爹不在家了。」眾人道:「好自在性兒!不在家了,往那裡去了?叫不將來!」於寬道:「實和你說了罷,休推睡里夢裡。剛纔提刑院打了俺每,押將出來。如今還要他正身見官去哩!」摟起腿來與永定瞧,教他進裡面去說: 「為你打俺每,有甚要緊!」一個個都躺在凳上聲疼叫喊。

那王三官兒越發不敢出來,只叫:「娘,怎麼樣兒?如何救我則可。」林氏道:「我女婦人家,如何尋人情去救得?」求了半日,見外邊眾人等得急了,要請老太太說話。那林氏又不出去,只隔著屏風說道:「你每略等他等,委的在莊上,不在家了。我這裡使小廝叫他去。」小張閑道:「老太太,快使人情他來!這個癤子終要出膿,只顧膿著不是事。俺每為他連累打了這一頓。剛纔老爹吩咐押出俺每來要他。他若不出來,大家都不得清凈,就弄的不好了。」

林氏聽言,連忙使小廝拿出茶來與眾人吃。王三官唬的鬼也似,逼他娘尋人情。直到至急之處,林氏方纔說道:「文嫂他只認的提刑西門官府家,昔年曾與他女兒說媒來,在他宅中走的熟。」王三官道:「就認的西門提刑也罷。快使小廝請他來。」林氏道:「他自從你前番說了他,使性兒一向不來走動,怎好又請他?他也不肯來。」王三官道:「好娘,如今事在至急,請他來,等我與他陪個禮兒便了。」林氏便使永定兒悄悄打後門出去,請了文嫂來。王三官再三央及他,一口一聲只叫: 「文媽,你認的提刑西門大官府,好歹說個人情救我。」這文嫂故意做出許多喬張致來,說道:「舊時雖故與他宅內大姑娘說媒,這幾年誰往他門上走!大人家深宅大院,不去纏他。」王三官連忙跪下說道:「文媽,你救我,恩有重報,不敢有忘。那幾個人在前邊只要出官,我怎去得?」文嫂只把眼看他娘,他娘道:「也罷,你便替他說說罷了。」文嫂道:「我獨自個去不得。三叔,你衣巾著,等我領你親自到西門老爹宅上,你自拜見央浼他,等我在旁再說,管情一天事就了了。」王三官道:「見今他眾人在前邊催逼甚急,只怕一時被他看見怎了?」文嫂道:「有甚難處勾當?等我出去安撫他,再安排些酒肉點心茶水哄他吃著,我悄悄領你從後門出去,幹事回來,他就便也不知道。」

這文嫂一面走出前廳,向眾人拜了兩拜,說道:「太太教我出來,多上覆列位哥每:本等三叔往莊上去了,不在家,使人請去了,便來也。你每略坐坐兒。吃打受罵,連累了列位。誰人不吃鹽米,等三叔來,教他知遇你們。你們千差萬差來人不差,恆屬大家只要圖了事。上司差派,不由自己。有了三叔出來,一天大事都了了。」眾人聽了,一齊道:「還是文媽見的多,你老人家早出來說恁句有南北的話兒,俺每也不急的要不的。執殺法兒只回不在家,莫不俺每自做出來的事?你恁帶累俺每吃官棒,上司要你,假推不在家。吃酒吃肉,教人替你不成?文媽,你是曉道理的,你出來,俺每還透個路兒與你──破些東西兒,尋個分上兒說說,大家了事。你不出來見俺每,這事情也要消繳,一個緝捕問刑衙門,平不答的就罷了?」文嫂兒道:「哥每說的是。你每略坐坐兒,我對太太說,安排些酒飯兒管待你每。你每來了這半日也餓了。」眾人都道:「還是我的文媽知人苦辣。不瞞文媽說,俺每從衙門裡打出來,黃湯兒也沒曾嘗著哩!」這文嫂走到後邊,一力竄掇,打了二錢銀子酒,買了一錢銀子點心,豬羊牛肉各切幾大盤,拿將出去,一壁哄他眾人在前邊大酒大肉吃著。

這王三官儒巾青衣,寫了揭帖,文嫂領著,帶上眼紗,悄悄從後門出來,步行徑往西門慶家來。到了大門首,平安兒認的文嫂,說道:「爹才在廳上,進去了。文媽有甚話說?」文嫂遞與他拜帖,說道:「哥哥,累你替他稟稟去。」連忙問王三官要了二錢銀子遞與他,那平安兒方進去替他稟知西門慶。西門慶見了手本拜帖,上寫著:「眷晚生王採頓首百拜。」一面先叫進文嫂,問了回話,然後才開大廳槅子門,使小廝請王三官進去。西門慶頭戴忠靖巾,便衣出來迎接,見王三衣巾進來,故意說道:「文嫂怎不早說?我褻衣在此。」便令左右:「取我衣服來。」慌的王三官向前攔住道:「尊伯尊便,小侄敢來拜瀆,豈敢動勞!」至廳內,王三官務請西門慶轉上行禮。西門慶笑道:「此是舍下。」再三不肯。西門慶居先拜下去,王三官說道:「小侄有罪在身,久仰,欠拜。」西門慶道:「彼此少禮。」王三官因請西門慶受禮,說道:「小侄人家,老伯當得受禮,以恕拜遲之罪。」務讓起來,受了兩禮。西門慶讓坐,王三官又讓了一回,然後挪座兒斜僉坐的。

少頃,吃了茶,王三官向西門慶說道:「小侄有事,不敢奉瀆尊嚴。」因向袖中取出揭帖遞上,隨即離座跪下。被西門慶一手拉住,說道:「賢契有甚話,但說何害!」王三官就說:「小侄不才,誠為得罪,望乞老伯念先父武弁一殿之臣,寬恕小侄無知之罪,完其廉恥,免令出官,則小侄垂死之日,實再生之幸也。銜結圖報,惶恐,惶恐!」西門慶展開揭帖,上面有小張閑等五人名字,說道:「這起光棍,我今日衙門裡,已各重責發落,饒恕了他,怎的又央你去?」王三官道:「他說老伯衙門中責罰了他,押出他來,還要小侄見官。在家百般辱罵喧嚷,索詐銀兩,不得安生,無處控訴,特來老伯這裡請罪。」又把禮帖遞上。西門慶一見,便道:「豈有此理!這起光棍可惡。我倒饒了他,如何倒往那裡去攪擾!」把禮帖還與王三官收了,道:「賢契請回,我且不留你坐。如今就差人拿這起光棍去。容日奉招。」王三官道:「豈敢!蒙老伯不棄,小侄容當叩謝。」千恩萬謝出門。西門慶送至二門首,說:「我褻服不好送的。」那王三官自出門來,還帶上眼紗,小廝跟隨去了。文嫂還討了西門慶話。西門慶吩咐:「休要驚動他,我這裡差人拿去。」

這文嫂同王三官暗暗到家。不想西門慶隨即差了一名節級、四個排軍,走到王招宣宅內。那起人正在那裡飲酒喧鬧,被公人進去不由分說都拿了,帶上鐲子。唬得眾人面如土色,說道:「王三官乾的好事,把俺每穩住在家,倒把鋤頭反弄俺每來了。」那個節級排軍罵道:「你這廝還胡說,當的甚麼?各人到老爹跟前哀告,討你那命是正經。」小張閑道:「大爺教導的是。」

不一時,都拿到西門慶宅門首,門上排軍並平安兒都張著手兒要錢,才替他稟。眾人不免脫下褶兒,並拿頭上簪圈下來,打發停當,方纔說進去。半日,西門慶出來坐廳,節級帶進去跪在廳下。西門慶罵道:「我把你這起光棍,我倒將就了你,你如何指稱我衙門往他家訛詐去?實說詐了多少錢?若不說,令左右拿拶子與我著實拶起來!」當下只說了聲,那左右排軍登時拿了五六把新拶子來伺候。小張閑等只顧叩頭哀告道:「小的每並沒訛詐分文財物,只說衙門中打出來,對他說聲。他家拿出些酒食來管待小的們,小的每並沒需索他的。」西門慶道:「你也不該往他家去。你這些光棍,設騙良家子弟,白手要錢,深為可恨!既不肯實供,都與我帶了衙門裡收監,明日嚴審取供,枷號示眾!」眾人一齊哀告,哭道:「天官爺,超生小的每罷,小的再不敢上他門纏擾了。休說枷號,這一送到監里去,冬寒時月,小的每都是死數。」西門慶道:「我把你這起光棍,饒出你去,都要洗心改過,務要生理。不許你挨坊靠院,引誘人家子弟,詐騙財物。再拿到我衙門裡來,都活打死了。」喝令:「叉出去!」眾人得了個性命,往外飛跑。正是:

敲碎玉籠飛彩鳳,頓開金鎖走蛟龍。

西門慶發了眾人去,回至後房,月娘問道:「這是那個王三官兒?」西門慶道:「此是王招宣府中三公子,前日李桂兒為那場事就是他。今日賊小淫婦兒不改,又和他纏,每月三十兩銀子教他包著。嗔道一向只哄著我!不想有個底腳裡人兒又告我說,教我差幹事的拿了這乾人,到衙門裡都夾打了。不想這乾人又到他家裡嚷賴,指望要詐他幾兩銀子,只說衙門中要他。他從沒見官,慌了,央文嫂兒拿了五十兩禮帖來求我說人情。我剛纔把那起人又拿了來,扎發了一頓,替他杜絕了。人家倒運,偏生這樣不肖子弟出來。──你家祖父何等根基,又做招宣,你又見入武學,放著那名兒不乾,家中丟著花枝般媳婦兒不去理論,白日黑夜只跟著這夥光棍在院里嫖弄。今年不上二十歲,年小小兒的,通不成器!」月娘道:「你乳老鴉笑話豬兒足,原來燈臺不照自。你自道成器的?你也吃這井裡水,無所不為,清潔了些甚麼兒?還要禁人!」幾句說的西門慶不言語了。

正擺上飯來吃,來安來報:「應二爹來了。」西門慶吩咐:「請書房裡坐,我就來。」王經連忙開了廳上書房門,伯爵進裡面坐了。良久,西門慶出來。聲喏畢,就坐在炕上,兩個說話。伯爵道:「哥,你前日在謝二哥家,怎老早就起身?」西門慶道:「我連日有勾當,又考察在邇,差人東京打聽消息。我比你每閑人兒?」伯爵又問:「哥,連日衙門中有事沒有?」西門慶道:「事,那日沒有!」伯爵又道:「王三官兒說,哥衙門中把小張閑他每五個,初八日晚夕,在李桂姐屋裡都拿的去了,只走了老孫、祝麻子兩個。今早解到衙門裡,都打出來了,眾人都往招宣府纏王三官去了。怎的還瞞著我不說?」西門慶道:「傻狗才,誰對你說來?你敢錯聽了。敢不是我衙門裡,敢是周守備府里?」伯爵道:「守備府中那裡管這閑事!」西門慶道:「只怕是京中提人?」伯爵道:「也不是。今早李銘對我說,那日把他一家子唬的魂也沒了,李桂兒至今唬的睡倒了,還沒曾起炕兒。怕又是東京下來拿人,今早打聽,方知是提刑院拿人。」西門慶道:「我連日不進衙門,並沒知道。李桂兒既賭過誓不接他,隨他拿亂去,又害怕睡倒怎的?」伯爵見西門慶迸著臉兒待笑,說道:「哥,你是個人,連我也瞞著起來。今日他告我說,我就知道哥的情。怎的祝麻子、老孫走了?一個緝捕衙門,有個走脫了人的?此是哥打著綿羊駒䮫戰,使李桂兒家中害怕,知道哥的手段。若都拿到衙門去,彼此絕了情意,都沒趣了。事情許一不許二。如今就是老孫、祝麻子見哥也有幾分慚愧。此是哥明修棧道,暗度陳倉的計策。休怪我說,哥這一著做的絕了。這一個叫做真人不露相,露相不真人。若明逞了臉,就不是乖人兒了。還是哥智謀大,見的多。」幾句說的西門慶撲吃的笑了,說道:「我有甚麼大智謀?」伯爵道:「我猜一定還有底腳裡人兒對哥說,怎得知道這等切?端的有鬼神不測之機!」西門慶道:「傻狗才,若要人不知,除非己莫為。」伯爵道:「哥衙門中如今不要王三官兒罷了。」 西門慶道:「誰要他做甚麼?當初幹事的打上事件,我就把王三官、祝麻子、老孫並李桂兒、秦玉芝名字都抹了,只拿幾個光棍來打了。」伯爵道:「他如今怎的還纏他?」西門慶道:「我實和你說罷,他指望訛詐他幾兩銀子。不想剛纔王三官親上門來拜見,與我磕了頭,陪了不是。我又差人把那幾個光棍拿了,要枷號,他眾人再三哀告說,再不敢上門纏他了。王三官一口一聲稱我是老伯,拿了五十兩禮帖兒,我不受他的。他到明日還要請我家中知謝我去。」伯爵失驚道:「真個他來和哥陪不是來了?」西門慶道:「我莫不哄你?」因喚王經:「拿王三官拜帖兒與應二爹瞧。」那王經向房子里取出拜帖,上面寫著:「眷晚生王採頓首百拜。」伯爵見了,極口稱贊道:「哥的所算,神妙不測。」西門慶吩咐伯爵:「你若看見他每,只說我不知道。」伯爵道:「我曉得。機不可泄,我怎肯和他說!」坐了一回,吃了茶,伯爵道:「哥,我去罷,只怕一時老孫和祝麻子摸將來。只說我沒到這裡。」西門慶道。「他就來,我也不見他。」一面叫將門上人來,都吩咐了:「但是他二人,只答應不在家。」西門慶從此不與李桂姐上門走動,家中擺酒也不叫李銘唱曲,就疏淡了。正是:

昨夜浣花溪上雨,綠楊芳草為何人?

