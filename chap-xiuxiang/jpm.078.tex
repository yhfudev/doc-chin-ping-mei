%# -*- coding: utf-8 -*-
%!TEX encoding = UTF-8 Unicode
%!TEX TS-program = xelatex
% vim:ts=4:sw=4
%
% 以上设定默认使用 XeLaTex 编译,并指定 Unicode 编码,供 TeXShop 自动识别

%第七十八回 
\chapter{林太太鴛幃再戰 如意兒莖露獨嘗}

詞曰:

鳳髻金泥帶,龍紋玉掌梳。去來窗下笑來扶,愛道畫眉深淺入時無?
弄筆偎人久,描花試手初。等閑含笑問狂夫,笑問歡情不減舊時麼?

話說西門慶陪大舅飲酒,至晚回家。到次日,荊都監早辰騎馬來拜謝,說道:「昨日見旨意下來,下官不勝歡喜,足見老翁愛厚,費心之至,實為銜結難忘。」說畢,茶湯兩換,荊都監起身,因問:「雲大人到幾時請俺們吃酒?」西門慶道:「近節這兩日也是請不成,直到正月間罷了。」送至大門,上馬而去。西門慶宰了一口鮮豬,兩壇浙江酒,一匹大紅絨金豸員領,一匹黑青妝花紵絲員領,一百果餡金餅,謝宋御史。就差春鴻拿貼兒,送到察院去。門吏人報進去,宋御史喚至後廳火房內,賞茶吃。等寫了回帖,又賞了春鴻三錢銀子。來見西門慶,拆開觀看,上寫著:

兩次造擾華府,悚愧殊甚。今又辱承厚貺,何以克當?外令親荊子事,已具本矣,相已知悉。連日渴仰豐標,容當面悉。使旋謹謝。侍生宋喬年拜大錦衣西門先生大人門下。

宋御史隨即差人,送了一百本歷日,四萬紙,一口豬來回禮。

一日,上司行下文書來,令吳大舅本衛到任管事。西門慶拜去,就與吳大舅三十兩銀子,四匹京段,交他上下使用。到二十四日,封了印來家,又備羊酒花紅軸文,邀請親朋,等吳大舅從衛中上任回來,迎接到家,擺大酒席與他作賀。又是何千戶東京家眷到了,西門慶寫月娘名字,送茶過去。到二十六日,玉皇廟吳道官十二個道眾,在家與李瓶兒念百日經,整做法事,大吹大打,各親朋都來送茶,請吃齋供,至晚方散,俱不在言表。至廿七日,西門慶打發各家送禮,應伯爵、謝希大、常峙節、傅伙計、甘伙計、韓道國、賁第傳、崔本,每家半口豬,半腔羊,一壇酒,二包米,一兩銀子,院中李桂姐、吳銀兒、鄭愛月兒,每人一套衣服,三兩銀子。吳月娘又與庵里薛姑子打齋,令來安兒送香油、米面、銀錢去,不在言表。看看到年除之日,穿梅表月,檐雪滾風,竹爆千門萬戶,家家貼春勝,處處挑桃符。西門慶燒了紙,又到於李瓶兒房,靈前祭奠。祭畢,置酒於後堂,合家大小歡樂。手下家人小廝並丫頭媳婦,都來磕頭。西門慶與吳月娘,俱有手帕、汗巾、銀錢賞賜。

到次日,重和元年新正月元旦,西門慶早起冠冕,穿大紅,天地上燒了紙,吃了點心,備馬就拜巡按賀節去了。月娘與眾婦人早起來,施朱傅粉,插花插翠,錦裙繡襖,羅襪弓鞋,妝點妖嬈,打扮可喜,都來月娘房裡行禮。那平安兒與該日節級在門首接拜貼,上門簿,答應往來官長士夫。玳安與王經穿著新衣裳,新靴新帽,在門首踢毽子,放炮仗,磕瓜子兒。眾伙計主管,伺候見節者,不計其數,都是陳敬濟一人管待。約晌午,西門慶往府縣拜了人回來,剛下馬,招宣府王三官兒衣巾著來拜。到廳上拜了西門慶四雙八拜,然後請吳月娘見。西門慶請到後邊,與月娘見了,出來前廳留坐。才拿起酒來吃了一盞,只見何千戶來拜。西門慶就叫陳敬濟管待陪王三官兒,他便往捲棚內陪何千戶坐去了。王三官吃了一回,告辭起身。陳敬濟送出大門,上馬而去。落後又是荊都監、雲指揮、喬大戶,皆絡繹而至。西門慶待了一日人,已酒帶半酣,至晚打發人去了,回到上房歇了一夜。到次早,又出去賀節,至晚歸來,家中已有韓姨夫、應伯爵、謝希大、常峙節、花子繇來拜。陳敬濟陪在廳上坐的。西門慶到了,見畢禮,重新擺上酒來飲酒。韓姨夫與花子繇隔門,先去了。剩下伯爵、希大、常峙節,坐個定光油兒不去。又撞見吳二舅來了,見了禮,又往後邊拜見月娘,出來一處坐的。直吃到掌燈已後方散。

西門慶已吃的酩酊大醉,送出伯爵,等到門首眾人去了。西門慶見玳安在旁站立,捏了一把手。玳安就知其意,說道:「他屋裡沒人。」這西門慶就撞入他房內。老婆早已在門裡迎接進去。兩個也無閑話,走到裡間,脫衣解帶就幹起來。原來老婆好並著腿乾,兩隻手扇著,只教西門慶攮他心子。那浪水熱熱一陣流出來,把床褥皆濕。西門慶龜頭蘸了藥,攮進去,兩手扳著腰,只顧揉搓,麈柄盡入至根,不容毫髮,婦人瞪目,口中只叫「親爺。」那西門慶問他:「你小名叫甚麼?說與我。」老婆道:「奴娘家姓葉,排行五姐。」西門慶口中喃喃吶吶,就叫葉「五兒」不絕。那老婆原是奶子出身,與賁四私通,被拐出來,占為妻子。今年三十二歲,甚麼事兒不知道!口裡如流水連叫「親爺」不絕,情濃一泄如註。西門慶扯出麈柄要抹,婦人攔住:「休抹,等淫婦下去,替你吮凈了罷。」西門慶滿心歡喜,婦人真個蹲下身子,雙手捧定那話,吮咂得乾乾凈凈,才繫上褲子。因問西門慶:「他怎的去恁些時不來?」西門慶道:「我這裡也盼他哩。只怕京中你夏老爹留住他使。」又與了老婆二、三兩銀子盤纏,因說:「我待與你一套衣服,恐賁四知道不好意思。不如與你些銀子兒,你自家治買罷。」開門送出來。玳安又早在鋪子里掩門等候。西門慶便往後邊去了。

看官聽說,自古上樑不正則下樑歪,原來賁四老婆先與玳安有姦,這玳安剛打發西門慶進去了,因傅伙計又沒在鋪子里上宿,他與平安兒打了兩大壺酒,就在老婆屋裡吃到有二更時分,平安在鋪子里歇了,他就和老婆在屋裡睡了一宿。有這等的事!正是:

滿眼風流滿眼迷,殘花何事濫如泥?
拾琴暫息商陵操,惹得山禽繞樹啼。

卻說賁四老婆晚夕同玳安睡了,因對他說:「我一時依了爹,只怕隔壁韓嫂兒傳嚷的後邊知道,也似韓伙計娘子,一時被你娘們說上幾句,羞人答答的,怎好相見?」玳安道:「如今家中,除了俺大娘和五娘不言語,別的不打緊。俺大娘倒也罷了,只是五娘快出尖兒。你依我,節間買些甚麼兒,進去孝順俺大娘。別的不稀罕,他平昔好吃蒸酥,你買一錢銀子果餡蒸酥、一盒好大壯瓜子送進去達初九日是俺五娘生日,你再送些禮去,梯己再送一盒瓜子與俺五娘。管情就掩住許多口嘴。」這賁四老婆真個依著玳安之言,第二日趕西門慶不在家,玳安就替他買了盒子,掇進月娘房中。月娘便道:「是那裡的?」玳安道:「是賁四嫂子送與娘吃的。」月娘道:「他男子漢又不在家,那討個錢來,又交他費心。」連忙收了,又回出一盒饅頭,一盒果子,說:「上覆他,多謝了。」

那日西門慶拜人回家,早又玉皇廟吳道官來拜,在廳上留坐吃酒。剛打發吳道官去了,西門慶脫了衣服,使玳安:「你騎了馬,問聲文嫂兒去:『俺爹今日要來拜拜太太。』看他怎的說?」玳安道:「爹,不消去,頭裡文嫂兒騎著驢子打門首過去了。他說明日初四,王三官兒起身往東京,與六黃公公磕頭去了。太太說,交爺初六日過去見節,他那裡伺候。」西門慶便道:「他真個這等說來?」玳安道:「莫不小的敢說謊!」這西門慶就入後邊去了。

開印,升廳畫卯,發放公事。先是雲理守家發貼兒,初五日請西門慶併合衛官員吃慶官酒。次日,何千戶娘子藍氏下貼兒,初六日請月娘姊妹相會。

且說那日西門慶同應伯爵、吳大舅三人起身到雲理守家。原來旁邊又典了人家一所房子,三間客位內擺酒,叫了一起吹打鼓樂迎接,都有桌面,吃至晚夕來家。巴不到次日,月娘往何千戶家吃酒去了。西門慶打選衣帽齊整,騎馬帶眼紗,玳安、琴童跟隨,午後時分,徑來王招宣府中拜節。王三官兒不在,送進貼兒去。文嫂兒又早在那裡,接了貼兒,連忙報與林太太說,出來,請老爺後邊坐。轉過大廳,到於後邊,掀起明簾,只見裡邊氍毹匝地,簾幕垂紅。少頃,林氏穿著大紅通袖袍兒,珠翠盈頭,與西門慶見畢禮數,留坐待茶,分付:「大官,把馬牽於後槽喂養。」茶罷,讓西門慶寬衣房內坐,說道:「小兒從初四日往東京與他叔岳父六黃太尉磕頭去了,只過了元宵才來。」西門慶一面喚玳安,脫去上蓋,裡邊穿著白綾襖子,天青飛魚氅衣,十分綽耀。婦人房裡安放桌席。須臾,丫鬟拿酒菜上來,杯盤羅列,餚饌堆盈,酒泛金波,茶烹玉蕊。婦人玉手傳杯,秋波送意,猜枚擲骰,笑語烘春。話良久,意洽情濃;飲多時,目邪心盪。看看日落黃昏,又早高燒銀燭。玳安、琴童自有文嫂兒管待,等閑不過這邊來。婦人又倒扣角門,僮僕誰敢擅入。酒酣之際,兩人共入裡間房內,掀開繡帳,關上窗戶,輕剔銀缸,忙掩朱戶。男子則解衣就寢,婦人即洗牝上床,枕設寶花,被翻紅浪。原來西門慶帶了淫器包兒來,安心要鏖戰這婆娘,早把胡僧藥用酒吃在腹中,那話上使著雙托子,在被窩中,架起婦人兩股,縱麈柄入牝中,舉腰展力,一陣掀騰鼓搗,連聲響亮。婦人在下,沒口叫親達達如流水。正是:招海旌幢秋色里,擊天鼙鼓月明中。但見:

迷魂陣罷,攝魄旗開。迷魂陣上,閃出一員酒金剛,色魔王能爭慣戰;攝魂旗下,擁一個粉骷髏,花狐狸百媚千嬌。這陣上,撲鼕鼕,鼓震春雷;那陣上,鬧挨挨,麝蘭靉靆。這陣上,復溶溶,被翻紅浪精神健;那陣上,刷剌剌,帳控銀鉤情意乖。這一個急展展,二十四解任徘徊;那一個忽剌剌,一十八滾難掙扎。鬥良久,汗浸浸,釵橫鬢亂;戰多時,喘吁吁,枕側衾歪。頃刻間,腫眉(月囊)眼;霎時下,肉綻皮開。正是:幾番鏖戰貪淫婦,不是今番這一遭。

當下西門慶就在這婆娘心口與陰戶燒了兩炷香,許下膽日家中擺酒,使人請他同三官兒娘子去看燈耍子。這婦人一段身心已被他拴縛定了,於是滿口應承都去。西門慶滿心歡喜,起來與他留連痛飲,至二更時分,把馬從後門牽出,作別回家。正是:

盡日思君倚畫樓,相逢不舍又頻留。
劉郎莫謂桃花老,浪把輕紅逐水流。

西門慶到家,有平安攔門稟說:「今日有薛公公家差人送請貼兒,請爹早往門外皇莊看春。又是雲二叔家送了五個貼兒,請五位娘吃節酒。」西門慶聽了,進入月娘房來。只見孟玉樓、潘金蓮都在房內坐的。月娘從何千戶家赴了席來家,正坐著說話。見西門慶進來,連忙道了萬福。因問:「你今日往那裡,這咱才來?」西門慶沒得說,只說:「我在應二哥家留坐。」月娘便說起今日何千戶家酒席上事:「原來何千戶娘子年還小哩,今年才十八歲,生的燈上人兒也似,一表人物,好標緻,知今博古,見我去,恰似會了幾遍,好不喜洽。嫁了何大人二年光景,房裡到使著四個丫頭,兩個養娘,兩房家人媳婦。」西門慶道:「他是內府生活所藍太監侄女兒,嫁與他陪了好少錢兒!」月娘道:「明日雲伙計家,又請俺每吃節酒,送了五個貼兒業,端的去不去?」西門慶說:「他既請你每,都去走走罷。」月娘道: 「留雪姐在家罷,只怕大節下,一時有個人客闖將來,他每沒處撾撓。」西門慶道:「也罷,留雪姐在家裡,你每四個去罷。明日薛太監請我看春,我也懶待去。這兩日春氣發也怎的,只害這腰腿疼。」月娘道:「你腰腿疼只怕是痰火,問任醫官討兩服藥吃不是,只顧挨著怎的?」西門慶道:「不妨事,由他。一發過了這兩日吃,心凈些。」因和月娘計較:「到明日燈節,咱少不的置席酒兒,請請何大人娘子。連周守備娘子,荊南崗娘子,張親家母,雲二哥娘子,連王三官兒母親,和大妗子、崔親家母,這幾位都會會。也只在十二三,掛起燈來。還叫王皇親家那起小廝扮戲耍一日。去年還有賁四在家,扎幾架煙火放,今年他東京去了,只顧不見來,卻教誰人看著扎?」那金蓮在旁插口道:「賁四去了,他娘子兒扎也是一般。」這西門慶就瞅了金蓮道:「這個小淫婦兒,三句話就說下道兒去了。」那月娘、玉樓也不採顧,就罷了。因說道:「那王官兒娘,咱每與他沒會過,人生面不熟,怎麼好請他?只怕他也不肯來。」西門慶道:「他既認我做親,咱送個貼兒與他,來不來,隨他就是了。」月娘又道:「我明日不往雲家去罷,懷著個臨月身子,只管往人家撞來撞去的,交人家唇齒。」玉樓道:「怕怎的,你身子懷的又不顯,怕還不是這個月的孩子,不妨事。大節下自恁散心,去走走兒才好。」說畢,西門慶吃了茶,就往後邊孫雪娥房裡去了。那潘金蓮見他往雪娥房中去,叫了大姐,也就往前邊去了。西門慶到於雪娥房中,交他打腿捏身上,捏了半夜。一宿晚景題過。

到次日早辰,只見應伯爵走來,對西門慶說:「昨日雲二嫂送了個貼兒,今日請房下陪眾嫂子坐。家中舊時有幾件衣服兒,都倒塌了。大正月不穿件好衣服,惹的人家笑話。敢來上覆嫂子,有上蓋衣服,借約兩套兒,頭面簪環,借約幾件兒,交他穿戴了去。」西門慶令王經:「你裡邊對你大娘說去。」伯爵道:「應寶在外邊拿著氈包並盒兒哩。哥哥,累你拿進去,就包出來罷。」那王經接氈包進去,良久抱出來,交與應寶,說道:「裡面兩套上色段子織金衣服,大小五件頭面,一雙環兒。」應寶接的去了。西門慶陪伯爵吃茶,說道:「今日薛內相又請我門外看春,怎麼得工夫去?吳親家廟裡又送貼兒,初九日年例打醮,也是去不成,教小婿去罷了。這兩日不知酒多了也怎的,只害腰疼,懶待動旦。」伯爵道:「哥,你還是酒之過,濕痰流註在這下部,也還該忌忌。」西門慶道:「這節間到人家,誰肯輕放了你,怎麼忌的住?」

正說著,只見玳安拿進盒兒來,說道:「何老爹家差人送請貼兒來,初九日請吃節酒。」西門慶道:「早是你看著,人家來請,你怎不去?」於是看盒兒內,放著三個請貼兒,一個雙紅僉兒,寫著「大寅丈四泉翁老先生大人」,一個寫「大都閫吳老先生大人」,一個寫著「大鄉望應老先生大人」,俱是「侍教生何永壽頓首拜」。玳安說:「他說不認的,教咱這裡轉送送兒去。」伯爵一見便說:「這個卻怎樣兒的?我還沒送禮兒去與他,怎好去?」西門慶道:「我這裡替你封上分帕禮兒,你差應寶早送去就是了。」一面令王經:「你封二錢銀子,一方手帕,寫你應二爹名字,與你應二爹。」因說:「你把這請貼兒袖了去,省的我又教人送。」只把吳大舅的差來安兒送去了。須臾,王經封了帕禮遞與伯爵。伯爵打恭說道:「又多謝哥,我後日早來會你,咱一同起身。」說畢,作辭去了。午間,吳月娘等打扮停當,一頂大轎,三頂小轎,後面又帶著來爵媳婦兒惠元,收疊衣服,一頂小轎兒,四名排軍喝道,琴童、春鴻、棋童、來安四個跟隨,往雲指揮家來吃酒。正是:

翠眉雲鬢畫中人,裊娜宮腰迥出塵。
天上嫦娥元有種,嬌羞釀出十分春。

不說月娘眾人吃酒去了。且說西門慶分付大門上平安兒:「隨問甚麼人,只說我不在。有貼兒接了就是了。」那平安經過一遭,那裡再敢離了左右,只在門首坐的。但有人客來望,只回不在家。西門慶因害腿疼,猛然想起任醫官與他延壽丹,用人乳吃。於是來到李瓶兒房中,叫迎春拿菜兒,篩酒來吃。迎春打發了,就走過隔壁,和春梅下棋去了。要茶要水,自有如意兒打發。西門慶見丫鬟不在屋裡,就在炕上斜靠著。露出那話,帶著銀托子,教他用口吮咂。一面斟酒自飲,因呼道: 「章四兒,我的兒,你用心替達達咂,我到明日,尋出件好妝花段子比甲兒來,你正月十二日穿。」老婆道:「看他可憐見。」咂弄勾一頓飯時,西門慶道:「我兒,我心裡要在你身上燒炷香兒。」老婆道:「隨爹揀著燒。」西門慶令他關上房門,把裙子脫了,仰臥在炕上。西門慶袖內還有燒林氏剩下的三個燒酒浸的香馬兒,撇去他抹胸兒,一個坐在他心口內,一個坐在他小肚兒底下,一個安在他蓋子上,用安息香一齊點著,那話下邊便插進牝中,低著頭看著拽,只顧沒棱露腦,往來迭進不已。又取過鏡臺來旁邊照看,須臾,那香燒到肉根前,婦人蹙眉嚙齒,忍其疼痛,口裡顫聲柔語,哼成一塊,沒口子叫:「達達,爹爹,罷了我了,好難忍他。」西門慶便叫道:「章四淫婦兒,你是誰的老婆?」婦人道:「我是爹的老婆。」西門慶教與他:「你說是熊旺的老婆,今日屬了我的親達達了。」那婦人回應道:「淫婦原是熊旺的老婆,今日屬了我的親達達了。」西門慶又問道:「我會肏不會?」婦人道:「達達會肏。」兩個淫聲艷語,無般言語不說出來。西門慶那話粗大,撐得婦人牝中滿滿,往來出入,帶的花心紅如鸚鵡舌,黑似蝙蝠翅,翻覆可愛。西門慶於是把他兩股扳拘在懷內,四體交匝,兩廂迎湊,那話盡沒至根,不容毫髮,婦人瞪目失聲,淫水流下,西門慶情濃樂極,精邈如泉涌。正是:

不知已透春消息,但覺形骸骨節熔。

西門慶燒了老婆身上三處春,開門尋了一件玄色段子妝花比甲兒與他。至晚,月娘眾人來家,對西門慶說:「原來雲二嫂也懷著個大身子,俺兩今日酒席上都遞了酒,說過,到明日兩家若分娩了,若是一男一女,兩家結親做親家;若都是男子,同堂攻書;若是女兒,拜做姐妹,一處做針指,來往親戚耍子。應二嫂做保證。」 西門慶聽的笑了。

,要轎子錢哩。」金蓮道:「我在這裡站著,他從多咱進去了?」琴童道:「姥姥打夾道里進去的。一來的轎子,該他六分銀子。」金蓮道:「我那得銀子?來人家來,怎不帶轎子錢兒走!」一面走到後邊,見了他娘,只顧不與他轎子錢,只說沒有。月娘道:「你與姥姥一錢銀子,寫帳就是了。」金蓮道:「我是不惹他,他的銀子都有數兒,只教我買東西,沒教我打發轎子錢。」坐了一回,大眼看小眼,外邊挨轎的催著要去。玉樓見不是事,向袖中拿出一錢銀子來,打發抬轎的去了。不一時,大妗子、二妗子、大師父來了,月娘擺茶吃了。潘姥姥歸到前邊他女兒房內來,被金蓮儘力數落了一頓,說道:「你沒轎子錢,誰教你來?恁出醜劃劃的,教人家小看!」潘姥姥道:「姐姐,你沒與我個錢兒,老身那討個錢兒來?好容易籌辦了這分禮兒來。」婦人道: 「指望問我要錢,我那裡討個錢兒與你?你看七個窟窿到有八個眼兒等著在這裡。今後你看有轎子錢便來他家來,沒轎子錢別要來。料他家也沒少你這個究親戚!休要做打踴的獻世包!『關王賣豆腐──人硬貨不硬』。我又聽不上人家那等屄聲顙氣。前日為你去了,和人家大嚷大鬧的,你知道也怎的?驢糞球兒面前光,卻不知裡面受凄惶。」幾句說的潘姥姥嗚嗚咽咽哭起來了。春梅道:「娘今日怎的,只顧說起姥姥來了。」一面安撫老人家,在裡邊炕上坐的,連忙點了盞茶與他吃。潘姥姥氣的在炕上睡了一覺,只見後邊請吃飯,才起來往後邊去了。

西門慶從衙門中來家,正在上房擺飯,忽有玳安拿進貼兒來說:「荊老爹升了東南統制,來拜爹。」西門慶見貼兒上寫:「新東南統制兼督漕運總兵官荊忠頓首拜。」慌的西門慶連忙穿衣,冠帶迎接出來。只見都總制穿著大紅麒麟補服、渾金帶進來,後面跟著許多僚掾軍牢。一面讓至大廳上敘禮畢,分賓主而坐,茶湯上來。荊統制說道:「前日陞官敕書才到,還未上任,徑來拜謝老翁。」西門慶道:「老總兵榮擢恭喜,大才必有大用,自然之道。吾輩亦有光矣,容當拜賀。」一面請寬尊服,少坐一飯。即令左右放卓兒,荊統制再三致謝道:「學生奉告老翁,一家尚未拜,還有許多薄冗,容日再來請教罷。」便要起身,西門慶那裡肯放,隨令左右上來,寬去衣服,登時打抹春台,收拾酒果上來。獸炭頓燒,暖簾低放。金壺斟下液,翠盞貯羊羔,才斟上酒來,只見鄭春、王相兩個小優兒來到,扒在面前磕頭。西門慶道:「你兩個如何這咱才來?」問鄭春:「那一個叫甚名字?」鄭春道:「他喚王相,是王桂的兄弟。」西門慶即令拿樂器上來彈唱。須臾,兩個小優哥唱了一套「霽景融和」。左右拿上兩盤攢盒點心嗄飯,兩瓶酒,打發馬上人等。荊統制道:「這等就不是了。學生叨擾,下人又蒙賜饌,何以克當?」即令上來磕頭。西門慶道:「一二日房下還要潔誠請尊正老夫人賞燈一敘,望乞下降。在座者惟老夫人、張親家夫人、同僚何天泉夫人,還有兩位舍親,再無他人。」荊統制道:「若老夫人尊票制,賤荊已定趨赴。」又問起:「周老總兵怎的不見升轉?」荊統制道:「我聞得周菊軒也只在三月間有京榮之轉。」西門慶道:「這也罷了。」坐不多時,荊統制告辭起身,西門慶送出大門,看著上馬喝道而去。

晚夕,潘金蓮上壽,後廳小優彈唱,遞了酒,西門慶便起身往金蓮房中去了。月娘陪著大妗子、潘姥姥、女兒鬱大姐、兩個姑子在上房會的飲酒。潘金蓮便陪西門慶在他房內,從新又安排上酒來,與西門慶梯己遞酒磕頭。落後潘姥姥來了,金蓮打發他李瓶兒這邊歇臥。他陪著西門慶自在飲酒,頑耍做一處。

卻說潘姥姥到那邊屋裡,如意、迎春讓他熱炕上坐著。先是姥姥看明間內靈前,供擺著許多獅仙五老定勝桌,旁邊掛著他影,因向前道了個問訊,說道:「姐姐好處生天去了。」進來坐在炕上,向如意兒、迎春道:「你娘勾了。官人這等費心追薦,受這般大供養,勾了。他是有福的。」如意兒道:「前日娘的生日,請姥姥,怎的不來?門外花大妗子和大妗子都在這裡來,十二個道士念經,好不大吹大打,揚幡道場,水火煉度,晚上才去了。」潘姥姥道:「幫年逼節,丟著個孩子在家,我來家中沒人,所以就不曾來。今日你楊姑娘怎的不見?」如意兒道:「姥姥還不知道,楊姑娘老病死了,從年裡俺娘念經就沒來,俺娘們都往北邊與他上祭去來。」 潘姥姥道:「可傷,他大如我,我還不曉的他老人家沒了。嗔道今日怎的不見他。」說了一回,如意兒道:「姥姥,有鐘甜酒兒,你老人家用些兒。」一面叫:「迎春姐,你放小卓兒在炕上,篩甜酒與姥姥吃杯。」不一時取到。飲酒之間,婆子又題起李瓶兒來:「你娘好人,有仁義的姐姐,熱心腸兒。我但來這裡,沒曾把我老娘當外人看承,一到就是熱茶熱水與我吃,還只恨我不吃。晚間和我坐著說話兒,我臨家去,好歹包些甚麼兒與我拿了去,再不曾空了我。不瞞你姐姐每說,我身上穿的這披襖兒,還是你娘與我的。正經我那冤家,半分折針兒也迸不出來與我。我老身不打誑語,阿彌陀佛,水米不打牙。他若肯與我一個錢兒,我滴了眼睛在地。你娘與了我些甚麼兒,他還說我小眼薄皮,愛人家的東西。想今日為轎子錢,你大包家拿著銀子,就替老身出幾分便怎的?咬定牙兒只說沒有,到教後邊西房裡姐姐,拿出一錢銀子來,打發抬轎的去了。歸到屋裡,還數落了我一頓,到明日有轎子錢,便教我來,沒轎子錢,休叫我上門走。我這去了不來了。來到這裡沒的受他的氣。隨他去,有天下人心狠,不似俺這短壽命。姐姐你每聽著我說,老身若死了,他到明日不聽人說,還不知怎麼收成結果哩!想著你從七歲沒了老子,我怎的守你到如今,從小兒交你做針指,往餘秀才家上女學去,替你怎麼纏手纏腳兒的,你天生就是這等聰明伶俐,到得這步田地?他把娘喝過來斷過去,不看一眼兒。」如意兒道:「原來五娘從小兒上學來,嗔道恁題起來就會識字深。」潘姥姥道:「他七歲兒上女學,上了三年,字仿也曾寫過,甚麼詩詞歌賦唱本上字不認的!」

人家熬不的,又早前靠後仰,打起盹來,方纔散了。

春梅便歸這邊來,推了推角門,開著,進入院內。只見秋菊正在明間板壁縫兒內,倚著春凳兒,聽他兩個在屋裡行房,怎的作聲喚,口中呼叫甚麼。正聽在熱鬧,不防春梅走到根前,向他腮頰上儘力打了個耳刮子,罵道:「賊少死的囚奴,你平白在這裡聽甚麼?」打的秋菊睜睜的,說道:「我這裡打盹,誰聽甚麼來,你就打我?」不想房裡婦人聽見,便問春梅,他和誰說話。春梅道:「沒有人,我使他關門,他不動。」於是替他摭過了。秋菊揉著眼,關上房門。春梅走到炕上,摘頭睡了。正是:

鶬鶊有意留殘景,杜宇無情戀晚暉。

一宿晚景題過。次日,潘金蓮生日,有傅伙計、甘伙計、賁四娘子、崔本媳婦、段大姐、吳舜臣媳婦、鄭三姐、吳二妗子,都在這裡。西門慶約會吳大舅、應伯爵,整衣冠,尊瞻視,騎馬喝道,往何千戶家赴席。那日也有許多官客,四個唱的,一起雜耍,周守備同席飲酒。至晚回家,就在前邊和如意兒歇了。

到初十日,發貼兒請眾官娘子吃酒,月娘便問西門慶說:「趁著十二日看燈酒,把門外的孟大姨和俺大姐,也帶著請來坐坐,省的教他知道惱,請人不請他。」西門慶道:「早是你說。」分付陳敬濟:「再寫兩個貼,差琴童兒請去。」這潘金蓮在旁,聽著多心,走到屋裡,一面攛掇潘姥姥就要起身。月娘道:「姥姥你慌去怎的?再消住一日兒是的。」金蓮道:「姐姐,大正月里,他家裡丟著孩子,沒人看,教他去罷。」慌的月娘裝了兩個盒子點心茶食,又與了他一錢轎子錢,管待打發去了。金蓮因對著李嬌兒說:「他明日請他有錢的大姨兒來看燈吃酒,一個老行貨子,觀眉觀眼的,不打發去了,平白教他在屋裡做甚麼?待要說是客人,沒好衣服穿。待要說是燒火的媽媽子,又不像。倒沒的教我惹氣。」因西門慶使玳安兒送了兩個請書兒,往招宣府,一個請林太太,一個請王三官兒娘子黃氏。又使他院中早叫李桂兒、吳銀兒、鄭愛月兒、洪四兒四個唱的,李銘、吳惠、鄭奉三個小優兒。不想那日賁四從東京來家,梳洗頭臉,打選衣帽齊整,來見西門慶磕頭。遞上夏指揮回書。西門慶問道:「你如何這些時不來?」賁四具言在京感冒打寒一節,「直到正月初二日,才收拾起身回來,夏老爹多上覆老爹,多承看顧。」西門慶照舊還把鑰匙教與他管絨線鋪。另打開一間,教吳二舅開鋪子賣綢絹,到明日松江貨舡到,都卸在獅子街房內,同來保發賣。且叫賁四叫花兒匠在家攢造兩架煙火,十二日要放與堂客看。

知道。」於是領了書禮,打在身邊,徑往李三家去了。

不說十一日來爵、春鴻同李三早雇了長行頭口,往兗州府去了。卻說十二日,西門慶家中請各堂客飲酒。那日在家不出門,約下吳大舅、謝希大、常峙節四位,晚夕來在捲棚內賞燈飲酒。王皇親家小廝,從早辰就挑了箱子來了,等堂客到,打銅鑼鼓迎接。周守備娘子有眼疾不得來,差人來回。止是荊統制娘子、張團練娘子、雲指揮娘子,並喬親家母、崔親家母、吳大姨、孟大姨,都先到了。只有何千戶娘子、王三官母親林太太並王三官娘子不見到。西門慶使排軍、玳安、琴童兒來回催邀了兩三遍,又使文嫂兒催邀。午間,只見林氏一頂大轎,一頂小轎跟了來。見了禮,請西門慶拜見,問:「怎的三官娘子不來?」林氏道:「小兒不在,家中沒人。」拜畢下來。止有何千戶娘子,直到晌午半日才來,坐著四人大轎,一個家人媳婦坐小轎跟隨,排軍抬著衣箱,又是兩個青衣人緊扶著轎扛,到二門裡才下轎。前邊鼓樂吹打迎接,吳月娘眾姊妹迎至儀門首。西門慶悄悄在西廂房,放下簾來偷瞧,見這藍氏年約不上二十歲,生的長挑身材,打扮的如粉妝玉琢,頭上珠翠堆滿,鳳翹雙插,身穿大紅通袖五彩妝花四獸麒麟袍兒,繫著金鑲碧玉帶,下襯著花錦藍裙,兩邊禁步叮咚,麝蘭撲鼻。但見:

儀容嬌媚,體態輕盈。姿性兒百伶百俐,身段兒不短不長。細彎彎兩道蛾眉,直侵入鬢;滴流流一雙鳳眼,來往踅人。嬌聲兒似囀日流鶯,嫩腰兒似弄風楊柳。端的是綺羅隊里生來,卻厭豪華氣象,珠翠叢中長大,那堪雅淡梳汝。開遍海棠花,也不問夜來多少;標殘楊柳絮,竟不知春意如何。輕移蓮步,有蕊珠仙子之風流;款蹙湘裙,似水月觀音之態度。正是:比花花解語,比玉玉生香。

這西門慶不見則已,一則魂飛天外,魄喪九霄,未曾體交,精魄先失。少頃,月娘等迎接進入後堂,相見敘禮已畢,請西門太拜見。西門慶得了這一聲,連忙整衣冠行禮,恍若瓊林玉樹臨凡,神女巫山降下,躬身施禮,心搖目盪,不能禁止。拜見畢下來,月娘先請在捲棚內擺過茶,然後大廳吹打,安席上坐,各依次序,當下林太太上席。戲文扮的是《小天香半夜朝元記》。唱的兩折下來,李桂姐、吳銀兒、鄭月兒、洪四兒四個唱的上去,彈唱燈詞。

西門慶在捲棚內,自有吳大舅、應伯爵、謝希大、常峙節、李銘、吳惠、鄭奉三個小優兒彈唱、飲酒,不住下來大廳格子外往裡觀覷。看官聽說,明月不常圓,彩雲容易散,樂極悲生,否極泰來,自然之理。西門慶但知爭名奪利,縱意奢淫,殊不知天道惡盈,鬼錄來追,死限臨頭。到晚夕堂中點起燈來,小優兒彈唱。還未到起更時分,西門慶陪人坐的,就在席上齁齁的打起睡來。伯爵便行令猜枚鬼混他,說道:「哥,你今日沒高興,怎的只打睡?」西門慶道:「我昨日沒曾睡,不知怎的,今日只是沒精神,要打睡。」只見四個唱的下來,伯爵教洪四兒與鄭月兒兩個彈唱,吳銀兒與李桂姐遞酒。

正耍在熱鬧處,忽玳安來報:「王太太與何老爹娘子起身了。」西門慶就下席來,黑影里走到二門裡首,偷看他上轎。月娘眾人送出來,前邊天井內看放煙火。藍氏已換了大紅遍地金貂鼠皮襖,林太太是白綾襖兒,貂鼠披風,帶著金釧玉珮。家人打燈籠,簇擁上轎而去。這西門慶正是餓眼將穿,饞涎空咽,恨不能就要成雙。見藍氏去了,悄悄從夾道進來。當時沒巧不成語,姻緣會湊,可霎作怪,來爵兒媳婦見堂客散了,正從後邊歸來,開房門,不想頂頭撞見西門慶,沒處藏躲。原來西門慶見媳婦子生的喬樣,安心已久,雖然不及來旺妻宋氏風流,也頗充得過第二。於是乘著酒興兒,雙關抱進他房中親嘴。這老婆當初在王皇親家,因是養主子,被家人不忿攘鬧,打發出來,今日又撞著這個道路,如何不從了?一面就遞舌頭在西門慶口中。兩個解衣褪褲,就按在炕沿子上,掇起腿來,被西門慶就聳了個不亦樂乎。正是:未曾得遇鶯娘面,且把紅娘去解饞。有詩為證:

燈月交光浸玉壺,分得清光照綠珠。
莫道使君終有婦,教人桑下覓羅敷。

