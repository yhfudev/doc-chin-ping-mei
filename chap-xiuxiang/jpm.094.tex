%# -*- coding: utf-8 -*-
%!TEX encoding = UTF-8 Unicode
%!TEX TS-program = xelatex
% vim:ts=4:sw=4
%
% 以上设定默认使用 XeLaTex 编译,并指定 Unicode 编码,供 TeXShop 自动识别

%第九十四回 
\chapter{大酒樓劉二撒潑 灑家店雪娥為娼}

詩曰:

骨肉傷殘產業荒,一身何忍去歸娼。
淚垂玉箸辭官舍,步蹴金蓮入教坊。
覽鏡自憐傾國色,向人初學倚門妝。
春來雨露寬如海,嫁得劉郎勝阮郎。

話說陳敬濟自從謝家酒樓上見了馮金寶,兩個又勾搭上前情。往後沒三日不和他相會,或一日敬濟有事不去,金寶就使陳三兒稍寄物事,或寫情書來叫他去。一次或五錢,或一兩。以後日間供其柴米,納其房錢。歸到廟中便臉紅。任道士問他何處吃酒來,敬濟只說:「在米鋪和伙計暢飲三杯,解辛苦來。」他師兄金宗明一力替他遮掩,晚夕和他一處盤弄那勾當,是不必說。朝來暮往,把任道士囊篋中細軟的本錢,也抵盜出大半花費了。

一日,也是合當有事。這灑家店的劉二,有名坐地虎,他是帥府周守備府中親隨張勝的小舅子,專一在馬頭上開娼店,倚強凌弱,舉放私債,與巢窩中各娼使用,加三討利。有一不給,搗換文書,將利作本,利上加利。嗜酒行兇,人不敢惹他。就是打粉頭的班頭,欺酒客的領袖。因見陳敬濟是宴公廟任道士的徒弟,白臉小廝,謝三家大酒上把粉頭鄭金寶兒佔住了,吃的楞楞睜睜,提著碗頭大的拳頭,走來謝家樓下,問:「金寶在那裡?」慌的謝三郎連忙聲喏,說道:「劉二叔叔,他在樓上第二間閣兒里便是。」這劉二大叉步上樓來。敬濟正與金寶在閣兒裡面飲酒,做一處快活,把房門關閉,外邊帘子掛著。被劉二一把手扯下帘子,大叫:「金寶兒出來!」唬的陳敬濟鼻口內氣兒也不敢出。這劉二用腳把門跺開,金寶兒只得出來相見,說:「劉二叔叔,有何說話?」劉二罵道:「賊淫婦,你少我三個月房錢,卻躲在這裡,就不去了。」金寶笑嘻嘻說道:「二叔叔,你家去,我使媽媽就送房錢來。」這劉二隻摟心一拳,打了老婆一交,把頭顱搶在階沿下磕破,血流滿地,罵道:「賊淫婦,還等甚送來,我如今就要!」看見陳敬濟在裡面,走向前把桌子只一掀,碟兒打得粉碎。那敬濟便道:「阿呀,你是甚麼人?走來撒野。」劉二罵道:「我肏你道士秫秫娘!」一手採過頭髮來,按在地下,拳捶腳踢無數。那樓上吃酒的人,看著都立睜了。店主人謝三初時見劉二醉了,不敢惹他,次後見打得人不像模樣,上樓來解勸,說道:「劉二叔,你老人家息怒。他不曉得你老人家大名,誤言衝撞,休要和他一般見識,看小人薄面,饒他去罷。」這劉二那裡依從,儘力把敬濟打了個發昏章第十一。叫將地方保甲,一條繩子,連粉頭都拴在一處墩鎖,分付:「天明早解到老爺府里去。」原來守備敕書上命他保障地方,巡捕盜賊,兼管河道。這裡拿了敬濟,任道士廟中尚還不知,只說晚夕米鋪中上宿未回。

卻說次日,地方保甲、巡河快手押解敬濟、金寶,雇頭口趕清晨早到府前伺候。先遞手本與兩個管事張勝、李安看,說是劉二叔地方喧鬧一起,宴公廟道士一名陳宗美,娼婦鄭金寶。眾軍牢都問他要錢,說道:「俺們是廳上動刑的,一班十二人,隨你罷。正經兩位管事的,你倒不可輕視了他。」敬濟道:「身邊銀錢倒有,都被夜晚劉二打我時,被人掏摸的去了。身上衣服都扯碎了,那得錢來?止有頭上關頂一根銀簪兒,拔下來,與二位管事的罷。」眾牢子拿著那根簪子,走來對張勝、李安如此這般說:「他一個錢兒不拿出來,止與了這根簪兒,還是鬧銀的。」張勝道:「你叫他近前,等我審問他。」眾軍牢不一時擁到跟前跪下,問:「你幾時與任道士做徒弟?俗名叫甚麼?我從未見你。」敬濟道:「小的俗名叫陳敬濟,原是好人家兒女,做道士不久。」張勝道:「你既做道士,便該習學經典,許你在外宿娼飲酒喧嚷?你把俺帥府衙門當甚麼些小衙門,不拿了錢兒來,這根簪子打水不渾,要他做甚?」還掠與他去。分付牢子:「等住回老爺升廳,把他放在頭一起。眼見這狗男女道士,就是個吝錢的,只許你白要四方施主錢糧!休說你為官事,你就來吃酒赴席,也帶方汗巾兒揩嘴。等動刑時,著實加力拶打這廝。」又把鄭金寶叫上去。鄭家有忘八跟著,上下打發了三四兩銀子。張勝說:「你系娼門,不過趁熟趕些衣食為生,沒甚大事。看老爺喜怒不同,看惱只是一兩拶子;若喜歡,只恁放出來也不知。」不一時,只見裡面雲板響,守備升廳,兩邊僚掾軍牢森列,甚是齊整。但見:

緋羅繳壁,紫綬桌圍。當廳額掛茜羅,四下簾垂翡翠。勘官守正,戒石上刻御制四行;人從謹廉,鹿角旁插令旗兩面。軍牢沉重,僚掾威儀。執大棍授事立階前,挾文書廳旁聽發放。雖然一路帥臣,果是滿堂神道。

當時,沒巧不成話,也是五百劫冤家聚會,姻緣合當湊著。春梅在府中,從去歲八月間,已生了個哥兒小衙內。今方半歲光景,貌如冠玉,唇若塗朱。守備喜似席上之珍,愛如無價之寶。未幾,大奶奶下世,守備就把春梅冊正,做了夫人。就住著五間正房,買了兩個養娘抱奶哥兒,一名玉堂,一名金匱;兩個小丫鬟服侍,一名翠花,一名蘭花;又有兩個身邊得寵彈唱的姐兒,都十六七歲,一名海棠,一名月桂,都在春梅房中侍奉。那孫二娘房中止使著一個丫鬟,名喚荷花兒,不在話下。每常這小衙內,只要張勝抱他外邊頑耍,遇著守備升廳,便在旁邊觀看。

當日,守備升廳坐下,放了告牌出去,各地方解進人來。頭一起就叫上陳敬濟並娼婦鄭金寶兒去。守備看了呈狀,便說道:「你這廝是個道士,如何不守清規,宿娼飲酒,騷擾地方,行止有虧。左右拿下去,打二十棍,追了度牒還俗。那娼婦鄭氏,拶一拶,敲五十敲,責令歸院當差。」兩邊軍牢向前,才待扯翻敬濟,攤去衣服,用繩索綁起,轉起棍來,兩邊招呼要打時,可霎作怪,張勝抱著小衙內,正在月臺上站立觀看,那小衙內看見打敬濟,便在懷裡攔不住,撲著要敬濟抱。張勝恐怕守備看見,忙走過來。那小衙內亦發大哭起來,直哭到後邊春梅跟前。春梅問:「他怎的哭?」張勝便說:「老爺廳上發放事,打那宴公廟陳道士,他就撲著要他抱,小的走下來,他就哭了。」

這春梅聽見是姓陳的,不免輕移蓮步,款蹙湘裙,走到軟屏後面探頭觀覷:「打的那人,聲音模樣,倒好似陳姐夫一般,他因何出家做了道士?」又叫過張勝,問他:「此人姓甚名誰?」張勝道:「這道士我曾問他來,他說俗名叫陳敬濟。」春梅暗道:「正是他了。」一面使張勝:「請下你老爺來。」這守備廳上打敬濟才打到十棍,一邊還拶著唱的,忽聽後邊夫人有請,分付牢子把棍且閣住休打,一面走下廳來。春梅說道:「你打的那道士,是我姑表兄弟,看奴面上,饒了他罷。」守備道:「夫人何不早說,我已打了他十棍,怎生奈何?」一面出來,分付牢子:「都與我放了。」唱的便歸院去了。守備悄悄使張勝:「叫那道士回來,且休去。問了你奶奶,請他相見。」這春梅才待使張勝請他到後堂相見,忽然沉吟想了一想,便又分付張勝:「你且叫那人去著,待我慢慢再叫他。」度牒也不曾追。

這陳敬濟打了十棍,出離了守備府,還奔來晏公廟。不想任道士聽見人來說:「你那徒弟陳宗美,在大酒樓上包著唱的鄭金寶兒,惹了灑家店坐地虎劉二,打得臭死,連老婆都拴了,解到守備府去了。行止有虧,便差軍牢來拿你去審問,追度牒還官。」這任道士聽了,一者老年的著了驚怕,二來身體胖大,因打開囊篋,內又沒有許多細軟東西,著了口重氣,心中痰湧上來,昏倒在地。眾徒弟慌忙向前扶救,請將醫者來灌下藥去,通不省人事。到半夜,嗚呼斷氣身亡。亡年六十三歲。第二日,陳敬濟來到,左右鄰人說:「你還敢廟裡去?你師父因為你,如此這般,得了口重氣,昨夜三更鼓死了。」這敬濟聽了,唬的忙忙似喪家之犬,急急如漏網之魚,復回清河縣城中來。正是:

鹿隨鄭相應難辯,蝶化莊周未可知。

話分兩頭。卻說春梅一面使張勝叫敬濟且去著,一面走歸房中,摘了冠兒,脫了繡服,倒在床上,便捫心撾被,聲疼叫喚起來。唬的合宅大小都慌了。下房孫二娘來問道:「大奶奶才好好的,怎的就不好起來?」春梅說:「你每且去,休管我。」落後守備退廳進來,見他躺在床上叫喚,也慌了。扯著他手兒問道:「你心裡怎的來?」也不言語,又問:「那個惹著你來?」也不做聲。守備道:「不是我剛纔打了你兄弟,你心內惱麼?」亦不應答。這守備無計奈何,走出外邊麻犯起張勝、李安來了:「你兩個早知他是你奶奶兄弟,如何不早對我說?卻教我打了他十下,惹的你奶奶心中不自在。我曾教你留下他,請你奶奶相見,你如何又放他去了?你這廝每卻討分曉!」張勝說:「小的曾稟過奶奶來,奶奶說且教他去著,小的才放他去了。」一面走入房中,哭哭啼啼,哀告春梅:「望乞奶奶在爺前方便一言。不然,爺要見責小的每哩。」這春梅睜圓星眼,剔起蛾眉,叫過守備近前說:「我自心中不好,乾他們甚事?那廝他不守本分,在外邊做道士,且奈他些時,等我慢慢招認他。」這守備才不麻犯張勝、李安了。

守備見他只管聲喚,又使張勝請下醫官來看脈,說:「老安人染了六欲七情之病,著了重氣在心。」討將藥來又不吃,都放冷了。丫頭每都不敢向前說話,請將守備來看著吃藥,只呷了一口,就不吃了。守備出去了,大丫鬟月桂拿過藥來,「請奶奶吃藥。」被春梅拿過來,匹臉只一潑,罵道:「賊浪奴才,你只顧拿這苦水來灌我怎的?我肚子里有甚麼?」教他跪在面前。孫二娘走來,問道:「月桂怎的?奶奶教他跪著。」海棠道:「奶奶因他拿藥與奶奶吃來,奶奶說:『我肚子里有甚麼?拿這藥來灌我。』教他跪著。」孫二娘道:「奶奶,你委的今一日沒曾吃甚麼。這月桂他不曉得,奶奶休打他,看我面上,饒他這遭罷。」分付海棠:「你往廚下熬些粥兒來,與你奶奶吃口兒。」春梅於是把月桂放起來。

那海棠走到廚下,用心用意熬了一小鍋粳米濃濃的粥兒,定了四碟小菜兒,用甌兒盛著,熱烘烘拿到房中。春梅躺在床上面朝里睡,又不敢叫,直待他番身,方纔請他:「有了粥兒在此,請奶奶吃粥。」春梅把眼合著,不言語。海棠又叫道:「粥晾冷了,請奶奶起來吃粥。」孫二娘在旁說道:「大奶奶,你這半日沒吃甚麼,這回你覺好些,且起來吃些個。」那春梅一骨碌子扒起來,教奶子拿過燈來,取粥在手,只呷了一口,往地下只一推。早是不曾把傢伙打碎,被奶子接住了。就大吆喝起來,向孫二娘說:「你平白叫我起來吃粥,你看賊奴才熬的好粥!我又不坐月子,熬這照面湯來與我吃怎麼?」分付奶子金匱:「你與我把這奴才臉上打與他四個嘴巴!」當下真個把海棠打了四個嘴巴。孫二娘便道:「奶奶,你不吃粥,卻吃些甚麼兒?卻不餓著你。」春梅道:「你教我吃,我心內攔著,吃不下去。」良久,叫過小丫鬟蘭花兒來,分付道:「我心內想些雞尖湯兒吃。你去廚房內,對那淫婦奴才,教他洗手做碗好雞尖湯兒與我吃。教他多放些酸筍,做的酸酸辣辣的我吃。」孫二娘便說:「奶奶分付他,教雪娥做去。你心下想吃的就是藥。」

這蘭花不敢怠慢,走到廚下對雪娥說:「奶奶教你做雞尖湯,快些做,等著要吃哩。」原來這雞尖湯,是雛雞脯翅的尖兒碎切的做成湯。這雪娥一面洗手剔甲,旋宰了兩隻小雞,退刷乾凈,剔選翅尖,用快刀碎切成絲,加上椒料、蔥花、芫荽、酸筍、油醬之類,揭成清湯。盛了兩甌兒,用紅漆盤兒,熱騰騰,蘭花拿到房中。春梅燈下看了,呷了一口,怪叫大罵起來:「你對那淫婦奴才說去,做的甚麼湯!精水寡淡,有些甚味?你們只教我吃,平白叫我惹氣!」慌的蘭花生怕打,連忙走到廚下對雪娥說:「奶奶嫌湯淡,好不罵哩。」這雪娥一聲兒不言語,忍氣吞聲,從新洗鍋,又做了一碗。多加了些椒料,香噴噴,教蘭花兒拿到房裡來。春梅又嫌忒咸了,拿起來照地下只一潑,早是蘭花躲得快,險些兒潑了一身。罵道:「你對那奴才說去,他不憤氣做與我吃。這遭做的不好,教他討分曉。」這雪娥聽見,千不合,萬不合,悄悄說了一句:「姐姐幾時這般大了,就抖摟起人來!」不想蘭花回到房裡,告春梅說了。這春梅不聽便罷,聽了此言,登時柳眉剔豎,星眼圓睜,咬碎銀牙,通紅了粉面,大叫:「與我採將那淫婦奴才來!」

須臾,使了奶娘丫鬟三四個,登時把雪娥拉到房中。春梅氣狠狠的一手扯住他頭髮,把頭上冠子跺了,罵道:「淫婦奴才,你怎的說幾時這般大?不是你西門慶家抬舉的我這般大!我買將你來伏侍我,你不憤氣,教你做口子湯,不是精淡,就是苦咸。你倒還對著丫頭說我幾時恁般大起來,摟搜索落我,要你何用?」一面請將守備來,採雪娥出去,當天井跪著。前邊叫將張勝、李安,旋剝褪去衣裳,打三十大棍。兩邊家人點起明晃晃燈籠,張勝、李安各執大棍伺候。那雪娥只是不肯脫衣裳。守備恐怕氣了他,在跟前不敢言語。孫二娘在旁邊再三勸道:「隨大奶奶分付打他多少,免褪他小衣罷。不爭對著下人,脫去他衣服,他爺體面上不好看的。只望奶奶高抬貴手,委的他的不是了。」春梅不肯,定要去他衣服打,說道:「那個攔我,我把孩子先摔殺了,然後我也一條繩子弔死就是了。留著他便是了。」於是也不打了,一頭撞倒在地,就直挺挺的昏迷,不省人事。守備唬的連忙扶起,說道:「隨你打罷,沒的氣著你。」當下可憐把這孫雪娥拖番在地,褪去衣服,打了三十大棍,打的皮開肉綻。一面使小牢子半夜叫將薛嫂兒來,即時罄身領出去辦賣。

春梅把薛嫂兒叫在背地,分付:「我只要八兩銀子,將這淫婦奴才好歹與我賣在娼門。隨你轉多少,我不管你。你若賣在別處,我打聽出來,只休要見我。」那薛嫂兒道:「我靠那裡過日子,卻不依你說?」當夜領了雪娥來家。那雪娥悲悲切切,整哭到天明。薛嫂便勸道:「你休哭了,也是你的晦氣,冤家撞在一處。老爺見你到罷了,只恨你與他有些舊仇舊恨,折挫你。連老爺也做不得主兒,見他有孩子,凡事依隨他。正經下邊孫二娘也讓他幾分。常言拐米倒做了倉官,說不的了,你休氣哭。」雪娥收淚,謝薛嫂:「只望早晚尋個好頭腦我去,只有飯吃罷。」薛嫂道:「他千萬分付,只教我把你送在娼門。我養兒養女,也要天理。等我替你尋個單夫獨妻,或嫁個小本經紀人家,養活得你來也罷。」那雪娥千恩萬福謝了。

薛嫂過了兩日,只見鄰居一個開店張媽走來叫:「薛媽,你這壁廂有甚娘子?怎的哭的悲切?」薛嫂便道:「張媽,請進來坐。」說道:「便是這位娘子,他是大人家出來的,因和大娘子合不著,打發出來,在我這裡嫁人。情願個單夫獨妻,免得惹氣。」張媽媽道:「我那邊下著一個山東賣綿花客人,姓潘,排行第五,年三十七歲,幾車花果,常在老身家安下。前日說他家有個老母有病,七十多歲,死了渾家半年光景,沒人伏侍。再三和我說,替他保頭親事,並無相巧的。我看來這位娘子年紀到相當,嫁與他做個娘子罷。」薛嫂道:「不瞞你老人家說,這位娘子大人家出身,不拘粗細都做的,針指女工,自不必說,又做的好湯水。今才三十五歲。本家只要三十兩銀子,倒好保與他罷。」張媽媽道:「有箱籠沒有?」薛嫂道:「止是他隨身衣服、簪環之類,並無箱籠。」張媽媽道:「既是如此,老身回去對那人說,教他自家來看一看。」說畢,吃茶,坐回去了。晚夕對那人說了,次日飯罷以後,果然領那人來相看。一見了雪娥好模樣兒,年小,一口就還了二十五兩,另外與薛嫂一兩媒人錢。薛嫂也沒爭競,就兌了銀子,寫了文書。晚夕過去,次日就上車起身。薛嫂教人改換了文書,只兌了八兩銀子交到府中,春梅收了,只說賣與娼門去了。

那人娶雪娥到張媽家,止過得一夜,到第二日,五更時分,謝了張媽媽,作別上了車,徑到臨清去了。此是六月天氣,日子長,到馬頭上才日西時分。到於灑家店,那裡有百十間房子,都下著各處遠方來的窠子行院唱的。這雪娥一領入一個門戶,半間房子,裡面炕上坐著個五六十歲的婆子,還有個十七頂老丫頭,打著盤頭揸髻,抹著鉛粉紅唇,穿著一弄兒軟絹衣服,在炕邊上彈弄琵琶。這雪娥看見,只叫得苦,才知道那漢子潘五是個水客。買他來做粉頭。起了他個名叫玉兒。這小妮子名喚金兒,每日拿廝鑼兒出去,酒樓上接客供唱,做這道路營生。這潘五進門不問長短,把雪娥先打了一頓,睡了兩日,只與他兩碗飯吃,教他學樂器彈唱,學不會又打,打得身上青紅遍了。引上道兒,方與他好衣穿,妝點打扮,門前站立,倚門獻笑,眉目嘲人。正是:遺蹤堪入府人眼,不買胭脂畫牡丹。有詩為證:

窮途無奔更無投,南去北來休更休。
一夜彩雲何處散,夢隨明月到青樓。

這雪娥在灑家店,也是天假其便。一日,張勝被守備差遣往河下買幾十石酒麴,宅中造酒。這灑家店坐地虎劉二,看見他姐夫來,連忙打掃酒樓乾凈,在上等閣兒里安排酒餚杯盤,請張勝坐在上面飲酒。酒博士保兒篩酒,稟問:「二叔,下邊叫那幾個唱的上來遞酒?」劉二分付:「叫王家老姐兒,趙家嬌兒,潘家金兒,玉兒四個上來,伏侍你張姑夫。」酒博士保兒應諾下樓。不多時,只聽得胡梯畔笑聲兒,一般兒四個唱的,打扮得如花似朵,都穿著輕紗軟絹衣裳,上的樓來,望上拜了四拜,立在旁邊。這張勝猛睜眼觀看,內中一個粉頭,可霎作怪,「到相老爺宅里打發出來的那雪娥娘子。他如何做這道路在這裡?」那雪娥亦眉眼掃見是張勝,都不做聲。這張勝便問劉二:「那個粉頭是誰家的?」劉二道:「不瞞姐夫,他是潘五屋裡玉兒、金兒,這個是王老姐,一個是趙嬌兒。」張勝道:「這潘家玉兒,我有些眼熟。」因叫他近前,悄悄問他:「你莫不是雪姑娘麼?怎生到於此處?」那雪娥聽見他問,便簇地兩行淚下,便道:「一言難盡。」如此這般,具說一遍。「被薛嫂攛瞞,把我賣了二十五兩銀子,賣在這裡供筵席唱,接客迎人。」這張勝平昔見他生的好,常是懷心。這雪娥席前殷勤勸酒,兩個說得入港。雪娥和金兒不免拿過琵琶來,唱個詞兒,與張勝下酒。唱畢,彼此穿杯換盞,倚翠偎紅,吃得酒濃時,常言:「世財紅粉歌樓酒,誰為三般事不迷?」這張勝就把雪娥來愛了。兩個晚夕留在閣兒里,就一處睡了。這雪娥枕邊風月,耳畔山盟,和張勝儘力盤桓,如魚似水,百般難述。次日起來,梳洗了頭面,劉二又早安排酒餚上來,與他姐夫扶頭。大盤大碗,饕食一頓,收起行裝,喂飽頭口,裝載米曲,伴當跟隨。臨出門,與了雪娥三兩銀子,分付劉二:「好生看顧他,休教人欺負。」自此以後,張勝但來河下,就在灑家店與雪娥相會。往後走來走去,每月與潘五幾兩銀子,就包住了他,不許接人。那劉二自恁要圖他姐夫歡喜,連房錢也不問他要了。各窠窩刮刷將來,替張勝出包錢,包定雪娥柴米。有詩為證:

豈料當年縱意為,貪淫倚勢把心欺。
禍不尋人人自取,色不迷人人自迷。

