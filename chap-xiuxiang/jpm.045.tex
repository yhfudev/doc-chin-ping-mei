%# -*- coding: utf-8 -*-
%!TEX encoding = UTF-8 Unicode
%!TEX TS-program = xelatex
% vim:ts=4:sw=4
%
% 以上设定默认使用 XeLaTex 编译,并指定 Unicode 编码,供 TeXShop 自动识别

%第四十五回 
\chapter{應伯爵勸當銅鑼 李瓶兒解衣銀姐}

詞曰:

徘徊。相期酒會,三千朱履,十二金釵。雅俗熙熙,下車成宴盡春台。好雍容、東山妓女,堪笑傲、北海樽壘。且追陪。鳳池歸去,那更重來!

話說西門慶因放假沒往衙門裡去,早晨起來,前廳看著,差玳安送兩張桌面與喬家去。一張與喬五太太,一張與喬大戶娘子,俱有高頂方糖、時鮮樹果之類。喬五太太賞了兩方手帕、三錢銀子,喬大戶娘子是一匹青絹,俱不必細說。

原來應伯爵自從與西門慶作別,趕到黃四家。黃四又早夥中封下十兩銀子謝他:「大官人吩咐教俺過節去,口氣只是搗那五百兩銀子文書的情。你我錢糧拿甚麼支持?」應伯爵道:「你如今還得多少才夠?」黃四道:「李三哥他不知道,只要靠著問那內臣借,一般也是五分行利。不如這裡藉著衙門中勢力兒,就是上下使用也省些。如今我算再借出五十個銀子來,把一千兩合用,就是每月也好認利錢。」應伯爵聽了,低了低頭兒,說道:「不打緊。假若我替你說成了,你夥計六人怎生謝我?」黃四道:「我對李三說,夥中再送五兩銀子與你。」伯爵道:「休說五兩的話。要我手段,五兩銀子要不了你的,我只消一言,替你每巧一巧兒,就在裡頭了。今日俺房下往他家吃酒,我且不去。明日他請俺們晚夕賞燈,你兩個明日絕早買四樣好下飯,再著上一壇金華酒。不要叫唱的,他家裡有李桂兒、吳銀兒,還沒去哩!你院里叫上六個吹打的,等我領著送了去。他就要請你兩個坐,我在旁邊,只消一言半句,管情就替你說成了。找出五百兩銀子來,共搗一千兩文書,一個月滿破認他三十兩銀子,那裡不去了,只當你包了一個月老婆了。常言道:秀才無假漆無真。進錢糧之時,香裡頭多放些木頭,蠟裡頭多摻些柏油,那裡查帳去?不圖打魚,只圖混水,藉著他這名聲兒,才好行事。」於是計議己定。到次日,李三、黃四果然買了酒禮,伯爵領著兩個小廝,抬送到西門慶家來。

西門慶正在前廳打發桌面,只見伯爵來到,作了揖,道及:「昨日房下在這裡打攪,回家晚了。」西門慶道:「我昨日周南軒那裡吃酒,回家也有一更天氣,也不曾見的新親戚,老早就去了。今早衙門中放假,也沒去。」說畢坐下,伯爵就喚李錦:「你把禮抬進來。」不一時,兩個抬進儀門裡放下。伯爵道:「李三哥、黃四哥再三對我說,受你大恩,節間沒甚麼,買了些微禮來,孝順你賞人。」只見兩個小廝向前磕頭。西門慶道:「你們又送這禮來做甚麼?我也不好受的,還教他抬回去。」伯爵道:「哥,你不受他的,這一抬出去,就醜死了。他還要叫唱的來伏侍,是我阻住他了,只叫了六名吹打的在外邊伺候。」西門慶向伯爵道:「他既叫將來了,莫不又打發他?不如請他兩個來坐坐罷。」伯爵得不的一聲兒,即叫過李錦來,吩咐:「到家對你爹說:老爹收了禮了,這裡不著人請去了,叫你爹同黃四爹早來這裡坐坐。」那李錦應諾下去。須臾,收進禮去。令玳安封二錢銀子賞他,磕頭去了。六名吹打的下邊伺候。

少頃,棋童兒拿茶來,西門慶陪伯爵吃了茶,就讓伯爵西廂房裡坐。因問伯爵:「你今日沒會謝子純?」伯爵道:「我早晨起來時,李三就到我那裡,看著打發了禮來,誰得閑去會他?」西門慶即使棋童兒:「快請你謝爹去!」不一時,書童兒放桌兒擺飯,兩個同吃了飯,收了傢伙去。西門慶就與伯爵兩個賭酒兒打雙陸。伯爵趁謝希大未來,乘先問西門慶道:「哥,明日找與李智、黃四多少銀子?」西門慶道:「把舊文書收了,另搗五百兩銀子文書就是了。」伯爵道:「這等也罷了。哥,你不如找足了一千兩,到明日也好認利錢。我又一句話,那金子你用不著,還算一百五十兩與他,再找不多兒了。」西門慶聽罷,道:「你也說的是。我明日再找三百五十兩與他罷,改一千兩銀子文書就是了,省的金子放在家,也只是閑著。」

兩個正打雙陸,忽見玳安兒來說道:「賁四拿了一座大螺鈿大理石屏鳳、兩架銅鑼銅鼓連鐺兒,說是白皇親家的,要當三十兩銀子,爹當與他不當?」西門慶道: 「你教賁四拿進來我瞧。」不一時,賁四與兩個人抬進去,放在廳堂上。西門慶與伯爵丟下雙陸,走出來看,原來是三尺闊五尺高可桌放的螺鈿描金大理石屏鳳,端的黑白分明。伯爵觀了一回,悄與西門慶道:「哥,你仔細瞧,恰好似蹲著個鎮宅獅子一般。兩架銅鑼銅鼓,都是彩畫金妝,雕刻雲頭,十分齊整。」在旁一力攛掇,說道:「哥,該當下他的。休說兩架銅鼓,只一架屏鳳,五十兩銀子還沒處尋去。」西門慶道:「不知他明日贖不贖。」伯爵道:「沒的說,贖甚麼?下坡車兒營生,及到三年過來,七本八利相等。」西門慶道:「也罷,教你姐夫前邊鋪子里兌三十兩與他罷。」剛打發去了,西門慶把屏鳳拂抹乾凈,安在大廳正面,左右看視,金碧彩霞交輝。因問:「吹打樂工吃了飯不曾?」琴童道:「在下邊吃飯哩。」西門慶道:「叫他吃了飯來吹打一回我聽。」於是廳內抬出大鼓來,穿廊下邊一帶安放銅鑼銅鼓,吹打起來,端的聲震雲霄,韻驚魚鳥。正吹打著,只見棋童兒請謝希大到了。進來與二人唱了喏,西門慶道:「謝子純,你過來估估這座屏風兒,值多少價?」謝希大近前觀看了半日,口裡只顧誇獎不已,說道:「哥,你這屏風,買得巧也得一百兩銀子,少也他不肯。」伯爵道:「你看,連這外邊兩架銅鑼銅鼓,帶鐺鐺兒,通共用了三十兩銀子。」那謝希大拍著手兒叫道:「我的南無耶,那裡尋本兒利兒!休說屏風,三十兩銀子還攪給不起這兩架銅鑼銅鼓來。你看這兩座架子,做的這工夫,朱紅彩漆,都照依官司里的樣範,少說也有四十斤響銅,該值多少銀子?怪不的一物一主,那裡有哥這等大福,偏有這樣巧價兒來尋你的。」

說了一回,西門慶請入書房裡坐的。不一時,李智、黃四也到了。西門慶說道:「你兩個如何又費心送禮來?我又不好受你的。」那李智、黃四慌的說道:「小人惶恐,微物胡亂與老爹賞人罷了。蒙老爹呼喚,不敢不來。」於是搬過座兒來,打橫坐了。須臾,小廝畫童兒拿了五盞茶上來,眾人吃了。少頃,玳安走上來請問: 「爹,在那裡放桌兒?」西門慶道:「就在這裡坐罷。」於是玳安與畫童兩個抬了一張八仙桌兒,騎著火盆安放。伯爵、希大居上,西門慶主位,李智、黃四兩邊打橫坐了。須臾,拿上春檠按酒,大盤大碗湯飯點心、各樣下飯。酒泛羊羔,湯浮桃浪。樂工都在窗外吹打。西門慶叫了吳銀兒席上遞酒,這裡前邊飲酒不題。

卻說李桂姐家保兒,吳銀兒家丫頭蠟梅,都叫了轎子來接。那桂姐聽見保兒來,慌的走到門外,和保兒兩個悄悄說了半日話,回到上房告辭要回家去。月娘再三留他道:「俺每如今便都往吳大妗子家去,連你每也帶了去。你越發晚了從他那裡起身,也不用轎子,伴俺每走百病兒,就往家去便了。」桂姐道:「娘不知,我家裡無人,俺姐姐又不在家,有我五姨媽那裡又請了許多人來做盒子會,不知怎麼盼我。昨日等了我一日,他不急時,不使將保兒來接我。若是閑常日子,隨娘留我幾日我也住了。」月娘見他不肯,一面教玉簫將他那原來的盒子,裝了一盒元宵、一盒白糖薄脆,交與保兒掇著,又與桂姐一兩銀子,打發他回去。這桂姐先辭月娘眾人,然後他姑娘送他到前邊,叫畫童替他抱了氈包,竟來書房門首,教玳安請出西門慶來說話。這玳安慢慢掀帘子進入書房,向西門慶請道:「桂姐家去,請爹說話。」 應伯爵道:「李桂兒這小淫婦兒,原來還沒去哩。」西門慶道:「他今日才家去。」一面走出前邊來。李姐與西門慶磕了四個頭,就道:「打攪爹娘這裡。」西門慶道:「你明日家去罷。」桂姐道:「家裡無人,媽使保兒拿轎子來接了。」又道:「我還有一件事對爹說:俺姑娘房裡那孩子,休要領出去罷。俺姑娘昨日晚夕又打了他幾下。說起來還小哩,也不知道甚麼,吃我說了他幾句,從今改了,他說再不敢了。不爭打發他出去,大節間,俺姑娘房中沒個人使,他心裡不急麼?自古木杓火杖兒短,強如手撥剌,爹好歹看我分上,留下這丫頭罷。」西門慶道:「既是你恁說,留下這奴才罷。」就吩咐玳安:「你去後邊對你大娘說,休要叫媒人去了。」玳安見畫童兒抱著桂姐氈包,說道:「拿桂姨氈包等我抱著,教畫童兒後邊說去罷。」那畫童應諾,一直往後邊去了。桂姐與西門慶說畢,又到窗子前叫道: 「應花子,我不拜你了,你娘家去。」伯爵道:「拉回賊小淫婦兒來,休放他去了,叫他且唱一套兒與我聽聽著。」桂姐道:「等你娘閑了唱與你聽。」伯爵道: 「恁大白日就家去了,便益了賊小淫婦兒了,投到黑還接好幾個漢子。」桂姐道:「汗邪了你這花子!」一面笑了出去。玳安跟著,打發他上轎去了。

西門慶與桂姐說了話,就後邊更衣去了。應伯爵向謝希大說:「李家桂兒這小淫婦兒,就是個真脫牢的強盜,越發賊的疼人子!恁個大節,他肯只顧在人家住著?鴇子來叫他,又不知家裡有甚麼人兒等著他哩。」謝希大道:「你好猜。」悄悄向伯爵耳邊,如此這般。說未數句,伯爵道:「悄悄兒說,哥正不知道哩。」不一時,西門慶走的腳步兒響,兩個就不言語了。這應伯爵就把吳銀兒摟在懷裡,和他一遞一口兒吃酒,說道:「是我這乾女兒又溫柔,又軟款,強如李家狗不要的小淫婦兒一百倍了。」吳銀兒笑道:「二爹好罵。說一個就一個,百個就百個,一般一方之地也有賢有愚,可可兒一個就比一個來?俺桂姐沒惱著你老人家!」西門慶道: 「你問賊狗才,單管只六說白道的!」伯爵道:「你休管他,等我守著我這乾女兒過日子。乾女兒過來,拿琵琶且先唱個兒我聽。」這吳銀兒不忙不慌,輕舒玉指,款跨鮫綃,把琵琶橫於膝上,低低唱了一回《柳搖金》。伯爵吃過酒,又遞謝希大,吳銀兒又唱了一套。這裡吳銀兒遞酒彈唱不題。

且說畫童兒走到後邊,月娘正和孟玉樓、李瓶兒、大姐、雪娥並大師父,都在上房裡坐的,只見畫童兒進來。月娘才待使他叫老馮來,領夏花兒出去,畫童便道: 「爹使小的對大娘說,教且不要領他出去罷了。」月娘道:「你爹教賣他,怎的又不賣他了?你實說,是誰對你爹說,教休要領他出去?」畫童兒道:「剛纔小的抱著桂姨氈包,桂姨臨去對爹說,央及留下了將就使罷。爹使玳安進來對娘說,玳安不進來,使小的進來,他就奪過氈包送桂姨去了。」這月娘聽了,就有幾分惱在心中,罵玳安道:「恁賊兩頭獻勤欺主的奴才,嗔道頭裡使他叫媒人,他就說道爹叫領出去,原來都是他弄鬼。如今又乾辦著送他去了,住回等他進後來,和他答話。」正說著,只見吳銀兒前邊唱了進來。月娘對他說:「你家蠟梅接你來了。李家桂兒家去了,你莫不也要家去了罷?」吳銀兒道:「娘既留我,我又家去,顯的不識敬重了。」因問蠟梅:「你來做甚麼?」蠟梅道:「媽使我來瞧瞧你。」吳銀兒問道:「家裡沒甚勾當?」蠟梅道:「沒甚事。」吳銀兒道:「既沒事,你來接我怎的?你家去罷。娘留下我,晚夕還同眾娘們往妗奶奶家走百病兒去。我那裡回來,才往家去哩。」說畢,蠟梅就要走。月娘道:「你叫他回來,打發他吃些甚麼兒。」吳銀兒道:「你大奶奶賞你東西吃哩。等著就把衣裳包了帶了家去,對媽媽說,休教轎子來,晚夕我走了家去。」因問:「吳惠怎的不來?」蠟梅道:「他在家裡害眼哩。」月娘吩咐玉簫領蠟梅到後邊,拿下兩碗肉,一盤子饅頭,一甌子酒,打發他吃。又拿他原來的盒子,裝了一盒元宵、一盒細茶食,回與他拿去。

原來吳銀兒的衣裳包兒放在李瓶兒房裡,李瓶兒早尋下一套上色織金緞子衣服、兩方銷金汗巾兒、一兩銀子,安放在他氈包內與他。那吳銀兒喜孜孜辭道:「娘,我不要這衣服罷。」又笑嘻嘻道:「實和娘說,我沒個白襖兒穿,娘收了這緞子衣服,不拘娘的甚麼舊白綾襖兒,與我一件兒穿罷。」李瓶兒道:「我的白襖兒寬大,你怎的穿?」叫迎春:「拿鑰匙,大櫥櫃里拿一匹整白綾來與銀姐。」「對你媽說,教裁縫替你裁兩件好襖兒。」因問:「你要花的,要素的?」吳銀兒道:「娘,我要素的罷,圖襯著比甲兒好穿。」笑嘻嘻向迎春說道:「又起動姐往樓上走一遭,明日我沒甚麼孝順,只是唱曲兒與姐姐聽罷了。」

須臾,迎春從樓上取了一匹松江闊機尖素白綾,下號兒寫著「重三十八兩」,遞與吳銀兒。銀兒連忙與李瓶兒磕了四個頭,起來又深深拜了迎春八拜。李瓶兒道: 「銀姐,你把這緞子衣服還包了去,早晚做酒衣兒穿。」吳銀兒道:「娘賞了白綾做襖兒,怎好又包了這衣服去?」於是又磕頭謝了。

不一時,蠟梅吃了東西,交與他都拿回家去了。月娘便說:「銀姐,你這等我才喜歡。休學李桂兒那等喬張致,昨日和今早,只象臥不住虎子一般,留不住的,只要家去。可可兒家裡就忙的恁樣兒?連唱也不用心唱了。見他家人來接,飯也不吃就去了。銀姐,你快休學他。」吳銀兒道:「好娘,這裡一個爹娘宅里,是那個去處?就有虛篢放著別處使,敢在這裡使?桂姐年幼,他不知事,俺娘休要惱他。」正說著,只見吳大妗子家使了小廝來定兒來請,說道:「俺娘上覆三姑娘,好歹同眾位娘並桂姐、銀姐,請早些過去罷。又請雪姑娘也走走。」月娘道:「你到家對你娘說,俺們如今便收拾去。二娘害腿疼不去,他在家看家了。你姑夫今日前邊有人吃酒,家裡沒人,後邊姐也不去。李桂姐家去了。連大姐、銀姐和我們六位去。你家少費心整治甚麼,俺們坐一回,晚上就來。」因問來定兒:「你家叫了誰在那裡唱?」來定兒道:「是鬱大姐。」說畢,來定兒先去了。月娘一面同玉樓、金蓮、李瓶兒、大姐並吳銀兒,對西門慶說了,吩咐奶子在家看哥兒,都穿戴收拾,共六頂轎子起身。派定玳安兒、棋童兒、來安兒三個小廝,四個排軍跟轎,往吳大妗子家來。正是:

萬井風光春落落,千門燈火夜沉沉。

