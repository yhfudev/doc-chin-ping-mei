%# -*- coding: utf-8 -*-
%!TEX encoding = UTF-8 Unicode
%!TEX TS-program = xelatex
% vim:ts=4:sw=4
%
% 以上设定默认使用 XeLaTex 编译,并指定 Unicode 编码,供 TeXShop 自动识别



\chapter*{金瓶梅序}
\addcontentsline{toc}{chapter}{金瓶梅序 -- 東吴弄珠客}


\begin{showcontents}{}


金瓶梅,穢書也。袁石公亟稱之,
亦自寄其牢騷耳,非有取於金瓶梅也。
然作者亦自有意,蓋為世戒,非為世勸也。
如諸婦多矣,而獨以潘金蓮,李瓶兒,春梅命名者,亦楚「檮杌」之意也。
蓋金蓮以姦死,瓶兒以孽死,春梅以淫死,較諸婦為更慘耳。
借(藉)西門慶以描畫世之大淨,應伯爵以描畫世之小丑(醜),諸淫婦以描畫世之丑(醜)婆淨婆,
令人讀之汗下。蓋為世戒,非為世勸也。

余嘗曰:讀金瓶梅而生憐憫心者,菩薩也;
生畏懼心者,君子也;
生歡喜心者,小人也;
生效(傚)法心者,乃禽獸耳。
余友人褚孝秀偕一少年同赴歌舞之筵,衍至「霸王夜宴」,
少年垂涎曰:「男兒何可不如此!」
褚孝秀曰:「也只為這烏江設此一着(著)耳。」
同座聞之,歎為有道之言。
若有人識得此意,方許他讀金瓶梅也。
不然,石公幾為導淫宣慾之尤矣!
奉勸世人,勿為西門慶之後車,可也。

%{\bigskip\mbox{}\fzqiti\large\hfill 東吳弄珠客題 \quad}
\tiJPM{東吳弄珠客題}
\footnote{金瓶梅词话的署名是:萬曆丁巳季冬東吴弄珠客漫書扵金閶道中}


%\piLiyuF{ % 李渔新刻绣像批评本
\chapter*{跋}
\addcontentsline{toc}{chapter}{金瓶梅跋 -- (明)谢肇淛}

%《金瓶梅》一书,不著作者名代。相传永陵中有金吾戚里,凭怙奢汰,淫纵无度,而其门客病之,采摭日逐行事,汇以成编,而托之西门庆也。书凡数百万言,为卷二十,始末不过数年事耳。
%其中朝野之政务,官私之晋接,闺闼之媟
%\zhuMINE{xie 亵狎的意思}
%语,市里之猥谈,与夫势交利合之态,心输背笑之局,桑中濮
%上之期,尊罍枕席之语,驵验{马会}之机械意智,粉黛之自媚争妍,狎客之从臾逢迎,奴怡之嵇唇淬{言卒}语,穷极境象,駥意快心。譬之范工抟泥,妍媸老少,人鬼万殊,不徒肖其貌,且并其神传之。信稗官之上乘,炉锤之妙手也。其不及《水浒传》者,以其猥琐淫蝶,无关名理。而或以为过之者,彼犹机轴相放,而此之面目各别,聚有自来,散有自去,读者竟想不到,唯恐易尽。此岂可与褒儒俗士见哉。此书向无镂版,抄写流传,参差散失。唯弇州家藏者最为完好。余于袁中郎得其十三,于丘诸城得其十五,稍为厘正,而阙所未备,以俟他日。有嗤余诲淫者,余不敢知。然溱洧之音,圣人不删,则亦中郎帐中必不可无之物也。仿此者有《玉娇丽》,然而乖彝败度,君子无取焉。

《金瓶梅》一書,不著作者名代。相傳永陵
中有金吾戚里,憑怙奢汰,淫縱無度,
而其門客病之,採摭日逐行事,匯以成編,而托之西門慶也。書凡數百萬言,為
卷二十,始末不過數年事耳。其中朝野之政務,官私之晉接,閨闥之媟\zhuMINE{xie 亵狎的意思}語,市里
之猥談,與夫勢交利合之態,心輸背笑之局,桑中濮上之期,尊罍枕席之語,駔
驓之機械意智,粉黛之自媚爭妍,狎客之從諛逢迎,奴佁之稽唇淬語,窮極境象,
駴意快心。譬之範工搏泥,妍媸老少,人鬼萬殊,不徒肖其貌,且并其神傳之。
信稗官之上乘,爐錘之妙手也。其不及《水滸傳》者,以其猥瑣淫媟,無關名理。
而或以為過之者,彼猶機軸相放,而此之面目各別,聚有自來,散有自去,讀者
意想不到,唯恐易盡。此豈可與褒儒俗士見哉。此書向無鏤版,鈔寫流傳,參差
散失。唯弇州家藏者最為完好。余于袁中郎得其十三,于丘諸城得其十五,稍為
釐正,而闕所未備,以俟他日。有嗤余誨淫者,余不敢知。然溱洧之音,聖人不
刪,則亦中郎帳中必不可無之物也。仿此者有《玉嬌麗》,然而乖彝敗度,君子
無取焉。



\tiJPM{(明)谢肇淛}
%\footnote{《金瓶梅资料汇编》190页,南开大学出版社1985年版。}
\footnote{〔明〕謝肇淛:〈金瓶梅跋〉(謝肇淛:《小草齋文集》,卷二十四),見黃霖編:《金瓶梅資料彙編》(北京:中華書局,1987),頁3-4。}
%} % 李渔新刻绣像批评本





\piZhangF{ % 张竹坡皋鹤堂批评第一奇书批评本
\chapter*{第一奇书序}
\addcontentsline{toc}{chapter}{第一奇书序 -- 张竹坡}

《金瓶》一书,传为凤洲门人之作也,或云即风洲手。然丽丽洋洋一百回内,其细针密线,每令观者望洋而叹。今经张子竹坡一批,不特照出作者金针之细,兼使其粉腻香浓,皆如狐穷秦镜,怪窘温犀、无不洞鉴原形,的是浑《艳异》旧手而出之者,信乎为凤洲作无疑也。然后知《艳异》亦淫,以其异而不显其艳;《金瓶》亦艳,以其不异则止觉其淫。故悬鉴燃犀,遂使雪月风花,瓶罄篦梳,陈茎落叶诸精灵等物,妆娇逞态,以欺世于数百年间,一旦潜形无地,蜂蝶留名,杏梅争色,竹坡其碧眼胡乎!向弄珠客教人生怜悯畏惧心,今后看官睹西门庆等各色幻物,弄影行间,能不怜悯,能不畏惧乎?其视金莲当作敝履观矣。不特作者解颐而谢觉,今天下失一《金瓶梅》,添一《艳异编》,岂不大奇!
时康熙岁次乙亥清明中浣,秦中觉天者谢颐题于皋鹤堂。

} % 张竹坡皋鹤堂批评第一奇书批评本

\piZhangF{ % 张竹坡皋鹤堂批评第一奇书批评本
\chapter*{第一奇书凡例}
\addcontentsline{toc}{chapter}{第一奇书凡例 -- 张竹坡}

一、此书非有意刊行,偶因一时文兴,借此一试目力,且成于十数天内,又非十年精思,故内中其大段结束精意,悉照作者。至于琐碎处,未暇请教当世,幸暂量之。

一、《水浒传》圣叹批,大抵皆腹中小批居多。予书刊数十回后,或以此为言。予笑曰:《水浒》是现成大段毕具的文字,如一百八人,各有一传,虽有穿插,实次第分明,故圣叹只批其字句也。若《金瓶》,乃隐大段精采于琐碎之中,只分别字名,细心者皆可为,而反失其大段精采也。然我后数十回内,亦随手补入小枇,是故欲知文字纲领者看上半部,欲随目成趣知文字细密者看下半部,亦何不可!

一、此书卷数浩繁,偶尔批成,适有工便,随刊呈世。其内或圈点不齐,或一二讹字,目力不到者,尚容细政,祈读时量之。

一、《金瓶》行世已久,予喜其文之整密,偶为当世同笔墨者闲中解颐。作《金瓶梅》者,或有所指,予则并无寓讽。设有此心,天地君亲其共恹之。

} % 张竹坡皋鹤堂批评第一奇书批评本

\piZhangF{ % 张竹坡皋鹤堂批评第一奇书批评本

\chapter*{杂录小引}
\addcontentsline{toc}{chapter}{杂录小引 -- 张竹坡}

凡看一书,必看其立架处,如《金瓶梅》内,房屋花园以及使用人等,皆其立架处也。何则?既要写他六房妻小,不得不派他六房居住。然全分开既难使诸人连合,全合拢又难使各人的事实入来,且何以见西门豪富。看他妙在将月、楼写在一处,娇儿在隐现之间。后文说挪厢房与大姐住,前又说大妗子见西门庆揭帘子进来,慌的往娇儿那边跑不迭,然则娇儿虽居厢房,却又紧连上房东间,或有门可通者也。雪娥在后院,近厨房。特特将金、瓶,梅三人,放在前边花园内,见得三人虽为侍妾,却似外室,名分不正,赘居其家,反不若李娇儿以娼家聚来,犹为名正言顺。则杀夫夺妻之事,断断非千金买妾之目。而金梅合,又分出瓶儿为一院,分者理势必然,必紧邻一墙者,为妒宠相争地步。而大姐住前厢,花园在仪门外,又为敬济偷情地步。见得西门庆一味自满托大,意谓惟我可以调弄人家妇女,谁敢狎我家春色,全不想这样妖淫之物,乃令其居于二门之外,墙头红杏,关且关不住,而况于不关也哉卜金莲固是冶容诲淫,而西门庆实是慢藏诲盗,然则固不必罪陈敬济也。故云写其房屋,是其间架处,犹欲耍狮子,先立一场;而唱戏先设一台。恐看官混混看过,故为之明白开出÷使看官如身入其中,然后好看书内有名人数进进出出,穿穿走走,做这些故事也。他如西门庆的家人妇女,皆书内听用者,亦录出之,令看者先已了了,俟后遇某人做某事,分外眼醒。而西门庆淫过妇人名数,开之足令看者伤心惨目,为之不忍也。若夫金莲,不异夏姬,故于其淫过者,亦录出之,令人知惧。

西门庆家人名数:来保(子僧保儿、小舅子刘仓)、来旺、玳安、来兴、平安、来安、书童、画童、琴
童、又琴童(天福儿改者)、棋童、来友、王显、春鸿、春燕、王经(系家丁)、来昭(暨铁棍儿)。后生(荣海)、司茶(郑纪)、烧火(刘包)、小郎(胡秀)、外甥小郎(崔本)、看坟(张安)。


西门庆家人媳妇:来旺媳妇(二,其一则宋蕙莲)、来昭媳妇(一丈青)、来保媳妇(惠祥)、来爵媳妇(惠元)、来兴媳妇(惠秀)。丫环:玉箫、小玉、兰香、小鸾、夏花、元霄儿、迎春、绣春、春梅、秋菊、中秋儿、翠儿。奶子:如意儿。

西门庆淫过妇女:李娇儿、卓丢儿、孟玉楼、潘金莲、李瓶儿、孙雪娥、春梅、迎春、绣春、兰香、宋蕙莲、来爵媳妇(惠元)、王六儿、责四嫂、如意儿、林太太、李桂姐、吴银儿、郑月儿。

意中人:何千户娘子(蓝氏)、王三官娘子(黄氏)、锦云。外宠:书童、王经、潘金莲、王六儿。

潘金莲淫过人目

张大户、西门庆、琴童、陈敬济、王潮儿。意中人:武二郎。外宠:西门庆。恶姻缘:武植。

藏春芙蓉镜:郓哥口、和尚耳,春梅秋波、猫儿眼中,铁棍舌畔、秋菊梦内。

附对:潘金莲品的箫,西门庆投的壶。

西门庆房屋

门面五间,到底七进(后要隔壁子虚房,共作花园)。

上房(月娘住)、西厢房(李娇儿住)、堂屋后三间(孙雪娥住)。

后院厨房、前院穿堂、大客屋、东厢房(大姐住)、西厢房。

仪门(仪门外,则花园也)。三间楼一院(潘金莲住)、又三间楼一院(李瓶儿住)。二人住楼在花园前,过花园方是后边。

花园门在仪门外,后又有角门,通看月娘后边也。金莲、瓶儿两院两角门,前又有一门,即花园门也。花园内,后有卷棚,翡翠轩,前有山子,山顶上卧云亭,半中间藏春坞雪洞也。花园外,即印子铺门面也。门面旁,开大门也。对门,乃要的乔亲家房子也。狮子街乃子虚迁去住者,瓶儿带来,后开绒线铺,又狮子街即打李外传处也。内仪门外,两道旁,乃群房,宋蕙莲等住者也。

} % 张竹坡皋鹤堂批评第一奇书批评本

\piZhangF{ % 张竹坡皋鹤堂批评第一奇书批评本

\chapter*{第一奇书目录}
\addcontentsline{toc}{chapter}{第一奇书目录 -- 张竹坡}


一回 热结 冷遇(悌字起)

二回 勾情 说技 三回  受贿 私挑

四回 幽欢 义愤 五回  捉奸 饮鸩

六回 瞒天 遇雨 七回  说媒 气骂

八回 占卦 烧灵 九回  偷娶 误打

十回 充配 玩赏(金瓶梅三字至此全起)

十一回 激打 梳笼 十二回 私仆 魇胜

十三回 密约 私窥 十四回 种孽 迎奸

十五回 赏灯 帮闲 十六回 择吉 追欢

十七回 弹奸 许嫁 十八回 脱祸 消魂

十九回 逻打 情感 二十回 趋奉 争风

二十一回 扫雪 替花(金瓶梅三人至此畅聚)

二十二回 偷期 正色 二十三回 输钞 潜踪

二十四回 戏娇 怒置 二十五回 秋千 醉谤

二十六回 递解 含羞 二十七回 私语 醉闹

二十八回 侥幸 糊涂

二十九回 冰鉴 兰汤(全部结果)

三十回覃恩 双喜 三十一回 构衅 为欢

三十二回 认女 惊儿 三十三回 罚唱 争风

三十四回 乞恩 说事 三十五回 报仇 媚客

三十六回 寄书 留饮 三十七回 说媒 包占

三十八回 棒槌 琵琶 三十九回 寄名 拜寿

四十回希宠 市爱 四十一回 联姻 同愤

四十二回 烟火 花灯 四十三回 争宠 卖富

四十四回 偷金 消夜 四十五回 劝当 解衣

四十六回 走雨 卜龟(两番结果)

四十七回 害主 枉法 四十八回 私情 捷径

四十九回 屈体 现身 五十回偷觑 嬉游

五十一回 品玉 输金 五十二回 山洞 花园

五十三回 惊欢 求子 五十四回 戏钏 诊瓶

五十五回 两庆 一诺 五十六回 助友 傲妻

五十七回 千金 一笑

五十八回 打狗 磨镜(孝子着书之意在此,教人以孝之意亦在此,此回以一个“孝”字照应一百回孝哥的“孝”字。)

五十九回 露阳 睹物 六十回死孽 生涯

六十一回 醉烧 病宴 六十二回 法遣 大哭

六十三回 传真 观戏 六十四回 三章 一帆

六十五回 同穴 守灵 六十六回 致赙 荐亡

六十七回 赏雪 入梦 六十八回 戏衔 密访

六十九回 初调 惊走 七十回朝房 庭参

七十一回 再梦 引奏 七十二回 抠打 义拜

七十三回 吹箫 试带 七十四回 偎玉 谈经

七十五回 含酸 撒泼(是作者一腔愤恨无可发泄处)

七十六回 娇撒 哭躲 七十七回 雪访 水战

七十八回 再战 独尝 七十九回 丧命 生儿

八十回售色 盗财 八十一回 拐财 欺主

八十二回 得双 冷面 八十三回 含恨 寄简

八十四回 碧霞 雪洞 八十五回 知情 惜泪

八十六回 唆打 解渴 八十七回 忘祸 祭兄

八十八回 感旧 埋尸 八十九回 寡妇 夫人

九十回盗拐 受辱 九十一回 爱嫁 怒打

九十二回 被陷 大闹 九十三回 义恤 娈淫

九十四回 酒楼 娼家 九十五回 窃玉 负心

九十六回 游旧 当面

九十七回 假续 真偕(一部真假总结,照转冷热二字)

九十八回 旧识 情遇 九十九回 醉骂 窃听

一百回路遇 幻化(孝字结)

} % 张竹坡皋鹤堂批评第一奇书批评本

\piZhangF{ % 张竹坡皋鹤堂批评第一奇书批评本

\chapter*{竹坡闲话}
\addcontentsline{toc}{chapter}{竹坡闲话 -- 张竹坡}

《金瓶梅》,何为而有此书也哉?曰:此仁人志士、孝子悌弟不得于时,上不能问诸天,下不能告诸人,悲愤鸣邑,而作秽言以泄其愤也。虽然,上既不可问诸天,下亦不能告诸人,虽作秽言以丑其仇,而吾所谓悲愤鸣邑者,未尝便慊然于心,解颐而自快也。夫终不能一畅吾志,是其言愈毒,而心愈悲,所谓“含酸抱阮”,以此固知玉楼一人,作者之自喻也。然其言既不能以泄吾愤,而终于“含酸抱阮”,作者何以又必有言哉?曰:作者固仁人也,志土也,孝子悌弟也。欲无言,而吾亲之仇也吾何如以处之?欲无言,而又吾兄之仇也吾何如以处之?且也为仇于吾天下万世也,吾又何如以公论之?是吾既不能上告天子以申其隐,又不能下告士师以求其平,且不能得急切应手之荆、聂以济乃事,则吾将止于无可如何而已哉!止于无可如何而已,亦大伤仁人志土、孝子悌弟之心矣。展转以思,惟此不律可以少泄吾愤,是用借西门氏以发之。虽然,我何以知作者必仁人志士、孝子悌弟哉?我见作者之以孝哥结也。“磨镜”一回,皆《蓼莪》遗
意,啾啾之声刺人心窝,此其所以为孝子也。至其以十兄弟对峙一亲哥哥,未复以二捣鬼为缓急相需之人,甚矣,《杀狗记》无此亲切也。

闲尝论之:天下最真者,莫若伦常;最假者,莫若财色。然而伦常之中,如君臣、朋友、夫妇,可合而成;若夫父子、兄弟,如水同源,如木同本,流分枝引,莫不天成。乃竟有假父、假子、假兄、假弟之辈。噫!此而可假,孰不可假?将富贵,而假者可真;贫贱,而真者亦假。富贵,热也,热则无不真;贫贱,冷也,冷则无不假。不谓“冷热”二字,颠倒真假一至于此!然而冷热亦无定矣。今日冷而明日热,则今日真者假,而明日假者真矣。今日热而明日冷,则今日之真者,悉为明日之假者矣。悲夫!本以嗜欲故,遂迷财色,因财色故,遂成冷热,因冷热故,遂乱真假。因彼之假者,欲肆其趋承,使我之真者皆遭其荼毒。所以此书独罪财色也。嗟嗟!假者一人死而百人来,真者一或伤而百难赎。世即有假聚为乐者,亦何必生死人之真骨肉以为乐也哉!

作者不幸,身遭其难,吐之不能,吞之不可,搔抓不得,悲号无益,借此以白泄。其志可悲,其心可悯矣。故其开卷,即以“冷热”为言,煞末又以“真假”为言。其中假父子矣,无何而有假母女;假兄弟矣,无何而有假弟妹;假夫妻矣,无何而有假外室;假亲戚矣,无何而有假孝子。满前役役营营,无非于假景中提傀儡。噫!识真假,则可任其冷热;守其真,则可乐吾孝悌。然而吾之亲父子已荼毒矣,则奈何?吾之亲手足已飘零矣,则奈何?上误吾之君,下辱吾之友,且殃及吾之同类,则奈何?是使吾欲孝,而已为不孝之人;欲弟,而已为不悌之人;欲忠欲信,而已放逐谗间于吾君、吾友之则。日夜咄咄,仰天太息,吾何辜而遭此也哉?曰:以彼之以假相聚故也。噫嘻!彼亦知彼之所以为假者,亦冷热中事乎?假子之子于假父也,以热故也。假弟、假女、假友,皆以热故也。彼热者,盖亦不知浮云之有聚散也。未几而冰山颓矣,未几而阀阅朽矣。当世驱己之假以残人之真者,不瞬息而己之真者亦飘泊无依。所为假者安在哉?彼于此时,应悔向日为假所误。然而人之真者,已黄土百年。彼留假傀儡,人则有真怨恨。怨恨深而不能吐,日酿一日,苍苍高天,茫茫碧海,吾何日而能忘也哉!眼泪洗面,椎心泣血,即百割此仇,何益于事!是此等酸法,一时一刻,酿成千百万年,死而有知,皆不能坏。此所以玉楼弹阮来,爱姐抱阮去,千秋万岁,此恨绵绵无绝期矣。故用普净以解冤偈结之。夫冤至于不可解之时,转而求其解,则此一刻之酸,当何如含耶?是愤已百二十分,酸又百二十分,不作《金瓶梅》,又何以消遣哉?甚矣!仁人志士、孝子悌弟,上不能告诸天,下不能告诸人,悲愤呜邑,而作秽言,以泄其愤。自云含酸,不是撒泼,怀匕囊锤,以报其人;是亦一举。乃作者固自有志,耻作荆、聂,寓复仇之义于百回微言之中,谁为刀笔之利不杀人于千古哉!此所以有《金瓶梅》也。


然则《金瓶梅》,我又何以批之也哉?我喜其文之洋洋一百回,而千针万线,同出一丝,又千曲万折,不露一线。闲窗独坐,读史、读诸家文,少暇,偶一观之曰:如此妙文,不为之递出金针,不几辜负作者千秋苦心哉!久之心恒怯焉,不敢遽操管以从事。盖其书之细如牛毛,乃千万根共具一体,血脉贯通,藏针伏线,千里相牵,少有所见,不禁望洋而退。迩来为穷愁所迫,炎凉所激,于难消遣时,恨不自撰一部世情书,以排遗闷怀。几欲下笔,而前后拮构,甚费经营,乃搁笔曰:“我且将他人炎凉之书,其所以前后经营者,细细算出,一者可以消我闷怀,二者算出古人之书,亦可算我今又经营一书。我虽未有所作,而我所以持往作书之法,不尽备于是乎!然则我自做我之《金瓶梅》,我何暇与人批《金瓶梅》也哉!

} % 张竹坡皋鹤堂批评第一奇书批评本

\piZhangF{ % 张竹坡皋鹤堂批评第一奇书批评本

\chapter*{冷热金针}
\addcontentsline{toc}{chapter}{冷热金针 -- 张竹坡}

《金瓶》以“冷热”二字开讲,抑熟不知此二字为一部之金钥乎?然于其点睛处,则未之知也。夫点睛处安在?曰:在温秀才、韩伙计。何则?韩者冷之别名,温者热之余气。故韩伙计于“加官”后即来,是热中之冷信。而温秀才自“磨镜”后方出,是冷字之先声。是知祸福倚伏,寒暑盗气,天道有然也。虽然,热与寒为匹,冷与温为匹,盖热者温之极,韩者冷之极也。故韩道国不出于冷局之后,而出热局之先,见热未极而冷已极。温秀才不来于热场之中,而来于冷局之首,见冷欲盛而热将尽也。噫嘻,一部言冷言热,何啻如花如火!而其点睛处乃以此二人,而数百年读者,亦不知其所以作韩、温二人之故。是作书者固难,而看书者为尤难,岂不信哉!

} % 张竹坡皋鹤堂批评第一奇书批评本

\piZhangF{ % 张竹坡皋鹤堂批评第一奇书批评本

\chapter*{寓意说}
\addcontentsline{toc}{chapter}{寓意说 -- 张竹坡}
稗官者,寓言也。其假捏一人,幻造一事,虽为风影之谈,亦必依山点石,借海扬波。故《金瓶》一部,有名人物不下百数,为之寻端竟委,大半皆属寓言。庶因物有名,托名摭事,以成此一百回
曲曲折折之书,如西门庆、潘金莲、王婆、武大、武二,《水浒传》中原有之人,《金瓶》因之者无论。然则何以有瓶、梅哉?瓶因庆生也。盖云贪欲嗜恶,面骸枯尽,瓶之罄矣。特特撰出瓶儿,直令千古风流人同声一哭。因瓶生情,则花瓶而子虚姓花,银瓶而银姐名银。瓶与屏通,窥春必于隙。屏号芙蓉,“玩赏芙蓉亭”盖为瓶儿插笋。而“私窥”一回卷首词内,必云“绣面芙蓉一笑开”。后“玩灯”一回《灯赋》内,荷花灯、芙蓉灯。盖金、瓶合传,是因瓶假屏,又因屏假芙蓉,浸淫以人于幻也。屏、风二字相连,则冯妈妈必随瓶儿,而当大理屏风、又点睛妙笔矣。芙蓉栽以正月,冶艳于中秋,摇落于九月,故瓶儿必生于九月十五,嫁以八月廿五,后病必于重阳,死以十月,总是《芙蓉谱》内时候。墙头物去,亲事杳然,瓶儿悔矣。故蒋文蕙将闻悔而来也者。然瓶儿终非所据,必致逐散,故又号竹山。总是瓶儿心事中生出此一人。如意为瓶儿后身,故为熊氏姓张。熊之所贵者胆也,是如意乃瓶胆一张耳。故瓶儿好倒插花,如意‘茎露独尝’,皆瓶与瓶胆之本色情景。官哥幻其名意,亦皆官窑哥窑,故以雪贼死之。瓶遇猫击,焉能不碎?银瓶坠井,千古伤心。故解衣而瓶儿死,托梦必于何家。银瓶失水矣,竹篮打水,成何益哉?故用何家蓝氏作意中人,以送西门之死,亦瓶之余意也。

至于梅,又因瓶而生。何则?瓶里梅花,春光无几。则瓶罄喻骨髓暗枯,瓶梅又喻衰朽在即。梅雪不相下,故春梅宠而雪娥辱,春梅正位而雪娥愈辱。月为梅花主人,故永福相逢,必云故主。而吴典恩之事,必用春梅襄事。冬梅为奇寒所迫,至春吐气,故“不垂别泪”,乃作者一腔炎凉痛恨发
于笔端。至周、舟同音,春梅归之,为载花舟。秀、臭同音,春梅遗臭载花舟且作粪舟。而周义乃野渡无人,中流荡漾,故永福寺里普净座前必用周义转世,为高留住儿,言须一篙留住,方登彼岸。

然则金莲,岂尽无寓意哉?莲与芰,类也;陈,旧也,败也;敬、茎同音。败茎芰荷,言莲之下场头。故金莲以敬济而败,“侥幸得金莲”,芰茎之罪。西门乃“打铁棍”,铁棍,芰茎影也,舍根而罪影,所谓糊涂。败茎不耐风霜,故至严州,而铁指甲一折即下。幸徐(山封)相救,风少劲即吹去矣。次后过街鼠寻风,是真朔风。风利如刀,刀利如风,残枝败叶,安得不摧哉!其父陈洪,已为露冷莲房坠粉红。其舅张团练搬去,又荷尽已无擎雨盖,留此败茎支持风雪,总写莲之不堪处。益知夏龙溪为金莲胜时写也。温秀才积至水秀才,至倪秀才,再至王潮儿,总言水枯莲谢,惟余数茎败叶潦倒污泥,所为风流不堪回首,无非为金莲污辱下贱写也。莲名金莲,瓶亦名金瓶,侍女偷金,莲、瓶相妒,斗叶输金,莲花飘萎,芸茎用事矣。他如宋蕙莲、王六儿,亦皆为金莲写也。写一金莲,不足以尽金莲之恶,且不足以尽西门、月娘之恶,故先写一宋金莲,再写一王六儿,总与潘金莲一而二,二而三者也。然而蕙莲,荻帘也,望子落,帘儿坠,含羞自缢,又为“叉竿挑帘”一回重作渲染。至王六儿,又黄芦儿别音,其娘家王母猪。黄芦与黄竹相类,其弟王经,亦黄芦茎之义。芦茎叶皆后空,故王六儿好干后庭花,亦随手成趣。芦亦有影,故看灯夜又用铁棍一觑春风,是芦荻皆莲之副,故曰二人皆为金莲写。此一部写金、写瓶、写梅之大梗概也。

若夫月娘为月,遍照诸花。生于中秋,故有桂儿为之女。“扫雪”而月娘喜,“踏雪”而月娘悲,月有阴晴明晦也。且月下吹箫,故用玉箫,月满兔肥,盈已必亏,故小玉成婚,平安即偷镀金钩子,到南瓦子里要。盖月照金钩于南瓦上,其亏可见。后用云里守人梦,月被云遮,小玉随之,与兔俱隐,情文明甚。

李娇儿,乃“桃李春风墙外枝”也。其弟李铭,言里明外暗,可发一笑。至贲四嫂与林太太,乃叶落林空,春光已去。贲四嫂姓叶,作“带水战”。西门庆将至其家,必云吩咐后生王显,是背面落水,显黄一叶也。林太太用文嫂相通,文嫂住捕衙厅前,女名金大姐,乃蜂衙中一黄蜂,所云蜂媒是也。此时爱月初宠,两番赏雪,雪月争寒,空林叶落,所莲花芙蓉,安能宁耐哉!故瓶死莲辱,独让春梅争香吐艳。而春鸿、春燕,又喻韶光迅速,送鸿迎燕,无有停息。来爵改名来友,见花事阑珊,燕莺遗恨。其妻惠元,三友会于园,看杜鹃啼血矣。内有玉箫勾引春风,外有玳安传消递息,箫有合欢之调,熏莲、惠元以之。箫有离别之音,故“三章约”乃阳关声。西门听之,能不动深悲耶?惹草粘花,必用玳安。一曰“嬉游蝴蝶巷”,再日“密访蜂媒”,已明其为蝶使矣,所谓“玳瑁斑花蝴蝶”非欤?书童则因箫而有名。盖篇内写月、写花、写雪,皆定名一人,惟风则止有冯妈妈。太守徐崶,虽亦一人。而非花娇月媚,正经脚色。故用书童与玉箫合,而萧疏之风动矣。未必云“私挂一帆”,可知其用意写风。然又通书为梳,故书童生于苏州府长熟县,字义可思。媚客之唱,必云“画损了掠儿稍”,接手云“贲四害怕”。“梳子在座,篦子害怕”,妙绝!《艳异》遗意,为男宠报仇。金莲必云“打了象牙”,明点牙梳。去必以瓶儿丧内,瓶坠簪折,牙梳零落,萧疏风起,春意阑珊,《阳关三叠》,大家将散场也。《金瓶》之大概寓言如此,其他剩意,不能殚述。推此观之,笔笔皆然。

至其写玉楼一人,则又作者经济学问,色色自喻皆到。试细细言之:玉楼簪上镌“玉楼人醉杏花天”,来自杨家,后嫁李家,遇薛嫂而受屈,遇陶妈妈而吐气,分明为杏无疑。可者,幸也。身毁名污,幸此残躯留于人世。而住居臭水巷。盖言元妄之来,遭此荼毒,污辱难忍,故着书以泄愤。嫁于李衙内,而李贵随之,李安往依之,以理为贵,以理为安。归于真定、枣强。真定,言吾心淡定;枣强,言黾勉工夫。所为勿助勿忘,此是作者学问。王杏庵送贫儿于晏公庙任道土为徒。晏,安也;任与人通,又与仁通。言“我若得志,必以仁道济天下,使天下匹夫匹妇,皆在晏安之内,以养其生;皆入于人伦之中,以复其性。”此作者之经济也。不谓有金道士淫之,又有陈三引之,言为今人声色货利浸淫已久,我方竭力养之教之,而今道又使其旧性复散,不可救援,相率而至于永福寺内,共作孤魂而后已。是可悲哉!夫永福寺,涌于腹下,此何物也?其内僧人,一曰胡僧,再曰道坚,一肖其形,一美其号。永福寺真生我之门死我户,故皆于死后同归于此,见色之利害。而万回长老,其回肠也哉。他如黄龙寺,脾也;相国寺,相火也。拜相国长老,归路避风黄龙,明言相火动而脾风发,故西门死气如牛吼,已先于东京言之矣。是玉皇庙,心也。二重殿后一重侧门,其心尚可问哉?故有吴道士主持结拜,心既无道,结拜何益?所以将玉皇庙始而永福寺结者,以此。

更有因一事而生数人者,则数名公同一义。如车(扯)淡、管世(事)宽、游守(手)、郝(好)贤(闲),四人共一寓意也。又如李智(枝)、黄四,梅、李尽黄,春光已暮,二人共一寓意也。又如‘带水战’一回,前云聂(捏)两湖、尚(上)小塘、汪北彦(沿),三人共一寓意也。又如安沈(枕)、宋(送)乔年,喻色欲伤生,二人共一寓意也。又有因——人而生数名者,应伯(白)爵(嚼)字光侯(喉),谢希(携)大(带)字子(紫)纯(唇),祝(住)实(十)念(年),孙天化(话)字伯(不)修(羞),常峙(时)节(借),卜(不)志(知)道,吴(无)典恩,云里守(手)字非(飞)去,白赖光字光汤,贲(背)第(地)传,傅(负)自新(心),甘(干)出身,韩道(捣)国(鬼)。因西门庆不肖,生出数名也。又有即物为名者,如吴神仙,乃镜也,名无夹,冰鉴照人无失也。黄真人,土也,瓶坠簪折,黄土伤心。末用楚云一人遥影,正是彩云易散。潘道士,撤也,死孽已成,撤着一做也。又有随手调笑,如西门庆父名达,盖明捏土音,言西门之达,即金莲所呼达达之达。设问其母何氏,当必云娘氏矣。桂姐接丁二官,打丁之人也。李(里)外传,取其传话之意。侯林儿,言树倒猢狲散。此皆掉手成趣处。他如张好问、白汝晃(谎)之类,不可枚举。随时会意,皆见作者狡滑之才。

若夫玉楼弹阮,爱姐继其后,抱阮以往湖州何官人家,依二捣鬼以终,是作者穷途有泪无可洒处,乃于爱河中捣此一篇鬼话。明亦无可如何之中,作书以自遣也。至其以孝哥结入一百回,用普净幻化,言惟孝可以消除万恶,惟孝可以永锡尔类,今使我不能全孝,抑曾反思尔之于尔亲,却是如何!千秋万岁,此恨绵绵,悠悠苍天,曷有其极,悲哉,悲哉!
\piWenlong{此批不必然,不必不然。在作者纯任其自然,批者欲求其所以然,遂未免强以为然。我谓有然有不然,不如视为莫知其然而然,斯统归于不期然而然,全付于天然,又何必争其然与不然哉!试起作者九原而问之,亦必哑然而笑,喟然而叹,悄然以悲,夷然不顾曰:其然岂其然乎?}


} % 张竹坡皋鹤堂批评第一奇书批评本

\piZhangF{ % 张竹坡皋鹤堂批评第一奇书批评本

\chapter*{苦孝说}
\addcontentsline{toc}{chapter}{苦孝说 -- 张竹坡}

夫人之有身,吾亲与之也。则吾之身,视亲之身为生死矣。若夫亲之血气衰老,归于大造,孝子有痛于中,是凡为人子者所同,而非一人独具之奇冤也。至于生也不幸,其亲为仇所算,则此时此际,以至千百万年,不忍一注目,不敢一存想,一息有知,一息之痛为无已。呜呼,痛哉!痛之不已,酿成奇酸,海枯石烂,其味深长。是故含此酸者,不敢独立默坐。苟独立默坐,则不知吾之身、吾之心、吾之骨肉,何以栗栗焉如刀斯割、如虫斯噬也。悲夫!天下尚有一境,焉能使斯人悦耳目、娱心志,一安其身也哉?苍苍高天,茫茫厚地,无可一安其身,必死用户庶几矣。然吾闻死而有有知之说,则奇痛尚在,是死亦无益于酸也。然则必何如而可哉?必何如而可,意者生而无我,死而亦无我。夫生而无我,死而亦无我,幻化之谓也。推幻化之谓,既不愿为人,又不愿为鬼,并不愿为水石。盖为水为石,犹必流石人之泪矣。呜呼!苍苍高天,茫茫厚地,何故而有我一人,致令幻化之难也?故作《金瓶梅》者,一曰“含酸”,再曰“抱阮”,结曰“幻化”,且必曰幻化孝哥儿,作者之心,其有余痛乎?则《金瓶梅》当名之曰《奇酸志》、《苦孝说》。呜呼!孝子,孝子,有苦如是!

} % 张竹坡皋鹤堂批评第一奇书批评本

\piZhangF{ % 张竹坡皋鹤堂批评第一奇书批评本

\chapter*{第一奇书非淫书论}
\addcontentsline{toc}{chapter}{第一奇书非淫书论 -- 张竹坡}

诗云“以尔车来,以我贿迁”,此非瓶儿等辈乎?又云“子不我思,岂无他人”,此非金、梅等辈乎??“狂且狡童”,此非西门、敬济等辈乎?乃先师手订,文公细注,岂不曰此淫风也哉!所以云“诗三百,一言以蔽之曰:思无邪。”注云:“诗有善有恶。善者起发人之善心,恶者惩创人之逆志。”圣贤着书立言之意,固昭然于千古也。今夫《金瓶梅》一书作者,亦是将《褰裳》、《风雨》、《箨兮》、《子衿》诸诗细为摹仿耳。夫微言之而文人知儆,显言之而流俗知惧。不意世之看者,不以为惩劝之韦弦,反以为行乐之符节,所以目为淫书,不知淫者自见其为淫耳。但目今旧板,现在金陵印刷,原本四处流行买卖。予小子悯作者之苦心,新同志之耳目,批此一书,其“寓意说”内,将其一部奸夫淫妇,翻批作草木幻影;一部淫词艳语,悉批作起伏奇文。至于以“睇”字起,“孝”字结,一片天命民彝,殷然慨侧,又以玉楼、杏庵照出作者学问经纶,使人一览无复有前此之《金瓶》矣。但恐不学风影等辈,借端恐虎,意在骗诈。夫现今通行发卖,原未禁止;小子穷愁着书,亦书生常事。又非借此沽名,本因家无寸土,欲觅蝇头以养生耳。即云奉行禁止,小子非套翻原板,固我自作我的《金瓶梅》。我的《金瓶梅》上洗淫乱而存孝悌,变帐簿以作文章,直使《金瓶》一书冰消瓦解,则算小子劈《金瓶梅》原板亦何不可!夫邪说当辟,而辟邪说者必就邪说而辟之,其说方息。今我辟邪说而人非之,是非之者必邪说也。若不予先辨明,恐当世君子为其所惑。况小子年始二十有六,素与人全无恩怨,本非借不律以泄愤懑;又非囊有余钱,借梨枣以博虚名:不过为糊口计。兰不当门,不锄何害?锄之何益?是用抒诚,以告仁人君子,共其量之。

} % 张竹坡皋鹤堂批评第一奇书批评本

\piZhangF{ % 张竹坡皋鹤堂批评第一奇书批评本

\chapter*{批评第一奇书《金瓶梅》读法}
\addcontentsline{toc}{chapter}{批评第一奇书《金瓶梅》读法 -- 张竹坡}

劈空撰出金、瓶、梅三个人来,看其如何收拢一块,如何发放开去。看其前半部止做金、瓶,后半部止做春梅。前半人家的金瓶,被他千方百计弄来,后半自己的梅花,却轻轻的被人夺去。(一)

起以玉皇庙,终以水福寺,而一回中已一齐说出,是大关键处。(二)

先是吴神仙总览其盛,后是黄真人少扶其衰,末是普净师一洗其业,是此书大照应处。(三)

“冰鉴定终身”,是一番结束,然独遗陈敬济。“戏笑卜龟儿”,又遗潘金莲。然金莲即从其自己口中补出,是故亦不遗金莲,当独遗西门庆与春梅耳。两番瓶儿托梦,盖又单补西门。而叶头陀相面,才为敬济一番结束也。(四)

未出金莲,先出瓶儿;既娶金莲,方出春梅;未娶金莲,却先娶玉楼;未娶瓶儿,又先出敬济。文字穿插之妙,不可名言。若夫夹写蕙莲、王六儿、贲四嫂、如意儿诸人,又极尽天工之巧矣。(五)

会看《金瓶》者,看下半部。亦惟会看者,单看上半部,如“生子加官”时,唱“韩湘子寻叔”、“叹浮生犹如一梦”等,不可枚举,细玩方知。(六)

《金瓶》有板定大章法。如金莲有事生气,必用玉楼在旁,百遍皆然,一丝不易,是其章法老处。他如西门至人家饮酒,临出门时,必用一人或一官来拜、留坐,此又是“生子加官”后数十回大章法。(七)

《金瓶》一百回,到底俱是两对章法,合其目为二百件事。然有一回前后两事,中用一语过节;又有前后两事,暗中一笋过下。如第一回,用玄坛的虎是也。又有两事两段写者,写了前一事半段,即写后一事半段,再完前半段,再完后半段者。有二事而参伍错综写者,有夹入他事写者。总之,以目中二事为条干,逐回细玩即知。(八)

《金瓶》一回,两事作对固矣,却又有两回作遥对者。如金莲琵琶、瓶儿象棋作一对,偷壶、偷金作一对等,又不可枚举。(九)

前半处处冷,令人不耐看;后半处处热,而人又看不出。前半冷,当在写最热处,玩之即知;后半热,看孟玉楼上坟,放笔描清明春色便知。(十)

内中有最没正经、没要紧的一人,却是最有结果的人,如韩爱姐是也。一部中,诸妇人何可胜数,乃独以爱姐守志结何哉?作者盖有深意存于其意矣。言爱姐之母为娼,而爱姐自东京归,亦曾迎人献笑,乃一留心敬济,之死靡他,以视瓶儿之于子虚,春梅之于守备,二人固当愧死。若金莲之遇

西门,亦可如爱姐之逢敬济,乃一之于琴童,再之于敬济,且下及王潮儿,何其比回心之娼妓亦不若哉?此所以将爱姐作结,以愧诸妇;且言爱姐以娼女回头,还堪守节,奈之何身居金屋而不改过海非,一竟丧廉寡耻,于死路而不返哉?(一一)

读《金瓶》,须看其大间架处。其大间架处,则分金、梅在一起,分瓶儿在一处,又必合金、瓶、梅在前院一处。金、梅合而瓶儿孤,前院近而金、瓶妒,月娘远而敬济得以下手也。(一二)

读《金瓶》,须看其入笋处。如玉皇庙讲笑话,插入打虎;请子虚,即插入后院紧邻;六回金莲才热,即借嘲骂处插入玉楼;借问伯爵连日那里,即插出桂姐;借盖卷棚即插入敬济,借翠管家插人王六儿;借翡
翠轩插入瓶儿生子;借梵僧药,插入瓶儿受病;借碧霞宫插入普净;借上坟插入李衙内;借拿皮袄插入玳安、小玉。诸如此类,不可胜数盖其用笔不露痕迹处也。其所以不露痕迹处,总之善用曲笔、逆笔,不肯另起头绪用直笔、顺笔也。夫此书头绪何限?若一一起之,是必不能之数也。我执笔时,亦必想用曲笔、逆笔,但不能如他曲得无迹、逆得不觉耳。此所以妙也。(一三)

《金瓶》有节节露破绽处。如窗内淫声,和尚偏听见;私琴童,雪娥偏知道;而裙带葫芦,更属险事;墙头密约,金莲偏看见;蕙莲偷期,金莲偏撞着;翡翠轩,自谓打听瓶儿;葡萄架,早已照人铁棍;才受赃,即动大巡之怒;才乞恩,便有平安之才;调婿后,西门偏就摸着;烧阴户,胡秀偏就看见。诸如此类,又不可胜数,总之,用险笔以写人情之可畏,而尤妙在既已露破,乃一语即解,绝不费力累赘。此所以为化笔也。(一四)

《金瓶》有特特起一事、生一人,而来既无端,去亦无谓,如书童是也。不知作者,盖几许经营,而始有书童之一人也。其描写西门淫荡,并及外宠,不必说矣。不知作者盖因一人之出门,而方写此书童也。何以言之?瓶儿与月娘始疏而终亲,金莲与月娘始亲而终疏。虽固因逐来昭、解来旺起衅,而未必至撒泼一番之甚也。夫竟至撒泼一番者,有玉箫不惜将月娘底里之言磬尽告之也。玉箫何以告之?曰有“三章约”在也。“三章”何以肯受?有书童一节故也。夫玉箫、书童不便突起炉灶,故写“藏壶构衅”于前也。然则遥遥写来,必欲其撒泼,何为也哉?必得如此,方于出门时月娘毫无怜惜,一弃不顾,而金莲乃一败涂地也。谁谓《金瓶》内有一无谓之笔墨也哉。(一五)

《金瓶》内正经写六个妇人,而其实止写得四个:月娘,玉楼,金莲,瓶儿是也。然月娘则以大纲故写之;玉楼虽写,则全以高才被屈,满肚牢骚,故又另出一机轴写之,然则以不得不写。写月娘,以不肯一样写;写玉楼,是全非正写也。其正写者,惟瓶儿、金莲。然而写瓶儿,又每以不言写之。夫以不言写之,是以不写处写之。以不写处写之,是其写处单在金莲也。单写金莲,宜乎金莲之恶冠于众人也。吁,文人之笔可惧哉!(一六)

《金瓶》内,有两个人为特特用意写之,其结果亦皆可观。如春梅与玳安儿是也。于同作丫鬟时,必用几遍笔墨描写春梅心高志大,气象不同;于众小厮内,必用层层笔墨,描写玳安色色可人。后文春梅作夫人,玳安作员外。作者必欲其如此何哉?见得一部炎凉书中翻案故也。何则?止知眼前作婢,不知即他日之夫人;止知眼前作仆,不知即他年之员外。不特他人转眼奉承,即月娘且转而以上宾待之,末路倚之。然则人之眼边前炎凉成何益哉!此是作者特特为人下砧砭也。因要他于污泥中为后文翻案,故不得不先为之抬高身分也。(一七)

李娇儿、孙雪娥,要此二人何哉?写一李娇儿,见其来遇金莲、瓶儿时,早已嘲风弄月,迎好卖俏,许多不肖事,种种可杀。是写金莲、瓶儿,乃实写西门之恶;写李娇儿,又虚写西门之恶。写出来的既已如此,其未写出来的时,又不知何许恶端不可问之事于从前也。作者何其深恶西门之如是!至孙雪娥,出身微贱,分不过通房,何其必劳一番笔墨写之哉?此又作者菩萨心也。夫以西门之恶,不写其妻作倡,何以报恶人?然既立意另一花样写月娘,断断不忍写月娘至于此也。玉楼本是无辜受毒,何忍更令其顶缸受报?李娇儿本是娼家,瓶儿更欲用之孽报于西门生前,而金莲更自有冤家债主在,且即使之为娼,于西门何损?于金莲似甚有益,乐此不苦,又何以言报也?故用写雪娥以至于为娼,以总张西门之报,且暗结宋蕙莲一段公案。至于张胜、敬济后事,则又情因文生,随手收拾。不然雪娥为娼,何以结果哉?(一八)

又娇儿色中之财,看其在家管库,临去拐财可见。王六儿财中之色,看其与西门交合时,必云做买卖,骗丫头房子,说合苗青。总是借色起端也。”(一九)

书内必写蕙莲,所以深潘金莲之恶于无尽也,所以为后文妒瓶儿时,小试行道之端也。何则?蕙莲才蒙爱,偏是他先知,亦如迎春唤猫。金莲睃见也。使春梅送火山洞,何异教西门早娶瓶儿,愿权在一块住也。蕙莲跪求,使尔舒心,且许多牢笼关锁,何异瓶儿来时,乘醉说一跳板走的话也。两舌雪娥,使激蕙莲,何异对月娘说瓶儿是非之处也。卒之来旺几死而未死,蕙莲可以不死而竟死,皆金莲为之也。作者特特于瓶儿进门加此一段,所以危瓶儿也。而瓶儿不悟,且亲密之,宜乎其祸不旋踵,后车终覆也。此深着金莲之恶。吾故曰:其小试行道之端,盖作者为不知远害者写一样子。若只随手看去,便说西门庆又刮上一家人媳妇子矣。夫西门庆,杀夫夺妻取其财,庇杀主之奴,卖朝廷之法,岂必于此特特撰此一事以增其罪案哉?然则看官每为作者瞒过了也。(二十)

后又写如意儿,何故哉?又作者明白奈何金莲,见其死蕙莲、死瓶儿之均属无益也。何则?蕙莲才死,金莲可一快。然而官哥生,瓶儿宠矣。及官哥死,瓶儿亦死,金莲又一大快。然而如意口脂,又从灵座生香,去掉一个,又来一个。金莲虽善固宠,巧于制人,于此能不技穷袖手,其奈之何?故作者写如意儿,全为金莲写,亦全为蕙莲、瓶儿愤也。(二—)

然则写桂姐、银儿、月儿诸妓,何哉?此则总写西门无厌,又见其为浮薄立品,市井为习。而于中写桂姐,特犯金莲;写银姐,特犯瓶儿;又见金、瓶二人,其气味声息,已全通娼家。虽未身为倚门之人,而淫心乱行,实臭味相投,彼娼妇犹步后尘矣。其写月儿,则另用香温玉软之笔,见西门一味粗鄙,虽章台春色,犹不能细心领略,故写月儿,又反衬西门也。(二二)

写王六儿、贲四嫂以及林太太何哉?曰:王六儿、贲四嫂、林太太三人是三样写法,三种意思。写王六儿干,专为财能致色一着做出来。你看西门在日,王六儿何等趋承,乃一旦拐财远遁。故知西门于六儿,借财图色,而王六儿亦借色求财。故西门死,必自王六儿家来,究竟财色两空。王六儿遇何官人,究竟借色求财。甚矣!色可以动人,尤未如财之通行无阻,人人皆爱也。然则写六儿,又似童讲财,故竟结入一百回内。至于贲四嫂,却为玳安写。盖言西门止知贪滥无厌,不知其左右亲随且上行下效,已浸淫乎欺主之风,而“窃玉成婚”,已伏线于此矣。若云陪写王六儿,犹是浅着。再至林太太,吾不知作者之心,有何干万愤懑,而于潘金莲发之。不但杀之割之,而并其出身之处、教习之人,皆欲致之死地而方畅也。何则?王招宣府内,故金莲旧时卖入学歌学舞之处也。今看其一腔机诈,丧廉寡耻,若云本自天生,则良心为不可必,而性善为不可据也。吾知其自二、三岁时,未必便如此淫荡也。使当日王招宣家男敦礼义,女尚贞廉,淫声不出于口,淫色不见于目,金莲虽淫荡,亦必化而为贞女。奈何堂堂招宣,不为天子招服远人,宣扬威德,而一裁缝家九岁女孩至其家,即费许多闲情,教其描眉画眼,弄粉涂朱,且教其做张做致,乔模乔样。其待小使女如此,则其仪型妻子可知矣。宜乎三官之不肖荒淫,林氏之荡闲(足俞)矩也。招宣实教之,夫复何尤!然则招宣教一金莲,以遗害无穷:身受其害者,前有武大,后有西门,而林氏为招宣还报,固其宜也。吾故曰:作者盖深恶金莲,而并恶及其出身之处,故写林太太也。然则张大户亦成金莲之恶者,何以不写?曰:张二官顶补西门千户之缺,而伯爵走动说娶娇儿,俨然又一西门,其受报亦必又有不可尽言者。则其不着笔墨处,又有无限烟波,直欲又藏一部大书于无笔处也。此所谓笔不到而意到者。(二三)

《金瓶》写月娘,人人谓西门氏亏此一人内助。不知作者写月娘之罪,纯以隐笔,而人不知也。何则?良人者,妻之所仰望而终身者也。若其夫千金买妾为宗嗣计,而月娘百依百顺,此诚《关雎》之雅,千古贤妇人也。若西门庆杀人之夫,劫人之妻,此真盗贼之行也。其夫为盗贼之行,而其妻不涕泣而告之,乃依违其间,视为路人,休戚不相关,而且自以好好先生为贤,其为心尚可问哉!至其于陈敬济,则作者已大书特书,月娘引贼人室之罪可胜言哉!至后识破奸情,不知所为分处之计,乃白口关门,便为处此已毕。后之逐敬济,送大姐,请春梅,皆随风弄柁,毫无成见;而听尼宣卷,胡乱烧香,全非妇女所宜。而后知“不甚读书”四字,误尽西门一生,且误尽月娘一生也。何则?使西门守礼,便能以礼刑其妻;今止为西门不读书,所以月娘虽有为善之资,而亦流于不知大礼,即其家常举动,全无举案之风,而徒多眉眼之处。盖写月娘,为一知学好而不知礼之妇人也。夫知学好矣,而不知礼,犹足遗害无穷,使敬济之恶归罪于己,况不学好者乎!然则敬济之罪,月娘成之,月娘之罪,西门庆刑于之过也。(二四)

文章有加一倍写法,此书则善于加倍写也。如写西门之执,更写蔡、宋二御史,更写六黄太尉,更写蔡太师,更写朝房,此加一倍热也。如写西门之冷,则更写一敬济在冷铺中,更写蔡太师充军,更写徽、钦北狩,真是加一倍冷。要之加一倍热,更欲写如西门之热者何限,而西门独倚财肆恶;加一倍冷者,正欲写如西门之冷者何穷,而西门乃不早见机也。(二五)

写月娘,必写其好佛者,人抑知作者之意乎?作者开讲,早已劝人六根清净,吾知其必以“空”结此“财色”二字也。安“空”字作结,必为僧乃可。夫西门不死,必不回头,而西门既死,又谁为僧?使月娘于西门一死,不顾家业,即削发入山,亦何与于西门说法?今必仍令西门自己受持方可。夫西门已死则奈何?作者几许踟蹰,乃以孝哥儿生于西门死之一刻,卒欲令其回头,受我度脱。总以圣贤心发菩萨愿,欲天下无终讳过之人,人无不改之过也。夫人之既死,犹望其改过于来生,然则作者之待西门何其忠厚慨恻,而劝勉于天下后世之人,何其殷殷不已也。是故既有此段大结束在胸中,若突然于后文生出一普净师幻化了去,无头无绪,一者落寻常窠臼,二者笔墨则脱落痕迹矣。故必先写月娘好佛,一路尸尸闪闪,如草蛇灰线。后又特笔
出碧霞宫,方转到雪涧,而又只一影普师,迟至十年,方才复收到永福寺。且于幻影中,将一部中有名人物,花开豆爆出来的,复一一烟消火灭了去。盖生离死别,各人传中皆自有结,此方是一总大结束。作者直欲使一部千针万线,又尽幻化了还之于太虚也。然则写月娘好佛,岂泛泛然为吃斋村妇闲写家常哉?此部书总妙在千里伏脉,不肯作易安之笔,没笋之物也是故妙绝群书。(二六)

又月娘好佛,内便隐三个姑子,许多隐谋诡计,教唆他烧夜香,吃药安胎,无所不为。则写好佛,又写月娘之隐恶也,不可不知。(二七)

内中独写玉楼有结果,何也?盖劝瓶儿、金莲二妇也。言不幸所天不寿,自己虽不能守,亦且静处金闺,令媒妁说合事成,虽不免扇坟之诮,然犹是孀妇常情。及嫁,而纨扇多悲,亦须宽心忍耐,安于数命。此玉楼俏心疡,高诸妇一着。春梅一味托大,玉楼一味胆小,故后日成就,春梅必竟有失身受嗜欲之危,而玉楼则一劳而永逸也。(二八)

陈敬济严州一事,岂不蛇足哉?不知作者一笔而三用也。一者为敬济堕落人冷铺作因,二者为大姐一死伏线,三者欲结玉楼实实遇李公子为百年知己,可偿在西门家三、四年之恨也。何以见之?玉楼不为敬济所动,固是心焉李氏,而李公子宁死不舍。天下有宁死不舍之情,非知己之情也哉?可必其无《白头吟》也。观玉楼之风韵嫣然,实是第一个美人,而西门乃独于一滥觞之金莲厚。故写一玉楼,明明说西门为市井之徒,知好淫,而且不知好色也。(二九)

玉楼来西门家,合婚过礼,以视“偷娶”“迎奸赴会”,何啻天壤?其吉凶气象已自不同。其嫁李衙内,则依然合婚行茶过礼,月娘送亲。以视老鸨争论,夜随来旺,王婆领出,不垂别泪,其明晦气象又自不同。故知作者特特写此一位真正美人,为西门不知风雅定案也。(三十)

金莲与瓶儿进门皆受辱。独玉楼自始至终无一褒贬。噫,亦有心人哉!(三一)

西门是混帐恶人,吴月娘是奸险好人,玉楼是乖人,金莲不是人,瓶儿是痴人,春梅是狂人,敬济是浮浪小人,娇儿是死人,雪娥是蠢人,宋蕙莲是不识高低的人,如意儿是顶缺之人。若王六儿与林太太等,直与李桂姐一流。总是不得叫做人。而伯爵、希大辈,皆是没良心的人。兼之蔡太师、蔡状元、宋御史,皆是枉为人也。(三二)

狮子街,乃武松报仇之地,西门几死其处。曾不数日,而于虚又受其害,西门徜徉来往。俟后王六儿,偏又为之移居此地。赏灯,偏令金莲两遍身历其处。写小入托大忘患,嗜恶不悔,一笔都尽。(三三)

《金瓶梅》是一部《史记》。然而《史记》有独传.有合传,却是分开做的。《金瓶梅》却是一百回共成一传,而千百人总合一传,内却又断断续续,各人自有一传,固知作《金瓶》者必能作《史记》也。何则?既已为其难,又何难为其易。(三四)

每见批此书者,必贬他书以褒此书。不知文章乃公共之物,此文妙,何妨彼文亦妙?我偶就此文之妙者而评之,而彼文之妙,固不掩此文之妙者也。即我自作一文,亦不得谓我之文出,而天下之文皆不妙,且不得谓天下更无妙文妙于此者。奈之何批此人之文,即若据为已有,而必使凡天下之文皆不如之。此其同心偏私狭隘,决做不出好文。夫做不出好文,又何能批人之好文哉!吾所谓《史记》易于《金瓶》,盖谓《史记》分做,而《金瓶》全做。即使龙门复生,亦必不谓予左袒《金瓶》。而予亦并非谓《史记》反不妙于《金瓶》,然而《金瓶》却全得《史记》之妙也。文章得失,惟有心者知之。我止赏其文之妙,何暇论其人之为古人,为后古之人,而代彼争论,代彼廉让也哉?(三五)

作小说者,概不留名,以其各有寓意,或暗指某人而作。夫作者既用隐恶扬善之笔,不存其人之姓名,并不露自己之姓名,乃后人必欲为之寻端竟委,说出名姓何哉?何其刻薄为怀也!且传闻之说,大都穿凿,不可深信。总之,作者无感慨,亦必不着书,一言尽之矣。其所欲说之人,即现在其书内。彼有感慨者,反不忍明言;我没感慨者,反必欲指出,真没搭撒、没要紧也。故“别号东楼”,“小名庆儿”之说,概置不问。即作书之人,亦止以“作者”称之。彼既不著名于书,予何多赘哉?近见《七才子书》,满纸王四,虽批者各自有意,而予则谓何不留此闲工,多曲折于其文之起尽也哉?偶记于此,以白当世。(三六)

《史记》中有年表,《金瓶》中亦有时日也。开口云西门庆二十七岁,吴神仙相面则二十九,至临死则三十三岁。而官哥则生于政和四年丙申,卒于政和五御丁酉。夫西门庆二十九岁生子,则丙申年;至三十三岁,该云庚子,而西门乃卒于“戊戌”。夫李瓶儿亦该云卒于政和五年,乃云“七年”,此皆作者故为参差之处。何则?此书独与他小说不同。看其三四年间,却是一日一时推着数去,无论春秋冷热,即某人生日,某
人某日来请酒,某月某日请某人,某日是某节令,齐齐整整捱去。若再将三五年间甲子次序,排得一丝不乱,是真个与西门计帐簿,有如世之无目者所云者也。故特特错乱其年谱,大约三五年间,其繁华如此。则内云某日某节,皆历历生动,不是死板一串铃,可以排头数去。而偏又能使看者五色眯目,真有如捱着一日日过去也。此为神妙之笔。嘻,技至此亦化矣哉!真千古至文,吾不敢以小说目之也。(三七)

一百回是一回,必须放开眼光作一回读,乃知其起尽处。(三八)

一百回不是一日做出,却是一日一刻创成。人想其创造之时,何以至于创成,便知其内许多起尽,费许多经营,许多穿插裁剪也。(三九)

看《金瓶》,把他当事实看,便被他瞒过,必须把他当文章看,方不被他瞒过也。(四十)

看《金瓶》,将来当他的文章看。犹须被他瞒过;必把他当自己的文章读,方不被他满过。(四一)

将他当自己的文章读,是矣。然又不如将他当自己才去经营的文章。我先将心与之曲折算出,夫而后谓之不能瞒我,方是不能瞒我也。(四二)

做文章,不过是“情理”二字。今做此一篇百回长文,亦只是“情理”二字。于一个人心中,讨出一个人的情理,则一个人的传得矣。虽前后夹杂众人的话,而此一人开口,是此一人的情理;非其开口便得情理,由于讨出这一人的情理方开口耳。是故写十百千人皆如写一人,而遂洋洋乎有此一百回大书也。(四三)

《金瓶》每于极忙时偏夹叙他事入内。如正未娶金莲,先插娶孟玉楼;娶玉楼时,即夹叙嫁大姐;生子时,即夹叙吴典恩借债;官哥临危时,乃有谢希大借银;瓶儿死时,乃人玉箫受约;择日出殡,乃有请六黄太尉等事;皆于百忙中,故作消闲之笔。非才富一石者何以能之?外加武松问傅伙计西门庆的话,百忙里说出“二两一月”等文,则又临时用轻笔讨神理,不在此等章法内算也。(四四)

《金瓶梅》妙在善于用犯笔而不犯也。如写一伯爵,更写一希大,然毕竟伯爵是伯爵,希大是希大,各人的身分,各人的谈吐,一丝不紊。写一金莲,更写一瓶儿,可谓犯矣,然又始终聚散,其言语举动,又各各不乱一丝。写一王六儿,偏又写一贲四嫂。写一李桂姐,偏又写一吴银姐、郑月儿。写一王婆,偏又写一薛媒婆、一冯妈妈、一文嫂儿、一陶媒婆。写一薛姑子,偏又写一王姑子、刘姑子。诸如此类,皆妙在、特特犯手,却又各各一款,绝不相同也。(四五)

《金瓶梅》于西门庆,不作一文笔;于月娘,不作一显笔;于玉楼,则纯用俏笔;于金莲,不作一钝笔;于瓶儿,不作一深笔;于春梅,纯用傲笔;于敬济,不作一韵笔;于大姐,不作一秀笔;于伯爵,不作一呆笔;于玳安儿,不着一蠢笔。此所以各各皆到也。(四六)

《金瓶梅》起头放过一男一女。结末又放去一男一女。如卜志道、卓丢儿,是起头放过者。楚云与李安,是结末放去者。夫起头放过去,乃云卜志道是花子虚的署缺者。不肯直出子虚,又不肯明是于十个中止写九个,单留一个缺去寻子虚顶补。故先着一人,随手去之,以出其缺,而便于出子虚,且于出子虚时,随手出瓶儿也。不然,先出子虚于十人之中,则将出瓶儿时又费笔墨。故卜志道虽为子虚署缺,又为瓶儿做楔子也。既云做一楔子,又何有顾意命名之义?而又必用一名,则只云“不知道”可耳,故云“卜志道”。至于丢儿,则又玉楼之署缺者。夫未娶玉楼,先娶此人,既娶玉楼,即丢开此人,岂如李瓶儿今日守灵,明朝烧纸,丫鬟奶子相伴空房,且一番两番托梦也。是诚丢开脑后之人,故云“丢儿”也。是其起头放过者,皆意在放过那人去,放人这人来也。至其结末放去者,’曰楚云者,盖为西门家中彩云易散作一影字。又见得美色无穷,人生有限,死到头来,虽有西子、王嫱,于我何涉?则又作者特特为起讲数语作证也。至于李安,则又与韩爱姐同意,而又为作者十二分满许之笔,写一孝子正人义士,以作中流砥柱也。何则?一部书中,上自蔡太师,下至侯林儿等辈,何止百有余人,并无一个好人,非迎奸卖俏之人,即附势趋炎之辈,使无李安一孝子,不几使良心种子灭绝手?看其写李安母子相依,其一篇话头,真见得守身如玉、不敢毁伤发肤之孝子。以视西门、敬济辈,真猪狗不如之人也。然则末节放过去的两人,又放不过众人,故特特放过此二人,以深省后人也。(四七)

写花子虚即于开首十人中,何以不便出瓶儿哉?夫作者于提笔时,固先有一瓶儿在其意中也。先有一瓶儿在其意中,其后如何偷期,如何迎奸,如何另嫁竹山,如何转嫁西门,其着数俱已算就。然后想到其夫,当令何名,夫不过令其应名而已,则将来虽有如无,故名之曰“子虚”。瓶本为花而有,故即姓花。忽然于出笔时,乃想叙西门氏正传也。于叙西门传中,不出瓶儿,何以入此公案?特叙瓶儿,则叙西门起头时,何以说隔壁一家姓花名某,某妻姓李名某也?此无头绪之笔,必不能人也。然则俟金莲进门再叙何如?夫他小
说,便有一件件叙去,另起头绪于中,惟《金瓶梅》,纯是太史公笔法。夫龙门文字中,岂有于一篇特特着意写之人,且十分有八分写此人之人,而于开卷第一回中不总出枢纽,如衣之领,如花之蒂,而谓之太史公之文哉?近人作一本传奇,于起头数折,亦必将有名人数点到。况《金瓶梅》为海内奇书哉!然则作者又不能自己另出头绪说。势必借结弟兄时,入花子虚也。夫使无伯爵一班人先与西门打热,则弟兄又何由而结?使写子虚亦在十人数内,终朝相见,则于第一回中西门与伯爵会时,子虚系你知我见之人,何以开口便提起“他家二嫂”?即提起二嫂,何以忽说“与咱院子止隔一墙?”而二嫂又何如好也哉?故用写子虚为会外之人,今日拉其人会,而因其邻墙,乃用西门数语,则瓶儿已出,邻墙已明,不言之表,子虚一家皆跃然纸上。因又算到不用卜志道之死,又何因想起拉子虚入会?作者纯以神工鬼斧之笔行文,故曲曲折折,止令看者眯目,而不令其窥彼金针之一度。吾故曰:纯是龙门文字。每于此等文字,使我悉心其中,曲曲折折,为之出入其起尽。何异人五岳三岛,尽览奇胜?我心乐此,不为疲也。(四八)

《金瓶》内,即一笑谈,一小曲,皆因时致宜,或直出本回之意,或足前回,或透下回,当于其下另自分注也。(四九)

《金瓶梅》一书,于作文之法无所不备,一时亦难细说,当各于本回前着明之。(五十)

《金瓶梅》说淫话,止是金莲与王六儿处多,其次则瓶儿,他如月娘、玉楼止一见,而春梅则惟于点染处描写之。何也?写月娘,惟“扫雪”前一夜,所以丑月娘、丑西门也。写玉楼,惟于“含酸”一夜,所以表玉楼之屈,而亦以丑西门也。是皆非写其淫荡之本意也。至于春梅,欲留之为炎凉翻案,故不得不留其身分,而止用影写也。至于百般无耻,十分不堪,有桂姐、月儿不能出之于口者,皆自金莲、六儿口中出之。其难堪为何如?此作者深罪西门,见得如此狗彘,乃偏喜之,真不是人也。故王六儿、潘金莲有日一齐动手,西门死矣。此作者之深意也。至于瓶儿,虽能忍耐,乃自讨苦吃,不关人事,而气死子虚,迎奸转嫁,亦去金莲不远,故亦不妨为之驰张丑态。但瓶儿弱而金莲狠,故写瓶儿之淫,略较金莲可些。而亦早自丧其命于试药之时,甚言女人贪色,不害人即自害也。吁,可畏哉!若蕙莲、如意辈,有何品行?故不妨唐突。而王招宣府内林太太者,我固云为金莲波及,则欲报应之人,又何妨唐突哉!(五一)

《金瓶梅》不可零星看,如零星,便止看其淫处也。故必尽数日之间,一气看完,方知作者起伏层次,贯通气脉,为一线穿下来也。(五二)

凡人谓《金瓶》是淫书者,想必伊止知看其淫处也。若我看此书,纯是一部史公文字。(五三)

做《金瓶梅》之人,若令其做忠臣孝子之文,彼必能又出手眼,摹神肖影,追魂取魄,另做出一篇忠孝文字也。我何以知之?我于其摹写奸夫淫妇知之。(五四)

今有和尚读《金瓶》,人必叱之,彼和尚亦必避人偷看;不知真正和尚方许他读《金瓶梅》。(五五)

今有读书者看《金瓶》,无论其父母师傅禁止之,即其自己亦不敢对人读。不知真正读书者,方能看《金瓶梅》,其避人读者,乃真正看淫书也。(五六)

作《金瓶》者,乃善才化身,故能百千解脱,色色皆到。不然正难梦见。(五七)

作《金瓶》者,必能转身证菩萨果。盖其立言处,纯是麟角凤嘴文字故也。(五八)

作《金瓶梅》者,必曾于患难穷愁,人情世故,一一经历过,人世最深,方能为众脚色摹神了。(五九)

作《金瓶梅》,若果必待色色历遍才有此书,则《金瓶梅》又必做不成也。何则?即如诸淫妇偷汉,种种不同,若必待身亲历而后知之,将何以经历哉?故知才子无所不通,专在一心也。(六十)

一心所通,实又真个现身一番,方说得一番。然则其写诸淫妇,真乃各现淫妇人身,为人说法者也。(六一)

其书凡有描写,莫不各尽人情。然则真千百化身现各色人等,为之说法者也。(六二)

其各尽人情,莫不各得天道。即千古算来,天之祸淫福善,颠倒权奸处,确乎如此。读之,似有一人亲曾执笔,在清河县前,西门家里,大大小小,前前后后,碟儿碗儿,一—记之,似真有其事,不敢谓为操笔伸纸做出来的。吾故曰:得天道也。(六三)

读《金瓶》,当看其白描处。子弟能看其白描处,必能自做出异样省力巧妙文字来也。(六四)

读《金瓶》,当看其脱卸处。子弟看其脱卸处,必能自出手眼,作过节文字也。(六五)

读《金瓶》,当看其避难处。子弟看其避难就易处,必能放重笔拿轻笔,异样使乖脱滑也。(六六)

读《金瓶》,当看其手闲事忙处。子弟会得,便许作繁衍文字也。(六七)

读《金瓶》,当看其穿插处。子弟会得,便许他作花团锦簇、五色眯人的文字也。(六八)

读《金瓶》,当看其结穴发脉、关锁照应处。子弟会得,才许他读《左》、《国》、《庄》、《骚》、《史》、子也。(六九)

读《金瓶》,当知其用意处。夫会得其处处所以用意处,方许他读《金瓶梅》,方许他自言读文字也。(七十)

幼时在馆中读文,见窗友为先生夏楚云:“我教你字宇想来,不曾教你囫轮吞。”予时尚幼,旁听此言,即深自儆省。于念文时,即一字一字作昆腔曲,拖长声,调转数四念之,而心中必将此一字,念到是我用出的一字方罢。犹记念的是“好古敏以求之”一句的文字,如此不三日,先生出会课题,乃“君子矜而不争”,予自觉做时不甚怯力而文成。先生大惊,以为抄写他人,不然何进益之速?予亦不能白。后先生留心验予动静,见予念文,以头伏桌,一手指文,一字一字唱之,乃大喜曰:“子不我欺”。且回顾同窗辈曰:“尔辈不若也”。今本不通,然思读书之法,断不可成片念过去。岂但读文,即如读《金瓶梅》小说,若连片念去,便味如嚼蜡,止见满篇老婆舌头而已,安能知其为妙文也哉!夫不看其妙文,然则止要看其妙事乎?是可一大揶揄。(七一)

读《金瓶》,必须静坐三月方可。否则眼光模糊,不能激射得到。(七二)

才不高,由于心粗,心粗由于气浮。心粗则气浮,气愈浮则心愈粗。岂但做不出好文,‘并亦看不出好文。遇此等人,切不可将《金瓶梅》与他读。(七三)

未读《金瓶梅》,而文字如是,既读《金瓶梅》,而文字犹如是。此人直须焚其笔砚,扶犁耕田为大快活,不必再来弄笔砚,自讨苦吃也。(七四)

做书者是诚才子矣,然到底是菩萨学问,不是圣贤学问,盖其专教人空也。若再进一步,到不空的所在,其书便不是这样做也。(七五)

《金瓶》以空结,看来亦不是空到地的,看他以孝哥结便知。然则所云“幻化”,乃是以孝化百恶耳。(七六)

《金瓶梅》到底有一种愤懑的气象,然则《金瓶梅》断断是龙门再世。(七七)

《金瓶梅》是部改过的书,观其以爱姐结便知。盖欲以三年之艾,治七年之病也。(七八)

《金瓶梅》究竟是大彻悟的人做的,故其中将僧尼之不肖处,一一写出。此方是真正菩萨,真正彻悟。(七九)

《金瓶梅》倘他当日发心不做此一篇市井的文字,他必能另出韵笔。,作花娇月媚如《西厢》等文字也。(八十)

《金瓶》必不可使不会做文的人读。夫不会做文字人读,则真有如俗云“读了《金瓶梅》”也。会做文字的人读《金瓶》,纯是读《史记》。(八一)

《金瓶梅》切不可令妇女看见。世有销金帐底,浅斟低唱之下,念一回于妻妾听者多多矣。不知男子中尚少知劝戒观感之人,彼女子中能观感者几人哉?少有效法,奈何奈何!至于其文法笔法,又非女子中所能学,亦不必学。即有精通书史者,则当以《左》、《国》、《风雅》、经史与之读也。然则,《金瓶梅》是不可看之书也,我又何以批之以误世哉?不知我正以《金瓶》为不可不看之妙文,特为妇人必不可看之书,恐前人呕心呕血做这妙文——虽本自娱,实亦欲娱千百世之锦绣才子者——乃为俗人所掩,尽付流水,是谓人误《金瓶》。何以谓西门庆误《金瓶》?使看官不作西门的事读,全以我此日文心,逆取他当日的妙笔,则胜如读一部《史记》。乃无如开卷便止知看西门庆如何如何,全不知作者行文的一片苦心,是故谓之西门庆误《金瓶梅》。然则仍依旧看官误看了西门庆的《金瓶梅》,不知为作者的《金瓶》也。常见一人批《金瓶梅》曰:“此西门之大帐簿”。其两眼无珠,可发一笑。夫伊于甚年月日,见作者雇工于西门庆家写帐簿哉?有读至敬济“弄一得双”,乃为西门大愤日:“何其剖其双珠!”不知先生又错看了也。金莲原非西门所固有,而作者特写一春梅,亦非欲为西门庆所能常有之人而写之也。此自是作者妙笔妙撰,以行此妙文,何劳先生为之旁生瞎气哉了故读《金瓶》者多,不善读《金瓶》者亦多。予因不揣,乃急欲批以请教。虽不敢谓能探作者之底里,然正因作者叫屈不歇,故不择狂瞽,代为争之。且欲使有志作文者,同醒一醒长日睡魔,少补文家之法律也。谁曰不宜?(八二)

《金瓶》是两半截书。上半截热,下半截冷;上半热中有冷,下牛冷中有热。(八三)

《金瓶梅》因西门庆一分人家,写好几分人家。如武大一家,花子虚一家,乔大户一家,陈洪一家,吴大舅一家,张大户一家,王招宣一家,应伯爵一家,周守备一家,何千户一家,夏提刑一家。他如悴云峰,在东京不算。伙计家以及女眷不往来者不算。凡这几家,大约清河县官员大户,屈指

已遍。而因一人写及一县,吁!元恶大惇论此回有几家,全倾其手,深遭荼毒也,可恨,可恨!(八四)

《金瓶梅》写西门庆无一亲人,上无父母,下无子孙,中无兄弟。幸而月娘犹不以继室自居。设也月娘因金莲终不通言对面,吾不知西门庆何乐乎为人也。乃于此不自改过自修,且肆恶无忌,宜乎就死不悔也。(八五)

书内写西门许多亲戚,通是假的。如乔亲家,假亲家也;翟亲家,愈假之亲家也;杨姑娘,谁氏之姑娘?假之姑娘也;应二哥,假兄弟也;谢子纯,假朋友也。至于花大舅、二舅,更属可笑,真假到没文理处也。敬济两番披麻戴孝,假孝子也。至于沈姨夫、韩姨夫,不闻有姨娘来,亦是假姨夫矣。惟吴大舅、二舅,而二舅又如鬼如蜮,吴大舅少可,故后卒得吴大舅略略照应也。彼西门氏并无一人,天之报施亦惨,而文人恶之者亦毒矣。奈何世人于一本九族之亲,乃漠然视之,且恨不排挤而去之,是何肺腑!(八六)

《金瓶》何以必写西门庆孤身一人,无一着己亲哉?盖必如此,方见得其起头热得可笑,后文一冷便冷到彻底,再不能热也。(八七)

作者直欲使此清河县之西门氏冷到彻底,并无一人。虽属寓言,然而其恨此等人,直使之千百年后,永不复望一复燃之灰。吁!文人亦狠矣哉!(八八)

《金瓶》内有一李安,是个孝子。却还有一个王杏庵,是个义士。安童是个义仆,黄通判是个益友,曾御史是忠臣,武二郎是个豪杰悌弟。谁谓一片淫欲世界中,天命民懿为尽灭绝也哉?(八九)

《金瓶》虽有许多好人,却都是男人,并无一个好女人。屈指不二色的,要算月娘一个。然却不知妇道以礼持家,往往惹出事端。至于爱姐,晚节固可佳,乃又守得不正经的节,且早年亦难清白。他如葛翠屏,娘家领去,作者固未定其末路,安能必之也哉?甚矣,妇人阴性,虽岂无贞烈者?然而

失守者易,且又在各人家教。观于此,可以禀型于之惧矣,齐家者可不慎哉?(九十)

《金瓶梅》内却有两个真人,一尊活佛,然而总不能救一个妖僧之流毒。妖僧为谁?施春药者也。(九一)

武大毒药,既出之西门庆家,则西门毒药,固有人现身而来。神仙、真人、活佛,亦安能逆天而救之也哉!(九二)

读《金瓶》,不可呆看,一呆看便错了。(九三)

读《金瓶》,必须置唾壶于侧,庶便于击。(九四)

读《金瓶》,必须列宝剑于右,或可划空泄愤。(九五)

读《金瓶》,必须悬明镜于前,庶能圆满照见。(九六)

读《金瓶》,必置大白于左,庶可痛饮,以消此世情之恶。(九七)

读《金瓶》,必置名香于几,庶可遥谢前人,感其作妙文,曲曲折折以娱我。(九八)

读《金瓶》,必须置香茗于案,以奠作者苦心。(九九)

《金瓶》亦并不晓得有甚圆通,我亦正批其不晓有甚圆通处也。(一百)

《金瓶》以“空”字起结,我亦批其以“空”字起结而已,到底不敢以“空”字诬我圣贤也。(百一)

《金瓶》以“空’’字起吉,我亦批其以“空”字起结而已,到底不敢以“空”字诬我圣贤也。(百二)

《金瓶》处处体贴人情天理,此是其真能悟彻了,此是其不空处也。(百三)

《金瓶梅》是大手笔,却是极细的心思做出来者。(百四)

《金瓶梅》是部惩人的书,故谓之戒律亦可。虽然又云《金瓶梅》是部人世的书,然谓之出世的书亦无不可。(百五)

金、瓶、梅三字连贯者,是作者自喻。此书内虽包藏许多春色,却一朵一朵,一瓣一瓣,费尽春工,当
注之金瓶,流香芝室,为千古锦绣才子作案头佳玩,断不可使村夫俗子作枕头物也。噫!夫金瓶梅花,全凭人力以补天王,则又如此书处处以文章夺化工之巧也夫。(百六)

此书为继《杀狗记》而作。看他随处影写兄弟,如何九之弟何十,杨大郎之弟杨二郎,周秀之弟周宣,韩道国之弟韩二捣鬼。惟西门庆、陈敬济无兄弟可想。(百七)

以玉楼弹阮起,爱姐抱阮结,乃是作者满肚皮猖狂之泪没处洒落,故以《金瓶梅》为大哭地也。(百八)

} % 张竹坡皋鹤堂批评第一奇书批评本


%\piWenlongF{ % 文龙在兹堂本手书批评本
%\chapter*{wenlong?}
%\addcontentsline{toc}{chapter}{wenlong?}
%} % 文龙在兹堂本手书批评本




\end{showcontents}

