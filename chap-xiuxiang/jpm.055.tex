%# -*- coding: utf-8 -*-
%!TEX encoding = UTF-8 Unicode
%!TEX TS-program = xelatex
% vim:ts=4:sw=4
%
% 以上设定默认使用 XeLaTex 编译,并指定 Unicode 编码,供 TeXShop 自动识别

%第五十五回 
\chapter{西門慶兩番慶壽旦 苗員外一諾送歌童}

詞曰:

師表方眷遇,魚水君臣,須信從來少。寶運當千,佳辰餘五,嵩岳誕生元老。帝遣阜安宗社,人仰雍容廊廟。願歲歲共祝眉壽,壽比山高。

卻說任醫官看了脈息,依舊到廳上坐下。西門慶便開言道:「不知這病癥端的何如?」任醫官道:「夫人這病,原是產後不慎調理,因此得來。目下惡路不凈,面帶黃色,飲食也沒些要緊,走動便覺煩勞。依學生愚見,還該謹慎保重。如今夫人兩手脈息虛而不實,按之散大。這病癥都只為火炎肝腑,土虛木旺,虛血妄行。若今番不治,後邊一發了不的。」說畢,西門慶道:「如今該用甚藥才好?」任醫官道:「只用些清火止血的藥──黃柏、知母為君,其餘再加減些,吃下看住,就好了。」西門慶聽了,就叫書童封了一兩銀子,送任醫官做藥本,任醫官作謝去了。不一時,送將藥來,李瓶兒屋裡煎服,不在話下。

且說西門慶送了任醫官去,回來與應伯爵說話。伯爵因說:「今日早晨,李三、黃四走來,說他這宗香銀子急的緊,再三央我來求哥。好歹哥看我面,接濟他這一步兒罷。」西門慶道:「既是這般急,我也只得依你了。你叫他明日來兌了去罷。」一面讓伯爵到小捲棚內,留他吃飯。伯爵因問:「李桂兒還在這裡住著哩?東京去的也該來了。」西門慶道:「正是,我緊等著還要打發他往揚州去,敢怕也只在早晚到也。」說畢,吃了飯,伯爵別去。到次日,西門慶衙門中回來,伯爵早已同李智、黃四坐在廳上等。見西門慶回來,都慌忙過來見了。西門慶進去換了衣服,就問月娘取出徐家討的二百五十兩銀子,又添兌了二百五十兩,叫陳敬濟拿了,同到廳上,兌與李三、黃四。因說道:「我沒銀子,因應二哥再三來說,只得湊與你。──我卻是就要的。」李三道:「蒙老爹接濟,怎敢遲延!如今關出這批銀子,一分也不敢動,就都送了來,」於是兌收明,千恩萬謝去了。伯爵也就要去,被西門慶留下。

正坐的說話,只見平安兒進來報說:「來保東京回來了。」伯爵道:「我昨日就說也該來了。」不一時,來保進到廳上,與西門慶磕了頭。西門慶便問:「你見翟爹麼?李桂姐事情怎樣了?」來保道:「小的親見翟爹。翟爹見了爹的書,隨即叫長班拿帖兒與朱太尉去說,小的也跟了去。朱太尉親吩咐說:『既是太師府中分上,就該都放了。因是六黃太尉送的,難以回他,如乃未到者,俱免提;已拿到的,且監些時。他內官性兒,有頭沒尾。等他性兒坦些,也都從輕處就是了。』」伯爵道:「這等說,連齊香兒也免提了?──造化了這小淫婦兒了!」來保道:「就是祝爹他每,也只好打幾下罷了。罪,料是沒了。」一面取出翟管家書遞上。西門慶看了說道:「老孫與祝麻子,做夢也不曉的是我這裡人情。」伯爵道:「哥,你也只當積陰騭罷了。」來保又說:「翟爹見小的去,好不歡喜,問爹明日可與老爺去上壽?小的不好回說不去,只得答應:『敢要來也。』翟爹說:『來走走也好,我也要與你爹會一會哩。』」西門慶道:「我到也不曾打點自去。既是這等說,只得要去走遭了。」因吩咐來保:「你辛苦了,且到後面吃些酒飯,歇息歇息。遲一兩日,還要趕到揚州去哩。」來保應諾去了。西門慶就要進去與李桂姐說知,向伯爵道:「你坐著,我就來。」伯爵也要去尋李三、黃四,乘機說道:「我且去著,再來罷。」一面別去。

西門慶來到月娘房裡,李桂姐已知道信了,忙走來與西門慶、月娘磕頭,謝道:「難得爹娘費心,救了我這一場大禍。拿甚麼補報爹娘!」月娘道:「你既在咱家恁一場,有些事兒,不與你處處,卻為著甚麼來?」桂姐道:「俺便賴爹娘可憐救了,只造化齊香兒那小淫婦兒,他甚相干?連他都饒了。他家賺錢賺鈔,帶累俺們受驚怕,俺每倒還只當替他說了個大人情,不該饒他才好!」西門慶笑道:「真造化了這小淫婦兒了。」說了一回,掛姐便要辭了家去,道:「我家媽還不知道這信哩,我家去說聲,免得他記掛,再同媽來與爹娘磕頭罷。」西門慶道:「也罷,我不留你,你且家去說聲著。」月娘道:「桂姐,你吃了飯去。」桂姐道:「娘,我不吃飯了。」一面又拜辭西門慶與月娘眾人。臨去,西門慶說道:「事便完了,你今後,這王三官兒也少招攬他了。」桂姐道:「爹說的是甚麼話,還招攬他哩!再要招攬他,就把身子爛化了。就是前日,也不是我招攬他。」月娘道:「不招攬他就是了,又平白說誓怎的?」一面叫轎子,打發桂姐去了。西門慶因告月娘說要上東京之事。月娘道:「既要去,須要早打點,省得臨時促忙促急。」西門慶道:「蟒袍錦繡、金花寶貝,上壽禮物,俱已完備,倒只是我的行李不曾整備。」月娘道:「行李不打緊。」西門慶說畢,就到前邊看李瓶兒去了。到次日,坐在捲棚內,叫了陳敬濟來,看著寫了蔡御史的書,交與來保,又與了他盤纏,叫他明日起早趕往揚州去,不題。

倏忽過了數日,看看與蔡太師壽誕將近,只得擇了吉日,吩咐琴童、玳安、書童、畫童四個小廝跟隨,各各收拾行李。月娘同玉樓、金蓮眾人,將各色禮物並冠帶衣服應用之物,共裝了二十餘扛。頭一日晚夕,妻妾眾人擺設酒餚和西門慶送行。吃完酒,就進月娘房裡宿歇。次日,把二十扛行李先打發出門,又發了一張通行馬牌,仰經過驛遞起夫馬迎送。各各停當,然後進李瓶兒房裡來,看了官哥兒,與李瓶兒說道:「你好好調理。要藥,叫人去問任醫官討。我不久便來家看你。」那李瓶兒閣著淚道:「路上小心保重。」直送出廳來,和月娘、玉樓、金蓮打夥兒送了出大門。西門慶乘了涼轎,四個小廝騎了頭口,望東京進發。迤邐行來,免不得朝登紫陌,夜宿郵亭,一路看了些山明水秀,相遇的無非都是各路文武官員進京慶賀壽誕,生辰扛不計其數。約行了十來日,早到東京。進了萬壽城門,那時天色將晚,趕到龍德街牌樓底下,就投翟家屋裡去住歇。

那翟管家聞知西門慶到了,忙出來迎接,各敘寒暄。吃了茶,西門慶叫玳安將行李一一交盤進翟家來。翟謙交府乾收了,就擺酒和西門慶洗塵。不一時,只見剔犀官桌上,擺上珍羞美味來,只好沒有龍肝鳳髓罷了,其餘般般俱有,便是蔡太師自家受用,也不過如此。當值的拿上酒來,翟謙先滴了天,然後與西門慶把盞。西門慶也回敬了。兩人坐下,糖果按酒之物,流水也似遞將上來。酒過兩巡,西門慶便對翟謙道:「學生此來,單為與老太師慶壽,聊備些微禮孝順太師,想不見卻。只是學生久有一片仰高之心,欲求親家預先稟過:但得能拜在太師門下做個乾生子,便也不枉了人生一世。不知可以啟口麼?」翟謙道:「這個有何難哉!我們主人雖是朝廷大臣,卻也極好奉承。今日見了這般盛禮,不惟拜做乾子,定然允從,自然還要升選官爵。」西門慶聽說,不勝之喜。飲夠多時,西門慶便推不吃酒了。翟管家道:「再請一杯,怎的不吃了?」西門慶道:「明日有正經事,不敢多飲。」再四相勸,只又吃了一杯。

翟管家賞了隨從人酒食,就請西門慶到後邊書房裡安歇。排下暖床綃帳,銀鉤錦被,香噴噴的。一班小廝扶侍西門慶脫衣上床。獨宿──西門慶一生不慣,那一晚好難捱過。巴到天明,正待起身,那翟家門戶重重掩著。直挨到巳牌時分,才有個人把鑰匙一路開將出來。隨後才是小廝拿手巾香湯進書房來。西門慶梳洗完畢,只見翟管家出來和西門慶廝見,坐下。當值的就托出一個朱紅盒子來,裡邊有三十來樣美味,一把銀壺斟上酒來吃早飯。翟謙道:「請用過早飯,學生先進府去和主翁說知,然後親家搬禮物進來。」西門慶道:「多勞費心!」酒過數杯,就拿早飯來吃了,收過家活。翟管家道:「且權坐一回,學生進府去便來。」

翟謙去不多時,就忙來家,向西門慶說:「老爺正在書房梳洗,外邊滿朝文武官員都伺候拜壽,未得廝見哩。學生已對老爺說過了,如今先進去拜賀罷,省的住回人雜。學生先去奉候,親家就來罷了。」說畢去了。西門慶不勝歡喜。便教跟隨人拉同翟家幾個伴當,先把那二十扛金銀緞匹抬到太師府前,一行人應聲去了。西門慶即冠帶,乘了轎來。只見亂哄哄,挨肩擦背,都是大小官員來上壽的。西門慶遠遠望見一個官員,也乘著轎進龍德坊來。西門慶仔細一看,卻認的是故人揚州苗員外。不想那苗員外也望見西門慶,兩個同下轎作揖,敘說寒溫。原來這苗員外也是個財主,他身上也現做著散官之職,向來結交在蔡太師門下,那時也來上壽,恰遇了故人。當下,兩個忙匆匆路次話了幾句,問了寓處,分手而別。

西門慶來到太師府前,但見:

堂開綠野,閣起凌煙。門前寬綽堪旋馬,閥閱嵬峨好豎旗。錦繡叢中,風送到畫眉聲巧;金銀堆里,日映出琪樹花香。左右活屏風,一個個夷光紅拂;滿堂死寶玩,一件件周鼎商彝。室掛明珠十二,黑夜裡何用燈油;門迎珠履三千,白日間盡皆名士。九州四海,大小官員,都來慶賀;六部尚書,三邊總督,無不低頭。正是:除卻萬年天子貴,只有當朝宰相尊。

西門慶恭身進了大門,翟管家接著,只見中門關著不開,官員都打從角門而入。西門慶便問:「為何今日大事,卻不開中門?」翟管家道:「中門曾經官家行幸,因此人不敢走。」西門慶和翟謙進了幾重門,門上都是武官把守,一些兒也不混亂。見了翟謙,一個個都欠身問管家:「從何處來?」翟管家答道:「舍親打山東來拜壽老爺的。」說罷,又走過幾座門,轉幾個彎,無非是畫棟雕梁,金張甲第。隱隱聽見鼓樂之聲,如在天上一般。西門慶又問道:「這裡民居隔絕,那裡來的鼓樂喧嚷?」翟管家道:「這是老爺教的女樂,一班二十四人,都曉得天魔舞、霓裳舞、觀音舞。但凡老爺早膳、中飯、夜宴,都是奏的。如今想是早膳了。」西門慶聽言未了,又鼻子里覺得異香馥馥,樂聲一發近了。翟管家道:「這裡與老爺書房相近了,腳步兒放鬆些。」

轉個迴廊,只見一座大廳,如寶殿仙宮。廳前仙鶴、孔雀種種珍禽,又有那瓊花、曇花、佛桑花,四時不謝,開的閃閃爍爍,應接不暇。西門慶還未敢闖進,交翟管家先進去了,然後挨挨排排走到堂前。只見堂上虎皮交椅上坐一個大猩紅蟒衣的,是太師了。屏風後列有二三十個美女,一個個都是宮樣妝束,執巾執扇,捧擁著他。翟管家也站在一邊。西門慶朝上拜了四拜,蔡太師也起身,就絨單上回了個禮。──這是初相見了。落後,翟管家走近蔡太師耳邊,暗暗說了幾句話下來,西門慶理會的是那話了,又朝上拜四拜,蔡太師便不答禮。──這四拜是認乾爺,因此受了。西門慶開言便以父子稱呼道:「孩兒沒恁孝順爺爺,今日華誕,特備的幾件菲儀,聊表千里鵝毛之意。願老爺壽比南山。」蔡太師道:「這怎的生受!」便請坐下。當值的拿了把椅子上來,西門慶朝上作了個揖道:「告坐了。」就西邊坐地吃茶。翟管家慌跑出門來,叫抬禮物的都進來。須臾,二十扛禮物擺列在階下。揭開了涼箱蓋,呈上一個禮目:大紅蟒袍一套、官綠龍袍一套、漢錦二十匹、蜀錦二十匹、火浣布二十匹、西洋布二十匹,其餘花素尺頭共四十匹、獅蠻玉帶一圍、金鑲奇南香帶一圍、玉杯犀杯各十對、赤金攢花爵杯八隻、明珠十顆,又另外黃金二百兩,送上蔡太師做贄見禮。蔡太師看了禮目,又瞧見抬上二十來扛,心下十分歡喜,說了聲「多謝!」便叫翟管家收進庫房去了。一面吩咐擺酒款待。西門慶因見他忙衝衝,就起身辭蔡太師。太師道:「既如此,下午早早來罷。」西門慶又作個揖,起身出來。蔡太師送了幾步,便不送了。西門慶依舊和翟管家同出府來。翟管家府內有事,也作別進去。

西門慶竟回到翟家來,脫下冠帶,已整下午飯,吃了一頓。回到書房,打了個盹,恰好蔡太師差舍人邀請赴席,西門慶謝了些扇金,著先去了。即便重整冠帶,又叫玳安封下許多賞封,做一拜匣盛了,跟隨著四個小廝,復乘轎望太師府來。蔡太師那日滿朝文武官員來慶賀的,各各請酒。自次日為始,分做三停:第一日是皇親內相,第二日是尚書顯要、衙門官員,第三日是內外大小等職。只有西門慶,一來遠客,二來送了許多禮物,蔡太師到十分歡喜,因此就是正日獨獨請他一個。見西門慶到了,忙走出軒下相迎。西門慶再四謙遜,讓:「爺爺先行。」自家屈著背,輕輕跨入檻內,蔡太師道:「遠勞駕從,又損隆儀。今日略坐,少表微忱。」西門慶道:「孩兒戴天履地,全賴爺爺洪福,些小敬意,何足掛懷!」兩個喁喁笑語,真似父子一般。二十四個美女,一齊奏樂,府乾當值的斟上酒來。蔡太師要與西門慶把盞,西門慶力辭不敢,只領的一盞,立飲而盡,隨即坐了桌席。西門慶叫書童取過一隻黃金桃杯,斟上一杯,滿滿走到蔡太師席前,雙膝跪下道:「願爺爺千歲!」蔡太師滿面歡喜道:「孩兒起來。」接過便飲個完。西門慶才起身,依舊坐下。那時相府華筵,珍奇萬狀,都不必說。西門慶直飲到黃昏時候,拿賞封賞了諸執役人,才作謝告別道:「爺爺貴冗,孩兒就此叩謝,後日不敢再來求見了。」出了府門,仍到翟家安歇。

次日,要拜苗員外,著玳安跟尋了一日,卻在皇城後李太監房中住下。玳安拿著帖子通報了,苗員外來出迎道:「學生正想個知心朋友講講,恰好來得湊巧。」就留西門慶筵燕。西門慶推卻不過,只得便住了。當下山餚海錯不記其數。又有兩個歌童,生的眉清目秀,頓開喉音,唱幾套曲兒。西門慶指著玳安、琴童向苗員外說道:「這班蠢材,只會吃酒飯,怎地比的那兩個!」苗員外笑道:「只怕伏侍不的老先生,若愛時,就送上也何難!」西門慶謙謝不敢奪人之好。飲到更深,別了苗員外,依舊來翟家歇。那幾日內相府管事的,各各請酒,留連了八九日。西門慶歸心如箭,便叫玳安收拾行李。翟管家苦死留住,只得又吃了一夕酒,重敘姻親,極其眷戀。次日早起辭別,望山東而行。一路水宿風餐,不在話下。

且說月娘家中,自從西門慶往東京慶壽,姊妹每望眼巴巴,各自在屋裡做些針指,通不出來閑耍。只有潘金蓮打扮的如花似玉,喬模喬樣,在丫鬢夥里,或是猜枚,或是抹牌,說也有,笑也有,狂的通沒些成色。嘻嘻哈哈,也不顧人看見,只想著與陳敬濟勾搭。每日只在花園雪洞內踅來踅去,指望一時湊巧。敬濟也一心想著婦人,不時進來尋撞,撞見無人便調戲,親嘴咂舌做一處,只恨人多眼多,不能盡情歡會。正是:

雖然未入巫山夢,卻得時逢洛水神。

一日,吳月娘、孟玉樓、李瓶兒同一處坐地,只見玳安慌慌跑進門來,見月娘眾人磕了頭,報道:「爹回來了。」月娘便問:「如今在那裡?」玳安道:「小的一路騎頭口,拿著馬牌先行,因此先到家。爹這時節,也差不上二十里遠近了。」月娘道:「你曾吃飯沒有?」玳安道:「從早上吃來,卻不曾吃中飯。」月娘便吩咐整飯伺候,一面就和六房姊妹同夥兒到廳上迎接。正是:

詩人老去鶯鶯在,公子歸時燕燕忙。

妻妾每在廳上等候多時,西門慶方到門前下轎了,眾妻妾一齊相迎進去。西門慶先和月娘廝見畢,然後孟玉樓、李瓶兒、潘金蓮依次見了,各敘寒溫。落後,書童、琴童、畫童也來磕了頭,自去廚下吃飯。西門慶把路上辛苦併到翟家住下、感蔡太師厚情請酒並與內相日日吃酒事情,備細說了一遍。因問李瓶兒:「孩子這幾時好麼?你身子吃的任醫官藥,有些應驗麼?我雖則往東京,一心只弔不下家裡。」李瓶兒道:「孩子也沒甚事,我身子吃藥後,略覺好些。」月娘一面收好行李及蔡太師送的下程,一面做飯與西門慶吃。到晚又設酒和西門慶接風。西門慶晚夕就在月娘房裡歇了。兩個是久旱逢甘雨,他鄉遇故知。歡愛之情,俱不必說。

次日,陳敬濟和大姐也來見了,說了些店裡的帳目。應伯爵和常峙節打聽的來家,都來探望。西門慶出來相見畢,兩個一齊說:「哥一路辛苦。」西門慶便把東京富麗的事情及太師管待情分,備細說了一遍。兩人只顧稱羨不已。當日,西門慶留二人吃了一日酒。常峙節臨起身向西門慶道:「小弟有一事相求,不知哥可照顧麼?」說著,只是低了臉,半含半吐。西門慶道:「但說不妨。」常峙節道:「實為住的房子不方便,待要尋間房子安身,卻沒有銀子。因此要求哥周濟些兒,日後少不的加些利錢送還哥。」西門慶道:「相處中說甚利錢!只我如今忙忙的,那討銀子?且待韓伙計貨船來家,自有個處。」說罷,常峙節、應伯爵作謝去了,不在話下。

且說苗員外自與西門慶相會,在酒席上把兩個歌童許下。不想西門慶歸心如箭,不曾別的他,竟自歸來。苗員外還道西門慶在京,差伴當來翟家問,才曉得西門慶家去了。苗員外自想道:「君子一言,快馬一鞭。我既許了他,怎麼失信!」於是叫過兩個歌童吩咐道:「我前日請山東西門大官人,曾把你兩個許下他。我如今就要送你到他家去,你們早收拾行李。」那兩個歌童一齊跪告道:「小的每伏侍的員外多年,員外不知費盡多少心力,教的俺每這些南曲,卻不留下自家歡樂,怎地到送與別人?」說罷,撲簌簌掉下淚來。那員外也覺慘然不樂,說道:「你也說的是,咱何苦定要送人?只是:『人而無信,不知其可也。』──那孔聖人說的話怎麼違得!如今也由不得你了,待咱修書一封,差人送你去,教他好生看覷你就是了。」兩個歌童違拗不過,只得應諾起來。苗員外就叫那門管先生寫著一封書信,寫那相送歌童之意。又寫個禮單兒,把些尺頭書帕封了,差家人苗實齎書,護送兩個歌童往西門慶家來。兩個歌童灑淚辭謝了員外,翻身上馬,迤邐同望山東大道而來。有日到了清河縣,三人下馬訪問,一直逕到縣牌坊西門慶家府里投下。

卻說西門慶自從東京到家,每日忙不迭,送禮的,請酒的,日日三朋四友,以此竟不曾到衙門裡去。那日稍閑無事,才到衙門裡升堂畫卯,把那些解到的人犯,同夏提刑一一審問一番。審問了半日,公事畢,方乘了一乘涼轎,幾個牢子喝道,簇擁來家。只見那苗實與兩個歌童已是候的久了,就跟著西門慶的轎子,隨到前廳,跪下稟說:「小的是揚州苗員外有書拜候老爹。」隨將書並禮物呈上。西門慶連忙說道:「請起來。」一面打開副啟,細細看了。見是送他歌童,心下喜之不勝,說道:「我與你員外意外相逢,不想就蒙你員外情投意合。酒後一言,就果然相贈,又不憚千里送來。你員外真可謂千金一諾矣。難得,難得!」兩個歌童從新走過,又磕了四個頭,說道:「員外著小的們伏侍老爹,萬求老爹青目!」西門慶道:「你起來,我自然重用。」一面叫擺酒飯,管待苗實並兩個歌童;一面整辦厚禮── 綾羅細軟,修書答謝員外;一面就叫兩個歌童,在於書房伺候。不想,韓道國老婆王六兒,因見西門慶事忙,要時常通個信兒,沒人往來,算計將他兄弟王經──才十五六歲,也生得清秀──送來伏侍西門慶,也是這日進門。西門慶一例收下,也叫在書房中伺候。

西門慶正在廳上分撥,忽伯爵走來。西門慶與他說知苗員外送歌童之事,就叫玳安裡面討出酒菜兒來,留他坐,就叫兩個歌童來唱南曲。那兩個歌童走近席前,並足而立,手執檀板,唱了一套《新水令》「小園昨夜放江梅」,果然是響遏行雲,調成白雪。伯爵聽了,歡喜的打跌,贊說道:「哥的大福,偏有這些妙人兒送將來。也難為這苗員外好情。」西門慶道:「我少不得尋重禮答他。」一面又與這歌童起了兩個名:一個叫春鴻,一個叫春燕。又叫他唱了幾個小詞兒,二人吃一回酒,伯爵方纔別去。正是:

風花弄影新鶯囀,俱是筵前歌舞人。

