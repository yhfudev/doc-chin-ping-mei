%# -*- coding: utf-8 -*-
%!TEX encoding = UTF-8 Unicode
%!TEX TS-program = xelatex
% vim:ts=4:sw=4
%
% 以上设定默认使用 XeLaTex 编译,并指定 Unicode 编码,供 TeXShop 自动识别

%第二十一回 
\chapter{吳月娘掃雪烹茶 應伯爵替花邀酒}

\begin{showcontents}{}


詞曰:

並刀如水,吳鹽勝雪,纖手破新橙。錦幄初溫,獸煙不斷,相對坐調笙。
低聲問向誰行宿,城上已三更。馬滑霜濃,不如休去,直至少人行。

話說西門慶從院中歸家,已一更天氣,到家門首,小廝叫開門,下了馬,踏著那亂瓊碎玉,到於後邊儀門首。只儀門半掩半開,院內悄無人聲。西門慶心內暗道: 「此必有蹺蹊。」於是潛身立於儀門內粉壁前,悄悄聽覷。只見小玉出來,穿廊下放桌兒。原來吳月娘自從西門慶與他反目以來,每月吃齋三次,逢七拜斗焚香,保佑夫主早早回心,西門慶還不知。只見小玉放畢香桌兒。少頃,月娘整衣出來,向天井內滿爐炷香,望空深深禮拜。祝曰:「妾身吳氏,作配西門。奈因夫主留戀煙花,中年無子。妾等妻妾六人,俱無所出,缺少墳前拜掃之人。妾夙夜憂心,恐無所托。是以發心,每夜於星月之下,祝贊三光,要祈佑兒夫,早早回心。棄卻繁華,齊心家事。不拘妾等六人之中,早見嗣息,
\piZhang{以此愈知其假如。} % 张旁批
\piLiyu{此尤人情所难。} % 绣像旁批
\piWenlong{先生何以知其假?便令是假,妇人能如此挽回丈夫之心,亦算叩头陪罪了。不知先生之意,果欲何为?} % 文龙旁批
以為終身之計,乃妾之素願也。」正是:

私出房櫳夜氣清,一庭香霧雪微明。
拜天訴盡衷腸事,無限徘徊獨自惺。

這西門慶不聽便罷,聽了月娘這一篇言語,不覺滿心慚感道:「原來我一向錯惱了他。他一篇都是為我的心,還是正經夫妻。」忍不住從粉壁前叉步走來,抱住月娘。月娘不防是他大雪裡來到,嚇了一跳,就要推開往屋裡走,被西門慶雙關抱住,說道:「我的姐姐!我西門慶死也不曉的,你一片好心,都是為我的。一向錯見了,丟冷了你的心,到今悔之晚矣。」月娘道:「大雪裡,你錯走了門兒了,敢不是這屋裡。我是那不賢良的淫婦,和你有甚情節?
\piZhang{以此知其假。} % 张旁批
\piWenlong{然则以毫拳加之乎?以笑脸迎之乎?月余不理之闷气,不望其倾吐,先生当更以假批之矣。} % 文龙旁批
那討為你的來?你平白又來理我怎的?咱兩個永世千年休要見面!」西門慶把月娘一手拖進房來。
\piLiyu{此正好德时,忽又插入好色,毕竟德不胜色,可叹,可叹。} % 绣像眉批
燈前看見他家常穿著:大紅潞綢對衿襖兒,軟黃裙子;頭上戴著貂鼠臥兔兒,金滿池嬌分心,
\piZhang{百忙里又写衣服。妙。又金满池娇,又照瓶儿。} % 张夹批
\piWenlong{幸而月娘尚有几分姿色,否则早被打入冷宫去矣。尚得一见,而使之闻之耶!} % 文旁批
越顯出他:

粉妝玉琢銀盆臉,蟬髻鴉鬟楚岫雲。

那西門慶如何不愛?連忙與月娘深深作了個揖,說道:「我西門慶一時昏昧,不聽你之良言,辜負你之好意。正是有眼不識荊山玉,拿著頑石一樣看。過後方知君子,千萬饒恕我則個。」
\piLiyu{有得。} % 绣像眉批
月娘道:「我又不是你那心上的人兒,凡是投不著你的機會,有甚良言勸你?隨我在這屋裡自生自活,你休要理他。我這屋裡也難安放你,趁早與我出去,我不著丫頭攆你。」
\piZhang{以此愈知其假,丑绝。不堪读下。} % 张旁批
\piZhang{岂敬夫之言。} % 张夹批
西門慶道:「我今日平白惹一肚子氣,大雪裡來家,逕來告訴你。」
\piZhang{岂敬夫之言。} % 张夹批
\piWenlong{诚非敬夫之言。试问:西门庆可是受人敬的东西?金莲之呼叱,直以儿呼之,亦不见其不受。此等人见之多矣。真有一句正经话,不能与之说者,必辱骂之而后可。或云;为之妻则不可。然月娘不过抢白,并未破口,曰:“我不着丫头撵你,望着管你的人去说。”酸味则有之,并非什么大不敬。闺门之内,更有甚于画眉者,买卖人之老婆,能如是亦罢了。先生之责人,无乃太苛乎 ?} % 文龙旁批
月娘道:「惹氣不惹氣,休對我說。我不管你,望著管你的人去說。」西門慶見月娘臉兒不瞧,就折疊腿裝矮子,跪在地下,殺雞扯脖,口裡姐姐長,姐姐短。月娘看不上,說道:「你真個恁涎臉涎皮的!我叫丫頭進來。」 一面叫小玉。那西門慶見小玉進來,連忙立起來,無計支出他去,說道:「外邊下雪了,一張香桌兒還不收進來?」小玉道:「香桌兒頭裡已收進來了。」月娘忍不住笑道:「沒羞的貨,
\piLiyu{弄一笑作收头,何等风韵。} % 绣像眉批
丫頭跟前也調個謊兒。」小玉出去,那西門慶又跪下央及。月娘道:「不看世人面上,一百年不理才好。」
\piWenlong{六人之中,到底月娘有点骨气,此后来所以能守而不嫁也。设身处地,略短取长,度理原情,因人论境,始可以评人。批者深恶吴氏,实不解其故,或者其夫妇间,别有一番景况。} % 文龙旁批
說畢,方才和他坐在一處,教玉簫捧茶與他吃。西門慶因他今日常家茶會,散後同邀伯爵到李家如何嚷鬧,告訴一遍:「如今賭了誓,再不踏院門了。」月娘道:「你踹不踹,不在於我。你拿響金白銀包著他,你不去,可知他另接了別個漢子?養漢老婆的營生,你拴住他身,拴不住他心。你長拿封皮封著他也怎的?」西門慶道:「你說的是。」於是打發丫鬟出去,脫衣上床,要與月娘求歡。月娘道:「教你上炕就撈食兒吃,今日只容你在我床上就夠了,要思想別的事,卻不能夠。」西門慶把那話露將出來,向月娘戲道: 「都是你氣的他,中風不語了。」月娘道:「怎的中風不語?」西門慶道:「他既不中風不語,如何大睜著眼就說不出話來?」月娘罵道:「好個汗邪的貨,教我有半個眼兒看的上你!」西門慶不由分說,把月娘兩隻白生生腿扛在肩膀上,那話插入牝中,一任其鶯恣蝶采,殢雨尤雲,未肯即休。正是得多少:

海棠枝上鶯梭急,翡翠梁間燕語頻。

不覺到靈犀一點,美愛無加,麝蘭半吐,脂香滿唇。西門慶情極,低聲求月娘叫達達;月娘亦低聲睥幃睨枕,態有餘妍,口呼親親不絕。是夜,兩人雨意雲情,並頭交頸而睡。正是:

亂鬢雙橫興已饒,情濃猶復厭通宵。
晚來獨向妝台立,淡淡春山不用描。

當夜夫妻交歡不題。卻表次日清晨,孟玉樓走到潘金蓮房中,未曾進門,先叫道:「六丫頭,起來了不曾?」春梅道:「俺娘才起來梳頭哩。三娘進屋裡坐。」玉樓進來,只見金蓮正在梳台前整掠香雲。因說道:「我有椿事兒來告訴你,你知道不知?」金蓮道:「我在這背哈喇子,誰曉的!」因問:「什麼事?」玉樓道:「他爹昨夜二更來家,走到上房裡,和吳家的好了,在他房裡歇了一夜。」金蓮道:「俺們何等勸著,他說一百年二百年,又怎的平白浪著,自家又好了?又沒人勸他!」玉樓道:「今早我才知道。俺大丫頭蘭香,在廚房內聽見小廝們說,昨日他爹同應二在院裡李桂兒家吃酒,看出淫婦的什麼破綻,把淫婦門窗戶壁都打了。大雪裡著惱來家,進儀門,看見上房燒夜香,想必聽見些什麼話兒,兩個才到一搭哩。磣死了。像他這等就沒的話說。若是別人,又不知怎的說浪!」金蓮接說道: 「早是與人家做大老婆,還不知怎樣久慣牢成!一個燒夜香,只該默默禱祝,誰家一徑倡揚,使漢子知道了。又沒人勸,自家暗裡又和漢子好了。硬到底才好,乾淨假撇清!」玉樓道:「也不是假撇清,他有心也要和,只是不好說出來的。他說他是大老婆不下氣,到叫俺們做分上,怕俺們久後玷言玷語說他,敢說你兩口子話差,也虧俺們說和。如今你我休教他買了乖兒去。你快梳了頭,過去和李瓶兒說去。咱兩個每人出五錢銀子,叫李瓶兒拿出一兩來,原為他的事起。今日安排一席酒,一者與他兩個把一杯,二者當家兒只當賞雪,耍戲一日,有何不可?」金蓮道:「說的是。不知他爹今日有勾當沒有?」玉樓道:「大雪裡有甚勾當?我來時兩口子還不見動靜,上房門兒才開,小玉拿水進去了。」這金蓮慌忙梳畢頭,和玉樓同過李瓶兒這邊來。李瓶兒還睡著在床上,迎春說:「三娘、五娘來了。」玉樓、金蓮進來,說道:「李大姐,好自在。這咱時懶龍才伸腰兒。」金蓮說舒進手去被窩裡,摸見薰被的銀香球兒,道:「李大姐生了蛋了。」就掀開被,見他一身白肉。那李瓶兒連忙穿衣不迭。玉樓道:「五姐,休鬼混他。李大姐,你快起來,俺們有椿事來對你說。如此這般,他爹昨日和大姐姐好了,咱每人五錢銀子,你便多出些兒,當初因為你起來。今日大雪裡,只當賞雪,咱安排一席酒兒,請他爹和大姐姐坐坐兒,好不好?」李瓶兒道:「隨姐姐教我出多少,奴出便了。」金蓮道: 「你將就只出一兩兒罷。你秤出來,俺好往後邊問李嬌兒、孫雪娥要去。」這李瓶兒一面穿衣纏腳,叫迎春開箱子,拿出銀子。拿了一塊,金蓮上等子秤,重一兩二錢五分。玉樓叫金蓮伴著李瓶兒梳頭:「等我往後邊問李嬌兒和孫雪娥要銀子去。」金蓮看著李瓶兒梳頭洗面,約一個時辰,只見玉樓從後邊來說道:「我早知也不幹這營生。大家的事,像白要他的。小淫婦說:『我是沒時運的人,漢子再不進我房裡來,我那討銀子?』求了半日,只拿出這根銀簪子來,你秤秤重多少?」金蓮取過等子來秤,只重三錢七分。因問:「李嬌兒怎的?」玉樓道:「李嬌兒初時只說沒有,『雖是錢日逐打我手裡使,都是叩數的。使多少交多少,那裡有富餘錢? 』我說:『你當家還說沒錢,俺們那個是有的?六月日頭,沒打你門前過也怎的?大家的事,你不出罷!』教我使性子走了出來,他慌了,使丫頭叫我回去,才拿出這銀子與我。沒來由,教我恁惹氣剌剌的!」金蓮拿過李嬌兒銀子來秤了秤,只四錢八分。因罵道:「好個奸滑的淫婦!隨問怎的,綁著鬼也不與人家足數,好歹短幾分。」玉樓道:「只許他家拿黃捍等子秤人的。人問他要,只象打骨禿出來一般,不知教人罵了多少!」一面連玉樓、金蓮共湊了三兩一錢;一面使繡春叫了玳安來。金蓮先問他:「你昨日跟了你爹去,在李家為什麼著了惱來?」玳安悉把在常家會茶散的早,邀應二爹和謝爹同到李家,他鴇子回說不在家,往五姨媽家做生日去了。「不想落後爹淨手,到後邊親看見粉頭和一個蠻子吃酒,爹就惱了。不由分說,叫俺眾人把淫婦家門窗戶壁盡力打了一頓,只要把蠻子、粉頭墩鎖在門上。多虧應二爹眾人再三勸住。爹使性騎馬回家,在路上發狠,到明日還要擺佈淫婦哩。」金蓮道:「賊淫婦!我只道蜜罐兒長年拿的牢牢的,如何今日也打了?」又問玳安:「你爹真個恁說來?」玳安道:「莫是小的敢哄娘!」金蓮道:「賊囚根子,他不揪不採,也是你爹的婊子,許你罵他?想著迎頭兒我們使著你,只推不得閒, 『爹使我往桂姨家送銀子去哩!』叫的桂姨那甜!如今他敗落了來,你主子惱了,連你也叫他淫婦來了!看我明日對你爹說不說。」玳安道:「耶樂!五娘這回日頭打西出來,從新又護起他家來了!莫不爹不在路上罵他淫婦,小的敢罵他?」金蓮道:「許你爹罵他罷了,原來也許你罵他?」玳安道:「早知五娘麻犯小的,小的也不對五娘說。」玉樓便道:「小囚兒,你別要說嘴。這裡三兩一錢銀子,你快和來興兒替我買東西去。今日俺們請你爹和大娘賞雪。你將就少落我們些兒,我教你五娘不告你爹說罷。」玳安道:「娘使小的,小的敢落錢?」於是拿了銀子同來興兒買東西去了。

且說西門慶起來,正在上房梳洗。只見大雪裡,來興買了雞鵝嗄飯,逕往廚房裡去了。玳安又提了一壇金華酒進來。便問玉簫:「小廝的東西,是那裡的?」玉簫回道:「今日眾娘置酒,請爹娘賞雪。」西門慶道:「金華酒是那裡的?」玳安道:「是三娘與小的銀子買的。」西門慶道:「啊呀!家裡見放著酒,又去買!」吩咐玳安:「拿鑰匙,前邊廂房有雙料茉莉酒,提兩壇攙著這酒吃。」於是在後廳明間內,設錦帳圍屏,放下梅花暖簾,爐安獸炭,擺列酒席。不一時,整理停當。李嬌兒、孟玉樓、潘金蓮、李瓶兒來到,請西門慶、月娘出來。當下李嬌兒把盞,孟玉樓執壺,潘金蓮捧菜,李瓶兒陪跪,頭一鍾先遞了與西門慶。西門慶接酒在手,笑道:「我兒,多有起動,孝順我老人家常禮兒罷!」那潘金蓮嘴快,插口道:「好老氣的孩兒!誰這裡替你磕頭哩?俺們磕著你,你站著。羊角蔥靠南牆──越發老辣!若不是大姐姐帶攜你,俺們今日與你磕頭?」一面遞了西門慶,從新又滿滿斟了一盞,請月娘轉上,遞與月娘。月娘道:「你們也不和我說,誰知你們平白又費這個心。」玉樓笑道:「沒什麼。俺們胡亂置了杯水酒兒,大雪,與你老公婆兩個散悶而已。姐姐請坐,受俺們一禮兒。」月娘不肯,亦平還下禮去。玉樓道:「姐姐不坐,我們也不起來。」相讓了半日,月娘才受了半禮。金蓮戲道:「對姐姐說過,今日姐姐有俺們面上,寬恕了他。下次再無禮,衝撞了姐姐,俺們也不管了。」望西門慶說道:「你裝憨打勢,還在上首坐,還不快下來,與姐姐遞個鐘兒,陪不是哩!」西門慶又是笑。良久,遞畢,月娘轉下來,令玉簫執壺,亦斟酒與眾姊妹回酒。惟孫雪娥跪著接酒,其餘都平敘姊妹之情。

於是西門慶與月娘居上座,其餘李嬌兒、孟玉樓、潘金蓮、李瓶兒、孫雪娥並西門大姐,都兩邊打橫。金蓮便道:「李大姐,你也該梯己與大姐姐遞杯酒兒,當初因為你的事起來,你做了老林,怎麼還恁木木的!」那李瓶兒真個就就走下席來要遞酒。被西門慶攔住,說道:「你休聽那小淫婦兒,他哄你。已是遞過一遍酒罷了,遞幾遍兒?」那李瓶兒方不動了。當下春梅、迎春、玉簫、蘭香一般兒四個家樂,琵琶、箏、弦子、月琴,一面彈唱起來,唱了一套《南石榴花》「佳期重會」。西門慶聽了,便問:「誰叫他唱這一套詞來?」玉簫道:「是五娘吩咐唱來。」西門慶就看著潘金蓮說道:「你這小淫婦,單管胡枝扯葉的!」金蓮道:「誰教他唱他來?沒的又來纏我。」月娘便道:「怎的不請陳姐夫來坐坐?」一面使小廝前邊請去。不一時,敬濟來到,向席上都作了揖,就在大姐下邊坐了。月娘令小玉安放了鍾箸,閤家歡飲。西門慶把眼觀看簾前那雪,如撏綿扯絮,亂舞梨花,下的大了。端的好雪。但見:

初如柳絮,漸似鵝毛。唰唰似數蟹行沙上,紛紛如亂瓊堆砌間。但行動衣沾六出,只頃刻拂滿蜂鬢。襯瑤台,似玉龍翻甲繞空舞;飄粉額,如白鶴羽毛連地落。正是:

凍合玉樓寒起粟,光搖銀海燭生花。

吳月娘見雪下在粉壁間太湖石上甚厚。下席來,教小玉拿著茶罐,親自掃雪,烹江南鳳團雀舌牙茶與眾人吃。正是:

白玉壺中翻碧浪,紫金盃內噴清香。

正喫茶中間,只見玳安進來,說道:「李銘來了,在前邊伺候。」西門慶道:「教他進來。」不一時,李銘進來向眾人磕了頭,走在旁邊。西門慶問道:「你往那裡去來?來得正好。」李銘道:「小的沒往那裡去,北邊酒醋門劉公公那裡,教了些孩子,小的瞧了瞧。記掛著爹娘內姐兒們,還有幾段唱未合拍,來伺候。」西門慶就將手內吃的那一盞木樨茶,遞與他吃。說道:「你吃了休去,且唱一個我聽。」李銘道:「小的知道。」一面下邊吃了茶上來,把箏弦調定,頓開喉音,並足朝上,唱了一套《冬景‧絳都春》。唱畢,西門慶令李銘近前,賞酒與他吃,教小玉拿壺滿斟,傾在銀琺琅桃兒鍾內。那李銘跪在地下,滿飲三杯。西門慶又叫在桌上拿了四碟菜,用盤子托著與李銘。那李銘走到下邊吃了,用絹兒把嘴抹了,走到上邊,直豎豎的靠著隔子站立。西門慶因把昨日桂姐家之事,告訴一遍。李銘道: 「小的並不知道,一向也不過那邊去。想起來不干桂姐事,都是俺三媽干的營生。爹也別要惱他,等小的見他說他便了。」當日飲酒到一更時分,妻妾俱各歡樂。先是陳敬濟、大姐往前邊去了。落後酒闌,西門慶又賞李銘酒,打發出門,分咐:「你到那邊,休說今日在我這裡。」李銘道:「爹吩咐,小的知道。」西門慶令左右送他出門,於是妻妾各散。西門慶還在月娘上房歇了。有詩為證:

赤繩緣分莫疑猜,扊扅夫妻共此懷。
魚水相逢從此始,兩情願保百年諧。

卻說次日雪晴,應伯爵、謝希大受了李家燒鵝瓶酒,恐怕西門慶擺佈他家,逕來邀請西門慶進裡邊陪禮。月娘早晨梳妝畢,正和西門慶在房中吃餅,只見玳安來說: 「應二爹和謝爹來了。」西門慶放下餅,就要往前走。月娘道:「兩個勾使鬼,又不知來做什麼。你亦發吃了出去,教他外頭等著去。慌的恁沒命的一般往外走怎的?大雪裡又不知勾了那去?」西門慶道:「你叫小廝把餅拿到前邊,我和他兩個吃罷。」說著,起身往外來。月娘吩咐:「你和他吃了,別要信著又勾引的往那裡去了。今日孟三姐晚夕上壽哩。」西門慶道:「我知道。」於是與應、謝二人相見聲喏,說道:「哥昨日著惱家來了,俺們甚是怪說他家:『從前已往,在你家使錢費物,雖故一時不來,休要改了腔兒才好,許你家粉頭背地偷接蠻子?冤家路兒窄,又被他親眼看見,他怎的不惱!休說哥惱,俺們心裡也看不過!』盡力說了他娘兒幾句,他也甚是沒意思。今日早請了俺兩個到家,娘兒們哭哭啼啼跪著,恐怕你動意,置了一杯水酒兒,好歹請你進去陪個不是。」西門慶道:「我也不動意。我再也不進去了。」伯爵道:「哥惱有理。但說起來,也不干桂姐事。這個丁二官原先是他姐姐桂卿的孤老,也沒說要請桂姐。只因他父親貨船搭在他鄉里陳監生船上,才到了不多兩日。這陳監生號兩淮,乃是陳參政的兒子。丁二官拿了十兩銀子,在他家擺酒請陳監生。才送這銀子來,不想你我到了他家,就慌了,躲不及,把個蠻子藏在後邊,被你看見了。實告不曾和桂姐沾身。今日他娘兒們賭身發咒,磕頭禮拜,央俺二人好歹請哥到那裡,把這委屈情由也對哥表出,也把惱解了一半。」西門慶道:「我已是對房下賭誓,再也不去,又惱什麼?你上覆他家,到不消費心。我家中今日有些小事,委的不得去。」慌的二人一齊跪下,說道:「哥,什麼話!不爭你不去,顯的我們請不得哥去,沒些面情了。到那裡略坐坐兒就來也罷。」當下二人死告活央,說的西門慶肯了。不一時,放桌兒,留二人吃餅。須臾吃畢,令玳安取衣服去。月娘正和孟玉樓坐著,便問玳安:「你爹要往那去?」玳安道:「小的不知,爹只叫小的取衣服。」月娘罵道:「賊囚根子,你還瞞著我不說!今日你三娘上壽哩。你爹但來晚了,我只打你這個賊囚根子。」玳安道:「娘打小的,管小的甚事?」月娘道:「不知怎的,聽見他這老子每來,恰似奔命的一般,吃著飯,丟下飯碗,往外不迭。又不知勾引遊魂撞屍,撞到多咱才來!」家中置酒等候不題。

且說西門慶被兩個邀請到李家,又早堂中置了一席齊整酒餚,叫了兩個妓女彈唱。李桂姐與桂卿兩個打扮迎接。老虔婆出來,跪著陪禮。姐兒兩個遞酒。應伯爵、謝希大在旁打諢耍笑,向桂姐道:「還虧我把嘴頭上皮也磨了半邊去,請了你家漢子來。就連酒兒也不替我遞一杯兒,只遞你家漢子!剛才若他撅了不來,休說你哭瞎了你眼,唱門詞兒,到明日諸人不要你,只我好說話兒將就罷了。」桂姐罵道:「怪應花子,汗邪了你!我不好罵出來的。可可兒的我唱門詞兒來?」應伯爵道: 「你看賊小淫婦兒!念了經打和尚,他不來慌的那腔兒,這回就翅膀毛兒干了。你過來,且與我個嘴溫溫寒著。」於是不由分說,摟過脖子來就親了個嘴。桂姐笑道:「怪攮刀子的,看推撒了酒在爹身上。」伯爵道:「小淫婦兒,會喬張致的,這回就疼漢子。『看撒了爹身上酒!』叫你爹那甜。我是後娘養的?怎的不叫我一聲兒?」桂姐道:「我叫你是我的孩兒。」伯爵道:「你過來,我說個笑話兒你聽:一個螃蟹與田雞結為兄弟,賭跳過水溝兒去便是大哥。田雞幾跳,跳過去了。螃蟹方欲跳,撞遇兩個女子來汲水,用草繩兒把他拴住,打了水帶回家去。臨行忘記了,不將去。田雞見他不來,過來看他,說道:『你怎的就不過去了?』螃蟹說: 『我過的去,倒不吃兩個小淫婦捩的恁樣了!』」桂姐兩個聽了,一齊趕著打,把西門慶笑的要不的。

不說這裡調笑頑耍,且說家中吳月娘一者置酒回席,二者又是玉樓上壽,吳大妗子、楊姑娘並兩個姑子,都在上房裡坐的。看看等到日落時分,不見西門慶來家,急的月娘要不的。金蓮拉著李瓶兒,笑嘻嘻向月娘說道:「大姐姐,他這咱不來,俺們往門首瞧他瞧去。」月娘道:「耐煩瞧他怎的!」金蓮又拉玉樓說:「咱三個打伙兒走走去。」玉樓道:「我這裡聽大師父說笑話兒哩,等聽說了笑話兒咱去。」那金蓮方住了腳,圍著兩個姑子聽說笑話兒,因說道:「大師父,你有,快些說。」那王姑子坐在坑上,就說了一個。金蓮道:「這個不好。再說一個。」王姑子又道:「一家三個媳婦兒,與公公上壽。先是大媳婦遞酒說:『公公好像一員官。』公公云:『我如何象官?』媳婦云:『坐在上面,家中大小都怕你,如何不像官?』次該二媳婦上來遞酒,說:『公公象虎威皂隸。』公公曰:『我如何象虎威皂隸?』媳婦云:『你喝一聲,家中大小都吃一驚,怎不像皂隸?』公公道:『你說的我好!』該第三媳婦遞酒,上來說:『公公也不像官,也不像皂隸。』公公道:『卻像什麼?』媳婦道:『公公像個外郎!』公公道:『我如何像個外郎?』媳婦道:『不像外郎,如何六房裡都串到?』」把眾人都笑了。金蓮道:「好禿子!把俺們都說在裡頭。那個外郎敢恁大膽!」說罷,金蓮、玉樓、李瓶兒同來到前邊大門首,瞧西門慶。玉樓問道:「今日他爹大雪裡那裡去了?」金蓮道:「我猜他一定往院中李桂兒那淫婦家去了。」玉樓道:「打了一場,賭誓再不去,如何又去?咱每賭什麼?管情不在他家。」金蓮道:「李大姐做證見,你敢和我拍手麼?我說今日往他家去了。前日打了淫婦家,昨日李銘那忘八先來打探子兒。今日應二和姓謝的,大清早晨,勾使鬼勾了他去。我猜老虔婆和淫婦鋪謀定計叫了去,不知怎的撮弄,陪著不是,還要回爐復帳,不知涎纏到多咱時候。有個來的成來不成,大姐姐還只顧等著他!」玉樓道:「就不來,小廝也該來家回一聲兒。」正說著,只見賣瓜子的過來,兩個正在門首買瓜子兒,忽然西門慶從東來了,三個往後跑不迭。

西門慶在馬上,教玳安先頭裡走:「你瞧是誰在大門首?」玳安走了兩步,說道:「是三娘、五娘、六娘在門首買瓜子哩。」西門慶到家下馬,進入後邊儀門首。玉樓、李瓶兒先去上房報月娘去了。獨有金蓮藏在粉壁背後黑影裡。西門慶撞見,嚇了一跳,說道:「怪小淫婦兒,猛可唬我一跳!你們在門首做什麼來?」金蓮道: 「你還敢說哩。你在那裡?這時才來,教娘們只顧在門首等著你。」西門慶進房中,月娘安排酒餚,教玉簫執壺,大姐遞酒。先遞了西門慶,然後眾姊妹都遞了,安席坐下。春梅、迎春下邊彈唱,吃了一回,都收下去。從新擺上玉樓上壽的酒,並四十樣細巧各樣的菜碟兒上來。壺斟美醞,盞泛流霞。讓吳大妗子上坐。吃到起更時分,大妗子吃不多酒,歸後邊去了。止是吳月娘同眾人陪西門慶擲骰猜枚行令。輪到月娘跟前,月娘道:「既要我行令,照依牌譜上飲酒:一個牌兒名,兩個骨牌名,合《西廂》一句。」月娘先說:「六娘子醉楊妃,落了八珠環,游絲兒抓住荼蘼架。」不遇。該西門慶擲,說:「虞美人,見楚漢爭鋒,傷了正馬軍,只聽耳邊金鼓連天震。」果然是個正馬軍,吃了一杯。該李嬌兒,說:「水仙子,因二士入桃源,驚散了花開蝶滿枝,只做了落紅滿地胭脂冷。」不遇。次該金蓮擲,說道: 「鮑老兒,臨老入花叢,壞了三綱五常,問他個非奸做賊拿。」果然是三綱五常,吃了一杯。輪該李瓶兒擲,說:「端正好,搭梯望月,等到春分晝夜停,那時節隔牆兒險化做望夫山。」不遇。該孫雪娥,說:「麻郎兒,見群鴉打鳳,絆住了折足雁,好教我兩下裡做人難。」不遇。落後該玉樓完令,說:「念奴嬌,醉扶定四紅沉,拖著錦裙襴,得多少春風夜月銷金帳。」正擲了四紅沉。月娘滿令,叫小玉:「斟酒與你三娘吃。」說道:「你吃三大杯才好!今晚你該伴新郎宿歇。」因對李瓶兒、金蓮眾人說:「吃畢酒,咱送他兩個歸房去。」金蓮道:「姐姐嚴令,豈敢不依!」把玉樓羞的要不的。

少頃酒闌,月娘等相送西門慶到玉樓房首方回。玉樓讓眾人坐,都不坐。金蓮便戲玉樓道:「我兒,好好兒睡罷。你娘明日來看你,休要淘氣!」因向月娘道:「親家,孩兒小哩,看我面上,凡是擔待些兒罷。」玉樓道:「六丫頭,你老米醋,挨著做。我明日和你答話。」金蓮道:「我媒人婆上樓子──老娘好耐驚耐怕兒。」 於是和李瓶兒、西門大姐一路去了。剛走到儀門首,不想李瓶兒被地滑了一交。這金蓮遂怪喬叫起來道:「這個李大姐,只像個瞎子,行動一磨子就倒了。我搊你去,倒把我一隻腳踩在雪裡,把人的鞋兒也踹泥了!」月娘聽見,說道:「就是儀門首那堆子雪。我吩咐了小廝兩遍,賊奴才,白不肯抬,只當還滑倒了。」 因叫小玉:「你拿個燈籠送送五娘、六娘去。」西門慶在房裡向玉樓道:「你看賊小淫婦兒!他踹在泥裡把人絆了一交,他還說人踹泥了他的鞋,恰是那一個兒,就沒些嘴抹兒。恁一個小淫婦!昨日叫丫頭們平白唱『佳期重會』,我就猜是他幹的營生。」玉樓道:「『佳期重會』是怎的說?」西門慶道:「他說吳家的不是正經相會,是私下相會。恰似燒夜香,有心等著我一般。」玉樓道:「六姐他諸般曲兒到都知道,俺們卻不曉的。」西門慶道:「你不知,這淫婦單管咬群兒。」

不說西門慶在玉樓房中宿歇。單表潘金蓮、李瓶兒兩個走著說話,走到儀門,大姐便歸前邊廂房去了。小玉打著燈籠,送二人到花園內。金蓮已帶半酣,拉著李瓶兒道:「二娘,我今日有酒了,你好歹送到我房裡。」李瓶兒道:「姐姐,你不醉。」須臾,送到金蓮房內。打發小玉回後邊,留李瓶兒坐,喫茶。金蓮又道:「你說你那咱不得來,虧了誰?誰想今日咱姊妹在一個跳板兒上走,不知替你頂了多少瞎缸,教人背地好不說我!奴只行好心,自有天知道罷了。」李瓶兒道:「奴知道姐姐費心,恩當重報,不敢有忘。」金蓮道:「得你知道,好了。」不一時,春梅拿茶來吃了,李瓶兒告辭歸房。金蓮獨自歇宿,不在話下。正是:

空庭高樓月,非復三五圓。
何須照床裡,終是一人眠。



\end{showcontents}

