%# -*- coding: utf-8 -*-
%!TEX encoding = UTF-8 Unicode
%!TEX TS-program = xelatex
% vim:ts=4:sw=4
%
% 以上设定默认使用 XeLaTex 编译,并指定 Unicode 编码,供 TeXShop 自动识别

%第七十二回 
\chapter{潘金蓮摳打如意兒 王三官義拜西門慶}

\begin{showcontents}{}


詞曰:

掉臂疊肩情態,炎涼冷暖紛紜。興來閹豎長兒孫,石女須教有孕。
莫使一朝勢謝,親生不若他生。爹爹媽媽向何親?掇轉窟臀不認。

話說西門慶與何千戶在路不題。單表吳月娘在家,因西門慶上東京,見家中婦女多,恐惹是非,吩咐平安無事關好大門,後邊儀門夜夜上鎖。姊妹每都不出來,各自在房做針指。若敬濟要往後樓上尋衣裳,月娘必使春鴻或來安兒跟出跟入。常時查門戶,凡事都嚴緊了。這潘金蓮因此不得和敬濟勾搭。只賴奶子如意備了舌,逐日只和如意兒合氣。

一日,月娘打點出西門慶許多衣服、汗衫、小衣,教如意兒同韓嫂兒漿洗。不想這邊春梅也洗衣裳,使秋菊問他借棒槌。這如意兒正與迎春捶衣,不與他,說道: 「前日你拿了個棒槌,使著罷了,又來要!趁韓嫂在這裡,要替爹捶褲子和汗衫兒哩。」那秋菊使性子走來對春梅說:「平白教我借,他又不與。迎春倒說拿去,如意兒攔住了不肯。」春梅道:「耶嚛,耶嚛!怎的這等生分?大白日里借不出個乾燈盞來。借個棒槌使使兒,就不肯與將來,替娘洗了這裹腳,教拿甚麼捶?秋菊,你往後邊問他們借來使使罷。」這潘金蓮正在房中炕上裹腳,忽然聽得,又因懷著仇恨,尋不著頭由兒,便罵道:「賊淫婦怎的不與?你自家問他要去,不與,罵那淫婦不妨事。」這春梅一衝性子,就一陣風走來李瓶兒那邊,說道:「那個是外人也怎的?棒槌借使使就不與。如今這屋裡又鑽出個當家的來了!」如意兒道:「耶嚛,耶嚛!放著棒槌拿去使不是,誰在這裡把住?就怒說起來。大娘吩咐,趁韓媽在這裡,替爹漿出這汗衫子和綿綢褲子來。秋菊來要,我說待我把你爹這衣服捶兩下兒著,就架上許多誑,說不與來?早是迎春姐聽著。」不想潘金蓮隨即跟了來,便罵道:「你這個老婆不要說嘴!死了你家主子,如今這屋裡就是你?你爹身上衣服不著你恁個人兒拴束,誰應的上他那心!俺這些老婆死絕了,教你替他漿洗衣服?你拿這個法兒降伏俺每,我好耐驚耐怕兒!」如意兒道:「五娘怎的說這話?大娘不吩咐,俺們好掉攬替爹整理的?」金蓮道:「賊歪剌骨,雌漢的淫婦,還強說甚麼嘴!半夜替爹遞茶兒扶被兒是誰來?討披襖兒穿是誰來?你背地乾的那繭兒,你說我不知道?就偷出肚子來,我也不怕!」如意道:「正經有孩子還死了哩,俺每到的那些兒!」這金蓮不聽便罷,聽了心頭火起,粉面通紅,走向前一把手把老婆頭髮扯住,只用手摳他腹。虧得韓嫂兒向前勸開了。金蓮罵道:「沒廉恥的淫婦,嘲漢的淫婦!俺每這裡還閑的聲喚,你來雌漢子,你在這屋裡是甚麼人?你就是來旺兒媳婦子從新又出世來了,我也不怕你!」那如意兒一壁哭著,一壁輓頭髮,說道:「俺每後來,也不知甚麼來旺兒媳婦子,只知在爹家做奶子。」金蓮道: 「你做奶子,行你那奶子的事,怎的在屋裡狐假虎威,成起精兒來?老娘成年拿雁,教你弄鬼兒去了!」

著謊兒成不的!他不攛瞞著,你家肯要他!想著一來時,餓答的個臉,黃皮寡瘦的,乞乞縮縮那個腔兒!吃了這二年飽飯,就生事兒,雌起漢子來了。你如今不禁下他來,到明日又教他上頭上臉的。一時捅出個孩子,當誰的?」玉樓笑道:「你這六丫頭,到且是有權屬。」說畢,坐了一回,兩個往後邊下棋去了。正是:

三光有影遺誰系?萬事無根只自生。

話休饒舌,有日後晌時分,西門慶來到清河縣。吩咐賁四、王經跟行李先往家去,他便送何千戶到衙門中,看著收拾打掃公廨乾凈住下,方纔騎馬來家。進入後廳,吳月娘接著,舀水凈面畢,就令丫鬟院子內放桌兒,滿爐焚香,對天地位下告許願心。月娘便問:「你為什麼許願心?」西門慶道:「休說起,我拾得性命來家。昨日十一月二十三日,剛過黃河,行到沂水縣八角鎮上,遭遇大風,沙石迷目,通行不得。天色又晚,百裡不見人,眾人都慌了。況馱垛又多,誠恐鑽出個賊來怎了?比及投到個古寺中,和尚又窮,夜晚連燈火也沒個兒,只吃些豆粥兒就過了一夜。次日風住,方纔起身,這場苦比前日更苦十分。前日雖熱,天還好些。這遭又是寒冷天氣,又耽許多驚怕。幸得平地還罷了,若在黃河遭此風浪怎了?我在路上就許了願心,到臘月初一日,宰豬羊祭賽天地。」月娘又問:「你頭裡怎不來家,卻往衙門裡做甚麼?」西門慶道:「夏龍溪已升做指揮直駕,不得來了。新升是匠作監何太監侄兒何千戶──名永壽,貼刑,不上二十歲,捏出水兒來的一個小後生,任事兒不知道。他太監再三央及我,凡事看顧教導他。我不送到衙門裡安頓他個住處,他知道甚麼?他如今一千二百兩銀子──也是我作成他──要了夏龍溪那房子,直待夏家搬取了家小去,他的家眷才搬來。前日夏大人不知什麼人走了風與他,他又使了銀子,央當朝林真人分上,對堂上朱太尉說,情願以指揮職銜再要提刑三年。朱太尉來對老爺說,把老爺難的要不得。若不是翟親家在中間竭力維持,把我撐在空地里去了。去時親家好不怪我,說我幹事不謹密。不知是什麼人對他說來。」月娘道:「不是我說,你做事有些三慌子火燎腿樣,有不的些事兒,告這個說一場,告那個說一場,恰似逞強賣富的。正是有心算無心,不備怎提備?人家悄悄乾的事兒停停妥妥,你還不知道哩!」西門慶又說:「夏大人臨來,再三央我早晚看顧看顧他家裡,容日你買分禮兒走走去。」月娘道:「他娘子出月初二日生日,就一事兒去罷。你今後把這狂樣來改了。常言道:『逢人且說三分清,未可全拋一片心。』老婆還有個裡外心兒,休說世人。」

正說著,只見玳安來說:「賁四問爹,要往夏大人家說去不去?」西門慶道:「你教他吃了飯去。」玳安應諾去了。李嬌兒、孟玉樓、孫雪娥、潘金蓮、大姐都來參見道萬福,問話兒,陪坐的。西門慶又想起前番往東京回來,還有李瓶兒在,一面走到他房內,與他靈床作揖,因落了幾點眼淚。如意兒、迎春、繡春都向前磕頭。月娘隨即使小玉請在後邊,擺飯吃了,一面吩咐拿出四兩銀子,賞跟隨小馬兒上的人,拿帖兒回謝周守備去了。又叫來興兒宰了半口豬、半腔羊、四十斤白麵、一包白米、一壇酒、兩腿火熏、兩隻鵝、十隻雞,又並許多油鹽醬醋之類,與何千戶送下程。又叫了一名廚役在那裡答應。

正在廳上打點,忽琴童兒進來說道:「溫師父和應二爹來望。」西門慶連忙請進溫秀才、伯爵來。二人連連作揖,道其風霜辛苦。西門慶亦道:「蒙二公早晚看家。」伯爵道:「我早起來時,忽聽房上喜鵲喳喳的叫。俺房下就先說:『只怕大官人來家了,你還不快走了瞧瞧去?』我便說:『哥從十二日起身,到今還未上半個月,怎能來得快?』房下說:『來不來,你看看去!』教我穿衣裳到宅里,不想哥真個來家了。恭喜恭喜!」因見許多下飯酒米裝在廳臺上,便問道:「送誰家的?」西門慶道:「新同僚何大人,一路同來,家小還未到。今在衙門中權住,送份下程與他。又發柬明日請他吃接風酒,再沒人,請二位與吳大舅奉陪。」伯爵道:「又一件:吳大舅與哥是官,溫老先生戴著方巾,我一個小帽兒怎陪得他坐!不知把我當甚麼人兒看,我惹他不笑話?」西門慶笑道:「這等把我買的緞子忠靖巾借與你戴著,等他問你,只說是我的大兒子,好不好?」說畢,眾人笑了。伯爵道:「說正經話,我頭八寸三,又戴不得你的。」溫秀才道:「學生也是八寸三分,倒將學生方巾與老翁戴戴何如?」西門慶道:「老先生不要借與他,他到明日借慣了,往禮部當官身去,又來纏你。」溫秀才笑道:「老先生好說,連我也扯下水去了。」少頃,拿上茶來吃了。溫秀才問:「夏公已是京任,不來了?」西門慶道:「他已做堂尊了,直掌鹵簿,穿麟服,使藤棍,如此華任,又來做甚麼!」須臾,看寫了帖子,抬下程出門,教玳安送去了。西門慶就拉溫秀才、伯爵到廂房內暖炕上坐去了。又使琴童往院里叫吳惠、鄭春、邵奉、左順四名小優兒明日早來伺候。

不一時,放桌兒陪二人吃酒。西門慶吩咐:「再取雙鐘箸兒,請你姐夫來坐坐。」良久,陳敬濟走來,作揖,打橫坐下。四人圍爐把酒來斟,因說起一路上受驚的話。伯爵道:「哥,你的心好,一福能壓百禍,就有小人,一時自然都消散了。」溫秀才道:「善人為邦百年,亦可以勝殘去殺。休道老先生為王事驅馳,上天也不肯有傷善類。」西門慶因問:「家中沒甚事?」敬濟道:「家中無事。只是工部安老爹那裡差人來問了兩遭,昨日還來問,我回說還沒來家哩。」

正說著,忽有平安來報:「衙門令史和眾節級來稟事。」西門慶即到廳上站立,令他進見。二人跪下:「請問老爹幾時上任?官司公用銀兩動支多少?」西門慶道: 「你們只照舊時整理就是了。」令史道:「去年只老爹一位到任,如今老爹轉正,何老爹新到任,兩事並舉,比舊不同。」西門慶道:「既是如此,添十兩銀子與他就是了。」二人應喏下去。西門慶又叫回來吩咐:「上任日期,你還問何老爹擇幾時。」二人道:「何老爹擇定二十六日。」西門慶道:「既如此,你每伺候就是了。」二人去了。就是喬大人來拜望道喜。西門慶留坐不肯,吃茶起身去了。西門慶進來,陪二人飲至掌燈方散。西門慶往月娘房裡歇了一宿。

到次日,家中置酒,與何千戶接風。文嫂又早打聽得西門慶來家,對王三官說了,具個柬帖兒來請。西門慶這裡買了一副豕蹄、兩尾鮮魚、兩隻燒鴨、一壇南酒,差玳安送去,與太太補生日之禮。他那裡賞了玳安三錢銀子,不在話下。正廳上設下酒,錦屏耀目,桌椅鮮明。吳大舅、應伯爵、溫秀才都來的早,西門慶陪坐吃茶,使人邀請何千戶。不一時,小優兒上來磕頭。伯爵便問:「哥,今日怎的不叫李銘?」西門慶道:「他不來我家來,我沒的請他去!」

正說話,只見平安忙拿帖兒稟說:「帥府周爺來拜,下馬了。」吳大舅、溫秀才、應伯爵都躲在西廂房內。西門慶冠帶出來,迎至廳上,敘禮畢,道及轉升恭喜之事。西門慶又謝他人馬。於是分賓主而坐。周守備問京中見朝之事,西門慶一一說了。周守備道:「龍溪不來,一定差人來取家小上京去。」西門慶道:「就取也待出月。如今何長官且在衙門權住著哩。夏公的房子與了他住,也是我替他主張的。」守備道:「這等更妙。」因見堂中擺設桌席,問道:「今日所延甚客?」西門慶道:「聊具一酌,與何大人接風。同僚之間,不好意思。」二人吃了茶,周守備起身,說道:「容日合衛列位,與二公奉賀。」西門慶道:「豈敢動勞,多承先施。」作揖出門,上馬而去。西門慶回來,脫了衣服,又陪三人在書房中擺飯。何千戶到午後方來,吳大舅等各相見敘禮畢,各敘寒溫。茶湯換罷,各寬衣服。何千戶見西門慶家道相稱,酒筵齊整。四個小優銀箏象板,玉阮琵琶,遞酒上坐。直飲至起更時分,何千戶方起身往衙門中去了。吳大舅、應伯爵、溫秀才也辭回去了。

西門慶打發小優兒出門,吩咐收了傢伙,就往前邊金蓮房中來。婦人在房內濃施朱粉,復整新妝,薰香澡牝,正盼西門慶進他房來,滿面笑容,向前替他脫衣解帶,連忙叫春梅點茶與他吃了,打發上床歇宿。端的被窩中相挨素體,枕席上緊貼酥胸,婦人雲雨之際,百媚俱生。西門慶抽拽之後,靈犀已透,睡不著,枕上把離言深講。交接後,淫情未足,又從下替他品簫。這婦人只要拴西門慶之心,又況拋離了半月在家,久曠幽懷,淫情似火,得到身,恨不得鑽入他腹中。將那話品弄了一夜,再不離口。西門慶要下床溺尿,婦人還不放,說道:「我的親親,你有多少尿,溺在奴口裡,替你咽了罷,省的冷呵呵的,熱身子下去凍著,倒值了多的。」西門慶聽了,越發歡喜無已,叫道:「乖乖兒,誰似你這般疼我!」於是真個溺在婦人口內。婦人用口接著,慢慢一口一口都咽了。西門慶問道:「好吃不好吃?」金蓮道:「略有些鹹味兒。你有香茶與我些壓壓。」西門慶道:「香茶在我白綾襖內,你自家拿。」這婦人向床頭拉過他袖子來,掏摸了幾個放在口內,才罷。正是:

侍臣不及相如渴,特賜金莖露一杯。

看官聽說:大抵妾婦之道,鼓惑其夫,無所不至,雖屈身忍辱,殆不為恥。若夫正室之妻,光明正大,豈肯為也!是夜,西門慶與婦人盤桓無度。

次早往衙門中與何千戶上任,吃公宴酒,兩院樂工動樂承應。午後才回家,排軍隨即抬了桌席來。王三官那裡又差人早來邀請。西門慶才收拾出來,左右來報:「工部安老爹來拜。」慌的西門慶整衣出來迎接。安郎中食寺丞的俸,系金鑲帶,穿白鷳補子,跟著許多官吏,滿面笑容,相攜到廳敘禮,彼此道及恭賀,分賓主坐下。安郎中道:「學生差人來問幾次,說四泉還未回。」西門慶道:「正是。京中要等見朝引奏,才起身回來。」須臾,茶湯吃罷,安郎中方說:「學生敬來有一事不當奉瀆:今有九江太府蔡少塘,乃是蔡老先生第九公子,來上京朝覲,前日有書來,早晚便到。學生與宋松泉、錢雲野、黃泰宇四人作東,欲借府上設席請他,未知允否?」西門慶道:「老先生尊命,豈敢有違。約定幾時?」安郎中道:「在二十七日。明日學生送分子過來,煩盛使一辦,足見厚愛矣。」說畢,又上了一道茶,作辭,起身上馬,喝道而去。

西門慶即出門,往王招宣府中來赴席。到門首,先投了拜帖。王三官連忙出來迎接,至廳上敘禮。大廳正面欽賜牌額,金字題曰「世忠堂」,兩邊門對寫著「喬木風霜古,山河(石帶)礪新」。王三官與西門慶行畢禮,尊西門慶上坐,他便傍設一椅相陪。須臾拿上茶來,交手遞了茶,左右收了去。彼此扳了些說話,然後安排酒筵遞酒。原來王三官叫了兩名小優兒彈唱。西門慶道:「請出老太太拜見拜見。」慌的王三官令左右後邊說。少頃,出來說道:「請老爹後邊見罷。」王三官讓西門慶進內。西門慶道:「賢契,你先導引。」於是逕入中堂。林氏又早戴著滿頭珠翠,身穿大紅通袖袍兒,腰系金鑲碧玉帶,下著玄錦百花裙,搽抹的如銀人也一般。西門慶一面施禮:「請太太轉上。」林氏道:「大人是客,請轉上。」讓了半日,兩個人平磕頭,林氏道:「小兒不識好歹,前日沖瀆大人。蒙大人又處斷了那些人,知感不盡。今日備了一杯水酒,請大人過來,老身磕個頭兒謝謝。如何又蒙大人賜將禮來?使我老身卻之不恭,受之有愧。」西門慶道:「豈敢。學生因為公事往東京去了,誤了與老太太拜壽。些須薄禮,胡亂送與老太太賞人。」因見文嫂兒在旁,便道:「老文,你取副盞兒來,等我與太太遞一杯壽酒。」一面呼玳安上來。原來西門慶氈包內,預備著一套遍地金時樣衣服,放在盤內獻上。林氏一見,金彩奪目,滿心歡喜。文嫂隨即捧上金盞銀台。王三官便要叫小優拿樂器進來彈唱。林氏道:「你叫他進來做甚麼?在外答應罷了。」當下,西門慶把盞畢,林氏也回奉了一盞與西門慶謝了。然後王三官與西門慶遞酒,西門慶才待還下禮去,林氏便道:「大人請起,受他一禮兒。」西門慶道:「不敢,豈有此禮?」林氏道:「好大人,怎這般說!你恁大職級,做不起他個父親!小兒自幼失學,不曾跟著好人。若是大人肯垂愛,凡事指教他為個好人,今日我跟前,就教他拜大人做了義父。但有不是處,一任大人教誨,老身並不護短。」西門慶道:「老太太雖故說得是,但令郎賢契,賦性也聰明,如今年少,為小試行道之端,往後自然心地開闊,改過遷善。老太太倒不必介意。」當下教西門慶轉上,王三官把盞,遞了三鐘酒,受其四拜之禮。遞畢,西門慶亦轉下與林氏作揖謝禮,林氏笑吟吟還了萬福。自此以後,王三官見著西門慶以父稱之。正是:常將壓善欺良意,權作尤雲殢雨心。復有詩以嘆之:

從來男女不通酬,賣俏營姦真可羞。
三官不解其中意,饒貼親娘還磕頭。

遞畢酒,林氏吩咐王三官:「請大人前邊坐,寬衣服。」玳安拿忠靖巾來換了。不一時,安席坐下。小優彈唱起來,廚役上來割道,玳安拿賞賜伺候。當下食割五道,歌吟二套,秉燭上來,西門慶起身告辭。王三官再三款留,又邀到他書院中。獨獨的三間小軒裡面,花竹掩映,文物瀟灑。正面懸著一個金粉箋扁,曰「三泉詩舫」,四壁掛四軸古畫。西門慶便問:「三泉是何人?」王三官只顧隱避,不敢回答。半日才說:「是兒子的賤號。」西門慶便一聲兒沒言語。抬過高壺來,又投壺飲酒。四個小優兒在旁彈唱。林氏後邊只顧打發添換菜蔬果碟兒上來。

,接抱其股,竭力扇磞的連聲響亮。一面令婦人呼叫大東大西,問道:「你怕我不怕?再敢管著!」婦人道:「怪奴才,不管著你好上天也!我曉的你也丟不開這淫婦,到明日,問了我方許你那邊去。他若問你要東西,須對我說,只不許你悄悄偷與他。若不依,我打聽出來,看我嚷不嚷!我就擯兌了這淫婦,也不差甚麼兒。又相李瓶兒來頭,教你哄了,險些不把我打到贅字號去。你這爛桃行貨子,豆芽萊──有甚正條捆兒也怎的?老娘如今也賊了些兒了。」說的西門慶笑了。當下兩個殢雨尤雲,纏到三更方歇。正是:

帶雨籠煙世所稀,妖嬈身勢似難支。
終宵故把芳心訴,留得東風不放歸。

兩個並頭交股睡到天明,婦人淫情未足,便不住手捏弄那話,登時把麈柄捏弄起來,叫道:「親達達,我一心要你身上睡睡。」一面爬伏在西門慶身上倒澆燭,接著他脖子只顧揉搓,教西門慶兩手扳住他腰,扳的緊緊的,他便在上極力抽提,一面爬伏在他身上揉一回,那話漸沒至根,餘者被托子所阻,不能入。婦人便道:「我的達達,等我白日里替你作一條白綾帶子,你把和尚與你的那末子藥裝些在裡面,我再墜上兩根長帶兒。等睡時,你扎他在根子上,卻拿這兩根帶扎拴後邊腰裡,拴的緊緊的,又柔軟,又得全放進,卻不強如這托子硬硬的,格的人疼?」西門慶道:「我的兒,你做下,藥在磁盒兒內,你自家裝上就是了。」婦人道:「你黑夜好歹來,咱兩個試試看好不好?」於是,兩個玩耍一番。

只見玳安拿帖兒進來,問春梅:「爹起身不曾?安老爹差人送分資來了。又抬了兩壇酒、四盆花樹進來。」春梅道:「爹還沒起身,教他等等兒。」玳安道:「他好少近路兒,還要趕新河口閘上回話哩。」不想西門慶在房中聽見,隔窗叫玳安問了話,拿帖兒進去,拆開看,上寫道:

奉去分資四封,共八兩。惟少塘桌席,餘者散酌而已。仰冀從者留神,足見厚愛之至。外具時花四盆,以供清玩;浙酒二樽,少助待客之需。希莞納,幸甚。

西門慶看了,一面起身,且不梳頭,戴著氈巾,穿著絨氅衣走出廳上,令安老爹人進見。遞上分資。西門慶見四盆花草:一盆紅梅、一盆白梅、一盆茉莉、一盆辛夷,兩壇南酒,滿心歡喜。連忙收了。發了回帖,賞了來人五錢銀子,因問:「老爹們明日多咱時分來?用戲子不用?」來人道:「都早來。戲子用海鹽的。」說畢,打發去了。西門慶叫左右把花草抬放藏春塢書房中擺放,一面使玳安叫戲子去,一面兌銀子與來安兒買辦。那日又是孟玉樓上壽,院中叫小優兒晚夕彈唱。


按下一頭。卻說應伯爵在家,拿了五個箋帖,教應保捧著盒兒,往西門慶對過房子內央溫秀才寫請書。要請西門慶五位夫人,二十八日家中做滿月。\piZhang{張夾批: 頭緒紛紜,卻一絲不亂。} % 張夾批
剛出門轉過街口,只見後邊一人高叫道:「二爹請回來!」伯爵扭頭回看是李銘,立住了腳。李銘走到跟前,問道:「二爹往那裏去?」\piZhang{又照管一處,文章真千百化身也。} % 張夾批
伯爵道:「我到溫師父那裏有些事兒去。」李銘道:「到家中還有句話兒說。」只見後邊一個閑漢,掇著盒兒,伯爵不免又到家堂屋內。李銘連忙磕了個頭,把盒兒掇進來放下,揭開卻是燒鴨二只、老酒二瓶,說道:「小人沒甚,這些微物兒孝順二爹賞人。小的有句話逕來央及二爹。」一面跪在地下不起來。伯爵一把手拉起來,說道:「傻孩兒,你有話只管說,怎的買禮來?」李銘道:「小的從小兒在爹宅內,答應這幾年,如今爹到看顧別人,不用小的了。就是桂姐那邊的事,各門各戶,小的實不知道。如今爹因怪那邊,連小的也怪了。這負屈銜冤,沒處伸訴,逕來告二爹。二爹到宅內見爹,千萬替小的加句美言兒說說。就是桂姐有些一差半錯,不幹小的事。爹動意惱小的不打緊,同行中人越發欺負小的了。」伯爵道:「你原來這些時沒往宅內答應去。」\piZhang{活賊。} % 張夾批
李銘道:「小的沒曾去。」伯爵道:「嗔道昨日擺酒與何老爹接風,叫了吳惠、鄭春、邵奉、左順在那裏答應,我說怎的不見你。我問你爹,你爹說:『他沒來,我沒的請他去!』傻孩兒,你還不走跳些兒還好?你與誰賭氣?」李銘道:「爹宅內不呼喚,小的怎的好去?前日他每四個在那裏答應,今日三娘上壽,安官兒早晨又叫了兩名去了;明日老爹擺酒,又是他們四個。倒沒小的,小的心裏怎麼有個不急的!只望二爹替小的說個明白,小的還來與二爹磕頭。」伯爵道:「我沒有個不替你說的。我從前已往不知替人完美了多少勾當,\piZhang{張夾批: 一總。} % 張夾批
你央及我這些事兒,我不替你說?你依著我,把這禮兒你還拿回去。你是那裏錢兒,我受你的!你如今就跟了我去,等我慢慢和你爹說。」李銘道:「二爹不收此禮,小的也不敢去了。雖然二爹不希罕,也盡小的一點窮心。」再三央告,伯爵把禮收了。討出三十文錢,打發拿盒人回去。於是同出門,來到西門慶對門房子裏。進到書院門首,搖的門環兒響,說道:「葵軒老先生在家麼?」溫秀才正在書窗下寫帖兒,忙應道:「請裏面坐。」畫童開門,伯爵在明間內坐的。溫秀才即出來相見,敘禮讓坐,說道:「老翁起來的早,往那裏去來?」伯爵道:「敢來煩瀆大筆寫幾個請書兒。如此這般,二十八日小兒滿月,請宅內他娘們坐坐。」\piZhang{伯爵生子滿月,正對玉樓生日,妙甚。} % 張夾批
\piWenlong{有何妙處?可笑。} % 文龍旁批
溫秀才道:「帖在那裏?將來學生寫。」伯爵即令應保取出五個帖兒,遞過去。溫秀才拿到房內,才寫得兩個,只見棋童慌走來說道:「溫師父,再寫兩個帖兒\piZhang{又是兩個。} % 張夾批
──大娘的名字,要請喬親家娘和大妗子去。頭裡琴童來取門外韓大姨和孟二妗子那兩個帖兒,打發去了不曾?」\piZhang{又是兩個。} % 張夾批
溫秀才道:「你姐夫看著,打發去這半日了。」\piZhang{如畫。} % 張夾批
棋童道:「溫師父寫了這兩個,還再寫上四個,\piZhang{又是四個。} % 張夾批
請黃四嬸、傅大娘、韓大嬸和甘伙計娘子的,我使來安兒來取。」不一時打發去了。只見來安來取這四個帖兒,\piZhang{寫的匆冗如畫。} % 張夾批
伯爵問:「你爹在家裡,是衙門中去了?」來安道:「爹今日沒往衙門裡去,在廳上看收禮哩。」溫秀才道:「老先生昨日王宅赴席來晚了。」伯爵問起那王宅,溫秀才道:「是招宣府中。」伯爵就知其故。良久,來安等了帖兒去,方纔與伯爵寫完。伯爵即帶了李銘過這邊來。


西門慶蓬著頭,只在廳上收禮,打發回帖,旁邊排擺桌面。見伯爵來,唱喏讓坐。伯爵謝前日厚情,因問:「哥定這桌席做什麼?」西門慶把安郎中來央浼作東,請蔡知府之事,告他說了一遍。伯爵道:「明日是戲子是小優?」西門慶道:「叫了一起海鹽子弟,我這裡又預備四名小優兒答應。」伯爵道:「哥,那四個?」西門慶道:「吳惠、邵奉、鄭春、左順。」伯爵道:「哥怎的不用李銘?」西門慶道:「他已有了高枝兒,又稀罕我這裡做什麼?」伯爵道:「哥怎的說這個話?你喚他,他才敢來。我也不知道你一向惱他。但是各人勾當,不乾他事。三嬸那邊幹事,他怎的曉得?你到休要屈了他。他今早到我那裡,哭哭啼啼告訴我:『休說小的姐姐在爹宅內,只小的答應該幾年,今日有了別人,到沒小的。』他再三賭身罰咒,並不知他三嬸那邊一字兒。你若惱他,卻不難為他了。他小人有什麼大湯水兒?你若動動意兒,他怎的禁得起!」便教李銘:「你過來,親自告訴你爹。你只顧躲著怎的?自古醜媳婦免不得見公婆。」

那李銘站在槅子邊,低頭斂足,就似僻廳鬼兒一般看著二人說話。聽得伯爵叫他,連忙走進去,跪著地下,只顧磕頭,說道:「爹再訪,那邊事小的但有一字知道,小的車碾馬踏,遭官刑揲死。爹從前已往,天高地厚之恩,小的一家粉身碎骨也報不過來。不爭今日惱小的,惹的同行人恥笑,他也欺負小的,小的再向那裡尋個主兒?」說畢,號淘痛哭,跪在地下只顧不起來。伯爵在旁道:「罷麼,哥也是看他一場。大人不見小人之過,休說沒他不是,就是他有不是處,他既如此,你也將就可恕他罷。」又叫李銘:「你過來,自古穿青衣抱黑柱,你爹既說開,就不惱你了,你往後也要謹慎些。」李銘道:「二爹說的是,知過必改,往後知道了。」西門慶沉吟半晌,便道:「既你二爹再三說,我不惱你了,起來答應罷。」伯爵道:」你還不快磕頭哩!」那李銘連忙磕個頭,立在旁邊。伯爵方纔令應保取出五個請帖兒來,遞與西門慶道:「二十八日小兒彌月,請列位嫂子過舍光降光降。」西門慶看畢,教來安兒:「連盒兒送與大娘瞧去。──管情後日去不成。實和你說,明日是你三娘生日,家中又是安郎中擺酒,二十八日他又要看夏大人娘子去,如何去的成?」伯爵道:「哥殺人哩!嫂子不去,滿園中果子兒,再靠著誰哩!我就親自進屋裡請去。」少頃,只見來安拿出空盒子來了:「大娘說,多上覆,知道了。」伯爵把盒兒遞與應保接去,笑了道:「哥,你就哄我起來。若是嫂子不去,我就把頭磕爛了,也好歹請嫂子走走去。」西門慶教伯爵:「你且休去,等我梳起頭來,咱每吃飯。」說畢,入後邊去了。

這伯爵便向李銘道:「如何?剛纔不是我這般說著,他甚是惱你。他有錢的性兒,隨他說幾句罷了。常言:嗔拳不打笑面。如今時年,尚個奉承的。拿著大本錢做買賣,還帶三分和氣。你若撐硬船兒,誰理你!全要隨機應變,似水兒活,才得轉出錢來。你若撞東牆,別人吃飯飽了,你還忍餓。你答應他幾年,還不知他性兒?明日交你桂姐趕熱腳兒來,兩當一:就與三娘做生日,就與他陪了禮兒來,一天事都了了。」李銘道:「二爹說的是。小的到家,過去就對三媽說。」說著,只見來安兒放桌兒,說道:「應二爹請坐,爹就出來。」

不一時,西門慶梳洗出來,陪伯爵坐的,問他:「你連日不見老孫、祝麻子?」伯爵道:「我令他來,他知道哥惱他。我便說:『還是哥十分情分,看上顧下,那日蜢蟲螞炸一例撲了去,你敢怎樣的!』他每發下誓,再不和王家小廝走。說哥昨日在他家吃酒來?他每也不知道。」西門慶道:「昨日他如此這般,置了一席大酒請我,拜認我做乾老子,吃到二更來了。他每怎的再不和他來往?只不乾礙著我的事,隨他去,我管他怎的?我不真是他老子,管他不成!」伯爵道:「哥這話說絕了。他兩個,一二日也要來與你服個禮兒,解釋解釋。」西門慶道:「你教他只顧來,平白服甚禮?」一面來安兒拿上飯來,無非是炮烹美口餚饌。西門慶吃粥,伯爵用飯。吃畢,西門慶問:「那兩個小優兒來了不曾?」來安道:「來了這一日了。」西門慶叫他和李銘一答兒吃飯。一個韓佐,一個邵謙,向前來磕了頭,下邊吃飯去了。

良久,伯爵起身,說道:「我去罷,家裡不知怎樣等著我哩。小人家兒幹事最苦,從爐臺底下直買到堂屋門首,那些兒不要買?」西門慶道:「你去幹了事,晚間來坐坐,與你三娘上壽,磕個頭兒,也是你的孝順。」伯爵道:「這個一定來,還教房下送人情來。」說畢,一直去了。正是:

酒深情不厭,知己話偏長。
莫負相欽重,明朝到草堂。



\end{showcontents}

