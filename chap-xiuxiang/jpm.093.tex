%# -*- coding: utf-8 -*-
%!TEX encoding = UTF-8 Unicode
%!TEX TS-program = xelatex
% vim:ts=4:sw=4
%
% 以上设定默认使用 XeLaTex 编译,并指定 Unicode 编码,供 TeXShop 自动识别

%第九十三回 
\chapter{王杏庵義恤貧兒 金道士孌淫少弟}

詩曰:

階前潛制淚,眾里自嫌身。
氣味如中酒,情懷似別人。
暖風張樂席,晴日看花塵。
儘是添愁處,深居乞過春。

話說陳敬濟,自從西門大姐死了,被吳月娘告了一狀,打了一場官司出來,唱的馮金寶又歸院中去了,剛刮剌出個命兒來。房兒也賣了,本錢兒也沒了,頭面也使了,傢伙也沒了。又說陳定在外邊打發人,克落了錢,把陳定也攆去了。家中日逐盤費不周,坐吃山空,不時往楊大郎家中,問他這半船貨的下落。一日,來到楊大郎門首,叫聲:「楊大郎在家不在?」不想楊光彥拐了他半船貨物,一向在外,賣了銀兩,四散躲閃。及打聽得他家中弔死了老婆,他丈母縣中告他,坐了半個月監,這楊大郎就驀地來家住著。聽見敬濟上門叫他,問貨船下落,一徑使兄弟楊二風出來,反問敬濟要人:「你把我哥哥叫的外面做買賣,這幾個月通無音信,不知拋在江中,推在河內,害了性命,你倒還來我家尋貨船下落?人命要緊,你那貨物要緊?」這楊二風平昔是個刁徒潑皮,耍錢搗子,胳膊上紫肉橫生,胸前上黃毛亂長,是一條直率光棍。走出來一把扯住敬濟,就問他要人。那敬濟慌忙掙開手跑出回家來。這楊二風故意拾了塊三尖瓦楔,將頭顱鑽破,血流滿面,趕將敬濟來,罵道:「我肏你娘娘!我見你家甚麼銀子來?你來我屋裡放屁,吃我一頓好拳頭。」那敬濟金命水命,走投無命,奔到家,把大門關閉如鐵桶相似,由著楊二風牽爹娘,罵父母,拿大磚砸門,只是鼻口內不敢出氣兒。又況才打了官司出來,夢條繩蛇也害怕,只得含忍過了。正是:

嫩草怕霜霜怕日,惡人自有惡人磨。

不消幾時,把大房賣了,找了七十兩銀子,典了一所小房,在僻巷內居住。落後兩個丫頭,賣了一個重喜兒,只留著元宵兒和他同鋪歇。又過了不上半月,把小房倒騰了,卻去賃房居住。陳安也走了,家中沒營運,元宵兒也死了,止是單身獨自,傢伙桌椅都變賣了,只落得一貧如洗。未幾,房錢不給,鑽入冷鋪內存身。花子見他是個富家勤兒,生得清俊,叫他在熱炕上睡,與他燒餅兒吃。有當夜的過來教他頂火夫,打梆子搖鈴。

那時正值臘月,殘冬時分,天降大雪,吊起風來,十分嚴寒。這工敬濟打了回梆子,打發當夜的兵牌過去,不免手提鈴串了幾條街巷。又是風雪,地下又踏著那寒冰,凍得聳肩縮背,戰戰兢兢。臨五更雞叫,只見個病花子躺在牆底下,恐怕死了,總甲分付他看守著,尋了把草叫他烤。這敬濟支更一夜,沒曾睡,就歪下睡著了。不想做了一夢,夢見那時在西門慶家,怎生受榮華富貴,和潘金蓮勾搭,頑耍戲謔,從睡夢中就哭醒來。眾花子說:「你哭怎的?」這敬濟便道:「你眾位哥哥,我的苦楚,你怎得知?頻年困苦痛妻亡,身上無衣口絕糧。馬死奴逃房又賣,隻身獨自在他鄉。 朝依肆店求遺饌,暮宿莊園倚敗牆。只有一條身後路,冷鋪之中去打梆。」

陳敬濟晚夕在冷鋪存身,白日間街頭乞食。

清河縣城內有一老者,姓王名宣,字廷用,年六十餘歲,家道殷實,為人心慈,仗義疏財,專一濟貧拔苦,好善敬神。所生二子,皆當家成立。長子王乾,襲祖職為牧馬所掌印正千戶;次子王震,充為府學庠生。老者門首搭了個主管,開著個解當鋪兒。每日豐衣足食,閑散無拘,在梵宇聽經,琳宮講道。無事在家門首施藥救人,拈素珠念佛。因後園中有兩株杏樹,道號為杏庵居士。

一日,杏庵頭戴重檐幅巾,身穿水合道服,在門首站立。只見陳敬濟打他門首過,向前扒在地下磕了個頭。忙的杏庵還禮不迭,說道:「我的哥,你是誰?老拙眼昏,不認的你。」這敬濟戰戰兢兢,站立在旁邊說道:「不瞞你老人家,小人是賣松槁陳洪兒子。」老者想了半日,說:「你莫不是陳大寬的令郎麼?」因見他衣服襤褸,形容憔悴,說道:「賢侄,你怎的弄得這般模樣?」便問:「你父親、母親可安麼?」敬濟道:「我爹死在東京,我母親也死了。」杏庵道:「我聞得你在丈人家住來?」敬濟道:「家外父死了,外母把我攆出來。他女兒死了,告我到官,打了一場官司。把房兒也賣了,有些本錢兒,都吃人坑了,一向閑著沒有營生。」 杏庵道:「賢侄,你如今在那裡居住?」敬濟半日不言語,說:「不瞞你老人家說,如此如此。」杏庵道:「可憐,賢侄你原來討吃哩。想著當初,你府上那樣根基人家。我與你父親相交,賢侄,你那咱還小哩,才扎著總角上學堂,怎就流落到此地位?可傷,可傷。你政治家甚親家?也不看顧你看顧兒。」敬濟道:「正是。俺張舅那裡,一向也久不上門,不好去的。」

問了一回話,老者把他讓到裡面客位里,令小廝放桌兒,擺出點心嗄飯來,教他儘力吃了一頓。見他身上單寒,拿出一件青布綿道袍兒,一頂氈帽,又一雙氈襪、綿鞋,又秤一兩銀子,五百銅錢,遞與他,分付說:「賢侄,這衣服鞋襪與你身上,那銅錢與你盤纏,賃半間房兒住;這一兩銀子,你拿著做上些小買賣兒,也好糊口過日子,強如在冷鋪中,學不出好人來。每月該多少房錢,來這裡,老拙與你。」這陳敬濟扒在地下磕頭謝了,說道:「小侄知道。」拿著銀錢,出離了杏庵門首。也不尋房子,也不做買賣,把那五百文錢,每日只在酒店麵店以了其事。那一兩銀子,搗了些白銅頓罐,在街上行使。吃巡邏的當土賊拿到該坊節級處,一頓拶打,使的罄盡,還落了一屁股瘡。不消兩日,把身上綿衣也輸了,襪兒也換嘴來吃了,依舊原在街上討吃。

一日,又打王杏庵門首所過,杏庵正在門首,只見敬濟走來磕頭,身上衣襪都沒了,止戴著那氈帽,精腳趿鞋,凍的乞乞縮縮。老者便問:「陳大官,做的買賣如何?房錢到了,來取房錢來了?」那陳敬濟半日無言可對。問之再三,方說如此這般,都沒了。老者便道:「阿呀,賢侄,你這等就不是過日子的道理。你又拈不的輕,負不的重,但做了些小活路兒,不強如乞食,免教人恥笑,有玷你父祖之名。你如何不依我說?」一面又讓到裡面,教安童拿飯來與他吃飽了。又與了他一條夾褲,一領白布衫,一雙裹腳,一弔銅錢,一鬥米:「你拿去務要做上了小買賣,賣些柴炭、豆兒、瓜子兒,也過了日子,強似這等討吃。」這敬濟口雖答應,拿錢米在手,出離了老者門,那消幾日,熟食肉面,都在冷鋪內和花子打夥兒都吃了。耍錢,又把白布衫、夾褲都輸了。大正月里,又抱著肩兒在街上走,不好來見老者,走在他門首房山牆底下,嚮日陽站立。

老者冷眼看見他,不叫他。他挨挨搶搶,又到根前扒在地下磕頭。老者見他還依舊如此,說道:「賢侄,這不是常策。咽喉深似海,日月快如梭,無底坑如何填得起?你進來,我與你說,有一個去處,又清閑,又安得你身,只怕你不去。」敬濟跪下哭道:「若得老伯見憐,不拘那裡,但安下身,小的情願就去。」杏庵道: 「此去離城不遠,臨清馬頭上,有座晏公廟。那裡魚米之鄉,舟船輻輳之地,錢糧極廣,清幽瀟灑。廟主任道士,與老拙相交極厚,他手下也有兩三個徒弟徒孫。我備分禮物,把你送與他做個徒弟出家,學些經典吹打,與人家應福,也是好處。」敬濟道:「老伯看顧,可知好哩。」杏庵道:「既然如此,你去,明日是個好日子,你早來,我送你去。」敬濟去了。這王老連忙叫了裁縫來,就替敬濟做了兩件道袍,一頂道髻,鞋襪俱全。

次日,敬濟果然來到。王老教他空屋裡洗了澡,梳了頭,戴上道髻,裡外換了新襖新褲,上蓋表絹道衣,下穿雲履氈襪,備了四盤羹果,一壇酒,一匹尺頭,封了五兩銀子。他便乘馬,雇了一匹驢兒與敬濟騎著,安童、喜童跟隨,兩個人擔了盒擔,出城門,徑往臨清馬頭晏公廟來。止七十里,一日路程。比及到晏公廟,天色已晚,王老下馬,進入廟來。只見青松鬱郁,翠柏森森,兩邊八字紅牆,正面三間朱戶,端的好座廟宇。但見:

山門高聳,殿閣棱層。高懸敕額金書,彩畫出朝入相。五間大殿,塑龍王一十二尊;兩下長廊,刻水族百千萬眾。旗竿凌漢,帥字招風。四通八達,春秋社禮享依時;雨順風調,河道民間皆祭賽。萬年香火威靈在,四境官民仰賴安。

山門下早有小童看見,報入方丈,任道士忙整衣出迎。王杏庵令敬濟和禮物且在外邊伺候。不一時,任道士把杏庵讓入方丈松鶴軒敘禮,說:「王老居上,怎生一向不到敝廟隨喜?今日何幸,得蒙下顧。」杏庵道:「只因家中俗冗所羈,久失拜望。」敘禮畢,分賓主而坐,小童獻茶。茶罷,任道士道:「老居士,今日天色已晚,你老人家不去罷了。」分付把馬牽入後槽喂息。杏庵道:「沒事不登三寶殿。老拙敬來有一事乾瀆,未知尊意肯容納否?」任道士道:「老居士有何見教?只顧分付,小道無不領命。」杏庵道:「今有故人之子,姓陳,名敬濟,年方二十四歲。生的資格清秀,倒也伶俐。只是父母去世太早,自幼失學。若說他父祖根基,也不是無名少姓人家,有一分家當,只因不幸遭官事沒了,無處棲身。老拙念他乃尊舊日相交之情,欲送他來貴宮作一徒弟,未知尊意如何?」任道士便道:「老居士分付,小道怎敢違阻?奈因小道命蹇,手下雖有兩三個徒弟,都不省事,沒一個成立的,小道常時惹氣,未知此人誠實不誠實?」杏庵道:「這個小的,不瞞尊師說,只顧放心,一味老實本分,膽兒又小,所事兒伶範,堪可作一徒弟。」任道士問:「幾時送來?」杏庵道:「見在山門外伺候。還有些薄禮,伏乞笑納。」慌的任道士道:「老居乾何不早說?」一面道:「有請。」於是抬盒人抬進禮物。任道士見帖兒上寫著:「謹具粗段一端,魯酒一樽,豚蹄一副,燒鴨二隻,樹果二盒,白金五兩。知生王宣頓首拜。」連忙稽首謝道:「老居士何以見賜許多重禮,使小道卻之不恭,受之有愧。」

只見陳敬濟頭戴金梁道髻,身穿青絹道衣,腳下雲履凈襪,腰系絲絛,生的眉清目秀,齒白唇紅,面如傅粉,走進來向任道士倒身下拜,拜了四雙八拜。任道士因問他:「多少青春?」敬濟道:「屬馬,交新春二十四歲了。」任道士見他果然伶俐,取了他個法名,叫做陳宗美。原來任道士手下有兩個徒弟,大徒弟姓金,名宗明;二徒弟姓徐,名宗順。他便叫陳宗美。王杏庵都請出來,見了禮數。一面收了禮物,小童掌上燈來,放卓兒,先擺飯,後吃酒。餚品杯盤,堆滿桌上,無非是雞蹄鵝鴨魚肉之類。王老吃不多酒,徒弟輪番勸勾幾巡,王老不勝酒力告辭。房中自有床鋪,安歇一宿。

到次日清晨,小童舀水凈面,梳洗盥漱畢,任道士又早來遞茶。不一時,擺飯,又吃了兩杯酒,喂飽頭口,與了抬盒人力錢。王老臨起身,叫過敬濟來分付:「在此好生用心習學經典,聽師父指教。我常來看你,按季送衣服鞋襪來與你。」又向任道士說:「他若不聽教訓,一任責治,老拙並不護短。」一面背地又囑付敬濟: 「我去後,你要洗心改正,習本等事業。你若再不安分,我不管你了。」那敬濟應諾道:「兒子理會了。」王老當下作辭任道士,出門上馬,離晏公廟,回家去了。

敬濟自此就在晏公廟做了道士。因見任道士年老赤鼻,身體魁偉,聲音洪亮,一部髭髯,能談善飲,只專迎賓送客。凡一應大小事,都在大徒弟金宗明手裡。那時,朝廷運河初開,臨清設二閘,以節水利。不拘官民,船到閘上,都來廟裡,或求神福,或來祭願,或設卦與笤,或做好事。也有佈施錢米的,也有饋送香油紙燭的,也有留松蒿蘆席的。這任道士將常署里多餘錢糧,都令家下徒弟在馬頭上開設錢米鋪,賣將銀子來,積攢私囊。

他這大徒弟金宗明,也不是個守本分的。年約三十餘歲,常在娼樓包占樂婦,是個酒色之徒。手下也有兩個清潔年少徒弟,同鋪歇臥,日久絮繁。因見敬濟生的齒白唇紅,面如傅粉,清俊乖覺,眼裡說話,就纏他同房居住。晚夕和他吃半夜酒,把他灌醉了,在一鋪歇臥。初時兩頭睡,便嫌敬濟腳臭,叫過一個枕頭上睡。睡不多回,又說他口氣噴著,令他弔轉身子,屁股貼著肚子。那敬濟推睡著,不理他。他把那話弄得硬硬的,直豎一條棍,抹了些唾津在頭上,往他糞門裡只一頂。原來敬濟在冷鋪里,被花子飛天鬼侯林兒弄過的,眼子大了,那話不覺就進去了。這敬濟口中不言,心內暗道:「這廝合敗。他討得十方便宜多了,把我不知當做甚麼人兒。與他個甜頭兒,且教他在我手內納些錢鈔。」一面故意聲叫起來。這金宗明恐怕老道士聽見,連忙掩住他口,說:「好兄弟,噤聲!隨你要的,我都依你。」敬濟道:「你既要勾搭我,我不言語,須依我三件事。」宗明道:「好兄弟,休說三件,就是十件事,我也依你。」敬濟道:「第一件,你既要我,不許你再和那兩個徒弟睡;第二件,大小房門鑰匙,我要執掌;第三件,隨我往那裡去,你休嗔我。你都依了我,我方依你此事。」金宗明道:「這個不打緊,我都依你。」當夜兩個顛來倒去,整狂了半夜。這陳敬濟自幼風月中撞,甚麼事不知道。當下被底山盟,枕邊海誓,淫聲艷語,摳吮舔品,把這金宗明哄得歡喜無盡。到第二日,果然把各處鑰匙都交與他手內,就不和那兩個徒弟在一處,每日只同他一鋪歇臥。

一日兩,兩日三,這金宗明便再三稱贊他老實。任道士聽信,又替他使錢討了一張度牒。自此以後,凡事並不防範。這陳敬濟因此常拿著銀錢往馬頭上遊玩,看見院中架兒陳三兒說:「馮金寶兒他鴇子死了,他又賣在鄭家,叫鄭金寶兒。如今又在大酒樓上趕趁哩,你不看他看去?」這小伙兒舊情不改,拿著銀錢,跟定陳三兒,徑往馬頭大酒樓上來。此不來倒好,若來,正是:

五百載冤家來聚會,數年前姻眷又相逢。

有詩為證:

人生莫惜金縷衣,人生莫負少年時。
有花欲折須當折,莫待無花空折枝。

原來這座酒樓乃是臨清第一座酒樓,名喚謝家酒樓。裡面有百十座閣兒,周圍都是綠欄桿,就緊靠著山岡,前臨官河,極是人煙鬧熱去處,舟船往來之所。怎見得這座酒樓齊整?但見:

雕檐映日,面棟飛雲。綠欄桿低接軒窗,翠簾櫳高懸戶牖。吹笙品笛,盡都是公子王孫;執盞擎杯,擺列著歌嫗舞女。消磨醉眼,依青天萬疊雲山;勾惹吟魂,翻瑞雪一河煙水。樓畔綠楊啼野鳥,門前翠柳系花驄。

這陳三兒引敬濟上樓,到一個閣兒里坐下。便叫店小二打抹春台,安排一分上品酒果下飯來擺著,使他下邊叫粉頭去了。須臾,只見樓梯響,馮金寶上來,手中拿著個廝鑼兒,見了敬濟,深深道了萬福。常言情人見情人,不覺簇地兩行淚下。正是:

數聲嬌語如鶯囀,一串珍珠落線買。

敬濟一見,便拉他一處坐,問道:「姐姐,你一向在那裡來?不見你。」這馮金寶收淚道:「自從縣中打斷出來,我媽著了驚謊,不久得病死了,把我賣在鄭五媽家。這兩日子弟稀少,不免又來在臨清馬頭上趕趁酒客。昨日聽見陳三兒說你在這裡開錢鋪,要見你一見。不期今日會見一面。可不想殺我也!」說畢,又哭了。敬濟取出袖中帕兒,替他抹了眼淚,說道:「我的姐姐,你休煩惱。我如今又好了,自從打出官司來,家業都沒了,投在這晏公廟,做了道士。師父甚是托我,往後我常來看你。」因問:「你如今在那裡安下?」金寶便道:「奴就在這橋西灑家店劉二那裡。有百十房子,四外行院窠子,妓女都在那裡安下,白日里便是這各酒樓趕趁。」說著,兩個挨身做一處飲酒。陳三兒燙酒上樓,拿過琵琶來。金寶彈唱了個曲兒與敬濟下酒,名《普天樂》:

淚雙垂,垂雙淚。三杯別酒,別酒三杯。鸞鳳對拆開,折開鸞鳳對。嶺外斜暉看看墜,看看墜,嶺外暉。天昏地暗,徘徊不舍,不舍徘徊。

兩人吃得酒濃時,朱免解衣雲雨,下個房兒。這陳敬濟一向不曾近婦女,久渴的人,今得遇金寶,儘力盤桓,尤雲殢雨,未肯即休。須臾事畢,各整衣衫。敬濟見天色晚了,與金寶作別,與了金寶一兩銀子,與了陳三兒百文銅錢,囑付:「姐姐,我常來看你,咱在這搭兒里相會。你若想我,使陳三兒叫我去。」下樓來,又打發了店主人謝三郎三錢銀子酒錢。敬濟回廟中去了。馮金寶送至橋邊方回。正是:

盼穿秋水因錢鈔,哭損花容為鄧通。



