%# -*- coding: utf-8 -*-
%!TEX encoding = UTF-8 Unicode
%!TEX TS-program = xelatex
% vim:ts=4:sw=4
%
% 以上设定默认使用 XeLaTex 编译,并指定 Unicode 编码,供 TeXShop 自动识别

%第三十二回 
\chapter{李桂姐趨炎認女 潘金蓮懷妒驚兒}

詩曰:

牛馬鳴上風,聲應在同類。小人非一流,要呼各相比。
吹彼塤與篪,翕翕騁志意。願游廣漠鄉,舉手謝時輩。

話說當日眾官飲酒席散,西門慶還留吳大舅、二舅、應伯爵、謝希大後坐。打發樂工等酒飯吃了,吩咐:「你每明日還來答應一日,我請縣中四宅老爹吃酒,俱要齊備些。臨了一總賞你每罷。」眾樂工道:「小的每無不用心,明日都是官樣新衣服來答應。」吃了酒飯,磕頭去了。良久,李桂姐、吳銀兒搭著頭出來,笑嘻嘻道: 「爹,晚了,轎子來了,俺每去罷。」應伯爵道:「我兒,你倒且是自在。二位老爹在這裡,不說唱個曲兒與老爹聽,就要去罷?」桂姐道:「你不說這一聲兒,不當啞狗賣。俺每兩日沒往家去,媽不知怎麼盼哩。」伯爵道:「盼怎的?玉黃李子兒,掐了一塊兒去了?」西門慶道:「也罷,教他兩個去罷,本等連日辛苦了。咱叫李銘、吳惠唱罷。」問道:「你吃了飯了?」桂姐道:「剛纔大娘留俺每吃了。」於是齊磕頭下去。西門慶道:「你二位後日還來走走,再替我叫兩個,不拘鄭愛香兒也罷,韓金釧兒也罷,我請親朋吃酒。」伯爵道:「造化了小淫婦兒,教他叫,又討提錢使。」桂姐道:「你又不是架兒,你怎曉得恁切?」說畢,笑的去了。伯爵因問:「哥,後日請誰?」西門慶道:「那日請喬老、二位老舅、花大哥、沈姨夫,並會中列位兄弟,歡樂一日。」伯爵道:「說不得,俺每打攪得哥忒多了。到後日,俺兩個還該早來,與哥做副東。」西門慶道:「此是二位下顧了。」說畢話,李銘、吳惠拿樂器上來,唱了一套。吳大舅等眾人方一齊起身。一宿晚景不題。

到次日,西門慶請本縣四宅官員。那日薛內相來的早,西門慶請至捲棚內待茶。薛內相因問:「劉家沒送禮來?」西門慶道:「劉老太監送過禮了。」良久,薛內相要請出哥兒來看一看:「我與他添壽。」西門慶推卻不得,只得教玳安後邊說去,抱哥兒出來。不一時,養娘抱官哥送出到角門首,玳安接到上面。薛內相看見,只顧喝采:「好個哥兒!」便叫:「小廝在那裡?」須臾,兩個青衣家人,戢金方盒拿了兩盒禮物:熌紅官緞一匹,福壽康寧鍍金銀錢四個,追金瀝粉彩畫壽星博郎鼓兒一個,銀八寶貳兩。說道:「窮內相沒什麼,這些微禮兒與哥兒耍子。」西門慶作揖謝道:「多蒙老公公費心。」看畢,抱哥兒回房不題。西門慶陪著吃了茶,就先擺飯。剛纔吃罷,忽報:「四宅老爹到了。」西門慶忙整衣冠,出二門迎接。乃是知縣李達天,並縣丞錢成、主簿任廷貴、典史夏恭基。各先投拜帖,然後廳上敘禮。請薛內相出見,眾官讓薛內相坐首席。席間又有尚舉人相陪。分賓坐定,普坐遞了一巡茶。少頃,階下鼓樂響動,笙歌擁奏,遞酒上坐。教坊呈上揭帖。薛內相揀了四摺《韓湘子升仙記》,又隊舞數回,十分齊整。薛內相心中大喜,喚左右拿兩弔錢出來,賞賜樂工。

不說當日眾官飲酒至晚方散,且說李桂姐到家,見西門慶做了提刑官,與虔婆鋪謀定計。次日,買了四色禮,做了一雙女鞋,教保兒挑著盒擔,絕早坐轎子先來,要拜月娘做乾娘。進來先向月娘笑嘻嘻拜了四雙八拜,然後才與他姑娘和西門慶磕頭。把月娘哄的滿心歡喜,說道:「前日受了你媽的重禮,今日又教你費心,買這許多禮來。」桂姐笑道:「媽說,爹如今做了官,比不得那咱常往裡邊走。我情願只做乾女兒罷,圖親戚來往,宅里好走動。」月娘忙教他脫衣服坐的,因問:「吳銀姐和那兩個怎的還不來?」桂姐道:「吳銀兒,我昨日會下他,不知怎的還不見來。前日爹吩咐教我叫了鄭愛香兒和韓金釧兒,我來時他轎子都在門首,怕不也待來。」言未了,只見銀兒和愛香兒,又與一個穿大紅紗衫年小的粉頭,提著衣裳包兒進來,先望月娘磕了頭。吳銀兒看見李桂姐脫了衣裳,坐在炕上,說道:「桂姐,你好人兒!不等俺每等兒,就先來了。」桂姐道:「我等你來,媽見我的轎子在門首,說道:『只怕銀姐先去了,你快去罷。』誰知你每來的遲。」月娘笑道: 「也不遲。」因問:「這位姐兒上姓?」吳銀兒道:「他是韓金釧兒的妹子玉釧兒。」不一時,小玉放桌兒,擺了八碟茶食,兩碟點心,打發四個唱的吃了。那李桂姐賣弄他是月娘乾女兒,坐在月娘炕上,和玉簫兩個剝果仁兒、裝果盒。吳銀兒三個在下邊杌兒上,一條邊坐的。那桂姐一徑抖搜精神,一回叫:「玉簫姐,累你,有茶倒一甌子來我吃。」一回又叫:「小玉姐,你有水盛些來,我洗這手。」那小玉真個拿錫盆舀了水,與他洗手。吳銀兒眾人都看的睜睜的,不敢言語。桂姐又道:「銀姐,你三個拿樂器來唱個曲兒與娘聽。我先唱過了。」月娘和李嬌兒對面坐著。吳銀兒見他這般說,只得取過樂器來。當下鄭愛香兒彈箏,吳銀兒琵琶,韓玉釧兒在旁隨唱,唱了一套《八聲甘州》「花遮翠樓」。

須臾唱畢,放下樂器。吳銀兒先問月娘:「爹今日請那幾位官客吃酒?」月娘道:「你爹今日請的都是親朋。」桂姐道:「今日沒有請那兩位公公?」月娘道:「今日沒有,昨日也只薛內相一位。那姓劉的沒來。」桂姐道:「劉公公還好,那薛公公慣頑,把人掐擰的魂也沒了。」月娘道:「左右是個內官家,又沒什麼,隨他擺弄一回子就是了。」桂姐道:「娘且是說的好,乞他奈何的人慌。」正說著,只見玳安兒進來取果盒,見他四個在屋裡坐著,說道:「客已到了一半,七八待上坐,你每還不快收拾上去?」月娘便問:「前邊有誰來了?」玳安道:「喬大爹、花大爹、大舅、二舅、謝爹都來了這一日了。」桂姐問道:「今日有應二花子和祝麻子二人沒有?」玳安道:「會中十位,一個兒也不少。應二爹從辰時就來了,爹使他有勾當去了,便道就來也。」桂姐道:「爺嚛!遭遭兒有這起攮刀子的,又不知纏到多早晚。我今日不出去,寧可在屋裡唱與娘聽罷。」玳安道:「你倒且是自在性兒。」拿出果盒去了。桂姐道:「娘還不知道,這祝麻子在酒席上,兩片子嘴不住,只聽見他說話,饒人那等罵著,他還不理。他和孫寡嘴兩個好不涎臉。」鄭愛香兒道:「常和應二走的那祝麻子,他前日和張小二官兒到俺那裡,拿著十兩銀子,要請俺家妹子愛月兒。俺媽說:『他才教南人梳弄了,還不上一個月,南人還沒起身,我怎麼好留你?』說著他再三不肯。纏的媽急了,把門倒插了,不出來見他。那張二官兒好不有錢,騎著大白馬,四五個小廝跟隨,坐在俺每堂屋裡只顧不去。急的祝麻了直撅兒跪在天井內,說道:『好歹請出媽來,收了這銀子。只教月姐兒一見,待一杯茶兒,俺每就去。』把俺每笑的要不的。只象告水災的,好個涎臉的行貨子!」吳銀兒道:「張小二官兒先包著董貓兒來。」鄭愛香兒道:「因把貓兒的虎口內火燒了兩醮,和他丁八著好一向了,這日才散走了。」因望著桂姐道:「昨日我在門外會見周肖兒,多上覆你,說前日同聶鉞兒到你家,你不在。」桂姐使了個眼色,說道:「我到爹宅里來,他請了俺姐姐桂卿了。」鄭愛香兒道:「你和他沒點兒相交,如何卻打熱?」桂姐道:「好肏的劉九兒,把他當個孤老,甚麼行貨子,可不砢磪殺我罷了。他為了事出來,逢人至人說了來,嗔我不看他。媽說:『你只在俺家,俺倒買些什麼看看你不打緊。你和別人家打熱,俺傻的不勻了。』真是硝子石望著南兒──丁口心!」說著都一齊笑了。月娘坐在炕上聽著他說,道:「你每說了這一日,我不懂,不知說的是那家話!」按下這裡不題。

卻說前邊各客都到齊了,西門慶冠冕著遞酒。眾人讓喬大戶為首,先與西門慶把盞。只見他三個唱的從後邊出來,都頭上珠冠(口疊)褻,身邊蘭麝濃香。應伯爵一見,戲道:「怎的三個零布在那裡來?攔住,休放他進來!」因問:「東家,李家桂兒怎不來?」西門慶道:「我不知道。」初是鄭愛香兒彈箏,吳銀兒琵琶,韓金釧兒撥板。啟朱唇,露皓齒,先唱《水仙子》「馬蹄金鑄就虎頭牌」一套。良久,遞酒畢,喬大戶坐首席,其次者吳大舅、二舅、花大哥、沈姨夫、應伯爵、謝希大、孫寡嘴、祝實念、常峙節、白賚光、傅自新、賁第傳,共十四人上席,八張桌兒。西門慶下席主位。說不盡歌喉宛轉,舞態蹁躚,酒若流波,餚如山疊。到了那酒過數巡,歌吟三套之間,應伯爵就在席上開口說道:「東家,也不消教他每唱了,翻來掉過去,左右只是這兩套狗撾門的,誰待聽!你教大官兒拿三個座兒來,教他與列位遞酒,倒還強似唱。」西門慶道:「且教他孝順眾尊親兩套詞兒著。你這狗才,就這等搖席破座的。」鄭愛香兒道:「應花子,你門背後放花兒 ──等不到晚了!」伯爵親自走下席來罵道:「怪小淫婦兒,什麼晚不晚?你娘那毴!」教玳安:「過來,你替他把刑法多拿了。」一手拉著一個,都拉到席上,教他遞酒。鄭愛香兒道:「怪行貨子,拉的人手腳兒不著地。」伯爵道:「我實和你說,小淫婦兒,時光有限了,不久青刀馬過,遞了酒罷,我等不的了。」謝希大便問:「怎麼是青刀馬?」伯爵道:「寒鴉兒過了,就是青刀馬。」眾人都笑了。

當下吳銀兒遞喬大戶,鄭愛香兒遞吳大舅,韓玉釧兒遞吳二舅,兩分頭挨次遞將來。落後吳銀兒遞到應伯爵跟前,伯爵因問:「李家桂兒怎的不來?」吳銀兒道: 「你老人家還不知道,李桂姐如今與大娘認義做乾女兒。我告訴二爹,只放在心裡。卻說人弄心,前日在爹宅里散了,都一答兒家去了,都會下了明日早來。我在家裡收拾了,只顧等他。誰知他安心早買了禮,就先來了,倒教我等到這咱晚。使丫頭往他家瞧去,說他來了,好不教媽說我。你就拜認與爹娘做乾女兒,對我說了便怎的?莫不攙了你什麼分兒?瞞著人幹事。嗔道他頭裡坐在大娘炕上,就賣弄顯出他是娘的乾女兒,剝果仁兒,定果盒,拿東拿西,把俺每往下躧。我還不知道,倒是裡邊六娘剛纔悄悄對我說,他替大娘做了一雙鞋,買了一盒果餡餅兒,兩隻鴨子,一大副膀蹄,兩瓶酒,老早坐了轎子來。」從頭至尾告訴一遍。伯爵聽了道:「他如今在這裡不出來,不打緊,我務要奈何那賊小淫婦兒出來。我對你說罷,他想必和他鴇子計較了,見你大爹做了官,又掌著刑名,一者懼怕他勢要,二者恐進去稀了,假著認乾女兒往來,斷絕不了這門兒親。我猜的是不是?我教與你個法兒,他認大娘做乾女,你到明日也買些禮來,卻認與六娘做乾女兒就是了。你和他都還是過世你花爹一條路上的人,各進其道就是了。我說的是不是?你也不消惱他。」吳銀兒道:「二爹說的是,我到家就對媽說。」說畢,遞過酒去,就是韓玉釧兒,挨著來遞酒。伯爵道:「韓玉姐起動起動,不消行禮罷。你姐姐家裡做什麼哩?」玉釧兒道:「俺姐姐家中有人包著哩,好些時沒出來供唱。」伯爵道:「我記的五月里在你那裡打攪了,再沒見你姐姐。」韓玉釧道:「那日二爹怎的不肯深坐,老早就去了?」伯爵道:「不是那日我還坐,坐中有兩個人不合節,又是你大老爹這裡相招,我就先走了。」韓玉釧兒見他吃過一杯,又斟出一杯。伯爵道:「罷罷,少斟些,我吃不得了!」玉釧道:「二爹你慢慢上,上過待我唱曲兒你聽。」伯爵道:「我的姐姐,誰對你說來?正可著我心坎兒。常言道:養兒不要屙金溺銀,只要見景生情。倒還是麗春院娃娃,到明日不愁沒飯吃,強如鄭家那賊小淫婦,歪剌骨兒,只躲滑兒,再不肯唱。」鄭愛香兒道:「應二花子,汗邪了你,好罵!」西門慶道:「你這狗才,頭裡嗔他唱,這回又索落他。」伯爵道:「這是頭裡帳,如今遞酒,不教他唱個兒?我有三錢銀子,使的那小淫婦鬼推磨。」韓玉釧兒不免取過琵琶來,席上唱了個小曲兒。

:「你這小淫婦,道你調子曰兒罵我,我沒的說,只是一味白鬼,把你媽那褲帶子也扯斷了。由他到明日不與你個功德,你也不怕不把將軍為神道。」桂姐道:「咱休惹他,哥兒拿出急來了。」鄭愛香笑道:「這應二花子,今日鬼酉上車兒──推醜,東瓜花兒──醜的沒時了。他原來是個王姑來子。」伯爵道:「這小歪剌骨兒,諸人不要,只我將就罷了。」桂姐罵道:「怪攮刀子,好乾凈嘴兒,擺人的牙花已闔了。爹,你還不打與他兩下子哩,你看他恁發訕。」西門慶罵道:「怪狗才東西!教他遞酒,你鬥他怎的!」走向席上打了他一下。伯爵道:「賊小淫婦兒!你說你倚著漢子勢兒,我怕你?你看他叫的『爹』那甜!」又道:「且休教他遞酒,倒便益了他。拿過刑法來,且教他唱一套與俺每聽著。他後邊躲了這會滑兒也夠了。」韓玉釧兒道: 「二爹,曹州兵備,管的事兒寬。」這裡前廳花攢錦簇,飲酒頑耍不題。

單表潘金蓮自從李瓶兒生了孩子,見西門慶常在他房裡宿歇,於是常懷嫉妒之心,每蓄不平之意。知西門慶前廳擺酒,在鏡臺前巧畫雙蛾,重扶蟬髩,輕點朱唇,整衣出房。聽見李瓶兒房中孩兒啼哭,便走入來問道:「他怎這般哭?」奶子如意兒道:「娘往後邊去了。哥哥尋娘,這等哭。」那潘金蓮笑嘻嘻的向前戲弄那孩兒,說道:「你這多少時初生的小人芽兒,就知道你媽媽。等我抱到後邊尋你媽媽去!」奶子如意兒說道:「五娘休抱哥哥,只怕一時撒了尿在五娘身上。」金蓮道: 「怪臭肉,怕怎的!拿襯兒托著他,不妨事。」一面接過官哥來抱在懷裡,一直往後去了。走到儀門首,一逕把那孩兒舉的高高的。不想吳月娘正在上房穿廊下,看著家人媳婦定添換菜碟兒,那潘金蓮笑嘻嘻看孩子說道:「『大媽媽,你做什麼哩?』你說:『小大官兒來尋俺媽媽來了。』」月娘忽抬頭看見,說道:「五姐,你說的什麼話?早是他媽媽沒在跟前,這咱晚平白抱出他來做甚麼?舉的恁高,只怕唬著他。他媽媽在屋裡忙著手哩。」便叫道:「李大姐你出來,你家兒子尋你來了。」那李瓶兒慌走出來,看見金蓮抱著,說道:「小大官兒好好兒在屋裡,奶子抱著,平白尋我怎的?看溺了你五媽身上尿。」金蓮道:「他在屋裡,好不哭著尋你,我抱出他來走走。」這李瓶兒忙解開懷接過來。月娘引逗了一回,吩咐:「好好抱進房裡去罷,休要唬著他!」李瓶兒到前邊,便悄悄說奶子:「他哭,你慢慢哄著他,等我來,如何教五娘抱到後邊尋我?」如意兒道:「我說來,五娘再三要抱了去。」那李瓶兒慢慢看著他餵了奶,就安頓他睡了。誰知睡下不多時,那孩子就有些睡夢中驚哭,半夜發寒潮熱起來。奶子喂他奶也不吃,只是哭。李瓶兒慌了。

且說西門慶前邊席散,打發四個唱的出門。月娘與了李桂姐一套重綃絨金衣服,二兩銀子,不必細說。西門慶晚夕到李瓶兒房裡看孩兒,因見孩兒只顧哭,便問: 「怎麼的?」李瓶兒亦不題起金蓮抱他後邊去一節,只說道:「不知怎的,睡了起來這等哭,奶也不吃。」西門慶道:「你好好拍他睡。」因罵如意兒:「不好生看哥兒,管何事?唬了他!」走過後邊對月娘說。月娘就知金蓮抱出來唬了他,就一字沒對西門慶說,只說:「我明日叫劉婆子看他看。」西門慶道:「休教那老淫婦來胡針亂灸的,另請小兒科太醫來看孩兒。」月娘不依他,說道:「一個剛滿月的孩子,什麼小兒科太醫。」到次日,打發西門慶早往衙門中去了,使小廝請了劉婆來看了,說是著了驚。與了他三錢銀子。灌了他些藥兒,那孩兒方纔得睡穩,不洋奶了。李瓶兒一塊石頭方落地。正是:

滿懷心腹事,盡在不言中。


