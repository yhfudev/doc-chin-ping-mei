%# -*- coding: utf-8 -*-
%!TEX encoding = UTF-8 Unicode
%!TEX TS-program = xelatex
% vim:ts=4:sw=4
%
% 以上设定默认使用 XeLaTex 编译,并指定 Unicode 编码,供 TeXShop 自动识别

%第七十九回 
\chapter{西門慶貪欲喪命 吳月娘失偶生兒}

詞曰:

青玉案

人生南北如岐路,世事悠悠等風絮,造化弄人無定據。翻來覆去,倒橫直豎,眼見都如許。
到如今空嗟前事,功名富貴何須慕,坎止流行隨所寓。玉堂金馬,竹籬茅舍,總是傷心處。

話說西門慶,姦耍了來爵老婆,復走到捲棚內,陪吳大舅、應伯爵、謝希大、常峙節飲酒。荊統制娘子、張團練娘子、喬親家母、崔親家母、吳大妗子、段大姐,坐了好一會,上罷元宵圓子,方纔起身去了。大妗子那日同吳舜臣媳婦都家去了。陳敬濟打發王皇親戲子二兩銀子唱錢,酒食管待出門。只四個唱的並小優兒,還在捲棚內彈唱遞酒。伯爵向西門慶說道:「明日花大哥生日,哥,你送了禮去不曾?」西門慶說道:「我早辰送過去了。」玳安道:「花大舅頭裡使來定兒送請貼兒來了。」伯爵道:「哥,你明日去不去?我好來會你。」西門慶道:「到明日看。再不,你先去罷。」少頃,四個唱的後邊去了,李銘等上來彈唱,那西門慶不住只在椅子上打睡。吳大舅道:「姐夫連日辛苦了,罷罷,咱每告辭罷。」於是起身。那西門慶又不肯,只顧攔著,留坐到二更時分才散。西門慶先打發四個唱的轎子去了,拿大鐘賞李銘等三人每人兩鐘酒,與了六錢唱錢,臨出門,叫回李銘分付:「我十五日要請你周爺和你荊爺、何老爹眾位,你早替我叫下四個唱的,休要誤了。」李銘跪下稟問:「爹叫那四個?」西門慶道:「樊百家奴兒,秦玉芝兒,前日何老爹那裡唱的一個馮金寶兒,並呂賽兒,好歹叫了來。」李銘應諾:「小的知道了。」磕了頭去了。

西門慶歸後邊月娘房裡來。月娘告訴:「今日林太太與荊大人娘子好不喜歡,坐到那咱晚才去了。酒席上再三謝我說:蒙老爹扶持,但得好處,不敢有忘。在出月往淮上催攢糧運去也。」又說:「何大娘子今日也吃了好些酒,喜歡六姐,又引到那邊花園山子上瞧了瞧。今日各項也賞了許多東西。」說畢,西門慶就在上房歇了。到半夜,月娘做了一夢,天明告訴西門慶說道:「敢是我日里看著他王太太穿著大紅絨袍兒,我黑夜就夢見你李大姐箱子內尋出一件大紅絨袍兒,與我穿在身上,被潘六姐匹手奪了去,披在他身上,教我就惱了,說道:『他的皮襖,你要的去穿了罷了,這件袍兒你又來奪。』他使性兒把袍兒上身扯了一道大口子,吃我大吆喝,和他罵嚷,嚷著就醒了。不想是南柯一夢。」西門慶道:「不打緊,我到明日替你尋一件穿就是了。自古夢是心頭想。」

到次日起來,頭沉,懶待往衙門中去,梳頭凈面,穿上衣裳,走來前邊書房中坐的。只見玉簫問如意兒擠了半甌子奶,徑到書房與西門慶吃藥。西門慶正倚靠床上,叫王經替他打腿。王經見玉簫來,就出去了。玉簫打發他吃了藥,西門慶就使他拿了一對金鑲頭簪兒,四個烏銀戒指兒,送到來爵媳婦子屋裡去。那玉簫明見主子使他乾此營生,又似來旺媳婦子那一本帳,連忙鑽頭覓縫,袖的去了。送到了物事,還走來回西門慶話,說道:「收了,改日與爹磕頭。」就拿回空甌子兒到上房去了。月娘叫小玉熬下粥,約莫等到飯時前後,還不見進來。

原來王經稍帶了他姐姐王六兒一包兒物事,遞與西門慶瞧,就請西門慶往他家去。西門慶打開紙包兒,卻是老婆剪下的一柳黑臻臻、光油油的青絲,用五色絨纏就了一個同心結托兒,用兩根錦帶兒拴著,做的十分細巧。又一件是兩個口的鴛鴦紫遍地金順袋兒,裡邊盛著瓜穰兒。西門慶觀玩良久,滿心歡喜,遂把順袋放在書廚內,錦托兒褪於袖中。正在凝思之際,忽見吳月娘驀地走來,掀開帘子,見他躺在床上,王經扒著替他打腿,便說道:「你怎的只顧在前頭,就不進去了,屋裡擺下粥了。你告我說,你心裡怎的,只是恁沒精神?」西門慶道:「不知怎的,心中只是不耐煩,害腿疼。」月娘道:「想必是春氣起了。你吃了藥,也等慢慢來。」一面請到房中,打發他吃粥。因說道:「大節下,你也打起精神兒來,今日門外花大舅生日,請你往那裡走走去。再不,叫將應二哥來,同你坐坐。」西門慶道:「他也不在,與花大舅做生日去了。你整治下酒菜兒,等我往燈市鋪子內和他二舅坐坐罷。」月娘道:「你騎馬去,我教丫鬟整理。」這西門慶一面分付玳安備馬,王經跟隨,穿上衣穿,徑到獅子街燈市裡來。但見燈市中車馬轟雷,燈球燦彩,遊人如蟻,十分熱鬧。

太平時序好風催,羅綺爭馳鬥錦回。
鰲山高聳青雲上,何處遊人不看來。

西門慶看了回燈,到房子門首下馬,進入裡面坐下。慌的吳二舅、賁四都來聲喏。門首買賣,甚是興盛。來昭妻一丈青又早書房內籠下火,拿茶吃了。不一時,吳月娘使琴童兒、來安兒拿了兩方盒點心嗄飯菜蔬,鋪內有南邊帶來豆酒,打開一壇,擺在樓上,請吳二舅與賁四輪番吃酒。樓窗外就看見燈市,來往人煙不斷。

吃至飯後時分,西門慶使王經對王六兒說去。王六兒聽見西門慶來,連忙整治下春台,果盒酒餚等候。西門慶分付來昭:「將這一桌酒菜,晚夕留著吳二舅、賁四在此上宿吃,不消拿回家去了。」又教琴童提送一壇酒,過王六兒這邊來。西門慶於是騎馬徑到他家。婦人打扮迎接到明間內,插燭也似磕了四個頭。西門慶道:「迭承你厚禮,怎的兩次請你不去?」王六兒說道:「爹倒說的好,我家中再有誰來?不知怎的,這兩日只是心裡不好,茶飯兒也懶待吃,做事沒入腳處。」西門慶道: 「敢是想你家老公?」婦人道:「我那裡想他!倒是見爹這一向不來,不知怎的怠慢著爹了,爹把我網巾圈兒打靠後了,只怕另有個心上人兒了。」西門慶笑道: 「那裡有這個理!倒因家中節間擺酒,忙了兩日。」婦人道:「說昨日爹家中請堂客來。」西門慶道:「便是你大娘吃過人家兩席節酒,須得請人回席。」婦人道: 「請了那幾位堂客?」西門慶便說某人某人,從頭訴說一遍。婦人道:「看燈酒兒,只請要緊的,就不請俺每請兒。」西門慶道:「不打緊,到明日十六,還有一席酒,請你每眾伙計娘子走走去。是必到跟前又推故不去了。」婦人道:「娘若賞個貼兒來,怎敢不去?」因前日他小大姐罵了申二姐,教他好不抱怨,說俺每。他那日原要不去來,倒是俺每攛掇了他去,落後罵了來,好不在這裡哭。俺每倒沒意思剌涑的。落後又教爹娘費心,送了盒子並一兩銀子來,安撫了他,才罷了。原來小大姐這等躁暴性子,就是打狗也看主人面。」西門慶道:「你不知這小油嘴,他好不兜達的性兒,著緊把我也擦刮的眼直直的。也沒見,他叫你唱,你就唱個兒與他聽罷了,誰教你不唱,又說他來?」婦人道:「耶嚛,耶嚛!他對我說,他幾時說他來,說小大姐走來指著臉子就罵起來,在我這裡好不三行鼻涕兩行眼淚的哭。我留他住了一夜,才打發他去了。」說了一回,丫頭拿茶吃了。老馮婆子又走來與西門慶磕頭。西門慶與了他約三四錢一塊銀子,說道:「從你娘沒了,就不往我那裡走走去。」婦人道:「沒他的主兒,那裡著落?倒常時來我這裡,和我做伴兒。」

西門慶道:「我知道。」兩個說話之間,又乾勾兩頓飯時,方纔精泄。解御下婦人腳帶來,摟在被窩內,並頭交股,醉眼朦朧,一覺直睡到三更時分方起。西門慶起來,穿衣凈手。婦人開了房門,叫丫鬟進來,再添美饌,復飲香醪,滿斟暖酒,又陪西門慶吃了十數杯。不覺醉上來,才點茶漱口,向袖中掏出一紙貼兒遞與婦人:「問甘伙計鋪子里取一套衣服你穿,隨你要甚花樣。」那婦人萬福謝了,方送出門。

王經打著燈籠,玳安、琴童籠著馬,那時也有三更天氣,陰雲密佈,月色朦朧,街市上人煙寂寞,閭巷內犬吠盈盈。打馬剛走到西首那石橋兒跟前,忽然一陣旋風,只見個黑影子,從橋底下鑽出來,向西門慶一撲。那馬見了只一驚跳,西門慶在馬上打了個冷戰,醉中把馬加了一鞭,那馬搖了搖鬃,玳安、琴童兩個用力拉著嚼環,收煞不住,雲飛般望家奔將來,直跑到家門首方止。王經打著燈籠,後邊跟不上。西門慶下馬腿軟了,被左右扶進,徑往前邊潘金蓮房中來。此這一來,正是:

失脫人家逢五道,濱冷餓鬼撞鐘馗。

,髓竭人亡。正是起頭所說:

二八佳人體似酥,腰間仗劍斬愚夫。
雖然不見人頭落,暗裡教君骨髓枯。

一宿晚景題過。到次日清早辰,西門慶起來梳頭,忽然一陣昏暈,望前一頭搶將去。早被春梅雙手扶住,不曾跌著磕傷了頭臉。在椅上坐了半日,方纔回過來。慌的金蓮連忙問道:「只怕你空心虛弱,且坐著,吃些甚麼兒著,出去也不遲。」一面使秋菊:「後邊取粥來與你爹吃。」那秋菊走到後邊廚下,問雪娥:「熬的粥怎麼了?爹如此這般,今早起來害了頭暈,跌了一交,如今要吃粥哩。」不想被月娘聽見,叫了秋菊,問其端的。秋菊悉把西門慶梳頭,頭暈跌倒之事,告訴一遍。月娘不聽便了,聽了魂飛天外,魄散九霄,一面分付雪娥快熬粥,一面走來金蓮房中看視。見西門慶坐在椅子上,問道:「你今日怎的頭暈?」西門慶道:「我不知怎的,剛纔就頭暈起來。」金蓮道:「早時我和春梅要跟前扶住了,不然好輕身子兒,這一交和你善哩!」月娘道:「敢是你昨日來家晚了,酒多了頭沉。」金蓮道: 「昨日往誰家吃酒?那咱晚才來。」月娘道:「他昨日和他二舅在鋪子里吃酒來。」不一時,雪娥熬了粥,教春梅拿著,打發西門慶吃。那西門慶拿起粥來,只吃了半甌兒,懶待吃,就放下了。月娘道:「你心裡覺怎的?」西門慶道:「我不怎麼,只是身子虛飄飄的,懶待動旦。」月娘道:「你今日不往衙門中去罷。」西門慶道:「我不去了。消一回,我往前邊看著姐夫寫貼兒,十五日請周菊軒、荊南崗、何大人眾官客吃酒。」月娘道:「你今日還沒吃藥,取奶來把那藥再吃上一服。是你連日著辛苦忙碌了。」一面教春梅問如意兒擠了奶來,用盞兒盛著,教西門慶吃了藥,起身往前邊去。春梅扶著,剛走到花園角門首,覺眼便黑了,身子晃晃蕩盪,做不的主兒,只要倒。春梅又扶回來了。月娘道:「依我且歇兩日兒,請人也罷了,那裡在乎這一時。且在屋裡將息兩日兒,不出去罷。」因說:「你心裡要吃甚麼,我往後邊做來與你吃。」西門慶道:「我心裡不想吃。」

月娘到後邊,從新又審問金蓮:「他昨日來家醉不醉?再沒曾吃酒?與你行甚麼事?」金蓮聽了,恨不的生出幾個口來,說一千個沒有:「姐姐,你沒的說,他那咱晚來了,醉的行禮兒也沒顧的,還問我要燒酒吃,教我拿茶當酒與他吃,只說沒了酒,好好打發他睡了。自從姐姐那等說了,誰和他有甚事來,倒沒的羞人子剌剌的。倒只怕別處外邊有了事來,俺每不知道。若說家裡,可是沒絲毫事兒。」月娘和玉樓都坐在一處,一面叫了玳安、琴童兩個到跟前審問他:「你爹昨日在那裡吃酒來?你實說便罷,不然有一差二錯,就在你這兩個囚根子身上。」那玳安咬定牙,只說獅子街和二舅、賁四吃酒,再沒往那裡去。落後叫將吳二舅來,問他,二舅道:「姐夫只陪俺每吃了沒多大回酒,就起身往別處去了。」這吳月娘聽了,心中大怒,待二舅去了,把玳安、琴童儘力數罵了一遍,要打他二人。二人慌了,方纔說出:「昨日在韓道國老婆家吃酒來。」那潘金蓮得不的一聲就來了,說道:「姐姐剛纔就埋怨起俺每來,正是冤殺旁人笑殺賊。俺每人人有面,樹樹有皮,姐姐那等說來,莫不俺每成日把這件事放在頭裡?」又道:「姐姐,你再問這兩個囚根子,前日你往何千戶家吃酒,他爹也是那咱時分才來,不知在誰家來。誰家一個拜年,拜到那咱晚!」玳安又恐怕琴童說出來,隱瞞不住,遂把私通林太太之事,備說一遍。月娘方纔信了,說道:「嗔道教我拿貼兒請他,我還說人生面不熟,他不肯來,怎知和他有連手。我說恁大年紀,描眉畫鬢,搽的那臉倒像膩抹兒抹的一般,乾凈是個老浪貨!」玉樓道:「姐姐,沒見一個兒子也長恁大人兒,娘母還乾這個營生。忍不住,嫁了個漢子,也休要出這個醜。」金蓮道:「那老淫婦有甚麼廉恥!」月娘道:「我只說他決不來,誰想他浪(扌扉) 著來了。」金蓮道:「這個,姐姐才顯出個皂白來了!像韓道國家這個淫婦,姐姐還嗔我罵他!乾凈一家子都養漢,是個明王八,把個王八花子也裁派將來,早晚好做勾使鬼。」月娘道: 「王三官兒娘,你還罵他老淫婦,他說你從小兒在他家使喚來。」那金蓮不聽便罷,聽了把臉掣耳朵帶脖子都紅了,便罵道:「汗邪了那賊老淫婦!我平日在他家做甚麼?還是我姨娘在他家緊隔壁住,他家有個花園,俺每小時在俺姨娘家住,常過去和他家伴姑兒耍子,就說我在他家來,我認的他是誰?也是個張眼露睛的老淫婦!」月娘道:「你看那嘴頭子!人和你說話,你罵他。」那金蓮一聲兒就不言語了。

月娘主張叫雪娥做了些水角兒,拿了前邊與西門慶吃。正走到儀門首,只見平安兒徑直往花園中走。被月娘叫住問道:「你做甚麼?」平安兒道:「李銘叫了四個唱的,十五日擺酒,因來回話。問擺的成擺不成。我說未發貼兒哩。他不信,教我進來稟爹。」月娘罵道:「怪賊奴才,還擺甚麼酒,問甚麼,還不回那王八去哩,還來稟爹娘哩。」把平安兒罵的往外金命水命去了。月娘走到金蓮房中,看著西門慶只吃了三四個水角兒,就不吃了。因說道:「李銘來回唱的,教我回倒他,改日子了,他去了。」西門慶點頭兒。

西門慶只望一兩日好些出來,誰知過了一夜,到次日,內邊虛陽腫脹,不便處發出紅瘰來,連腎囊都腫得明滴溜如茄子大。但溺尿,尿管中猶如刀子犁的一般。溺一遭,疼一遭。外邊排軍、伴當備下馬伺候,還等西門慶往衙門裡大發放,不想又添出這樣癥候來。月娘道:「你依我拿貼兒回了何大人,在家調理兩日兒,不去罷。你身子恁虛弱,趁早使小廝請了任醫官,教瞧瞧。你吃他兩貼藥過來。休要只顧耽著,不是事。你偌大的身量,兩日通沒大好吃甚麼兒,如何禁的?」那西門慶只是不肯吐口兒請太醫,只說:「我不妨事,過兩日好了,我還出去。」雖故差人拿貼兒送假牌往衙門裡去,在床上睡著,只是急躁,沒好氣。西門慶只望一兩日好些出來,誰知過了一夜,到次日,內邊虛陽腫脹,不便處發出紅瘰來,連腎囊都腫得明滴溜如茄子大。但溺尿,尿管中猶如刀子犁的一般。溺一遭,疼一遭。外邊排軍、伴當備下馬伺候,還等西門慶往衙門裡大發放,不想又添出這樣癥候來。月娘道:「你依我拿貼兒回了何大人,在家調理兩日兒,不去罷。你身子恁虛弱,趁早使小廝請了任醫官,教瞧瞧。你吃他兩貼藥過來。休要只顧耽著,不是事。你偌大的身量,兩日通沒大好吃甚麼兒,如何禁的?」那西門慶只是不肯吐口兒請太醫,只說:「我不妨事,過兩日好了,我還出去。」雖故差人拿貼兒送假牌往衙門裡去,在床上睡著,只是急躁,沒好氣。

應伯爵打聽得知,走來看他。西門慶請至金蓮房中坐的。伯爵聲喏道:「前日打攪哥,不知哥心中不好,嗔道花大舅那裡不去。」西門慶道:「我心中若好時,也去了。不知怎的懶待動旦。」伯爵道:「哥,你如今心內怎樣的?」西門慶道:「不怎的,只是有些頭暈,起來身子軟,走不的。」伯爵道:「我見你面容發紅色,只怕是火。教人看來不曾?」西門慶道:「房下說請任後溪來看我,我說又沒甚大病,怎好請他的。」伯爵道:「哥,你這個就差了,還請他來看看,怎的說。吃兩貼藥,散開這火就好了。春氣起,人都是這等痰火舉發舉發。昨日李銘撞見我,說你使他叫唱的,今日請人擺酒,說你心中不好,改了日子。把我唬了一跳,我今日才來看哥。」西門慶道:「我今日連衙門中拜牌也沒去,送假牌去了。」伯爵道:「可知去不的,大調理兩日兒出門。」吃畢茶道:「我去罷,再來看哥。李桂姐會了吳銀兒,也要來看你哩。」西門慶道:「你吃了飯去。」伯爵道:「我一些不吃。」揚長出去了。

西門慶於是使琴童往門外請了任醫官來,進房中診了脈,說道:「老先生此貴恙,乃虛火上炎,腎水下竭,不能既濟,此乃是脫陽之症。須是補其陰虛,方纔好得。」說畢,作辭起身去了。一面封了五錢銀子,討將藥來,吃了。止住了頭暈,身子依舊還軟,起不來。下邊腎囊越發腫痛,溺尿甚難。西門慶於是使琴童往門外請了任醫官來,進房中診了脈,說道:「老先生此貴恙,乃虛火上炎,腎水下竭,不能既濟,此乃是脫陽之症。須是補其陰虛,方纔好得。」說畢,作辭起身去了。一面封了五錢銀子,討將藥來,吃了。止住了頭暈,身子依舊還軟,起不來。下邊腎囊越發腫痛,溺尿甚難。

到後晌時分,李桂姐、吳銀兒坐轎子來看。每人兩個盒子,進房與西門慶磕頭,說道:「爹怎的心裡不自在?」西門慶道:「你姐兒兩個自恁來看看便了,如何又費心買禮兒。」因說道:「我今年不知怎的,痰火發的重些。」桂姐道:「還是爹這節間酒吃的多了,清潔他兩日兒,就好了。」坐了一回,走到李瓶兒那邊屋裡,與月娘眾人見節。請到後邊,擺茶畢,又走來到前邊,陪西門慶坐的說話兒。只見伯爵又陪了謝希大、常峙節來望。西門慶教玉簫搊扶他起來坐的,留他三人在房內,放桌兒吃酒。謝希大道:「哥,用了些粥不曾?」玉簫把頭扭著不答應。西門慶道:「我還沒吃粥,咽不下去。」希大道:「拿粥,等俺每陪哥吃些粥兒還好。」不一時,拿將粥來。西門慶拿起粥來,只扒了半盞兒,就吃不下了。月娘和李桂姐、吳銀兒都在李瓶兒那邊坐的。伯爵問道:「李桂姐與銀姐來了,怎的不見?」西門慶道:「在那邊坐的。」伯爵因令來安兒:「你請過來,唱一套兒與你爹聽。」吳月娘恐西門慶不耐煩,攔著,只說吃酒哩,不教過來。眾人吃了一回酒,說道: 「哥,你陪著俺每坐,只怕勞碌著你。俺每去了,你自在側側兒罷。」西門慶道:「起動列位掛心。」三人於是作辭去了。

應伯爵走出小院門,叫玳安過來分付:「你對你大娘說,應二爹說來,你爹面上變色,有些滯氣,不好,早尋人看他。大街上胡太醫最治的好痰火,何不使人請他看看,休要耽遲了。」玳安不敢怠慢,走來告訴月娘。月娘慌進房來,對西門慶說:「方纔應二哥對小廝說,大街上胡太醫看的痰火好,你何不請他來看看你?」西門慶道:「胡太醫前番看李大姐不濟,又請他?」月娘道:「藥醫不死病,佛度有緣人。看他不濟,只怕你有緣,吃了他的藥兒好了是的。」西門慶道:「也罷,你請他去。」不一時,使棋童兒請了胡太醫來。適有吳大舅來看,陪他到房中看了脈。對吳大舅、陳敬濟說:「老爹是個下部蘊毒,若久而不治,卒成溺血之疾。乃是忍便行房。」又卦了五星藥金,討將藥來吃下去,如石沉大海一般,反溺不出來。月娘慌了,打發桂姐、吳銀兒去了,又請何老人兒子何春泉來看。又說:「是癃閉便毒,一團膀胱邪火,趕到這下邊來。四肢經絡中,又有濕痰流聚,以致心腎不交。」封了五錢藥金,討將藥來,越發弄的虛陽舉發,麈柄如鐵,晝夜不倒。潘金蓮晚夕不管好歹,還騎在他身上,倒澆蠟燭掇弄,死而複蘇者數次。

到次日,何千戶要來望,先使人來說。月娘便對西門慶道:「何大人要來看你,我扶你往後邊去罷,這邊隔二騙三,不是個待人的。」那西門慶點頭兒。於是月娘替他穿上暖衣,和金蓮肩搭搊扶著,方離了金蓮房,往後邊上房,鋪下被褥高枕,安頓他在明間炕上坐的。房中收拾乾凈,焚下香。不一時,何千戶來到,陳敬濟請他到於後邊臥房,看見西門慶坐在病榻上,說道:「長官,我不敢作揖。」因問:「貴恙覺好些?」西門慶告訴:「上邊火倒退下了,只是下邊腫毒,當不的。」何千戶道:「此系便毒。我學生有一相識,在東昌府探親,昨日新到舍下,乃是山西汾州人氏,姓劉號桔齋,年半百,極看的好瘡毒。我就使人請他來看看長官貴恙。」 西門慶道:「多承長官費心,我這裡就差人請去。」何千戶吃畢茶,說道:「長官,你耐煩保重。衙門中事,我每日委答應的遞事件與你,不消掛意。」西門慶舉手道:「只是有勞長官了。」作辭出門。西門慶這裡隨即差玳安拿貼兒,同何家人請了這劉桔齋來。看了脈,並不便處,連忙上了藥,又封一貼煎藥來。西門慶答賀了一匹杭州絹,一兩銀子。吃了他頭一盞藥,還不見動靜。

那日不想鄭月兒送了一盒鴿子雛兒,一盒果餅頂皮酥,坐轎子來看。進門與西門慶磕頭,說道:「不知道爹不好,桂姐和銀姐好人兒,不對我說聲兒,兩個就先來了。看的爹遲了,休怪。」西門慶道:「不遲,又起動你費心,又買禮來。」愛月兒笑道:「甚麼大禮,惶恐。」因說:「爹清減的恁樣的,每日飲饌也用些兒?」 月娘道:「用的倒好了,吃不多兒。今日早辰,只吃了些粥湯兒,剛纔太醫看了去了。」愛月兒道:「娘,你分付姐把鴿子雛兒頓爛一個兒來,等我勸爹進些粥兒。你老人家不吃,恁偌大身量,一家子金山也似靠著你,卻怎麼樣兒的。」月娘道:「他只害心口內攔著,吃不下去。」愛月兒道:「爹,你依我說,把這飲撰兒就懶待吃,須也強吃些兒,怕怎的?人無根本,水食為命。終須用些兒。不然,越發淘淥的身子空虛了。」不一時,頓爛了鴿子雛兒,小玉拿粥上來,十香甜醬瓜茄,粳粟米粥兒。這鄭月兒跳上炕去,用盞兒托著,跪在西門慶身邊,一口口喂他。強打著精神,只吃了上半盞兒。揀兩箸兒鴿子雛兒在口內,就搖頭兒不吃了。愛月兒道:「一來也是藥,二來還虧我勸爹,卻怎的也進了些飲饌兒!」玉簫道:「爹每常也吃,不似今日月姐來,勸著吃的多些。」月娘一面擺茶與愛月兒吃,臨晚管待酒饌,與了他五錢銀子,打發他家去。愛月兒臨出門,又與西門慶磕頭,說道:「爹,你耐煩將息兩日兒,我再來看你。」

比及到晚夕,西門慶又吃了劉桔齋第二貼藥,遍身疼痛,叫了一夜。到五更時分,那不便處腎囊脹破了,流了一灘鮮血,龜頭上又生出疳瘡來,流黃水不止。西門慶不覺昏迷過去。月娘眾人慌了,都守著看視,見吃藥不效,一面請了劉婆子,在前邊捲棚內與西門慶點人燈挑神,一面又使小廝往周守備家內訪問吳神仙在那裡,請他來看,因他原相西門慶今年有嘔血流膿之災,骨瘦形衰之病。賁四說:「也不消問周老爹宅內去,如今吳神仙見在門外土地廟前,出著個卦肆兒,又行醫,又賣卦。人請他,不爭利物,就去看治。」月娘連忙就使琴童把這吳神仙請將來。進房看了西門慶不似往時,形容消減,病體懨懨,勒著手帕,在於臥榻。先診了脈息,說道:「官人乃是酒色過度,腎水竭虛,太極邪火聚於慾海,病在膏肓,難以治療。吾有詩八句,說與你聽。只因他:醉飽行房戀女娥,精神血脈暗消磨。遺精溺血與白濁,燈盡油乾腎水枯。 當時只恨歡娛少,今日翻為疾病多。玉山自倒非人力,總是盧醫怎奈何!」

月娘見他說治不的了,道:「既下藥不好,先生看他命運如何?」吳神仙掐指尋紋,打算西門慶八字,說道:「屬虎的,丙寅年,戊申月,壬午日,丙辰時。今年戊戌,流年三十三年,算命,見行癸亥運。雖然是火土傷官,今年戊土來克壬水。正月又是戊寅月,三戊沖辰,怎麼當的?雖發財發福,難保壽源。有四句斷語不好。說道:命犯災星必主低,身輕煞重有災危。時日若逢真太歲,就是神仙也皺眉。」

月娘道:「命不好,請問先生還有解麼?」神仙道:「白虎當頭,喪門坐命,神仙也無解,太歲也難推。造物已定,神鬼莫移。」月娘只得拿了一匹布,謝了神仙,打發出門。月娘見求神問卜皆有凶無吉,心中慌了。到晚夕,天井內焚香,對天發願,許下「兒夫好了,要往泰安州頂上與娘娘進香掛袍三年」。孟玉樓又許下逢七拜鬥,獨金蓮與李嬌兒不許願心。

西門慶自覺身體沉重,要便發昏過去,眼前看見花子虛、武大在他跟前站立,問他討債,又不肯告人說,只教人廝守著他。見月娘不在跟前,一手拉著潘金蓮,心中舍他不的,滿眼落淚,說道:「我的冤家,我死後,你姐妹們好好守著我的靈,休要失散了。」那金蓮亦悲不自勝,說道:「我的哥哥,只怕人不肯容我。」西門慶道:「等他來,等我和他說。」不一時,吳月娘進來,見他二人哭的眼紅紅的,便道:「我的哥哥,你有甚話,對奴說幾句兒,也是我和你做夫妻一場。」西門慶聽了,不覺哽咽哭不出聲來,說道:「我覺自家好生不濟,有兩句遺言和你說:我死後,你若生下一男半女,你姊妹好好待著,一處居住,休要失散了,惹人家笑話。」指著金蓮說:「六兒從前的事,你耽待他罷。」說畢,那月娘不覺桃花臉上滾下珍珠來,放聲大哭,悲慟不止。西門慶囑付了吳月娘,又把陳敬濟叫到跟前,說道:「姐夫,我養兒靠兒,無兒靠婿。姐夫就是我的親兒一般。我若有些山高水低,你發送了我入土。好歹一家一計,幫扶著你娘兒每過日子,休要教人笑話。」 又分付:「我死後,段子鋪里五萬銀子本錢,有你喬親家爹那邊,多少本利都找與他。教傅伙計把貸賣一宗交一宗,休要開了。賁四絨線鋪,本銀六千五百兩,吳二舅綢絨鋪是五千兩,都賣盡了貨物,收了來家。又李三討了批來,也不消做了,教你應二叔拿了別人家做去罷。李三、黃四身上還欠五百兩本錢,一百五十兩利錢未算,討來發送我。你只和傅伙計守著家門這兩個鋪子罷。印子鋪佔用銀二萬兩,生藥鋪五千兩,韓伙計、來保松江船上四千兩。開了河,你早起身,往下邊接船去。接了來家,賣了銀子併進來,你娘兒每盤纏。前邊劉學官還少我二百兩,華主簿少我五十兩,門外徐四鋪內,還欠我本利三百四十兩,都有合同見在,上緊使人摧去。到日後,對門並獅子街兩處房子都賣了罷,只怕你娘兒們顧攬不過來。」說畢,哽哽咽咽的哭了。陳敬濟道:「爹囑咐,兒子都知道了。」不一時,傅伙計、甘伙計、吳二舅、賁四、崔本都進來看視問安。西門慶一一都分付了一遍。眾人都道:「你老人家寬心,不妨事。」一日來問安看者,也有許多。見西門慶不好的沉重,皆嗟嘆而去。

過了兩日,月娘痴心,只指望西門慶還好,誰知天數造定,三十三歲而去。到於正月二十一日,五更時分,相火燒身,變出風來,聲若牛吼一般,喘息了半夜。挨到巳牌時分,嗚呼哀哉,斷氣身亡。正是:

三寸氣在千般用,一旦無常萬事休。

古人有幾句格言,說得好:

為人多積善,不可多積財。
積善成好人,積財惹禍胎。
石崇當日富,難免殺身災。
鄧通饑餓死,錢山何用哉!
今人非古比,心地不明白。
只說積財好,反笑積善呆。
多少有錢者,臨了沒棺材。

原來西門慶一倒頭,棺材尚未曾預備。慌的吳月娘叫了吳二舅與賁四到跟前,開了箱子拿四四錠元寶,教他兩個看材板去。剛纔打發去了,不防忽一陣就害肚裡疼,急撲進去床上倒下,就昏暈不省人事。孟玉樓與潘金蓮、孫雪娥都在那邊屋裡,七手八腳,替西門慶戴唐巾,裝柳穿衣服。忽聽見小玉來說:「俺娘跌倒在床上。」 慌的玉樓、李嬌兒就來問視,月娘手按著害肚內疼,就知道決撒了。玉樓教李嬌兒守著月娘,他就來使小廝快請蔡老娘去。李嬌兒又使玉簫前邊教如意兒來。比及玉樓回到上房裡面,不見了李嬌兒。原來李嬌兒趕月娘昏沉,房內無人,箱子開著,暗暗拿了五錠元寶,往他屋裡去了。手中拿將一搭紙,見了玉樓,只說:「尋不見草紙,我往房裡尋草紙去來。」那玉樓也不留心,且守著月娘,拿榪子伺候,見月娘看看疼的緊了。

不一時,蔡老娘到了,登時生下一個孩兒來。這屋裡裝柳西門慶停當,口內才沒氣兒,合家大小放聲號哭起來。蔡老娘收裹孩兒,剪去臍帶,煎定心湯與月娘吃了。扶月娘暖炕上坐的。月娘與了蔡老娘三兩銀子,蔡老娘嫌少,說道:「養那位哥兒賞了我多少,還與我多少便了。休說這位哥兒是大娘生養的。」月娘道:「比不得當時,有當家的老爹在此,如今沒了老爹,將就收了罷。待洗三來,再與你一兩就是了。」那蔡老娘道:「還賞我一套衣服兒罷。」拜謝去了。

月娘蘇醒過來,看見箱子大開著,便罵玉簫:「賊臭肉,我便昏了,你也昏了?箱子大開著,恁亂烘烘人走,就不說鎖鎖兒。」玉簫道:「我只說娘鎖了箱子,就不曾看見。」於是取鎖來鎖。玉樓見月娘多心,就不肯在他屋裡,走出對著金蓮說:「原來大姐姐恁樣的,死了漢子,頭一日就防範起人來了。」殊不知李嬌兒已偷了五錠元寶在屋裡去了。

當下吳二舅、賁四往尚推官家買了一付棺材板來,教匠人解鋸成槨。眾小廝把西門慶抬出,停當在大廳上,請了陰陽徐先生來批書。不一時,吳大舅也來了。吳二舅、眾伙計都在前廳熱亂,收燈捲畫,蓋上紙被,設放香燈幾席。來安兒專一打磨。徐先生看了手,說道:「正辰時斷氣,合家都不犯凶煞。」請問月娘:「三日大殮,擇二月十六破土,三十齣殯,有四七多日子。」一面管待徐先生去了,差人各處報喪,交牌印往何千戶家去,家中披孝搭棚,俱不必細說。

到三日,請僧人念倒頭經,挑出紙錢去。合家大小都披麻帶孝。女婿陳敬濟斬衰泣杖,靈前還禮。月娘在暗房中出不來。李嬌兒與玉樓陪待堂客;潘金蓮管理庫房,收祭桌;孫雪娥率領家人媳婦,在廚下打發各項人茶飯。傅伙計、吳二舅管帳、賁四管孝帳;來興管廚;吳大舅與甘伙計陪待人客。蔡老娘來洗了三,月娘與了一套綢絹衣裳打發去了。就把孩兒起名叫孝哥兒,未免送些喜面。親鄰與眾街坊鄰舍都說:「西門慶大官人正頭娘子生了一個墓生兒子,就與老子同日同時,一頭斷氣,一頭生兒,世間有這等蹊蹺古怪事。」

不說眾人理亂這樁事。且說應伯爵聞知西門慶沒了,走來弔孝哭泣,哭了一回。吳大舅、二舅正在捲棚內看著與西門慶傳影,伯爵走來,與眾人見禮,說道:「可傷,做夢不知哥沒了。」要請月娘拜見,吳大舅便道:「舍妹暗房出不來,如此這般,就是同日添了個娃兒。」伯爵愕然道:「有這等事!也罷也罷,哥有了個後代,這家當有了主兒了。」落後陳敬濟穿著一身重孝,走來與伯爵磕頭。伯爵道:「姐夫姐夫,煩惱。你爹沒了,你娘兒每是死水兒了,家中凡事要你仔細。有事不可自家專,請問你二位老舅主張。不該我說,你年幼,事體還不大十分歷練。」吳大舅道:「二哥,你沒的說。我自也有公事,不得閑,見有他娘在。」伯爵道: 「好大舅,雖故有嫂子,外邊事怎麼理的?還是老舅主張。自古沒舅不生,沒舅不長。一個親娘舅,比不的別人。你老人家就是個都根主兒,再有誰大?」因問道: 「有了發引日期沒有?」吳大舅道:「擇二月十六日破土,三十日出殯,也在四七之外。」不一時,徐先生來到,祭告入殮,將西門慶裝入棺材內,用長命丁釘了,安放停當,題了名旌:「誥封武略將軍西門公之柩」。

那日何千戶來弔孝。靈前拜畢,吳大舅與伯爵陪侍吃茶,問了發引的日期。何千戶分付手下該班排軍,原答應的,一個也不許動,都在這裡伺候。直過發引之後,方許回衙門當差。又委兩名節級管領,如有違誤,呈來重治。又對吳大舅說:「如有外邊人拖欠銀兩不還者,老舅只顧說來,學生即行追治。」弔老畢,到衙門裡一面行文開缺,申報東京本衛去了。

話分兩頭。卻說來爵、春鴻同李三,一日到兗州察院,投下了書禮,宋御史見西門慶書上要討古器批文一節,說道:「你早來一步便好。昨日已都派下各府買辦去了。」尋思間,又見西門慶書中封著金葉十兩,又不好違阻了的。便留下春鴻、來爵、李三在公廨駐札。隨即差快手拿牌,趕回東平府批文來,封回與春鴻書中,又與了一兩路費,方取路回清河縣。往返十日光景。走進城,就聞得路上人說:「西門大官人死了,今日三日,家中念經做齋哩。」這李三就心生姦計,路上說念來爵、春鴻:「將此批文按下,只說宋老爺沒與來。咱每都投到大街張二老爹那裡去罷。你二人不去,我每人與你十兩銀子,到家隱住,不拿出來就是了。」那來爵見財物倒也肯了,只春鴻不肯,口裡含糊應諾。

到家,見門首挑著紙錢,僧人做道場,親朋弔喪者不計其數,這李三就分路回家去了。來爵、春鴻見吳大舅、陳敬濟磕了頭,問:「討批文如何?怎的李三不來?」 那來爵欲說不肯,這春鴻把宋御史書連批都拿出來,遞與大舅,悉把李三路上與的十兩銀子,說的言語,如此這般教他隱下,休拿出來,同他投往張二官家去:「小的怎敢忘恩負義?徑奔家來。」吳大舅一面走到後邊,告訴月娘:「這個小的兒,就是個知恩的。叵耐李三這廝短命,見姐夫沒了幾日,就這等壞心。」因把這件事就對應伯爵說:「李智、黃四借契上本利還欠六百五十兩銀子,趁著剛纔何大人分付,把這件事寫紙狀子,呈到衙門裡,教他替俺追追這銀子來,發送姐夫。他同寮間自恁要做分上,這些事兒莫道不依。」伯爵慌了,說道:「李三卻不該行此事。老舅快休動意,等我和他說罷。」於是走到李三家,請了黃四來,一處計較。說道:「你不該先把銀子遞與小廝,倒做了管手。狐狸打不成,倒惹了一屁股臊。如今恁般,要拿文書提刑所告你每哩。常言道官官相護,何況又同寮之間,你等怎抵鬥的他過!依我,不如悄悄遂二十兩銀子與吳大舅,只當兗州府幹了事來了。我聽得說,這宗錢糧他家已是不做了,把這批文難得掣出來,咱投張二官那裡去罷。你每二人再湊得二百兩,少不也拿不出來,再備辦一張祭桌,一者祭奠大官人,二者交這銀子與他。另立一紙欠結,你往後有了買賣,慢慢還他就是了。這個一舉兩得,又不失了人情,有個始終。」黃四道:「你說的是。李三哥,你幹事忒慌速了些。」真個到晚夕,黃四同伯爵送了二十兩銀子到吳大舅家,如此這般,「討批文一節,累老舅張主張主。」這吳大舅已聽見他妹子說不做錢糧,何況又黑眼見了白晃晃銀子,如何不應承,於是收了銀子。

到次日,李智、黃四備了一張插桌,豬首三牲,二百兩銀子,來與西門慶祭奠。吳大舅對月娘說了,拿出舊文書,從新另立了四百兩一紙欠帖,饒了他五十兩,餘者教他做上買賣,陸續交還。把批文交付與伯爵手內,同往張二官處合夥,上納錢糧去了,不在話下。正是:金逢火煉方知色,人與財交便見心。有詩為證:

造物於人莫強求,勸君凡事把心收。
你今貪得收人業,還有收人在後頭。

