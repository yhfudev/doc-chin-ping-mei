%# -*- coding: utf-8 -*-
% !TeX encoding = UTF-8 Unicode
% !TeX spellcheck = en_US
% !TeX TS-program = xelatex
%~ \XeTeXinputencoding "UTF-8"
% vim:ts=4:sw=4
%
% 以上設定默認使用 XeLaTex 編譯,並指定 Unicode 編碼,供 TeXShop 自動識別

%第九回
\chapter{西門慶偷娶潘金蓮 武都頭誤打李皂隸}

\begin{showcontents}{}



詩曰:

感郎耽夙愛,著意守香奩。
歲月多忘遠,情綜任久淹。
于飛期燕燕,比翼誓鶼鶼。
細數從前意,時時屈指尖。

話說西門慶與潘金蓮燒了武大靈,到次日,又安排一席酒,請王婆作辭,就把迎兒交付與王婆看養。因商量道:「武二回來,卻怎生不與他知道六姐是我娶了才好?」王婆笑道:「有老身在此,任武二那廝怎地兜達,我自有話回他。大官人只管放心!」西門慶聽了,滿心歡喜,又將三兩銀子謝他。當晚就將婦人箱籠,都打發了家去,剩下些破桌、壞凳、舊衣裳,都與了王婆。到次日初八,一頂轎子,四個燈籠,婦人換了一身艷色衣服,王婆送親,玳安跟轎,把婦人抬到家中來。那條街上,遠近人家無一不知此事,都懼怕西門慶有錢有勢,不敢來多管,只編了四句口號,說得好:

堪笑西門不識羞,先奸後娶醜名留。
轎內坐著浪淫婦,後邊跟著老牽頭。

西門慶娶婦人到家,收拾花園內樓下三間與他做房。一個獨獨小角門兒進去,院內設放花草盆景。白日間人跡罕到,極是一個幽僻去處。一邊是外房,一邊是臥房。西門慶旋用十六兩銀子買了一張黑漆歡門描金床,大紅羅圈金帳幔,寶象花揀妝,桌椅錦杌,擺設齊整。大娘子吳月娘房裡使著兩個丫頭,一名春梅,一名玉簫。西門慶把春梅叫到金蓮房內,令他伏侍金蓮,趕著叫娘。卻用五兩銀子另買一個小丫頭,名叫小玉,伏侍月娘。又替金蓮六兩銀子買了一個上灶丫頭,名喚秋菊。排行金蓮做第五房。先頭陳家娘子陪嫁的,名喚孫雪娥,約二十年紀,生的五短身材,有姿色。西門慶與他戴了鬏髻,排行第四,以此把金蓮做個第五房。此事表過不題。

這婦人一娶過門來,西門慶就在婦人房中宿歇,如魚似水,美愛無加。到第二日,婦人梳妝打扮,穿一套艷色服,春梅捧茶,走來後邊大娘子吳月娘房裡,拜見大小,遞見面鞋腳。月娘在座上仔細觀看,這婦人年紀不上二十五六,生的這樣標緻。但見:

眉似初春柳葉,常含著雨恨雲愁;臉如三月桃花,暗帶著風情月意。纖腰裊娜,拘束的燕懶鶯慵;檀口輕盈,勾引得峰狂蝶亂。玉貌妖嬈花解語,芳容窈窕玉生香。

吳月娘從頭看到腳,風流往下跑;從腳看到頭,風流往上流。論風流,如水泥晶盤內走明珠;語態度,似紅杏枝頭籠曉日。看了一回,口中不言,心內想道:「小廝每來家,只說武大怎樣一個老婆,不曾看見,不想果然生的標緻,怪不的俺那強人愛他。」金蓮先與月娘磕了頭,遞了鞋腳。月娘受了他四禮。次後李嬌兒、孟玉樓、孫雪娥,都拜見了,平敘了姊妹之禮,立在旁邊。月娘叫丫頭拿個坐兒教他坐,吩咐丫頭、媳婦趕著他叫五娘。這婦人坐在旁邊,不轉睛把眾人偷看。見吳月娘約三九年紀,生的面如銀盆,眼如杏子,舉止溫柔,持重寡言。第二個李嬌兒,乃院中唱的,生的肌膚豐肥,身體沉重,雖數名妓者之稱,而風月多不及金蓮也。第三個就是新娶的孟玉樓,約三十年紀,生得貌若梨花,腰如楊柳,長挑身材,瓜子臉兒,稀稀多幾點微麻,自是天然俏麗,惟裙下雙灣與金蓮無大小之分。第四個孫雪娥,乃房裡出身,五短身材,輕盈體態,能造五鮮湯水,善舞翠盤之妙。這婦人一抹兒都看在心裡。過三日之後,每日清晨起來,就來房裡與月娘做針指,做鞋腳,凡事不拿強拿,不動強動。指著丫頭趕著月娘,一口一聲只叫大娘,快把小意兒貼戀幾次,把月娘喜歡得沒入腳處,稱呼他做六姐。衣服首飾揀心愛的與他,吃飯喫茶都和他在一處。因此,李嬌兒眾人見月娘錯敬他,都氣不忿,背後常說:「俺們是舊人,到不理論。他來了多少時,便這等慣了他。大姐姐好沒分曉!」西門慶自娶潘金蓮來家,住著深宅大院,衣服頭面又相趁,二人女貌郎才,正在妙年之際,凡事如膠似漆,百依百隨,淫慾之事,無日無之。且按下不題。

單表武松,八月初旬到了清河縣,先去縣裡納了回書。知縣見了大喜,已知金寶交得明白,賞了武松十兩銀子,酒食管待,不必細說。武松回到下處,換了衣服鞋襪,戴了一頂新頭巾,鎖了房門,一徑投紫石街來。兩邊眾鄰舍看見武松回來,都吃一驚,捏兩把汗,說道:「這番蕭牆禍起了!這個太歲歸來,怎肯干休!」武松走到哥哥門前,揭起簾子,探身入來,看見小女迎兒在樓穿廊下攆線。叫聲哥哥也不應,叫聲嫂嫂也不應,道:「我莫不耳聾了,如何不見哥嫂聲音?」向前便問迎兒。那迎兒見他叔叔來,嚇的不敢言語。武松道:「你爹娘往那裡去了?」迎兒只是哭,不做聲。正問間,隔壁王婆聽得是武二歸來,生怕決撒了,慌忙走過來。武二見王婆過來,唱了喏,問道:「我哥哥往那裡去了?嫂嫂也怎的不見?」婆子道:「二哥請坐,我告訴你。你哥哥自從你去後,到四月間得個拙病死了。」武二道:「我哥哥四月幾時死的?得什麼病?吃誰的藥來?」王婆道:「你哥哥四月二十頭,猛可地害起心疼起來,病了八九日,求神問卜,什麼藥不吃到?醫治不好,死了。」武二道:「我的哥哥從來不曾有這病,如何心疼便死了?」王婆道:「都頭卻怎的這般說?天有不測風雲,人有旦夕禍福。今晚脫了鞋和襪,未審明朝穿不穿。誰人保得常沒事?」武二道:「我哥哥如今埋在那裡?」王婆道:「你哥哥一倒了頭,家中一文錢也沒有,大娘子又是沒腳蟹,那裡去尋墳地?虧左近一個財主舊與大郎有一面之交,捨助一具棺木,沒奈何放了三日,抬出去火葬了。」武二道:「如今嫂嫂往那裡去了?」婆子道:「他少女嫩婦的,又沒的養贍過日子。胡亂守了百日孝,他娘勸他,前月嫁了外京人去了。丟下這個業障丫頭子,教我替他養活。專等你回來交付與你,也了我一場事。」武二聽言,沉吟了半晌,便撇下王婆出門去,逕投縣前下處。開了門進房裡,換了一身素衣,便叫土兵街上打了一條麻絛,買了一雙綿褲,一頂孝帽戴在頭上;又買了些果品點心、香燭冥紙、金銀錠之類,歸到哥哥家,從新安設武大靈位。安排羹飯,點起香燭,鋪設酒餚,掛起經幡紙繒,安排得端正。約一更已後,武二拈了香,撲翻身便拜,道:「哥哥陰魂不遠,你在世時,為人軟弱,今日死後,不見分明。你若負屈含冤,被人害了,托夢與我,兄弟替你報冤雪恨!」把酒一面澆奠了,燒化冥紙,武二便放聲大哭。終是一路上來的人,哭的那兩邊鄰舍無不淒惶。武二哭罷,將這羹飯酒餚和土兵、迎兒吃了。討兩條蓆子,教土兵房外旁邊睡,迎兒房中睡,他便自把條蓆子,就武大靈桌子前睡。

約莫將半夜時分,武二翻來覆去那裡睡得著,口裡只是長吁氣。那土兵齁齁的卻似死人一般,挺在那裡。武二爬將起來看時,那靈桌子上琉璃燈半明半滅。武二坐在蓆子上,自言自語,口裡說道:「我哥哥生時懦弱,死後卻無分明。」說猶未了,只見那靈桌子下捲起一陣冷風來。但見:

無形無影,非霧非煙。盤旋似怪風侵骨冷,凜冽如殺氣透肌寒。昏昏暗暗,靈前燈火失光明;慘慘幽幽,壁上紙錢飛散亂。隱隱遮藏食毒鬼,紛紛飄逐影魂幡。

那陣冷風,逼得武二毛髮皆豎起來。定睛看時,見一個人從靈桌底下鑽將出來,叫聲:「兄弟!我死得好苦也!」武二看不仔細,卻待向前再問時,只見冷氣散了,不見了人。武二一交跌翻在蓆子上坐的,尋思道:「怪哉!似夢非夢。剛才我哥哥正要報我知道,又被我的神氣衝散了。想來他這一死,必然不明。」聽那更鼓,正打三更三點。回頭看那土兵,正睡得好。於是咄咄不樂,只等天明,卻再理會。

看看五更雞叫,東方漸明。土兵起來燒湯,武二洗漱了,喚起迎兒看家,帶領土兵出了門。在街上訪問街坊鄰舍:「我哥哥怎的死了?嫂嫂嫁得何人去了?」那街坊鄰舍明知此事,都懼怕西門慶,誰肯來管?只說:「都頭,不消訪問,王婆在緊隔壁住,只問王婆就知了。」有那多口的說:「賣梨的鄆哥兒與仵作何九,二人最知詳細。」這武二竟走來街坊前去尋鄆哥。只見那小猴子手裡拿著個柳籠簸羅兒,正糴米回來。武二便叫鄆哥道:「兄弟!」唱喏。那小廝見是武二叫他,便道:「武都頭,你來遲了一步兒,須動不得手。只是一件,我的老爹六十歲,沒人養贍,我卻難保你們打官司。」武二道:「好兄弟,跟我來。」引他到一個飯店樓上,武二叫貨賣造兩分飯來。武二對鄆哥道:「兄弟,你雖年幼,倒有養家孝順之心。我沒什麼──」向身邊摸出五兩碎銀子,遞與鄆哥道:「你且拿去與老爹做盤費。待事務畢了,我再與你十來兩銀子做本錢。你可備細說與我:哥哥和甚人合氣?被甚人謀害了?家中嫂嫂被那一個娶去?你一一說來,休要隱匿。」這鄆哥一手接過銀子,自心裡想道:「這些銀子,老爹也夠盤費得三五個月,便陪他打官司也不妨。」一面說道:「武二哥,你聽我說,卻休氣苦。」於是把賣梨兒尋西門慶,後被王婆怎地打他,不放進去,又怎地幫扶武大捉姦,西門慶怎的踢中了武大,心疼了幾日,不知怎的死了,從頭至尾細說了一遍。武二聽了,便道:「你這話卻是實麼?」又問道:「我的嫂子實嫁與何人去了?」鄆哥道:「你嫂子吃西門慶抬到家,待搗吊底子兒,自還問他實也是虛!」武二道:「你休說謊。」鄆哥道:「我便官府面前,也只是這般說。」武二道:「兄弟,既然如此,討飯來吃。」須臾,吃了飯。武二還了飯錢,兩個下樓來,吩咐鄆哥:「你回家把盤纏交與老爹,明日早上來縣前,與我作證。」又問:「何九在那裡居住?」鄆哥道:「你這時候還尋何九?他三日前聽見你回,便走的不知去向了。」這武二放了鄆哥家去。

到第二日,早起,先在陳先生家寫了狀子,走到縣門前。只見鄆哥也在那裡伺候,一直奔到廳上跪下,聲冤起來。知縣看見,認的是武松,便問:「你告什麼?因何聲冤?」武二告道:「小人哥哥武大,被豪惡西門慶與嫂潘氏通姦,踢中心窩,王婆主謀,陷害性命。何九朦朧入殮,燒燬屍傷。見今西門慶霸佔嫂子在家為妾。見有這個小廝鄆哥是證見。望相公作主則個。」因遞上狀子。知縣接著,便問:「何九怎的不見?」武二道:「何九知情在逃,不知去向。」知縣於是摘問了鄆哥口詞,當下退廳與佐二官吏通同商議。原來知縣、縣丞、主簿、典史,上下都是與西門慶有首尾的,因此官吏通同計較,這件事難以問理。知縣隨出來叫武松道:「你也是個本縣中都頭,怎不省得法度?自古捉姦見雙,殺人見傷。你那哥哥屍首又沒了,又不曾捉得他奸。你今只憑這小廝口內言語,便問他殺人的公事,莫非公道忒偏向麼?你不可造次,須要自己尋思。」武二道:「告稟相公,這都是實情,不是小人捏造出來的。只望相公拿西門慶與嫂潘氏、王婆來,當堂盡法一番,其冤自見。若有虛誣,小人情願甘罪。」知縣道:「你且起來,待我從長計較。可行時,便與你拿人。」武二方才起來,走出外邊,把鄆哥留在屋裡,不放回家。

早有人把這件事報與西門慶得知。西門慶聽得慌了,忙叫心腹家人來保、來旺,身邊帶著銀兩,連夜將官吏都買囑了。到次日早晨,武二在廳上指望告稟知縣,催逼拿人。誰想這官人受了賄賂,早發下狀子來,說道:「武松,你休聽外人挑撥,和西門慶做對頭。這件事欠明白,難以問理。聖人云:經目之事,猶恐未真;背後之言,豈能全信?你不可一時造次。」當該吏典在旁,便道:「都頭,你在衙門裡也曉得法律,但凡人命之事,須要屍、傷、病、物、蹤,五件事俱完,方可推問。你那哥哥屍首又沒了,怎生問理?」武二道:「若恁的說時,小人哥哥的冤仇,難道終不能報便罷了?既然相公不准所告,且卻有理。」遂收了狀子,下廳來。來到下處,放了鄆哥歸家,不覺仰天長歎一聲,咬牙切齒,口中罵淫婦不絕。

武松是何等漢子,怎消洋得這口惡氣!一直走到西門慶生藥店前,要尋西門慶廝打。正見他開舖子的傅夥計在櫃身裡面,見武二狠狠的走來,問道:「你大官人在宅上麼?」傅夥計認的是武二,便道:「不在家了。都頭有甚話說?」武二道:「且請借一步說句。」傅夥計不敢不出來,被武二引到僻靜巷口。武二翻過臉來,用手撮住他衣領,睜圓怪眼說道:「你要死,卻是要活?」傅夥計道:「都頭在上,小人又不曾觸犯了都頭,都頭何故發怒?」武二道:「你若要死,便不要說;若要活時,對我實說。西門慶那廝如今在那裡?我的嫂子被他娶了多少日子?一一說來,我便罷休?」那傅夥計是個小膽的人,見武二發作,慌了手腳,說道:「都頭息怒,小人在他家,每月二兩銀子雇著,小人只開舖子,並不知他們閒帳。大官人本不在家,剛才和一相知,往獅子街大酒樓上吃酒去了。小人並不敢說謊。」武二聽了此言,方才放了手,大叉步飛奔到獅子街來。嚇的傅夥計半日移腳不動。那武二逕奔到獅子街橋下酒樓前來。

且說西門慶正和縣中一個皂隸李外傳在樓上吃酒。原來那李外傳專一在府縣前綽攬些公事,往來聽氣兒撰些錢使。若有兩家告狀的,他便賣串兒;或是官吏打點,他便兩下裡打背。因此縣中就起了他這個渾名,叫做李外傳。那日見知縣回出武松狀子,討得這個消息,便來回報西門慶知道。因此西門慶讓他在酒樓上飲酒,把五兩銀子送他。正吃酒在熱鬧處,忽然把眼向樓窗下看,只見武松似凶神般從橋下直奔酒樓前來。已知此人來意不善,不覺心驚,欲待走了,卻又下樓不及,遂推更衣,走往後樓躲避。武二奔到酒樓前,便問酒保道:「西門慶在此麼?」酒保道:「西門大官人和一相識在樓上吃酒哩。」武二撥步撩衣,飛搶上樓去。早不見了西門慶,只見一個人坐在正面,兩個唱的粉頭坐在兩邊。認的是本縣皂隸李外傳,就知是他來報信,不覺怒從心起,便走近前,指定李外傳罵道:「你這廝,把西門慶藏在那裡去了?快說了,饒你一頓拳頭!」李外傳看見武二,先嚇呆了,又見他惡狠狠逼緊來問,那裡還說得出話來!武二見他不則聲,越加惱怒,便一腳把桌子踢倒,碟兒盞兒都打得粉碎。兩個粉頭嚇得魂都沒了。李外傳見勢頭不好,強掙起身來,就要往樓下跑。武二一把扯回來道:「你這廝,問著不說,待要往那裡去?且吃我一拳,看你說也不說!」早颼的一拳,飛到李外傳臉上。李外傳叫聲啊呀,忍痛不過,只得說道:「西門慶才往後樓更衣去了,不干我事,饒我去罷!」武二聽了,就趁勢兒用雙手將他撮起來,隔著樓窗兒往外只一兜,說道:「你既要去,就饒你去罷!」撲通一聲,倒撞落在當街心裡。武二隨即趕到後樓來尋西門慶。此時西門慶聽見武松在前樓行兇,嚇得心膽都碎,便不顧性命,從後樓窗一跳,順著房簷,跳下人家後院內去了。武二見西門慶不在後樓,只道是李外傳說謊,急轉身奔下樓來,見李外傳已跌得半死,直挺挺在地下,還把眼動。氣不過,兜襠又是兩腳,早已哀哉斷氣身亡。眾人道:「這是李皂隸,他怎的得罪都頭來?為何打殺他?」武二道:「我自要打西門慶,不料這廝悔氣,卻和他一路,也撞在我手裡。」那地方保甲見人死了,又不敢向前捉武二,只得慢慢挨上來收籠他,那裡肯放鬆!連酒保王鸞並兩個粉頭包氏、牛氏都拴了,竟投縣衙裡來。此時哄動了獅子街,鬧了清河縣,街上議論的人,不計其數。卻不知道西門慶不該死,倒都說是西門慶大官人被武松打死了。正是:

李公吃了張公釀,鄭六生兒鄭九當。
世間幾許不平事,都付時人話短長。



\piWenlongF{
文禹门云:此回脱卸《水浒传》,归入《金瓶梅》正传。李外传之传,读作去声,方合本旨,故用之以脱卸西门庆。《水浒》为里传,此书为外传也。

独是武松一口恶气,未能出得,看者能勿怏怏乎?惟其怏怏也!方可与看《金瓶梅》。必须怏怏到底,方知《金瓶梅》不是淫书也。或曰:假耳,何必怏怏?予曰:既知是假,又何必看。第恐看到中间,又转以为真。斯不若怏怏者,尚有天理良心也。

}



\end{showcontents}


